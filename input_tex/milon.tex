% problems:
% Psukim - not recognized as MotoChapter
% section_title_secondary:
%   gets into page title - shouldn't be there
%   gets into thumbs - shouldn't be there
%   because of these, Stam font has been lost (probably because it's not ``Otiyut'' but ``otiut and friends''
%   make sure no harm is done to other sections and to Web, when finished here
%   make sure that it's correctly spelled
%   make sure that it matches book
%   then delete all ``ZZ'' and ``unite_lines'' prints


% debug warning messages

% https://www.ctan.org/pkg/addlines - might be useful in later stages (adding or subing line or few lines from specific pages to fix appearance)

\documentclass[usegeometry,12pt]{scrbook}

\usepackage{nag}

%%%%%%%%%%%        PAGE SIZE        %%%%%%%%%%%
\usepackage[paperheight=10.7in, paperwidth=20.7cm, top=3.5cm, bottom=1.8cm, headsep=1.3cm, left=2.3cm, right=2.3cm]{geometry}
% add ``showframe'' parameter for debugging page layout

%%%%%%%%%%%        THUMBS        %%%%%%%%%%%
\usepackage{polythumbs}

%%%%%%%%%%%        HEADING        %%%%%%%%%%%

\usepackage{fancyhdr}
\pagestyle{fancy}
\fancyhf{}

\newcommand{\fancyheadlo}{\mynormalsize{\PolythumbWrapperLeft\hspace{-1mm}\thepage}}
\newcommand{\fancyheadre}{\mynormalsize{\thepage\PolythumbWrapperRight}}
\newcommand{\setfancyheadtitleboth}[1] {\fancyhead[CE,CO]{\mynormalsize\removelinebreaks{#1}}}

\fancyhead[LO]{\fancyheadlo}
\fancyhead[RE]{\fancyheadre}

\pagenumbering{alph}			% change page numbers to Hebrew letters
\setcounter{page}{-1}			% make page 1 first


%%%%%%%%%%%        HEADING - `plain' with page number only       %%%%%%%%%%%
\fancypagestyle{plain}{%
    \fancyhf{}
    \fancyhead[LO]{\fancyheadlo}
    \fancyhead[RE]{\fancyheadre}
}


%%%%%%%%%%%        COLUMNS        %%%%%%%%%%%
\usepackage{multicol}		% allow using 2 columns
\setlength{\columnsep}{1.5pc}	% increase space between columns


%%%%%%%%%%%        HEBREW        %%%%%%%%%%%
\usepackage{polyglossia}		% allow Hebrew
\setdefaultlanguage{hebrew}	
\rightfootnoterule   			% move footnote ruler to the right, until https://github.com/reutenauer/polyglossia/issues/5 will be fixed

% 31.8.20 - this patch doesn't compile with newer MikTeX, commenting it
% TODO: check if it's still required, or maybe the problem has been fixed
%D % Patch received from Markus Kohm‏ by mail, on 15.1.19 - to fix footnote recurrence problem - he doesn't promise it's 100% good...
%D \usepackage{xpatch}
%D \makeatletter
%D \xpatchcmd{\hebrew@globalnumbers}{\protect}{}{}{\PAtchFailed}
%D \makeatother


%%%%%%%%%%%    CHAPTERS/PARTS - Koma script style    %%%%%%%%%%%
% from: https://tex.stackexchange.com/questions/470051/koma-script-how-to-create-an-empty-page-after-each-chapter-heading
%\RedeclareSectionCommand[style=part]{chapter}
\renewcommand*{\partformat}{}                % remove part number
\renewcommand*{\chapterformat}{}             % remove chapter number
\renewcommand*{\partpagestyle}{empty}        % remove page number from part
\renewcommand*{\chapterpagestyle}{empty}     % remove page number from chapter
\setkomafont{disposition}{\normalcolor}


%%%%%%%%%%%%    Sections (etc.) heading style    %%%%%%%%%%%%%%%%%%%%%%%%%%%%%%

%\RedeclareSectionCommand[style=chapter]{subsubsection}
\addtokomafont{subsubsection}{\center\large}
\addtokomafont{chapter}{\fontsize{24}{54}\selectfont}
\renewcommand*{\raggedsection}{\centering}

\renewcommand*{\chapterheadendvskip}{\clearpage}

%%%%%%%%%%%        UNINDENT        %%%%%%%%%%%
\setlength{\parindent}{0pt}		% avoid space at beginning of paragraph


%%%%%%%%%%%        FOOTNOTES    %%%%%%%%%%%
\deffootnote[1.5em]{1.75em}{1.75em}{\makebox[1.5em][r]{\myfootnotesize{\thefootnotemark.\space}}}	% unindent footnotes, change footnote marker

\setlength{\skip\footins}{2em} % vertical space between text body and footnote ruler

% based on http://tex.stackexchange.com/questions/59501/how-do-i-add-a-blank-line-between-the-footnote-line-and-the-footnotes?rq=1
\let\oldfootnoterule\footnoterule
\def\footnoterule{\oldfootnoterule \vskip1em\relax}      % vertical space between ruler and footnotes


% keep all footnotes on same page. taken from https://tex.stackexchange.com/questions/32208/footnote-runs-onto-second-page
% \interfootnotelinepenalty=10000
% -
% the previous setting solved that problem, but created new problem with strange spacing in beginning of ``Tora'' section.
% trying to mitigate both, with lowering to 5000
\interfootnotelinepenalty=5000

% define "reset footnote counte" command
\newcommand{\resetfootnotecounter}{\setcounter{footnote}{0}}
        

% Footnote mark in text - fix size
% based on https://tex.stackexchange.com/questions/47324/superscript-outside-math-mode/140703
\deffootnotemark{\scriptsize{$^{\mathrm{\thefootnotemark}}$}}


%%%%%%%%%%%        MISC        %%%%%%%%%%%

% commented, because Rav asked to make it denser. keeping for future reference...
%\usepackage[onehalfspacing]{setspace}            % instead of `linespread` - which caused strange footnotes spacing 

\setlength{\parskip}{1ex plus 0.2ex minus 0.2ex}		% increase spacing between paragraphs

% for Moto's minipage centering
\usepackage{adjustbox}


% to make sure that subsubsection heading isn't at end of page
\usepackage{needspace}


% fixes column balancing - my packaging of https://tex.stackexchange.com/questions/471655/avoiding-parskip-at-end-of-column/472148?noredirect=1#comment1193617_472148
\usepackage{hebcolumnbal}     


%%%%%%%%%%%        FONTS        %%%%%%%%%%%
% regular Hebrew fonts declarations
\usepackage{fontspec}
\newfontfamily\hebrewfont{Times New Roman}[Script=Hebrew,LetterSpace=3]
\newfontfamily\hebrewfonttt{Miriam}[Script=Hebrew,AutoFakeBold=1.5]

% declare new specific fonts (and new commands)
\newfontfamily\hebrewfonthadassah{Hadassah Friedlaender}[Script=Hebrew]
\newfontfamily\hebrewfontstam{Guttman Stam}[Script=Hebrew]


% copied from https://tex.stackexchange.com/questions/23450/font-problems-hebrew-with-koma-document-classes-error-message?rq=1
\setmainfont[Mapping=tex-text]{Times New Roman}
\setsansfont[Mapping=tex-text]{Times New Roman}
\setmonofont[Mapping=tex-text]{Times New Roman}


%%%%%%%%%%%        COMMANDS        %%%%%%%%%%%
\newcommand\mysectiontitlesecondarysize{\fontsize{18}{34}\selectfont}
\newcommand\mystamsize{\fontsize{14}{14}\selectfont}              % used by Stam
\newcommand\mylarge{\fontsize{11}{14}\selectfont}
\newcommand\mynormalsize{\fontsize{10}{14}\selectfont}
\newcommand\mysmall{\fontsize{9}{14}\selectfont}
\newcommand\myfootnotesize{\fontsize{9}{12}\selectfont}

\usepackage{xstring}

% copied from https://tex.stackexchange.com/questions/35884/easy-way-to-remove-formatting-e-g-linebreaks
\newcommand{\removelinebreaks}[1]{%
  \begingroup\def\\{}#1\endgroup}


\newcommand{\mybookname}[2]{
    \end{multicols}

    \addtokomafont{part}{\vspace{-3cm}\fontsize{35}{35}\hebrewfonthadassah}    

    \part[#2]{#1}

    \addtokomafont{part}{\hebrewfont\Huge}

    \addPolythumb{#2}

    \setfancyheadtitleboth{#1}
    \begin{multicols}{2}
}

\newcommand{\mytitle}[2]{
    \end{multicols}

    \part[#2]{#1}

    \addPolythumb{#2}

    \setfancyheadtitleboth{#1}
    \resetfootnotecounter
    \begin{multicols}{2}
}




\newcommand{\mychapter}[2]{
    \end{multicols}
    \part[#2]{#1}
    
    \addPolythumb{#2}

    \setfancyheadtitleboth{#1}
    \resetfootnotecounter
    \begin{multicols}{2}
}


\newcommand{\mymotochapter}[2]{
    \end{multicols}
    \chapter[#2]{#1}

    \addPolythumb{#2}

    \setfancyheadtitleboth{#1}
    \resetfootnotecounter
    \begin{multicols}{2}
}


\newcommand{\myintrochapter}[2]{
    \end{multicols}

    \chapter[#2]{#1}
    \thispagestyle{plain}

    \addPolythumb{#2}

    \setfancyheadtitleboth{#1, הגדרות מבוא}
    \resetfootnotecounter
    \begin{multicols}{2}
}



\newcommand{\mysubsection}[1]{
    \end{multicols}
    \subsection{#1}
    \setfancyheadtitleboth{#1}
    \resetfootnotecounter
    \begin{multicols}{2}
}

\newcommand{\myletterweaktitle}[1]{
    \end{multicols}
    \newpage
    \thispagestyle{plain}

    \vspace*{0.5cm}
    \subsubsection{#1}

    \resetfootnotecounter
    \begin{multicols}{2}
}


\newcommand{\mylettertitle}[1]{
    \end{multicols}
    \newpage
    \thispagestyle{plain}
   
    \vspace*{0.5cm}
    \subsubsection{#1}

    % the following two are the diff between this, and \myletterweaktitle
    \replacePolythumb{#1}
    \fancyhead[CO]{\mynormalsize{אות #1}}

    \resetfootnotecounter
    \begin{multicols}{2}
}

\newcommand{\myletterslave}[1]{
    \end{multicols}
    \needspace{3cm}
    \subsubsection{#1}
    \resetfootnotecounter
    \begin{multicols}{2}
}

\newcommand{\myfootnote}[1]{\footnote{\myfootnotesize{#1}}}
\newcommand{\mycircle}[1]{\textmd{\mynormalsize{#1}}}
\newcommand{\hebrewmakaf}{\iffont{Miriam(0)}{\,\,־}{־}}       % Miriam font has bug, and needs another space before the Makaf, else just print regular Makaf
\newcommand{\mynewline}{\\*}



%%%%%%%%%%%%%%%
% copied from https://tex.stackexchange.com/questions/14377/how-can-i-test-for-the-current-font

\usepackage{ifthen}
\makeatletter
\newcommand{\showfont}{encoding: \f@encoding{},
  family: \f@family{},
  series: \f@series{}
  shape: \f@shape{},
  size: \f@size{}
}
\newcommand{\iffont}[3]{\ifthenelse{\equal{\f@family}{#1}}{#2}{#3}}
\makeatother


%%%%%%%%%%%%%%%


\newcommand{\ערך}[1]{\textbf{{\mylarge #1}}}
\newcommand{\משנה}[1]{\textbf{{\mynormalsize #1}}}
\newcommand{\הגדרה}[1]{{\mynormalsize #1}}
% what about fakes? (bolded)
\newcommand{\מקור}[1]{{\mysmall #1}}

\newcommand{\צערך}[1]{\משנה{#1}}
\newcommand{\צמשנה}[1]{\textbf{{\mynormalsize #1}}}
\newcommand{\צהגדרה}[1]{{\mysmall #1}}
\newcommand{\צהגדרהמודגשת}[1]{\textbf{{\mysmall #1}}}
\newcommand{\צמקור}[1]{{\myfootnotesize #1}}


\newcommand{\תערך}[1]{\texttt{\ערך{#1}}}
\newcommand{\תמשנה}[1]{\texttt{\צמשנה{#1}}}
\newcommand{\תהגדרה}[1]{\texttt{\צהגדרה{#1}}}
\newcommand{\תמקור}[1]{\texttt{\צמקור{#1}}}

\newcommand{\מעוין}[1]{\mynormalsize{#1}}
\newcommand{\מעויןמרכזי}[1]{

     \centerline{\mynormalsize{#1}}
     \vspace{2mm}
}

\newcommand{\stamletter}[1]{\mystamsize\hebrewfontstam{#1}\hebrewfont}


%\newcommand{\תקלה}[1]{\הגדרה{#1}}
\newcommand{\תקלה}[1]{#1}
%%%%%%%%%%%        DATA        %%%%%%%%%%%

\begin{document}
\begin{multicols}{2}

\mybookname{מילון הראיה}{הראיה מילון}
\mylettertitle{א}
\ערך{אב }\הגדרה{- הנושא המוליד, המחולל את התולדות }\מקור{[ר״מ קיז]}\צהגדרה{. }

\ערך{אב }\הגדרה{- המקים את הבית, המדריך את התולדות, המאיר את ארחות חייהם בהשפעתו הרוחנית }\מקור{[שם קיח]}\צהגדרה{. }

\ערך{אב }\הגדרה{- הרועה הנאמן. מדריך, העומד במעלות נפשו הרבה יותר גבוה מהמעלה של הצעירות של }\הגדרה{הבנים\mycircle{°}}\הגדרה{, הצריכה לקבל את }\הגדרה{השפעתו\mycircle{°}}\הגדרה{ }\צהגדרה{[עפ״י }\צהגדרה{ע}\צהגדרה{״ר ב סה].}

\ערך{אבנט}\הגדרה{ - מכוון בתור אמצעי, בין החלק העליון מקום הכחות הנפשיים, לבין החלק התחתון שבגוף, מקום הכחות הגופניים השפלים, שמורה אמנם על היחש החזק שיש לכחות השפלים אל הכחות הנפשיים, עד }\הגדרה{שהקדושה\mycircle{°}}\הגדרה{ המעלה את הנטיות הנפשיות, פועלת להגביל יפה את סדרי הפעולות הטבעיות לצד המעלה והקדושה}\צהגדרה{ [}\צהגדרה{ע}\צהגדרה{״א ג ב ד].}\\\הגדרה{ע״ע חגורה, יסוד הויתה.}

\ערך{אבר }\הגדרה{- }\משנה{אבר הכנף\mycircle{°}}\הגדרה{ - הכח הפנימי המניע את העפיפה }\מקור{[עפ״י ע״א ב ח יב]}\צהגדרה{. }

\משנה{אגרת רב שרירא גאון}\צהגדרה{ - מסמך}\צהגדרה{-היסוד }\צהגדרה{לסדר ההשתלשלות של כל התורה}\צהגדרה{-שבעל}\צהגדרה{}\צהגדרה{-פה\mycircle{°}}\צהגדרה{, שהוא כמגדל בנוי לתלפיות של היהדות, עליו תלוי כל שלטי הגבורים במלחמתה של תורה ואמתת דורותיה. בסיס האמונים לחתימת תורת אמת של ״חיי עולם הנטועים }\צהגדרה{בתוכנו״\mycircle{°}}\צהגדרה{ מאז היותנו לעם ד׳ אלהינו, בקבלת מתנתה, ונמשכים לו באחרית חתימת התלמוד עם המשך דבריהם של }\צהגדרה{הגאונים\mycircle{°}}\צהגדרה{ מוסרי עניניו }\צמקור{[ל״י ב (מהדורת בית אל תשס״ג) מט, נא].}

\ערך{אד }\הגדרה{- ענן }\מקור{[ר״מ קיח]}\צהגדרה{. }

\ערך{אד }\הגדרה{- לישנא דתברא }\מקור{[ר״מ קיח]}\צהגדרה{. }

\ערך{״אדם״ }\הגדרה{- כנוי לגויה. ציור האדם השפל והנבזה על שם האדמה אשר לוקח משם, להוראת היות חומרו שפל מאד, כי הוא המדרגה הפחותה מן הדצח״מ שהוא הדומם }\מקור{[עפ״י ע״א יבמות סג.]}\צהגדרה{.}\\\ערך{״אדם״ }\הגדרה{- כנוי לנפש. ציור מדרגה גבוהה, כמו שכתוב }\הגדרה{״בצלם\mycircle{°}}\הגדרה{ אלקים עשה את האדם״, היינו מצד }\הגדרה{נשמתו\mycircle{°}}\הגדרה{ הרוממה, אשר היא נאצלת מתחת }\הגדרה{כסא-הכבוד\mycircle{°}}\הגדרה{, ועל שם ״אדמה לעליון״}\myfootnote{ ישעיה יד יד.\label{1}}\הגדרה{ }\מקור{[עפ״י ע״א יבמות סג.]}\צהגדרה{.}\\\הגדרה{ע״ע ״אנוש״. ע״ע גבר. ע״ע איש.}

\ערך{אדם }\הגדרה{- נפש שכלית קשורה בחומר }\מקור{[ע״א ג ב קצט]}\צהגדרה{. }\\\ערך{אדם }\הגדרה{- }\משנה{צורת\mycircle{°} האדם }\הגדרה{- }\הגדרה{המחשבה-העליונה\mycircle{°}}\הגדרה{ העושה את האדם לאדם, }\הגדרה{התורה\mycircle{°}}\הגדרה{ }\מקור{[ע״א ד ט יז]}\צהגדרה{. }\\\משנה{צורת האדם הפנימית }\הגדרה{- שכלו ומוסרו }\מקור{[פנ׳ א]}\צהגדרה{. }\\\ערך{אדם }\הגדרה{- }\משנה{כחו הרוחני }\הגדרה{- ע׳ במדור נפשיות, רוח, הכח הרוחני (של האדם).}\\\ערך{אדם}\הגדרה{ - }\משנה{סגולת\mycircle{°} האדם}\myfootnote{ \textbf{כונס בקרבו }\textbf{את}\textbf{ כל }\textbf{סגולת}\textbf{ ההויה }\textbf{וכו׳} - ש״ק קובץ א קעב: ״האדם הוא תמצית מלאה שההויה כולה משתקפת בו״.\label{2}}\הגדרה{ - מציינת את רוממותו הבאה בעקב שפלותו }\משנה{- }\הגדרה{יצור מושפל עד עמקי החומר, ועם זה כונס בקרבו את כל סגולת ההויה הרוחנית המלאה. שדוקא בהשתפלותו אל המורד הארצי הרי הוא רוכס את כל ההויה מראש היש עד סופו}\צהגדרה{ [}\צהגדרה{ע}\צהגדרה{״א ד ט קה].}\\\ערך{אדם }\הגדרה{- }\משנה{נשמת האדם בכל חגויה השונים}\הגדרה{ - פרח רז עולם (של) החיבור הנעלה שממנו מתגלה הכבדות }\הגדרה{הארצית\mycircle{°}}\הגדרה{ עם השאיפה השמימית המנצחתה, של שפעת החיים היציריים המשתפלת דרגה אחר דרגה, עד שיוצרת את }\הגדרה{החמריות\mycircle{°}}\הגדרה{, עם }\הגדרה{המאור-העליון\mycircle{°}}\הגדרה{, השפעה של הוית הישות, הרוחני, האצילי, השכלי, והמוסרי, הקדוש והמצוחצח }\מקור{[עפ״י א״ק ב תקכד]}\צהגדרה{.}\\\הגדרה{יצירה שבה מתגלה האור ההויתי בכל עזו ותקפו. הכח המרכזי, שההויה חודרת באורה כולה אל הויתו, ומשלמת את תכונתה על ידו }\מקור{[פנק׳ ג של]}\צהגדרה{.}\\\ערך{אדם }\הגדרה{- }\משנה{תעודת האדם שנוצר בגללה}\myfootnote{ \textbf{תעודת}\textbf{ האדם} - ע״ע ע״ר א קפא ד״ה מכלל ופרט וכלל. א״ק ב תקלד. ע׳ במדור מונחי קבלה ונסתר, ״תוספת״.\label{3}}\הגדרה{ - להוסיף }\הגדרה{אור\mycircle{°}}\הגדרה{ }\הגדרה{רצוני\mycircle{°}}\הגדרה{ }\הגדרה{עליון\mycircle{°}}\הגדרה{ }\הגדרה{בעזוז\mycircle{°}}\הגדרה{ החיים הפרטיים, }\הגדרה{להעלותם\mycircle{°}}\הגדרה{ אל }\הגדרה{עלוי\mycircle{°}}\הגדרה{ }\הגדרה{הכלל\mycircle{°}}\הגדרה{, ולהוסיף בכלל }\הגדרה{זיו\mycircle{°}}\הגדרה{ צביוני חדש ע״י עושר הבא ממשפלים. (לעסוק }\הגדרה{בתורה-לשמה\mycircle{°}}\הגדרה{) }\מקור{[א״ק א מד]}\צהגדרה{. }\\\הגדרה{להשלים את }\הגדרה{מלכות-שמים\mycircle{°}}\הגדרה{ }\צהגדרה{<שהיא מופיעה בכל היש בהדר גאונה, והולכת היא ומשתפלת בהעולמים המעשיים, בתהומות מאד עמוקים, בירידות מאד חשוכים>.}\הגדרה{ וברצונו הטוב והאיתן של האדם, שיצא אל הפועל בהיותו מתעלה להיות אוחז במשטר האלהי בהמון עולמים, }\צהגדרה{<שבפליאות נוראות נתגלה ע״י הבהקת אורם של אדירי הקודש שבדורות הקדמונים, ויצא במלא יקרתו בהתגלות האלהית שבאור }\צהגדרה{התורה\mycircle{°}}\צהגדרה{ }\צהגדרה{ונשמת-ישראל\mycircle{°}}\צהגדרה{, הפרטית והכללית, בין כל עמי הארץ>.}\הגדרה{ בכח חסון זה }\הגדרה{יתקן\mycircle{°}}\הגדרה{ האדם ויעלה את החלק הירוד }\הגדרה{שבמלכות\mycircle{°}}\הגדרה{ }\הגדרה{האצילות-האלהית\mycircle{°}}\הגדרה{, שירדה להיות מנהגת עולמי עד, בצורתם המוקצבה. ובזה יקשור נזר ועטרה למלכות שדי בכל העולמים כולם, והמגמה היצירתית תצא אל הפועל בכל יפעת אידיאליה, מראשית המחשבה עד סוף המעשה וכולה אומרת }\הגדרה{כבוד\mycircle{°}}\הגדרה{ }\מקור{[עפ״י קובץ ח קעב]}\צהגדרה{.}\\\משנה{מגמת היצירה האנושית }\הגדרה{- הנשמה החושבת, ההוגה דעה, }\הגדרה{המציירת\mycircle{°}}\הגדרה{ ציורי }\הגדרה{קודש\mycircle{°}}\הגדרה{ }\מקור{[א״ק ג שלד]}\צהגדרה{.}\\\משנה{תעודת האדם }\הגדרה{- להיות משכיל ובן חורין, }\הגדרה{מתענג-על-ד׳\mycircle{°}}\הגדרה{ ומתעלס בידיעת האמת ושמח }\הגדרה{בכבוד\mycircle{°}}\הגדרה{ יוצר כל }\מקור{[עפ״י קבצ׳ א נח]}\צהגדרה{.}\\\משנה{התפקיד האנושי }\הגדרה{- להיות איש חי מכיר ובעל השכלה }\מקור{[קובץ א קצה]}\צהגדרה{.}\\\הגדרה{ע״ע חיי האדם. ע׳ במדור פסוקים ובטויי חז״ל, צלם אלהים, חותם צלם אלהים מוטבע באדם. ע״ע דמות האדם. ע׳ במדור אדם הראשון, תעודת האדם.}

\ערך{אדנות מוחלטה }\הגדרה{- }\הגדרה{היכולת\mycircle{°}}\הגדרה{ }\הגדרה{החפשית\mycircle{°}}\הגדרה{ האין-סופית, המצויה תמיד בפועל }\הגדרה{בגבורה-של-מעלה\mycircle{°}}\הגדרה{, היא האדנות המוחלטה והמלוכה האמיתית שהיא עומדת למעלה מכל }\הגדרה{שם\mycircle{°}}\הגדרה{, מכל בטוי ומכל }\הגדרה{קריאה\mycircle{°}}\הגדרה{, שהרי האפשרות אין לה קץ ותכלית, והיכולת אין לה גבול והגדרה. }\הגדרה{מלכות-אין-סוף\mycircle{°}}\הגדרה{ במובן העליון, }\הגדרה{המלוכה-העליונה\mycircle{°}}\הגדרה{ }\מקור{[ע״ר א מו]}\צהגדרה{. }\\\הגדרה{ע׳ במדור שמות כינויים ותארים אלהיים, ״אדון עולם״}\myfootnote{ ע׳ עטרת ראש להרד״ב, שער ראש השנה סי׳ ה.\label{4}}\הגדרה{. }

\ערך{אדריכל }\הגדרה{- פועל (את) הבנין }\מקור{[א״ק ב שנ]}\צהגדרה{. }\\\צהגדרה{ }\\\משנה{אהבה }\צהגדרה{- ההתיחסות הנאמנה, הישרה, ההגונה, המתאימה אל האמת המציאותית, מתוך שייכות נכונה וזיקה רצויה, הכרה מלאה ושלמה של המציאות, של הענין שהיא מתייחסת אליו }\צמקור{[עפ״י ל״י ב רלד].}\\\צהגדרה{מצב גדלותי, רוחני, אינטלקטואלי הכרתי נשמתי, שייכות חיונית, קישור התדבקות והזדהות, מתוך חכמה אמיתית והכרה אמיתית }\צמקור{[עפ״י שי׳ 63, 4-5].}

\ערך{אהבה }\הגדרה{- עדן החיים, התשוקה האלהית של העלאת נר החיים }\מקור{[מ״ר 24]}\צהגדרה{. }\\\משנה{שלימות האהבה}\הגדרה{ - השמחה הגמורה ואור הנפש, שעמה כל טוב ואושר ובה כלולים נועם החכמה וההשגה ואהבתה }\צהגדרה{[}\צהגדרה{ע}\צהגדרה{״א א ד לו].}\\\מעוין{◊}\הגדרה{ }\הגדרה{האמונה\mycircle{°}}\הגדרה{ והאהבה הן עצם החיים בעוה״ז }\הגדרה{ובעוה״ב\mycircle{°}}\הגדרה{ }\מקור{[א׳ סט]}\צהגדרה{. }

\ערך{אהבה }\הגדרה{- }\משנה{שמרי האהבה }\הגדרה{- ע״ע תאוות. }

\ערך{אהבה }\הגדרה{- }\משנה{עבודת אהבה }\הגדרה{- זהירות בפרטי כל מצות ודקדוקי תורה מכח השפעת כללות התורה הדבקה בלב בחוזק והכרה ברורה }\מקור{[עפ״י א״ת ג ג]}\צהגדרה{.}\\\משנה{עבודת ד׳ וכל מעגל טוב מאהבה }\הגדרה{- מידיעת }\הגדרה{הטוב\mycircle{°}}\הגדרה{ הגנוז בהם }\מקור{[עפ״י ע״ר א רפו]}\צהגדרה{. }\\\הגדרה{מהכרה אמיתית אל הטוב והשלימות }\מקור{[ע״ר א שסז-ח (ע״א א ג לב)]}\צהגדרה{.}\\\משנה{באהבה}\הגדרה{ - בדרך חפץ פנימי והכרה עצמית }\מקור{[ל״ה 55]}\צהגדרה{. }\\\משנה{כח העבודה מאהבה}\הגדרה{ - }\מעוין{◊ }\הגדרה{אינו בא כי אם לפי מדת הידיעה הבאה בלימוד של קביעות ועשירות רבה במקצעות השונים של תורת }\הגדרה{המוסר\mycircle{°}}\הגדרה{ }\הגדרה{והיראה\mycircle{°}}\הגדרה{, שאי אפשר כלל להמצא מבלעדי לימוד בסדר נכון, למגרס תחילה בבקיאות מלמטה למעלה, ואחר כך למסבר בעומק עיון ודעה שלמה }\מקור{[ל״ה 188]}\צהגדרה{.}\\\הגדרה{ע״ע עבודה מאהבה, עבודת ד׳ מאהבה ותלמוד תורה-לשמה.}

\ערך{אהבה אלהית }\הגדרה{- }\משנה{האהבה האלהית העליונה, המבוסמת בבשמי הדעה העליונה }\הגדרה{- ההרגשה הנשמתית היותר חודרת ופנימית, אשר }\הגדרה{בכנסת-ישראל\mycircle{°}}\הגדרה{ בכללותה, בנשמות אישיה היחידים, }\הגדרה{בחביון-עז\mycircle{°}}\הגדרה{ נשמת כללותה, ובכל אשד הרוח המשתפך בכל פלגות תולדותיה }\מקור{[ע״א ד ט פח]}\צהגדרה{. }\\\הגדרה{}\הגדרה{אהבת-ד׳\mycircle{°}}\הגדרה{ }\הגדרה{אלהי-ישראל\mycircle{°}}\הגדרה{, עצם החיים (בישראל), }\הגדרה{נשמת-האומה\mycircle{°}}\הגדרה{ ועצם חייה }\צהגדרה{[}\צהגדרה{אג}\צהגדרה{׳ א מד].}\\\משנה{אהבה אלהית }\הגדרה{- הנטיה היותר }\הגדרה{חפשית\mycircle{°}}\הגדרה{ }\הגדרה{ונצחית\mycircle{°}}\הגדרה{ של רוח החיים, שהופעתה באה מסקירת הגודל הבלתי מוקצב, של }\הגדרה{אור\mycircle{°}}\הגדרה{ }\הגדרה{הקודש\mycircle{°}}\הגדרה{ המקיף עולמי נצח ממעל לכל חק וקצב, שאור }\הגדרה{החסד\mycircle{°}}\הגדרה{ }\הגדרה{הנאמן\mycircle{°}}\הגדרה{ מתעלה שם, השופע ויורד בכל מלא }\הגדרה{חנו\mycircle{°}}\הגדרה{, ממעל לכל חק }\הגדרה{ומשפט\mycircle{°}}\הגדרה{, וכל פנות שהוא פונה הכל הוא רק }\הגדרה{לטובה\mycircle{°}}\הגדרה{ }\הגדרה{ולברכה\mycircle{°}}\הגדרה{ לאור }\הגדרה{ולחיים\mycircle{°}}\הגדרה{, וכל מעשה וכל תנועה מחוללת אך }\הגדרה{נועם\mycircle{°}}\הגדרה{ }\הגדרה{והוד\mycircle{°}}\הגדרה{ קודש }\מקור{[עפ״י א״י כט, ע״ר א יד]}\צהגדרה{.}\\\משנה{האהבה העליונה }\צהגדרה{- }\צהגדרה{אהבת-עולם\mycircle{°}}\צהגדרה{ }\צהגדרה{ואהבה-רבה\mycircle{°}}\צהגדרה{, אשר לישראל את ד׳ אלהיהם }\צהגדרה{ואביהם-שבשמים\mycircle{°}}\צהגדרה{ מלך}\צהגדרה{-עולמים, הבוחר בעמו ומלמדו ומדריכו }\צמקור{[ל״י א (מהדורת בית אל תשס״ב) צג]. }\\\ערך{האהבה}\myfootnote{ \textbf{ההכרה האמיתית וכו׳ מכבוד}\textbf{-}\textbf{אל, הנשקף מהבריאה וכו׳ וכו׳ }\textbf{שומע קול ד׳ הקורא אליו }\textbf{וכו׳}\textbf{ ומרגיש שהוא}\textbf{ וכו׳ }\textbf{שואף את חייו יחד עם מקור}\textbf{-}\textbf{החיים}\textbf{,}\textbf{ וכל היצור כולו ניצב לו כאורגן שלם אדיר נחמד ואהוב, שהוא אחד מאבריו, המקבל מכולו ונותן לכולו, ויונק יחד עמו זיו חייו ממקור החיים} - ע׳ במדור שמות כינויים ותארים אלהיים, ״מלכנו״. ושם, ״אבינו״. ע״ע ע״ר א רמט, ד״ה ברוך. ושם, רפט ד״ה ברכנו. ושם ב ג ד״ה אמר ר׳ עקיבא. קבצ׳ ב קלז [87]. פנק׳ ב רד מט. ע״ע ״שמע״. ע׳ במדור פסוקים ובטויי חז״ל, ברוך שם כבוד מלכותו. (את ההבחנה בעניין האיר לי רוני שיין).\label{5}}\ערך{ }\הגדרה{- }\צהגדרה{תכלית התעודה האנושית}\הגדרה{. ההכרה האמיתית כשמתגברת באדם כראוי, }\הגדרה{מכבוד-אל\mycircle{°}}\הגדרה{ הכללי, הנשקף מכל }\הגדרה{הדר\mycircle{°}}\הגדרה{ הבריאה וסדריה }\הגדרה{הגשמיים\mycircle{°}}\הגדרה{ }\הגדרה{והרוחניים\mycircle{°}}\הגדרה{, בעבר, בהוה ובעתיד, }\צהגדרה{<שגם זה האחרון מוצץ הוא יפה למי שמבקש }\צהגדרה{ודורש-את-אלהים\mycircle{°}}\צהגדרה{ באמת וחפץ שלם>}\הגדרה{ אותה ההכרה כשהיא מתעצמת יפה באדם, רק היא מטבעת עליו את חותמו האמיתי, את אופיו הטבעי להקרא בשם }\הגדרה{אדם\mycircle{°}}\הגדרה{. }\צהגדרה{רק אז הוא מרגיש שהוא חי חיים נצחיים ומכובדים. <הוא מכיר כי הדרכים שהחיים מתגלים בהם, לפי ערכנו ביחש מצבנו החומרי, שונים המה, ובכל השינויים ההווים והעתידים לבבו }\צהגדרה{בוטח-בשם-ד׳\mycircle{°}}\צהגדרה{ אלהי עולם מחיה החיים }\צהגדרה{וחי-העולמים\mycircle{°}}\צהגדרה{> מצב נפש כזה כשהוא מתאים גם כן לכל סדרי החיים הפנימיים, הנפשיים והגופניים, חיי המשפחה והחברה, וכשהוא צועד בעוזו להיות גם כן מתפלש להיות המוסר הציבורי עומד על תילו ומכונו, אז הארץ מוכרחת להתמלא דעה, ותורת ד׳ היא נובעת ממעמקי הלב }\צהגדרהמודגשת{- }\צהגדרה{כל }\הגדרה{אדם שומע קול ד׳ הקורא אליו ושש ושמח לעשות רצון קונו וחפץ צורו, שהוא צורו הפרטי וצור העולמים כולם; ומרגיש הוא אז, שהוא האדם, שואף את חייו יחד עם }\הגדרה{מקור-החיים\mycircle{°}}\הגדרה{, וכל היצור כולו ניצב לו כאורגן שלם אדיר נחמד ואהוב, שהוא אחד מאבריו, המקבל מכולו ונותן לכולו, ויונק יחד עמו זיו חייו ממקור החיים }\מקור{[עפ״י ל״ה 149]}\צהגדרה{.}\\\משנה{מתק האהבה }\הגדרה{- רוחב }\הגדרה{הדעת\mycircle{°}}\הגדרה{, והנועם אשר }\הגדרה{לעדן-העליון\mycircle{°}}\הגדרה{ }\מקור{[א״ק ג ראש דבר כט]}\צהגדרה{. }\\\משנה{אהבת צור-העולמים\mycircle{°}}\הגדרה{ - זיו }\הגדרה{השכינה\mycircle{°}}\הגדרה{, הכרה שכלית והרגשית, ללכת }\הגדרה{בדרכי-ד׳\mycircle{°}}\הגדרה{ באהבת אמת והכרה עמוקה }\הגדרה{פנימית\mycircle{°}}\הגדרה{ }\מקור{[עפ״י ע״א ג ב נ]}\צהגדרה{. }\\\משנה{זיקי אהבת אלהים }\הגדרה{- מציאת }\הגדרה{אור-ד׳\mycircle{°}}\הגדרה{ בעומק רגש, בתוכן דעה }\מקור{[א״ק ג ריא]}\צהגדרה{. }\\\הגדרה{ע״ע אהבת ד׳. ע״ע אהבת ד׳ העליונה. ע״ע יראת הגודל. }

\ערך{אהבה לעומת טוב}\הגדרה{ - ע׳ בנספחות, מדור מחקרים.}

\ערך{אהבה מינית }\הגדרה{- ע׳ במדור הנטייה המינית.}

\ערך{אהבה קדושה }\הגדרה{- }\משנה{האהבה הקדושה}\הגדרה{ - }\הגדרה{אהבת-ד׳\mycircle{°}}\הגדרה{ וכל העולמים, אהבת כל היקום וכל היצור }\צהגדרה{[}\צהגדרה{א}\צהגדרה{״ק ג רעט-רפ].}

\ערך{״אהבה רבה״ }\הגדרה{- ע׳ במדור פסוקים ובטויי חז״ל.}

\ערך{אהבת אור ד׳}\הגדרה{ - אהבת החיים של }\הגדרה{הצדיק-האמיתי\mycircle{°}}\הגדרה{, <שאיננה כלל אותה הנטיה הגסה של אהבת החיים המרופדת בשכרון של נטיות החומר הגסים המצוי אצל רוב הבריות, כי אם> אהבת חיקוי }\הגדרה{לחסד-עליון\mycircle{°}}\הגדרה{ בעולמו, המתפשטת על פני כל היצור }\מקור{[קבצ׳ א קעד]}\צהגדרה{.}

\ערך{אהבת ד׳ }\הגדרה{- הרגשת השתוקקות תמיד }\הגדרה{לטוב\mycircle{°}}\הגדרה{ }\הגדרה{ולאמת\mycircle{°}}\הגדרה{ שהאדם מרגיש באמת בנקודת }\הגדרה{נשמתו\mycircle{°}}\הגדרה{ הפנימית, <שהכל הוא בכלל טוב או בכלל אמת> }\מקור{[עפ״י קבצ׳ ב קלג]}\צהגדרה{.}\\\הגדרה{(האהבה) מצד }\הגדרה{כבודו\mycircle{°}}\הגדרה{ }\הגדרה{וחסדיו\mycircle{°}}\הגדרה{ שעשה }\מקור{[מא״ה א פט]}\צהגדרה{. }\\\צהגדרה{גלויה היסודי של }\צהגדרה{האמונה\mycircle{°}}\צהגדרה{ הגדולה }\צמקור{[נ״ה יא].}\\\מעוין{◊ }\משנה{אהבת ד׳}\הגדרה{ באה כשישים האדם לבבו להדמות }\הגדרה{לדרכי\mycircle{°}}\הגדרה{ }\הגדרה{השי״ת\mycircle{°}}\הגדרה{, אז ע״י ההדמות תולד האהבה, וכפי רוב הדמיון יהי׳ רוב האהבה }\מקור{[ע״א א ב לו (ע״ר ב קכג)]}\צהגדרה{. }\\\מעוין{◊}\הגדרה{ האהבה באה מצד השלמות שבנמצאים, שמצדה הם כולם נמצאים באמיתת מציאותו יתברך }\מקור{[ע״א ג ב קעא]}\צהגדרה{.}\\\מעוין{◊ }\הגדרה{דעת כל המציאות לאמתתה לכל סעיפיה, כפי היכולת לאדם, <שמכלל }\הגדרה{דעת-ד׳\mycircle{°}}\הגדרה{>. דעת הטבע לכל סעיפיו גיאוגרפיה והתכונה, הרפואה וחכמת הנפש, תכונות העמים וכל הנלוה להם, המביאים גם כן }\הגדרה{לאהבה-העליונה\mycircle{°}}\הגדרה{ הזכה בכללות האנושיות }\מקור{[עפ״י קבצ׳ ב קלא]}\צהגדרה{.}\\\הגדרה{ע״ע אהבה אלהית. ע״ע יראת ד׳. ע״ע בנספחות, מדור מחקרים, אהבה ויראה. }

\ערך{אהבת ד׳ העליונה }\הגדרה{- אהבת השלמות המוחלטת והגמורה של }\הגדרה{סיבת\mycircle{°}}\הגדרה{ הכל, מחולל כל ומחיה את כל }\מקור{[א״ק ב תמב]}\צהגדרה{. }\\\משנה{אהבת ד׳ הבהירה }\הגדרה{- האהבה המרוממה והעדינה }\הגדרה{לאין-סוף\mycircle{°}}\הגדרה{ }\מקור{[קובץ ה צה]}\צהגדרה{. }\\\משנה{אהבת השי״ת }\הגדרה{- }\מעוין{◊}\הגדרה{ בהכרת האמת של המציאות האלהית מצד עצמה, המקור }\הגדרה{לשמו-הגדול\mycircle{°}}\הגדרה{ ב״ה }\מקור{[קבצ׳ א קלז]}\צהגדרה{. }\\\משנה{אור אהבת ד׳ }\הגדרה{- }\הגדרה{עדן\mycircle{°}}\הגדרה{ החיים, מגמת החיים, עצם החיים, בהירות החיים, ומעין חיי החיים, עליון מכל הגה, מכל רצון והסברה, מכל שאיפה }\הגדרה{פנימית\mycircle{°}}\הגדרה{, ומכל }\הגדרה{הזרחה\mycircle{°}}\הגדרה{ }\הגדרה{יפעתית\mycircle{°}}\הגדרה{, הכל בה, והכל ממנה }\מקור{[קובץ ו רמא]}\צהגדרה{.  }\\\הגדרה{ע׳ במדור פסוקים ובטויי חז״ל, אהבה רבה. ע״ע אהבת שם ד׳. }

\משנה{״אהבת חנם״}\myfootnote{ ע׳ א״ק ג שכד.\label{6}}\הגדרה{ }\צהגדרה{- אהבה שגם כשיש במציאות דברים שכאילו מעכבים לה אעפ״כ תתגבר על כולם ותקבע-חנם }\צמקור{[ל״י א קיג]. }\\\צהגדרה{אהבה שאינה תלויה בדבר <כאהבת ד׳ לישראל, ברית עולם> }\צמקור{[ק״ת נה].}

\ערך{״אהבת חסד״ }\הגדרה{- ע׳ במדור פסוקים ובטויי חז״ל.}

\ערך{״אהבת חסד״}\הגדרה{ - }\משנה{(לעומת ״תורת חיים״)}\הגדרה{ - ע׳ במדור פסוקים ובטויי חז״ל.}

\ערך{״אהבת עולם״ }\הגדרה{- ע׳ במדור פסוקים ובטויי חז״ל.}

\ערך{אהבת שם ד׳ }\הגדרה{- אהבת הלימוד והידיעה של מציאות השי״ת ודרכיו, וכל המכשירים המביאים לזה }\מקור{[קבצ׳ א קלו]}\צהגדרה{. }\\\הגדרה{ע״ע אהבת ד׳. ע״ע אהבת ד׳ העליונה.}

\ערך{אהבת תורה }\הגדרה{- ע׳ במדור תורה.}

\ערך{אוביקטיבי }\הגדרה{- חיצוני}\myfootnote{ ע׳ בנספחות, מדור מחקרים, אוביקטיבי סוביקטיבי. ושם, חיצון, עולם חיצון.\label{7}}\הגדרה{ }\מקור{[עפ״י א״ק ג צג]}\צהגדרה{.}\\\צהגדרה{ע״ע סוביקטיבי.}

\ערך{אוהל }\הגדרה{- בית הדירה, העלול להיות מוכן למסעות, שמרשם בתוכן הרוחני העליון את העליות הנכספות. מסמן את היסוד המטלטל, את הצביון של ההכנה אשר לתנועה, שכונתה היא תמיד השתנות ועליה לצד }\הגדרה{האושר-העליון\mycircle{°}}\הגדרה{, לקראת }\הגדרה{הזיו\mycircle{°}}\הגדרה{ של מעלה}\צהגדרה{ [}\צהגדרה{ע}\צהגדרה{״ר א מג].}\\\הגדרה{ע״ע משכן.}

\ערך{״אוהל״ לעומת ״בית״ }\הגדרה{- ע׳ במדור פסוקים ובטויי חז״ל,}\משנה{ ״}\הגדרה{יושב בבית״ לעומת ״יושב אוהל״.}

\ערך{אויב }\הגדרה{- מי שהשנאה (אצלו) בכח לא בפועל }\מקור{[מ״ש שכז]}\צהגדרה{.}\\\הגדרה{מבקש רעה בציורו ונטיתו הרוחנית }\מקור{[ע״ר א לד]}\צהגדרה{.}\\\הגדרה{ע״ע קם להרע.}

\ערך{אולפן }\הגדרה{- לימוד <בתרגום> }\מקור{[ר״מ ב]}\צהגדרה{. }

\ערך{אומה }\הגדרה{- }\משנה{טבע האומה, הרוחני והחומרי }\הגדרה{- הטבע הפסיכולוגי של האומה, וטבע התולדה והמורשה של האבות והגזע, וטבע הגיאוגרפי של ארץ נחלתה }\מקור{[עפ״י קבצ׳ ב מה (ב״ר שכו-ז)]}\צהגדרה{. }\\\ערך{אומה }\הגדרה{- }\משנה{צורת\mycircle{°} האומה }\הגדרה{- נשמתה ואורח חייה }\מקור{[עפ״י ע״א ד ה סא]}\צהגדרה{. }\\\הגדרה{ע״ע עמים, הצד המהותי בחיי העמים.}

\ערך{אומה }\הגדרה{- }\משנה{האומה (הישראלית) כולה בצרופה הכללי }\הגדרה{- אֵם החיים שלנו. האופן }\הגדרה{הכללי\mycircle{°}}\הגדרה{ של כל }\הגדרה{ישראל\mycircle{°}}\הגדרה{ בתור גוש אחד, המחבר את כל האישים הפרטיים להיות }\הגדרה{לעם\mycircle{°}}\הגדרה{ אחד, הכולל ג״כ את כל הדורות כולם בהערכה אחת }\מקור{[עפ״י א׳ עו, ע״ר ב פד]}\צהגדרה{. }\\\הגדרה{ע״ע עם. ע״ע גוי. }

\ערך{אומה }\הגדרה{- }\משנה{רוח האומה היחידי}\הגדרה{ - השאיפה אל הטוב האלהי המונח בטבע נשמתה }\מקור{[א׳ נב]}\צהגדרה{.}\\\הגדרה{ע״ע רוח ישראל. ע״ע רוח ד׳. ע׳ במדור תורה, תורה שבכתב, תורה שבכתב ברום תפארתה ותורה שבעל פה שניהם יחד.}

\ערך{אומה }\הגדרה{- }\משנה{שכינת האומה }\הגדרה{- רוח החיים של השאיפה האלהית המקושרת בתוכן הסגנון הצבורי של הצורה הלאומית }\מקור{[א׳ קו]}\צהגדרה{. }\\\הגדרה{ע׳ במדור מונחי קבלה ונסתר, ״שושנה עליונה״. }

\ערך{אומה }\הגדרה{- }\משנה{ישראל}\הגדרה{ - }\הגדרה{כנסת-ישראל\mycircle{°}}\הגדרה{ המוגבלה בגבול נחלת ישראל }\מקור{[א׳ מב]}\צהגדרה{.}\\\הגדרה{מקום מנוחתה של }\הגדרה{האידיאה-האלהית\mycircle{°}}\הגדרה{ על המרחב ההיסתורי הכללי }\מקור{[א׳ קח]}\צהגדרה{.}\\\ערך{אומה הישראלית}\הגדרה{ - }\משנה{התכלית הכללית של האומה הישראלית}\הגדרה{ - להודיע את }\הגדרה{שם-ד׳\mycircle{°}}\הגדרה{ בעולם כולו ע״י מציאותה והנהגתה }\מקור{[ל״ה 119 (פנק׳ ב עו)]}\צהגדרה{.}\\\הגדרה{}\הגדרה{חטיבה\mycircle{°}}\הגדרה{ אחת בעולם, המצויינת בתקותה לעצמה לא בשביל עצמה, כ״א בשביל הטוב הכללי, שהוא חן השכל הטוב, }\הגדרה{המוסר\mycircle{°}}\הגדרה{ }\הגדרה{והיושר\mycircle{°}}\הגדרה{ האמיתי, שא״א שיבנה כ״א ע״י }\הגדרה{תיקון-עולם-במלכות-שדי\mycircle{°}}\הגדרה{}\מקור{ [ע״ר א שפו (ע״א ב ט רצ)]}\צהגדרה{.}\\\הגדרה{ע״ע ישראל, מהותם העצמית הנותנת להם את אופים המיוחד.}

\ערך{אומה כללית }\הגדרה{- }\הגדרה{תמצית\mycircle{°}}\הגדרה{ של המין האנושי הפועלת עליו בלי הרף בעיבוד }\הגדרה{צורתו\mycircle{°}}\הגדרה{ }\הגדרה{הרוחנית\mycircle{°}}\הגדרה{ }\מקור{[קובץ ה קצו]}\צהגדרה{.}\\\הגדרה{ע׳ במדור פסוקים ובטויי חז״ל, עם לבדד.}

\ערך{אוצר החיים }\הגדרה{- }\הגדרה{אורה-של-תורה\mycircle{°}}\הגדרה{ במקורה }\מקור{[ע״ר א קמז]}\צהגדרה{. }\\\הגדרה{הצד העליון של התורה, היקר בעצמו מכל החיים כולם, }\משנה{אוצר חיים}\הגדרה{ עליונים נעלים ונשאים מכל חיי זמן ועולם}\מקור{ [ע״א ד ט ז]}\צהגדרה{.}\\\הגדרה{ע׳ במדור מונחי קבלה ונסתר, ״אורייתא מבינה נפקת״. ע׳ במדור תורה, תורה, שורש התורה. }

\ערך{אוצר הטוב }\הגדרה{- מקור }\הגדרה{חי-העולמים\mycircle{°}}\הגדרה{ }\מקור{[אג׳ א קי]}\צהגדרה{.}

\ערך{אוצר עליון}\הגדרה{ - }\משנה{האוצר העליון}\הגדרה{ - מקור }\הגדרה{הברכות\mycircle{°}}\הגדרה{}\צהגדרה{ [}\צהגדרה{א}\צהגדרה{״ק א קיט].}

\ערך{אור }\הגדרה{- יסוד ואומץ המשכת החיים }\מקור{[עפ״י א״ק ב רצז (ע״ט טז)]}\צהגדרה{. }\\\הגדרה{כל יסוד החיים, חיי החיים, }\הגדרה{זיום\mycircle{°}}\הגדרה{ }\הגדרה{ותפארתם\mycircle{°}}\הגדרה{ }\מקור{[עפ״י ע״א ד יא יג]}\צהגדרה{. }\\\הגדרה{כח }\הגדרה{הרוחניות\mycircle{°}}\הגדרה{ של השכל הגדול, של החפץ הכביר, של המרץ }\הגדרה{הנשגב\mycircle{°}}\הגדרה{ }\מקור{[מ״ר 296 (קבצ׳ ב עא)]}\צהגדרה{. }\\\משנה{האור הגדול הכללי}\צהגדרה{ - }\צהגדרה{שלמות החיים ובריאותם הנמשכת ממקור אמתתם, המתגלה על ידי כל פרטיותם של דברי התורה, טיפולם וקליטתם, במלא כל הנפש ובכל תפוצות חדריה }\צמקור{[א״}\צהגדרה{ל}\צמקור{ מג].}\\\הגדרה{ע״ע ״אור החיים״. ע׳ במדור מונחי קבלה ונסתר, אורות. }\\\ערך{אור }\הגדרה{- }\משנה{האור בעצם }\הגדרה{- }\הגדרה{אור-חדש\mycircle{°}}\הגדרה{ של }\הגדרה{תשובה-עליונה\mycircle{°}}\הגדרה{, המ״ט }\הגדרה{שערי\mycircle{°}}\הגדרה{ }\הגדרה{בינה\mycircle{°}}\הגדרה{ [}\צהגדרה{ח״פ לב:].}\\\צהגדרה{גילוי אמיתת המציאות המשוכללת }\צהגדרה{בהופעת\mycircle{°}}\צהגדרה{ הקרנת }\צהגדרה{הזרחתה\mycircle{°}}\צהגדרה{ [עפ״י}\צמקור{ פנק׳ א תרלו (}\צהגדרה{ב״א ד יא}\צמקור{)}\צהגדרה{]. }\\\הגדרה{ע׳ בנספחות, מדור מחקרים, אור, זוהר, זיו. ושם, אור, זיו, ברק. }

\ערך{אור }\הגדרה{- }\משנה{(לעומת כלי\mycircle{°}) }\הגדרה{- נשמתו הרוחנית של הכלי <שהוא לבושו המעשי החיצון> }\מקור{[עפ״י א׳ קנח]}\צהגדרה{.}\\\הגדרה{התוכן (לעומת הסגנון)}\מקור{ [ע״א ד יב ה]}\צהגדרה{.}\\\הגדרה{החיים העצמיים של מחשבת ההויה (לעומת ההויה) }\מקור{[עפ״י ע״ר א כו, וא״ק ד ת (א״ה 1098)]}\צהגדרה{. }\\\הגדרה{אצילות האלהות בתור נפש ההויה, (מבחינתה הפנימית), בדיבורים מצד הסתכלות השירית שברוח הקודש }\מקור{[עפ״י א״ק ב שמח]}\צהגדרה{. }\\\צהגדרה{משמעות עמוקה, תוכן }\צהגדרה{רוחני\mycircle{°}}\צהגדרה{ }\צמקור{[פנק׳ א תרלז (שי׳ 6, 25)]. }

\ערך{אור }\הגדרה{- }\משנה{(לעומת חיים\mycircle{°}) }\הגדרה{- }\הגדרה{דעה\mycircle{°}}\הגדרה{, רוח-הבטה }\מקור{[עפ״י א׳ יא]}\צהגדרה{. }

\ערך{׳אור׳ לעומת ׳מאור׳ }\הגדרה{- ע׳ בנספחות, מדור מחקרים. }

\ערך{אור}\הגדרה{ - כללות הרגשה וידיעת מציאות. ערך }\הגדרה{ההשגה\mycircle{°}}\הגדרה{ והרצון הגמור }\צהגדרה{<כי מה שלמעלה מההשגה האנושית אין לקרות כ״א בשם }\צהגדרה{חושך\mycircle{°}}\צהגדרה{ מצד ההעלם, וכשיש העלם לחושך של מעלה, נגבל בגדר השגה ונעשה }\צהגדרהמודגשת{אור}\צהגדרה{>}\צמקור{ [עפ״י מא״ה ד כא-כב]}\צהגדרה{.}

\ערך{אור }\הגדרה{- ההרגשה הנפשית וההבנה של הידיעה }\מקור{[ע״א ב ט קכד]}\צהגדרה{.}

\ערך{אור }\הגדרה{- }\משנה{אוצר האור }\הגדרה{- חיי החיים העליונים, מקור כל החיים ושרש כל ההויות }\מקור{[ע״ר א סז]}\צהגדרה{. }\\\ערך{אור }\הגדרה{- }\משנה{האור הפנימי (של המושג מאורו של אלקים-חיים, צור ישעינו, לגדולי המשיגים) }\הגדרה{- החיים האמיתיים שאין לנו מלה ליחסם, כמו שהם נמצאים }\הגדרה{במקור-החיים\mycircle{°}}\הגדרה{ יתברך שמו }\מקור{[עפ״י ע״א ג ב נב]}\צהגדרה{. }

\ערך{אור אין סוף }\הגדרה{- ע׳ במדור מונחי קבלה ונסתר. או במדור שמות כינויים ותארים אלהיים.}

\ערך{אור אין סופי}\הגדרה{ - }\משנה{האור האין סופי}\הגדרה{ - ההארה }\הגדרה{האלהית\mycircle{°}}\הגדרה{ המקיפה והממלאה את כל, את כל }\הגדרה{הנשמות\mycircle{°}}\הגדרה{ ואת כל }\הגדרה{העולמים\mycircle{°}}\הגדרה{ }\מקור{[פנק׳ א שצט]}\צהגדרה{.}

\ערך{אור ״אל עליון קונה שמים וארץ״}\myfootnote{ בראשית יד יט.\label{8}}\הגדרה{ - }\הגדרה{החפץ-האלהי\mycircle{°}}\הגדרה{, המהוה את היש כולו, המעמידו ומחייהו, הדוחפו לעילוייו בכל קומתו המעשיית והרוחנית מריש דרגין עד סופם. }\הגדרה{הנבואה-העליונה\mycircle{°}}\הגדרה{ של פה אל פה אדבר בו, הנשפעת }\הגדרה{לנאמן\mycircle{°}}\הגדרה{ }\הגדרה{בית\mycircle{°}}\הגדרה{, להקים עדות ביעקב לעולמי עולמים, לקומם תבל ומלאה, בנשמת ד׳ יוצר כל }\מקור{[עפ״י ע״א ד ט טז]}\צהגדרה{. }

\ערך{אור אלהי }\הגדרה{- }\הגדרה{אור\mycircle{°}}\הגדרה{ }\הגדרה{האמת\mycircle{°}}\הגדרה{ }\הגדרה{הצדק\mycircle{°}}\הגדרה{ }\הגדרה{והדעת\mycircle{°}}\הגדרה{ }\מקור{[ע״ה קכח]}\צהגדרה{. }\\\הגדרה{}\הגדרה{זוהר\mycircle{°}}\הגדרה{ גדול של שכל בהיר וחשק אדיר של רצון כביר מאד }\מקור{[א״ק ג רטז]}\צהגדרה{. }

\ערך{אור אלהי }\הגדרה{- }\משנה{האור האלהי }\הגדרה{- המגמה }\הגדרה{השעשועית\mycircle{°}}\הגדרה{ }\הגדרה{הפנימית\mycircle{°}}\הגדרה{ של היצירה כולה, }\הגדרה{המזריחה\mycircle{°}}\הגדרה{ }\הגדרה{באור\mycircle{°}}\הגדרה{ }\הגדרה{יפעתה\mycircle{°}}\הגדרה{ על פני כל היקום, מחייה הפנימיים }\מקור{[א״ק ג קפח]}\צהגדרה{. }\\\משנה{נועם אור אלוה נורא הוד }\הגדרה{- מקור הנעימות ומעין }\הגדרה{העדנים\mycircle{°}}\הגדרה{, אוצר }\הגדרה{ההופעות\mycircle{°}}\הגדרה{ ומקור מקורות החיים }\מקור{[ר״מ עו]}\צהגדרה{. }\\\משנה{האור האלהי }\הגדרה{-  }\הגדרה{מקור-החיים\mycircle{°}}\הגדרה{ ומקור כל העדן ורוממות כל }\הגדרה{אושר-עליון\mycircle{°}}\הגדרה{. אור }\הגדרה{חיי-החיים\mycircle{°}}\הגדרה{ }\מקור{[עפ״י קבצ׳ א רטז (ג״ר 124)]}\צהגדרה{. }\\\הגדרה{חיי החיים }\מקור{[ע״א ג ב רכו]}\צהגדרה{.}\\\הגדרה{מקור החיים }\הגדרה{והשמחה\mycircle{°}}\הגדרה{ }\מקור{[ע״א ג ב צט]}\צהגדרה{.}\\\משנה{מקור האור האלהי}\הגדרה{ - נחל עדנים שאין לו סוף, ומקור עדן נצחי לכל נשמת חיים, המהפך את הכל }\הגדרה{לאור-חיים\mycircle{°}}\הגדרה{. המאור הפנימי, הכח הכמוס האלהי שיש במגמת הוייתה של האומה בעולם, שהוא הסוד של כל ההויה כולה }\מקור{[עפ״י קבצ׳ א קעה]}\צהגדרה{.}\\\משנה{אור אלהי עליון }\הגדרה{- המרחב של אין סוף לבהירות והשלמת חיי עולמים בעד הכל }\מקור{[ע״א ד ט סה]}\צהגדרה{. }\\\הגדרה{ע״ע אור עליון. ע״ע אור ד׳.}\myfootnote{ \textbf{אור}\textbf{ אלהי, אור אלהים, אור ד׳, אור עליון }- בין מושגים אלה התקשתי למצוא הבדל, מכל מקום חולקו ההגדרות למחלקות שונות על פי המונחים השונים.\label{9}}\הגדרה{ ע׳ במדור שמות כינויים ותארים אלהיים, אלהי, המקור האלהי. }

\ערך{אור אלהי }\הגדרה{- }\משנה{האור האלהי }\הגדרה{- }\הגדרה{נשמת-האומה\mycircle{°}}\הגדרה{ השרשית }\מקור{[א׳ קנח]}\צהגדרה{. }\\\הגדרה{}\הגדרה{הזיו\mycircle{°}}\הגדרה{ }\הגדרה{הטהור\mycircle{°}}\הגדרה{ הממלא נפשות טהורות }\מקור{[ע״א ג ב נ]}\צהגדרה{. }

\ערך{אור\mycircle{°} אלהים\mycircle{°} }\הגדרה{- תעודת ההויה, מקור }\הגדרה{הנשמות\mycircle{°}}\הגדרה{, }\הגדרה{מלא-כל\mycircle{°}}\הגדרה{, רוח }\הגדרה{ישראל\mycircle{°}}\הגדרה{ המופשט }\מקור{[עפ״י א״ת יב א]}\צהגדרה{. }\\\הגדרה{אור החיים היותר יפים, היותר }\הגדרה{טהורים\mycircle{°}}\הגדרה{ היותר }\הגדרה{מאירים\mycircle{°}}\הגדרה{ }\מקור{[ע״א ד ו מ]}\צהגדרה{. }

\ערך{אור ד׳\mycircle{°}}\הגדרה{ - }\הגדרה{העילוי\mycircle{°}}\הגדרה{ }\הגדרה{העליון\mycircle{°}}\הגדרה{, שממעל }\הגדרה{למקור-החיים\mycircle{°}}\הגדרה{, יסוד המרחב העליון של }\הגדרה{הזוהר\mycircle{°}}\הגדרה{ הבלתי מוגבל שכל }\הגדרה{עולמי-עולמים\mycircle{°}}\הגדרה{ אינם כדאיים לו, שהוא מובדל מכל אורות עולמים, שכל תכונה של אורה בהם הרי היא מכוונת לראות על ידה גופים חשכים, שבעצמם אינם מערך מהות האורה, אבל האור בעצמו איננו דבר נראה, כי לא נתגלה בעולם לפי מדתו הכח הרואה את מהות האור. אמנם }\משנה{אור ד׳}\הגדרה{ במעלת הרחבת }\הגדרה{אצילות\mycircle{°}}\הגדרה{ מקורו, הוא האור שאור נראה בו ועל ידו }\מקור{[עפ״י ע״ר א כא]}\צהגדרה{. }\\\הגדרה{אור האורים, שאי-אפשר לנו לבטאו ואיננו יכול להתלבש באותיות של שום מבטא גם לא של שום רעיון }\מקור{[א׳ קלא]}\צהגדרה{. }\\\הגדרה{מקור }\הגדרה{חיי-החיים\mycircle{°}}\הגדרה{ ב״ה }\מקור{[ע״ר א קנה]}\צהגדרה{. }\\\הגדרה{}\הגדרה{חיי-החיים\mycircle{°}}\הגדרה{, היסוד העליון מקור חיי אור }\הגדרה{העולמים\mycircle{°}}\הגדרה{ }\מקור{[עפ״י א״ק ג צה]}\צהגדרה{. }\\\הגדרה{יסוד כל היש, ויותר מכל היש באין קץ }\מקור{[קובץ א תתיא]}\צהגדרה{.}\\\הגדרה{}\הגדרה{צרור-החיים\mycircle{°}}\הגדרה{ }\מקור{[שם רמ]}\צהגדרה{. }\\\הגדרה{אור }\הגדרה{האמת\mycircle{°}}\הגדרה{ }\מקור{[עפ״י ע״א ב ט קנא]}\צהגדרה{. }\\\הגדרה{אלהי עולם. }\הגדרה{הטוהר\mycircle{°}}\הגדרה{, }\הגדרה{הטוב-המוחלט\mycircle{°}}\הגדרה{, האמת המזהרת, }\הגדרה{הנצח\mycircle{°}}\הגדרה{ בכל מלא }\הגדרה{הודו\mycircle{°}}\הגדרה{ }\מקור{[עפ״י א״ק א קפב]}\צהגדרה{.}\\\משנה{אור ד׳ מחולל כל }\הגדרה{- זוהר האמת, }\הגדרה{הוד\mycircle{°}}\הגדרה{ }\הגדרה{אור-החיים\mycircle{°}}\הגדרה{, }\הגדרה{שבמקור-הקודש\mycircle{°}}\הגדרה{ }\מקור{[עפ״י א״ק א ג (מ״ר 402)]}\צהגדרה{. }\\\משנה{אור ד׳ וכבודו}\הגדרה{\mycircle{°} - הקודש-העליון }\מקור{[מ״ר 345]}\צהגדרה{. }\\\משנה{אור ד׳ העליון }\הגדרה{- כולל הכל, ומקור הכל וחיי כל }\מקור{[ע״ר א רח]}\צהגדרה{. }\\\הגדרה{ע״ע אור עליון. ע״ע אור אלהי.}\footref{9} \\\ערך{אור ד׳ ממרומיו\mycircle{°}}\הגדרה{ - }\הגדרה{היש-העליון\mycircle{°}}\הגדרה{, }\הגדרה{הרוחניות\mycircle{°}}\הגדרה{ }\הגדרה{והטוהר\mycircle{°}}\הגדרה{ המעולה }\מקור{[עפ״י א״ק ג רפו]}\צהגדרה{. }\\\משנה{אור ד׳ }\הגדרה{- אור פני המלך המתנשא לכל לראש מעל כל ענין העולמות }\מקור{[ע״ר א רפט]}\צהגדרה{.}\\\הגדרה{}\הגדרה{הארת\mycircle{°}}\הגדרה{ היש האמיתי }\הגדרה{וזיו\mycircle{°}}\הגדרה{ החיים האלהיים }\מקור{[ע״א ד ט מז]}\צהגדרה{. }\\\הגדרה{הקדושה השרשית העצמית, המצואה בפועל, }\הגדרה{הוד\mycircle{°}}\הגדרה{ חיי }\הגדרה{הקודש\mycircle{°}}\הגדרה{, המרומם ונשא מכל שרעף ורעיון }\מקור{[עפ״י ע״ר א ט]}\צהגדרה{. }\\\הגדרה{}\הגדרה{הנס\mycircle{°}}\הגדרה{ המוחלט }\מקור{[ע״ר א מט]}\צהגדרה{.}\\\הגדרה{האור העליון שהוא הרבה למעלה מן }\הגדרה{הטבעיות\mycircle{°}}\הגדרה{}\מקור{ [ע״א ד ט מא]}\צהגדרה{.}\\\הגדרה{}\הגדרה{טוב-העליון\mycircle{°}}\הגדרה{ }\מקור{[ע״ר א שעב, רפט]}\צהגדרה{. }\\\ערך{אור ד׳ המהוה הישות }\הגדרה{- }\הגדרה{זרוע-ד׳\mycircle{°}}\הגדרה{ אשר נגלתה, יסוד ההשתלמות הבלתי פוסקת }\מקור{[עפ״י א״ק ב תקל]}\צהגדרה{. }\\\משנה{אור ד׳ וכבודו }\הגדרה{- אמיתת }\הגדרה{הרצון-הכללי\mycircle{°}}\הגדרה{ אשר }\הגדרה{בנשמת\mycircle{°}}\הגדרה{ היקום כולו }\מקור{[שם ג לט]}\צהגדרה{. }\\\משנה{אור ד׳ }\הגדרה{- המגמה האלהית היותר ברורה ותהומית לאין חקר }\מקור{[פנק׳ ג שלא]}\צהגדרה{.}\\\הגדרה{}\הגדרה{נשמת-העולמים\mycircle{°}}\הגדרה{ }\צהגדרה{[קבצ׳ ב קנה}\משנה{]}\צהגדרה{.}\\\הגדרה{שפעת החיים הנובעים ושוטפים ממקור חיי העולמים }\מקור{[קבצ׳ א רכג]}\צהגדרה{.}\\\משנה{אור ד׳ בעולמו }\הגדרה{- אור }\הגדרה{השכינה\mycircle{°}}\הגדרה{, נשמת העולמים, הוד }\הגדרה{האידיאליות\mycircle{°}}\הגדרה{ האלהית החיה בכל }\מקור{[א״ק ב שסח, א״ש יד ד]}\צהגדרה{. }\\\הגדרה{}\הגדרה{אורו-של-משיח\mycircle{°}}\הגדרה{ }\מקור{[א״ק ב תקסא]}\צהגדרה{. }\\\משנה{אור ד׳ }\הגדרה{- }\הגדרה{גאולה\mycircle{°}}\הגדרה{ רוחנית עליונה, נהירה אל }\הגדרה{ד׳\mycircle{°}}\הגדרה{ ואל טובו }\מקור{[א׳ צא]}\צהגדרה{.}\\\הגדרה{״אורן של ישראל״, }\הגדרה{רוח-ה׳\mycircle{°}}\הגדרה{ השורה על }\הגדרה{כלל\mycircle{°}}\הגדרה{-ישראל }\הגדרה{ותפארתם\mycircle{°}}\הגדרה{ הכללית }\מקור{[ע״א א ד לג]}\צהגדרה{. }\\\הגדרה{}\הגדרה{אור-תורה\mycircle{°}}\הגדרה{, }\הגדרה{אור-חיים\mycircle{°}}\הגדרה{ }\מקור{[קובץ ו קפח]}\צהגדרה{.}\\\משנה{אור ד׳ אשר באומה\mycircle{°}}\הגדרה{ - }\הגדרה{השראת-השכינה\mycircle{°}}\הגדרה{ }\הגדרה{וקדושת\mycircle{°}}\הגדרה{ }\הגדרה{התורה\mycircle{°}}\הגדרה{ }\הגדרה{והמצוה\mycircle{°}}\הגדרה{ }\מקור{[עפ״י ע״ר א קעג]}\צהגדרה{. }\\\משנה{אור ד׳ בעולם }\הגדרה{- }\הגדרה{האמת\mycircle{°}}\הגדרה{ }\הגדרה{והצדק\mycircle{°}}\הגדרה{ של דעת }\הגדרה{הקודש\mycircle{°}}\הגדרה{ }\מקור{[ע״ר א רו]}\צהגדרה{. }\\\משנה{אור ד׳ וטובו }\הגדרה{- }\הגדרה{הטוב\mycircle{°}}\הגדרה{ והצדק }\צהגדרה{[ל״ה }\צמקור{118}\צהגדרה{]. }\\\ערך{אור ד׳ בארץ }\הגדרה{- התורה האמיתית והשכל הצלול והבהיר }\מקור{[אג׳ א קיז]}\צהגדרה{. }\\\משנה{אור ד׳ }\הגדרה{- קדושת התורה והיהדות הנאמנה, מורשה קהלת יעקב }\צהגדרה{[מ״ר }\צמקור{367}\צהגדרה{].}\\\הגדרה{אור צדק עולמים אשר }\הגדרה{בתורת-חיים\mycircle{°}}\הגדרה{ }\צהגדרה{[מ״ר }\צמקור{366}\צהגדרה{].}\\\הגדרה{}\הגדרה{המוסר\mycircle{°}}\הגדרה{ }\הגדרה{האלהי\mycircle{°}}\הגדרה{ המתגלה }\הגדרה{בתורה\mycircle{°}}\הגדרה{, במסורת, }\הגדרה{בשכל\mycircle{°}}\הגדרה{ }\הגדרה{וביושר\mycircle{°}}\הגדרה{ }\מקור{[א״ק ג א]}\צהגדרה{. }\\\הגדרה{}\הגדרה{צמאון-אלהי\mycircle{°}}\הגדרה{ }\מקור{[עפ״י קובץ ז רח]}\צהגדרה{.}\\\הגדרה{}\הגדרה{נועם\mycircle{°}}\הגדרה{ הקודש }\מקור{[א״ק ב שי]}\צהגדרה{. }\\\הגדרה{}\הגדרה{זיו\mycircle{°}}\הגדרה{ אור החכמה והשגת האמת }\מקור{[פנ׳ ח]}\צהגדרה{. }

\ערך{אור ד׳ העליון }\הגדרה{- }\משנה{(לעומת אור-הדעת-התחתון\mycircle{°}) }\הגדרה{- }\הגדרה{אור-המקיף\mycircle{°}}\הגדרה{ הגדול ורחב מרחבי שחקים }\מקור{[ע״א ד ט כט]}\צהגדרה{. }

\ערך{אור האמת }\הגדרה{- ע״ע אמת. }

\ערך{אור הגדול}\הגדרה{ -}\משנה{ האור הגדול}\הגדרה{ - התוכן }\הגדרה{האלהי\mycircle{°}}\הגדרה{ שאי אפשר להגותו }\הגדרה{ולשערו\mycircle{°}}\הגדרה{ }\מקור{[פנק׳ א שסב]}\צהגדרה{.}

\ערך{אור הגלוי }\הגדרה{- }\משנה{האור הגלוי}\הגדרה{\mycircle{°} - האור הנראה של }\הגדרה{אור-התורה\mycircle{°}}\הגדרה{ וחכמת ישראל כולה }\הגדרה{בקדושתה\mycircle{°}}\הגדרה{ }\הגדרה{וטהרתה\mycircle{°}}\הגדרה{, בבינתה והכרתה, בכבודה וישרותה, בעושר סעיפיה בעומק הגיונותיה ובאומץ מגמותיה. תלמודה של תורה בכל הרחבתה והסתעפו(יו)תיה, בדעת וכשרון, ברגש חי וקדוש, וברצון אדיר }\הגדרה{וחסון\mycircle{°}}\הגדרה{ לחיות את אותם החיים הטהורים והקדושים אשר האור המלא הזה מתאר אותם לפנינו. }\הגדרה{(אור-הקודש-החבוי\mycircle{°}}\הגדרה{) בהיותו מתקרב מאד אל מושגינו, אל צרכינו הזמניים, ואל מאויינו הלאומיים }\מקור{[מא״ה ג (מהדורת תשס״ד) קכג, קכה]}\צהגדרה{. }\\\הגדרה{ע״ע אור קודש חבוי. }

\ערך{אור הדעת התחתון }\הגדרה{- }\משנה{(לעומת אור-ד׳-העליון\mycircle{°}) }\הגדרה{- }\הגדרה{אור-הפנימי\mycircle{°}}\הגדרה{ המרוכז באוצר הדעת אשר לבן-אדם }\מקור{[ע״א ד ט כט]}\צהגדרה{. }

\ערך{אור ההשואה }\הגדרה{- ע׳ במדור מונחי קבלה ונסתר.  }

\ערך{אור החיים }\הגדרה{- מקור }\הגדרה{זיו\mycircle{°}}\הגדרה{ החיים }\מקור{[עפ״י א״ק ב שכט]}\צהגדרה{. }\\\הגדרה{שאיפת הגדלת כחותיהם }\מקור{[ע״א ד ו מא]}\צהגדרה{.}\\\הגדרה{ע״ע אור. ע״ע אור חיים. }\\\ערך{אור החיים }\הגדרה{- }\הגדרה{אור-ד׳\mycircle{°}}\הגדרה{, }\הגדרה{זיו\mycircle{°}}\הגדרה{ }\הגדרה{החכמה\mycircle{°}}\הגדרה{ האלהית, ואור פני מלך חוטר מגזע ישי }\מקור{[ע״א ב ט קנב]}\צהגדרה{. }\\\צהגדרה{ }\\\ערך{אור החיים העליונים }\הגדרה{- }\הגדרה{הנשגב-הכללי\mycircle{°}}\הגדרה{ }\צהגדרה{[}\צהגדרה{א}\צהגדרה{״ק ג רפ].}

\ערך{אור העליון}\הגדרה{ - היושר האמיתי, העצמיות והקדושה בבירורה }\מקור{[קובץ ו רסט]}\צהגדרה{.}

\ערך{אור העתיד}\הגדרה{ - }\הגדרה{הופעת\mycircle{°}}\הגדרה{ }\הגדרה{כבוד-ד׳\mycircle{°}}\הגדרה{ בעולם }\מקור{[א״ק ב קפב]}\צהגדרה{.}

\ערך{אור הפנימי}\הגדרה{ - }\משנה{האור הפנימי (של המושג מאורו של אלקים-חיים, צור ישעינו, לגדולי המשיגים)}\הגדרה{ - ע״ע אור, האור הפנימי.}

\ערך{אור השכינה }\הגדרה{- ע׳ במדור מונחי קבלה ונסתר, שכינה. }

\ערך{אור התורה }\הגדרה{- ע׳ במדור תורה.}\\\משנה{אורה של תורה }\הגדרה{- ע׳ שם. }

\ערך{אור חדש }\הגדרה{- ע״ע אור קודש.}

\ערך{״אור חדש״ }\הגדרה{- אוצר חיים חדש ומלא }\הגדרה{רעננות\mycircle{°}}\הגדרה{, }\הגדרה{נשמות-חדשות\mycircle{°}}\הגדרה{, מלאות הופעת חיים גאיוניים, ממשלת }\הגדרה{עולמי-עולמים\mycircle{°}}\הגדרה{, הפורחת ועולה, המשחקת בכל עת לפני }\הגדרה{הדר\mycircle{°}}\הגדרה{ }\הגדרה{אל\mycircle{°}}\הגדרה{ עליון, האצולות }\הגדרה{מזיו\mycircle{°}}\הגדרה{ }\הגדרה{החכמה\mycircle{°}}\הגדרה{ }\הגדרה{והגבורה\mycircle{°}}\הגדרה{ של מעלה }\מקור{[א״ק ג שסח]}\צהגדרה{. }

\ערך{אור חיים }\הגדרה{- אור קיום של }\הגדרה{הדר\mycircle{°}}\הגדרה{ נצח נצחים }\מקור{[א״ק ג נח]}\צהגדרה{. }

\ערך{אור חיים}\myfootnote{ \textbf{אור חיים} - לבירור ההבחנה בין ״\textbf{אור}״ ל״\textbf{חיים}״, ע׳ הוד הקרח הנורא פרק א סי׳ ג, ד, ובעיקר בעמ׳ לו. \label{10}}\ערך{ }\הגדרה{- }\הגדרה{דעה\mycircle{°}}\הגדרה{ }\הגדרה{ורצון\mycircle{°}}\הגדרה{, רוח-הבטה }\הגדרה{ומציאות-מלאה\mycircle{°}}\הגדרה{ }\מקור{[עפ״י א׳ יא]}\צהגדרה{. }\\\הגדרה{ע״ע ״אור החיים״. ע׳ בנספחות, מדור מחקרים, אור וחיים.}

\ערך{אור חַי-העולמים\mycircle{°} }\הגדרה{- }\הגדרה{אור-עליון\mycircle{°}}\הגדרה{, מקור מקורות, חיי החיים }\מקור{[קובץ ה צט]}\צהגדרה{.}

\ערך{אור חֵי-העולמים\mycircle{°} }\הגדרה{- }\הגדרה{הענין-האלהי\mycircle{°}}\הגדרה{}\מקור{ [א׳ סו]}\צהגדרה{.}\\\הגדרה{}\הגדרה{הטוב-הכללי\mycircle{°}}\הגדרה{, הטוב האלהי השורה }\הגדרה{בעולמות\mycircle{°}}\הגדרה{ כולם. נשמת-כל, האצילית, }\הגדרה{בהודה\mycircle{°}}\הגדרה{ }\הגדרה{וקדושתה\mycircle{°}}\הגדרה{ }\מקור{[עפ״י א״ש פרק ב]}\צהגדרה{. }\\\הגדרה{החיים האלהיים ההולכים ושופעים, המחיים כל חי, השולחים אורם מרום גובהם עד שפל תחתיות ארץ, המתפשטים על אדם ועל בהמה יחד }\מקור{[עפ״י ע״ט י]}\צהגדרה{. }\\\הגדרה{הרצון הכללי, הרצון העולמי }\צהגדרה{[}\צהגדרה{א}\צהגדרה{״ק ג נ]. }

\ערך{אור עליון }\הגדרה{- }\משנה{האור העליון }\הגדרה{- חייו ומקור שפעו, מחוללו ומהוהו של העולם }\מקור{[עפ״י ע״א ד ט נב]}\צהגדרה{. }\\\הגדרה{יסוד הכל ומקורו }\מקור{[קובץ א תרלו]}\צהגדרה{.}\\\הגדרה{}\הגדרה{הזיו\mycircle{°}}\הגדרה{ }\הגדרה{האלהי\mycircle{°}}\הגדרה{, יוצר כל }\מקור{[א״ק א קצב]}\צהגדרה{. }\\\משנה{האור העליון שבהויה}\הגדרה{ - }\הגדרה{העילוי\mycircle{°}}\הגדרה{ }\הגדרה{הרוחני\mycircle{°}}\הגדרה{ }\מקור{[קובץ א קסח]}\צהגדרה{.}\\\משנה{האור העליון }\הגדרה{- }\הגדרה{זוהר\mycircle{°}}\הגדרה{ }\הגדרה{הצחצחות\mycircle{°}}\הגדרה{ של }\הגדרה{קדש-הקדשים\mycircle{°}}\הגדרה{ }\מקור{[א״ק ג רח]}\צהגדרה{. }\\\הגדרה{מקור מקוריות כל חיים וכל יש }\מקור{[קובץ ה נ]}\צהגדרה{. }\\\הגדרה{מקור מקורות, חיי החיים, אור חי העולמים }\מקור{[שם צט]}\צהגדרה{. }\\\הגדרה{מקור החיים והעונג }\מקור{[שם כה]}\צהגדרה{. }\\\הגדרה{בהירות חיים והויה מלאה זיו }\הגדרה{קדש\mycircle{°}}\הגדרה{. החיים }\הגדרה{העליונים\mycircle{°}}\הגדרה{ ברום ערכם, בהופיעם ממכון }\הגדרה{הטוב\mycircle{°}}\הגדרה{ }\הגדרה{והעלוי\mycircle{°}}\הגדרה{ }\הגדרה{הנשגב\mycircle{°}}\הגדרה{ }\מקור{[עפ״י ע״ר א קצג]}\צהגדרה{. }\\\הגדרה{לשד חיי העולמים הזולף בחסדי אבות ממקור }\הגדרה{הברכה\mycircle{°}}\הגדרה{,  מיסוד עולם שהוא קודם ונעלה מכל }\הגדרה{הגבלה\mycircle{°}}\הגדרה{ וחוקיות מוטבעה }\מקור{[אג׳ ג נח]}\צהגדרה{. }\\\משנה{האור העליון הבלתי מוגבל }\הגדרה{- }\הגדרה{המוסר\mycircle{°}}\הגדרה{ האלהי המוחלט }\מקור{[א״ת ד ד]}\צהגדרה{.}\\\הגדרה{ע״ע אור אלהי. ע״ע אור ד׳.}\footref{9} \הגדרה{ע׳ במדור מונחי קבלה ונסתר, אור אין סוף. }

\ערך{אור קודש}\myfootnote{ \textbf{אור }\textbf{קודש} - בא״ק ג רפו הנוסח הוא: אור חדש.\label{11}}\ערך{ }\הגדרה{- טללי }\הגדרה{שפעת\mycircle{°}}\הגדרה{ }\הגדרה{חכמה\mycircle{°}}\הגדרה{ }\הגדרה{וציורים\mycircle{°}}\הגדרה{ }\הגדרה{עליונים\mycircle{°}}\הגדרה{, }\הגדרה{נשגבים\mycircle{°}}\הגדרה{ }\הגדרה{ונעימים\mycircle{°}}\הגדרה{, שהם משתפכים לתוך }\הגדרה{הנשמה\mycircle{°}}\הגדרה{, מודיעים לה }\הגדרה{זיוים\mycircle{°}}\הגדרה{ עליונים, מנשאים אותה }\הגדרה{לרוממות\mycircle{°}}\הגדרה{ }\הגדרה{מעלה\mycircle{°}}\הגדרה{, מקרבים לה את }\הגדרה{היש-העליון\mycircle{°}}\הגדרה{, את }\הגדרה{הרוחניות\mycircle{°}}\הגדרה{ }\הגדרה{והטוהר\mycircle{°}}\הגדרה{ המעולה, את }\הגדרה{אור-ד׳-ממרומיו\mycircle{°}}\הגדרה{ }\מקור{[קובץ ו קנ]}\צהגדרה{. }\\\הגדרה{ע״ע אור, האור בעצם.}

\ערך{אור קודש חבוי }\הגדרה{- }\הגדרה{האור\mycircle{°}}\הגדרה{ }\הגדרה{הקדוש\mycircle{°}}\הגדרה{ הגנוז, (ה)מקור }\הגדרה{האלהי\mycircle{°}}\הגדרה{ של }\הגדרה{התורה\mycircle{°}}\הגדרה{, }\הגדרה{החוסן\mycircle{°}}\הגדרה{ של }\הגדרה{הנבואה\mycircle{°}}\הגדרה{, }\הגדרה{סגולתה\mycircle{°}}\הגדרה{ של }\הגדרה{רוח-הקודש\mycircle{°}}\הגדרה{ והמחזה }\הגדרה{העליון\mycircle{°}}\הגדרה{. האור הגנוז של אור הנבואה ורוח הקודש. מעין החיים של שורש התורה האלהית ומכון כל חזון ומראה עליון. אור הקודש של חמדת עולמים הגנוזה, שורש התורה האלהית ומקור הנבואה ורוח הקודש המיוחד לישראל }\מקור{[עפ״י מא״ה ג (מהדורת תשס״ד) קכב-ה]}\צהגדרה{.}\\\הגדרה{ע״ע אור הגלוי. ע׳ במדור אליליות ודתות, חושך חבוי.}

\ערך{אורגן }\הגדרה{- גוף חי, מסודר }\מקור{[רצי״ה א״ש ה הערה 1]}\צהגדרה{.}

\ערך{אורגניסמוס }\הגדרה{- הקישור העצמי שיש להגוף עם הנשמה }\מקור{[קובץ ה קנה]}\צהגדרה{.}\\\ערך{אורגניסמוס }\הגדרה{- }\משנה{(האורגניות הכללית שביצירה כולה) }\הגדרה{- קישור }\הגדרה{ושילוב\mycircle{°}}\הגדרה{ החלקים זה בזה בכל צומח ובכל חי ובאדם. כל החלקים שביש (ה)צריכים זה לזה, ותהומות רבה והררי עד (ש)הם זה בזה משולבים ומצורפים }\מקור{[עפ״י א״ק ב תיז]}\צהגדרה{. }\\\משנה{חק האורגניות}\הגדרה{ - היחש החי של השפעה ושל קבלה, (ה)הולך וחורז ומקיף את כל המצוי, את החומריות ואת הרוחניות, את הפעולות, המנהגים, ההרגשות ואת המחשבות}\מקור{ [ע״א ד ו מא]}\צהגדרה{.}

\ערך{״אורה״ }\הגדרה{- ע׳ במדור פסוקים ובטויי חז״ל. }\\\ערך{אורה }\הגדרה{- }\משנה{האורה }\הגדרה{- }\הגדרה{העילוי\mycircle{°}}\הגדרה{ }\הגדרה{הרוחני\mycircle{°}}\הגדרה{ }\מקור{[עפ״י קובץ א קסח]}\צהגדרה{.}

\ערך{אורה }\הגדרה{- }\משנה{אורה אלהית }\הגדרה{- שלמות הכל, ושלמות }\הגדרה{העדן\mycircle{°}}\הגדרה{ של מקור הכל, שאין לנו שום מושג ממנה כי-אם מה שאנו חשים את מציאותה ומתענגים מזיוה בכל עומק נפש רוח ונשמה }\מקור{[עפ״י א״ק ג רצ, א׳ קיא]}\צהגדרה{. }\\\הגדרה{הגודל והשיגוב האלהי }\מקור{[קובץ ו צ]}\צהגדרה{. }

\ערך{אורה }\הגדרה{- }\משנה{אורה אלהית }\הגדרה{- }\הגדרה{החוסן\mycircle{°}}\הגדרה{ המלא, האור העליון, המון החיים ומקור יממיהם }\מקור{[עפ״י אוה״ק ב תמז]}\צהגדרה{.}\\\הגדרה{האורה האנושית בכללה המתגלה }\הגדרה{באורן-של-ישראל\mycircle{°}}\הגדרה{}\צהגדרה{ [}\צהגדרה{אג}\צהגדרה{׳ א מג].}\\\משנה{האורה הכללית }\הגדרה{- המשך החיים הנובע מהתשוקה העליונה והכללית של }\הגדרה{קרבת-אלהים\mycircle{°}}\הגדרה{}\מקור{ [מ״ר 38]}\צהגדרה{.}\\\משנה{אורה עליונה }\הגדרה{- }\הגדרה{אור\mycircle{°}}\הגדרה{ }\הגדרה{חכמת\mycircle{°}}\הגדרה{ כל }\הגדרה{עולמים\mycircle{°}}\הגדרה{ }\מקור{[א׳ כט]}\צהגדרה{. }

\ערך{אורה }\הגדרה{- }\משנה{האורה הכללית }\הגדרה{- דעת }\הגדרה{היהדות\mycircle{°}}\הגדרה{ בכל הדרת נשמתה הפנימית, העולה מעומקה של }\הגדרה{תורה\mycircle{°}}\הגדרה{, ומאוצר ההרגשה האלהית הבאה בהתמדת התלמוד והעיון בדברים שהם כבשונו של עולם, עם המכשירים המוסריים והעיוניים הדרושים לזה }\מקור{[עפ״י א״ה 913]}\צהגדרה{. }

\ערך{אורה אלהית }\הגדרה{- החפץ }\הגדרה{הציורי\mycircle{°}}\הגדרה{ והמעשי, לשלטון של }\הגדרה{עילוי\mycircle{°}}\הגדרה{ כל }\הגדרה{עז\mycircle{°}}\הגדרה{ של }\הגדרה{צדק\mycircle{°}}\הגדרה{ }\הגדרה{ואור\mycircle{°}}\הגדרה{ }\מקור{[ע״ה קלב]}\צהגדרה{. }\\\הגדרה{השלמות, המעשית והשכלית, הרגשית והתכונית, השלמות במילואה }\מקור{[א״ק ב שעה]}\צהגדרה{. }\\\משנה{אורה }\הגדרה{- שלמות בכל תיקונה. חיי }\הגדרה{קודש\mycircle{°}}\הגדרה{ }\הגדרה{וטוהר\mycircle{°}}\הגדרה{ }\מקור{[עפ״י שם רפז, שכט]}\צהגדרה{. }\\\משנה{האורה העליונה }\הגדרה{- גדולת החיים של הכרת האלהות האמיתית, הכוללת את כל תענוגי הרוח וכל העדנים עמם ברום עוזם}\מקור{ [קובץ ז עו]}\צהגדרה{.}

\ערך{אורה חיצונית }\הגדרה{- נימוסים אנושיים טובים ויפים, תיקוני מדינה וממלכה נוחים ונעימים. התקדמות, סדרים, }\הגדרה{פאר\mycircle{°}}\הגדרה{ ונעימות חיצונית הדורשים עמם חכמה מעשית רבה ללכת קוממיות ולהיות גוי איתן מלא חכמה מעשית וכליל }\הגדרה{יופי\mycircle{°}}\הגדרה{ }\מקור{[עפ״י ע״א ג ב טז]}\צהגדרה{.}\\\הגדרה{ע״ע אורה פנימית.}

\ערך{אורה פנימית }\הגדרה{- }\הגדרה{אורה-של-תורה\mycircle{°}}\הגדרה{, }\הגדרה{רוח-הקודש\mycircle{°}}\הגדרה{ }\הגדרה{והנבואה\mycircle{°}}\הגדרה{, ששופע }\הגדרה{בישראל\mycircle{°}}\הגדרה{ ביחוד ממקום }\הגדרה{בית-המקדש\mycircle{°}}\הגדרה{ יצאה }\הגדרה{האורה\mycircle{°}}\הגדרה{, היא }\הגדרה{האורה-האלהית\mycircle{°}}\הגדרה{ שמאירה בישראל לבדם ואין לזרים חלק בו. כח }\הגדרה{הקדושה\mycircle{°}}\הגדרה{ המיוחדת שהיא מעלה את ישראל למצב רם ברוח קדושה }\הגדרה{ודעת-אלהים\mycircle{°}}\הגדרה{ }\הגדרה{ודרכיו\mycircle{°}}\הגדרה{, שכולה אומרת }\הגדרה{כבוד-אלהים\mycircle{°}}\הגדרה{ }\מקור{[עפ״י ע״א ג ב טז]}\צהגדרה{.}\\\הגדרה{ע״ע אורה חיצונית.}

\ערך{אורה רוחנית }\הגדרה{- }\משנה{האורה הרוחנית }\הגדרה{- }\הגדרה{הגבורה\mycircle{°}}\הגדרה{ הגמורה המנצחת את כל העולמים וכל כחותיהם }\מקור{[א׳ פד]}\צהגדרה{. }

\ערך{אורה שכלית }\הגדרה{- }\משנה{האורה השכלית}\הגדרה{ - הדעות הקבועות וארחות }\הגדרה{הדעה\mycircle{°}}\הגדרה{ }\מקור{[ע״ר א קסח]}\צהגדרה{.}\\\מעוין{◊ }\משנה{האורה השכלית}\הגדרה{ באה מרוב תורה ודעת, מהרבה }\הגדרה{שימוש-של-חכמים\mycircle{°}}\הגדרה{,  ומהרבה דעת העולם והחיים }\מקור{[א״ק א רמ]}\צהגדרה{. }

\ערך{אורה של תורה }\הגדרה{- ע׳ במדור תורה.}

\ערך{אורה של תורה }\הגדרה{- ע׳ במדור תורה, אור התורה. }

\משנה{״אורות״ }\צהגדרה{- }\צמשנה{(עניינו של ספר אורות) }\צהגדרה{- שלמות גילוי אמתת קדושת עצמיותם של ישראל וערכם האלהי העליון הנצחי}\צמקור{ [א׳ קפז].}

\ערך{אורות הקדש }\הגדרה{- החיים בחיים (ה)עליונים ברום עולמים }\הגדרה{בצחצחות\mycircle{°}}\הגדרה{ }\הגדרה{אידיאליהם\mycircle{°}}\הגדרה{, ספוגי }\הגדרה{קדש-קדשים\mycircle{°}}\הגדרה{ }\מקור{[ע״ר א קפ]}\צהגדרה{. }\\\הגדרה{ע״ע מוסר הקודש. ר׳ חכמת הקודש.}

\ערך{״אורך ימים״}\myfootnote{ תהילים כג ו, צג ה\label{12}}\הגדרה{ - כל }\הגדרה{הימים\mycircle{°}}\הגדרה{ בעמדם בצביונם המלא, (בהיותם) בתוכן }\הגדרה{העליון\mycircle{°}}\הגדרה{, בקשר החיים עם }\הגדרה{הנצחיות\mycircle{°}}\הגדרה{ }\הגדרה{האלהית\mycircle{°}}\הגדרה{, (ששם) אין }\הגדרה{הזמן\mycircle{°}}\הגדרה{ עובר, הכל קיים, (כש)כל העשוי בהם עומד ומזהיר, ומשביע את }\הגדרה{הנשמה\mycircle{°}}\הגדרה{ }\הגדרה{זיו\mycircle{°}}\הגדרה{ }\הגדרה{וצחצחות\mycircle{°}}\הגדרה{ ושובע נעימות. מלוא הימים, (כאשר) הצירוף של כל השיגוב, שנעשה מכל פרטי החיים }\הגדרה{בטוהר\mycircle{°}}\הגדרה{ }\הגדרה{קדושתם\mycircle{°}}\הגדרה{, מתעלה בזיו ונהורא בהירה }\מקור{[עפ״י ע״ר ב עח]}\צהגדרה{. }\\\משנה{באורך ימים}\צהגדרה{ כלולים הם כל הימים וכל }\צהגדרה{ההשפעות\mycircle{°}}\צהגדרה{, כל }\צהגדרה{ההופעות\mycircle{°}}\צהגדרה{ וכל }\צהגדרה{ההזרחות\mycircle{°}}\צהגדרה{, כל המדעים וכל ההרגשות, כל צדדי ההסתכלות, וכל ארחות }\צהגדרה{הדעה\mycircle{°}}\צהגדרה{ }\מקור{[א״ק א סו]}\צהגדרה{. }

\ערך{״אורך ימים״}\myfootnote{ ברכת קריאת שמע שבערבית.\label{13}}\ערך{ }\הגדרה{- }\משנה{(תאר למעלה שבמצוות כ״אורך ימינו״ לעומת ״חיינו״) }\הגדרה{- התועלת המגיעה בשלמות }\הגדרה{הנשמה\mycircle{°}}\הגדרה{ במה שנעלם ואינו נרגש כלל אבל הוא מקנה לה קנין נשגב }\מקור{[ע״ר א תיב (פנק׳ ג רסח)]}\צהגדרה{. }\\\הגדרה{ע״ע ״חיים״, תאר למעלה שבמצוות (לעומת אורך ימים).}

\ערך{אורך ימים }\הגדרה{- השלמת החיים היוצאת חוץ לגבול התעודה הפרטית. שמאריכים הם על המדה המוגבלת לפרטיותו ויספיק האדם תעודת החיים בעד העתיד בעד דור יבוא }\מקור{[עפ״י ע״א ג א נח (ח״פ מב.)]}\צהגדרה{. }\\\הגדרה{הנביעה של }\הגדרה{האורה\mycircle{°}}\הגדרה{ }\הגדרה{הרוחנית\mycircle{°}}\הגדרה{ המתגברת ועולה על ידי }\הגדרה{הברכה\mycircle{°}}\הגדרה{ הפנימית של }\הגדרה{הנשמה\mycircle{°}}\הגדרה{, שהיא באה ביחוד מהמקור של ההוקרה הפנימית והתוכית של הצד החיצוני המוכר }\הגדרה{בחכמה\mycircle{°}}\הגדרה{, שזהו התוכן הפרטי שבקניני הרוח, שהיא הכרה מפורדת לחלקים שונים, (העושה את) הימים מבורכים גם בפרטיותם }\מקור{[עפ״י שם ד יג ט]}\צהגדרה{.}\\\הגדרה{ע״ע אורך שנים. }

\ערך{אורך שנים }\הגדרה{- האורה הכללית של השנים. תפיסת חיים העולה בצורה כללית, מפני הוקרת התוכן }\הגדרה{האצילי\mycircle{°}}\הגדרה{ של }\הגדרה{החכמה\mycircle{°}}\הגדרה{ (ה)מביאה נהרה אחדותית בנפש האדם, ודחיפת החיים היוצאת ממנה היא משאת נפש לתוכן ההכללה של החכמה בצורתה הבהירה והמקפת }\מקור{[עפ״י ע״א ד יג ט]}\צהגדרה{. }\\\הגדרה{ע״ע אורך ימים.}

\ערך{אורן של צדיקים }\הגדרה{- }\הגדרה{האור\mycircle{°}}\הגדרה{ }\הגדרה{הרענן\mycircle{°}}\הגדרה{, }\הגדרה{שהקודש-העליון\mycircle{°}}\הגדרה{ חי במלא }\הגדרה{חפשו\mycircle{°}}\הגדרה{ הנאדר בתוכו }\מקור{[א״ק ג ק]}\צהגדרה{. }

\ערך{אושר }\הגדרה{- }\משנה{האושר}\הגדרה{ - }\מעוין{◊ }\הגדרה{מקור }\הגדרה{המנוחה\mycircle{°}}\הגדרה{ }\הגדרה{והבטחה\mycircle{°}}\הגדרה{ }\צהגדרה{[}\צהגדרה{ע}\צהגדרה{״ר ב סד].}

\ערך{אושר }\הגדרה{- }\משנה{(נשמתי) }\הגדרה{- תחושת }\הגדרה{הנשמה\mycircle{°}}\הגדרה{ את עדונה הגדול, את }\הגדרה{זיו\mycircle{°}}\הגדרה{ חייה המלא עדני עד, מהמון זרמי חיי עולם וישות אדירה, השוטפים בקרבה פנימה, מנחת דשן }\הגדרה{קרבת\mycircle{°}}\הגדרה{ }\הגדרה{אלהים-חיים\mycircle{°}}\הגדרה{, ואור קדושתו המלאה על כל גדותיה }\מקור{[עפ״י ע״ר א קח-ט]}\צהגדרה{. }\\\משנה{אושר נעלה }\הגדרה{- קדושת החיים }\הגדרה{במוסר\mycircle{°}}\הגדרה{ מדות טובות }\הגדרה{ודעת-אלקים\mycircle{°}}\הגדרה{ }\מקור{[פנ׳ פה]}\צהגדרה{. }\\\משנה{אושר }\הגדרה{- שלמות אמיתית בדעת ובהנהגה }\מקור{[ע״א ג ב נה]}\צהגדרה{.}\\\משנה{המצב המאושר }\הגדרה{- המצב }\הגדרה{הנפשי\mycircle{°}}\הגדרה{, }\הגדרה{שנועם\mycircle{°}}\הגדרה{ ד׳, }\הגדרה{ועונג\mycircle{°}}\הגדרה{ }\הגדרה{אהבה\mycircle{°}}\הגדרה{ ושיקוק }\הגדרה{עליון\mycircle{°}}\הגדרה{ מופיע בתוך }\הגדרה{הנשמה\mycircle{°}}\הגדרה{ במצב של מנוחה וקביעות }\מקור{[א״ק ב תקח]}\צהגדרה{. }

\ערך{אושר }\הגדרה{- }\משנה{קץ האושר }\הגדרה{- שיהפך }\הגדרה{העונג\mycircle{°}}\הגדרה{ היותר חמרי ויותר מלוכלך בניוול }\הגדרה{לקודש\mycircle{°}}\הגדרה{ אידיאלי עליון }\מקור{[פנק׳ ג שלט]}\צהגדרה{.}

\ערך{אושר העולם }\הגדרה{- }\הגדרה{השמחה\mycircle{°}}\הגדרה{ העדינה תולדתו של }\הגדרה{העדן\mycircle{°}}\הגדרה{ }\הגדרה{האציל\mycircle{°}}\הגדרה{ הבא }\הגדרה{מההארה\mycircle{°}}\הגדרה{ של }\הגדרה{זיו\mycircle{°}}\הגדרה{ הרעיונות של עומק }\הגדרה{האמונה\mycircle{°}}\הגדרה{, הרפודה }\הגדרה{באהבה-האלהית\mycircle{°}}\הגדרה{ }\הגדרה{והדבקות\mycircle{°}}\הגדרה{ הגדולה והרחבה שזיו }\הגדרה{שדי\mycircle{°}}\הגדרה{ פרוש עליה }\מקור{[א״א 127]}\צהגדרה{. }\\\משנה{תכלית האושר}\הגדרה{ - התכלית }\הגדרה{האידיאלית\mycircle{°}}\הגדרה{ האחרונה, שהיא }\הגדרה{עצת-ד׳\mycircle{°}}\הגדרה{ }\הגדרה{וברכתו\mycircle{°}}\הגדרה{ לאדם ולעולם }\מקור{[ל״ה 158]}\צהגדרה{.}\\\הגדרה{ע׳ במדור פסוקים ובטויי חז״ל, שמחת ד׳ במעשיו.}\\\ערך{אושר עליון\mycircle{°}}\הגדרה{ - שיהיה ״ד׳ אחד ושמו }\הגדרה{אחד״\mycircle{°}}\הגדרה{ }\מקור{[א׳ קס]}\צהגדרה{. }\\\הגדרה{ע״ע תֹּם. }

\ערך{״אות מן התורה״}\myfootnote{ ע׳ מגלה עמוקות על ואתחנן אופן קצז. בית עולמים קלט.: ד״ה תיקונא ״נשמות ישראל הם אותיות התורה״.\label{14}}\הגדרה{ - נשמה מישראל }\מקור{[עפ״י א״ת יא ב]}\צהגדרה{.}

\ערך{אותיות }\הגדרה{- }\משנה{כ״ב אתוון}\myfootnote{ \textbf{כ}\textbf{״ב אותיות} - בביאור ס׳ קהלת לרמ״ד וואלי, עמ׳ קעד ״א״ת, דהיינו כללות ההשפעה מא׳ ועד ת׳״. ושם קעו: ״בגין דאיהו עמודא דאמצעיתא, שהוא כלול מכל האורות, דהיינו מלת ״את״ שרומזת אל הכללות מא׳ ועד ת׳״. ובבאורו לאיכה, עמ׳ קמא: ״כי כ״ב אותיות הם רומזים אל הכללות כידוע״.\label{15}}\משנה{ }\הגדרה{- כלל ההנהגה}\מקור{ [ג״ר 29]}\צהגדרה{. }

\ערך{אז }\הגדרה{- מורה על העבר, אבל לא רק בדרך }\הגדרה{פרוזי\mycircle{°}}\הגדרה{, סיפור של מאורע שאינו מרותק עם רגשי הנפש והתפעלו(יו)תיה השיריות, כ״א באורח שירי, ומצב נפשי מרומם }\מקור{[ר״מ קיט]}\צהגדרה{. }

\ערך{אח }\הגדרה{- הקרוב היותר מקורב, המגובל בגבול האחדות }\מקור{[ר״מ קכ]}\צהגדרה{. }

\ערך{אחדות }\הגדרה{- }\משנה{האחדות }\הגדרה{- יחוד שלטון }\הגדרה{השי״ת\mycircle{°}}\הגדרה{ בהתחלת }\הגדרה{הסבות\mycircle{°}}\הגדרה{ הראשיות, המסבבות כל המון המעשים, שהן בערך }\הגדרה{השמים\mycircle{°}}\הגדרה{, ובגמר כל תכליתם, שהן בערך }\הגדרה{הארץ\mycircle{°}}\הגדרה{, ובכל האמצעים הרבים השונים ומסובכים, אשר ביניהם, שהם בערך }\הגדרה{ד׳-רוחות\mycircle{°}}\הגדרה{ העולם, שכאילו מחברים את השמים עם הארץ }\מקור{[ע״ר א רמה]}\צהגדרה{.}\\\משנה{אחדות ד׳}\צהגדרה{ - }\הגדרה{הדעה היותר עליונה של מושג }\הגדרה{האלהות\mycircle{°}}\הגדרה{ }\מקור{[ל״ה 224]}\צהגדרה{. }\\\משנה{אחדות הרוחנית}\הגדרה{ - }\הגדרה{שם-ד׳\mycircle{°}}\הגדרה{ }\הגדרה{אחד\mycircle{°}}\הגדרה{ השוכן }\הגדרה{בישראל\mycircle{°}}\הגדרה{ }\מקור{[ע״א ב ביכורים לט]}\צהגדרה{.}\\\משנה{אחדות }\הגדרה{- }\הגדרה{מגמה\mycircle{°}}\הגדרה{ אחת עשירה ואדירה, כוללת כל, וברוכה בכל - }\הגדרה{אור-החיים\mycircle{°}}\הגדרה{ היותר מאירים ויותר שלמים. המקור האחד של כל }\הגדרה{האידיאלים\mycircle{°}}\הגדרה{ היותר נשאים, שאנחנו מוצאים בנפשנו פנימה, שכל זמן שהם }\הגדרה{עולים\mycircle{°}}\הגדרה{ ומתבכרים, הם באים אליו }\מקור{[עפ״י ע״ה קנב]}\צהגדרה{. }\\\הגדרה{התוכן של השאיפה היותר נאצלה השיכת לכל הנברא בהתאחד הכל למטרתו היותר עליונה }\מקור{[עפ״י ע״ר א קנט]}\צהגדרה{. }\\\הגדרה{השאיפה ותגבורת }\הגדרה{חיל\mycircle{°}}\הגדרה{ החיים בהתרכזות עשירה של חטיבה כללית, בכל, וביחוד במציאות }\הגדרה{הרוחנית\mycircle{°}}\הגדרה{ }\הגדרה{והאידיאלית\mycircle{°}}\הגדרה{, המתלבשת גם כן יפה }\הגדרה{בהחמרית\mycircle{°}}\הגדרה{ והריאלית בכל מלא }\הגדרה{עולמים\mycircle{°}}\הגדרה{ כולם }\מקור{[עפ״י מ״ר 16]}\צהגדרה{. }\\\הגדרה{}\הגדרה{הכלליות\mycircle{°}}\הגדרה{ }\הגדרה{הקדושה\mycircle{°}}\הגדרה{ ברוממות קודש קדשה }\מקור{[ר״מ קסח]}\צהגדרה{. }\\\משנה{יסוד האחדות-העליונה\mycircle{°}}\הגדרה{ - }\הגדרה{המציאות\mycircle{°}}\הגדרה{ }\הגדרה{ההויתית\mycircle{°}}\הגדרה{ המתגלה כחטיבה אחת }\מקור{[עפ״י א״ה 916]}\צהגדרה{. }\\\הגדרה{ע׳ במדור שמות כינויים ותארים אלהיים, ״אחד״ (תאר כלפי מעלה). ע״ע יחוד ד׳ בעולם. }\\\ערך{אחדות }\הגדרה{- }\משנה{האחדות האלהית }\הגדרה{- }\הגדרה{הרוח\mycircle{°}}\הגדרה{ }\הגדרה{האצילי\mycircle{°}}\הגדרה{ המקיף את כל הנטיות כולן ומאחדם עם כל המון הכוחות הגשמיים והרוחניים למטרה }\הגדרה{מוסרית\mycircle{°}}\הגדרה{ עליונה }\מקור{[עפ״י א״ק ג ש]}\צהגדרה{. }\\\הגדרה{}\הגדרה{הטהרה\mycircle{°}}\הגדרה{ העליונה של הנקודה }\הגדרה{האמונית\mycircle{°}}\הגדרה{}\מקור{ [קבצ׳ ב נ]}\צהגדרה{.}\\\ערך{אחדות אלהים }\הגדרה{- }\הגדרה{הטוב-העליון\mycircle{°}}\הגדרה{, הטוב המגלה שאין כל }\הגדרה{רע\mycircle{°}}\הגדרה{ מצוי }\מקור{[פנק׳ ד עה]}\צהגדרה{.}\\\ערך{אחדות }\הגדרה{- }\משנה{האחדות המופעה בעולם }\הגדרה{- מתבארת ע״י הקשור שיש בין }\הגדרה{המצוה\mycircle{°}}\הגדרה{ }\הגדרה{התורית\mycircle{°}}\הגדרה{ בכלל ההתגלות של }\הגדרה{דבר-ד׳\mycircle{°}}\הגדרה{ ובין כל הסדר העולמי במערכי }\הגדרה{הטבע\mycircle{°}}\הגדרה{ וכל מוסדי ההויה כולם. זהו תוכן המברר את ה}\משנה{אחדות האלהית\mycircle{°} בעולם}\הגדרה{, שהכל מתאים לתוכן אחד והכל מתקשר לאגודה אחדותית אחת }\מקור{[ע״ר א כה]}\צהגדרה{. }\\\הגדרה{ע׳ במדור מונחי קבלה ונסתר, ״יחוד תחתון״. ע׳ בנספחות, מדור מחקרים, אחדות ויחוד. }

\ערך{אחדות }\הגדרה{- }\משנה{האחדות העולמית }\הגדרה{- הצד של השיווי שיש למצוא בהויה כולה, עד למעלה למעלה, }\הגדרה{לדימוי-הצורה-ליוצרה\mycircle{°}}\הגדרה{. הגשמיות והרוחניות, הציור והשכל, השפל והנישא, הם כולם תואמים, מתאחדים ומוקשים }\מקור{[עפ״י ע״ט סז]}\צהגדרה{. }

\ערך{אחדות אין סופית }\הגדרה{- ע׳ במדור מונחי קבלה ונסתר, יחוד עליון. ושם, אור אין סוף. }

\ערך{אחדות עליונה }\הגדרה{- }\הגדרה{הדעת\mycircle{°}}\הגדרה{, }\הגדרה{אוצר-החיים\mycircle{°}}\הגדרה{ אשר בנשמת }\הגדרה{חי-העולמים\mycircle{°}}\הגדרה{ }\מקור{[א״ק א קעד]}\צהגדרה{. }\\\הגדרה{}\הגדרה{אושר\mycircle{°}}\הגדרה{ }\הגדרה{ותענוג\mycircle{°}}\הגדרה{, למעלה מכל }\הגדרה{אחדות\mycircle{°}}\הגדרה{ ומכל צחצחות, שורש }\הגדרה{נשמתן\mycircle{°}}\הגדרה{ של צדיקים אשר עם המלך ישבו במלאכתו }\מקור{[שם ג מד]}\צהגדרה{. }\\\הגדרה{הסוד העליון של }\הגדרה{האורה-האלהית\mycircle{°}}\הגדרה{ }\הגדרה{בראשית\mycircle{°}}\הגדרה{ התחלת הופעתה, (אשר אין) יכולת בידי בן אדם להשכיל בכחו בשכלו ובאורח מדעו וחושיו, }\הגדרה{לצייר\mycircle{°}}\הגדרה{ בהויתו, איזה הערכה מסוד האחדות העליונה }\הגדרה{מלמעלה\mycircle{°}}\הגדרה{ }\הגדרה{למטה\mycircle{°}}\הגדרה{ }\מקור{[עפ״י ח״פ מה.]}\צהגדרה{. }\\\ערך{אחדות מוחלטה }\הגדרה{- אור הפשטות העליונה, יסוד }\הגדרה{העדן-העליון\mycircle{°}}\הגדרה{ }\מקור{[ר״מ קג]}\צהגדרה{. }\\\הגדרה{מקור כל }\הגדרה{הקדושה\mycircle{°}}\הגדרה{, מכון כל }\הגדרה{העונג\mycircle{°}}\הגדרה{ הנצחי, ובסיס כל השלמות ההולכת ומתעלה }\הגדרה{עדי-עד\mycircle{°}}\הגדרה{}\מקור{ [פנק׳ א תו]}\צהגדרה{.}\\\משנה{מלא האחדות המוחלטה }\הגדרה{- מקור חיי כל החיים }\הגדרה{אור-אין-סוף\mycircle{°}}\הגדרה{, אדון כל היש ומלא כל ההויה, מקור כל הרחמים ואב כל החסדים וכל גבורות נעם, כל פאר ויפעה וכל תפארת קדש, שומע }\הגדרה{תפלה\mycircle{°}}\הגדרה{ ומאזין עתירה, בלא קץ ותכלית }\מקור{[ע״ר א סה]}\צהגדרה{.}\\\הגדרה{ע״ע יחוד ד׳ בעולם. ע׳ במדור מונחי קבלה ונסתר, תפארה.}\\\ערך{אחדות שלמה }\הגדרה{- }\משנה{האחדות השלמה }\הגדרה{- }\הגדרה{קודש\mycircle{°}}\הגדרה{ }\הגדרה{ד׳\mycircle{°}}\הגדרה{, קדוש }\הגדרה{ישראל\mycircle{°}}\הגדרה{ }\מקור{[א״ק ג ס]}\צהגדרה{. }

\ערך{אחור }\הגדרה{- צד הטפל שבכל דבר הוא אחוריו }\מקור{[ע״א א ב מג, פנק׳ ג ער]}\צהגדרה{. }\\\הגדרה{ע״ע פנים.}

\ערך{אחור }\הגדרה{- הצד }\הגדרה{החיצוני\mycircle{°}}\הגדרה{, שהוא הרבה כהה והרבה חלוש מהצד שהחיים הפנימיים של }\הגדרה{רוח-ד׳\mycircle{°}}\הגדרה{ אשר במלא עולמו יונקים ממנו }\מקור{[עפ״י מ״ר 249]}\צהגדרה{. }\\\הגדרה{ע״ע פנים.}

\ערך{אחור }\הגדרה{- }\משנה{ההנהגה האלהית שהיא לאחור }\הגדרה{- ההנהגה שהיא לצד ההתחסרות }\מקור{[עפ״י ע״ר ב סז]}\צהגדרה{. }\\\הגדרה{ע״ע פנים.}

\ערך{אחור ופנים במציאות הרוחניות }\הגדרה{- ע״ע פנים ואחור במציאות הרוחניות. }

\ערך{אחוריים }\הגדרה{- }\הגדרה{השגה\mycircle{°}}\הגדרה{ כללית סתומה, שאינה מתפרטת בפירוט האור בתכונה }\הגדרה{פרצופית\mycircle{°}}\הגדרה{, כ״א מתבזקת בהתבזקות כללית כמראה האחוריים שאין בו פירוט }\הגדרה{זיו\mycircle{°}}\הגדרה{ הפנים בכל נתוח אבריו נושאי החושים העליונים }\מקור{[ר״מ קפ]}\צהגדרה{.}\\\משנה{האחוריים של הפנים המחשביים }\הגדרה{- נחלים הנעשים מהאורים הגדולים המתפשטים מהפנים-המחשביים, נחלי חכמה מוקשבת, מחוללת אורה בינה והשכל לימודיים, המתלבשים בהמון התלמוד }\מקור{[עפ״י שם קפד]}\צהגדרה{. }\\\הגדרה{ע״ע פנים, פנים מחשביים. ע׳ במדור משה, הראני נא את כבודך וגו׳. ע׳ במדור תיאורים אלהיים, אחוריים, ראיית אחוריים.}

\ערך{אחוריים }\הגדרה{- }\משנה{מראה אחוריים }\הגדרה{- ע׳ במדור תיאורים אלהיים.  }

\ערך{״אחסנתין״}\myfootnote{ באור הגר״א על משלי, ד ד, ח כא, יד יח, יט יד, כז כז.\label{16}}\ערך{ }\הגדרה{- קשר }\הגדרה{הקדושה\mycircle{°}}\הגדרה{ שהוא ירושה מאבות שיש לאדם בענין }\הגדרה{עבודת-ד׳\mycircle{°}}\הגדרה{ }\מקור{[עפ״י מא״ה ג קעה]}\צהגדרה{. }\\\הגדרה{כח קדושת טבע הנפש שהיא מורשה לישראל }\מקור{[ה׳ רי]}\צהגדרה{. }\\\הגדרה{ע״ע ״עטרין״. }

\ערך{אט }\הגדרה{- ההוראה להליכה בנחת }\מקור{[ר״מ קכ]}\צהגדרה{. }

\משנה{אידיאה }\צהגדרה{- מין }\צהגדרה{נשמה\mycircle{°}}\צהגדרה{, המשכת כח של }\צהגדרה{צורה\mycircle{°}}\צהגדרה{ }\צהגדרה{רוחנית\mycircle{°}}\צהגדרה{, שיש לכל דבר שבעולם }\צמקור{[עפ״י א״ל רמד].}\\\צהגדרה{הערך הרוחני האידיאלי של המציאות, פנימיות הדברים. מציאותיות רוחנית או הרוחניות המציאותית }\צמקור{[שי׳ ב 304, 303].}\\\צהגדרה{מציאות התוכן הרוחני שהוא שורש המציאות }\צמקור{[שם 39, 6-5].}\\\צהגדרה{שורש }\צהגדרה{רוחני\mycircle{°}}\צהגדרה{, מציאותי, שעומד ביסוד המציאות של כל דבר. מציאות רוחנית, שהיא מקור המציאות }\צהגדרה{החומרית\mycircle{°}}\צהגדרה{ }\צמקור{[עפ״י מה״ה ג רטז].}\\\צמשנה{״למהלך האידיאות״ }\צהגדרה{- יסודי עולם רוחניים, מחשבתיים, מציאותיים, שמופיעים במהלך הדורות של קורא הדורות }\צמקור{[שם].}\\\הגדרה{ר׳ בנספחות, מדור מחקרים, אידיאה.}

\ערך{אידיאה אלהית }\הגדרה{- הרעיון האלהי שההכנה אליו הנמצאת באיזה אופן גלוי או נסתר ישר או מעוות, בכל הלבבות של האנושיות, לכל פלגותיה, משפחותיה וגוייה, ומחוללת דתות ורגשי-אמונה שונים סדרים ונמוסים }\מקור{[עפ״י א׳ קב]}\צהגדרה{. }\\\מעוין{◊}\הגדרה{ סגנון המחשבה של הרעיון }\הגדרה{הרוחני\mycircle{°}}\הגדרה{ בהתבררותו ביותר, בגימור קויו הרשמיים ברוח האומנות אשר להסתוריה מבטא את ה}\משנה{אידיאה האלהית }\מקור{[עפ״י א׳ קב]}\צהגדרה{. }\\\ערך{האידיאה האלהית בישראל }\הגדרה{- הנטיה הרוחנית של }\הגדרה{כנסת-ישראל\mycircle{°}}\הגדרה{ שהיא אורה ונשמתה של הנטיה הלאומית המעשית שלה }\מקור{[עפ״י א׳ קד, קנח]}\צהגדרה{. }

\ערך{אידיאה האלהית המוחלטת }\הגדרה{- }\הגדרה{השכינה-העליונה\mycircle{°}}\הגדרה{ }\מקור{[א׳ קיב]}\צהגדרה{. }\\\הגדרה{הכשרון אל השכלול העליון והגמור המאיר את העולם כלו בכבודו }\מקור{[א׳ קה]}\צהגדרה{.}\\\הגדרה{ע׳ בנספחות, מדור מחקרים, אידיאה אלהית ואידיאה לאומית. }

\ערך{אידיאה דתית }\הגדרה{- }\משנה{האידיאה הדתית }\הגדרה{- }\הגדרה{ההופעה\mycircle{°}}\הגדרה{ האלהית המוקטנת המיוחדת לצד הפרטיות }\מקור{[א׳ קי]}\צהגדרה{. }\\\הגדרה{התוכן }\צהגדרה{}\צהגדרה{המוסרי\mycircle{°}}\צהגדרה{ הבא}\הגדרה{ בתור תולדה מהכרת האחדות האלהית בתור גורם למעמד מוסרי יפה לכל יחיד, המביאו לחיי נצח טובים. ולמעולים וחשובים ננעץ בו גם-כן התעוררות מוסרית אדירה לשמה, המתנוצצת מהעולם האלהי, <המתגלה יפה כשהשיקוע החומרי ודרישת ההנאה הגסה, אפילו בצדדיה היפים והעדינים, המצוי בעולם האלילי המסוגל לו, סר מהם> }\מקור{[עפ״י קבצ׳ ג קיד-קטו]}\צהגדרה{.}\\\הגדרה{ע״ע דת.}

\ערך{אידיאה לאומית }\הגדרה{- תוכן הסגנון הצבורי של }\הגדרה{הצורה\mycircle{°}}\הגדרה{ הלאומית }\מקור{[עפ״י א׳ קו]}\צהגדרה{. }\\\הגדרה{הנטיה הקבוצית שבצורה הלאומית }\מקור{[עפ״י שם קב-ג]}\צהגדרה{. }\\\מעוין{◊}\הגדרה{ סגנון החיים הסדרניים של החברה בהתבררותו ביותר, בגימור קויו הרשמיים ברוח האומנות אשר להסתוריה מבטא את ה}\משנה{אידיאה הלאומית }\מקור{[עפ״י שם קב]}\צהגדרה{. }\\\ערך{האידיאה הלאומית בישראל }\הגדרה{- הנטיה הלאומית המעשית שהיא לבושה החיצון של הנטיה הרוחנית של }\הגדרה{כנסת-ישראל\mycircle{°}}\הגדרה{ }\מקור{[עפ״י שם קנח]}\צהגדרה{. }\\\הגדרה{ע׳ בנספחות, מדור מחקרים, אידיאה אלהית ואידיאה לאומית. }

\משנה{אידיאה לאומית}\myfootnote{ ע׳ בנספחות, מדור מחקרים, אידיאה אלהית ואידיאה לאומית.\label{17}}\הגדרה{ }\צהגדרה{- }\צהגדרה{כנסת ישראל }\צמקור{[שי׳ ב 235].}\\\הגדרה{שכינת האומה }\מקור{[א׳ קו]}\צהגדרה{.}\\\הגדרה{הלבוש של כנסת-ישראל, החודרת בעצמת חייה בכל פרט ופרט מישראל, ובכל מעשיו ותנועותיו, שיחיו ושיגיו, שאיפותיו וקניניו הפרטיים }\מקור{[עפ״י קובץ ו קמג]}\צהגדרה{. }

\ערך{אידיאה העליונה }\הגדרה{- }\משנה{האידיאה העליונה }\הגדרה{- }\הגדרה{הרצון\mycircle{°}}\הגדרה{ (רצון }\הגדרה{ד׳\mycircle{°}}\הגדרה{) }\הגדרה{הקדוש\mycircle{°}}\הגדרה{ והנשא }\מקור{[א״ק א קמב]}\צהגדרה{. }

\משנה{אידיאל }\צהגדרה{- המשך של }\צהגדרה{האידיאה\mycircle{°}}\צהגדרה{ }\צמקור{[שי׳ 39, 6-5]. }\\\משנה{אידיאל }\צהגדרה{- הצדק העולמי, שהוא גם הטוב, האור הפיוט וכו׳ וכו׳, נשוא מאויי לבה של הכללות }\צמקור{[ד״ל כו].}\\\ערך{אידיאל }\הגדרה{- }\משנה{התוכן האידיאלי של העולם }\הגדרה{- }\הגדרה{סוד\mycircle{°}}\הגדרה{ }\הגדרה{האלהי\mycircle{°}}\הגדרה{ המוחלט שבהויה }\מקור{[א״א 18]}\צהגדרה{. }\\\צהגדרה{ }\\\ערך{אידיאל }\הגדרה{- }\משנה{האידיאל המוזער של העולם}\הגדרה{ - האידיאל הנופל בההויה המצומצמה }\מקור{[א״ק ב תנה]}\צהגדרה{.}

\ערך{אידיאל }\הגדרה{- }\משנה{האידיאל המלא של העולם }\הגדרה{- האידיאל המרומם }\הגדרה{האלהי\mycircle{°}}\הגדרה{ במלא מילואו, יסוד העולם היותר }\הגדרה{עתיק\mycircle{°}}\הגדרה{, מציאות העולם היותר ממשית. שהמציאות }\הגדרה{הזעירה\mycircle{°}}\הגדרה{ (של האידיאל המוזער, הנופל בההויה המצומצמה) רק מיניקת לשדו העליון היא מתקיימת }\הגדרה{ומתברכת\mycircle{°}}\הגדרה{ }\מקור{[א״ק ב תנה]}\צהגדרה{.}

\ערך{אידיאל }\הגדרה{- }\משנה{התגלמות של אידיאל }\הגדרה{- ע״ע התגלמות. }

\ערך{אידיאל }\הגדרה{-}\משנה{ הגילוי האידיאלי של כל דבר נעלה בחיי-הרוח\mycircle{°} המתפשט במציאות }\הגדרה{- }\הגדרה{חפשו\mycircle{°}}\הגדרה{ ויסוד מציאותו מתוך רצון מלא נדבה }\מקור{[עפ״י ע״ר א פג-ד]}\צהגדרה{. }\\\הגדרה{ע״ע חסד, (לעומת ברית). ע״ע ברית, (לעומת חסד). }

\ערך{אידיאל לימודי}\הגדרה{ - ע״ע לימוד, האידיאל הלימודי.}

\ערך{אידיאליות }\הגדרה{- }\הגדרה{הרעיון\mycircle{°}}\הגדרה{, }\הגדרה{ההתרוממות\mycircle{°}}\הגדרה{ }\הגדרה{הרוחנית\mycircle{°}}\הגדרה{ }\מקור{[עפ״י ע״ר א ז]}\צהגדרה{. }

\ערך{אידיאליות }\הגדרה{- }\הגדרה{האהבה\mycircle{°}}\הגדרה{ }\הגדרה{לדרכי-ד׳\mycircle{°}}\הגדרה{ הנטועה בנפשו הלאומית (של }\הגדרה{ישראל\mycircle{°}}\הגדרה{) }\הגדרה{פנימה\mycircle{°}}\הגדרה{, והיא הולכת ועולה, פורחת ומתגדלת, לרגלי כל מה שמתגבר }\הגדרה{מקור-ישראל\mycircle{°}}\הגדרה{ }\מקור{[ע״ה קלה]}\צהגדרה{. }\\\ערך{אידיאליות העליונה }\הגדרה{- גילוי אור }\הגדרה{הקדש\mycircle{°}}\הגדרה{ שבשאיפה }\הגדרה{הפנימית\mycircle{°}}\הגדרה{ של }\הגדרה{הרוח\mycircle{°}}\הגדרה{ }\מקור{[חד׳ תשס״ח קמה]}\צהגדרה{.}\\\תערך{אידיאליות }\תהגדרה{- }\תמשנה{האידיאליות הנשמתית }\תהגדרה{- התשוקה העליונה והרוממה }\תהגדרה{להתדבקות\mycircle{°}}\תהגדרה{ אלהית, תוכן }\תהגדרה{האמונה\mycircle{°}}\תהגדרה{ }\תמקור{[עפ״י מ״ר 494]. }\\\הגדרה{ע״ע תשוקה, התשוקה האידיאלית. }

\ערך{אידיאליות }\הגדרה{- }\משנה{האידיאליות }\הגדרה{- }\הגדרה{האצילות\mycircle{°}}\הגדרה{ }\מקור{[א״ק א עט]}\צהגדרה{. }\\\משנה{אידיאליות אלהית }\הגדרה{- }\הגדרה{האידיאליות\mycircle{°}}\הגדרה{ הגמורה, כל }\הגדרה{התפארת\mycircle{°}}\הגדרה{, כל }\הגדרה{ההוד\mycircle{°}}\הגדרה{ שביסוד המציאות }\מקור{[א״ק ג קפד]}\צהגדרה{. }

\ערך{אידיאלים }\הגדרה{- }\משנה{האידיאלים היותר נשאים, שהם הולכים ושואבים תמיד עלוי\mycircle{°} וצחצוח\mycircle{°} ממקור העצמיות\mycircle{°} העליונה\mycircle{°}}\הגדרה{ - פלגות נהרי }\הגדרה{אורותיה\mycircle{°}}\הגדרה{ של עריגת }\הגדרה{הנשמה\mycircle{°}}\הגדרה{ לעצמיות האלהית }\מקור{[עפ״י מ״ר 507]}\צהגדרה{. }\\\משנה{האידיאלים האלהיים\mycircle{°}}\הגדרה{ - }\הגדרה{השמות\mycircle{°}}\הגדרה{ האלהיים, }\הגדרה{דרכי-ד׳\mycircle{°}}\הגדרה{, }\הגדרה{חפציו\mycircle{°}}\הגדרה{, }\הגדרה{האצילות\mycircle{°}}\הגדרה{, }\הגדרה{הספירות\mycircle{°}}\הגדרה{, }\הגדרה{המדות\mycircle{°}}\הגדרה{, השבילים, הנתיבות, }\הגדרה{השערים\mycircle{°}}\הגדרה{ }\הגדרה{והפרצופים\mycircle{°}}\הגדרה{, שמקצת תכנם }\הגדרה{האידיאלי\mycircle{°}}\הגדרה{ חקוק וקבוע הוא גם כן }\הגדרה{בנפש\mycircle{°}}\הגדרה{ האדם,}\myfootnote{ \textbf{האידיאלים}\textbf{ האלהיים} - \textbf{שמקצת תכנם האידיאלי חקוק וקבוע הוא גם כן בנפש האדם, אשר עשהו האלהים ישר} - ע׳ שיחות על אהבה לר״י אברבנאל, מוסד ביאליק, תשמ״ג, עמ׳ 450-440. ע״ע מלבי״ם, איוב לו א-ד. ״האידעען הנטועים בנפשו, הם אמתיות מונחלות וטבועות בנפש נפש ממקור מחצבה, ירושה לה מאלהי הרוחות בעודה בחביון עוזה, והם קדושים וטהורים אלהיים; האידעען קראם בשם דעי, ודעים. (כן אצל רש״ט גפן בממדים, הנבואה והאדמתנות עמ׳ 103); ייחס מלין אלה לאלהים, כי הוא הטביע דעים אלה בנפשו להשיג על פיהם את סודותיו ואמתיותיו; אצל ד׳ הדעים האלה הם בתמימות ובשלמות, וא״כ תמים דעים נמצא עמך, והוא האלהות הנמצא טבוע בשורש נפשך״. ע״ע בנספחות, מדור מחקרים, אידיאה.\label{18}}\הגדרה{ אשר עשהו האלהים }\הגדרה{ישר\mycircle{°}}\הגדרה{ }\מקור{[ע״ה קמה]}\צהגדרה{. }\\\משנה{אידיאלים כלליים }\הגדרה{- מטע }\הגדרה{ד׳\mycircle{°}}\הגדרה{. }\הגדרה{המגמות\mycircle{°}}\הגדרה{ העולמיות כולן, }\הגדרה{אצילות\mycircle{°}}\הגדרה{ }\הגדרה{האורות-העליונים\mycircle{°}}\הגדרה{ ברוממות תעודתם, שנשמתה של האומה הישראלית, ונשמתו של כל יחיד מישראל }\הגדרה{בפנימיות\mycircle{°}}\הגדרה{ מהותה, }\הגדרה{מאור\mycircle{°}}\הגדרה{ זה היא אצולה }\מקור{[עפ״י ע״ר ב קנח]}\צהגדרה{.}\\\הגדרה{הכח הנסתר של }\הגדרה{האצילות\mycircle{°}}\הגדרה{ }\הגדרה{שבנשמת-האומה\mycircle{°}}\הגדרה{ }\מקור{[ע״ה קנב]}\צהגדרה{.}\\\משנה{האידיאלים הנשאים }\צהגדרה{-}\הגדרה{ }\הגדרה{המגמות\mycircle{°}}\הגדרה{ האחרונות של מעשה }\הגדרה{התורה\mycircle{°}}\הגדרה{ }\הגדרה{והמצוה\mycircle{°}}\הגדרה{ }\צהגדרה{[}\צהגדרה{א}\צהגדרה{״ק ג שכב].}\\\הגדרה{ע׳ במדור מונחי קבלה ונסתר, תפארת, התפארת האלהית. ע׳ במדור שמות כינויים ותארים אלהיים, אלהות. ע״ע בנספחות, בסוף מדור מחקרים, שני מכתבים לברור דברים בשיטת הרב. ע״ע רוממות האידיאלים. }

\ערך{אילת השחר}\myfootnote{ תהילים כב א.\label{19}}\ערך{ }\הגדרה{- היופי העולמי }\הגדרה{שנאזר\mycircle{°}}\הגדרה{ מגבורה וצמצומים }\מקור{[קובץ ה ר]}\צהגדרה{. }

\ערך{איש }\הגדרה{- התוכן הממולא בכח המפעל וההשפעה, ששפעותיו עשירות הנה בכל הארחים, }\הגדרה{לטוב\mycircle{°}}\הגדרה{ }\הגדרה{ולרע\mycircle{°}}\הגדרה{, לבנין ולסתירה, רק הוא בכללות קיבוץ כל חלקיו, ממלא הוא תוכן של }\הגדרה{אישיות\mycircle{°}}\הגדרה{, }\הגדרה{צורה\mycircle{°}}\הגדרה{ ממולא(ה) בטפוס שלם, העומד הכן לפעול ולהשפיע, לשכלל ולהשלים}\מקור{ [ר״מ קכז]}\צהגדרה{.}\\\משנה{יסוד השלמתו }\הגדרה{- }\הגדרה{השכל\mycircle{°}}\הגדרה{ העומד בראש }\הגדרה{והרגש\mycircle{°}}\הגדרה{ עוזר על ידו }\צהגדרה{[}\צהגדרה{ע}\צהגדרה{״א ג ב ריג].}\\\הגדרה{ע״ע אשה. ע״ע אנשים. ע״ע גבר. ע״ע ״אדם״. ע״ע ״אנוש״.}

\ערך{״איש״ לעומת ״בעל״}\הגדרה{ - }\משנה{בעל }\הגדרה{- יקרא על שם הבעילה, כמו ״האי מטרא בעלא דארעא״}\myfootnote{ תענית ו: \label{20}}\הגדרה{, ו}\משנה{איש}\הגדרה{ - יקרא על שם המשפיע, יען כי הוא הנותן לה לחם לאכול ובגד ללבוש }\מקור{[ע״א יבמות סב:, סי׳ י]}\צהגדרה{.}

\ערך{אישיות }\הגדרה{- }\הגדרה{צורה\mycircle{°}}\הגדרה{ ממולא(ה) בטפוס שלם, העומד הכן לפעול ולהשפיע, לשכלל ולהשלים }\מקור{[ר״מ קכז]}\צהגדרה{. }

\ערך{איתן העולם }\הגדרה{- ע׳ במדור מדרגות והערכות אישיותיות. }

\ערך{איתנות }\הגדרה{- הקביעות העזיזה }\מקור{[ל״ה 162]}\צהגדרה{. }\\\ערך{איתניות ברוח }\הגדרה{- }\הגדרה{עז\mycircle{°}}\הגדרה{ הרצון הכביר המתגלה באופן בריא וחזק עם כל תנועה נפשית וגופית }\מקור{[ע״א ד ו פג]}\צהגדרה{. }

\ערך{אך }\הגדרה{- מלת המיעוט, הצערת הנושא ממובנו הקדום, קציצת איזה סעיפים מתכונתו, <נרדף הוא עם התוכן של ההכאה הבא בתואר זה> }\מקור{[ר״מ קכא-ב]}\צהגדרה{. }

\ערך{אכילה }\הגדרה{- }\משנה{(לעומת טעימה\mycircle{°})}\הגדרה{ - תתיחס מצד התועלת, התועלת של תכלית האכילה לחזק הגוף ולהמשיך החיים הבאה אחר העיכול, שע״ז יבא פעל אכל, מאוּכַּל}\צהגדרה{ [עפ״י }\צהגדרה{ע}\צהגדרה{״א ג א לה].}\\\הגדרה{ע״ע מזון, לקיחת מזון. ע״ע מאכל.}

\ערך{אַל }\הגדרה{- הוראה שלילית }\מקור{[ר״מ קכב]}\צהגדרה{.}

\ערך{אֵל }\הגדרה{- הוראת הכח }\מקור{[ר״מ קכב]}\צהגדרה{. }

\ערך{אלהִי }\הגדרה{- }\משנה{הנקודה האלהית }\הגדרה{- מכון השלמות המוחלטה }\מקור{[ע״א ד ט קלד]}\צהגדרה{. }\\\הגדרה{ע״ע אמונה, נקודת האמונה. }

\ערך{אלהִי }\הגדרה{- }\משנה{עידון אלהי }\הגדרה{- (העידון) של }\הגדרה{החכמה\mycircle{°}}\הגדרה{ }\הגדרה{והמישרים\mycircle{°}}\הגדרה{, של הצדק }\הגדרה{והאורה\mycircle{°}}\הגדרה{ העדינה }\הגדרה{הרוחנית\mycircle{°}}\הגדרה{ }\מקור{[ע״א ד ו מ]}\צהגדרה{. }

\ערך{אלהים}\myfootnote{ ע׳ בהערה במדור שמות כינויים ותארים אלהיים, אלהים.\label{21}}\הגדרה{ - מנהיג ושולט }\מקור{[ע״ר א קיד]}\צהגדרה{. }\\\הגדרה{כל כח שבנבראים שיש לו איזה השפעה, }\צהגדרה{<שמוכרחת היא להיות מוגבלת, מאחר שהכח המשפיע בעצמו הוא מוגבל, בבחינת התחלתו ובבחינת סופו, וכל מוגבל הרי }\צהגדרה{מדת-הדין\mycircle{°}}\צהגדרה{ המצומצמת טבועה בתוכו>}\צמקור{ [עפ״י ע״ר ב פח]}\צהגדרה{.}

\ערך{אַלף }\הגדרה{- }\הגדרה{תרגום\mycircle{°}}\הגדרה{ של למד }\מקור{[ר״מ קיז]}\צהגדרה{. }\\\הגדרה{תרגום של למוד, <בעברית הוא ג״כ ממקור הארמי> ובא על המדרגה הירודה של הלמוד, הצד המתנוצץ אל התלמיד משפעתו של הרב, כתכונת האחורים לעומת הפנים }\מקור{[שם פג]}\צהגדרה{. }\\\הגדרה{הלומד מאחר, והוא בערך העתקה ואחורים של הפנים המאירים באור השכל המקורי. ההתלמדות האולפנית, מדת התרגום, שהוא הכשר וחינוך המביא אל המעמד המקורי העתיד }\מקור{[עפ״י שם קיג]}\צהגדרה{. }\\\הגדרה{הלימוד המתורגם, לימוד חינוכי, לימוד מכשיר, <שמביא באחריתו ללמוד מקורי, להבנת הלב> }\מקור{[עפ״י שם קיח]}\צהגדרה{. }\\\הגדרה{ע׳ במדור אותיות, למ״ד. }

\ערך{אלף }\הגדרה{- }\הגדרה{שור\mycircle{°}}\הגדרה{ }\מקור{[ר״מ פג]}\צהגדרה{. }

\ערך{אֵם }\הגדרה{- הוראת האמהות }\מקור{[ר״מ קכג]}\צהגדרה{. }\\\הגדרה{המשפעת שפע חיים ליצור בהולדו }\מקור{[ע״א ד ו עג]}\צהגדרה{.}\\\הגדרה{ע׳ במדור פסוקים ובטויי חז״ל.}

\ערך{אמונה}\myfootnote{ \textbf{אמונה} - הגדרות האמונה השונות חולקו לשתי מחלקות (שגבולן סומן ב ◊◊) : הקבוצה הראשונה עד ערך ׳תוכן האמונה׳ (ולא עד בכלל), מרכזת הגדרות למושג אמונה בסתם, ומקורותיה. ומערך ׳תוכן האמונה׳ ואילך הובאו הגדרות לבחינותיה השונות של האמונה, ענייניה, תכניה והשלכותיה.\newline
\textbf{גדר האמונה בכללה} - אמונות ודעות לרס״ג הקדמת המחבר, ״צריכים לבאר מה היא האמונה, ונאמר כי היא ענין עולה בלב לכל דבר ידוע בתכונה אשר הוא עליה, וכאשר תצא חמאת העיון יקבלנה השכל ויקיפנה ויכניסנה בלבבות ותמזג בהם, ויהיה בהם האדם מאמין בענין אשר הגיע אליו״. ובשומר אמונים הקדמון, ויכוח ראשון סי׳ לג (מיסוד המורה נבוכים, ח״א פ״נ) ״כי האמונה אינה האמירה בפה, כי אם התאמת הדבר במחשבת הלב והצטיירו בשכל״.\label{22}}\ערך{ }\הגדרה{- }\משנה{גדר האמונה בכללה }\הגדרה{- שאמיתת הענין הנודע קבועה בקרבו, לא מצד הידיעה לבדה כ״א מצד מנוחת הנפש הגמורה כשהוא מקבל אותה בקבלה שלמה, מבלי שיסתער בו מאומה נגד זה }\מקור{[ע״ר א שלז]}\צהגדרה{. }

\ערך{אמונה }\הגדרה{- }\הגדרה{דת\mycircle{°}}\הגדרה{}\מקור{ [קבצ׳ ב נג]}\צהגדרה{.}\\\הגדרה{ע״ע אמונה, נשמת האמונה}\צהגדרה{.}

\ערך{אמונה }\הגדרה{- }\משנה{מצות האמונה }\הגדרה{- מצוה מעשית, שיהי׳ הלב נמשך לאמונת }\הגדרה{התורה\mycircle{°}}\הגדרה{ לעיון }\הגדרה{העבודה\mycircle{°}}\הגדרה{ }\הגדרה{והמצות\mycircle{°}}\הגדרה{. שכל הדברים הברורים המאומתים שהודיענו }\הגדרה{השי״ת\mycircle{°}}\הגדרה{ }\הגדרה{בתורתו\mycircle{°}}\הגדרה{, נראה להשתדל שכשם שהם מאומתים מצד אמתתם בשכל, כן יהיו הרהורי הלב מלאים מהם, והיינו על ידי שירבה המשא ומתן בענפיהם בעיון והשכל איש איש כפי רוחב לבו}\צהגדרה{ [פנק׳ ג כו (}\צהגדרה{מא}\צהגדרה{״ה ב רעג)].}\\\משנה{צורת מצות האמונה האלהית }\הגדרה{- שהאלהים יתעלה, שהוא העושה כל המעשה הגדול הזה, אשר עינינו רואות אותו מסודר }\הגדרה{בעצה\mycircle{°}}\הגדרה{ ובחכמה גדולה ועמוקה ומדוקדקת מאד, הוא שהוציא אותנו }\הגדרה{ממצרים\mycircle{°}}\הגדרה{ מבית עבדים ונתן לנו את }\הגדרה{התורה\mycircle{°}}\הגדרה{, ולו שייכים ומיוחסים כל הדברים אשר הם שייכים ומיוחסים לאלהים <בין שמתבררים ע״פ אמתת הברור השכלי ובין שמתבררים ע״פ התורה והקבלה השלמה> }\צהגדרה{[עפ״י }\צהגדרה{ע}\צהגדרה{״ר א שלו].}

\משנה{אמונה}\צהגדרה{ - }\צמשנה{אמת האמונה}\myfootnote{ \textbf{האמונה}\textbf{ }- \textbf{הידיעה}\textbf{ }\textbf{ההכרה} - מוסיף רבנו שם באגרת: לכן בלשון התורה שבכתב: אברהם אבינו ״האמין בד׳״; ובלשון חז״ל בתושבע״פ: ״הכיר את בוראו״. \newline
ובהמשך דברי רבנו שם: ״הכרה ברורה של הסתכלות האמונה הזאת, נמשכת ומתבהרת לנו, בשלשלת הדורות, מתוך זכירת ימות עולם אל בינת שנות דור ודור, המפגישה אותנו עם הופעתנו המיוחדת הגדולה, שאין דומה ודוגמה לה בכל מערכת האנושיות, ושלשלת דורותיה והשפעתנו בתוכה״. וע׳ בפרקי הקיום הכשרון וההשפעה, שבס׳ כארי יתנשא.\newline
\label{23}}\צהגדרה{ - }\צהגדרה{הידיעה ההכרה ההבנה הברורה הפשוטה שכל הנמצאים שבעולם כולם נמשכים ומתגלים לנו מתוך מקור המציאות, וכלשונו של הרמב}\צהגדרה{״}\צהגדרה{ם}\צהגדרה{: ׳}\צהגדרה{שכל הנמצאים הם מתוך אמתת מציאותו}\צהגדרה{׳ }\צמקור{[אגרת רבינו, מי}\צהגדרה{״}\צמקור{ז בכסלו תשל}\צהגדרה{״}\צמקור{ז}\צהגדרה{. }\צמקור{מובא בשי׳ ג 241, הערה 89].}\\\צמשנה{רוממות-רוח-אמונה ואמתת-דעת-עליון }\צהגדרה{- תקף בהירותה של הכרת עולם ומלאו, שלמותו ומקוריותו }\צמקור{[ל״י ב (מהדורת בית אל תשס״ז) תסד].}

\ערך{אמונה }\הגדרה{- }\משנה{עיקר מגמת הדרכת האמונה }\הגדרה{- לבסס את כח }\הגדרה{המדמה\mycircle{°}}\הגדרה{ }\הגדרה{בטהרת\mycircle{°}}\הגדרה{ }\הגדרה{השכל\mycircle{°}}\הגדרה{ }\הגדרה{והמוסר\mycircle{°}}\הגדרה{ }\מקור{[פנק׳ ד שסט]}\צהגדרה{.}

\ערך{אמונה }\הגדרה{- }\משנה{כח האמונה }\הגדרה{- (הכח) להקשיב יפה ולקבל ממה שנמסר }\מקור{[א״א 66]}\צהגדרה{. }\\\הגדרה{להיות מוכן לקבל בתורת ידיעה את המסור לנו מאבותינו }\מקור{[שם 94]}\צהגדרה{.}\\\הגדרה{הקבלה את מה שנאמר מפי גדולי עולם}\מקור{ [ע״א ב ט רעב]}\צהגדרה{. }

\ערך{אמונה טבעית הסתכלותית }\הגדרה{- האמונה המופעה מהעולם }\מקור{[מ״ר 70]}\צהגדרה{. }\\\ערך{אמונה ניסית מסורתית }\הגדרה{- האמונה המופעה מהתורה }\מקור{[מ״ר 70]}\צהגדרה{. }\\\ערך{אמונה תוכית }\הגדרה{- האמונה המופעה ממעמקי הנשמה }\מקור{[מ״ר 70]}\צהגדרה{. }\\\הגדרה{ע״ע תשובה טבעית. תשובה אמונית. תשובה שכלית. }

\ערך{אמונה }\הגדרה{- שלמות המדעים המושכלים שהם שורשי }\הגדרה{התורה\mycircle{°}}\הגדרה{ }\מקור{[א״ה א 702 (מהדורת תשס״ב ח״ב 30)]}\צהגדרה{. }

\תערך{אמונה }\הגדרה{- }\תמשנה{רוח\mycircle{°} האמונה הקדושה\mycircle{°} }\הגדרה{- }\תהגדרה{הכרה עצמית-פנימית }\תהגדרה{כ״טביעת-עין״\mycircle{°}}\תהגדרה{, שהאדם מכיר את העולם מתוך עצמותו }\תמקור{[מ״ר 488]. }

\תערך{אמונה }\הגדרה{- }\תמשנה{כח האמונה }\הגדרה{-}\תהגדרה{ הצלם}\תקלה{-}\תהגדרה{}\תהגדרה{האלהי\mycircle{°}}\תהגדרה{ המאיר מתוך תוכם של ברואיו של }\תהגדרה{הקב״ה\mycircle{°}}\תהגדרה{, והוא עצם מהותה של הנשמה, התשוקה }\תהגדרה{להתדבקות-אלהית\mycircle{°}}\תהגדרה{ }\תמקור{[עפ״י מ״ר 487]. }

\תערך{אמונה }\הגדרה{- }\תמשנה{עומק האמונה }\הגדרה{-}\תהגדרה{ }\תהגדרה{להתדבק\mycircle{°}}\תהגדרה{ בבורא כל }\תהגדרה{העולמים\mycircle{°}}\תהגדרה{ ב״ה, בכל סדרי המחשבה והמעשה, והוא גילוי הכח השלם (של האמונה) }\תמקור{[מ״ר 488].}\\\הגדרה{ע׳ במדור הכרה והשכלה והפכן, מחשבה, המחשבה היסודית. }\\\משנה{אמונה }\צהגדרה{- הידיעה-ההכרה הבהירה והמלאה בד׳ אלהים, מקור חיי כל עולמים, הממלאה את כל חדרי הנפש, הרוח והנשמה, הכליות והלב והגוף כולו, ואשר לפיכך הם כולם ממולאים מתוכה }\צהגדרה{אהבה-עליונה\mycircle{°}}\צהגדרה{ של }\צהגדרה{דבקות\mycircle{°}}\צהגדרה{ חיונית }\צהגדרה{לאבינו-שבשמים\mycircle{°}}\צהגדרה{, אשר מלך בטרם כל יצור נברא }\צהגדרהמודגשת{- }\צהגדרה{ואחריו }\צמקור{[נ״ה ט].}\\\צהגדרה{ענג רוממות ושבע נפש פנימי, (של) מלא ההכרה הברה והתמה ורווי עז }\צהגדרה{הבטחון\mycircle{°}}\צהגדרה{ (ש)על ידי השקפת האחדות העליונה, שהנהגת }\צהגדרה{ההשגחה\mycircle{°}}\צהגדרה{ העליונה מסבבת כל המעשים וכולם מלאים הם }\צהגדרה{מזיו\mycircle{°}}\צהגדרה{ הטוב הכללי המקיף הכל וכוללם יחד, של ״מה דעבד רחמנא לטב עביד״, ״עושה שלום ובורא הכל״ }\צמקור{[עפ״י א״ל קצה].}\\\צמשנה{כח האמונה }\צהגדרה{- ההכרה העליונה והדבקה באמתת צדיקו של עולם, עז מלכותו ומקור חיותו, אשר מראשית דרכו מאז ועד אחרית מופיעה המשכת כל המפעלים ושעשועי התולדה }\צמקור{[נ״ה יט].}\\\צהגדרה{מציאות נפשית פנימית של זיקת האדם, הנברא }\צהגדרה{בצלם-אלהים\mycircle{°}}\צהגדרה{, אל מקורו, יוצרו, רבון-העולמים, שהיא ממלאה אותו כולו ומתוך-כך מתגלית בסידורי מעשיו ודורותיו היחידיים והצבוריים }\צמקור{[עפ״י ל״י ג קעז (מהדורת בית אל ב תשס״ג תה)].}\\\צהגדרה{כ}\צהגדרה{ִּ}\צהגדרה{וו}\צהגדרה{ּ}\צהגדרה{נ}\צהגדרה{ָ}\צהגדרה{ה העצמי של תפיסת-עולם-ואדם שלמה וכוללת, בטבעיותו }\צהגדרה{הרוחנית\mycircle{°}}\צהגדרה{ והחיונית }\צמקור{[עפ״י ל״י ב (מהדורת בית אל תשס״ג) תז].}

\ערך{אמונה }\הגדרה{- ע״ע בקשת אלהים. ע׳ במדור פסוקים ובטויי חז״ל, דרישת ד׳. }

\ערך{אמונה }\הגדרה{- ע״ע הכרה אלהית. ע׳ במדור פסוקים ובטויי חז״ל, דעת אלהים.}

\ערך{אמונה }\הגדרה{- }\משנה{כח האמונה }\הגדרה{- שלמות תמימות טבעו הרוחני של האדם }\צהגדרה{[עפ״י }\צהגדרה{ע}\צהגדרה{״א ג ב קנג].}\\\הגדרה{בבחי׳ נפש, הרגשה, כח המקבל שבהוי׳ }\מקור{[קבצ׳ ג קכג]}\צהגדרה{.}\\\ערך{אמונה }\הגדרה{- }\מעוין{◊}\הגדרה{ היסוד הטבעי לכל }\הגדרה{טובה\mycircle{°}}\הגדרה{ ולכל }\הגדרה{מוסר\mycircle{°}}\הגדרה{ }\הגדרה{עליון\mycircle{°}}\הגדרה{ הנעוץ בעומק הטבע }\הגדרה{הישר\mycircle{°}}\הגדרה{, האנושי. מקור חייו הטבעיים, }\הגדרה{הרוחניים\mycircle{°}}\הגדרה{ (של האדם) הנותנים לו שלות לב ושמחת נפש בעוה״ז ואחרית ותקוה }\הגדרה{לעוה״ב\mycircle{°}}\הגדרה{ }\מקור{[עפ״י מ״ר 225]}\צהגדרה{.}\\\הגדרה{תוכן הנפש היותר עדין, והמקור לכל התרבות האידיאלית שהאנושיות כולה עורגת אליה}\מקור{ [פנק׳ ב קצד]}\צהגדרה{. }

\ערך{אמונה }\הגדרה{-}\משנה{ הנטיה האמונית }\הגדרה{- מקור }\הגדרה{הקדושה\mycircle{°}}\הגדרה{ בעולם כולו }\מקור{[קבצ׳ א נד]}\צהגדרה{.}\\\ערך{אמונה }\הגדרה{- }\משנה{נשמת האמונה }\הגדרה{- }\הגדרה{אור-החיים\mycircle{°}}\הגדרה{ האלהיים שבתוך }\הגדרה{הדת\mycircle{°}}\הגדרה{ }\מקור{[עפ״י קבצ׳ ב נג]}\צהגדרה{.}

\ערך{אמונה }\הגדרה{- }\משנה{עיקר האמונה }\הגדרה{- }\מעוין{◊}\הגדרה{ עיקר האמונה היא }\הגדרה{בגדולת\mycircle{°}}\הגדרה{ שלמות }\הגדרה{אין-סוף\mycircle{°}}\הגדרה{. שכל מה שנכנס בתוך הלב הרי זה ניצוץ בטל לגמרי לגבי מה שראוי להיות משוער, ומה שראוי להיות משוער אינו עולה כלל בסוג של ביטול לגבי מה שהוא באמת }\מקור{[א׳ קכד]}\צהגדרה{. }\\\משנה{שורש האמונה }\הגדרה{- להביע בפנימיות הנשמה את גדולת-א״ס }\מקור{[מ״ה אמונה לד]}\צהגדרה{. }\\\הגדרה{מתוך }\הגדרה{החכמה\mycircle{°}}\הגדרה{ הכמוסה }\הגדרה{בנשמה\mycircle{°}}\הגדרה{, שעל ידה היא מכירה בגודל }\הגדרה{האמת-האלהית\mycircle{°}}\הגדרה{, אלא שאינו יכול להוציא אל הפועל את הפרטים, מתוך כך הוא מתקשר הרבה להאמין בהפרטים שהוא מרגיש בפנימיות לבבו, שהם הם מגלים את האמת הגדולה בכל האופנים שדרכה להגלות }\משנה{-}\הגדרה{ במעשה, דיבור ומחשבה, ברגש, בדמיון, במזג ובנטיות נפשיות, ובמעוף חיים פנימי ועליון מכל הגיון לבב ומכל הקשבה מוגבלת}\מקור{ [עפ״י קובץ א תתמט]}\צהגדרה{.}\\\תמשנה{נקודת האמונה }\הגדרה{-}\תהגדרה{ יחס האדם להאין-סוף ברוך הוא, וכל העולמים כולם מתרכזים בנקודה אין-סופית זו }\תמקור{[מ״ר 487]. }\\\הגדרה{ע״ע אלהי, הנקודה האלהית. ע׳ במדור מונחי קבלה ונסתר, לאשתאבא בגופא דמלכא.}

\ערך{אמונה }\הגדרה{- }\משנה{חוש האמונה }\הגדרה{- החוש הטבעי היסודי של הנשמה, שנובע באדם מנשמת }\הגדרה{חֵי-העולמים\mycircle{°}}\הגדרה{ מנשמת כל היקום, כל היש }\מקור{[עפ״י א״א 112]}\צהגדרה{. }\\\מעוין{◊ }\צמשנה{היחש של האמונה בטהרתו}\צהגדרה{ בא מפני עצמיות הטבע הפנימי של הנפש האנושית שהיא קשורה בקשר הוייתה בשורש חיי כל החיים, במקור כל ההויה }\מקור{[ע״א ד ט קכה]}\צהגדרה{. }

\ערך{אמונה}\myfootnote{ בע״א ד יא יג מציין הרב שתי בחינות באמונה. סגולית נשמתית - ״\textbf{האמונה האלהית העליונה} שאינה מתדמה כלל לשום רישום של ידיעה והכרה בעולם, כי היא הסגולה של כל יסוד החיים, של אורם, של חיי חייהם, של זיום ותפארתם. והסגולה הזאת מצד ערכה הפנימי, שאין לו שום הערכה בשום צד המתדמה לו בתואר חצוני, הוא ענין סגולי בישראל, לא מצד בחירת נפשם בפרט אלא מצד מחצב הקדושה וסגולת ירושת אבות שלהם״. ולאידך גיסא, הכרתית חיצונית - ״\textbf{כח}\textbf{ האמונה מצד התגלותה החצונית}, שאפשר לו להגלות בפועל, בהכרה, ברגש, בביטוי ובמעשה, ששם אין האור הגנוז של שלמות חיי האמונה זורח״. בהתאם חולקו ההגדרות כאן לשתי קבוצות. \label{24}}\ערך{ }\הגדרה{- }\משנה{(בד׳) }\הגדרה{- }\הגדרה{שירת-החיים\mycircle{°}}\הגדרה{, }\הגדרה{שירת\mycircle{°}}\הגדרה{ המציאות, שירת ההויה. שירת העולם }\הגדרה{העליונה\mycircle{°}}\הגדרה{}\myfootnote{ \textbf{האמונה}\textbf{ היא שירת העולם העליונה} - ע׳ זוהר תרומה קלט: ״מ״י, רזא דעלמא עלאה איהו דהא מתמן נפקא שירותא לאתגליא רזא דמהימנותא״.\label{25}}\הגדרה{. ומקורה הוא הטבע }\הגדרה{האלהי\mycircle{°}}\הגדרה{ שבעומק }\הגדרה{הנשמה\mycircle{°}}\הגדרה{, }\הגדרה{העונג\mycircle{°}}\הגדרה{ של ההסתכלות }\הגדרה{הפנימית\mycircle{°}}\הגדרה{ של }\הגדרה{אושר\mycircle{°}}\הגדרה{ אין סוף }\מקור{[א״א 66, 123]}\צהגדרה{. }\\\הגדרה{שירת חיינו}\myfootnote{ \textbf{שירת חיינו} - ע״ע הסברת והגדרת האמונה בס׳ כארי יתנשא, פרק שלישי, האלהות, עמ׳ 30-32.\label{26}}\הגדרה{, (ה)מכון ששם האמת העליונה שוכנת בכל }\הגדרה{יפעתה\mycircle{°}}\הגדרה{ }\הגדרה{והדרה\mycircle{°}}\הגדרה{ }\מקור{[קבצ׳ ב קיב]}\צהגדרה{.}\\\הגדרה{האמת הגדולה, התוכן העזיז, מלא הקודש, משך החיים האמתיים, וכל שיגוב ואידור במילואו }\צהגדרה{[}\צהגדרה{א}\צהגדרה{״ק א עא].}\\\משנה{האמונה העליונה }\הגדרה{- שירת העולם ואמת העולם }\מקור{[א׳ קכז]}\צהגדרה{. }\\\משנה{הארת האמונה }\הגדרה{- }\הגדרה{הארה\mycircle{°}}\הגדרה{ כללית למעלה מכל הערכים ובזה היא מבססת את הערכים כולם }\מקור{[שם קכה (א״א 68)]}\צהגדרה{. }\\\משנה{אור האמונה }\הגדרה{- שורש כל הקדושות }\מקור{[א״א 141]}\צהגדרה{. }\\\משנה{האמונה האלהית }\הגדרה{- }\הגדרה{המחשבה\mycircle{°}}\הגדרה{ היותר }\הגדרה{עליונה\mycircle{°}}\הגדרה{, שמתוך גבהה היא משתפלת בכל השדרות גם היותר שפלות, ובכל דרגה ושדרה היא מתארת כפי ערכה }\מקור{[מ״ה אמונה ג (א״א 7-66)]}\צהגדרה{. }\\\משנה{יסוד האמונה }\הגדרה{- התוכן המקיים את הצביון העליון הבלתי מדוד ושקול, הבלתי מצומצם ומתואר, בתוך כל מה שמתואר ומוגבל, המחיה את כל החוקים, בהתמשכם }\הגדרה{מראשית\mycircle{°}}\הגדרה{ נביעתם עד }\הגדרה{אחרית\mycircle{°}}\הגדרה{ }\הגדרה{המגמות\mycircle{°}}\הגדרה{ כולן, עד אין קץ למורד }\מקור{[ע״ר א לה-ו]}\צהגדרה{.}\\\משנה{שרש האמונה בטהרתה\mycircle{°}}\הגדרה{ - חיבור הקדושה הרוממה, של }\הגדרה{אור-אין-סוף\mycircle{°}}\הגדרה{, עם הקדושה החודרת בכל העולמים ובכל היצורים כולם }\מקור{[עפ״י ע״ר א קיד]}\צהגדרה{.}\\\תמשנה{תוכן האמונה }\הגדרה{-}\תהגדרה{ }\תהגדרה{האידיאליות\mycircle{°}}\תהגדרה{ }\תהגדרה{הנשמתית\mycircle{°}}\תהגדרה{, שהיא התשוקה העליונה והרוממה }\תהגדרה{להתדבקות-אלהית\mycircle{°}}\תהגדרה{, שתנאיה הם השואה גמורה ומוחלטת בין המעשה }\תהגדרה{והמחשבה-האלהית\mycircle{°}}\תהגדרה{ }\תמקור{[מ״ר 494]. }\\\משנה{אמונה, יסודה העצמי }\הגדרה{- נקודת }\הגדרה{הציור\mycircle{°}}\הגדרה{ האלהי ברקמת החיים הפנימית (של) היחיד או הצבור, המשפחה או }\הגדרה{האומה\mycircle{°}}\הגדרה{, המפלגה או הסיעה, העושה את הרקמה }\הגדרה{הנפשית\mycircle{°}}\הגדרה{ היותר חטיבית, יותר איתנה ויותר חודרת }\מקור{[עפ״י א״א 78]}\צהגדרה{. }\\\משנה{האמונה האלהית }\הגדרה{- }\הגדרה{הסגולה\mycircle{°}}\הגדרה{ של כל יסוד }\הגדרה{החיים\mycircle{°}}\הגדרה{, של }\הגדרה{אורם\mycircle{°}}\הגדרה{, של חיי חייהם, של }\הגדרה{זיום\mycircle{°}}\הגדרה{ }\הגדרה{ותפארתם\mycircle{°}}\הגדרה{ }\מקור{[ע״א ד יא יג]}\צהגדרה{. }\\\משנה{נקודת קדושת האמונה האלהית }\הגדרה{- יסוד }\הגדרה{יראת-ד׳\mycircle{°}}\הגדרה{ באמת }\מקור{[א״א 95]}\צהגדרה{. }\\\משנה{אמונה }\הגדרה{- }\הגדרה{היראה\mycircle{°}}\הגדרה{, התוכן }\הגדרה{האצילי\mycircle{°}}\הגדרה{, }\הגדרה{המקודש\mycircle{°}}\הגדרה{, האלהי, של האומה }\מקור{[א״ש טו יא]}\צהגדרה{. }\\\הגדרה{היחס }\הגדרה{האמיתי\mycircle{°}}\הגדרה{ הפנימי אל יסוד המציאות במקורו, המשיב לנשמה את צביונה }\מקור{[עפ״י א״א 3-82 (מ״ר 74)]}\צהגדרה{. }\\\הגדרה{}\הגדרה{תהילת-ד׳\mycircle{°}}\הגדרה{ וסקירת השלמות העליונה }\מקור{[א״ק ד תיא (קובץ ז קמג)]}\צהגדרה{.}\\\משנה{העומק התמציתי של מהות האמונה }\הגדרה{- יסוד }\הגדרה{העלוי\mycircle{°}}\הגדרה{ הנשמתי היותר בהיר של האדם }\מקור{[א״א 58]}\צהגדרה{. }\\\משנה{מהות האמונה }\הגדרה{- התעמקות חיה בהענינים האלהיים }\מקור{[א״א 68 (קובץ א שלה)]}\צהגדרה{. }\\\משנה{רז האמונה }\הגדרה{- אור האלהי שבנשמה, המבקש את }\הגדרה{הרוח-הטהור\mycircle{°}}\הגדרה{, את הקודש הנשגב בחיים, בהרגשה ובידיעה }\מקור{[קובץ ח ע]}\צהגדרה{. }\\\מעוין{◊}\הגדרה{ האמונה }\הגדרה{והאהבה\mycircle{°}}\הגדרה{ הן עצם החיים בעוה״ז }\הגדרה{ובעוה״ב\mycircle{°}}\הגדרה{ }\מקור{[עפ״י א׳ סט]}\צהגדרה{. }\\\משנה{האמונה }\הגדרה{- אינה לא }\הגדרה{שכל\mycircle{°}}\הגדרה{ ולא }\הגדרה{רגש\mycircle{°}}\הגדרה{, אלא גלוי עצמי היותר יסודי של מהות הנשמה, <שצריך להדריך אותה בתכונתה, וכשאין משחיתים את דרכה הטבעי לה, איננה צריכה לשום תוכן אחר לסעדה, אלא היא מוצאה בעצמה את הכל> }\מקור{[מ״ר 70]}\צהגדרה{. }\\\משנה{אמונה }\הגדרה{- }\משנה{הגיונה העליון }\הגדרה{- הגילוי האלהי שבנשמה שלמעלה מכל }\הגדרה{דעת\mycircle{°}}\הגדרה{ }\מקור{[ע״ט יח (א״א 56)]}\צהגדרה{. }\\\משנה{חוש האמונה האלהית }\הגדרה{- דעת הדעות והרגשת ההרגשות, שמחברת את ההויה }\הגדרה{הרוחנית\mycircle{°}}\הגדרה{ של האדם, המציאותית בפועל, עם ההויה הרוחנית העליונה, ומערבת את החיים שלו עם החיים המציאותיים הרמים מכל גבול ונעדרי כל חולשה פיסית }\מקור{[מ״ר 1]}\צהגדרה{.}\\\הגדרה{הרגש של הכלל, (ש)הכל אצור ב}\צהגדרה{ו <בעומק הרגש מונח הכל }\צהגדרהמודגשת{-}\צהגדרה{ כל הפרטים החבויים בגנזי השכל העליון, אשר רק חלקים ממנו הולכים ומתגלים על פי ארחות החיפוש והמחקר, הרגש הוא תופש בהם שלא בהדרגה, איננו צריך לעזר ההתחכמות בשביל להכיר את טובם>}\צמקור{ [קבצ׳ ב פח (פנק׳ ד סח)]}\צהגדרה{. }\\\הגדרה{ע׳ בנספחות, מדור מחקרים, בטחון לעומת אמונה.}

\ערך{אמונה}\footref{24} \הגדרה{- המסקנה היותר מזהירה של הלימודים היותר }\הגדרה{רמים\mycircle{°}}\הגדרה{ ונשאים המוארים }\הגדרה{באור\mycircle{°}}\הגדרה{ }\הגדרה{עליון\mycircle{°}}\הגדרה{ }\מקור{[א״י נ]}\צהגדרה{.}\\\מעוין{◊}\הגדרה{ האמונה מקפת את כל הידיעות לעשות חטיבה כללית מכל הפרטים כולם, ובזה היא נותנת חיים נצחיים לכל הזוכים }\הגדרה{לאורה\mycircle{°}}\הגדרה{, והיא מחיה בלשד פנימיותה את }\הגדרה{המוסר\mycircle{°}}\הגדרה{, את חיי החברה ואת חיי היחיד, כשם שהיא מחיה את כל }\הגדרה{העולמים\mycircle{°}}\הגדרה{ כולם, }\הגדרה{מראש\mycircle{°}}\הגדרה{ ועד סוף }\מקור{[א״ק ג קז (א״א 68)]}\צהגדרה{. }\\\מעוין{◊ ◊}

\ערך{אמונה}\footref{22} \myfootnote{ צריך להבחין בסדרת הגדרות האמונה הבאות (שגבולן סומן ב ◊) בין האמונה בבחינתה כממד אונטי, לבין האמונה כממד פסיכי; ובהגדרות המשתפות ביניהם.\label{27}}\הגדרה{ - }\משנה{תוכן האמונה }\הגדרה{- היסוד הקדמון לכל היש, החובק כל המציאות והממלא אותה עצמת הקיום וההמשכה ההוייתית. שורה היא האמונה בכל הנברא ונוצר ונעשה, שורה היא ברומי מרומים, משתפלת היא בשפלי שפלים. }\הגדרה{וקדמותה\mycircle{°}}\הגדרה{ העליונה של אור האמונה וצחצחות טבעיותה האיתן, זהו המכריח את כל החיים האנושיים להיות מוטבעים על פיה }\מקור{[ע״א ד ט קכה]}\צהגדרה{. }\\\מעוין{◊ }\תמשנה{תוכן האמונה}\תהגדרה{ אינו רעיון כ״א מציאות גמורה, הנמצאת בכל חלקי היצירה, גם בדומם, והוא סוד קיומו של עולם }\תמקור{[מ״ר 489].}\\\תמשנה{כח האמונה }\תהגדרה{הוא כח כללי ולא פרטי, אור מקיף הנובע ממקור ההויה כולה ומתפשט על הכל }\תמקור{[מ״ר 489]. }\\\תערך{אמונה }\הגדרה{-}\תהגדרה{ עריגת כל }\תהגדרה{העולמים\mycircle{°}}\תהגדרה{ כולם }\תהגדרה{לחי-העולמים\mycircle{°}}\תהגדרה{ }\תמקור{[מ״ר 494]. }

\ערך{אמונה }\הגדרה{- }\משנה{האמונה הרבה}\myfootnote{ \textbf{האמונה}\textbf{ הרבה} - ע׳ י׳ מאמרות לרמ״ע, אכ״ח ח״ג יא, ביד יהודה ס״ק יד.\label{28}}\הגדרה{ - }\הגדרה{האמונה-האלהית\mycircle{°}}\הגדרה{ (ה)חוקקת את התפקידים של כל כוחות ההויה, שיעשו את עבודתם בכל }\הגדרה{עז\mycircle{°}}\הגדרה{ ומרץ, בכל איתניות של קיום ונצחון, שהכח הכביר של אור האמונה האלהית מאיר עליהם, למסור בידם תוכנים וענינים מתוקנים עומדים על מכונם, ולקלקל ולהפסיד את צורתם וערכם, כדי להוציא ע״י קלקול זה ערכים יותר נפלאים בחדוש פנים }\הגדרה{לעילוי\mycircle{°}}\הגדרה{ ולשבח. }\הגדרה{מהסתעפות\mycircle{°}}\הגדרה{ }\משנה{האמונה הרבה}\הגדרה{, הכוללת כל היקום, מתגלה }\הגדרה{הדר\mycircle{°}}\הגדרה{ כח המחיה והמפריא, המעודד והמחדש, בכל תוכן שיוצאת ממנה ערות של חיים, אחרי שכבר צללי }\הגדרה{המות\mycircle{°}}\הגדרה{ באו ופרשו עליו את אפלתם. האדם, כל כוחות החיים וההויה, מתעוררים אחרי בלותם, בכח }\הגדרה{אור\mycircle{°}}\הגדרה{ האמונה-האלהית-העליונה, מיסוד }\הגדרה{אמונת-אומן\mycircle{°}}\הגדרה{ של }\הגדרה{חכמת\mycircle{°}}\הגדרה{ }\הגדרה{ודעת\mycircle{°}}\הגדרה{, אשר באמונת העתות, בכל מסיביהן ותמורותיהן }\מקור{[עפ״י ע״ר א ג]}\צהגדרה{. }\\\הגדרה{האמונה המסדרת כוחות ההויה }\מקור{[עפ״י רצי״ה שם ב תלט]}\צהגדרה{. }\\\הגדרה{ע׳ במדור תורה, ״אמון רבתא״. }

\ערך{האמונה הגדולה }\הגדרה{- אמונת עולמים. האמונה הגדולה השרויה בעומק האהבה, הפועלת ברוח חיים }\הגדרה{בשכל-עליון\mycircle{°}}\הגדרה{ ומפואר, בסדר והתאמה בכללות המון היצורים }\הגדרה{והעולמים\mycircle{°}}\הגדרה{ כולם. האמונה הפנימית, היודעת את כבודה, את אשרה וגבורתה המדושנת עונג פנימי, המכירה שהיא בכל מושלת, שהיא מחלקת במדה ובמשקל של }\הגדרה{צדק\mycircle{°}}\הגדרה{ }\הגדרה{ויושר\mycircle{°}}\הגדרה{, }\הגדרה{אור\mycircle{°}}\הגדרה{ }\הגדרה{וחיים\mycircle{°}}\הגדרה{, לכל המון יצורים לאין תכלית, על פי סדר וערך של יחושים נאמנים, דרוכים ברוח }\הגדרה{שלום\mycircle{°}}\הגדרה{ }\הגדרה{ואמת\mycircle{°}}\הגדרה{ }\מקור{[עפ״י א׳ מב-מג]}\צהגדרה{. }

\ערך{אמונה }\הגדרה{- }\משנה{״אֹמֶן״}\myfootnote{ \textbf{אומן} - עפ״י הקדמת ר״י הארוך בר קלונימוס האשכנזי (המיוחסת לראב״ד) לס״י, הנתיב הג׳.\label{29}}\משנה{ }\הגדרה{- }\הגדרה{החפש\mycircle{°}}\הגדרה{, }\הגדרה{אב\mycircle{°}}\הגדרה{ }\הגדרה{האמונה-העליונה\mycircle{°}}\הגדרה{, השכל המקודש שהוא יסוד }\הגדרה{החכמה\mycircle{°}}\הגדרה{ הקדומה, שמכוחו האמונה נאצלת }\מקור{[עפ״י א״א 128, שם 17, (77)]}\צהגדרה{. }\\\הגדרה{ע׳ במדור שמות כינויים ותארים אלהיים, ״אהיה אשר אהיה״. ע׳ במדור הכרה והשכלה והפכן, השכלה העליונה. ע׳ במדור פסוקים ובטויי חז״ל, אמונות אמיתיות.}

\ערך{אמונה }\הגדרה{- }\משנה{״אמונת אומן\mycircle{°}״}\myfootnote{ \textbf{אמונת}\textbf{ אומן} - עפ״י ישעיה כה א.\label{30}}\משנה{ }\הגדרה{- }\הגדרה{האמונה\mycircle{°}}\הגדרה{ }\הגדרה{העליונה\mycircle{°}}\הגדרה{ }\הגדרה{שגורל\mycircle{°}}\הגדרה{ העתיד בידה הוא. המתק הגדול של }\הגדרה{הנועם-העליון\mycircle{°}}\הגדרה{ אשר לאור עולם שלעתיד, נועם ד׳ המתבקש בכל שכלול של }\הגדרה{קדושה\mycircle{°}}\הגדרה{ }\מקור{[א״א 128, 80]}\צהגדרה{. }\\\הגדרה{הנקודה הגורלית העליונה של כל האדם וכל היצור, בה מתגלה הערך היותר עצמי וטפוסי, היותר פנימי ומכוון אל הטוהר המהותי של האדם, שבה ספיגת כל החיים כולם, כל }\הגדרה{הזיו\mycircle{°}}\הגדרה{ }\הגדרה{והזוהר\mycircle{°}}\הגדרה{ שבעולם, כל }\הגדרה{השלום\mycircle{°}}\הגדרה{ }\הגדרה{והאושר\mycircle{°}}\הגדרה{ שבעולם, כל }\הגדרה{החיל\mycircle{°}}\הגדרה{ }\הגדרה{והחוסן\mycircle{°}}\הגדרה{ שבעולם, }\הגדרה{השושנה-העליונה\mycircle{°}}\הגדרה{ חבצלת השרון }\מקור{[עפ״י שם 133]}\צהגדרה{. }\\\הגדרה{האמונה הרוממה המושכלת, מעמק האמונה }\הגדרה{בטהרתה\mycircle{°}}\הגדרה{ }\הגדרה{הפנימית\mycircle{°}}\הגדרה{, שהמחשבות הנובעות ממנה הן }\הגדרה{החפשיות\mycircle{°}}\הגדרה{ באמת, שאין עליהן שום עול של הגבלה, והן המתדמות ביותר למקור היצירה האלהית, }\הגדרה{״ואהיה\mycircle{°}}\הגדרה{ אצלו אמון״. מקור כל השמחות }\הגדרה{והשעשועים\mycircle{°}}\הגדרה{ האציליים, ״משחקת לפניו בכל עת ושעשועי את בני אדם״ }\מקור{[עפ״י שם 55, 128, 80]}\צהגדרה{. }\\\הגדרה{}\הגדרה{האמונה-האלהית\mycircle{°}}\הגדרה{ העליונה, שבישראל, מצד ערכה הפנימי, שאין לו שום הערכה בשום צד המתדמה לו בתואר חיצוני, הוא ענין }\הגדרה{סגולי\mycircle{°}}\הגדרה{, לא מצד בחירת נפשם בפרט אלא מצד הקדושה וסגולת ירושת }\הגדרה{אבות\mycircle{°}}\הגדרה{ }\מקור{[עפ״י ע״א ד יא יג]}\צהגדרה{. }\\\הגדרה{}\הגדרה{האמונה-האמיתית\mycircle{°}}\הגדרה{. האמונה העליונה }\מקור{[קובץ א קס, תקפט]}\צהגדרה{.}\\\הגדרה{האמונה המחי׳ את בני׳ }\הגדרה{באור-ד׳\mycircle{°}}\הגדרה{ ובשמירת }\הגדרה{התורה\mycircle{°}}\הגדרה{ }\הגדרה{והמצוה\mycircle{°}}\הגדרה{ באמת ובתמים }\מקור{[אג׳ א קטו]}\צהגדרה{.}\\\משנה{האמונה הגדולה }\הגדרה{- }\הגדרה{הציור\mycircle{°}}\הגדרה{ }\הגדרה{העליון\mycircle{°}}\הגדרה{ של ההשכלה }\הגדרה{הרוחנית\mycircle{°}}\הגדרה{ (ש)איננו מצטמצם בשכל }\הגדרה{הגיוני\mycircle{°}}\הגדרה{, (שאורו) הוא }\הגדרה{זיו\mycircle{°}}\הגדרה{ החיים החזק; }\הגדרה{חוש-האמונה-האלקי\mycircle{°}}\הגדרה{, בכל גודל }\הגדרה{עזוזו\mycircle{°}}\הגדרה{, זהו החיים }\הגדרה{האמיתיים\mycircle{°}}\הגדרה{, החיים ששום }\הגדרה{מות\mycircle{°}}\הגדרה{ אין עמם, החיים ששמחתם איננה נחלשת משום צרה יגון ואנחה, החיים שהטובה }\הגדרה{והברכה\mycircle{°}}\הגדרה{ שרויים בם, עמם דבקים }\הגדרה{שמחת\mycircle{°}}\הגדרה{ עולמים בלא עצב, }\הגדרה{חדוה\mycircle{°}}\הגדרה{ עליונה רחבה ועשירה, בלא שום דלדול ורפיון }\מקור{[עפ״י א״א 131 (מ״ר 75)]}\צהגדרה{. }\\\הגדרה{הכח הרוחני שבעומק קדושת הנשמה, רוח החיים הפועם במלוא הנשמה, רוח אלהים שבלב, החי בתוכיותה של הנשמה }\צהגדרה{[עפ״י א״ת ב }\צמקור{105}\צהגדרה{].}

\ערך{אמונה }\הגדרה{- }\משנה{היסוד העליון של האמונה }\הגדרה{- }\הגדרה{הארת\mycircle{°}}\הגדרה{ }\הגדרה{הקודש\mycircle{°}}\הגדרה{ בצורה העליונה שלמעלה מן }\הגדרה{הטבע\mycircle{°}}\הגדרה{, המושלת על הטבע ומשכללתו }\מקור{[א״א 128]}\צהגדרה{. }\\\הגדרה{האמונה-העליונה, האמונה-הגדולה, אמונת-אומן }\מקור{[עפ״י שם 131, 133]}\צהגדרה{. }\\\הגדרה{ע״ע אמונה, האמונה הרבה. ע״ע השכלה העליונה.}\\\מעוין{◊}

\ערך{אמונה }\הגדרה{- }\משנה{היסוד הטבעי של האמונה }\הגדרה{- עריגת הקודש שבנפש האדם }\מקור{[א״א 128]}\צהגדרה{. }\\\משנה{טבע האמונה-האלהית\mycircle{°}}\הגדרה{ - עומק הטבע הנפשי, תאות }\הגדרה{הדבקות-האלהית\mycircle{°}}\הגדרה{ ברעיון ובחפץ פנימי, (ש)היא תאוה וחמדה עליונה, חזקה מכל התאוות שבעולם }\מקור{[שם 116]}\צהגדרה{. }\\\משנה{האמונה הטבעית }\הגדרה{- עריגת הקודש, החשק הפנימי, של הדבקות-האלהית }\מקור{[עפ״י שם 108 (קובץ ז קמא)]}\צהגדרה{. }\\\משנה{הארת האמונה הטבעית }\הגדרה{- אור אלהים המפעם בנשמה בכחו הגדול מצד עצמו <חוץ ממה שהוא מואר באורה של תורה, של מורשת אבות וקבלה> }\מקור{[א״א 107 (קובץ ז פ)]}\צהגדרה{. }\\\מעוין{◊ }\משנה{שני צדדים בטבע האמונה: צד החסד שלה }\הגדרה{- התוך הרוחני, הגרעין האידיאלי, שבאמונה, הצורה השכלית של החובה והמצוה האלהית העליונה שבה, הצורה הטבעית של האמונה, <בצורה הטבעית של הקודש }\הגדרה{שבכנסת-ישראל\mycircle{°}}\הגדרה{ חקוקים הם כל תוכני הצורות של המצות כולן וסעיפיהן, כל התורה כולה>; ו}\משנה{צד הגבורה שבה }\הגדרה{- ההמשכה הטבעית, התאוה הנפשית הפנימית, החומר הרוחני של האמונה, עריגת הקודש של כל העולם, שאפשר לה להיות בישרי לב שבכל האדם כולו }\מקור{[עפ״י שם 128, 117]}\צהגדרה{. }\\\הגדרה{ע׳ במדור מועדים וחגים, פסח. ושם, סוכות. ע״ע יהדות טבעית, ע״ע יהדות ניסית. ע׳ במדור פסוקים ובטויי חז״ל, ״עמוסי בטן״. ע׳ במדור אליליות ודתות, קליפת האגוז. ושם, ״יצרא דעבודה זרה״. ע״ע רליגיוזיות, הטבעיות הרליגיוזית. }

\ערך{אמונה }\הגדרה{- }\משנה{עבודת האמונה }\הגדרה{- שכלול האמונה ורעיית-האמונה}\myfootnote{ \textbf{רעיית}\textbf{ האמונה} - עפ״י תהילים לז ג.\label{31}}\הגדרה{, במעשים טובים ובמדות טובות }\הגדרה{בתורה\mycircle{°}}\הגדרה{ }\הגדרה{וחכמה\mycircle{°}}\הגדרה{ עליונה }\מקור{[עפ״י א״א 76, 77]}\צהגדרה{. }

\ערך{אמונה }\הגדרה{- }\משנה{עיקרי אמונה }\הגדרה{- ע״ע עקרים. }

\ערך{אמונה }\הגדרה{- }\משנה{רעיית האמונה }\הגדרה{- ע״ע אמונה, עבודת האמונה. }

\תערך{אמונה }\הגדרה{- }\תמשנה{מגמת האמונה הטהורה }\הגדרה{-}\תהגדרה{ השואת המעשה לתוכן המחשבה והבאת הרמוניה שלמה ביניהם }\תמקור{[מ״ר 493]. }

\ערך{אמונה }\הגדרה{- }\משנה{״אמונות אמיתיות״}\הגדרה{ - ע׳ במדור פסוקים ובטויי חז״ל, אמונות אמיתיות.}

\ערך{אמונה }\הגדרה{- }\משנה{״אמונות מוכרחות״ }\הגדרה{- ע׳ במדור פסוקים ובטויי חז״ל, אמונות מוכרחות.}

\ערך{״אמונה באמונה״ }\הגדרה{- ר׳ ״ללמוד אמונה באמונה״.}

\ערך{אמונה בברית\mycircle{°}}\הגדרה{ - }\הגדרה{הכונניות\mycircle{°}}\הגדרה{ המונחות ביצירת הרגש, שכשיתפתח עפ״י מדתו יהיה תמיד מתאים אל התכונה הרוממה של }\הגדרה{הדעות\mycircle{°}}\הגדרה{ }\הגדרה{הטהורות\mycircle{°}}\הגדרה{ }\מקור{[ע״ר א שפה]}\צהגדרה{.}\\\הגדרה{ע׳ במדור פסוקים ובטויי חז״ל, זכירת הברית. }

\ערך{אמונה עליונה }\הגדרה{- ע׳ במדור פסוקים ובטויי חז״ל, אמונות אמיתיות. }

\ערך{אמונה שלמה}\הגדרה{ - אמונה ביכולת ד׳ ובבריאת העולם יש מאין }\מקור{[פנק׳ א תי]}\צהגדרה{.}

\ערך{אמונה תחתונה }\הגדרה{- ע׳ במדור פסוקים ובטויי חז״ל, אמונות מוכרחות. }

\משנה{אמונת עתנו }\צהגדרה{- האמונה המתגלה מתוך כל מקוריותה ושלמותה של רציפות הדורות, בכל תקפה בממשות ההופעה האלהית של עתנו זאת }\צמקור{[ל״י ג קיז (מהדורת בית אל תשס״ג, ב תז)].}\\\צהגדרה{ראיית }\צהגדרה{יד-ד׳\mycircle{°}}\צהגדרה{ }\צהגדרה{אלהי-ישראל\mycircle{°}}\צהגדרה{ קורא דורותינו אלה ומכונן פעליהם, המתגלה בהם, בחוסן ישועותינו, בתקומת עמו ונחלתו וחזרת שכינתו, בקיבוץ נדחיו וכינוסם ובבנין בית חייו }\צמקור{[ל״י א ריח-ט]. }

\ערך{אמירה }\הגדרה{- ההופעה המלולית, הבאה מסגולת כח הביטוי שבאדם, המתחלת מראשית הציור המחשבתי, שהוא מצויר אצל בעל המבטא. ומפני שראשית היסוד של צמיחת ההבטאה באה מתוך }\הגדרה{הרעיון\mycircle{°}}\הגדרה{, נקרא זה בשם }\משנה{אמירה}\הגדרה{; והמקור הוא אמור, כמו ראש אמיר, כלומר התגלות הדבור ביחסו להמדבר בעצמו, טרם שבא להתגלם בצורה המשפעת כבר על השומע }\מקור{[עפ״י ע״ר ב נד]}\צהגדרה{. }\\\הגדרה{ע״ע אֹמר. ע״ע דבור, ע״ע קול. ע׳ בנספחות, מדור מחקרים, ״אֹמר״ לעומת ״דבור״. }

\ערך{אמיתי }\הגדרה{- }\הגדרה{פנימי\mycircle{°}}\הגדרה{ }\מקור{[א״א 82]}\צהגדרה{. }\\\הגדרה{}\הגדרה{כללי\mycircle{°}}\הגדרה{ }\מקור{[ע״א ג ב קסו]}\צהגדרה{. }\\\הגדרה{ע״ע אמת, באמת. ע״ע אמת, להיות חפץ באמת. }

\ערך{אמיתיות מקובלות}\הגדרה{ - דברים רבים, שלבד אמתתם ההגיונית הנשענת בדרך כללי מבירורה של }\הגדרה{תורה\mycircle{°}}\הגדרה{ הגלוי והמבורר, ״אתה הראת לדעת כי ד׳ הוא האלהים׳״, עוד הם נתמכים ביסוד ההכרה הפנימית הכללית שבאומה}\myfootnote{ \textbf{ההכרה הפנימית הכללית }\textbf{שבאומה} - ע״ע במדור תורה, תורה שבעל פה, יסודה של תושבע״פ השמירה של קבלת אבותינו מה שהתנחלו באומה, ובהערה שם.\label{32}}\הגדרה{, שהעבר וההוה שלה מרוכס ומקושר בקשורים אמיצים גלויים ומוחשים בהכרה גלויה ואמתית לכל מי שהולך בדרכיה הטבעיים לה }\צהגדרה{[}\צהגדרה{ע}\צהגדרה{״א ג ב קז].}\\\הגדרה{ע׳ במדור פסוקים ובטויי חז״ל, הגוי כולו, קבלת ״הגוי כולו״. וע׳ בנספחות, מדור מחקרים, ודאות באמיתותה של התורה.}

\ערך{אֹֽמֶר}\הגדרה{ - הענף העליון של הבטוי. הבטוי הפנימי מצד המבטא }\מקור{[ע״ר ב נד]}\צהגדרה{.}\\\הגדרה{ע״ע אמירה. ע״ע דבור. ע״ע קול.}

\ערך{אֹֽמֶר }\הגדרה{- }\משנה{אֹמֶר ההויה כולה }\הגדרה{- רוחה הפנימי של ההויה בסודה הכלול בקרבה, באותו המצב הראוי להתגלות ליצורים בעלי שכל והרגשה, המכירים כבר ציורים שכלים }\מקור{[ע״ר ב נד]}\צהגדרה{. }\\\הגדרה{ע״ע קול, קול ההויה כולה. ע״ע דבור, דבור ההויה הכללית. }

\ערך{אמת }\הגדרה{- השלמות המוחלטת של האלהות }\מקור{[ע״א ד ח לה]}\צהגדרה{. }\\\משנה{אור האמת }\הגדרה{- ברק }\הגדרה{העצם\mycircle{°}}\הגדרה{ }\הגדרה{ממקור-המקורות\mycircle{°}}\הגדרה{ }\הגדרה{עדי-עד\mycircle{°}}\הגדרה{ }\מקור{[ע״ה קנה]}\צהגדרה{. }\\\משנה{האמת כשהיא לעצמה }\הגדרה{- }\הגדרה{העליוניות\mycircle{°}}\הגדרה{ המוחלטה }\צהגדרה{[}\צהגדרה{א}\צהגדרה{״ק ג ל].}\\\ערך{אמת }\הגדרה{- }\משנה{״אמת ד׳ לעולם״ }\הגדרה{- ההנהגה }\הגדרה{העליונה\mycircle{°}}\הגדרה{, }\הגדרה{שבמדת-הדין\mycircle{°}}\הגדרה{, קשר }\הגדרה{הקדש-העליון\mycircle{°}}\הגדרה{ }\מקור{[עפ״י ע״ר א ריג]}\צהגדרה{. }\\\הגדרה{מדת הדין כמו שהיה קודם }\הגדרה{בריאת\mycircle{°}}\הגדרה{ העולם }\מקור{[מא״ה א קלו]}\צהגדרה{. }\\\צמשנה{אור האמת }\צהגדרה{- }\צהגדרה{מדת-הדין-העליונה\mycircle{°}}\צהגדרה{ }\צמקור{[ע״ר ב תפח]. }\\\משנה{אמת }\הגדרה{- }\הגדרה{אור-ד׳\mycircle{°}}\הגדרה{, מחולל כל }\מקור{[א״ק א ג (מ״ר 402)]}\צהגדרה{. }\\\הגדרה{היסוד של החיים השלמים, החיים המקיפים את כל ומלאים את הכל הנובעים מאור }\הגדרה{צור-העולמים\mycircle{°}}\הגדרה{, }\הגדרה{הכל-יוכל-וכוללם-יחד\mycircle{°}}\הגדרה{ }\מקור{[ע״א ד יב לב]}\צהגדרה{. }\\\משנה{האמת שלמעלה מכל גבולים }\הגדרה{- }\הגדרה{התענוג\mycircle{°}}\הגדרה{ }\הגדרה{האצילי\mycircle{°}}\הגדרה{ שלמעלה מכל פגמים, }\הגדרה{העז\mycircle{°}}\הגדרה{ של }\הגדרה{העדן\mycircle{°}}\הגדרה{ }\הגדרה{שכולו-אומר-כבוד\mycircle{°}}\הגדרה{ }\מקור{[עפ״י א״א 77]}\צהגדרה{. }\\\משנה{גבורת האמת }\הגדרה{- מציאות חיה וקיימת, עליונה, }\הגדרה{בחביון-העוז\mycircle{°}}\הגדרה{ האלהי, באופן יותר עשיר, יותר קיים ויותר אמיץ בהוייתו, ויותר נעלה בהופעת רוממות אצילות שלמותו, מכל מה שכל רעיון יוכל לצייר ולתפוס }\מקור{[ע״ט נ]}\צהגדרה{. }\\\הגדרה{}\הגדרה{סוד\mycircle{°}}\הגדרה{ המציאות העצמית, שלמעלה }\הגדרה{מהזיו\mycircle{°}}\הגדרה{ }\הגדרה{האידיאלי\mycircle{°}}\הגדרה{ דלגבי דידן }\מקור{[שם שם]}\צהגדרה{. }\\\משנה{זוהר האמת }\הגדרה{- ההוד של }\הגדרה{אור-החיים\mycircle{°}}\הגדרה{ שבמקור }\הגדרה{הקודש\mycircle{°}}\הגדרה{ }\מקור{[א״ק א ג (מ״ר 402)]}\צהגדרה{. }\\\משנה{אמת אלהית }\הגדרה{- החפץ המרומם והנשגב של }\הגדרה{המגמה\mycircle{°}}\הגדרה{ האידיאלית אשר בעליונות הקודש }\מקור{[א׳ יא]}\צהגדרה{. }\\\משנה{האמת העליונה }\הגדרה{- }\הגדרה{אור\mycircle{°}}\הגדרה{ }\הגדרה{הטוהר\mycircle{°}}\הגדרה{ של שיגוב }\הגדרה{החכמה\mycircle{°}}\הגדרה{ השוכן ברום }\הגדרה{חוסן\mycircle{°}}\הגדרה{ }\הגדרה{אמונת\mycircle{°}}\הגדרה{ }\הגדרה{ישראל\mycircle{°}}\הגדרה{ }\מקור{[עפ״י א״ק ב רפה]}\צהגדרה{. }\\\משנה{האמת העליונה }\הגדרה{- }\הגדרה{האמת-האלהית\mycircle{°}}\הגדרה{, שהיא נתונה מיד רבון כל המעשים ע״פ הסדר של אמיתת המציאות, ע״פ המהות העליון שלה }\מקור{[ע״ר א נד]}\צהגדרה{. }

\ערך{אמת }\הגדרה{- }\משנה{מדת האמת }\הגדרה{- הדין בלא צדקה }\מקור{[מא״ה קסד]}\צהגדרה{. }

\ערך{אמת }\הגדרה{- היסוד ההגיוני המופשט והקר, בעל המופתים והמשפטים המחייבים}\צהגדרה{ [}\צהגדרה{א}\צהגדרה{״ה (מהדורת תשס״ב) ב }\צמקור{81}\צהגדרה{ (א״ב ג)].}

\ערך{אמת }\הגדרה{- }\משנה{(לעומת שלום\mycircle{°}) }\הגדרה{- שלמות הקיום של כל נמצא מצד עצמו ופרטיותו }\מקור{[עפ״י מ״ש קכ (מא״ה ב יד)]}\צהגדרה{. }\\\משנה{כח האמת האלהית }\הגדרה{- מקור }\הגדרה{הקדושה\mycircle{°}}\הגדרה{ הכללית של }\הגדרה{ישראל\mycircle{°}}\הגדרה{, (ה)נותן כחו בהם לכל יחיד פרטי, להתקדש בכח עצמי ולהעשות בזה מוכן להוסיף קדושה מיוחדת מצדו (לאיזה נושא או ענין) }\מקור{[עפ״י ע״ר ב רנה]}\צהגדרה{. }

\ערך{אמת }\הגדרה{- היסוד הנצחי המוציא את דבר }\הגדרה{המשפט\mycircle{°}}\הגדרה{ בקו }\הגדרה{הצדק\mycircle{°}}\הגדרה{ הגמור }\מקור{[עפ״י ע״א ד יב לח]}\צהגדרה{. }\\\משנה{האמת הגדולה }\הגדרה{- מקור משפטי ד׳ }\מקור{[ע״ר ב ס]}\צהגדרה{. }\\\משנה{אור האמת }\הגדרה{- נשמת }\הגדרה{הצדק\mycircle{°}}\הגדרה{ }\הגדרה{העליון\mycircle{°}}\הגדרה{, אשר במשפטי ד׳ }\הגדרה{הישרים\mycircle{°}}\הגדרה{ }\מקור{[שם שם]}\צהגדרה{. }\\\משנה{האמת האלהית העליונה}\הגדרה{ - מגמת כל חמדת עולמים, הגנוזה במשפטי ד׳}\מקור{ [עפ״י ע״ר ב נט]}\צהגדרה{.}

\ערך{אמת }\הגדרה{- }\מעוין{◊}\הגדרה{ דבר נצחי ומתקיים}\myfootnote{ \textbf{אמת}\textbf{ דבר נצחי ומתקיים} - ע׳ מגן וצינה דף י.\label{33}}\הגדרה{ }\מקור{[ע״ר א רכז]}\צהגדרה{. }\\\הגדרה{הנצח }\מקור{[מ״ר 159]}\צהגדרה{. }\\\הגדרה{ע״ע שקר. ע״ע כזב. ע׳ בנספחות, מדור מחקרים, צדק ואמת. ע״ע חסד, אמת (משפט), ורחמים. }

\ערך{אמת }\הגדרה{- }\משנה{גאולת האמת שלמעלה מכל גבולים }\הגדרה{- }\הגדרה{האידיאליות\mycircle{°}}\הגדרה{ }\הגדרה{בתענוג\mycircle{°}}\הגדרה{ }\הגדרה{האצילי\mycircle{°}}\הגדרה{ שלמעלה מכל פגמים, }\הגדרה{העז\mycircle{°}}\הגדרה{ של }\הגדרה{העדן\mycircle{°}}\הגדרה{ }\הגדרה{שכולו-אומר-כבוד\mycircle{°}}\הגדרה{ }\מקור{[עפ״י א״א 77]}\צהגדרה{. }

\ערך{אמת }\הגדרה{- }\משנה{באמת }\הגדרה{- }\משנה{(גישה אמיתית בשיפוט) }\הגדרה{- בלא נטיה של חפץ להטיב לשום צד }\מקור{[אג׳ א קנט]}\צהגדרה{. }

\ערך{אמת - }\משנה{״לכל אשר יקראוהו באמת״ }\הגדרה{- התכלית האמיתי, של עיקר קיום החיים <ולא הבלי עוה״ז הכלים> }\מקור{[עפ״י ע״ר א רכז, מא״ה, ענייני תפילה, שד]}\צהגדרה{. }

\ערך{אמת }\הגדרה{- }\משנה{להיות חפץ באמת }\הגדרה{- להיות חי ופועל לפי הדיעות היותר אמיתיות וההרגשות היותר }\הגדרה{קדושות\mycircle{°}}\הגדרה{ }\הגדרה{לטוב\mycircle{°}}\הגדרה{ }\הגדרה{ולחסד\mycircle{°}}\הגדרה{, כמעשה גדולי העולם אשר נגשו אל ד׳ במעשיהם הבהירים למלא את העולם חסד ואמת }\מקור{[ע״א ג ב קפד]}\צהגדרה{. }\\\הגדרה{השאיפה }\הגדרה{לציורי\mycircle{°}}\הגדרה{ המושכלות מצד עצמם ונצחיותם }\מקור{[א׳ לה]}\צהגדרה{.}\\\הגדרה{ע״ע אמיתי. }

\ערך{אמת }\הגדרה{- }\משנה{חיי אמת }\הגדרה{- ע״ע חיים, חיי אמת. }

\ערך{אן }\הגדרה{- הוראת השאלה ביחס המקום של איזה מבוקש }\מקור{[ר״מ קכד]}\צהגדרה{.}\\\ערך{אן }\הגדרה{- }\משנה{בצורה רוחנית }\הגדרה{- דרישת המטרה התכליתית ממחזה כללי המופיע בהמון פרקים, בתבנית ברקי אורות ונצוצי חיים מבריקים, בצורה זעירה ומעולמת. והשאלה חודרת היא, אן מונחת היא המטרה המרכזית של כל המון בריות הללו שהם בלי תכלית }\מקור{[שם]}\צהגדרה{.}

\ערך{״אנוֹש״ }\הגדרה{- יאמר (על האדם) ע״ש הכח היותר חלוש }\הגדרה{שבנפש\mycircle{°}}\הגדרה{ והוא הרצון הסתמי בלא טעם דעת ובחירה ושכל, רק רצון לבד ונטי׳ דומה ממש לרצון כל בע״ח למיניהם }\מקור{[עפ״י פ״א קעז, קעה]}\צהגדרה{.}\\\הגדרה{ע״ע ״אדם״.}

\ערך{אני }\הגדרה{- עצמיותי האנושית }\צהגדרה{[}\צהגדרה{ע}\צהגדרה{״ר א מה].}

\ערך{אנרכיא }\הגדרה{- }\משנה{אנארכיזם הגשמי האינדיבידואלי }\הגדרה{- אהבה עצמית רבה וגדולה }\מקור{[אג׳ א קעד, קעה]}\צהגדרה{.}

\ערך{אנשים }\הגדרה{- }\משנה{(לעומת נשים\mycircle{°}) }\הגדרה{- הכח הפועל בעולם (בחברה) }\מקור{[ע״א ד ו כז]}\צהגדרה{.}\\\הגדרה{ע״ע איש.}

\ערך{אסתטי }\הגדרה{- }\משנה{החוש האסתטי }\הגדרה{- הרגש של }\הגדרה{היופי\mycircle{°}}\הגדרה{ וההידור }\מקור{[ע״א ג ב סז]}\צהגדרה{. }

\ערך{אף }\הגדרה{- הוראה לדבר נטפל שאינו עומד לעצמו, כ״א הוא מצטרף וטפל לדברים אחרים, גדולים ועקרים יותר ממנו }\מקור{[ר״מ קכד]}\צהגדרה{. }

\ערך{אף }\הגדרה{- ע׳ במדור נפשיות.}\\\הגדרה{ }\\\ערך{אף }\הגדרה{- }\משנה{(אלקי) }\הגדרה{- ע׳ במדור תיאורים אלהיים. }

\ערך{אף }\הגדרה{- }\משנה{(בתאור הפנים) }\הגדרה{- ע׳ במדור גוף האדם אבריו ותנועותיו.}

\ערך{אף }\הגדרה{- ע׳ במדור גוף האדם אבריו ותנועותיו.}

\תערך{אפיקורסות }\הגדרה{- }\תמשנה{תכן האפיקורסות }\הגדרה{-}\תהגדרה{ הסתלקות האדם מחבור של }\תהגדרה{קדושה\mycircle{°}}\תהגדרה{, מהתקשרות אלהית. המחשבה הגרועה, של הסתלקות מוחלטת מאלהות }\תמקור{[מ״ר 493]. }\\\צהגדרה{מהלך מחשבה. מהלך רוח חומרני, מטריאליסטי, המנותק }\צהגדרה{מעוה״ב\mycircle{°}}\צהגדרה{, מנותק מקשר עם הנצח }\צמקור{[שי׳ ת״ת 138].}\\\הגדרה{ע״ע כפירה (שלילת האמונה). ע״ע שלילה. ר׳ במדור מדרגות והערכות אישיותיות, אפיקורס. ושם, כופר. ע׳ במדור פסוקים ובטויי חז״ל, העושה תורתו עיתים הרי זה מיפר (תורה) [ברית]. }

\ערך{אץ }\הגדרה{- }\מעוין{◊}\הגדרה{ מורה מהירות }\מקור{[ר״מ קכה]}\צהגדרה{. }

\ערך{אץ }\הגדרה{- לחיצה ודחיפה, מקושר עם צרות ביחש }\הגדרה{המקום\mycircle{°}}\הגדרה{ }\מקור{[ר״מ קכה]}\צהגדרה{. }

\ערך{אצילות }\הגדרה{- }\משנה{אצילות רצונית }\הגדרה{- }\הגדרה{המוסריות\mycircle{°}}\הגדרה{ העליונה המתגלה ע״י }\הגדרה{קדושה\mycircle{°}}\הגדרה{ }\הגדרה{וחסידות\mycircle{°}}\הגדרה{ }\הגדרה{טהורה\mycircle{°}}\הגדרה{ ועליונה }\מקור{[ע״ט קכ]}\צהגדרה{. }\\\הגדרה{הרצון התמים והבהיר של האדם התופס את קצה }\הגדרה{זיוה\mycircle{°}}\הגדרה{ של }\הגדרה{האצילות-האלהית\mycircle{°}}\הגדרה{ }\מקור{[עפ״י א״ק ב שמט]}\צהגדרה{. }

\ערך{אר }\הגדרה{- (מורה) קללה ומארה }\מקור{[ר״מ קכו]}\צהגדרה{. }

\ערך{אר }\הגדרה{- הוראת אור, }\הגדרה{הארה\mycircle{°}}\הגדרה{ }\הגדרה{וזריחה\mycircle{°}}\הגדרה{ }\מקור{[ר״מ קכו]}\צהגדרה{. }

\ערך{אר }\הגדרה{- (מורה) לקיטה ותלישת פירות }\מקור{[ר״מ קכו]}\צהגדרה{. }

\ערך{ארוכה }\הגדרה{- רפוי מתמיד וממושך למחלות כרוניות, מתוך קלקולים מתמידים }\מקור{[עפ״י מ״ר 473]}\צהגדרה{. }\\\הגדרה{הרפואה המדרגת המשיבה את הכחות שנתרופפו }\מקור{[שם 371]}\צהגדרה{. }\\\הגדרה{רפואה טבעית פנימית, לתקן הטבע, והוא לשון תקון כמו ״אריך לנא למחזא״}\myfootnote{ \textbf{אריך}\textbf{ לנא למחזא} - עזרא ד יד.\label{34}}\הגדרה{. דרושה למחלות פנימיות. שבחה של דרך רפואה זו הוא להיות קרוב למזון יותר מלרפואה, כדי לחזק את טבע הגוף בעצמו מבלי להוסיף כח זר על הכח הטבעי }\מקור{[עפ״י משפט כהן, פתיחה טו]}\צהגדרה{. }\\\הגדרה{ע״ע רפואה. ע״ע חולי. ע״ע מיחוש.}

\ערך{ארוסין }\הגדרה{- היסוד החוקי }\הגדרה{האצילי\mycircle{°}}\הגדרה{ }\הגדרה{שבדבקות\mycircle{°}}\הגדרה{ (שבין בני הזוג), שמתוך מעלתו אין בו התפסה לחקוי }\הגדרה{חמרי\mycircle{°}}\הגדרה{ כלל }\מקור{[עפ״י ע״ר א לה]}\צהגדרה{. }

\ערך{ארץ }\הגדרה{- }\משנה{הארץ בכלל }\הגדרה{- העולם בכלל }\מקור{[ע״א ד ו מו]}\צהגדרה{. }\\\הגדרה{גמר כל תכליתם של }\הגדרה{הסבות\mycircle{°}}\הגדרה{ הראשיות [של כל }\הגדרה{מגמה\mycircle{°}}\הגדרה{ ותכלית] המסבבות כל המון המעשים }\מקור{[עפ״י ע״ר א רמה, ע״א א ב ד]}\צהגדרה{. }\\\ערך{ארץ}\צהגדרה{ - המציאות}\צמקור{ [פנ׳ קלג]}\צהגדרה{.}\\\הגדרה{המציאות המעשית }\מקור{[ע״ר א שכה]}\צהגדרה{.}\\\הגדרה{גשמותה של המציאות }\מקור{[ע״א ב ט ל]}\צהגדרה{. }\\\הגדרה{כלל כל }\הגדרה{העולמים\mycircle{°}}\הגדרה{ }\הגדרה{החמריים\mycircle{°}}\הגדרה{ כולם }\מקור{[עפ״י ע״ר א קטו]}\צהגדרה{. }\\\הגדרה{העולם החומרי }\מקור{[ע״ר ב פ]}\צהגדרה{.}\\\הגדרה{נבכי החומר ובמעמקי מסילותיו המסובכות }\מקור{[ע״א ד ט קד]}\צהגדרה{.}\\\הגדרה{השטח התחתיתי }\מקור{[עפ״י ע״ר ב סז]}\צהגדרה{.}\\\הגדרה{כחות החמריים, מחשבות האדם, סדרי החיים והחברה, וכל הנוגע לכל תהומות, עד מעמקי שפל }\מקור{[קובץ ה סח]}\צהגדרה{. }\\\הגדרה{הענינים }\הגדרה{החומריים\mycircle{°}}\הגדרה{ המדיניים האקונומיים }\מקור{[ע״א ג א לג]}\צהגדרה{. }\\\הגדרה{ע״ע שמים. ע׳ במדור מונחי קבלה ונסתר, אחרית. ע׳ במדור מונחי קבלה ונסתר, ברתא. }

\ערך{ארץ }\הגדרה{- }\משנה{תכונה ארצית}\הגדרה{ - ברכת הטבע וכל כחותיו }\מקור{[ע״ר א רט]}\צהגדרה{. }\\\צהגדרה{ }\\\משנה{ארץ}\צהגדרה{ - }\מעוין{◊}\צהגדרה{ מכון הטבע של }\צהגדרה{האומה\mycircle{°}}\צהגדרה{ }\צמקור{[צ״צ קא]}\צהגדרה{.}\\\צהגדרה{ר׳ שפה.}

\משנה{ארץ}\צהגדרה{ - }\צמשנה{קדושת הארץ}\הגדרה{ - }\צהגדרה{קדושת }\צהגדרה{הכלל\mycircle{°}}\צהגדרה{ }\צמקור{[שי׳ ה 212]. }\\\צהגדרה{כלליות הקדושה של כל הקדושות }\צמקור{[רצי״ה ג״ר 133].}

\ערך{ארץ אשור }\הגדרה{- (}\משנה{׳האובדן בארץ אשור׳}\הגדרה{)}\myfootnote{ ישעיה כז יג.\label{35}}\הגדרה{ - <אשור - לשון הבטה והשקפה> האובדן בארץ בדעות רעות ורוח שטות המביא לידי עבירה. גלות טעות השכל והתרופפות האמונה (בהשגחה וחסרון בטחון וכיו״ב, או זלזול הורים ומורים מפני מיעוט יקרת התורה בנפשו), מפני טרדות עוה״ז והבליו המשכחים את האמת }\מקור{[עפ״י מ״ש סו-ח]}\צהגדרה{.}\\\הגדרה{ע״ע ארץ מצרים. ע׳ במדור מקומות וארצות, אשור. ע׳ במדור פסוקים ובטויי חז״ל, האבדים בארץ אשור והנדחים בארץ מצרים. }

\ערך{ארץ הרוחנית }\הגדרה{- }\הגדרה{הציורים\mycircle{°}}\הגדרה{ ההולכים עם ההסברים של הידיעות }\הגדרה{האלהיות\mycircle{°}}\הגדרה{, רפידת המחשבה }\מקור{[פנ׳ קלו]}\צהגדרה{. }

\ערך{ארץ מצרים\mycircle{°} }\הגדרה{- }\משנה{(׳הנדחות בארץ מצרים׳)}\footref{35} \הגדרה{- גלות תאוות עוה״ז, <כי מצרים היא ערות הארץ ומקור התאוות והחומריות כולן> }\מקור{[מ״ש סו]}\צהגדרה{.}\\\הגדרה{ע״ע ארץ אשור. ע׳ במדור מקומות וארצות, מצרים. ע׳ במדור פסוקים ובטויי חז״ל, האבדים בארץ אשור והנדחים בארץ מצרים. }

\ערך{אש }\הגדרה{- החומר השורף והמכלה, המחמם והמאיר, העושה את הפעולות ההפכיות בתכונתן, הכל לפי ערכם של מקבלי המפעלים }\מקור{[ר״מ קכז]}\צהגדרה{. }\\\משנה{כח האש וחומו  }\הגדרה{- הפועל הגורם להמפעלים שיעשו }\מקור{[ע״ר א קכט]}\צהגדרה{. }

\ערך{אִשָּׁה }\הגדרה{-}\משנה{ יסוד השלמתה }\הגדרה{- עדינות }\הגדרה{הרגש\mycircle{°}}\הגדרה{ }\הגדרה{הטהור\mycircle{°}}\הגדרה{ }\הגדרה{והטוב\mycircle{°}}\הגדרה{, }\הגדרה{והשכל\mycircle{°}}\הגדרה{ יעזר על ידו כפי המדה האפשרית }\מקור{[ע״א ג ב ריג]}\צהגדרה{.}\\\הגדרה{ע״ע איש. ע׳ במדור הכרה והשכלה והפכן, בינה יתירה (באשה). ע״ע גברת.}

\ערך{את }\הגדרה{- מלת הצירוף, הוראת הטפלה, הערכת }\הגדרה{התוספת\mycircle{°}}\הגדרה{ }\מקור{[ר״מ קכז]}\צהגדרה{. }

\ערך{את }\הגדרה{- הכלי היסודי לעבודת האדמה להוציא מחיה מן הארץ }\מקור{[ר״מ קכח]}\צהגדרה{. }

\mylettertitle{ב}
\ערך{בא }\הגדרה{- הוראת התכנסות הנושא אל המקום הראוי ע״י תנועה מוקדמת }\מקור{[ר״מ קכח]}\צהגדרה{.}

\ערך{באור }\הגדרה{- יחשו של כל מאמר בודד, לא רק לפי ערכו והדבר המבוצר בתוכו בלבד, כ״א עפ״י ערך כל אותן ההשפעות שאפשר לו להשפיע, לכשיתבאר, כאשר ״מעין ישיתוהו״ על עולם הרעיונות ההולכים בדרך ישרה, הוא פתוח בדרך מפולש לעולם הגדול של ההשכלות מלאות }\הגדרה{זיו\mycircle{°}}\הגדרה{, ומעורר בדרך פתחו להכניס אל תוכו ועל ידו תלי תלים של ידיעות והרחבות שכליות, שנותנות אומץ וגבורה לנפשות ההוגות בהם. ״מ״ם פתוחה }\משנה{-}\הגדרה{ }\הגדרה{מאמר-פתוח\mycircle{°}}\הגדרה{״ }\מקור{[ע״א א, הקדמה, יד-טו]}\צהגדרה{.}\\\הגדרה{ע״ע פרוש. ע״ע דרש.}

\ערך{באור }\הגדרה{- }\משנה{דרך הביאור }\הגדרה{- הדרישה הבנויה ע״פ ערכי הכללים, שהם דומים לדרישת כל רעיון לא רק מצד עצמו, כ״א מצד הרעיונות שמטבעו להוליד ע״פ דרך ישרה, כן דרישת התורה שע״פ הכללים, אין הפרטים נולדים ומסתעפים זה מזה, כ״א כולם יחד יוצאים הם מהכללים הראשיים יסודי ועקרי התורה וסתרי טעמיה הגדולים }\מקור{[ע״א א, הקדמה, יז]}\צהגדרה{.}\\\הגדרה{ע׳ במדור תורה, דרישת התורה בדרך הכהן. }

\ערך{בג }\הגדרה{- מזון }\מקור{[ר״מ קכח]}\צהגדרה{.}

\ערך{בד }\הגדרה{- מגזרת בדד, ההתבודדות הפרטית }\מקור{[ר״מ קכט]}\צהגדרה{.}\\\צהגדרה{ }\\\ערך{בד }\הגדרה{- }\הגדרה{שקר\mycircle{°}}\הגדרה{ }\מקור{[ר״מ קכט]}\צהגדרה{.}

\ערך{בד }\הגדרה{- ענף של אילן }\מקור{[ר״מ קכט]}\צהגדרה{.}

\ערך{בד }\הגדרה{- מוט }\מקור{[ר״מ קכט]}\צהגדרה{.}

\ערך{בד }\הגדרה{- ע״ע בוץ.}

\ערך{בה }\הגדרה{- תואר הוראה של יחס פנימי, למושג הנקבי }\מקור{[ר״מ קל]}\צהגדרה{.}

\ערך{בהלה }\הגדרה{- טירוף של רצונות ומחשבות שכל אחת דוחה את חבירתה עד אפס מקום של חשבון נקי }\מקור{[ע״א ג ב רכג]}\צהגדרה{.}\\\הגדרה{נפילת ערך וחרדה}\מקור{ [ע״ר ב עא]}\צהגדרה{.}

\ערך{בו }\הגדרה{- מתאר את היחש הפנימי במושג הזכרי }\מקור{[ר״מ קלא]}\צהגדרה{.}

\ערך{בוץ }\הגדרה{- }\הגדרה{פשתן\mycircle{°}}\הגדרה{, <הנקרא ג״כ בד על שם שהוא עולה בד בבד> מורה על שמירת הגבולים, על כח צומח פורה, שעם זה הוא שומר חק וגבול, ואינו מתערב בחלקי חיים ותוכן של נושא אחר, שומר את המצר, ומגין על }\הגדרה{הצדק\mycircle{°}}\הגדרה{ }\מקור{[ר״מ קלו]}\צהגדרה{.}

\ערך{בושה }\הגדרה{- }\משנה{אמיתתה במקורה }\הגדרה{- }\הגדרה{יראה-עליונה\mycircle{°}}\הגדרה{, }\הגדרה{יראת-ד׳\mycircle{°}}\הגדרה{ }\מקור{[עפ״י ר״מ קמ, א״ש יד כד]}\צהגדרה{. }

\ערך{בחירה }\הגדרה{- }\משנה{״בחר בנו מכל העמים״ }\הגדרה{- נתן לנו יתרון ומעלה }\הגדרה{ודבקות\mycircle{°}}\הגדרה{ בו ית׳ על כל עם ולשון }\מקור{[עפ״י מ״ש קמב (ה׳ קפד)]}\צהגדרה{.}\\\צהגדרה{יצר אותנו להיות לו לעם-נחלה, להתגלות }\צהגדרה{צלם-האלהים\mycircle{°}}\צהגדרה{ שבאדם בקרבנו בתור עם}\myfootnote{ ע׳ אור החיים בראשית א כז. ״״ויברא אלהים את האדם בצלמו״. כי ברא האדם בב׳ צלמים, הראשון צלם הניכר בכל אדם ואפילו בבני אדם הריקים מהקדושה אשר לא מבני ישראל המה, ועליהם אמר ״בצלמו״ פירוש: של הנברא; והב׳ הם בחינת המאושרים, עם ישראל נחלת שדי, כנגד אלו אמר ״בצלם אלהים בראו״, הרי זה בא ללמדנו כי יש בנבראים ב׳ צלמים צלם הניכר וצלם אלהים רוחני נעלם, והבן״.\label{36}}\צהגדרה{, באדם הצבורי המופיע בנו במלוא שיעור-קומתנו }\צמקור{[ל״י ג קיד-קטו (מהדורת בית אל תשס״ב ב תד)].}\\\צהגדרה{הופעת קדושת עצמיותנו הצבורית, מתוך }\צהגדרה{השראת-שכינתו\mycircle{°}}\צהגדרה{ על כלנו כאחד, בהוציאו אותנו מבית}\צהגדרה{-עבדים }\צהגדרה{ממצרים\mycircle{°}}\צהגדרה{, }\צהגדרה{ובקרבו-אותנו-לפני-הר-סיני\mycircle{°}}\צהגדרה{ }\צמקור{[ל״י א (מהדורת בית אל תשס״ב) רכו].}\\\משנה{בחירתנו מכל העמים }\צהגדרה{- יצירת מהותנו. הוית עצמותנו הצבורית בכל ממשותה ושכלולה, תקפה ותפארתה}\צמקור{ [עפ״י שם, רכז].}\\\הגדרה{ע״ע בחירת ישראל.}

\ערך{בחירה }\הגדרה{- }\משנה{(בעם ישראל, לעומת סגולה\mycircle{°}) }\הגדרה{- ההערכה הגלויה של }\הגדרה{קדושתם\mycircle{°}}\הגדרה{ של }\הגדרה{ישראל\mycircle{°}}\הגדרה{ }\מקור{[ע״ר ב פ]}\צהגדרה{.}\\\מעוין{◊ }\צמשנה{הבחירה}\צהגדרה{ הגלויה, מתבררת ע״י המדות הקדושות והמובחרות, שבהן }\צהגדרה{נתעטרו\mycircle{°}}\צהגדרה{ בני }\צהגדרה{יעקב\mycircle{°}}\צהגדרה{ בכללם, עד שהכל מכירים שהבחירה האלהית ראויה להם }\צמקור{[ע״ר א רב].}\\\הגדרה{ע׳ במדור מדתם ועניינם הרוחני של אישי התנ״ך, יעקב, מדת התאר יעקב (לעומת ישראל). ושם, ישראל, מדת התאר ישראל (לעומת יעקב). ע״ע ישראל לעומת ישורון. ע׳ במדור מונחי קבלה ונסתר, קוב״ה דרגא על דרגא סתים וגליא וכו׳. ע׳ במדור פסוקים ובטויי חז״ל, בית יעקב לעומת בני ישראל. ושם ממלכת כהנים וגוי קדוש. ושם בני בכורי לעומת בנים. ע׳ בנספחות, מדור מחקרים, ״בחרתי בכם ויחדתי שמי עליכם״ לעומת ״אני בכבודי מתהלך ביניכם״.}

\ערך{בחירה }\הגדרה{- }\משנה{בחירת ד׳ בכהנים\mycircle{°} }\הגדרה{- הטבעה טבעית רוחנית עליונה }\הגדרה{בסגולת\mycircle{°}}\הגדרה{ נפשותם}\צהגדרה{ [}\צהגדרה{ע}\צהגדרה{״ר א קסא].}\\\הגדרה{ע׳ במדור אישים, אהרן. ושם, ״אהרן ובניו״. ושם, ״בני אהרן״. ע״ע כהונה, קדושתה.}

\תערך{בחירה}\תהגדרה{ - חופש הפעולה של האדם }\תמקור{[נ״א ד 39].}

\ערך{בחירה גלויה }\הגדרה{- }\משנה{הבחירה הגלויה }\הגדרה{- }\הגדרה{הבחירה-החפשית\mycircle{°}}\הגדרה{ הנמצאת בפועל, המתגלמת בבני אדם, שהמשפט המורגש מתראה על ידה, שבחיים המתגלים לפנינו לעולם לא נמצא אותה במילואה }\מקור{[עפ״י אג׳ א שיט, א״ק ג לד]}\צהגדרה{. }\\\הגדרה{הבחירה שפרטי הכוחות והפרטיות הדקות שבמציאות ביחש לשכר ועונש נחלקים על פיה }\מקור{[עפ״י פנ׳ כג]}\צהגדרה{. }\\\מעוין{◊}\הגדרה{ עקר הבחירה וראשית יסודה הוא בבחינת }\הגדרה{הרוח\mycircle{°}}\הגדרה{, שהוא מדרגת האדם }\מקור{[ע״ר א רמט]}\צהגדרה{. }\\\הגדרה{ע״ע בחירה צפונה. ע״ע בחירה כמוסה. }

\ערך{בחירה  גנוזה }\הגדרה{- ע״ע בחירה כמוסה. }

\ערך{בחירה חפשית }\הגדרה{- }\הגדרה{החופש\mycircle{°}}\הגדרה{ הגמור, המחולל בקרבו }\הגדרה{רצון\mycircle{°}}\הגדרה{ שאין בו שום מועקה מבחוץ, שעל ידו מתגלה }\הגדרה{העצמות\mycircle{°}}\הגדרה{ של מהות החיים, המקבלים את התכנית של }\הגדרה{הגורל\mycircle{°}}\הגדרה{ }\הגדרה{הטוב\mycircle{°}}\הגדרה{ או }\הגדרה{הרע\mycircle{°}}\הגדרה{, שרק החלק הקטן (ממנה), המתגלה בתור בחירה מעשית }\הגדרה{(בחירה-גלויה\mycircle{°}}\הגדרה{), מתוה לפנינו בגלוי את ארחות הטוב והרע }\מקור{[עפ״י אג׳ א שיט]}\צהגדרה{. }\\\תהגדרה{חופש הבחירה, חופש הפעולה של האדם. חוקי }\תהגדרה{האפשר\mycircle{°}}\תהגדרה{, כח ואפשרות בחירת מעשים בזולת מעשים ותועלתם }\תמקור{[נ״א ד 39].}\\\הגדרה{ע״ע התגלות המהות.}

\ערך{בחירה כמוסה }\הגדרה{- הבחירה שאיננה על פי התוכן }\הגדרה{המוסרי\mycircle{°}}\הגדרה{ המתגלה, אלא על פי }\הגדרה{האידיאל\mycircle{°}}\הגדרה{ }\הגדרה{העליון\mycircle{°}}\הגדרה{, שעל פי }\הגדרה{הצפיה-העליונה\mycircle{°}}\הגדרה{, למעלה מהתנאים שההויה נמצאת בהם כעת. }\הגדרה{ההזרחות\mycircle{°}}\הגדרה{ שבאות מתוכן זה הם }\הגדרה{אורות\mycircle{°}}\הגדרה{ }\הגדרה{הנשמה\mycircle{°}}\הגדרה{ }\הגדרה{הפנימית\mycircle{°}}\הגדרה{ של כל היש, והן כוללות את העבר ההוה והעתיד, למעלה מסדר זמנים וצורתם, וכל זה כלול }\הגדרה{בשם-ההויה\mycircle{°}}\הגדרה{, כסדרו ובכל אופני צירופיו }\מקור{[א״ק ג כג (ע״ט ב)]}\צהגדרה{. }\\\הגדרה{הבחירה שכל מערכת }\הגדרה{המשפט\mycircle{°}}\הגדרה{ של כל היש מתנהגת על ידה }\מקור{[א״ק ג לד]}\צהגדרה{.}\\\הגדרה{הבחירה שפרטי הכוחות והפרטיות הדקות שבמציאות בכלל (להוציא מדרגות ה׳שכר ועונש׳) נחלקים למדריגותיהם על פיה ע״פ היסוד ד״הכל }\הגדרה{צפוי״\mycircle{°}}\הגדרה{ }\מקור{[עפ״י פנ׳ כג]}\צהגדרה{. }\\\משנה{בחירה צפונה }\הגדרה{- יסוד כל חק }\הגדרה{ומשפט\mycircle{°}}\הגדרה{. הבחירה השמה את המערכות לפי מדרגותיהן, מגדולי המציאות עד קטניהם }\מקור{[עפ״י א״ק ג לד]}\צהגדרה{. }\\\הגדרה{גלגול (הזכות או החובה), האמצעות של המסבבים את הדברים הרשמיים בהם נעוץ כח החפץ הגמור, הנכלל בכחות המציאות שלא לחסר ממנה את מושגי המוסר והרשע והצדק וכל העלילות הגדולות המסובבות מהם ועל ידם, באין שום גרעון, ״כי כל אשר יעשה האלהים הוא יהיה לעולם עליו אין להוסיף וממנו אין לגרוע והאלהים עשה שיראו }\הגדרה{מלפניו״\mycircle{°}}\הגדרה{. וההכרה המחברת את מושג המשפט הקבוע, עם מושג }\הגדרה{החופש\mycircle{°}}\הגדרה{, להתאימם עם }\הגדרה{העז\mycircle{°}}\הגדרה{ והמשפט המלא את כל היקום, כי ״אלהים שופט צדיק״, היא }\משנה{הבחירה הכמוסה}\הגדרה{ הגלויה רק ליוצר כל במקור }\הגדרה{החכמה-האלהית\mycircle{°}}\הגדרה{ }\מקור{[ע״א ג ב רד]}\צהגדרה{. }\\\משנה{הבחירה הגנוזה}\myfootnote{ ע״ע הערת הרצי״ה אג׳ א עמ׳ שפה-שפו, לעמ׳ שיט.\label{37}}\הגדרה{ - }\הגדרה{הבחירה-החפשית\mycircle{°}}\הגדרה{ הגמורה, שהיא עצם המהותיות שלנו, המיטב והעיקרי שבהויתנו, המתגלה בכל מלואה ועשרה רק לצפיה-העליונה }\מקור{[עפ״י אג׳ א שיט, שם ב מב, ע״ר ב קנז]}\צהגדרה{.}\\\הגדרה{ע׳ במדור מונחי קבלה ונסתר, יובל, עלמא דיובלא.  ע׳ בנספחות, מדור מחקרים, ידיעה ובחירה.}

\ערך{בחירת ישראל }\הגדרה{-}\משנה{ מטרת בחירת\mycircle{°} ישראל על פי ד׳ בהתגלות אלהות על ידי אותות ומופתים גלויים}\הגדרה{ - כדי שיהיו מוכנים }\הגדרה{לקרבת-אלהים\mycircle{°}}\הגדרה{ היותר נעלה, שהיא יסוד העליון והתכליתי למין האנושי, ושמשפע מוסרם בצירוף הכח האלהי שכבר נגבל בהיסתוריה הברורה שלהם ובארץ הקודש, לכשתצא מן הכח אל הפועל גדולתם ותפארתם כמו שראוי להיות לעם הנושא את היסוד היותר מעולה וכולל לכל המין האנושי בידו, שהוא יסוד הרוחניות של הרחבת דעת השם בחיים האנושיים, אז מאיליה תצא הפעולה לכל העולם ברב הוד והדר }\מקור{[ל״ה 168]}\צהגדרה{.}

\ערך{בטול }\הגדרה{- התכללות }\מקור{[עפ״י קובץ ה צה]}\צהגדרה{. }\\\ערך{בטול}\myfootnote{ \textbf{הטעות שיש במהותיות עצמותית}\textbf{, }\textbf{והתגברות החפץ לאשתאבה בגופא דמלכא} - ע׳ לקוטי תורה לרש״ז, שיר השירים א. ״והנה כלה יש בו ב׳ פירושים. הא׳ לשון כליון וכו׳, והב׳ מלשון כלתה נפשי והיינו תשוקת הנפש לידבק וליכלל באורו ית׳״. ע״ע פנק׳ א תכד סי׳ מח.\label{38}}\ערך{ }\הגדרה{- }\משנה{בטול אל האור-האלהי\mycircle{°}}\הגדרה{ - }\הגדרה{להשתאב-בגופא-דמלכא\mycircle{°}}\הגדרה{ }\מקור{[עפ״י א״ק ב שצח]}\צהגדרה{. }\\\הגדרה{}\הגדרה{כלות-הנפש-לאלהים\mycircle{°}}\הגדרה{ }\מקור{[עפ״י שם, ע״ר א מז, סז]}\צהגדרה{. }\\\משנה{ביטול גמור של מהות עצמו }\הגדרה{- הכרת הנשמה את כל הטעות שיש במהותיות עצמותית, והתגברות החפץ לאשתאבה בגופא דמלכא, בשלמות אין סוף של נועם העליון. }\הגדרה{הענוה\mycircle{°}}\הגדרה{ הגמורה, והשפלות העמוקה, שהעצמיות היא בה רק }\הגדרה{שירים\mycircle{°}}\הגדרה{, כלומר ענין של חסרון שנשאר בלתי כלול בשלמות העליונה }\מקור{[עפ״י קובץ א שיב]}\צהגדרה{. }\\\הגדרה{ע״ע כניעה, ההכנעה מפני האלהות. ע׳ במדור פסוקים ובטויי חז״ל, ונחנו מה. ושם, ואנכי תולעת ולא איש. ושם, כלה שארי ולבבי צור לבבי וחלקי אלהים לעולם.}

\ערך{בטול }\הגדרה{- }\משנה{בטול פנימי עדין }\הגדרה{- (בטול עצמי) המשפיל את הצד המכוער שבנו ומרומם את כל מהות הטוב והעדין}\צהגדרה{ <ואינו}\צהגדרה{ מטשטש את אומץ החיים}\צהגדרה{>}\צמקור{ [עפ״י א״ש יד כא]}\צהגדרה{. }

\ערך{בטחון }\הגדרה{- דעת }\הגדרה{בבינת-לב\mycircle{°}}\הגדרה{ פנימית ואדירה }\מקור{[עפ״י ע״א ג ב צ]}\צהגדרה{. }\\\מעוין{◊}\הגדרה{ בא מתוך }\הגדרה{הדבקות-האלהית\mycircle{°}}\הגדרה{, הבאה מתוך }\הגדרה{האמונה-השלמה\mycircle{°}}\הגדרה{}\myfootnote{ \textbf{הדבקות}\textbf{-}\textbf{האלהית, הבאה מתוך האמונה}\textbf{-}\textbf{השלמה} - ע׳ במדור הכרה והשכלה והפכן, ״טביעות עין״.\label{39}}\הגדרה{ }\מקור{[עפ״י ע״ר ב פו]}\צהגדרה{. }\\\מעוין{◊ }\משנה{הבטחון}\הגדרה{ כולל בתוכו את הדבקות האמונית בשלמותה, וממשיך את }\הגדרה{אור-החיים\mycircle{°}}\הגדרה{ }\הגדרה{ממקור-החיים\mycircle{°}}\הגדרה{, }\הגדרה{מחי-העולמים\mycircle{°}}\הגדרה{ ברוך הוא, לכל מי שמתעטר בקדושת האמונה והדבקות האמיתית }\מקור{[עפ״י שם]}\צהגדרה{.}\\\ערך{בטחון}\myfootnote{ \textbf{יסוד הבטחון} - \textbf{הוא לא שהאדם בטוח שמה שהוא דורש ימלא ד׳, כי אפשר שמה שהוא חושב, שהוא הטוב, הוא ההפך מהאמת. אלא שהוא בטוח בחסד עליון וכו׳} - ע״ע ע״א ג א מג. צבי לצדיק לרי״מ חרל״פ פרק ד. \newline
להבחנת מדרגות הבטחון השונות ע׳ ע״א שם שם מו, ובע״א ב ט קעג. ובקבצ׳ ב עמ׳ יח סי׳ טז: ״שיטת הלאומיות שאומרים וכו׳ שאין להשתמש בבטחון לענין לאומי״.\newline
\textbf{אדם הנברא בצלם-אלה}\textbf{ים}\textbf{ הוא תמצית כל ואחוד הכל, ומצד הכל הלא אין אבוד ולא הירוס} - ע״ע ע״ר א שלב (א״ק ב תקג). ע״ע ע״ר א קעג ד״ה וטעונה ״לא אבדת ציורים ורשומים פרטיים הוא הענין של מלוי הקדש, אלא הרמתם העשירה עם כל רכוש ציוריהם לרום מעלות הקדש״. ושם שם א ד״ה אני, שם שם רטז ד״ה ד׳ צבאות, שם שם מט ד״ה וצור חבלי, ושם שם רכב ד״ה דעו. ובע״ר ב סב ד״ה דרשתי שם שם סג ד״ה הביטו, שם שם קנז ד״ה חלק, ושם שם רנה.\newline
\textbf{תכלית הבטחון} - \textbf{האסונות, הנכונים לבא על בני אדם, באופן כזה שאין הזהירות האנושית יכולה להגן הרי הם סרים מן הבוטח }- ע״ע שם בע״א ג ב קצב, ובמשפט כהן עמ׳ שכז-ח, שנט. ובעזרת כהן עמ׳ קמא. ע״א א א קא ״גם על ההשגחה האלקית ראוי שיקבע בנפשו שלא יאתה להיות סומך כ״א במה שאין ידו מגעת להשתדל בעצמו״. ושם ח״ב ט קכ ״הבטחון הוא מוגדר כשנשלים את חק ההשתדלות במה שהוא בידינו, ובמה שאין יכולת שלנו מגיע(ה) לזה, שם הוא מקום הבטחון. כי במקום שהיכולת מתגלה חלילה להשתמש בבטחון, שאין זה בטחון כ״א הוללות ומסה כלפי מעלה״. ע״ע שם שם כג, ושם ג ב קצו. ע״ע רמב״ם, פיהמ״ש פסחים נו., עקדה שער כו, בראשית דף רכא.-רכו:, וברבנו בחיי עה״ת, שמות יג יח. ובבאור הגר״א על משלי יד טז, (מהד׳ פיליפ) עמ׳ 173 ד״ה וכסיל מתעבר ובוטח ״הכסיל עובר במקום שיש לטעות או במקום סכנה ובוטח בה׳ שלא יבא לידי רע, והוא בטחון הכסילים כי מי מכריח אותו לילך במקום סכנה״. (אך ע׳ שם ג ה, עמ׳ 49 50 ובהערה 24 שם וצ״ע, ואולי י״ל עפ״י דבריו שם טז כ, עמ׳ 197, בחלוק בין עוה״ז ועוה״ב, ותורה ותפילה. מ״מ, אין שיטה זו עולה בקנה אחד עם שיטת האמונות ודעות לרס״ג י טו שאומר על מי שאומר שבטוח בד׳ על עניני עוה״ז בלא השתדלות, שהיא דעה זרה, דא״כ יאמר ג״כ על עניני עוה״ב, ומה תכלית התוהמ״צ. ואמנם, כבר נחלקו בענין זה אבות העולם ראשונים ואחרונים. וע׳ באלפי מנשה, ח״א, פרק צח. ובמערכתו של ר״ש מאלצאן, אבן שלמה, ליקוטים בסוף הספר דף סט.-עא. ואא״ל על פיהם). ע״ע אבן ישראל ח״ג, בהקדמה.\label{40}}\הגדרה{ - }\משנה{יסוד הבטחון המעלה את האדם לתכונת קדושה\mycircle{°} עליונה, רוממות נפש וגדולת קדש }\הגדרה{- <יסוד בטחון זה, הוא לא אותו }\הגדרה{הציור\mycircle{°}}\הגדרה{, שהאדם יצייר בעצמו שהוא בטוח, שמה שהוא דורש ומבקש, וחושב שדרוש לו, ימלא }\הגדרה{ד׳\mycircle{°}}\הגדרה{, כי אפשר שמה שהוא חושב, שהוא הטוב, הוא ההפך מהאמת. אלא> שהוא בטוח }\הגדרה{בחסד\mycircle{°}}\הגדרה{ }\הגדרה{עליון\mycircle{°}}\הגדרה{, שברא את העולם ובנה אותו, וכוננו, ומשגיח עליו ברב חסד, ועל כן אין מקום לשום דאגה, לשום עצבון רוח, כי הלא יודעים אנו, שחסד אל נטוי על כל יצוריו, והננו נכנסים תחת כנפי חסדו בכל רגע }\מקור{[ע״ר א רכ]}\צהגדרה{.}\\\משנה{יסוד (הבטחון ב)שם-ד׳\mycircle{°}}\הגדרה{ - לבטוח שההנהגה עוזרת לקנות השלמות האמיתית }\מקור{[עפ״י ע״א א א נג]}\צהגדרה{. }\\\משנה{יסוד הבטחון והשמחה }\הגדרה{- נובע מהבירור הפנימי שאין לחפוץ, גם לעניני עצמו, כ״א את מה שהוא חפץ אדון כל ב״ה. ואז ימלא אדם שמחה ואומץ לב כפי מדת בירור דבר זה בלבבו, ולפי ערך התאמת כל ארחות חייו לזאת המדה העליונה }\מקור{[פנ׳ לד]}\צהגדרה{. }\\\משנה{בטחון נשגב\mycircle{°} עליון\mycircle{°}}\הגדרה{ - }\הגדרה{החסיון\mycircle{°}}\הגדרה{ }\הגדרה{האידיאלי\mycircle{°}}\הגדרה{ הבא מתוך }\הגדרה{ההופעה\mycircle{°}}\הגדרה{ העליונה, בלא שום מבט על }\הגדרה{הגורל\mycircle{°}}\הגדרה{ הנופל בחלקה של האישיות הפרטית. כי מתוך השגוב העליון }\הגדרה{והזיו-האלהי\mycircle{°}}\הגדרה{, של מקור כל השלמות ושורש כל }\הגדרה{תענוג\mycircle{°}}\הגדרה{ }\הגדרה{ואור\mycircle{°}}\הגדרה{, הכל }\הגדרה{מתבטל\mycircle{°}}\הגדרה{ מרוב }\הגדרה{נועם\mycircle{°}}\הגדרה{ }\מקור{[עפ״י ע״ר ב עד]}\צהגדרה{. }\\\הגדרה{הבהירות של הידיעה האלהית העליונה וחשק הלב הפנימי בהתמלאותם של }\הגדרה{האידיאלים-האלהיים\mycircle{°}}\הגדרה{ במלא כל היש, והבירור הגמור שכן הוא, ושהכל הולך לחפץ הטוב העליון, (המביאים) שמחת הנשמה הפנימית, וכל דאגה עצבית מתגרשת, }\הגדרה{וחדות\mycircle{°}}\הגדרה{ ד׳ מתמלאת בכל מהותו של אדם }\מקור{[עפ״י קובץ ה קכה]}\צהגדרה{. }\\\משנה{יסוד הבטחון }\הגדרה{- יסוד הבטחון בא מתוך }\הגדרה{החסן\mycircle{°}}\הגדרה{ אשר לנו }\הגדרה{באלהים\mycircle{°}}\הגדרה{ }\הגדרה{סלה\mycircle{°}}\הגדרה{. כשאדם הנברא }\הגדרה{בצלם-אלהים\mycircle{°}}\הגדרה{ הלא הוא באמת תמצית כל ואחוד הכל, ומצד הכל הלא אין אבוד ולא הירוס, לא השפלה ולא }\הגדרה{ירידה\mycircle{°}}\הגדרה{, כ״א כולו אומר }\הגדרה{כבוד\mycircle{°}}\הגדרה{ }\הגדרה{וחיים\mycircle{°}}\הגדרה{, וכשהאדם מכיר את }\הגדרה{עוזו\mycircle{°}}\הגדרה{ באלהי-הצבאות, }\הגדרה{ד׳-צבאות\mycircle{°}}\הגדרה{, הלא הוא מלא בטחון }\מקור{[ע״ר א קנ]}\צהגדרה{. }\\\משנה{תכלית הבטחון }\הגדרה{- }\הגדרה{קרבת-אלהים\mycircle{°}}\הגדרה{ הנמשכת מן הבטחון, וגבורת הנפש }\הגדרה{בעז-ד׳\mycircle{°}}\הגדרה{ הנמשכת ממנה בעת צר, שהאדם מוצא לו תמיד }\הגדרה{מחסה\mycircle{°}}\הגדרה{ }\הגדרה{בשם-ד׳\mycircle{°}}\הגדרה{, וגם בעת אשר כל המסיבות }\הגדרה{הטבעיות\mycircle{°}}\הגדרה{ כבר חדלו כח להצילו מרעתו, }\הגדרה{עז-ד׳\mycircle{°}}\הגדרה{ }\הגדרה{ישגבהו\mycircle{°}}\הגדרה{ תמיד, והאסונות, הנכונים לבא על בני אדם, באופן כזה שאין הזהירות האנושית יכולה להגן, הרי הם סרים מן הבוטח, כשם שרגשי הפחד הדמיוני סרים מפני }\הגדרה{האור\mycircle{°}}\הגדרה{ של הבטחון, ונפשו מלאה אומץ, וסדר שלותי קבוע בה }\מקור{[עפ״י ע״א ג ב קצב, ע״ר ב עה]}\צהגדרה{. }\\\משנה{הבטחון מדתו }\הגדרה{- הרחבת כח העז והגבורה, אפילו במה שהוא למעלה מגבולי כח היכולת הקבועה בכחות האדם הגלויים, כי אין מעצור לד׳ להושיע ולעזור גם לאין כח. }\צהגדרהמודגשת{מדת הבטחון }\צהגדרה{באה לאחר שיאזר האדם בגבורה בכל אשר תשיג ידו בכחותיו החומריים והרוחניים, ובבאו לגבול ששם יש לפניו מעצור כח המוגבל החלש, אל יפול לבבו. <וזאת היא }\צהגדרהמודגשת{מדת הבטחון}\צהגדרה{ שצריכה להתחבר תמיד עם מדת הגבורה, המועלת למלא את נפש האדם כבוד ועז. וכשהיא מתחברת עם הדעה השלמה והמוסר האמיתי, היא מדרכת את האדם בדרך ד׳ העליונה, ומכשרתו להיות בד׳ מבטחו. ומעלתו, שיהיה כבוד ד׳ חופף עליו לעשות לו ניסים, בין גלויים בין נסתרים, במערכות סדרי הטבע>}\צמקור{ [עפ״י ל״ה 177 (פנק׳ ב קכא)]}\צהגדרה{.}\\\צהגדרה{יסוד החיים הלא הוא הכח לפעול ולעשות, כל איש לפי ערכו, וכל חברה לפי ערכה. החיים הטובים המה, שתהיינה הפעולות מסודרות יפה ועולות תמיד במעלה בהוספת ערך והשלמה. והנה האדם הוא איננו חפשי גמור, פעמים רבות יתיצבו לו כצר מונעים רבים שיעכבוהו שלא יוכל ללכת מהלך החיים שלו, שלא יוכל לפעול לפי תכונתו וערכו, ואז הוא צריך להתגבר עליהם בכל עז. ועל זה צריך }\משנה{שיבטח בד׳, }\הגדרה{שאם אפילו כחותיו לא יספיקו לו, מכל מקום ״אין מעצור לד׳ להושיע״, ותשועת ד׳ תשגבהו להסיר המניעות, למען יוכל לפעול ולעבוד ולחיות כראוי}\מקור{ [עפ״י ל״ה 240]}\צהגדרה{. }\\\הגדרה{ע׳ בנספחות, מדור מחקרים, בטחון לעומת אמונה.}

\ערך{בי }\הגדרה{- מבטא, מבליט, את המהותיות הפנימית של הנושא, המכריז על עצמו את התגלותו העצמית, ומודיע את הגנוז בקרבו }\מקור{[ר״מ קלב]}\צהגדרה{.}\\\ערך{בי }\הגדרה{- מורה על }\הגדרה{העצמיות\mycircle{°}}\הגדרה{ המיוחדה של האדם ותוכן החיים הטבעי שלו, שהיא הבסיס לקבל עליה את }\הגדרה{האור\mycircle{°}}\הגדרה{ }\הגדרה{העליון\mycircle{°}}\הגדרה{ של }\הגדרה{הנשמה\mycircle{°}}\הגדרה{ }\מקור{[ע״ר א ג]}\צהגדרה{. }\\\הגדרה{התוכן המורגשי של האדם בהויתו הפרטית. אותו תוכן שיש עמו ג״כ חבור להצד הפרטי, המסמן את הפירוט היחידי של האדם באשר הוא מוגבל ומצומצם }\מקור{[שם סז]}\צהגדרה{. }\\\ערך{בי}\הגדרה{ - מעמקי }\הגדרה{הנשמה\mycircle{°}}\הגדרה{, הריכוז היותר }\הגדרה{פנימי\mycircle{°}}\הגדרה{ ויותר }\הגדרה{כללי\mycircle{°}}\הגדרה{}\מקור{ [ר״מ קלב]}\צהגדרה{.}

\ערך{ביזה }\הגדרה{- מה ששוללים דרך מלחמה }\מקור{[ר״מ קלא]}\צהגדרה{. }

\ערך{״בית ד׳״ }\הגדרה{- }\הגדרה{בית-המקדש\mycircle{°}}\הגדרה{ מצד היותו מרכז }\הגדרה{הקדושה\mycircle{°}}\הגדרה{ וע״י נשפעת קדושה גדולה }\הגדרה{בעבודת-ישראל\mycircle{°}}\הגדרה{ בכל }\הגדרה{המצות\mycircle{°}}\הגדרה{. נגד ענין }\הגדרה{המצות\mycircle{°}}\הגדרה{ והעבודה המעשית }\מקור{[עפ״י ע״א א א ז]}\צהגדרה{. }\\\הגדרה{ע״ע ״היכל ד׳״.}

\ערך{״בית ד׳״ }\הגדרה{-  ע׳ במדור פסוקים ובטויי חז״ל.}

\ערך{בית דין הגדול }\הגדרה{- מרכזנו הדתי, היסוד העיקרי לביאור התורה לפרטיה הנולדים בהמשך החיים, היושב ״במקום אשר יבחר ד׳״, שמשם הוראה צריכה לצאת לכל ישראל }\צהגדרה{[}\צהגדרה{א}\צהגדרה{״ה ב (מהדורת תשס״ב) }\צמקור{127}\צהגדרה{].}

\ערך{בית הגדול }\הגדרה{- }\משנה{הבית הגדול }\הגדרה{- כל העולמים כולם בכל }\הגדרה{הדר\mycircle{°}}\הגדרה{ }\הגדרה{כונניותם\mycircle{°}}\הגדרה{ }\מקור{[ר״מ קמג]}\צהגדרה{.}\\\הגדרה{ע׳ במדור פסוקים ובטויי חז״ל, בית ד׳.}

\ערך{בית הכנסת }\הגדרה{- מקום הקיבוץ הציבורי }\הגדרה{לעבודת\mycircle{°}}\הגדרה{ }\הגדרה{השי״ת\mycircle{°}}\הגדרה{ }\מקור{[ע״א א א נו]}\צהגדרה{.}\\\הגדרה{המכון לקיבוץ עבודת השי״ת והרמת כח }\הגדרה{האמונה\mycircle{°}}\הגדרה{ }\הגדרה{ויראת-ד׳\mycircle{°}}\הגדרה{ בלבבות }\צהגדרה{[}\צהגדרה{ע}\צהגדרה{״א ג א כה].}\\\הגדרה{(בית) שתעודתו היא עבודת השם ית׳, <שהוא }\הגדרה{המקום\mycircle{°}}\הגדרה{ היותר גבוה שבחיים, שכל פינות החיים הפרטיים כולן אליו יפנו וע״י יתעלו ויתרוממו}\צהגדרה{>}\צמקור{ [עפ״י שם]}\צהגדרה{.}\\\צהגדרה{הבית של ההתכנסות הפנימית לשם }\צהגדרה{ד׳-אלהי-ישראל\mycircle{°}}\צהגדרה{, להופעת רוחו ושייכות }\צהגדרה{מצוותו\mycircle{°}}\צהגדרה{ }\צמקור{[ל״י א נו].}\\\ערך{״אחורי בית הכנסת״}\myfootnote{ ברכות ו:.\label{41}}\ערך{ }\הגדרה{- }\משנה{באיכותו וערכו}\הגדרה{ - הצד הטפל של בית הכנסת - השגת המבוקש }\הגדרה{בתפילה\mycircle{°}}\הגדרה{ }\מקור{[עפ״י פנק׳ ג ער]}\צהגדרה{. }\\\ערך{פנים בית הכנסת}\הגדרה{ - }\משנה{באיכותו וערכו}\הגדרה{ - החלק העיקרי של בית הכנסת - תכליתו להרבות }\הגדרה{כבודו\mycircle{°}}\הגדרה{ של השי״ת בלב כל הנכנסים בתוכו, ולתכלית זו באה }\הגדרה{התפילה\mycircle{°}}\הגדרה{}\צהגדרה{ [עפ}\צהגדרה{״י פנק׳ ג ער].}\\\הגדרה{ע׳ בנספחות, מדור מחקרים, רנה ותפילה. ע׳ במדור פסוקים ובטויי חז״ל, המתפלל אחורי בית הכנסת נקרא רשע. ושם, מסיר אזנו משמוע תורה גם תפילתו תועבה. }

\ערך{בית הכסא }\הגדרה{- מקום התגלות שפלות החומריות האנושית מצד עולמו הפנימי }\מקור{[ע״א ג א יג]}\צהגדרה{.}\\\מעוין{◊}\הגדרה{ האמצעי }\הגדרה{לטהרה\mycircle{°}}\הגדרה{ של הלכלוך הטבעי המחליא באי נקיונו בהשארתו בגויה }\מקור{[ע״א ג ב עג]}\צהגדרה{.}\\\הגדרה{ע׳ במדור מלאכים ושדים, שעיר של בית הכסא.}

\ערך{בית המרחץ }\הגדרה{- מקום התגלות שפלות החומריות האנושית מצד עולמו החיצוני }\מקור{[ע״א ג א יג]}\צהגדרה{.}

\ערך{בך }\הגדרה{- מבטא את היחש החודר בפנימיותו של נושא חוצי, העומד לנכח הנושא העצמי, המביע את הרעיון }\מקור{[ר״מ קלב]}\צהגדרה{.}

\ערך{בכור }\הגדרה{- }\משנה{(ענינו) }\הגדרה{- היסוד הראשי של משך החיים}\מקור{ [עפ״י ע״ר א מב]}\צהגדרה{.}

\ערך{בכורה }\הגדרה{- תכונה, שמחיבת להיות משפיע ופועל פעולה חנוכית על יתר הבנים והבנות, שבאים אחריו }\מקור{[ע״ר א קו]}\צהגדרה{.}\\\הגדרה{ע׳ במדור מצוות, הלכות, מנהגים וטעמיהן, בכורות, קדושת הבכורות. }

\ערך{בל }\הגדרה{- הוראת שלילה }\מקור{[ר״מ קלג]}\צהגדרה{. }

\ערך{בם }\הגדרה{- מורה חדירת הנושא בתוך התוכנים הרבים העומדים בריחוק מקום מהנושא המתאר }\מקור{[ר״מ קלג]}\צהגדרה{. }

\ערך{בן }\הגדרה{- התולדה האיתנה, העובדת והמסדרת, היורשת את }\הגדרה{ההארות\mycircle{°}}\הגדרה{ העליונות שהן הן הגורמות את החידוש התולדתי }\מקור{[ר״מ קלד]}\צהגדרה{. }

\ערך{בן}\הגדרה{ - }\מעוין{◊ }\הגדרה{המיוחש }\הגדרה{לאב\mycircle{°}}\הגדרה{ ביחש הקשר הנשמתי היותר חזק }\מקור{[עפ״י ע״ר א פו]}\צהגדרה{.}\\\הגדרה{ע׳ במדור פסוקים ובטויי חז״ל, נחלת ד׳ בנים. }

\ערך{בן}\הגדרה{ - }\משנה{״בנים״}\הגדרה{ - הצעירות, הצריכה לקבל את }\הגדרה{השפעתו\mycircle{°}}\הגדרה{, של }\הגדרה{האב\mycircle{°}}\הגדרה{ הגדול }\צהגדרה{[}\צהגדרה{ע}\צהגדרה{״ר ב סה].}

\ערך{בן }\הגדרה{- }\משנה{(בעבודת ד׳ לעומת עבד) }\הגדרה{- ע׳ במדור מדרגות והערכות אישיותיות. }

\ערך{בסום }\הגדרה{- }\משנה{התבסמות העולם}\myfootnote{ בש״ק, קובץ א קט: ״התבסמות העולם ע״י כל המשך הדורות, ע״י ביסום היותר עליון של גילויי השכינה בישראל וע״י נסיונות הזמנים, התגדלות היחש החברותי, והתרחבות המדעים, זיקקה הרבה את רוח האדם, עד שאע״פ שלא נגמרה עדיין טהרתו, מ״מ חלק גדול מהגיונותיו ושאיפת רצונו הטבעי הנם מכוונים מצד עצמם אל הטוב האלהי״. ושם שצד: ״העולם בהתבסמותו הולך הוא ומתעלה בתוכיותו. האדם מוצא את חפצו, הולך וטוב בערכו הפנימי״.\label{42}}\הגדרה{ - }\הגדרה{תיקון\mycircle{°}}\הגדרה{ והתעלות }\מקור{[עפ״י פנק׳ ג שט, שי]}\צהגדרה{.}\\\ערך{בסומה של הנשמה}\הגדרה{ - שאיבתה ממעין }\הגדרה{הקדושה\mycircle{°}}\הגדרה{ האלהית הפנימית. סוד הקדושה }\הגדרה{האצילית\mycircle{°}}\הגדרה{ הפנימית, הופעת הנשמה }\צהגדרה{[עפ״י }\צהגדרה{ע}\צהגדרה{״ר א קנג].}\\\הגדרה{ע״ע מתבסם.}

\ערך{בצבוץ }\הגדרה{- תיאור לכל רושם צמחני }\מקור{[עפ״י ר״מ קלו]}\צהגדרה{. }

\ערך{בצבוץ }\הגדרה{- ההפריה המתבודדת בחוגיה, וההתגלות הקולית }\מקור{[ר״מ קלו]}\צהגדרה{. }

\ערך{בִּצָה }\הגדרה{- מקום המוכשר לגידול, מסמל את הבסיס הדוגמתי בהתוכנים }\הגדרה{הרוחנים\mycircle{°}}\הגדרה{, בגליפת המושגים, במהות השכלתם }\הגדרה{וציור\mycircle{°}}\הגדרה{ אמיתת }\הגדרה{צדקם\mycircle{°}}\הגדרה{, המשפיעים על היסוד המעשי, ערכי }\הגדרה{הצדק\mycircle{°}}\הגדרה{ }\הגדרה{והמישרים\mycircle{°}}\הגדרה{ }\מקור{[ר״מ קלו]}\צהגדרה{. }

\ערך{בק }\הגדרה{- ענין של התרוקנות }\מקור{[ר״מ קלו]}\צהגדרה{. }

\ערך{בֹּקֶר }\הגדרה{- }\הגדרה{עת\mycircle{°}}\הגדרה{ }\הגדרה{ההזרחה\mycircle{°}}\הגדרה{ של }\הגדרה{האורה\mycircle{°}}\הגדרה{ האמיתית, אשר תביא להכרת החיים במהותם העצמית }\מקור{[ע״ר ב עג]}\צהגדרה{. }

\ערך{בקורת }\הגדרה{- }\משנה{מטרת הבקורת }\הגדרה{- להגיה אורות מאופל. לברר בחופש והרחבה, על צד השקר המועט, המוכרח להמצא בתוך האמת הגדולה והמרובה, ועל ניצוץ האמת המתגלה בתוך עומק החושך של השקר }\צהגדרה{[}\צהגדרה{מ}\צהגדרה{״ר }\צמקור{288}\צהגדרה{].}

\ערך{בקשת אלהים }\הגדרה{- דרישת חיים של אמת פנימית }\מקור{[עפ״י קובץ ד פז]}\צהגדרה{.}\\\הגדרה{ע׳ במדור פסוקים ובטויי חז״ל, דרישת ד׳.}

\ערך{בר }\הגדרה{- המזון המבריא בכל הערכים }\מקור{[ר״מ קלז]}\צהגדרה{. }

\ערך{בר }\הגדרה{- הבנה של חוצה, העומד(ת) מחוץ }\הגדרה{להפרגוד\mycircle{°}}\הגדרה{ אשר }\הגדרה{חביון-עז\mycircle{°}}\הגדרה{ קודש הקדשים של הדממה העליו(נה) אצור שמה }\מקור{[עפ״י ר״מ קלח]}\צהגדרה{. }

\ערך{בר }\הגדרה{- תרגום }\הגדרה{בן\mycircle{°}}\הגדרה{ }\מקור{[עפ״י ר״מ קלח]}\צהגדרה{. }

\ערך{ברה }\הגדרה{- מנצחת את כל צללי המחשכים }\מקור{[ע״ר ב נז]}\צהגדרה{. }

\ערך{ברוך }\הגדרה{- }\משנה{(משמעותו בברכת המצוות)}\הגדרה{ - }\הגדרה{מקור-חיי-החיים\mycircle{°}}\הגדרה{, אוצר הטוב והקודש, ששפעת כל ברכת ההויה שמה היא גנוזה, המוער בבאנו להוציא מן הכח אל הפועל את האור הקדוש של }\הגדרה{מצוה\mycircle{°}}\הגדרה{ מעשית, בהתגלות המפעלית, ובהארת היפעה האלהית, הנובעת מראש מקור אור החיים העליונים של }\הגדרה{חי-העולמים\mycircle{°}}\הגדרה{, מתמשכת אז שפעת }\הגדרה{ברכה\mycircle{°}}\הגדרה{, ההולכת ומפלסת לה את נתיבה בהתחשפות האורה של החיים המעשיים, ומעין החיים }\הגדרה{מתברך\mycircle{°}}\הגדרה{ במקורו, בהיותו מוכן להתגבר בשטף ברכותיו ע״י אותו השביל החדש, ההולך ומתבלט ע״י מפעלנו במעשה המצוה הבאה ממרום החפץ האלוהי העליון, מקור החיים והטוב, אל תחתית מעמקי העולם, הנמצר במצריו החמריים וכוחותיו המעשיים }\מקור{[עפ״י ע״ר א ז]}\צהגדרה{. }

\ערך{בריאה }\הגדרה{- }\משנה{הבריאה }\הגדרה{- הממשיות המוגלמת }\מקור{[א״א 125]}\צהגדרה{. }

\ערך{בריאה}\myfootnote{ הזכיר הרב מרדכי גלובמן בנ״א ה עמ׳ 22 מדברי אבן עזרא, בראשית א א ״רובי ממפרשים אמרו שהבריאה להוציא יש מאין וכן ״אם בריאה יברא ה׳״. והנה שכחו ״ויברא אלהים את התנינים״ ושלש בפסוק אחד ״ויברא אלהים את האדם״. ו״ברא חשך״ שהוא הפך האור שהוא יש. ויש דקדוק המלה ברא לשני טעמים: זה האחד, והשני ״לא ברה אתם לחם״. וזה השני ה״א תחת אלף כי כמוהו ״להברות את דוד״ כי הוא מהבנין הכבד הנוסף ואם היה באל״ף היה כמו ״להבריאכם״ ומצאנו מהבנין הכבד ״ובראת לך״. ואיננו כמו ״ברו לכם איש״ רק כמו ״וברא אתהן״. וטעמו לגזור ולשום גבול נגזר והמשכיל יבין״. והביא את האברבנאל בראש אמנה ש״ברא״ הונח בהנחה ראשונה יש מאין, ומזה הושאל על כל בריאה ניסית או נשגבה היוצאת מגדר הטבע. אמנם גם פירושי ראשונים אלו לא יעמודו במבחן דברי האדרת אליהו, בראשית א א, ד״ה ברא ״הבינו כל מפרשי הדת שמורה על דבר מחודש יש מאין. אבל מה יאמרו ״ויברא אלהים התנינים הגדולים״ וכן ״ויברא אלהים את האדם בצלמו״ וכן מה שתקנו קדמונינו בכל ברכת הנהנין ״בורא פרי האדמה״, ״בורא פרי העץ״, ונשאר כללם הידוע מעל״. על כן פירש הגר״א שם ״מלת בריאה הונח להורות על חידוש העצם אשר אין בכח הנבראים אפי׳ כולם חכמים ונבונים לחדשו... וכן תיקנו ״בורא פרי״ כי אינו בכח כל הנבראים לחדשו בעבור שהוא עצם פועל ה׳״. ״ברא הוא עצם הדבר ואפילו יש מיש״. ובמטפחת ספרים ליעב״ץ, פרק ח ד ״לשון בריאה מורה על יש מיש על דרך האמת. ויתכן גם בריאה אין מיש״ וכו׳ עש״ע שהאריך, (ע״ע ע״ט לב ד״ה לא). וע׳ בדרך חיים למהר״ל, רי ושם שכא ״לשון בריאה נאמר על הצורה הנבדלת האלקית שדבק בנבראים, וזה כי האדם כתיב בפי׳ בצלם אלקים עשה את האדם, שתדע מזה כי דבק בצורת האדם ענין אלקי, וכן בשמים וארץ שהם כלל העולם, אין ספק שדבק בהם ענין אלקי ולכך כתיב לשון בריאה. וכן התנינים הגדולים שהכתוב מפרש שהם תנינים גדולים, ולפי גדלם עד שהם בריאה נפלאה דבק בהם ענין האלקי נאמר אצלם לשון בריאה. כי כל הנבראים יש בהם דבר זה כמו שיתבאר רק התורה הזכירה לשון בריאה באלו שלשה, כי באלו שלשה מפורסם ונראה לגמרי לעין ובשאר דברים אינו נראה״. והרש״ט גפן, בממדים, הנבואה והאדמתנות, תורת הנבואה הטהורה, מאמר שני, עיון בנבואה ובמופתים, פרק יג הגדיר בריאה: ״יציאת היש ממה שאיננו נופל תחת הציור באופן בלתי מובן ובלתי מושג לא לשכל ולא לחוש ומבלעדי כל הכרח״. עע״ש פרק יד. ושם, מעשה בראשית והאדמתנות, סוף דבר, א, הגדיר: ״הבריאה הוא השתכללות צורות הזמן והמקום על פי כוח נסתר ונעלם, בדעת האדם והכרתו״. עע״ש הערה 11. ובקסת הסופר לר״א מרקוס, בראשית א א ״ברא - הוציא יש מאין שלא כפי טבע הנברא״, עע״ש עמ׳ ב-ד בהרחבת דברים נפלאה. ע״ע מנֹפת צוף, למו״ר הרב יהונתן שמחה בלס, ח״ב עמ׳ 825 ״המונח ״ברא״ ראוי להמצאת מציאות ראשונית שלאחר מכן נותרה כבררת מחדל״. כדרכו של הרב ברב דבריו, על פי הסברו בסוגיה יעלו כל הפירושים בקנה אחד.\label{43}}\הגדרה{ - }\הגדרה{התהוות\mycircle{°}}\הגדרה{ העולם וכל אשר לו מאותו }\הגדרה{החפץ\mycircle{°}}\הגדרה{ }\הגדרה{הקדום\mycircle{°}}\הגדרה{, המלא }\הגדרה{עז\mycircle{°}}\הגדרה{, המעוטר }\הגדרה{בגבורה\mycircle{°}}\הגדרה{ ובכל }\הגדרה{אור\mycircle{°}}\הגדרה{ }\הגדרה{קדשי-קדשים\mycircle{°}}\הגדרה{, עדינות }\הגדרה{הטוב\mycircle{°}}\הגדרה{, }\הגדרה{החסדים\mycircle{°}}\הגדרה{ הנאמנים }\הגדרה{עדי-עד\mycircle{°}}\הגדרה{ }\מקור{[א״ק ג ע]}\צהגדרה{. }\\\הגדרה{היש המצומצם שאנו פוגשים }\הגדרה{בציור\mycircle{°}}\הגדרה{ של הויה, }\הגדרה{שבחופש\mycircle{°}}\הגדרה{ ולמטרה ידועה, נלחצה בצמצומה}\מקור{ [עפ״י קובץ ז קנא]}\צהגדרה{.}\\\הגדרה{היצירה המוחלטה. היכולת הבלתי תנאית ממציאה הכל, על-פי היסוד החפצי}\מקור{ [עפ״י קובץ ה קפה]}\צהגדרה{.}\\\צהגדרה{הוצאת יש מאין}\צמקור{ [ק״ת עז].}\\\משנה{בריאת העולם }\הגדרה{- }\הגדרה{הופעת\mycircle{°}}\הגדרה{ האור של }\הגדרה{הקדושה\mycircle{°}}\הגדרה{ }\הגדרה{העליונה\mycircle{°}}\הגדרה{ }\הגדרה{התורנית\mycircle{°}}\הגדרה{,  בתור }\הגדרה{אור-החיים\mycircle{°}}\הגדרה{ של }\הגדרה{קבלת-מלכות-שמים\mycircle{°}}\הגדרה{, של }\הגדרה{כבוד-המלכות\mycircle{°}}\הגדרה{, המאיר בהויה והיצירה כולה }\מקור{[עפ״י ע״ר א קיא]}\צהגדרה{. }\\\תערך{בריאה ראשונה }\הגדרה{- }\תמשנה{״בראשית ברא״ }\הגדרה{- }\תהגדרה{(התהוות) שלא על דרך }\תהגדרה{השתלשלות\mycircle{°}}\תהגדרה{ אלא בכונה ראשונה }\תמקור{[עפ״י נ״א ה 22-21]. }\\\הגדרה{ע״ע נברא. ע״ע מחשבה אלהית על דבר העולם. ע׳ בנספחות, מדור מחקרים, תכלית הבריאה. ע׳ במדור שמות כינויים ותארים אלהיים, בורא. ר׳ יצירה. ר׳ עשיה.}

\ערך{בריאה }\הגדרה{- }\משנה{עולם הבריאה }\הגדרה{- ע׳ במדור מונחי קבלה ונסתר. }

\ערך{בריאות }\הגדרה{- המצב הטוב המסכים אל כלל המציאות, מבלי שיופרע הסדר ביציאת פרט אחד מהסכמתו אל הכלל }\מקור{[עפ״י ע״א א ה נח]}\צהגדרה{. }

\ערך{בריאות רוחנית}\הגדרה{ - }\משנה{הבריאות הרוחנית}\הגדרה{ - ההרגשות הנפשיות כולן, במצבן הנורמלי. הרגשת היופי, האהבה, נטיית הגבורה, חשק החיים הבריאים}\מקור{ [עפ״י קבצ׳ ב קכז]}\צהגדרה{.}

\ערך{בּרִיכַה }\הגדרה{- ע׳ במדור גוף האדם אבריו ותנועותיו.}

\ערך{בריקה }\הגדרה{- פעולת התקפה (רוחנית) חזקה בפתאומיותה }\מקור{[רצי״ה א״ש ב הערה 3]}\צהגדרה{. }\\\הגדרה{ע׳ בנספחות, מדור מחקרים, אור, זיו, ברק. }

\ערך{ברירות }\הגדרה{- הזיכוך המחשבי והמעשי }\מקור{[ר״מ קלח]}\צהגדרה{. }

\ערך{ברית }\הגדרה{- קשר שכלי, נמוסי או טבעי, בין שני נושאים }\מקור{[עפ״י ע״ר א שפד]}\צהגדרה{. }\\\הגדרה{ע״ע אמונה בברית. ע׳ במדור פסוקים ובטויי חז״ל, זכירת הברית. }

\ערך{ברית }\הגדרה{- }\משנה{שורש ברית, וכריתת ברית במובן המוסרי }\הגדרה{- שהענין החיובי }\הגדרה{ואידיאלי\mycircle{°}}\הגדרה{, הנובע מתמצית }\הגדרה{המוסר\mycircle{°}}\הגדרה{ היותר נעלה ונשגב, יהיה מוטבע עמוק וחזק בכל טבע הלב והנפש, עד שלא יצטרך לא זירוז ולא חזוק ולא סייג לשמירתו, כי-אם יהיה מוחש וקבוע, כמו שטבועה, למשל, בלב אדם }\הגדרה{ישר\mycircle{°}}\הגדרה{ מניעת רציחה וכדומה מן השלילות הרעות שכבר הספיק כח המוסר הכללי לקלטן יפה }\מקור{[מ״ה ברית א (פנ׳ ה)]}\צהגדרה{. }

\ערך{ברית }\הגדרה{- }\משנה{יסוד הברית }\הגדרה{- הפעולות }\הגדרה{המוסריות\mycircle{°}}\הגדרה{, ביחוד הדתיות, המכוונות לכבד את }\הגדרה{ד׳\mycircle{°}}\הגדרה{ לפי }\הגדרה{ציור\mycircle{°}}\הגדרה{ }\הגדרה{המדמה\mycircle{°}}\הגדרה{. }\הגדרה{השגחת-ד׳\mycircle{°}}\הגדרה{, שימצאו דתות לכל אומה, המחזקות את הצדק בעולם. ושתמצא בהן אחת יסודית, שמחזקת מעוז הציורים האמיתיים, ומקשרתם אל השכל המעשי }\צהגדרהמודגשת{-}\הגדרה{ }\הגדרה{תורת-ישראל\mycircle{°}}\הגדרה{ המאירה באורה הפנימי בבית ישראל ומפיצה קרנים ג״כ לבני נח. והוא אות ברית בין אלקים ובין האדם בכללו, שמונע עכ״פ מהשחתה }\מקור{[עפ״י פנק׳ א קמד (קבצ׳ א נז)]}\צהגדרה{. }\\\הגדרה{ע׳ במדור מונחי קבלה ונסתר, קשת. }

\ערך{ברית }\הגדרה{- }\משנה{פגם הברית }\הגדרה{- ע׳ במדור הנטייה המינית.}

\ערך{ברית }\הגדרה{- הטבע של קדושת }\הגדרה{היהדות\mycircle{°}}\הגדרה{. עצם ההויה הנפשית והטבע הרוחני וגם הגופני, של }\הגדרה{הכלל\mycircle{°}}\הגדרה{ כולו ושל כל אחד ואחד מישראל. הטבע היהדותי במעשה, ברעיון, ברגש ובמחשבה, ברצון ובמציאות }\מקור{[עפ״י א״ש יז ד]}\צהגדרה{.}\\\משנה{קדושת הברית }\הגדרה{- אור הטבע הישראלי הנקי }\מקור{[א׳ מד]}\צהגדרה{.}\\\הגדרה{ע׳ במדור פסוקים ובטויי חז״ל, הפרת ברית. ע׳ במדור מלאכים ושדים, אליהו. }

\ערך{ברית }\הגדרה{- המושג העצמי של התוכן אשר }\הגדרה{לנצח\mycircle{°}}\הגדרה{ העומד למעלה מכל מושג מוסבר באיזה }\הגדרה{הגיון\mycircle{°}}\הגדרה{ מוגבל }\מקור{[ע״ר א רב (ע״א ב ט קנז)]}\צהגדרה{. }\\\מעוין{◊ }\הגדרה{הברית מיוסדת על תוכן קים, שאיננו נופל תחת שום שינוי }\מקור{[שם צז]}\צהגדרה{. }\\\ערך{ברית }\הגדרה{- }\משנה{ תוכן הברית }\הגדרה{- }\הגדרה{הזכרון\mycircle{°}}\הגדרה{ העולמי שאינו סובל שום הגבלה }\הגדרה{ציורית\mycircle{°}}\הגדרה{ כלל }\מקור{[עפ״י שם רב]}\צהגדרה{. }\\\הגדרה{ע׳ במדור מונחי קבלה ונסתר, רזא דברית.}

\ערך{ברית }\הגדרה{- }\משנה{(לעומת חסד\mycircle{°}) }\הגדרה{- מעלת בטחונו וחוזק מציאותו, של כל דבר נעלה }\הגדרה{בחיי-הרוח\mycircle{°}}\הגדרה{ המתפשט במציאות }\מקור{[עפ״י ע״ר א פג]}\צהגדרה{. }\\\הגדרה{}\הגדרה{הודאיות-המוחלטת\mycircle{°}}\הגדרה{ }\מקור{[עפ״י א״ק א רז, ע״ר א פג-פד, רב]}\צהגדרה{. }\\\הגדרה{ע׳ במדור מונחי קבלה ונסתר, אחרית, לעומת הראשית בחיי הרוח. ע׳ במדור פסוקים ובטויי חז״ל, ברית עולם. ושם, נתתי את תורתי בקרבם ועל לבם אכתבנה.}

\ערך{ברית }\הגדרה{- }\משנה{הברית שכרת ד׳ עם ישראל }\הגדרה{- שאי אפשר כלל שיהיה ח״ו כלל-ישראל נבדל ונפרד מקדושת }\הגדרה{שמו\mycircle{°}}\הגדרה{ הגדול ב״ה }\מקור{[מ״ש שיד (מא״ה ג רג)]}\צהגדרה{. }\\\משנה{כריתת ברית שכרת השי״ת עם ישראל }\הגדרה{- שאע״פ שהזמן פועל שינויים גדולים בעולם, ובני האדם הפועלים בזמן הם חפשים }\הגדרה{בבחירתם\mycircle{°}}\הגדרה{ והענין ארוך מאד, א״כ היה נראה לכאורה שאפשר הדבר שיצאו הדברים בכללם חוץ למטרת }\הגדרה{החכמה-העליונה\mycircle{°}}\הגדרה{ שכיון הבורא יתברך ח״ו, ע״י בני-אדם הפועלים שינויים רבים בבחירתם ע״י הזמן. ע״כ השי״ת }\הגדרה{בחר-בישראל\mycircle{°}}\הגדרה{ וצוה אותם לקדש חדשים ושנים. פי׳ שע״י כחן של ישראל ופעולתן בעצמם ובעולם, תהי׳ ערובה בטוחה שכל הדברים יחזרו אל תכליתם, והזמן יפעול פעולה }\הגדרה{מקודשת\mycircle{°}}\הגדרה{, היינו פעולה המגעת אל התכלית העליונה שכיון השי״ת ולא פעולה של }\הגדרה{חול\mycircle{°}}\הגדרה{ }\מקור{[מ״ש שנ]}\צהגדרה{. }\\\הגדרה{ע׳ במדור פסוקים ובטויי חז״ל, ברית עולם. ע׳ במדור מועדים וחגים, קידוש הזמנים.}

\ערך{ברכה}\myfootnote{ רקאנאטי עה״ת עקב: ״הברכה היא אצילות תוספת המשכה מאפיסת המחשבה שהיא מקור החיים״. ובשל״ה עה״ת, וזאת הברכה, תורה אור, ד״ה וכבר כתבתי: ״ענין ברכה הוא התפשטות בשפע רב תמיד נצחי״. \newline
\label{44}}\הגדרה{ - תוספת }\הגדרה{אור\mycircle{°}}\הגדרה{ ויתרון }\מקור{[עפ״י א״ק ב תקלד]}\צהגדרה{. }\\\הגדרה{תוספת חיים עצמיים מקוריים }\מקור{[עפ״י שם רצד]}\צהגדרה{. }\\\הגדרה{תוספת מעלה, }\הגדרה{הופעה\mycircle{°}}\הגדרה{ }\הגדרה{ועליה\mycircle{°}}\הגדרה{ }\מקור{[ע״ר א ריז]}\צהגדרה{. }\\\הגדרה{ההוספה התמידית במעלה }\מקור{[שם]}\צהגדרה{. }\\\הגדרה{התוספת התדירית, }\הגדרה{בשפעת\mycircle{°}}\הגדרה{ אור }\הגדרה{הקדש\mycircle{°}}\הגדרה{ וחיי }\הגדרה{האמת\mycircle{°}}\הגדרה{ }\מקור{[שם]}\צהגדרה{. }\\\הגדרה{התוספת וההגדלה }\מקור{[שם סב]}\צהגדרה{. }\\\הגדרה{שפעת }\הגדרה{חידוש\mycircle{°}}\הגדרה{ ומקור חיים }\מקור{[ע״א ד ט ק]}\צהגדרה{. }\\\הגדרה{}\הגדרה{שפעת\mycircle{°}}\הגדרה{ חיים טובים, נעימים }\הגדרה{ורעננים\mycircle{°}}\הגדרה{ }\מקור{[עפ״י א״ק ג קפח]}\צהגדרה{. }\\\הגדרה{שפע החיים, העז והעצמה }\מקור{[ע״ר א ריד]}\צהגדרה{. }\\\הגדרה{השפעה והשלמה}\מקור{ [עפ״י ע״א ג ב מט]}\צהגדרה{.}\\\הגדרה{ענין הַבְרָכָה ובְּרֵכַת-מים המשקה את הארץ }\מקור{[מא״ה ד קסד]}\צהגדרה{.}\\\הגדרה{ע״ע מתברך. ע״ע מברך. ע׳ במדור מונחי קבלה ונסתר, ״יחוד ברכה קדושה״. ע״ע קדושה.}\\\ערך{ברכה }\הגדרה{- }\משנה{הברכה הכללית }\הגדרה{- הברכה הנכנסת בעומק הפנימי של החיים, ברכת }\הגדרה{שלום\mycircle{°}}\הגדרה{ הפנימי שעל ידה ימצא האדם שהחיים המה טובים כשהם לעצמם, וממילא אין עמם מחסור כשהם מתמלאים עם הדרישות המעשיות }\צהגדרה{[}\צהגדרה{ע}\צהגדרה{״א ג ב רכו].}\\\ערך{ברכה }\הגדרה{- }\משנה{ברכה המושפעת מחסד\mycircle{°} אל עליון }\הגדרה{- }\הגדרה{האורות-העליונים\mycircle{°}}\הגדרה{ כשהם מופיעים על }\הגדרה{הנשמה\mycircle{°}}\הגדרה{, על }\הגדרה{נשמת-הכלל\mycircle{°}}\הגדרה{ ועל נשמת הפרט, המרחיבים את מהותה, מעצמים את הויתה, ומעלים אותה למרומי }\הגדרה{האושר\mycircle{°}}\הגדרה{ }\הגדרה{הנצחי\mycircle{°}}\הגדרה{ }\מקור{[עפ״י ע״ר א קנח]}\צהגדרה{. }\\\ערך{ברכה }\הגדרה{- }\משנה{הברכה היסודית של העולם }\הגדרה{- }\הגדרה{ההתעלות\mycircle{°}}\הגדרה{ התדירית של }\הגדרה{דעת-ד׳\mycircle{°}}\הגדרה{, }\הגדרה{בגדלה\mycircle{°}}\הגדרה{, }\הגדרה{ביפעתה\mycircle{°}}\הגדרה{ }\הגדרה{ובטהרתה\mycircle{°}}\הגדרה{ }\מקור{[ע״ר א נ]}\צהגדרה{. }\\\הגדרה{ע״ע מתברך. ע׳ ברוך.}

\משנה{ברכה }\צהגדרה{- }\מעוין{◊ }\צהגדרה{הזכרת }\צהגדרה{השם\mycircle{°}}\צהגדרה{ היא היסוד הפנימי השרשי של הברכה, }\צהגדרה{והמלכות\mycircle{°}}\צהגדרה{ היא מהותה העצמית הממשית}\צמקור{ [א״ל קיט].}

\ערך{ברכה לעומת הודאה }\הגדרה{- ע׳ בנספחות, מדור מחקרים. }

\ערך{״ברכה לצורך״ }\הגדרה{- }\מעוין{◊}\הגדרה{ }\הגדרה{ההופעות\mycircle{°}}\הגדרה{ }\הגדרה{העליונות\mycircle{°}}\הגדרה{, המופעות בנשמתנו מעולמי התעלומה, תפקידן הוא לרומם בנו את התוכן המהותי של כל עצמות חיינו אל רום }\הגדרה{הנצח\mycircle{°}}\הגדרה{, אל }\הגדרה{ההוד\mycircle{°}}\הגדרה{ האלהי הנשגב ברוממות קדשו. ואם יופיעו המון }\הגדרה{הארות\mycircle{°}}\הגדרה{, ורכוז לא יהיה להם בעצמיות מהות החיים שלנו, הרי הן לנו כאבודות. על כן אין לנו רשות לברך ברכה כי אם לצורך, וברכה שאינה צריכה, ומה גם ברכה לבטלה, הרי היא לנו נשיאת עון וחלול שם שמים. הבהקת האורה הרוחנית מתגברת היא בתעצומתה על ידי הברכה, מתאדרת היא ההארה הרוחנית בנשמתנו מעולם הנעלם, בא לידי גלוי ע״י בטוי הברכה המון רב של מחשבות רוממות וציורים נאדרים בקודש. ומתי הם לברכה באמת }\משנה{-}\הגדרה{ בזמן שיש להם רכוז בנטית החיים שלנו במהלך קדשם. ברכה לצורך }\משנה{-}\הגדרה{ צורך גבוה וצורך הדיוט }\מקור{[ע״ר א לא]}\צהגדרה{. }\\\הגדרה{ברכה שיצאה מפינו והשיגה את הרכוז בחיי המפעל ובהכרה המפורשה }\מקור{[עפ״י שם]}\צהגדרה{. }

\ערך{ברכת ד׳ }\הגדרה{- החיים המלאים }\הגדרה{צדק\mycircle{°}}\הגדרה{ ע״פ תכונתם ואופיים }\צהגדרה{[}\צהגדרה{ע}\צהגדרה{״א ב ט ער].}

\ערך{ברכת ד׳ }\הגדרה{- }\משנה{״לברך את שמך״ }\הגדרה{- ע׳ במדור פסוקים ובטויי חז״ל. }

\ערך{ברכת הדיוט הצריכה לגבוה}\צהגדרה{ - }\הגדרה{ע׳ במדור פסוקים ובטויי חז״ל. }

\ערך{״בשם אלהינו נדגול״ }\הגדרה{- ע׳ במדור מדרגות והערכות אישיותיות, דוגלים בשם ד׳.}

\ערך{״בשם כל ישראל״}\הגדרה{ - ע״ע ישראל, ״שם כל ישראל״.}

\ערך{בת }\הגדרה{- תיאור המין הנקבי שבכל נושא, ביחש להערך של המוליד והמחדש אותו, או המוציאו עכ״פ אל הפועל הגמור, כלומר ביחש ההורים }\הגדרה{האב\mycircle{°}}\הגדרה{ או }\הגדרה{האם\mycircle{°}}\הגדרה{, או ביחש להנושא המיני הזכרי, שגם הוא מיוחש אל ההורים כלומר }\הגדרה{הבן\mycircle{°}}\הגדרה{ }\מקור{[ר״מ קמ]}\צהגדרה{. }

\ערך{בת }\הגדרה{- מלת תואר, המדה }\מקור{[ר״מ קמ]}\צהגדרה{. }

\mylettertitle{ג}
\ערך{גא }\הגדרה{- תאור הרוממות הגשמית }\מקור{[עפ״י ר״מ קמא]}\צהגדרה{. }

\ערך{גאוה עליונה }\הגדרה{- }\משנה{הגאוה העליונה }\הגדרה{- }\מעוין{◊}\הגדרה{ כשאור }\הגדרה{הקודש-העליון\mycircle{°}}\הגדרה{ של המקור הראשי אשר }\הגדרה{להחכמה\mycircle{°}}\הגדרה{ הקדומה מופיע בהארת }\הגדרה{אורה\mycircle{°}}\הגדרה{ בתוך }\הגדרה{הגמול\mycircle{°}}\הגדרה{ העולמי, }\הגדרה{ומשפט\mycircle{°}}\הגדרה{ }\הגדרה{הצדק\mycircle{°}}\הגדרה{ מבהיק את אורו בבהירותו, }\משנה{הגאוה העליונה}\הגדרה{ מתגלה בעולם }\מקור{[ר״מ קמא]}\צהגדרה{. }\\\הגדרה{ע׳ במדור תיאורים אלהיים, גאות ד׳. }

\משנה{גאונות }\צהגדרה{- מקוריות יסודיות }\צמקור{[שי׳ א 67].}

\ערך{גאונות רוחנית }\הגדרה{- עצם כחות הנפש, קדושת הרגש וגדולת הכשרונות, בהתקבצם יחד, באיש אחד מיוחד ומצויין}\מקור{ [עפ״י א״י כד]}\צהגדרה{.}

\ערך{גאונים }\הגדרה{-}\משנה{ תקופת הגאונים }\הגדרה{- התקופה הגדולה שאחר חתימת }\הגדרה{התלמוד\mycircle{°}}\הגדרה{, שהיתה תקופה הרת עולם בחיי הרוח הפנימיים של אומתנו. הנוגעת לעיקרה ויסודה של חכמת ישראל. תורת }\הגדרה{ההלכה\mycircle{°}}\הגדרה{ }\הגדרה{והאגדה\mycircle{°}}\הגדרה{ של רבותינו הקדמונים בבבל, גאוני סורא ופומבדיתא, אשר מידם נמסר לנו המבצר הגדול לחומת אש דת }\הגדרה{התלמוד\mycircle{°}}\הגדרה{ הבבלי כולו }\צהגדרה{[מ״ר }\צמקור{315}\צהגדרה{].}

\ערך{גאות עולמים }\הגדרה{- גאות }\הגדרה{קדש\mycircle{°}}\הגדרה{ המתנשא }\הגדרה{מימות-עולם\mycircle{°}}\הגדרה{, המתעלה מכל }\הגדרה{עלוי\mycircle{°}}\הגדרה{ }\הגדרה{ופאר\mycircle{°}}\הגדרה{, למעלה מכל תכן של שאיפה היותר נאצלה השיכת לכל דבר נברא, גם בהתאחד הכל למטרתו היותר עליונה, שהיא תשוקת קדשם של עם ד׳ }\מקור{[עפ״י ע״ר א קנט]}\צהגדרה{. }\\\הגדרה{ע׳ במדור תיאורים אלהיים, גאות ד׳. ושם, מתנשא מימות עולם. ע׳ במדור שמות כינויים ותארים אלהיים, גאה.}

\ערך{גבהי גבוהים }\הגדרה{- מקור שרשו, חיי נשמתו (של האדם), אור חיי נשמת כל העולמים, אור }\הגדרה{אל-עליון\mycircle{°}}\הגדרה{ }\הגדרה{טובו\mycircle{°}}\הגדרה{ }\הגדרה{והדרו\mycircle{°}}\הגדרה{ }\מקור{[עפ״י א״ת י ב]}\צהגדרה{.}

\ערך{גבוה }\הגדרה{- }\משנה{(לעומת נמוך\mycircle{°}}\הגדרה{) - }\הגדרה{כללי\mycircle{°}}\הגדרה{ }\הגדרה{ומופשט\mycircle{°}}\הגדרה{ }\מקור{[עפ״י קובץ ג קז]}\צהגדרה{. }

\ערך{גבוה }\הגדרה{- }\משנה{(לעומת רם\mycircle{°}) }\הגדרה{- }\מעוין{◊}\הגדרה{ מצטיין ג״כ ביחש להמשך הדבר, ההולך וגבה מתחתית מצבו עד }\הגדרה{הרום-העליון\mycircle{°}}\הגדרה{ }\מקור{[ע״ר א קיב]}\צהגדרה{. }

\ערך{גבולים }\הגדרה{- זמנים ומקומות, מעשים ומחשבות, שאיפות ורצונות מוגבלים }\מקור{[ע״ר א קצג]}\צהגדרה{. }\\\הגדרה{ע״ע מגביל. }



\end{multicols}
\end{document}
