\mybookname{מילון הראיה}{מילון הראיה}
\mylettertitle{א}

\ערך{אב }\הגדרה{- הנושא המוליד, המחולל את התולדות }\מקור{[ר״מ קיז]}\צהגדרה{. }

\ערך{אב }\הגדרה{- המקים את הבית, המדריך את התולדות, המאיר את ארחות חייהם בהשפעתו הרוחנית }\מקור{[שם קיח]}\צהגדרה{. }

\ערך{אב }\הגדרה{- הרועה הנאמן. מדריך, העומד במעלות נפשו הרבה יותר גבוה מהמעלה של הצעירות של הבנים\mycircle{°}, הצריכה לקבל את השפעתו\mycircle{°} }\מקור{[עפ״י ע״ר ב סה]}\צהגדרה{.}

\ערך{אבנט}\הגדרה{ - מכוון בתור אמצעי, בין החלק העליון מקום הכחות הנפשיים, לבין החלק התחתון שבגוף, מקום הכחות הגופניים השפלים, שמורה אמנם על היחש החזק שיש לכחות השפלים אל הכחות הנפשיים, עד שהקדושה\mycircle{°} המעלה את הנטיות הנפשיות, פועלת להגביל יפה את סדרי הפעולות הטבעיות לצד המעלה והקדושה}\צהגדרה{ }\מקור{[ע״א ג ב ד]}\צהגדרה{.}

\הגדרה{ע״ע חגורה, יסוד הויתה.}

\ערך{אבר }\הגדרה{- }\משנה{אבר הכנף\mycircle{°}}\הגדרה{ - הכח הפנימי המניע את העפיפה }\מקור{[עפ״י ע״א ב ח יב]}\צהגדרה{. }

\משנה{אגרת רב שרירא גאון}\צהגדרה{ - מסמך\hebrewmakaf היסוד לסדר ההשתלשלות של כל התורה\hebrewmakaf שבעל\hebrewmakaf פה\mycircle{°}, שהוא כמגדל בנוי לתלפיות של היהדות, עליו תלוי אלף המגן כל שלטי הגבורים במלחמתה של תורה ואמתת דורותיה. בסיס האמונים לחתימת תורת אמת של ״חיי עולם הנטועים בתוכנו״\mycircle{°} מאז היותנו לעם ד׳ אלהינו, בקבלת מתנתה, ונמשכים לו באחרית חתימת התלמוד עם המשך דבריהם של הגאונים\mycircle{°} מוסרי עניניו }\צמקור{[ל״י ב (מהדורת בית אל תשס״ג) מט, נא].}

\משנה{ע״ע גאונים, תקופת הגאונים.}

\ערך{אד }\הגדרה{- ענן }\מקור{[ר״מ קיח]}\צהגדרה{. }

\ערך{אד }\הגדרה{- לישנא דתברא }\מקור{[ר״מ קיח]}\צהגדרה{. }

\ערך{״אדם״ }\הגדרה{- כנוי לגויה. ציור האדם השפל והנבזה על שם האדמה אשר לוקח משם, להוראת היות חומרו שפל מאד, כי הוא המדרגה הפחותה מן הדצח״מ שהוא הדומם }\מקור{[עפ״י ע״א יבמות סג.]}\צהגדרה{.}

\ערך{״אדם״ }\הגדרה{- כנוי לנפש. ציור מדרגה גבוהה, כמו שכתוב ״בצלם\mycircle{°} אלקים עשה את האדם״, היינו מצד נשמתו\mycircle{°} הרוממה, אשר היא נאצלת מתחת כסא\hebrewmakaf הכבוד\mycircle{°}, ועל שם ״אדמה לעליון״}\myfootnote{ ישעיה יד יד.\label{1}}\הגדרה{ }\מקור{[עפ״י ע״א יבמות סג.]}\צהגדרה{.}

\הגדרה{ע״ע ״אנוש״. ע״ע גבר. ע״ע איש.}

\ערך{אדם }\הגדרה{- נפש שכלית קשורה בחומר }\מקור{[ע״א ג ב קצט]}\צהגדרה{. }

\ערך{אדם }\הגדרה{- }\משנה{צורת\mycircle{°} האדם }\הגדרה{- המחשבה\hebrewmakaf העליונה\mycircle{°} העושה את האדם לאדם, התורה\mycircle{°} }\מקור{[ע״א ד ט יז]}\צהגדרה{. }

\משנה{צורת האדם הפנימית }\הגדרה{- שכלו ומוסרו }\מקור{[פנ׳ א]}\צהגדרה{. }

\ערך{אדם }\הגדרה{- }\משנה{כחו הרוחני }\הגדרה{- ע׳ במדור נפשיות, רוח, הכח הרוחני (של האדם).}

\ערך{אדם}\הגדרה{ - }\משנה{סגולת\mycircle{°} האדם}\myfootnote{ \textbf{כונס בקרבו את כל סגולת ההויה וכו׳} - ש״ק קובץ א קעב: ״האדם הוא תמצית מלאה שההויה כולה משתקפת בו״.\label{2}}\הגדרה{ - מציינת את רוממותו הבאה בעקב שפלותו - יצור מושפל עד עמקי החומר, ועם זה כונס בקרבו את כל סגולת ההויה הרוחנית המלאה. שדוקא בהשתפלותו אל המורד הארצי הרי הוא רוכס את כל ההויה מראש היש עד סופו}\צהגדרה{ }\מקור{[ע״א ד ט קה]}\צהגדרה{.}

\ערך{אדם }\הגדרה{- }\משנה{נשמת האדם בכל חגויה השונים}\הגדרה{ - פרח רז עולם (של) החיבור הנעלה שממנו מתגלה הכבדות הארצית\mycircle{°} עם השאיפה השמימית המנצחתה, של שפעת החיים היציריים המשתפלת דרגה אחר דרגה, עד שיוצרת את החמריות\mycircle{°}, עם המאור\hebrewmakaf העליון\mycircle{°}, השפעה של הוית הישות, הרוחני, האצילי, השכלי, והמוסרי, הקדוש והמצוחצח }\מקור{[עפ״י א״ק ב תקכד]}\צהגדרה{.}

\הגדרה{יצירה שבה מתגלה האור ההויתי בכל עזו ותקפו. הכח המרכזי, שההויה חודרת באורה כולה אל הויתו, ומשלמת את תכונתה על ידו }\מקור{[פנק׳ ג של]}\צהגדרה{.}

\ערך{אדם }\הגדרה{- }\משנה{תעודת האדם שנוצר בגללה}\myfootnote{ \textbf{תעודת האדם} - ע״ע ע״ר א קפא ד״ה מכלל ופרט וכלל. א״ק ב תקלד. ע׳ במדור מונחי קבלה ונסתר, ״תוספת״.\label{3}}\הגדרה{ - להוסיף אור\mycircle{°} רצוני\mycircle{°} עליון\mycircle{°} בעזוז\mycircle{°} החיים הפרטיים, להעלותם\mycircle{°} אל עלוי\mycircle{°} הכלל\mycircle{°}, ולהוסיף בכלל זיו\mycircle{°} צביוני חדש ע״י עושר הבא ממשפלים. (לעסוק בתורה\hebrewmakaf לשמה\mycircle{°}) }\מקור{[א״ק א מד]}\צהגדרה{. }

\הגדרה{להשלים את מלכות\hebrewmakaf שמים\mycircle{°} }\צהגדרה{<שהיא מופיעה בכל היש בהדר גאונה, והולכת היא ומשתפלת בהעולמים המעשיים, בתהומות מאד עמוקים, בירידות מאד חשוכים>.}\הגדרה{ וברצונו הטוב והאיתן של האדם, שיצא אל הפועל בהיותו מתעלה להיות אוחז במשטר האלהי בהמון עולמים, }\צהגדרה{<שבפליאות נוראות נתגלה ע״י הבהקת אורם של אדירי הקודש שבדורות הקדמונים, ויצא במלא יקרתו בהתגלות האלהית שבאור התורה\mycircle{°} ונשמת\hebrewmakaf ישראל\mycircle{°}, הפרטית והכללית, בין כל עמי הארץ>.}\הגדרה{ בכח חסון זה יתקן\mycircle{°} האדם ויעלה את החלק הירוד שבמלכות\mycircle{°} האצילות\hebrewmakaf האלהית\mycircle{°}, שירדה להיות מנהגת עולמי עד, בצורתם המוקצבה. ובזה יקשור נזר ועטרה למלכות שדי בכל העולמים כולם, והמגמה היצירתית תצא אל הפועל בכל יפעת אידיאליה, מראשית המחשבה עד סוף המעשה וכולה אומרת כבוד\mycircle{°} }\מקור{[עפ״י קובץ ח קעב]}\צהגדרה{.}

\משנה{מגמת היצירה האנושית }\הגדרה{- הנשמה החושבת, ההוגה דעה, המציירת\mycircle{°} ציורי קודש\mycircle{°} }\מקור{[א״ק ג שלד]}\צהגדרה{.}

\משנה{תעודת האדם }\הגדרה{- להיות משכיל ובן חורין, מתענג\hebrewmakaf על\hebrewmakaf ד׳\mycircle{°} ומתעלס בידיעת האמת ושמח בכבוד\mycircle{°} יוצר כל }\מקור{[עפ״י קבצ׳ א נח]}\צהגדרה{.}

\משנה{התפקיד האנושי }\הגדרה{- להיות איש חי מכיר ובעל השכלה }\מקור{[קובץ א קצה]}\צהגדרה{.}

\הגדרה{ע״ע חיי האדם. ע׳ במדור פסוקים ובטויי חז״ל, צלם אלהים, חותם צלם אלהים מוטבע באדם. ע״ע דמות האדם. ע׳ במדור אדם הראשון, תעודת האדם.}

\ערך{אדנות מוחלטה }\הגדרה{- היכולת\mycircle{°} החפשית\mycircle{°} האין\hebrewmakaf סופית, המצויה תמיד בפועל בגבורה\hebrewmakaf של\hebrewmakaf מעלה\mycircle{°}, היא האדנות המוחלטה והמלוכה האמיתית שהיא עומדת למעלה מכל שם\mycircle{°}, מכל בטוי ומכל קריאה\mycircle{°}, שהרי האפשרות אין לה קץ ותכלית, והיכולת אין לה גבול והגדרה. מלכות\hebrewmakaf אין\hebrewmakaf סוף\mycircle{°} במובן העליון, המלוכה\hebrewmakaf העליונה\mycircle{°} }\מקור{[ע״ר א מו]}\צהגדרה{. }

\הגדרה{ע׳ במדור שמות כינויים ותארים אלהיים, ״אדון עולם״}\myfootnote{ ע׳ עטרת ראש להרד״ב, שער ראש השנה סי׳ ה.\label{4}}\הגדרה{. }

\ערך{אדריכל }\הגדרה{- פועל (את) הבנין }\מקור{[א״ק ב שנ]}\צהגדרה{. }

\צהגדרה{ }

\ערך{אהבה }\הגדרה{-}\צהגדרה{ ההתיחסות הנאמנה, הישרה, ההגונה, המתאימה אל האמת המציאותית, מתוך שייכות נכונה וזיקה רצויה, הכרה מלאה ושלמה של המציאות, של הענין שהיא מתייחסת אליו }\צמקור{[עפ״י ל״י ב רלד].}

\צהגדרה{מצב גדלותי, רוחני, אינטלקטואלי הכרתי נשמתי, שייכות חיונית, קישור התדבקות והזדהות, מתוך חכמה אמיתית והכרה אמיתית }\צמקור{[עפ״י שי׳ 63, 4\hebrewmakaf 5].}

\ערך{אהבה }\הגדרה{- עדן החיים, התשוקה האלהית של העלאת נר החיים }\מקור{[מ״ר 24]}\צהגדרה{. }

\משנה{שלימות האהבה}\הגדרה{ - השמחה הגמורה ואור הנפש, שעמה כל טוב ואושר ובה כלולים נועם החכמה וההשגה ואהבתה }\מקור{[ע״א א ד לו]}\צהגדרה{.}

\מעוין{◊}\הגדרה{ האמונה\mycircle{°} }\משנה{והאהבה}\הגדרה{ הן עצם החיים בעוה״ז ובעוה״ב\mycircle{°} }\מקור{[א׳ סט]}\צהגדרה{. }

\ערך{אהבה }\הגדרה{- }\משנה{שמרי האהבה }\הגדרה{- ע״ע תאוות. }

\ערך{אהבה }\הגדרה{- }\משנה{עבודת אהבה }\הגדרה{- זהירות בפרטי כל מצות ודקדוקי תורה מכח השפעת כללות התורה הדבקה בלב בחוזק והכרה ברורה }\מקור{[עפ״י א״ת ג ג]}\צהגדרה{.}

\משנה{עבודת ד׳ וכל מעגל טוב מאהבה }\הגדרה{- מידיעת הטוב\mycircle{°} הגנוז בהם }\מקור{[עפ״י ע״ר א רפו]}\צהגדרה{. }

\הגדרה{מהכרה אמיתית אל הטוב והשלימות }\מקור{[ע״ר א שסז\hebrewmakaf ח (ע״א א ג לב)]}\צהגדרה{.}

\משנה{באהבה}\הגדרה{ - בדרך חפץ פנימי והכרה עצמית }\מקור{[ל״ה 55]}\צהגדרה{. }

\משנה{כח העבודה מאהבה}\הגדרה{ - }\מעוין{◊ }\הגדרה{אינו בא כי אם לפי מדת הידיעה הבאה בלימוד של קביעות ועשירות רבה במקצעות השונים של תורת המוסר\mycircle{°} והיראה\mycircle{°}, שאי אפשר כלל להמצא מבלעדי לימוד בסדר נכון, למגרס תחילה בבקיאות מלמטה למעלה, ואחר כך למסבר בעומק עיון ודעה שלמה }\מקור{[ל״ה 188]}\צהגדרה{.}

\הגדרה{ע״ע עבודה מאהבה, עבודת ד׳ מאהבה ותלמוד תורה\hebrewmakaf לשמה.}

\ערך{אהבה אלהית }\הגדרה{- }\משנה{האהבה האלהית העליונה, המבוסמת בבשמי הדעה העליונה }\הגדרה{- ההרגשה הנשמתית היותר חודרת ופנימית, אשר בכנסת\hebrewmakaf ישראל\mycircle{°} בכללותה, בנשמות אישיה היחידים, בחביון\hebrewmakaf עז\mycircle{°} נשמת כללותה, ובכל אשד הרוח המשתפך בכל פלגות תולדותיה }\מקור{[ע״א ד ט פח]}\צהגדרה{. }

\הגדרה{אהבת\hebrewmakaf ד׳\mycircle{°} אלהי\hebrewmakaf ישראל\mycircle{°}, עצם החיים (בישראל), נשמת\hebrewmakaf האומה\mycircle{°} ועצם חייה }\מקור{[אג׳ א מד]}\צהגדרה{.}

\משנה{אהבה אלהית }\הגדרה{- הנטיה היותר חפשית\mycircle{°} ונצחית\mycircle{°} של רוח החיים, שהופעתה באה מסקירת הגודל הבלתי מוקצב, של אור\mycircle{°} הקודש\mycircle{°} המקיף עולמי נצח ממעל לכל חק וקצב, שאור החסד\mycircle{°} הנאמן\mycircle{°} מתעלה שם, השופע ויורד בכל מלא חנו\mycircle{°}, ממעל לכל חק ומשפט\mycircle{°}, וכל פנות שהוא פונה הכל הוא רק לטובה\mycircle{°} ולברכה\mycircle{°} לאור ולחיים\mycircle{°}, וכל מעשה וכל תנועה מחוללת אך נועם\mycircle{°} והוד\mycircle{°} קודש }\מקור{[עפ״י א״י כט, ע״ר א יד]}\צהגדרה{.}

\ערך{האהבה העליונה }\צהגדרה{- אהבת\hebrewmakaf עולם\mycircle{°} ואהבה\hebrewmakaf רבה\mycircle{°}, אשר לישראל את ד׳ אלהיהם ואביהם\hebrewmakaf שבשמים\mycircle{°} מלך\hebrewmakaf עולמים, הבוחר בעמו ומלמדו ומדריכו }\צמקור{[ל״י א (מהדורת בית אל תשס״ב) צג]. }

\ערך{האהבה}\myfootnote{ \textbf{ההכרה האמיתית וכו׳ מכבוד\hebrewmakaf אל, הנשקף מהבריאה וכו׳ וכו׳ שומע קול ד׳ הקורא אליו וכו׳ ומרגיש שהוא וכו׳ שואף את חייו יחד עם מקור\hebrewmakaf החיים, וכל היצור כולו ניצב לו כאורגן שלם אדיר נחמד ואהוב, שהוא אחד מאבריו, המקבל מכולו ונותן לכולו, ויונק יחד עמו זיו חייו ממקור החיים} - ע׳ במדור שמות כינויים ותארים אלהיים, ״מלכנו״. ושם, ״אבינו״. ע״ע ע״ר א רמט, ד״ה ברוך. ושם, רפט ד״ה ברכנו. ושם ב ג ד״ה אמר ר׳ עקיבא. קבצ׳ ב קלז [87]. פנק׳ ב רד מט. ע״ע ״שמע״. ע׳ במדור פסוקים ובטויי חז״ל, ברוך שם כבוד מלכותו. (את ההבחנה בעניין האיר לי אהרן משה שיין).\label{5}}\ערך{ }\הגדרה{- }\צהגדרה{תכלית התעודה האנושית}\הגדרה{. ההכרה האמיתית כשמתגברת באדם כראוי, מכבוד\hebrewmakaf אל\mycircle{°} הכללי, הנשקף מכל הדר\mycircle{°} הבריאה וסדריה הגשמיים\mycircle{°} והרוחניים\mycircle{°}, בעבר, בהוה ובעתיד, }\צהגדרה{<שגם זה האחרון מוצץ הוא יפה למי שמבקש ודורש\hebrewmakaf את\hebrewmakaf אלהים\mycircle{°} באמת וחפץ שלם>}\הגדרה{ אותה ההכרה כשהיא מתעצמת יפה באדם, רק היא מטבעת עליו את חותמו האמיתי, את אופיו הטבעי להקרא בשם אדם\mycircle{°}. }\צהגדרה{רק אז הוא מרגיש שהוא חי חיים נצחיים ומכובדים. <הוא מכיר כי הדרכים שהחיים מתגלים בהם, לפי ערכנו ביחש מצבנו החומרי, שונים המה, ובכל השינויים ההווים והעתידים לבבו בוטח\hebrewmakaf בשם\hebrewmakaf ד׳\mycircle{°} אלהי עולם מחיה החיים וחי\hebrewmakaf העולמים\mycircle{°}> מצב נפש כזה כשהוא מתאים גם כן לכל סדרי החיים הפנימיים, הנפשיים והגופניים, חיי המשפחה והחברה, וכשהוא צועד בעוזו להיות גם כן מתפלש להיות המוסר הציבורי עומד על תילו ומכונו, אז הארץ מוכרחת להתמלא דעה, ותורת ד׳ היא נובעת ממעמקי הלב - כל }\הגדרה{אדם שומע קול ד׳ הקורא אליו ושש ושמח לעשות רצון קונו וחפץ צורו, שהוא צורו הפרטי וצור העולמים כולם; ומרגיש הוא אז, שהוא האדם, שואף את חייו יחד עם מקור\hebrewmakaf החיים\mycircle{°}, וכל היצור כולו ניצב לו כאורגן שלם אדיר נחמד ואהוב, שהוא אחד מאבריו, המקבל מכולו ונותן לכולו, ויונק יחד עמו זיו חייו ממקור החיים }\מקור{[עפ״י ל״ה 149]}\צהגדרה{.}

\משנה{מתק האהבה }\הגדרה{- רוחב הדעת\mycircle{°}, והנועם אשר לעדן\hebrewmakaf העליון\mycircle{°} }\מקור{[א״ק ג ראש דבר כט]}\צהגדרה{. }

\משנה{אהבת צור\hebrewmakaf העולמים\mycircle{°}}\הגדרה{ - זיו השכינה\mycircle{°}, הכרה שכלית והרגשית, ללכת בדרכי\hebrewmakaf ד׳\mycircle{°} באהבת אמת והכרה עמוקה פנימית\mycircle{°} }\מקור{[עפ״י ע״א ג ב נ]}\צהגדרה{. }

\משנה{זיקי אהבת אלהים }\הגדרה{- מציאת אור\hebrewmakaf ד׳\mycircle{°} בעומק רגש, בתוכן דעה }\מקור{[א״ק ג ריא]}\צהגדרה{. }

\הגדרה{ע״ע אהבת ד׳. ע״ע אהבת ד׳ העליונה. ע״ע יראת הגודל. }

\ערך{אהבה לעומת טוב}\הגדרה{ - ע׳ בנספחות, מדור מחקרים.}

\ערך{אהבה מינית }\הגדרה{- ע׳ במדור הנטייה המינית.}

\ערך{אהבה קדושה }\הגדרה{- }\משנה{האהבה הקדושה}\הגדרה{ - אהבת\hebrewmakaf ד׳\mycircle{°} וכל העולמים, אהבת כל היקום וכל היצור }\מקור{[א״ק ג רעט\hebrewmakaf רפ]}\צהגדרה{.}

\ערך{״אהבה רבה״ }\הגדרה{- ע׳ במדור פסוקים ובטויי חז״ל.}

\ערך{אהבת אור ד׳}\הגדרה{ - אהבת החיים של הצדיק\hebrewmakaf האמיתי\mycircle{°}, <שאיננה כלל אותה הנטיה הגסה של אהבת החיים המרופדת בשכרון של נטיות החומר הגסים המצוי אצל רוב הבריות, כי אם> אהבת חיקוי לחסד\hebrewmakaf עליון\mycircle{°} בעולמו, המתפשטת על פני כל היצור }\מקור{[קבצ׳ א קעד]}\צהגדרה{.}

\ערך{אהבת ד׳ }\הגדרה{- הרגשת השתוקקות תמיד לטוב\mycircle{°} ולאמת\mycircle{°} שהאדם מרגיש באמת בנקודת נשמתו\mycircle{°} הפנימית, <שהכל הוא בכלל טוב או בכלל אמת> }\מקור{[עפ״י קבצ׳ ב קלג (פנק׳ ד שסב)]}\צהגדרה{.}

\צהגדרה{השגת המושכלות המופשטים שבענינים האלהיים ואהבה אמיתית את הטוב\mycircle{°} את היושר\mycircle{°}, והרדיפה ללכת בדרכי\hebrewmakaf ד׳\mycircle{°} }\מקור{[עפ״י פנק׳ א תקסג]}\צהגדרה{.}

\הגדרה{(האהבה) מצד כבודו\mycircle{°} וחסדיו\mycircle{°} שעשה }\מקור{[מא״ה א פט]}\צהגדרה{. }

\צהגדרה{גלויה היסודי של האמונה\mycircle{°} הגדולה }\צמקור{[נ״ה יא].}

\מעוין{◊ }\משנה{אהבת ד׳}\הגדרה{ באה כשישים האדם לבבו להדמות לדרכי\mycircle{°} השי״ת\mycircle{°}, אז ע״י ההדמות תולד האהבה, וכפי רוב הדמיון יהי׳ רוב האהבה }\מקור{[ע״א א ב לו (ע״ר ב קכג)]}\צהגדרה{. }

\מעוין{◊ }\משנה{האהבה}\הגדרה{ באה מצד השלמות שבנמצאים, שמצדה הם כולם נמצאים באמיתת מציאותו יתברך }\מקור{[ע״א ג ב קעא]}\צהגדרה{.}

\משנה{אהבת ד׳ }\הגדרה{- דעת\hebrewmakaf ד׳\mycircle{°} <שבכללה היא גם כן דעת כל המציאות לאמתתה לכל סעיפיה, כפי היכולת לאדם: דעת הטבע לכל סעיפיו גיאוגרפיה והתכונה, הרפואה וחכמת הנפש, תכונות העמים וכל הנלוה להם, המביאים גם כן לאהבה\hebrewmakaf העליונה\mycircle{°} הזכה בכללות האנושיות> }\מקור{[עפ״י קבצ׳ ב קלא]}\צהגדרה{.}

\הגדרה{ע״ע אהבה אלהית. ע״ע יראת ד׳. ע״ע בנספחות, מדור מחקרים, אהבה ויראה. }

\ערך{אהבת ד׳ העליונה }\הגדרה{- אהבת השלמות המוחלטת והגמורה של סיבת\mycircle{°} הכל, מחולל כל ומחיה את כל }\מקור{[א״ק ב תמב]}\צהגדרה{. }

\משנה{אהבת ד׳ הבהירה }\הגדרה{- האהבה המרוממה והעדינה לאין\hebrewmakaf סוף\mycircle{°} }\מקור{[קובץ ה צה]}\צהגדרה{. }

\משנה{אהבת השי״ת}\myfootnote{ בהכרת האמת של המציאות האלהית מצד עצמה, המקור לשמו\hebrewmakaf הגדול  - ע׳ א״ק ד ת.\label{6}}\משנה{ }\הגדרה{- }\מעוין{◊}\הגדרה{ בהכרת האמת של המציאות האלהית מצד עצמה, המקור לשמו\hebrewmakaf הגדול\mycircle{°} ב״ה }\מקור{[קבצ׳ א קלז]}\צהגדרה{. }

\משנה{אור אהבת ד׳ }\הגדרה{- עדן\mycircle{°} החיים, מגמת החיים, עצם החיים, בהירות החיים, ומעין חיי החיים, עליון מכל הגה, מכל רצון והסברה, מכל שאיפה פנימית\mycircle{°}, ומכל הזרחה\mycircle{°} יפעתית\mycircle{°}, הכל בה, והכל ממנה }\מקור{[קובץ ו רמא]}\צהגדרה{.  }

\הגדרה{ע׳ במדור פסוקים ובטויי חז״ל, אהבה רבה. ע״ע אהבת שם ד׳. }

\משנה{״אהבת חנם״}\myfootnote{ ע׳ א״ק ג שכד.\label{7}}\הגדרה{ }\צהגדרה{- אהבה שגם כשיש במציאות דברים שכאילו מעכבים לה אעפ״כ תתגבר על כולם ותקבע\hebrewmakaf חנם }\צמקור{[ל״י א קיג]. }

\צהגדרה{אהבה שאינה תלויה בדבר <כאהבת ד׳ לישראל, ברית עולם> }\צמקור{[ק״ת נה].}

\ערך{״אהבת חסד״ }\הגדרה{- ע׳ במדור פסוקים ובטויי חז״ל.}

\ערך{״אהבת חסד״}\הגדרה{ - }\משנה{(לעומת ״תורת חיים״)}\הגדרה{ - ע׳ במדור פסוקים ובטויי חז״ל.}

\ערך{״אהבת עולם״ }\הגדרה{- ע׳ במדור פסוקים ובטויי חז״ל.}

\ערך{אהבת שם ד׳ }\הגדרה{- אהבת הלימוד והידיעה של מציאות השי״ת ודרכיו, וכל המכשירים המביאים לזה }\מקור{[קבצ׳ א קלו]}\צהגדרה{. }

\הגדרה{ע״ע אהבת ד׳. ע״ע אהבת ד׳ העליונה.}

\ערך{אהבת תורה }\הגדרה{- ע׳ במדור תורה.}

\ערך{אוביקטיבי }\הגדרה{- חיצוני}\myfootnote{ ע׳ בנספחות, מדור מחקרים, אוביקטיבי סוביקטיבי. ושם, חיצון, עולם חיצוני.\label{8}}\הגדרה{ }\מקור{[עפ״י א״ק ג צג]}\צהגדרה{.}

\צהגדרה{ע׳ במדור הכרה והשכלה והפכן, סוביקטיבי.}\הגדרה{ ע״ע חצון, עולם חיצון.}

\ערך{אוהל }\הגדרה{- שם בית הדירה, העלול להיות מוכן למסעות, המרשם בתוכן הרוחני העליון (של האדם) את העליות הנכספות. האוהל מסמן את היסוד המטלטל, את הצביון של ההכנה אשר לתנועה, שכונתה היא תמיד השתנות ועליה לצד האושר\hebrewmakaf העליון\mycircle{°}, לקראת הזיו\mycircle{°} של מעלה}\צהגדרה{ }\מקור{[ע״ר א מג]}\צהגדרה{.}

\הגדרה{ע״ע משכן.}

\ערך{״אוהל״ לעומת ״בית״ }\הגדרה{- ע׳ במדור מדרגות והערכות אישיותיות, }\משנה{״}\הגדרה{יושב בבית״ לעומת ״יושב אוהל״.}

\ערך{אויב }\הגדרה{- מי שהשנאה (אצלו) בכח לא בפועל }\מקור{[מ״ש שכז]}\צהגדרה{.}

\הגדרה{מבקש רעה בציורו ונטיתו הרוחנית }\מקור{[ע״ר א לד]}\צהגדרה{.}

\הגדרה{ע״ע קם להרע.}

\ערך{אולפן }\הגדרה{- לימוד <בתרגום> }\מקור{[ר״מ ב]}\צהגדרה{. }

\ערך{אומה }\הגדרה{- }\משנה{טבע האומה, הרוחני והחומרי }\הגדרה{- הטבע הפסיכולוגי של האומה, וטבע התולדה והמורשה של האבות והגזע, וטבע הגיאוגרפי של ארץ נחלתה }\מקור{[עפ״י קבצ׳ ב מה (ב״ר שכו\hebrewmakaf ז)]}\צהגדרה{. }

\ערך{אומה }\הגדרה{- }\משנה{צורת\mycircle{°} האומה }\הגדרה{- נשמתה ואורח חייה }\מקור{[עפ״י ע״א ד ה סא]}\צהגדרה{. }

\הגדרה{ע״ע עמים, הצד המהותי בחיי העמים. ע׳ במדור פסוקים ובטויי חז״ל, שבעים אומות.}

\ערך{אומה }\הגדרה{- }\משנה{האומה (הישראלית) כולה בצרופה הכללי }\הגדרה{- אֵם החיים שלנו. האופן הכללי\mycircle{°} של כל ישראל\mycircle{°} בתור גוש אחד, המחבר את כל האישים הפרטיים להיות לעם\mycircle{°} אחד, הכולל ג״כ את כל הדורות כולם בהערכה אחת }\מקור{[עפ״י א׳ עו, ע״ר ב פד]}\צהגדרה{. }

\הגדרה{ע״ע עם. ע״ע גוי. }

\ערך{אומה }\הגדרה{- }\משנה{רוח האומה היחידי}\myfootnote{ אולי צ״ל: יחודי. ע׳ בנספחות, מחקרים, כללים להבנה נכונה בקריאת כתבי הרב, יחידי.\label{9}}\הגדרה{ - השאיפה אל הטוב האלהי המונח בטבע נשמתה }\מקור{[א׳ נב]}\צהגדרה{.}

\הגדרה{ע״ע רוח ישראל. ע״ע רוח ד׳. ע׳ במדור תורה, תורה שבכתב, תורה שבכתב ברום תפארתה ותורה שבעל פה שניהם יחד.}

\ערך{אומה }\הגדרה{- }\משנה{שכינת האומה }\הגדרה{- רוח החיים של השאיפה האלהית המקושרת בתוכן הסגנון הצבורי של הצורה הלאומית }\מקור{[א׳ קו]}\צהגדרה{. }

\הגדרה{ע׳ במדור מונחי קבלה ונסתר, ״שושנה עליונה״. ע״ע אידיאה לאומית.}

\ערך{אומה }\הגדרה{- }\משנה{ישראל}\הגדרה{ - כנסת\hebrewmakaf ישראל\mycircle{°} המוגבלה בגבול נחלת ישראל }\מקור{[א׳ מב]}\צהגדרה{.}

\הגדרה{מקום מנוחתה של האידיאה\hebrewmakaf האלהית\mycircle{°} על המרחב ההיסתורי הכללי }\מקור{[א׳ קח]}\צהגדרה{.}

\ערך{אומה הישראלית}\הגדרה{ - }\משנה{התכלית הכללית של האומה הישראלית}\הגדרה{ - להודיע את שם\hebrewmakaf ד׳\mycircle{°} בעולם כולו ע״י מציאותה והנהגתה }\מקור{[ל״ה 119 (פנק׳ ב עו)]}\צהגדרה{.}

\הגדרה{חטיבה\mycircle{°} אחת בעולם, המצויינת בתקותה לעצמה לא בשביל עצמה, כ״א בשביל הטוב הכללי, שהוא חן השכל הטוב, המוסר\mycircle{°} והיושר\mycircle{°} האמיתי, שא״א שיבנה כ״א ע״י תיקון\hebrewmakaf עולם\hebrewmakaf במלכות\hebrewmakaf שדי\mycircle{°}}\צהגדרה{ }\מקור{[ע״ר א שפו (ע״א ב ט רצ)]}\צהגדרה{.}

\הגדרה{ע״ע ישראל, מהותם העצמית הנותנת להם את אופים המיוחד.}

\ערך{אומה כללית }\הגדרה{- תמצית\mycircle{°} של המין האנושי הפועלת עליו בלי הרף בעיבוד צורתו\mycircle{°} הרוחנית\mycircle{°} }\מקור{[קובץ ה קצו]}\צהגדרה{.}

\הגדרה{ע׳ במדור פסוקים ובטויי חז״ל, עם לבדד.}

\ערך{אוצר החיים }\הגדרה{- אורה\hebrewmakaf של\hebrewmakaf תורה\mycircle{°} במקורה }\מקור{[ע״ר א קמז]}\צהגדרה{. }

\הגדרה{הצד העליון של התורה, היקר בעצמו מכל החיים כולם, }\משנה{אוצר חיים}\הגדרה{ עליונים נעלים ונשאים מכל חיי זמן ועולם}\צהגדרה{ }\מקור{[ע״א ד ט ז]}\צהגדרה{.}

\הגדרה{ע׳ במדור מונחי קבלה ונסתר, ״אורייתא מבינה נפקת״. ע׳ במדור תורה, תורה, שורש התורה. }

\ערך{אוצר הטוב }\הגדרה{- מקור חי\hebrewmakaf העולמים\mycircle{°} }\מקור{[אג׳ א קי]}\צהגדרה{.}

\ערך{אוצר עליון}\הגדרה{ - }\משנה{האוצר העליון}\הגדרה{ - מקור הברכות\mycircle{°}}\צהגדרה{ }\מקור{[א״ק א קיט]}\צהגדרה{.}

\ערך{אור }\הגדרה{- יסוד ואומץ המשכת החיים }\מקור{[עפ״י א״ק ב רצז (ע״ט טז)]}\צהגדרה{. }

\הגדרה{כל יסוד החיים, חיי החיים, זיום\mycircle{°} ותפארתם\mycircle{°} }\מקור{[עפ״י ע״א ד יא יג]}\צהגדרה{. }

\הגדרה{כח הרוחניות\mycircle{°} של השכל הגדול, של החפץ הכביר, של המרץ הנשגב\mycircle{°} }\מקור{[מ״ר 296 (קבצ׳ ב עא)]}\צהגדרה{. }

\משנה{האור הגדול הכללי}\צהגדרה{ - שלמות החיים ובריאותם הנמשכת ממקור אמתתם, המתגלה על ידי כל פרטיותם של דברי התורה, טיפולם וקליטתם, במלא כל הנפש ובכל תפוצות חדריה }\צמקור{[א״ל מג].}

\הגדרה{ע״ע אור החיים. ע׳ במדור מונחי קבלה ונסתר, אורות. }

\ערך{אור }\הגדרה{- }\משנה{האור בעצם }\הגדרה{- אור\hebrewmakaf חדש\mycircle{°} של תשובה\hebrewmakaf עליונה\mycircle{°}, המ״ט שערי\mycircle{°} בינה\mycircle{°} [}\צהגדרה{ח״פ לב:].}

\צהגדרה{גילוי אמיתת המציאות המשוכללת בהופעת\mycircle{°} הקרנת הזרחתה\mycircle{°} }\צמקור{[עפ״י פנק׳ א תרלו (ב״א ד יא)]. }

\הגדרה{ע׳ בנספחות, מדור מחקרים, אור, זוהר, זיו. ושם, אור, זיו, ברק. }

\ערך{אור }\הגדרה{- }\צמשנה{האור הנשגב }\צהגדרה{- החיים המלאים הלאומיים הממולאים מטל חיים ממלכתיים אלהיים ממשיים וחזיוניים, העומדים אחר כותלנו }\מקור{[פנק׳ ד ריז-ח]}\צהגדרה{.}

\ערך{אור }\הגדרה{- }\משנה{(לעומת כלי\mycircle{°}) }\הגדרה{- נשמתו הרוחנית של הכלי <שהוא לבושו המעשי החיצון> }\מקור{[עפ״י א׳ קנח]}\צהגדרה{.}

\הגדרה{התוכן (לעומת הסגנון)}\צהגדרה{ }\מקור{[ע״א ד יב ה]}\צהגדרה{.}

\הגדרה{החיים העצמיים של מחשבת ההויה (לעומת ההויה) }\מקור{[עפ״י ע״ר א כו, וא״ק ד ת (א״ה 1098)]}\צהגדרה{. }

\הגדרה{אצילות האלהות בתור נפש ההויה, (מבחינתה הפנימית), בדיבורים מצד הסתכלות השירית שברוח הקודש }\מקור{[עפ״י א״ק ב שמח]}\צהגדרה{. }

\ערך{אור }\משנה{וכלים }\צהגדרה{- משמעות עמוקה, תוכן רוחני\mycircle{°}. }\צהגדרהמודגשת{כלים }\צהגדרה{- הגילויים של האור }\צמקור{[פנק׳ א תרלז (שי׳ 6, 25)]. }

\ערך{אור }\הגדרה{- }\משנה{(לעומת חיים\mycircle{°}) }\הגדרה{- דעה\mycircle{°}, רוח\hebrewmakaf הבטה }\מקור{[עפ״י א׳ יא]}\צהגדרה{. }

\ערך{׳אור׳ לעומת ׳מאור׳ }\הגדרה{- ע׳ בנספחות, מדור מחקרים. }

\ערך{אור}\הגדרה{ - כללות הרגשה וידיעת מציאות. ערך ההשגה\mycircle{°} והרצון הגמור }\צהגדרה{<כי מה שלמעלה מההשגה האנושית אין לקרות כ״א בשם חושך\mycircle{°} מצד ההעלם, וכשיש העלם לחושך של מעלה, נגבל בגדר השגה ונעשה }\צהגדרהמודגשת{אור}\צהגדרה{> }\מקור{[עפ״י מא״ה ד כא-כב]}\צהגדרה{.}

\ערך{אור }\הגדרה{- ההרגשה הנפשית וההבנה של הידיעה }\מקור{[ע״א ב ט קכד]}\צהגדרה{.}

\ערך{אור }\הגדרה{- }\משנה{אוצר האור }\הגדרה{- חיי החיים העליונים, מקור כל החיים ושרש כל ההויות }\מקור{[ע״ר א סז]}\צהגדרה{. }

\ערך{אור }\הגדרה{- }\משנה{האור הפנימי (של המושג מאורו של אלקים\hebrewmakaf חיים, צור ישעינו, לגדולי המשיגים) }\הגדרה{- החיים האמיתיים שאין לנו מלה ליחסם, כמו שהם נמצאים במקור\hebrewmakaf החיים\mycircle{°} יתברך שמו }\מקור{[עפ״י ע״א ג ב נב]}\צהגדרה{. }

\ערך{אור אין סוף }\הגדרה{- ע׳ במדור מונחי קבלה ונסתר. או במדור שמות כינויים ותארים אלהיים.}

\ערך{אור אין סופי}\הגדרה{ - }\משנה{האור האין סופי}\הגדרה{ - ההארה האלהית\mycircle{°} המקיפה והממלאה את כל, את כל הנשמות\mycircle{°} ואת כל העולמים\mycircle{°} }\מקור{[פנק׳ א שצט]}\צהגדרה{.}

\ערך{אור ״אל עליון קונה שמים וארץ״}\myfootnote{ בראשית יד יט.\label{10}}\הגדרה{ - החפץ\hebrewmakaf האלהי\mycircle{°}, המהוה את היש כולו, המעמידו ומחייהו, הדוחפו לעילוייו בכל קומתו המעשיית והרוחנית מריש דרגין עד סופם. הנבואה\hebrewmakaf העליונה\mycircle{°} של פה אל פה אדבר בו, הנשפעת לנאמן\mycircle{°} בית\mycircle{°}, להקים עדות ביעקב לעולמי עולמים, לקומם תבל ומלאה, בנשמת ד׳ יוצר כל }\מקור{[עפ״י ע״א ד ט טז]}\צהגדרה{. }

\ערך{אור אלהי }\הגדרה{- אור\mycircle{°} האמת\mycircle{°} הצדק\mycircle{°} והדעת\mycircle{°} }\מקור{[ע״ה קכח]}\צהגדרה{. }

\הגדרה{זוהר\mycircle{°} גדול של שכל בהיר וחשק אדיר של רצון כביר מאד }\מקור{[א״ק ג רטז]}\צהגדרה{. }

\ערך{אור אלהי }\הגדרה{- }\משנה{האור האלהי }\הגדרה{- המגמה השעשועית\mycircle{°} הפנימית\mycircle{°} של היצירה כולה, המזריחה\mycircle{°} באור\mycircle{°} יפעתה\mycircle{°} על פני כל היקום, מחייה הפנימיים }\מקור{[א״ק ג קפח]}\צהגדרה{. }

\משנה{נועם אור אלוה נורא הוד }\הגדרה{- מקור הנעימות ומעין העדנים\mycircle{°}, אוצר ההופעות\mycircle{°} ומקור מקורות החיים }\מקור{[ר״מ עו]}\צהגדרה{. }

\משנה{האור האלהי }\הגדרה{-  מקור\hebrewmakaf החיים\mycircle{°} ומקור כל העדן ורוממות כל אושר\hebrewmakaf עליון\mycircle{°}. אור חיי\hebrewmakaf החיים\mycircle{°} }\מקור{[עפ״י קבצ׳ א רטז (פנק׳ א תקיח, ג״ר 124)]}\צהגדרה{. }

\הגדרה{חיי החיים }\מקור{[ע״א ג ב רכו]}\צהגדרה{.}

\הגדרה{מקור החיים והשמחה\mycircle{°} }\מקור{[ע״א ג ב צט]}\צהגדרה{.}

\משנה{מקור האור האלהי}\הגדרה{ - נחל עדנים שאין לו סוף, ומקור עדן נצחי לכל נשמת חיים, המהפך את הכל לאור\hebrewmakaf חיים\mycircle{°}. המאור הפנימי, הכח הכמוס האלהי שיש במגמת הוייתה של האומה בעולם, שהוא הסוד של כל ההויה כולה }\מקור{[עפ״י קבצ׳ א קעה]}\צהגדרה{.}

\משנה{אור אלהי עליון }\הגדרה{- המרחב של אין סוף לבהירות והשלמת חיי עולמים בעד הכל }\מקור{[ע״א ד ט סה]}\צהגדרה{. }

\הגדרה{ע״ע אור עליון. ע״ע אור ד׳.}\myfootnote{ \textbf{אור אלהי, אור אלהים, אור ד׳, אור עליון }- בין מושגים אלה התקשתי למצוא הבדל, מכל מקום חולקו ההגדרות למחלקות שונות על פי המונחים השונים.\label{11}}\הגדרה{ ע׳ במדור שמות כינויים ותארים אלהיים, אלהי, המקור האלהי. }

\ערך{אור אלהי }\הגדרה{- }\משנה{האור האלהי }\הגדרה{- נשמת\hebrewmakaf האומה\mycircle{°} השרשית }\מקור{[א׳ קנח]}\צהגדרה{. }

\הגדרה{הזיו\mycircle{°} הטהור\mycircle{°} הממלא נפשות טהורות }\מקור{[ע״א ג ב נ]}\צהגדרה{. }

\ערך{אור\mycircle{°} אלהים\mycircle{°} }\הגדרה{- תעודת ההויה, מקור הנשמות\mycircle{°}, מלא\hebrewmakaf כל\mycircle{°}, רוח ישראל\mycircle{°} המופשט }\מקור{[עפ״י א״ת יב א]}\צהגדרה{. }

\הגדרה{אור החיים היותר יפים, היותר טהורים\mycircle{°} היותר מאירים\mycircle{°} }\מקור{[ע״א ד ו מ]}\צהגדרה{. }

\ערך{אור ד׳\mycircle{°}}\הגדרה{ - העילוי\mycircle{°} העליון\mycircle{°}, שממעל למקור\hebrewmakaf החיים\mycircle{°}, יסוד המרחב העליון של הזוהר\mycircle{°} הבלתי מוגבל שכל עולמי\hebrewmakaf עולמים\mycircle{°} אינם כדאיים לו, שהוא מובדל מכל אורות עולמים, שכל תכונה של אורה בהם הרי היא מכוונת לראות על ידה גופים חשכים, שבעצמם אינם מערך מהות האורה, אבל האור בעצמו איננו דבר נראה, כי לא נתגלה בעולם לפי מדתו הכח הרואה את מהות האור. אמנם }\משנה{אור ד׳}\הגדרה{ במעלת הרחבת אצילות\mycircle{°} מקורו, הוא האור שאור נראה בו ועל ידו }\מקור{[עפ״י ע״ר א כא]}\צהגדרה{. }

\הגדרה{אור האורים, שאי\hebrewmakaf אפשר לנו לבטאו ואיננו יכול להתלבש באותיות של שום מבטא גם לא של שום רעיון }\מקור{[א׳ קלא]}\צהגדרה{. }

\הגדרה{מקור חיי\hebrewmakaf החיים\mycircle{°} ב״ה }\מקור{[ע״ר א קנה]}\צהגדרה{. }

\הגדרה{חיי\hebrewmakaf החיים\mycircle{°}, היסוד העליון מקור חיי אור העולמים\mycircle{°} }\מקור{[עפ״י א״ק ג צה]}\צהגדרה{. }

\הגדרה{יסוד כל היש, ויותר מכל היש באין קץ }\מקור{[קובץ א תתיא]}\צהגדרה{.}

\הגדרה{צרור\hebrewmakaf החיים\mycircle{°} }\מקור{[שם רמ]}\צהגדרה{. }

\הגדרה{אור האמת\mycircle{°} }\מקור{[עפ״י ע״א ב ט קנא]}\צהגדרה{. }

\הגדרה{אלהי עולם. הטוהר\mycircle{°}, הטוב\hebrewmakaf המוחלט\mycircle{°}, האמת המזהרת, הנצח\mycircle{°} בכל מלא הודו\mycircle{°} }\מקור{[עפ״י א״ק א קפב]}\צהגדרה{.}

\משנה{אור ד׳ מחולל כל }\הגדרה{- זוהר האמת, הוד\mycircle{°} אור\hebrewmakaf החיים\mycircle{°}, שבמקור\hebrewmakaf הקודש\mycircle{°} }\מקור{[עפ״י א״ק א ג (מ״ר 402)]}\צהגדרה{. }

\משנה{אור ד׳ וכבודו}\הגדרה{\mycircle{°} - הקודש\hebrewmakaf העליון }\מקור{[מ״ר 345]}\צהגדרה{. }

\משנה{אור ד׳ העליון }\הגדרה{- כולל הכל, ומקור הכל וחיי כל }\מקור{[ע״ר א רח]}\צהגדרה{. }

\הגדרה{ע״ע אור עליון. ע״ע אור אלהי.}\footref{11}

\ערך{אור ד׳ ממרומיו\mycircle{°}}\הגדרה{ - היש\hebrewmakaf העליון\mycircle{°}, הרוחניות\mycircle{°} והטוהר\mycircle{°} המעולה }\מקור{[עפ״י א״ק ג רפו]}\צהגדרה{. }

\משנה{אור ד׳ }\הגדרה{- אור פני המלך המתנשא לכל לראש מעל כל ענין העולמות }\מקור{[ע״ר א רפט]}\צהגדרה{.}

\הגדרה{הארת\mycircle{°} היש האמיתי וזיו\mycircle{°} החיים האלהיים }\מקור{[ע״א ד ט מז]}\צהגדרה{. }

\הגדרה{הקדושה השרשית העצמית, המצואה בפועל, הוד\mycircle{°} חיי הקודש\mycircle{°}, המרומם ונשא מכל שרעף ורעיון }\מקור{[עפ״י ע״ר א ט]}\צהגדרה{. }

\הגדרה{הנס\mycircle{°} המוחלט }\מקור{[ע״ר א מט]}\צהגדרה{.}

\הגדרה{האור העליון שהוא הרבה למעלה מן הטבעיות\mycircle{°}}\צהגדרה{ }\מקור{[ע״א ד ט מא]}\צהגדרה{.}

\הגדרה{טוב\hebrewmakaf העליון\mycircle{°} }\מקור{[ע״ר א שעב, רפט]}\צהגדרה{. }

\ערך{אור ד׳ המהוה הישות }\הגדרה{- זרוע\hebrewmakaf ד׳\mycircle{°} אשר נגלתה, יסוד ההשתלמות הבלתי פוסקת }\מקור{[עפ״י א״ק ב תקל]}\צהגדרה{. }

\משנה{אור ד׳ וכבודו }\הגדרה{- אמיתת הרצון\hebrewmakaf הכללי\mycircle{°} אשר בנשמת\mycircle{°} היקום כולו }\מקור{[שם ג לט]}\צהגדרה{. }

\משנה{אור ד׳}\ערך{ - המגמה האלהית היותר ברורה ותהומית לאין חקר }\מקור{[פנק׳ ג שלא]}\צהגדרה{.}

\ערך{נשמת\hebrewmakaf העולמים\mycircle{°} }\צהגדרה{[קבצ׳ ב קנה}\משנה{]}\צהגדרה{.}

\הגדרה{שפעת החיים הנובעים ושוטפים ממקור חיי העולמים }\מקור{[קבצ׳ א רכג (פנק׳ א תקכה)]}\צהגדרה{.}

\משנה{אור ד׳ בעולמו }\הגדרה{- אור השכינה\mycircle{°}, נשמת העולמים, הוד האידיאליות\mycircle{°} האלהית החיה בכל }\מקור{[א״ק ב שסח, א״ש יד ד]}\צהגדרה{. }

\הגדרה{אורו\hebrewmakaf של\hebrewmakaf משיח\mycircle{°} }\מקור{[א״ק ב תקסא]}\צהגדרה{. }

\משנה{אור ד׳ - }\הגדרה{גאולה\mycircle{°} רוחנית עליונה, נהירה אל ד׳\mycircle{°} ואל טובו }\מקור{[א׳ צא]}\צהגדרה{.}

\הגדרה{״אורן של ישראל״, רוח\hebrewmakaf ה׳\mycircle{°} השורה על כלל\mycircle{°}\hebrewmakaf ישראל ותפארתם\mycircle{°} הכללית }\מקור{[ע״א א ד לג]}\צהגדרה{. }

\הגדרה{אור\hebrewmakaf תורה\mycircle{°}, אור\hebrewmakaf חיים\mycircle{°} }\מקור{[קובץ ו קפח]}\צהגדרה{.}

\משנה{אור ד׳ אשר באומה\mycircle{°}}\הגדרה{ - השראת\hebrewmakaf השכינה\mycircle{°} וקדושת\mycircle{°} התורה\mycircle{°} והמצוה\mycircle{°} }\מקור{[עפ״י ע״ר א קעג]}\צהגדרה{. }

\משנה{אור ד׳ בעולם }\הגדרה{- האמת\mycircle{°} והצדק\mycircle{°} של דעת הקודש\mycircle{°} }\מקור{[ע״ר א רו]}\צהגדרה{. }

\משנה{אור ד׳ וטובו }\הגדרה{- הטוב\mycircle{°} והצדק }\מקור{[ל״ה 118]}\צהגדרה{. }

\ערך{אור ד׳ בארץ }\הגדרה{- התורה האמיתית והשכל הצלול\mycircle{°} והבהיר\mycircle{°} }\מקור{[אג׳ א קיז]}\צהגדרה{. }

\משנה{אור ד׳ }\הגדרה{- קדושת התורה והיהדות הנאמנה, מורשה קהלת יעקב }\מקור{[מ״ר 367]}\צהגדרה{.}

\הגדרה{אור צדק עולמים אשר בתורת\hebrewmakaf חיים\mycircle{°} }\מקור{[מ״ר 366]}\צהגדרה{.}

\הגדרה{המוסר\mycircle{°} האלהי\mycircle{°} המתגלה בתורה\mycircle{°}, במסורת, בשכל\mycircle{°} וביושר\mycircle{°} }\מקור{[א״ק ג א]}\צהגדרה{. }

\הגדרה{צמאון\hebrewmakaf אלהי\mycircle{°} }\מקור{[עפ״י קובץ ז רח]}\צהגדרה{.}

\הגדרה{נועם\mycircle{°} הקודש }\מקור{[א״ק ב שי]}\צהגדרה{. }

\הגדרה{זיו\mycircle{°} אור החכמה והשגת האמת }\מקור{[פנ׳ ח]}\צהגדרה{. }

\ערך{אור ד׳ העליון }\הגדרה{- }\משנה{(לעומת אור\hebrewmakaf הדעת\hebrewmakaf התחתון\mycircle{°}) }\הגדרה{- אור\hebrewmakaf המקיף\mycircle{°} הגדול ורחב מרחבי שחקים }\מקור{[ע״א ד ט כט]}\צהגדרה{. }

\ערך{אור האמת }\הגדרה{- ע״ע אמת. }

\ערך{אור הגדול}\הגדרה{ -}\משנה{ האור הגדול}\הגדרה{ - התוכן האלהי\mycircle{°} שאי אפשר להגותו ולשערו\mycircle{°} }\מקור{[פנק׳ א שסב]}\צהגדרה{.}

\ערך{אור הגלוי }\הגדרה{- }\משנה{האור הגלוי}\הגדרה{\mycircle{°} - האור הנראה של אור\hebrewmakaf התורה\mycircle{°} וחכמת\hebrewmakaf ישראל\mycircle{°} כולה בקדושתה\mycircle{°} וטהרתה\mycircle{°}, בבינתה והכרתה, בכבודה וישרותה, בעושר סעיפיה בעומק הגיונותיה ובאומץ מגמותיה. תלמודה של תורה בכל הרחבתה והסתעפו(יו)תיה, בדעת וכשרון, ברגש חי וקדוש, וברצון אדיר וחסון\mycircle{°} לחיות את אותם החיים הטהורים והקדושים אשר האור המלא הזה מתאר אותם לפנינו. (אור\hebrewmakaf הקודש\hebrewmakaf החבוי\mycircle{°}) בהיותו מתקרב מאד אל מושגינו, אל צרכינו הזמניים, ואל מאויינו הלאומיים }\מקור{[מא״ה ג (מהדורת תשס״ד) קכג, קכה]}\צהגדרה{. }

\הגדרה{ע״ע אור קודש חבוי. }

\ערך{אור הדעת התחתון }\הגדרה{- }\משנה{(לעומת אור\hebrewmakaf ד׳\hebrewmakaf העליון\mycircle{°}) }\הגדרה{- אור\hebrewmakaf הפנימי\mycircle{°} המרוכז באוצר הדעת אשר לבן\hebrewmakaf אדם }\מקור{[ע״א ד ט כט]}\צהגדרה{. }

\ערך{אור ההשואה }\הגדרה{- ע׳ במדור מונחי קבלה ונסתר.  }

\ערך{אור החיים }\הגדרה{- מקור זיו\mycircle{°} החיים }\מקור{[עפ״י א״ק ב שכט]}\צהגדרה{. }

\הגדרה{שאיפת הגדלת כחותיהם }\מקור{[ע״א ד ו מא]}\צהגדרה{.}

\הגדרה{ע״ע אור. ע״ע אור חיים. }

\ערך{אור החיים }\הגדרה{- אור\hebrewmakaf ד׳\mycircle{°}, זיו\mycircle{°} החכמה\mycircle{°} האלהית, ואור פני מלך חוטר מגזע ישי }\מקור{[ע״א ב ט קנב]}\צהגדרה{. }

\צהגדרה{ }

\ערך{אור החיים העליונים }\הגדרה{- הנשגב\hebrewmakaf הכללי\mycircle{°} }\מקור{[א״ק ג רפ]}\צהגדרה{.}

\ערך{אור העליון}\הגדרה{ - היושר האמיתי, העצמיות והקדושה בבירורה }\מקור{[קובץ ו רסט]}\צהגדרה{.}

\ערך{אור העתיד}\הגדרה{ - הופעת\mycircle{°} כבוד\hebrewmakaf ד׳\mycircle{°} בעולם }\מקור{[א״ק ב קפב]}\צהגדרה{.}

\ערך{אור הפנימי}\הגדרה{ - }\משנה{האור הפנימי (של המושג מאורו של אלקים\hebrewmakaf חיים, צור ישעינו, לגדולי המשיגים)}\הגדרה{ - ע״ע אור, האור הפנימי.}

\ערך{אור השכינה }\הגדרה{- ע׳ במדור מונחי קבלה ונסתר, שכינה. }

\ערך{אור התורה }\הגדרה{- ע׳ במדור תורה.}

\משנה{אורה של תורה }\הגדרה{- ע׳ שם. }

\ערך{אור חדש }\הגדרה{- ע״ע אור קודש.}

\ערך{״אור חדש״ }\הגדרה{- אוצר חיים חדש ומלא רעננות\mycircle{°}, נשמות\hebrewmakaf חדשות\mycircle{°}, מלאות הופעת חיים גאיוניים, ממשלת עולמי\hebrewmakaf עולמים\mycircle{°}, הפורחת ועולה, המשחקת בכל עת לפני הדר\mycircle{°} אל\mycircle{°} עליון, האצולות מזיו\mycircle{°} החכמה\mycircle{°} והגבורה\mycircle{°} של מעלה }\מקור{[א״ק ג שסח]}\צהגדרה{. }

\ערך{אור חיים }\הגדרה{- אור קיום של הדר\mycircle{°} נצח נצחים }\מקור{[א״ק ג נח]}\צהגדרה{. }

\ערך{אור חיים}\myfootnote{ \textbf{אור חיים} - לבירור ההבחנה בין ״\textbf{אור}״ ל״\textbf{חיים}״, ע׳ הוד הקרח הנורא פרק א סי׳ ג, ד, ובעיקר בעמ׳ לו. \label{12}}\ערך{ }\הגדרה{- דעה\mycircle{°} ורצון\mycircle{°}, רוח\hebrewmakaf הבטה ומציאות\hebrewmakaf מלאה\mycircle{°} }\מקור{[עפ״י א׳ יא]}\צהגדרה{. }

\הגדרה{ע״ע אור החיים. ע׳ בנספחות, מדור מחקרים, אור וחיים.}

\ערך{אור חַי\hebrewmakaf העולמים\mycircle{°} }\הגדרה{- אור\hebrewmakaf עליון\mycircle{°}, מקור מקורות, חיי החיים }\מקור{[קובץ ה צט]}\צהגדרה{.}

\ערך{אור חֵי\hebrewmakaf העולמים\mycircle{°} }\הגדרה{- הענין\hebrewmakaf האלהי\mycircle{°}}\צהגדרה{ }\מקור{[א׳ סו]}\צהגדרה{.}

\הגדרה{הטוב\hebrewmakaf הכללי\mycircle{°}, הטוב האלהי השורה בעולמות\mycircle{°} כולם. נשמת\hebrewmakaf כל, האצילית, בהודה\mycircle{°} וקדושתה\mycircle{°} }\מקור{[עפ״י א״ש פרק ב]}\צהגדרה{. }

\הגדרה{החיים האלהיים ההולכים ושופעים, המחיים כל חי, השולחים אורם מרום גובהם עד שפל תחתיות ארץ, המתפשטים על אדם ועל בהמה יחד }\מקור{[עפ״י ע״ט י]}\צהגדרה{. }

\הגדרה{הרצון הכללי, הרצון העולמי }\מקור{[א״ק ג נ]}\צהגדרה{. }

\ערך{אור עליון }\הגדרה{- }\משנה{האור העליון }\הגדרה{- חייו ומקור שפעו, מחוללו ומהוהו של העולם }\מקור{[עפ״י ע״א ד ט נב]}\צהגדרה{. }

\הגדרה{יסוד הכל ומקורו }\מקור{[קובץ א תרלו]}\צהגדרה{.}

\הגדרה{הזיו\mycircle{°} האלהי\mycircle{°}, יוצר כל }\מקור{[א״ק א קצב]}\צהגדרה{. }

\משנה{האור העליון }\הגדרה{- זוהר\mycircle{°} הצחצחות\mycircle{°} של קדש\hebrewmakaf הקדשים\mycircle{°} }\מקור{[א״ק ג רח]}\צהגדרה{. }

\הגדרה{מקור מקוריות כל חיים וכל יש }\מקור{[קובץ ה נ]}\צהגדרה{. }

\הגדרה{מקור מקורות, חיי החיים, אור חי העולמים }\מקור{[שם צט]}\צהגדרה{. }

\הגדרה{מקור החיים והעונג }\מקור{[שם כה]}\צהגדרה{. }

\הגדרה{ע׳ במדור מונחי קבלה ונסתר, אור אין סוף. ע״ע אור אלהי. ע״ע אור ד׳.}\תקלה{11}

\ערך{אור עליון }\הגדרה{- }\משנה{האור העליון שבהויה}\הגדרה{ - העילוי\mycircle{°} הרוחני\mycircle{°} }\מקור{[קובץ א קסח]}\צהגדרה{.}

\הגדרה{בהירות חיים והויה מלאה זיו קדש\mycircle{°}. החיים העליונים\mycircle{°} ברום ערכם, בהופיעם ממכון הטוב\mycircle{°} והעלוי\mycircle{°} הנשגב\mycircle{°} }\מקור{[עפ״י ע״ר א קצג]}\צהגדרה{. }

\הגדרה{לשד חיי העולמים הזולף בחסדי אבות ממקור הברכה\mycircle{°},  מיסוד עולם שהוא קודם ונעלה מכל הגבלה\mycircle{°} וחוקיות מוטבעה }\מקור{[אג׳ ג נח]}\צהגדרה{. }

\משנה{האור העליון הבלתי מוגבל }\הגדרה{- המוסר\mycircle{°} האלהי המוחלט }\מקור{[א״ת ד ד]}\צהגדרה{.}

\ערך{אור קודש}\myfootnote{ \textbf{אור קודש} - בא״ק ג רפו הנוסח הוא: אור חדש.\label{13}}\ערך{ }\הגדרה{- טללי שפעת\mycircle{°} חכמה\mycircle{°} וציורים\mycircle{°} עליונים\mycircle{°}, נשגבים\mycircle{°} ונעימים\mycircle{°}, שהם משתפכים לתוך הנשמה\mycircle{°}, מודיעים לה זיוים\mycircle{°} עליונים, מנשאים אותה לרוממות\mycircle{°} מעלה\mycircle{°}, מקרבים לה את היש\hebrewmakaf העליון\mycircle{°}, את הרוחניות\mycircle{°} והטוהר\mycircle{°} המעולה, את אור\hebrewmakaf ד׳\hebrewmakaf ממרומיו\mycircle{°} }\מקור{[קובץ ו קנ]}\צהגדרה{.}

\ערך{אור קודש }\משנה{בו מלאה הנשמה }\הגדרה{- מאויי נצח והוד עליון ושאיפת חיים אצילית לאין חקר }\מקור{[ע״ר א סח]}\צהגדרה{.}

\הגדרה{ע״ע אור, האור בעצם.}

\ערך{אור קודש חבוי }\הגדרה{- האור\mycircle{°} הקדוש\mycircle{°} הגנוז, (ה)מקור האלהי\mycircle{°} של התורה\mycircle{°}, החוסן\mycircle{°} של הנבואה\mycircle{°}, סגולתה\mycircle{°} של רוח\hebrewmakaf הקודש\mycircle{°} והמחזה העליון\mycircle{°}. האור הגנוז של אור הנבואה ורוח הקודש. מעין החיים של שורש התורה האלהית ומכון כל חזון ומראה עליון. אור הקודש של חמדת עולמים הגנוזה, שורש התורה האלהית ומקור הנבואה ורוח הקודש המיוחד לישראל }\מקור{[עפ״י מא״ה ג (מהדורת תשס״ד) קכב\hebrewmakaf ה]}\צהגדרה{.}

\הגדרה{ע״ע אור הגלוי. ע׳ במדור אליליות ודתות, חושך חבוי.}

\ערך{אורגן }\הגדרה{- גוף חי, מסודר }\מקור{[רצי״ה א״ש ה הערה 1]}\צהגדרה{.}

\צהגדרה{ע׳ בנספחות, מדור מחקרים, ארגון, מאורגן.}

\ערך{אורגניסמוס }\הגדרה{- הקישור העצמי שיש להגוף עם הנשמה }\מקור{[קובץ ה קנה]}\צהגדרה{.}

\ערך{אורגניסמוס }\הגדרה{- }\משנה{(האורגניות הכללית שביצירה כולה) }\הגדרה{- קישור ושילוב\mycircle{°} החלקים זה בזה בכל צומח ובכל חי ובאדם. כל החלקים שביש (ה)צריכים זה לזה, ותהומות רבה והררי עד (ש)הם זה בזה משולבים ומצורפים }\מקור{[עפ״י א״ק ב תיז]}\צהגדרה{. }

\משנה{חק האורגניות}\הגדרה{ - היחש החי של השפעה ושל קבלה, (ה)הולך וחורז ומקיף את כל המצוי, את החומריות ואת הרוחניות, את הפעולות, המנהגים, ההרגשות ואת המחשבות}\צהגדרה{ }\מקור{[ע״א ד ו מא]}\צהגדרה{.}

\ערך{״אורה״ }\הגדרה{- ע׳ במדור פסוקים ובטויי חז״ל. }

\ערך{אורה }\הגדרה{- }\משנה{האורה }\הגדרה{- העילוי\mycircle{°} הרוחני\mycircle{°} }\מקור{[עפ״י קובץ א קסח]}\צהגדרה{.}

\ערך{אורה }\הגדרה{- }\משנה{אורה אלהית }\הגדרה{- שלמות הכל, ושלמות העדן\mycircle{°} של מקור הכל, שאין לנו שום מושג ממנה כי\hebrewmakaf אם מה שאנו חשים את מציאותה ומתענגים מזיוה בכל עומק נפש רוח ונשמה }\מקור{[עפ״י א״ק ג רצ, א׳ קיא]}\צהגדרה{. }

\הגדרה{הגודל והשיגוב האלהי }\מקור{[קובץ ו צ]}\צהגדרה{. }

\ערך{אורה }\הגדרה{- }\משנה{אורה אלהית }\הגדרה{- החוסן\mycircle{°} המלא, האור העליון, המון החיים ומקור יממיהם }\מקור{[עפ״י אוה״ק ב תמז]}\צהגדרה{.}

\הגדרה{האורה האנושית בכללה המתגלה באורן\hebrewmakaf של\hebrewmakaf ישראל\mycircle{°}}\צהגדרה{ }\מקור{[אג׳ א מג]}\צהגדרה{.}

\משנה{האורה הכללית }\הגדרה{- המשך החיים הנובע מהתשוקה העליונה והכללית של קרבת\hebrewmakaf אלהים\mycircle{°}}\צהגדרה{ }\מקור{[מ״ר 38]}\צהגדרה{.}

\משנה{אורה עליונה }\הגדרה{- אור\mycircle{°} חכמת\mycircle{°} כל עולמים\mycircle{°} }\מקור{[א׳ כט]}\צהגדרה{. }

\ערך{אורה }\הגדרה{- }\משנה{האורה הכללית }\הגדרה{- דעת היהדות\mycircle{°} בכל הדרת נשמתה הפנימית, העולה מעומקה של תורה\mycircle{°}, ומאוצר ההרגשה האלהית הבאה בהתמדת התלמוד והעיון בדברים שהם כבשונו של עולם, עם המכשירים המוסריים והעיוניים הדרושים לזה }\מקור{[עפ״י א״ה 913]}\צהגדרה{. }

\ערך{אורה אלהית }\הגדרה{- החפץ הציורי\mycircle{°} והמעשי, לשלטון של עילוי\mycircle{°} כל עז\mycircle{°} של צדק\mycircle{°} ואור\mycircle{°} }\מקור{[ע״ה קלב]}\צהגדרה{. }

\הגדרה{השלמות, המעשית והשכלית, הרגשית והתכונית, השלמות במילואה }\מקור{[א״ק ב שעה]}\צהגדרה{. }

\משנה{אורה }\הגדרה{- שלמות בכל תיקונה. חיי קודש\mycircle{°} וטוהר\mycircle{°} }\מקור{[עפ״י שם רפז, שכט]}\צהגדרה{. }

\משנה{האורה העליונה }\הגדרה{- גדולת החיים של הכרת האלהות האמיתית, הכוללת את כל תענוגי הרוח וכל העדנים עמם ברום עוזם}\צהגדרה{ }\מקור{[קובץ ז עו]}\צהגדרה{.}

\ערך{אורה חיצונית }\הגדרה{- נימוסים אנושיים טובים ויפים, תיקוני מדינה וממלכה נוחים ונעימים. התקדמות, סדרים, פאר\mycircle{°} ונעימות חיצונית הדורשים עמם חכמה מעשית רבה ללכת קוממיות ולהיות גוי איתן מלא חכמה מעשית וכליל יופי\mycircle{°} }\מקור{[עפ״י ע״א ג ב טז]}\צהגדרה{.}

\הגדרה{ע״ע אורה פנימית.}

\ערך{אורה פנימית }\הגדרה{- אורה\hebrewmakaf של\hebrewmakaf תורה\mycircle{°}, רוח\hebrewmakaf הקודש\mycircle{°} והנבואה\mycircle{°}, ששופע בישראל\mycircle{°} ביחוד ממקום בית\hebrewmakaf המקדש\mycircle{°} יצאה האורה\mycircle{°}, היא האורה\hebrewmakaf האלהית\mycircle{°} שמאירה בישראל לבדם ואין לזרים חלק בו. כח הקדושה\mycircle{°} המיוחדת שהיא מעלה את ישראל למצב רם ברוח קדושה ודעת\hebrewmakaf אלהים\mycircle{°} ודרכיו\mycircle{°}, שכולה אומרת כבוד\hebrewmakaf אלהים\mycircle{°} }\מקור{[עפ״י ע״א ג ב טז]}\צהגדרה{.}

\הגדרה{ע״ע אורה חיצונית.}

\ערך{אורה רוחנית }\הגדרה{- }\משנה{האורה הרוחנית }\הגדרה{- הגבורה\mycircle{°} הגמורה המנצחת את כל העולמים וכל כחותיהם }\מקור{[א׳ פד]}\צהגדרה{. }

\ערך{אורה שכלית }\הגדרה{- }\משנה{האורה השכלית}\הגדרה{ - הדעות הקבועות וארחות הדעה\mycircle{°} }\מקור{[ע״ר א קסח]}\צהגדרה{.}

\מעוין{◊ }\משנה{האורה השכלית}\הגדרה{ באה מרוב תורה ודעת, מהרבה שימוש\hebrewmakaf של\hebrewmakaf חכמים\mycircle{°},  ומהרבה דעת העולם והחיים }\מקור{[א״ק א רמ]}\צהגדרה{. }

\ערך{אורה של תורה }\הגדרה{- ע׳ במדור תורה.}

\ערך{אורה של תורה }\הגדרה{- ע׳ במדור תורה, אור התורה. }

\משנה{״אורות״ }\צהגדרה{- }\צמשנה{(עניינו של ספר אורות) }\צהגדרה{- שלמות גילוי אמתת קדושת עצמיותם של ישראל וערכם האלהי העליון הנצחי}\צמקור{ [א׳ קפז].}

\ערך{אורות הקדש }\הגדרה{- החיים בחיים (ה)עליונים ברום עולמים בצחצחות\mycircle{°} אידיאליהם\mycircle{°}, ספוגי קדש\hebrewmakaf קדשים\mycircle{°} }\מקור{[ע״ר א קפ]}\צהגדרה{. }

\הגדרה{ע״ע מוסר הקודש. ר׳ חכמת הקודש.}

\ערך{״אורך ימים״}\myfootnote{ תהילים כג ו, צג ה\label{14}}\הגדרה{ - כל הימים\mycircle{°} בעמדם בצביונם המלא, (בהיותם) בתוכן העליון\mycircle{°}, בקשר החיים עם הנצחיות\mycircle{°} האלהית\mycircle{°}, (ששם) אין הזמן\mycircle{°} עובר, הכל קיים, (כש)כל העשוי בהם עומד ומזהיר, ומשביע את הנשמה\mycircle{°} זיו\mycircle{°} וצחצחות\mycircle{°} ושובע נעימות. מלוא הימים, (כאשר) הצירוף של כל השיגוב, שנעשה מכל פרטי החיים בטוהר\mycircle{°} קדושתם\mycircle{°}, מתעלה בזיו ונהורא בהירה }\מקור{[עפ״י ע״ר ב עח]}\צהגדרה{. }

\משנה{באורך ימים}\צהגדרה{ כלולים הם כל הימים וכל ההשפעות\mycircle{°}, כל ההופעות\mycircle{°} וכל ההזרחות\mycircle{°}, כל המדעים וכל ההרגשות, כל צדדי ההסתכלות, וכל ארחות הדעה\mycircle{°} }\צמקור{[א״ק א סו]. }

\ערך{״אורך ימים״}\myfootnote{ ברכת קריאת שמע שבערבית.\label{15}}\ערך{ }\הגדרה{- }\משנה{(תאר למעלה שבמצוות כ״אורך ימינו״ לעומת ״חיינו״) }\הגדרה{- התועלת המגיעה בשלמות הנשמה\mycircle{°} במה שנעלם ואינו נרגש כלל אבל הוא מקנה לה קנין נשגב }\מקור{[ע״ר א תיב (פנק׳ ג רסח)]}\צהגדרה{. }

\הגדרה{ע״ע ״חיים״, תאר למעלה שבמצוות (לעומת אורך ימים).}

\ערך{אורך ימים }\הגדרה{- השלמת החיים היוצאת חוץ לגבול התעודה הפרטית. שמאריכים הם על המדה המוגבלת לפרטיותו ויספיק האדם תעודת החיים בעד העתיד בעד דור יבוא }\מקור{[עפ״י ע״א ג א נח (ח״פ מב.)]}\צהגדרה{. }

\הגדרה{הנביעה של האורה\mycircle{°} הרוחנית\mycircle{°} המתגברת ועולה על ידי הברכה\mycircle{°} הפנימית של הנשמה\mycircle{°}, שהיא באה ביחוד מהמקור של ההוקרה הפנימית והתוכית של הצד החיצוני המוכר בחכמה\mycircle{°}, שזהו התוכן הפרטי שבקניני הרוח, שהיא הכרה מפורדת לחלקים שונים, (העושה את) הימים מבורכים בפרטיותם }\מקור{[עפ״י שם ד יג ט]}\צהגדרה{.}

\הגדרה{ע״ע אורך שנים. }

\ערך{אורך שנים }\הגדרה{- האורה הכללית של השנים. תפיסת חיים העולה בצורה כללית, מפני הוקרת התוכן האצילי\mycircle{°} של החכמה\mycircle{°} (ה)מביאה נהרה אחדותית בנפש האדם, ודחיפת החיים היוצאת ממנה היא משאת נפש לתוכן ההכללה של החכמה בצורתה הבהירה והמקפת }\מקור{[עפ״י ע״א ד יג ט]}\צהגדרה{. }

\הגדרה{ע״ע אורך ימים.}

\ערך{אורן של צדיקים }\הגדרה{- האור\mycircle{°} הרענן\mycircle{°}, שהקודש\hebrewmakaf העליון\mycircle{°} חי במלא חפשו\mycircle{°} הנאדר בתוכו }\מקור{[א״ק ג ק]}\צהגדרה{. }

\ערך{אושר }\הגדרה{- }\משנה{האושר}\הגדרה{ - }\מעוין{◊ }\הגדרה{מקור המנוחה\mycircle{°} והבטחה\mycircle{°} }\מקור{[ע״ר ב סד]}\צהגדרה{.}

\ערך{אושר }\הגדרה{- }\משנה{(נשמתי) }\הגדרה{- תחושת הנשמה\mycircle{°} את עדונה הגדול, את זיו\mycircle{°} חייה המלא עדני עד, מהמון זרמי חיי עולם וישות אדירה, השוטפים בקרבה פנימה, מנחת דשן קרבת\mycircle{°} אלהים\hebrewmakaf חיים\mycircle{°}, ואור קדושתו המלאה על כל גדותיה }\מקור{[עפ״י ע״ר א קח\hebrewmakaf ט]}\צהגדרה{. }

\משנה{אושר נעלה }\הגדרה{- קדושת החיים במוסר\mycircle{°} מדות טובות ודעת\hebrewmakaf אלקים\mycircle{°} }\מקור{[פנ׳ פה]}\צהגדרה{. }

\משנה{אושר }\הגדרה{- שלמות אמיתית בדעת ובהנהגה }\מקור{[ע״א ג ב נה]}\צהגדרה{.}

\משנה{המצב המאושר }\הגדרה{- המצב הנפשי\mycircle{°}, שנועם\mycircle{°} ד׳, ועונג\mycircle{°} אהבה\mycircle{°} ושיקוק עליון\mycircle{°} מופיע בתוך הנשמה\mycircle{°} במצב של מנוחה וקביעות }\מקור{[א״ק ב תקח]}\צהגדרה{. }

\ערך{אושר }\הגדרה{- }\משנה{קץ האושר }\הגדרה{- שיהפך העונג\mycircle{°} היותר חמרי ויותר מלוכלך בניוול לקודש\mycircle{°} אידיאלי עליון }\מקור{[פנק׳ ג שלט]}\צהגדרה{.}

\ערך{אושר העולם }\הגדרה{- השמחה\mycircle{°} העדינה תולדתו של העדן\mycircle{°} האציל\mycircle{°} הבא מההארה\mycircle{°} של זיו\mycircle{°} הרעיונות של עומק האמונה\mycircle{°}, הרפודה באהבה\hebrewmakaf האלהית\mycircle{°} והדבקות\mycircle{°} הגדולה והרחבה שזיו שדי\mycircle{°} פרוש עליה }\מקור{[א״א 127]}\צהגדרה{. }

\משנה{תכלית האושר}\הגדרה{ - התכלית האידיאלית\mycircle{°} האחרונה, שהיא עצת\hebrewmakaf ד׳\mycircle{°} וברכתו\mycircle{°} לאדם ולעולם }\מקור{[ל״ה 158]}\צהגדרה{.}

\הגדרה{ע׳ במדור תיאורים אלהיים, שמחת ד׳ במעשיו.}

\ערך{אושר עליון\mycircle{°}}\הגדרה{ - שיהיה ״ד׳ אחד ושמו אחד״\mycircle{°} }\מקור{[א׳ קס]}\צהגדרה{. }

\הגדרה{ע״ע תֹּם. }

\ערך{״אות מן התורה״}\myfootnote{ ע׳ מגלה עמוקות על ואתחנן אופן קצז. בית עולמים קלט.: ד״ה תיקונא ״נשמות ישראל הם אותיות התורה״.\label{16}}\הגדרה{ - נשמה מישראל }\מקור{[עפ״י א״ת יא ב]}\צהגדרה{.}

\ערך{אותיות }\הגדרה{- }\משנה{כ״ב אתוון}\myfootnote{ \textbf{כ״ב אותיות} - בביאור ס׳ קהלת לרמ״ד וואלי, עמ׳ קעד ״א״ת, דהיינו כללות ההשפעה מא׳ ועד ת׳״. ושם קעו: ״בגין דאיהו עמודא דאמצעיתא, שהוא כלול מכל האורות, דהיינו מלת ״את״ שרומזת אל הכללות מא׳ ועד ת׳״. ובבאורו לאיכה, עמ׳ קמא: ״כי כ״ב אותיות הם רומזים אל הכללות כידוע״.\label{17}}\משנה{ }\הגדרה{- כלל ההנהגה}\צהגדרה{ }\מקור{[ג״ר 29]}\צהגדרה{. }

\הגדרה{סדרים לגילוי המאורות\mycircle{°}, }\צהגדרה{<ויש גילוי מצד השמיעה\mycircle{°} וגילוי מצד הראי׳\mycircle{°}, הרי ב׳ אותיות לכל ספי׳\mycircle{°}, ואחת הכוללת ענין חיבור האורות להאותיות בתחילת המחשבה, ואחת בסוף המעשה, הרי כ״ב> }\מקור{[פנק׳ ג צ]}\צהגדרה{.}

\ערך{אז }\הגדרה{- מורה על העבר, אבל לא רק בדרך פרוזי\mycircle{°}, ספור של מאורע שאינו מרותק עם רגשי הנפש והתפעלו(יו)תיה השיריות, כ״א באורח שירי, ומצב נפשי מרומם }\מקור{[ר״מ קיט]}\צהגדרה{. }

\ערך{אח }\הגדרה{- הקרוב היותר מקורב, המגובל בגבול האחדות }\מקור{[ר״מ קכ]}\צהגדרה{. }

\ערך{אחדות }\הגדרה{- }\משנה{האחדות }\הגדרה{- יחוד שלטון השי״ת\mycircle{°} בהתחלת הסבות\mycircle{°} הראשיות, המסבבות כל המון המעשים, שהן בערך השמים\mycircle{°}, ובגמר כל תכליתם, שהן בערך הארץ\mycircle{°}, ובכל האמצעים הרבים השונים ומסובכים, אשר ביניהם, שהם בערך ד׳\hebrewmakaf רוחות\mycircle{°} העולם, שכאילו מחברים את השמים עם הארץ }\מקור{[ע״ר א רמה]}\צהגדרה{.}

\משנה{אחדות ד׳}\צהגדרה{ - }\הגדרה{הדעה היותר עליונה של מושג האלהות\mycircle{°} }\מקור{[ל״ה 224]}\צהגדרה{. }

\משנה{אחדות הרוחנית}\הגדרה{ - שם\hebrewmakaf ד׳\mycircle{°} אחד\mycircle{°} השוכן בישראל\mycircle{°} }\מקור{[ע״א ב ביכורים לט]}\צהגדרה{.}

\משנה{אחדות }\הגדרה{- מגמה\mycircle{°} אחת עשירה ואדירה, כוללת כל, וברוכה בכל - אור\hebrewmakaf החיים\mycircle{°} היותר מאירים ויותר שלמים. המקור האחד של כל האידיאלים\mycircle{°} היותר נשאים, שאנחנו מוצאים בנפשנו פנימה, שכל זמן שהם עולים\mycircle{°} ומתבכרים, הם באים אליו }\מקור{[עפ״י ע״ה קנב]}\צהגדרה{. }

\הגדרה{התוכן של השאיפה היותר נאצלה השיכת לכל הנברא בהתאחד הכל למטרתו היותר עליונה }\מקור{[עפ״י ע״ר א קנט]}\צהגדרה{. }

\הגדרה{השאיפה ותגבורת חיל\mycircle{°} החיים בהתרכזות עשירה של חטיבה כללית, בכל, וביחוד במציאות הרוחנית\mycircle{°} והאידיאלית\mycircle{°}, המתלבשת גם כן יפה בהחמרית\mycircle{°} והריאלית בכל מלא עולמים\mycircle{°} כולם }\מקור{[עפ״י מ״ר 16]}\צהגדרה{. }

\הגדרה{הכלליות\mycircle{°} הקדושה\mycircle{°} ברוממות קודש קדשה }\מקור{[ר״מ קסח]}\צהגדרה{. }

\משנה{יסוד האחדות\hebrewmakaf העליונה\mycircle{°}}\הגדרה{ - המציאות\mycircle{°} ההויתית\mycircle{°} המתגלה כחטיבה אחת }\מקור{[עפ״י א״ה 916]}\צהגדרה{. }

\הגדרה{ע׳ במדור שמות כינויים ותארים אלהיים, ״אחד״ (תאר כלפי מעלה). ע״ע יחוד ד׳ בעולם. }

\ערך{אחדות }\הגדרה{- }\משנה{האחדות האלהית }\הגדרה{- הרוח\mycircle{°} האצילי\mycircle{°} המקיף את כל הנטיות כולן ומאחדם עם כל המון הכוחות הגשמיים והרוחניים למטרה מוסרית\mycircle{°} עליונה }\מקור{[עפ״י א״ק ג ש]}\צהגדרה{. }

\הגדרה{הטהרה\mycircle{°} העליונה של הנקודה האמונית\mycircle{°}}\צהגדרה{ }\מקור{[קבצ׳ ב נ]}\צהגדרה{.}

\ערך{אחדות אלהים }\הגדרה{- הטוב\hebrewmakaf העליון\mycircle{°}, הטוב המגלה שאין כל רע\mycircle{°} מצוי }\מקור{[פנק׳ ד עה]}\צהגדרה{.}

\ערך{אחדות }\הגדרה{- }\משנה{האחדות המופעה בעולם }\הגדרה{- מתבארת ע״י הקשור שיש בין המצוה\mycircle{°} התורית\mycircle{°} בכלל ההתגלות של דבר\hebrewmakaf ד׳\mycircle{°} ובין כל הסדר העולמי במערכי הטבע\mycircle{°} וכל מוסדי ההויה כולם. זהו תוכן המברר את ה}\משנה{אחדות האלהית\mycircle{°} בעולם}\הגדרה{, שהכל מתאים לתוכן אחד והכל מתקשר לאגודה אחדותית אחת }\מקור{[ע״ר א כה]}\צהגדרה{. }

\הגדרה{ע׳ במדור מונחי קבלה ונסתר, ״יחוד תחתון״. ע׳ בנספחות, מדור מחקרים, אחדות ויחוד. }

\ערך{אחדות }\הגדרה{- }\משנה{האחדות העולמית }\הגדרה{- הצד של השיווי שיש למצוא בהויה כולה, עד למעלה למעלה, לדימוי\hebrewmakaf הצורה\hebrewmakaf ליוצרה\mycircle{°}. הגשמיות והרוחניות, הציור והשכל, השפל והנישא, הם כולם תואמים, מתאחדים ומוקשים }\מקור{[עפ״י ע״ט סז]}\צהגדרה{. }

\ערך{אחדות אין סופית }\הגדרה{- ע׳ במדור מונחי קבלה ונסתר, יחוד עליון. ושם, אור אין סוף. }

\ערך{אחדות עליונה }\הגדרה{- המציאות ההוייתית (כ)חטיבה\mycircle{°} אחת }\מקור{[עפ״י פנק׳ ד רסז]}\צהגדרה{.}

\ערך{אחדות עליונה }\הגדרה{- הדעת\mycircle{°}, אוצר\hebrewmakaf החיים\mycircle{°} אשר בנשמת חי\hebrewmakaf העולמים\mycircle{°} }\מקור{[א״ק א קעד]}\צהגדרה{. }

\הגדרה{אושר\mycircle{°} ותענוג\mycircle{°}, למעלה מכל אחדות\mycircle{°} ומכל צחצחות, שורש נשמתן\mycircle{°} של צדיקים אשר עם המלך ישבו במלאכתו }\מקור{[שם ג מד]}\צהגדרה{. }

\הגדרה{הסוד העליון של האורה\hebrewmakaf האלהית\mycircle{°} בראשית\mycircle{°} התחלת הופעתה, (אשר אין) יכולת בידי בן אדם להשכיל בכחו בשכלו ובאורח מדעו וחושיו, לצייר\mycircle{°} בהויתו, איזה הערכה מסוד האחדות העליונה מלמעלה\mycircle{°} למטה\mycircle{°} }\מקור{[עפ״י ח״פ מה.]}\צהגדרה{. }

\ערך{אחדות מוחלטה }\הגדרה{- אור הפשטות העליונה, יסוד העדן\hebrewmakaf העליון\mycircle{°} }\מקור{[ר״מ קג]}\צהגדרה{. }

\הגדרה{מקור כל הקדושה\mycircle{°}, מכון כל העונג\mycircle{°} הנצחי, ובסיס כל השלמות ההולכת ומתעלה עדי\hebrewmakaf עד\mycircle{°}}\צהגדרה{ }\מקור{[פנק׳ א תו]}\צהגדרה{.}

\משנה{מלא האחדות המוחלטה }\הגדרה{- מקור חיי כל החיים אור\hebrewmakaf אין\hebrewmakaf סוף\mycircle{°}, אדון כל היש ומלא כל ההויה, מקור כל הרחמים ואב כל החסדים וכל גבורות נעם, כל פאר ויפעה וכל תפארת קדש, שומע תפלה\mycircle{°} ומאזין עתירה, בלא קץ ותכלית }\מקור{[ע״ר א סה]}\צהגדרה{.}

\הגדרה{ע׳ במדור מונחי קבלה ונסתר, תפארה.}

\ערך{אחדות שלמה }\הגדרה{- }\משנה{האחדות השלמה }\הגדרה{- קודש\mycircle{°} ד׳\mycircle{°}, קדוש ישראל\mycircle{°} }\מקור{[א״ק ג ס]}\צהגדרה{. }

\ערך{אחור }\הגדרה{- צד הטפל שבכל דבר הוא אחוריו }\מקור{[ע״א א ב מג, פנק׳ ג ער]}\צהגדרה{. }

\הגדרה{ע״ע פנים.}

\ערך{אחור }\הגדרה{- הצד החיצוני\mycircle{°}, שהוא הרבה כהה והרבה חלוש מהצד שהחיים הפנימיים של רוח\hebrewmakaf ד׳\mycircle{°} אשר במלא עולמו יונקים ממנו }\מקור{[עפ״י מ״ר 249]}\צהגדרה{. }

\הגדרה{ע״ע פנים.}

\ערך{אחור }\הגדרה{- }\משנה{ההנהגה האלהית שהיא לאחור }\הגדרה{- ההנהגה שהיא לצד ההתחסרות }\מקור{[עפ״י ע״ר ב סז]}\צהגדרה{. }

\הגדרה{ע״ע פנים.}

\ערך{אחור ופנים במציאות הרוחניות }\הגדרה{- ע״ע פנים ואחור במציאות הרוחניות. }

\ערך{אחוריים }\הגדרה{- השגה\mycircle{°} כללית סתומה, שאינה מתפרטת בפירוט האור בתכונה פרצופית\mycircle{°}, כ״א מתבזקת בהתבזקות כללית כמראה האחוריים שאין בו פירוט זיו\mycircle{°} הפנים בכל נתוח אבריו נושאי החושים העליונים }\מקור{[ר״מ קפ]}\צהגדרה{.}

\משנה{האחוריים של הפנים המחשביים }\הגדרה{- נחלים הנעשים מהאורים הגדולים המתפשטים מהפנים\hebrewmakaf המחשביים, נחלי חכמה מוקשבת, מחוללת אורה בינה והשכל לימודיים, המתלבשים בהמון התלמוד }\מקור{[עפ״י שם קפד]}\צהגדרה{. }

\הגדרה{ע״ע פנים, פנים מחשביים. ע׳ במדור משה, הראני נא את כבודך וגו׳. ע׳ במדור תיאורים אלהיים, אחוריים, ראיית אחוריים.}

\ערך{אחוריים }\הגדרה{- }\משנה{מראה אחוריים }\הגדרה{- ע׳ במדור תיאורים אלהיים.  }

\ערך{״אחסנתין״}\myfootnote{ באור הגר״א על משלי, ד ה, ח כא, יד יח, יט יד, כז כז. ממקור זוהר ח״ג רצא. (הערת מו״ר הרצב״י טאו).\label{18}}\ערך{ }\הגדרה{- קשר הקדושה\mycircle{°} שהוא ירושה מאבות שיש לאדם בענין עבודת\hebrewmakaf ד׳\mycircle{°} }\מקור{[עפ״י מא״ה ג קעה]}\צהגדרה{. }

\הגדרה{כח קדושת טבע הנפש שהיא מורשה לישראל }\מקור{[ה׳ רי]}\צהגדרה{. }

\הגדרה{ע״ע ״עטרין״. }

\ערך{אט }\הגדרה{- ההוראה להליכה בנחת }\מקור{[ר״מ קכ]}\צהגדרה{. }

\משנה{אידיאה }\צהגדרה{- מין נשמה\mycircle{°}, המשכת כח של צורה\mycircle{°} רוחנית\mycircle{°}, שיש לכל דבר שבעולם }\צמקור{[עפ״י א״ל רמד].}

\צהגדרה{הערך הרוחני האידיאלי של המציאות, פנימיות הדברים. מציאותיות רוחנית או הרוחניות המציאותית }\צמקור{[שי׳ ב 304, 303].}

\צהגדרה{מציאות התוכן הרוחני שהוא שורש המציאות }\צמקור{[שם 39, 6\hebrewmakaf 5].}

\צהגדרה{שורש רוחני\mycircle{°}, מציאותי, שעומד ביסוד המציאות של כל דבר. מציאות רוחנית, שהיא מקור המציאות החומרית\mycircle{°} }\צמקור{[עפ״י מה״ה ג רטז].}

\צמשנה{״למהלך האידיאות״ }\צהגדרה{- יסודי עולם רוחניים, מחשבתיים, מציאותיים, שמופיעים במהלך הדורות של קורא הדורות }\צמקור{[שם].}

\הגדרה{ר׳ בנספחות, מדור מחקרים, אידיאה.}

\ערך{אידיאה אלהית }\הגדרה{- הרעיון האלהי שההכנה אליו הנמצאת באיזה אופן גלוי או נסתר ישר או מעוות, בכל הלבבות של האנושיות, לכל פלגותיה, משפחותיה וגוייה, ומחוללת דתות ורגשי\hebrewmakaf אמונה שונים סדרים ונמוסים }\מקור{[עפ״י א׳ קב]}\צהגדרה{. }

\מעוין{◊}\הגדרה{ סגנון המחשבה של הרעיון הרוחני\mycircle{°} בהתבררותו ביותר, בגימור קויו הרשמיים ברוח האומנות אשר להסתוריה מבטא את ה}\משנה{אידיאה האלהית }\מקור{[עפ״י א׳ קב]}\צהגדרה{. }

\ערך{האידיאה האלהית בישראל }\הגדרה{- הנטיה הרוחנית של כנסת\hebrewmakaf ישראל\mycircle{°} שהיא אורה ונשמתה של הנטיה הלאומית המעשית שלה }\מקור{[עפ״י א׳ קד, קנח]}\צהגדרה{. }

\ערך{אידיאה האלהית המוחלטת }\הגדרה{- השכינה\hebrewmakaf העליונה\mycircle{°} }\מקור{[א׳ קיב]}\צהגדרה{. }

\הגדרה{הכשרון אל השכלול העליון והגמור המאיר את העולם כלו בכבודו }\מקור{[א׳ קה]}\צהגדרה{.}

\הגדרה{ע׳ בנספחות, מדור מחקרים, אידיאה אלהית ואידיאה לאומית. }

\ערך{אידיאה דתית }\הגדרה{- }\משנה{האידיאה הדתית }\הגדרה{- ההופעה\mycircle{°} האלהית המוקטנת המיוחדת לצד הפרטיות המבססת את המוסר האישי הפרטי, הדאגה לחיי\hebrewmakaf הנצח האישיים הפרטיים, הדיוק הפרטי של כל מעשה בודד המקושר ברוח הכללי }\מקור{[עפ״י א׳ קי]}\צהגדרה{. }

\הגדרה{התוכן המוסרי\mycircle{°} הבא בתור תולדה מהכרת האחדות האלהית בתור גורם למעמד מוסרי יפה לכל יחיד, המביאו לחיי נצח טובים. ולמעולים וחשובים ננעץ בו גם\hebrewmakaf כן התעוררות מוסרית אדירה לשמה, המתנוצצת מהעולם האלהי, <המתגלה יפה כשהשיקוע החומרי ודרישת ההנאה הגסה, אפילו בצדדיה היפים והעדינים, המצוי בעולם האלילי המסוגל לו, סר מהם> }\מקור{[עפ״י קבצ׳ ג קיד-קטו]}\צהגדרה{.}

\הגדרה{ע״ע דת.}

\ערך{אידיאה לאומית }\הגדרה{- תוכן הסגנון הצבורי של הצורה\mycircle{°} הלאומית }\מקור{[עפ״י א׳ קו]}\צהגדרה{. }

\הגדרה{הנטיה הקבוצית שבצורה הלאומית }\מקור{[עפ״י שם קב\hebrewmakaf ג]}\צהגדרה{. }

\מעוין{◊}\הגדרה{ סגנון החיים הסדרניים של החברה בהתבררותו ביותר, בגימור קויו הרשמיים ברוח האומנות אשר להסתוריה מבטא את ה}\משנה{אידיאה הלאומית }\מקור{[עפ״י שם קב]}\צהגדרה{. }

\הגדרה{ע׳ בנספחות, מדור מחקרים, אידיאה אלהית ואידיאה לאומית. }

\משנה{אידיאה לאומית}\myfootnote{ ע׳ בנספחות, מדור מחקרים, אידיאה אלהית ואידיאה לאומית. \label{19}}\הגדרה{ }\צהגדרה{- כנסת\hebrewmakaf ישראל\mycircle{°} }\צמקור{[שי׳ ב 235].}

\הגדרה{שכינת\hebrewmakaf האומה\mycircle{°} }\מקור{[א׳ קו]}\צהגדרה{.}

\הגדרה{כנסת\hebrewmakaf ישראל – שהיא הלבוש\mycircle{°} לתשוקה האלהית, לדבקות קדושה, ושמחת אור עליון – החודרת בעצמת חייה בכל פרט ופרט מישראל, ובכל מעשיו ותנועותיו, שיחיו ושיגיו, שאיפותיו וקניניו הפרטיים }\מקור{[עפ״י קובץ ו קמג]}\צהגדרה{. }

\ערך{אידיאה העליונה }\הגדרה{- }\משנה{האידיאה העליונה }\הגדרה{- הרצון\mycircle{°} (רצון ד׳\mycircle{°}) הקדוש\mycircle{°} והנשא }\מקור{[א״ק א קמב]}\צהגדרה{. }

\משנה{אידיאל }\צהגדרה{- המשך של האידיאה\mycircle{°} }\צמקור{[שי׳ 39, 6\hebrewmakaf 5]. }

\משנה{אידיאל }\צהגדרה{- הצדק\hebrewmakaf העולמי\mycircle{°}, שהוא גם הטוב\mycircle{°}, האור\mycircle{°} הפיוט וכו׳ וכו׳, נשוא מאויי לבה של הכללות, }\צמקור{<המושגים האלה וכאלה יוכלו לבאר את מובני }\צהגדרהמודגשת{האידיאלים\hebrewmakaf האלהיים\mycircle{°}}\צמקור{ ולהתיחס אליהם בהיותם מחוברים, נהגים ונאמרים אתם יחד, ולא אם אלה יתורגמו ויפורשו ויעתקו באלה בפני עצמם>}\צהגדרה{) }\צמקור{[ד״ל (מהדורת תשס״ה) אגרות יט, כ].}

\ערך{אידיאל }\הגדרה{- }\משנה{התוכן האידיאלי של העולם }\הגדרה{- סוד\mycircle{°} האלהי\mycircle{°} המוחלט שבהויה }\מקור{[א״א 18]}\צהגדרה{. }

\צהגדרה{ }

\ערך{אידיאל }\הגדרה{- }\משנה{האידיאל המוזער של העולם}\הגדרה{ - האידיאל הנופל בההויה המצומצמה }\מקור{[א״ק ב תנה]}\צהגדרה{.}

\ערך{ע׳ במדור מונחי קבלה ונסתר, זעיר.}

\ערך{אידיאל }\הגדרה{- }\משנה{האידיאל המלא של העולם }\הגדרה{- האידיאל המרומם האלהי\mycircle{°} במלא מילואו, יסוד העולם היותר עתיק\mycircle{°}, מציאות העולם היותר ממשית. שהמציאות הזעירה\mycircle{°} (של האידיאל המוזער, הנופל בההויה המצומצמה) רק מיניקת לשדו העליון היא מתקיימת ומתברכת\mycircle{°} }\מקור{[א״ק ב תנה]}\צהגדרה{.}

\ערך{ע׳ במדור מונחי קבלה ונסתר, עתיק.}

\ערך{אידיאל }\הגדרה{- }\משנה{התגלמות של אידיאל }\הגדרה{- ע״ע התגלמות. }

\ערך{אידיאל }\הגדרה{-}\משנה{ הגילוי האידיאלי של כל דבר נעלה בחיי\hebrewmakaf הרוח\mycircle{°} המתפשט במציאות }\הגדרה{- חפשו\mycircle{°} ויסוד מציאותו מתוך רצון מלא נדבה }\מקור{[עפ״י ע״ר א פג\hebrewmakaf ד]}\צהגדרה{. }

\הגדרה{ע״ע חסד, (לעומת ברית). ע״ע ברית, (לעומת חסד). }

\ערך{אידיאל לימודי}\הגדרה{ - ע״ע לימוד, האידיאל הלימודי.}

\ערך{אידיאליות }\הגדרה{- הרעיון\mycircle{°}, ההתרוממות\mycircle{°} הרוחנית\mycircle{°} }\מקור{[עפ״י ע״ר א ז]}\צהגדרה{. }

\ערך{אידיאליות }\הגדרה{- האהבה\mycircle{°} לדרכי\hebrewmakaf ד׳\mycircle{°} הנטועה בנפשו הלאומית (של ישראל\mycircle{°}) פנימה\mycircle{°}, והיא הולכת ועולה, פורחת ומתגדלת, לרגלי כל מה שמתגבר מקור\hebrewmakaf ישראל\mycircle{°} }\מקור{[ע״ה קלה]}\צהגדרה{. }

\ערך{אידיאליות העליונה }\הגדרה{- גילוי אור הקדש\mycircle{°} שבשאיפה הפנימית\mycircle{°} של הרוח\mycircle{°} }\מקור{[חד׳ תשס״ח קמה]}\צהגדרה{.}

\תערך{אידיאליות }\תהגדרה{- }\תמשנה{האידיאליות הנשמתית }\תהגדרה{- התשוקה העליונה והרוממה להתדבקות\mycircle{°} אלהית, תוכן האמונה\mycircle{°} }\תמקור{[עפ״י מ״ר 494]. }

\הגדרה{ע״ע תשוקה, התשוקה האידיאלית. }

\ערך{אידיאליות }\הגדרה{- }\משנה{האידיאליות }\הגדרה{- האצילות\mycircle{°} }\מקור{[א״ק א עט]}\צהגדרה{. }

\משנה{אידיאליות אלהית }\הגדרה{- האידיאליות\mycircle{°} הגמורה, כל התפארת\mycircle{°}, כל ההוד\mycircle{°} שביסוד המציאות }\מקור{[א״ק ג קפד]}\צהגדרה{. }

\ערך{אידיאלים }\הגדרה{- }\משנה{האידיאלים היותר נשאים, שהם הולכים ושואבים תמיד עלוי\mycircle{°} וצחצוח\mycircle{°} ממקור העצמיות\mycircle{°} העליונה\mycircle{°}}\הגדרה{ - פלגות נהרי אורותיה\mycircle{°} של עריגת הנשמה\mycircle{°} לעצמיות האלהית }\מקור{[עפ״י מ״ר 507]}\צהגדרה{. }

\משנה{האידיאלים האלהיים\mycircle{°}}\הגדרה{ - השמות\mycircle{°} האלהיים, דרכי\hebrewmakaf ד׳\mycircle{°}, חפציו\mycircle{°}, האצילות\mycircle{°}, הספירות\mycircle{°}, המדות\mycircle{°}, השבילים, הנתיבות, השערים\mycircle{°} והפרצופים\mycircle{°}, שמקצת תכנם האידיאלי\mycircle{°} חקוק וקבוע הוא גם כן בנפש\mycircle{°} האדם,}\myfootnote{ \textbf{האידיאלים האלהיים} - \textbf{שמקצת תכנם האידיאלי חקוק וקבוע הוא גם כן בנפש האדם, אשר עשהו האלהים ישר} - ע׳ שיחות על אהבה לר״י אברבנאל, מוסד ביאליק, תשמ״ג, עמ׳ 450\hebrewmakaf 440. ע״ע מלבי״ם, איוב לו א\hebrewmakaf ד. ״האידעען הנטועים בנפשו, הם אמתיות מונחלות וטבועות בנפש נפש ממקור מחצבה, ירושה לה מאלהי הרוחות בעודה בחביון עוזה, והם קדושים וטהורים אלהיים; האידעען קראם בשם דעי, ודעים. (כן אצל רש״ט גפן בממדים, הנבואה והאדמתנות עמ׳ 103); ייחס מלין אלה לאלהים, כי הוא הטביע דעים אלה בנפשו להשיג על פיהם את סודותיו ואמתיותיו; אצל ד׳ הדעים האלה הם בתמימות ובשלמות, וא״כ תמים דעים נמצא עמך, והוא האלהות הנמצא טבוע בשורש נפשך״. ע״ע בנספחות, מדור מחקרים, אידיאה.\label{20}}\הגדרה{ אשר עשהו האלהים ישר\mycircle{°} }\מקור{[ע״ה קמה]}\צהגדרה{. }

\משנה{אידיאלים כלליים }\הגדרה{- מטע ד׳\mycircle{°}. המגמות\mycircle{°} העולמיות כולן, אצילות\mycircle{°} האורות\hebrewmakaf העליונים\mycircle{°} ברוממות תעודתם, שנשמתה של האומה הישראלית, ונשמתו של כל יחיד מישראל בפנימיות\mycircle{°} מהותה, מאור\mycircle{°} זה היא אצולה }\מקור{[עפ״י ע״ר ב קנח]}\צהגדרה{.}

\הגדרה{הכח הנסתר של האצילות\mycircle{°} שבנשמת\hebrewmakaf האומה\mycircle{°} }\מקור{[ע״ה קנב]}\צהגדרה{.}

\משנה{האידיאלים הנשאים }\צהגדרה{-}\הגדרה{ המגמות\mycircle{°} האחרונות של מעשה התורה\mycircle{°} והמצוה\mycircle{°} }\מקור{[א״ק ג שכב]}\צהגדרה{.}

\הגדרה{ע׳ במדור מונחי קבלה ונסתר, תפארת, התפארת האלהית. ע׳ במדור שמות כינויים ותארים אלהיים, אלהות. ע״ע בנספחות, בסוף מדור מחקרים, שני מכתבים לברור דברים בשיטת הרב. ע״ע רוממות האידיאלים. }

\ערך{אילת השחר}\myfootnote{ תהילים כב א.\label{21}}\ערך{ }\הגדרה{- היופי העולמי שנאזר\mycircle{°} מגבורה וצמצומים }\מקור{[קובץ ה ר]}\צהגדרה{. }

\ערך{איש }\הגדרה{- התוכן הממולא בכח המפעל וההשפעה, ששפעותיו עשירות הנה בכל הארחים, לטוב\mycircle{°} ולרע\mycircle{°}, לבנין ולסתירה, רק הוא בכללות קיבוץ כל חלקיו, ממלא הוא תוכן של אישיות\mycircle{°}, צורה\mycircle{°} ממולא(ה) בטפוס שלם, העומד הכן לפעול ולהשפיע, לשכלל ולהשלים}\צהגדרה{ }\מקור{[ר״מ קכז]}\צהגדרה{.}

\ערך{איש }\משנה{- יסוד השלמתו }\הגדרה{- השכל\mycircle{°} העומד בראש והרגש\mycircle{°} עוזר על ידו }\מקור{[ע״א ג ב ריג]}\צהגדרה{.}

\הגדרה{ע״ע אשה. ע״ע אנשים. ע״ע גבר. ע״ע ״אדם״. ע״ע ״אנוש״.}

\ערך{״איש״ לעומת ״בעל״}\הגדרה{ - }\משנה{בעל }\הגדרה{- יקרא על שם הבעילה, כמו ״האי מטרא בעלא דארעא״}\myfootnote{ תענית ו: \label{22}}\הגדרה{, ו}\משנה{איש}\הגדרה{ - יקרא על שם המשפיע, יען כי הוא הנותן לה לחם לאכול ובגד ללבוש }\מקור{[ע״א יבמות סב:, סי׳ י]}\צהגדרה{.}

\ערך{אישיות }\הגדרה{- צורה\mycircle{°} ממולא(ה) בטפוס שלם, העומד הכן לפעול ולהשפיע, לשכלל ולהשלים }\מקור{[ר״מ קכז]}\צהגדרה{. }

\ערך{איתן העולם }\הגדרה{- ע׳ במדור מדרגות והערכות אישיותיות. }

\ערך{איתנות }\הגדרה{- הקביעות העזיזה }\מקור{[ל״ה 162]}\צהגדרה{. }

\ערך{איתניות ברוח }\הגדרה{- עז\mycircle{°} הרצון הכביר המתגלה באופן בריא וחזק עם כל תנועה נפשית וגופית }\מקור{[ע״א ד ו פג]}\צהגדרה{. }

\ערך{אך }\הגדרה{- מלת המיעוט\mycircle{°}, הצערת הנושא ממובנו הקדום, קציצת איזה סעיפים מתכונתו, <נרדף הוא עם התוכן של ההכאה הבא בתואר זה> }\מקור{[ר״מ קכא\hebrewmakaf ב]}\צהגדרה{. }

\ערך{אכילה }\הגדרה{- }\משנה{(לעומת טעימה\mycircle{°})}\הגדרה{ - תתיחס מצד התועלת, התועלת של תכלית האכילה לחזק הגוף ולהמשיך החיים, הבאה אחר העיכול, שע״ז יבא פעל אכל, מאוּכַּל}\צהגדרה{ }\מקור{[עפ״י ע״א ג א לה]}\צהגדרה{.}

\הגדרה{ע״ע מזון, לקיחת מזון. ע״ע מאכל.}

\ערך{אַל }\הגדרה{- הוראה שלילית }\מקור{[ר״מ קכב]}\צהגדרה{.}

\ערך{אֵל }\הגדרה{- הוראת הכח }\מקור{[ר״מ קכב]}\צהגדרה{. }

\ערך{אלהִי }\הגדרה{- }\משנה{הנקודה האלהית }\הגדרה{- מכון השלמות המוחלטה }\מקור{[ע״א ד ט קלד]}\צהגדרה{. }

\הגדרה{ע״ע אמונה, נקודת האמונה. }

\ערך{אלהִי }\הגדרה{- }\משנה{עידון אלהי }\הגדרה{- (העידון) של החכמה\mycircle{°} והמישרים\mycircle{°}, של הצדק והאורה\mycircle{°} העדינה הרוחנית\mycircle{°} }\מקור{[ע״א ד ו מ]}\צהגדרה{. }

\ערך{אלהים}\myfootnote{ ע׳ בהערה במדור שמות כינויים ותארים אלהיים, אלהים.\label{23}}\הגדרה{ - מנהיג ושולט }\מקור{[ע״ר א קיד]}\צהגדרה{. }

\הגדרה{כל כח שבנבראים שיש לו איזה השפעה, }\צהגדרה{<שמוכרחת היא להיות מוגבלת, מאחר שהכח המשפיע בעצמו הוא מוגבל, בבחינת התחלתו ובבחינת סופו, וכל מוגבל הרי מדת\hebrewmakaf הדין\mycircle{°} המצומצמת טבועה בתוכו> }\מקור{[עפ״י ע״ר ב פח]}\צהגדרה{.}

\ערך{אַלף }\הגדרה{- תרגום\mycircle{°} של למד }\מקור{[ר״מ קיז]}\צהגדרה{. }

\הגדרה{תרגום של למוד, <בעברית הוא ג״כ ממקור הארמי> ובא על המדרגה הירודה של הלמוד, הצד המתנוצץ אל התלמיד משפעתו של הרב, כתכונת האחורים לעומת הפנים }\מקור{[שם פג]}\צהגדרה{. }

\הגדרה{הלומד מאחר, והוא בערך העתקה ואחורים של הפנים המאירים באור השכל המקורי. ההתלמדות האולפנית, מדת התרגום, שהוא הכשר וחינוך המביא אל המעמד המקורי העתיד }\מקור{[עפ״י שם קיג]}\צהגדרה{. }

\הגדרה{הלימוד המתורגם, לימוד חינוכי, לימוד מכשיר, <שמביא באחריתו ללמוד מקורי, להבנת הלב> }\מקור{[עפ״י שם קיח]}\צהגדרה{. }

\הגדרה{ע׳ במדור אותיות, למ״ד. }

\ערך{אלף }\הגדרה{- שור\mycircle{°} }\מקור{[ר״מ פג]}\צהגדרה{. }

\ערך{אֵם }\הגדרה{- הוראת האמהות }\מקור{[ר״מ קכג]}\צהגדרה{. }

\הגדרה{המשפעת שפע חיים ליצור בהולדו }\מקור{[ע״א ד ו עג]}\צהגדרה{.}

\הגדרה{ע׳ במדור פסוקים ובטויי חז״ל.}

\ערך{אמונה}\myfootnote{ \textbf{אמונה} - הגדרות האמונה השונות חולקו לשתי מחלקות (שגבולן סומן ב ◊◊) : הקבוצה הראשונה עד ערך ׳תוכן האמונה׳ (ולא עד בכלל), מרכזת הגדרות למושג אמונה בסתם, ומקורותיה. ומערך ׳תוכן האמונה׳ ואילך הובאו הגדרות לבחינותיה השונות של האמונה, ענייניה, תכניה והשלכותיה.\textbf{גדר האמונה בכללה} - אמונות ודעות לרס״ג הקדמת המחבר, ״צריכים לבאר מה היא האמונה, ונאמר כי היא ענין עולה בלב לכל דבר ידוע בתכונה אשר הוא עליה, וכאשר תצא חמאת העיון יקבלנה השכל ויקיפנה ויכניסנה בלבבות ותמזג בהם, ויהיה בהם האדם מאמין בענין אשר הגיע אליו״. ובשומר אמונים הקדמון, ויכוח ראשון סי׳ לג (מיסוד המורה נבוכים, ח״א פ״נ) ״כי האמונה אינה האמירה בפה, כי אם התאמת הדבר במחשבת הלב והצטיירו בשכל״.\label{24}}\ערך{ }\הגדרה{- }\משנה{גדר האמונה בכללה }\הגדרה{- שאמיתת הענין הנודע קבועה בקרבו, לא מצד הידיעה לבדה כ״א מצד מנוחת הנפש הגמורה כשהוא מקבל אותה בקבלה שלמה, מבלי שיסתער בו מאומה נגד זה }\מקור{[ע״ר א שלז]}\צהגדרה{. }

\ערך{אמונה }\הגדרה{- דת\mycircle{°}}\צהגדרה{ }\מקור{[קבצ׳ ב נג]}\צהגדרה{.}

\הגדרה{ע״ע אמונה, נשמת האמונה}\צהגדרה{.}

\ערך{אמונה }\הגדרה{- }\משנה{מצות האמונה }\הגדרה{- מצוה מעשית, שיהי׳ הלב נמשך לאמונת התורה\mycircle{°} לעיון העבודה\mycircle{°} והמצות\mycircle{°}. שכל הדברים הברורים המאומתים שהודיענו השי״ת\mycircle{°} בתורתו\mycircle{°}, נראה להשתדל שכשם שהם מאומתים מצד אמתתם בשכל, כן יהיו הרהורי הלב מלאים מהם, והיינו על ידי שירבה המשא ומתן בענפיהם בעיון והשכל איש איש כפי רוחב לבו}\צהגדרה{ }\מקור{[פנק׳ ג כו (מא״ה ב רעג)]}\צהגדרה{.}

\משנה{צורת מצות האמונה האלהית }\הגדרה{- שהאלהים יתעלה, שהוא העושה כל המעשה הגדול הזה, אשר עינינו רואות אותו מסודר בעצה\mycircle{°} ובחכמה גדולה ועמוקה ומדוקדקת מאד, הוא שהוציא אותנו ממצרים\mycircle{°} מבית עבדים ונתן לנו את התורה\mycircle{°}, ולו שייכים ומיוחסים כל הדברים אשר הם שייכים ומיוחסים לאלהים <בין שמתבררים ע״פ אמתת הברור השכלי ובין שמתבררים ע״פ התורה והקבלה השלמה> }\מקור{[עפ״י ע״ר א שלו]}\צהגדרה{.}

\משנה{אמונה }\צהגדרה{- התכּונות התפיסת אל השכליות האלהית העליונה המוחלטת הכוללת\hebrewmakaf כל, המושכלת ומוכרת ומבוררת בכל מלוא הדעת הלבב והנפש והחיות }\צמקור{[עפ״י שי׳ ה 40]}\הגדרה{. }

\צמשנה{אמת האמונה}\myfootnote{ \textbf{האמונה }- \textbf{הידיעה ההכרה} - מוסיף רבנו שם באגרת: לכן בלשון התורה שבכתב: אברהם אבינו ״האמין בד׳״; ובלשון חז״ל בתושבע״פ: ״הכיר את בוראו״. ובהמשך דברי רבנו שם: ״הכרה ברורה של הסתכלות האמונה הזאת, נמשכת ומתבהרת לנו, בשלשלת הדורות, מתוך זכירת ימות עולם אל בינת שנות דור ודור, המפגישה אותנו עם הופעתנו המיוחדת הגדולה, שאין דומה ודוגמה לה בכל מערכת האנושיות, ושלשלת דורותיה והשפעתנו בתוכה״. וע׳ בפרקי הקיום הכשרון וההשפעה, שבס׳ כארי יתנשא.\label{25}}\צהגדרה{ - הידיעה ההכרה ההבנה הברורה הפשוטה שכל הנמצאים שבעולם כולם נמשכים ומתגלים לנו מתוך מקור המציאות, וכלשונו של הרמב״ם: ׳שכל הנמצאים הם מתוך אמתת מציאותו׳ }\צמקור{[אגרת רבינו, מי״ז בכסלו תשל״ז. מובא בשי׳ ג 241, הערה 89].}

\צמשנה{רוממות\hebrewmakaf רוח\hebrewmakaf אמונה ואמתת\hebrewmakaf דעת\hebrewmakaf עליון }\צהגדרה{- תקף בהירותה של הכרת עולם ומלאו, שלמותו ומקוריותו }\צמקור{[ל״י ב (מהדורת בית אל תשס״ז) תסד].}

\ערך{אמונה }\הגדרה{- }\משנה{עיקר מגמת הדרכת האמונה }\הגדרה{- לבסס את כח המדמה\mycircle{°} בטהרת\mycircle{°} השכל\mycircle{°} והמוסר\mycircle{°} }\מקור{[פנק׳ ד שסט]}\צהגדרה{.}

\ערך{אמונה }\הגדרה{- }\משנה{כח האמונה }\הגדרה{- (הכח) להקשיב יפה ולקבל ממה שנמסר }\מקור{[א״א 66]}\צהגדרה{. }

\הגדרה{להיות מוכן לקבל בתורת ידיעה את המסור לנו מאבותינו }\מקור{[שם 94]}\צהגדרה{.}

\הגדרה{הקבלה את מה שנאמר מפי גדולי עולם}\צהגדרה{ }\מקור{[ע״א ב ט רעב]}\צהגדרה{. }

\ערך{אמונה טבעית הסתכלותית }\הגדרה{- האמונה המופעה מהעולם }\מקור{[מ״ר 70]}\צהגדרה{. }

\ערך{אמונה ניסית מסורתית }\הגדרה{- האמונה המופעה מהתורה }\מקור{[מ״ר 70]}\צהגדרה{. }

\ערך{אמונה תוכית }\הגדרה{- האמונה המופעה ממעמקי הנשמה }\מקור{[מ״ר 70]}\צהגדרה{. }

\הגדרה{ע״ע תשובה טבעית. תשובה אמונית. תשובה שכלית. }

\ערך{אמונה }\הגדרה{- שלמות המדעים המושכלים שהם שורשי התורה\mycircle{°} }\מקור{[א״ה א 702 (מהדורת תשס״ב ח״ב 30)]}\צהגדרה{. }

\תערך{אמונה }\הגדרה{- }\תמשנה{רוח\mycircle{°} האמונה הקדושה\mycircle{°} }\הגדרה{- }\תהגדרה{הכרה עצמית\hebrewmakaf פנימית כ״טביעת\hebrewmakaf עין״\mycircle{°}, שהאדם מכיר את העולם מתוך עצמותו }\תמקור{[מ״ר 488]. }

\תערך{אמונה }\הגדרה{- }\תמשנה{כח האמונה }\הגדרה{-}\תהגדרה{ הצלם\hebrewmakaf האלהי\mycircle{°} המאיר מתוך תוכם של ברואיו של הקב״ה\mycircle{°}, והוא עצם מהותה של הנשמה, התשוקה להתדבקות\hebrewmakaf אלהית\mycircle{°} }\תמקור{[עפ״י מ״ר 487]. }

\תערך{אמונה }\הגדרה{- }\תמשנה{עומק האמונה }\הגדרה{-}\תהגדרה{ להתדבק\mycircle{°} בבורא כל העולמים\mycircle{°} ב״ה, בכל סדרי המחשבה והמעשה, והוא גילוי הכח השלם (של האמונה) }\תמקור{[מ״ר 488].}

\הגדרה{ע׳ במדור הכרה והשכלה והפכן, מחשבה, המחשבה היסודית. }

\משנה{אמונה }\צהגדרה{- הידיעה\hebrewmakaf ההכרה הבהירה והמלאה בד׳ אלהים, מקור חיי כל עולמים, הממלאה את כל חדרי הנפש, הרוח והנשמה, הכליות והלב והגוף כולו, ואשר לפיכך הם כולם ממולאים מתוכה אהבה\hebrewmakaf עליונה\mycircle{°} של דבקות\mycircle{°} חיונית לאבינו\hebrewmakaf שבשמים\mycircle{°}, אשר מלך בטרם כל יצור נברא - ואחריו }\צמקור{[נ״ה ט].}

\צהגדרה{ההשכלה וההכרה המבוררת בכל מלוא הדעת לבב ונפש וחיות, בהתכּונות התפיסה אל השכליות האלהית העליונה המוחלטת הכוללת כל }\צמקור{[עפ״י שי׳ ה 40 הערה 63].}

\צהגדרה{ענג רוממות ושבע נפש פנימי, (של) מלא ההכרה הברה והתמה ורווי עז הבטחון\mycircle{°} (ש)על ידי השקפת האחדות העליונה, שהנהגת ההשגחה\mycircle{°} העליונה מסבבת כל המעשים וכולם מלאים הם מזיו\mycircle{°} הטוב הכללי המקיף הכל וכוללם יחד, של ״מה דעבד רחמנא לטב עביד״, ״עושה שלום ובורא הכל״ }\צמקור{[עפ״י א״ל קצה].}

\צמשנה{כח האמונה }\צהגדרה{- ההכרה העליונה והדבקה באמתת צדיקו של עולם, עז מלכותו ומקור חיותו, אשר מראשית דרכו מאז ועד אחרית מופיעה המשכת כל המפעלים ושעשועי התולדה }\צמקור{[נ״ה יט].}

\צהגדרה{מציאות נפשית פנימית של זיקת האדם, הנברא בצלם\hebrewmakaf אלהים\mycircle{°}, אל מקורו, יוצרו, רבון\hebrewmakaf העולמים, שהיא ממלאה אותו כולו ומתוך\hebrewmakaf כך מתגלית בסידורי מעשיו ודורותיו היחידיים והצבוריים }\צמקור{[עפ״י ל״י ג קעז (מהדורת בית אל ב תשס״ג תה)].}

\צהגדרה{כִּווּנָה העצמי (שלא מתוך שטחי עניינים אחרים, של רגשיות או מוסריות או שכליות, של תועלתיות או חברתיות או הגיוניות) של תפיסת\hebrewmakaf עולם\hebrewmakaf ואדם שלמה וכוללת, בטבעיותו הרוחנית\mycircle{°} והחיונית }\צמקור{[עפ״י ל״י ב (מהדורת בית אל תשס״ג) תז].}

\ערך{אמונה }\הגדרה{- ע״ע בקשת אלהים. ע׳ במדור פסוקים ובטויי חז״ל, דרישת ד׳. }

\ערך{אמונה }\הגדרה{- ע׳ במדור הכרה והשכלה והפכן, הכרה אלהית. ע׳ במדור פסוקים ובטויי חז״ל, דעת אלהים.}

\ערך{אמונה }\הגדרה{- }\משנה{כח האמונה }\הגדרה{- שלמות תמימות טבעו הרוחני של האדם }\מקור{[עפ״י ע״א ג ב קנג]}\צהגדרה{.}

\הגדרה{בבחי׳ נפש, הרגשה, כח המקבל שבהוי׳ }\מקור{[קבצ׳ ג קכג]}\צהגדרה{.}

\ערך{אמונה }\הגדרה{- }\מעוין{◊}\הגדרה{ היסוד הטבעי לכל טובה\mycircle{°} ולכל מוסר\mycircle{°} עליון\mycircle{°} הנעוץ בעומק הטבע הישר\mycircle{°}, האנושי. מקור חייו הטבעיים, הרוחניים\mycircle{°} (של האדם) הנותנים לו שלות לב ושמחת נפש בעוה״ז ואחרית ותקוה לעוה״ב\mycircle{°} }\מקור{[עפ״י מ״ר 225]}\צהגדרה{.}

\הגדרה{תוכן הנפש היותר עדין, והמקור לכל התרבות האידיאלית שהאנושיות כולה עורגת אליה}\צהגדרה{ }\מקור{[פנק׳ ב קצד]}\צהגדרה{. }

\ערך{אמונה }\הגדרה{-}\משנה{ הנטיה האמונית }\הגדרה{- מקור הקדושה\mycircle{°} בעולם כולו }\מקור{[קבצ׳ א נד]}\צהגדרה{.}

\ערך{אמונה }\הגדרה{- }\משנה{נשמת האמונה }\הגדרה{- אור\hebrewmakaf החיים\mycircle{°} האלהיים שבתוך הדת\mycircle{°} }\מקור{[עפ״י קבצ׳ ב נג]}\צהגדרה{.}

\ערך{אמונה }\הגדרה{- }\משנה{עיקר האמונה }\הגדרה{- }\מעוין{◊}\הגדרה{ עיקר האמונה היא בגדולת\mycircle{°} שלמות אין\hebrewmakaf סוף\mycircle{°}. שכל מה שנכנס בתוך הלב הרי זה ניצוץ בטל לגמרי לגבי מה שראוי להיות משוער, ומה שראוי להיות משוער אינו עולה כלל בסוג של ביטול לגבי מה שהוא באמת }\מקור{[א׳ קכד]}\צהגדרה{. }

\משנה{שורש האמונה }\הגדרה{- להביע בפנימיות הנשמה את גדולת\hebrewmakaf א״ס }\מקור{[מ״ה אמונה לד]}\צהגדרה{. }

\הגדרה{מתוך החכמה\mycircle{°} הכמוסה בנשמה\mycircle{°}, שעל ידה היא מכירה בגודל האמת\hebrewmakaf האלהית\mycircle{°}, אלא שאינו יכול להוציא אל הפועל את הפרטים, מתוך כך הוא מתקשר הרבה להאמין בהפרטים שהוא מרגיש בפנימיות לבבו שהם הם מגלים את האמת הגדולה בכל האופנים שדרכה להגלות - במעשה, דיבור ומחשבה, ברגש, בדמיון, במזג ובנטיות נפשיות, ובמעוף חיים פנימי ועליון מכל הגיון לבב ומכל הקשבה מוגבלת}\צהגדרה{ }\מקור{[עפ״י קובץ א תתמט]}\צהגדרה{.}

\תמשנה{נקודת האמונה }\הגדרה{-}\תהגדרה{ יחס האדם להאין\hebrewmakaf סוף ברוך הוא, וכל העולמים כולם מתרכזים בנקודה אין\hebrewmakaf \hebrewmakaf \hebrewmakaf \hebrewmakaf \hebrewmakaf סופית זו }\תמקור{[מ״ר 487]. }

\הגדרה{ע״ע אלהי, הנקודה האלהית. ע׳ במדור מונחי קבלה ונסתר, לאשתאבא בגופא דמלכא.}

\ערך{אמונה }\הגדרה{- }\משנה{חוש האמונה }\הגדרה{- החוש הטבעי היסודי של הנשמה, שנובע באדם מנשמת חֵי\hebrewmakaf העולמים\mycircle{°} מנשמת כל היקום, כל היש }\מקור{[עפ״י א״א 112]}\צהגדרה{. }

\מעוין{◊ }\צמשנה{היחש של האמונה בטהרתו}\צהגדרה{ בא מפני עצמיות הטבע הפנימי של הנפש האנושית שהיא קשורה בקשר הוייתה בשורש חיי כל החיים, במקור כל ההויה }\צמקור{[ע״א ד ט קכה]. }

\ערך{אמונה}\myfootnote{ בע״א ד יא יג מציין הרב שתי בחינות באמונה. סגולית נשמתית - ״\textbf{האמונה האלהית העליונה} שאינה מתדמה כלל לשום רישום של ידיעה והכרה בעולם, כי היא הסגולה של כל יסוד החיים, של אורם, של חיי חייהם, של זיום ותפארתם. והסגולה הזאת מצד ערכה הפנימי, שאין לו שום הערכה בשום צד המתדמה לו בתואר חצוני, הוא ענין סגולי בישראל, לא מצד בחירת נפשם בפרט אלא מצד מחצב הקדושה וסגולת ירושת אבות שלהם״. ולאידך גיסא, הכרתית חיצונית - ״\textbf{כח האמונה מצד התגלותה החצונית}, שאפשר לו להגלות בפועל, בהכרה, ברגש, בביטוי ובמעשה, ששם אין האור הגנוז של שלמות חיי האמונה זורח״. בהתאם חולקו ההגדרות כאן לשתי קבוצות. \label{26}}\ערך{ }\הגדרה{- }\משנה{(בד׳) }\הגדרה{- שירת\hebrewmakaf החיים\mycircle{°}, שירת\mycircle{°} המציאות, שירת ההויה. שירת העולם העליונה\mycircle{°}}\myfootnote{ \textbf{האמונה היא שירת העולם העליונה} - ע׳ זוהר תרומה קלט: ״מ״י, רזא דעלמא עלאה איהו דהא מתמן נפקא שירותא לאתגליא רזא דמהימנותא״.\label{27}}\הגדרה{. ומקורה הוא הטבע האלהי\mycircle{°} שבעומק הנשמה\mycircle{°}, העונג\mycircle{°} של ההסתכלות הפנימית\mycircle{°} של אושר\mycircle{°} אין סוף }\צהגדרה{[א״א }\צמקור{66, 123}\צהגדרה{]. }

\הגדרה{שירת חיינו}\myfootnote{ \textbf{שירת חיינו} - ע״ע הסברת והגדרת האמונה בס׳ כארי יתנשא, פרק שלישי, האלהות, עמ׳ 30-32.\label{28}}\הגדרה{, (ה)מכון ששם האמת העליונה שוכנת בכל יפעתה\mycircle{°} והדרה\mycircle{°} }\מקור{[קבצ׳ ב קיב]}\צהגדרה{.}

\הגדרה{האמת הגדולה, התוכן העזיז, מלא הקודש, משך החיים האמתיים, וכל שיגוב ואידור במילואו }\מקור{[א״ק א עא]}\צהגדרה{.}

\משנה{האמונה העליונה }\הגדרה{- שירת העולם ואמת העולם }\מקור{[א׳ קכז]}\צהגדרה{. }

\משנה{הארת האמונה }\הגדרה{- הארה\mycircle{°} כללית למעלה מכל הערכים ובזה היא מבססת את הערכים כולם }\מקור{[שם קכה (א״א 68)]}\צהגדרה{. }

\משנה{אור האמונה }\הגדרה{- שורש כל הקדושות }\מקור{[א״א 141]}\צהגדרה{. }

\משנה{האמונה האלהית }\הגדרה{- המחשבה\mycircle{°} היותר עליונה\mycircle{°}, שמתוך גבהה היא משתפלת בכל השדרות גם היותר שפלות, ובכל דרגה ושדרה היא מתארת כפי ערכה }\צהגדרה{[מ״ה אמונה ג (א״א }\צמקור{7\hebrewmakaf 66}\צהגדרה{)]. }

\משנה{יסוד האמונה }\הגדרה{- התוכן המקיים את הצביון העליון הבלתי מדוד ושקול, הבלתי מצומצם ומתואר, בתוך כל מה שמתואר ומוגבל, המחיה את כל החוקים, בהתמשכם מראשית\mycircle{°} נביעתם עד אחרית\mycircle{°} המגמות\mycircle{°} כולן, עד אין קץ למורד }\מקור{[ע״ר א לה\hebrewmakaf ו]}\צהגדרה{.}

\משנה{שרש האמונה בטהרתה\mycircle{°}}\הגדרה{ - חיבור הקדושה הרוממה, של אור\hebrewmakaf אין\hebrewmakaf סוף\mycircle{°}, עם הקדושה החודרת בכל העולמים ובכל היצורים כולם }\מקור{[עפ״י ע״ר א קיד]}\צהגדרה{.}

\תמשנה{תוכן האמונה }\הגדרה{-}\תהגדרה{ האידיאליות\mycircle{°} הנשמתית\mycircle{°}, שהיא התשוקה העליונה והרוממה להתדבקות\hebrewmakaf אלהית\mycircle{°}, שתנאיה הם השואה גמורה ומוחלטת בין המעשה והמחשבה\hebrewmakaf האלהית\mycircle{°} }\תמקור{[מ״ר 494]. }

\משנה{אמונה, יסודה העצמי }\הגדרה{- נקודת הציור\mycircle{°} האלהי ברקמת החיים הפנימית (של) היחיד או הצבור, המשפחה או האומה\mycircle{°}, המפלגה או הסיעה, העושה את הרקמה הנפשית\mycircle{°} היותר חטיבית, יותר איתנה ויותר חודרת }\מקור{[עפ״י א״א 78 (קובץ ז עו)]}\צהגדרה{. }

\משנה{האמונה האלהית }\הגדרה{- הסגולה\mycircle{°} של כל יסוד החיים\mycircle{°}, של אורם\mycircle{°}, של חיי חייהם, של זיום\mycircle{°} ותפארתם\mycircle{°} }\מקור{[ע״א ד יא יג]}\צהגדרה{. }

\משנה{נקודת קדושת האמונה האלהית }\הגדרה{- יסוד יראת\hebrewmakaf ד׳\mycircle{°} באמת }\מקור{[א״א 95]}\צהגדרה{. }

\משנה{אמונה }\הגדרה{- היראה\mycircle{°}, התוכן האצילי\mycircle{°}, המקודש\mycircle{°}, האלהי, של האומה }\מקור{[א״ש טו יא]}\צהגדרה{. }

\הגדרה{היחס האמיתי\mycircle{°} הפנימי אל יסוד המציאות במקורו, המשיב לנשמה את צביונה }\צהגדרה{[עפ״י א״א 3\hebrewmakaf }\צמקור{82 (}\צהגדרה{מ״ר 74)]. }

\הגדרה{תהילת\hebrewmakaf ד׳\mycircle{°} וסקירת השלמות העליונה }\מקור{[א״ק ד תיא (קובץ ז קמג)]}\צהגדרה{.}

\משנה{העומק התמציתי של מהות האמונה }\הגדרה{- יסוד העלוי\mycircle{°} הנשמתי היותר בהיר של האדם }\מקור{[א״א 58]}\צהגדרה{. }

\משנה{מהות האמונה }\הגדרה{- התעמקות חיה בהענינים האלהיים }\מקור{[א״א 68 (קובץ א שלה)]}\צהגדרה{. }

\משנה{רז האמונה }\הגדרה{- אור האלהי שבנשמה, המבקש את הרוח\hebrewmakaf הטהור\mycircle{°}, את הקודש הנשגב בחיים, בהרגשה ובידיעה }\מקור{[קובץ ח ע]}\צהגדרה{. }

\מעוין{◊}\הגדרה{ האמונה והאהבה\mycircle{°} הן עצם החיים בעוה״ז ובעוה״ב\mycircle{°} }\מקור{[עפ״י א׳ סט]}\צהגדרה{. }

\משנה{האמונה }\הגדרה{- אינה לא שכל\mycircle{°} ולא רגש\mycircle{°}, אלא גלוי עצמי היותר יסודי של מהות הנשמה, <שצריך להדריך אותה בתכונתה, וכשאין משחיתים את דרכה הטבעי לה, איננה צריכה לשום תוכן אחר לסעדה, אלא היא מוצאה בעצמה את הכל> }\מקור{[מ״ר 70]}\צהגדרה{. }

\משנה{אמונה }\הגדרה{- }\משנה{הגיונה העליון }\הגדרה{- הגילוי האלהי שבנשמה שלמעלה מכל דעת\mycircle{°} }\מקור{[ע״ט יח (א״א 56)]}\צהגדרה{. }

\משנה{חוש האמונה האלהית }\הגדרה{- דעת הדעות והרגשת ההרגשות, שמחברת את ההויה הרוחנית\mycircle{°} של האדם, המציאותית בפועל, עם ההויה הרוחנית העליונה, ומערבת את החיים שלו עם החיים המציאותיים הרמים מכל גבול ונעדרי כל חולשה פיסית }\מקור{[מ״ר 1]}\צהגדרה{.}

\הגדרה{הרגש של הכלל, (ש)הכל אצור ב}\צהגדרה{ו <בעומק הרגש מונח הכל - כל הפרטים החבויים בגנזי השכל העליון, אשר רק חלקים ממנו הולכים ומתגלים על פי ארחות החיפוש והמחקר, הרגש הוא תופש בהם שלא בהדרגה, איננו צריך לעזר ההתחכמות בשביל להכיר את טובם> }\מקור{[קבצ׳ ב פח (פנק׳ ד סח)]}\צהגדרה{. }

\הגדרה{ע׳ בנספחות, מדור מחקרים, בטחון לעומת אמונה.}

\ערך{אמונה}\footref{26}\הגדרה{ - המסקנה היותר מזהירה של הלימודים היותר רמים\mycircle{°} ונשאים המוארים באור\mycircle{°} עליון\mycircle{°} }\מקור{[א״י נ]}\צהגדרה{.}

\מעוין{◊}\הגדרה{ האמונה מקפת את כל הידיעות לעשות חטיבה כללית מכל הפרטים כולם, ובזה היא נותנת חיים נצחיים לכל הזוכים לאורה\mycircle{°}, והיא מחיה בלשד פנימיותה את המוסר\mycircle{°}, את חיי החברה ואת חיי היחיד, כשם שהיא מחיה את כל העולמים\mycircle{°} כולם, מראש\mycircle{°} ועד סוף }\מקור{[א״ק ג קז (א״א 68)]}\צהגדרה{. }

\משנה{◊◊}

\ערך{אמונה}\footref{24}\הגדרה{ }\myfootnote{ צריך להבחין בסדרת הגדרות האמונה הבאות (שגבולן סומן ב ◊) בין האמונה בבחינתה כממד אונטי, לבין האמונה כממד פסיכי; ובהגדרות המשתפות ביניהם.\label{29}}\הגדרה{ - }\משנה{תוכן האמונה }\הגדרה{- היסוד הקדמון לכל היש, החובק כל המציאות והממלא אותה עצמת הקיום וההמשכה ההוייתית. שורה היא האמונה בכל הנברא ונוצר ונעשה, שורה היא ברומי מרומים, משתפלת היא בשפלי שפלים. וקדמותה\mycircle{°} העליונה של אור האמונה וצחצחות טבעיותה האיתן, זהו המכריח את כל החיים האנושיים להיות מוטבעים על פיה }\מקור{[ע״א ד ט קכה]}\צהגדרה{. }

\מעוין{◊ }\תמשנה{תוכן האמונה}\תהגדרה{ אינו רעיון כ״א מציאות גמורה, הנמצאת בכל חלקי היצירה, גם בדומם, והוא סוד קיומו של עולם }\תמקור{[מ״ר 489].}

\תמשנה{כח האמונה }\תהגדרה{הוא כח כללי ולא פרטי, אור מקיף הנובע ממקור ההויה כולה ומתפשט על הכל }\תמקור{[מ״ר 489]. }

\תערך{אמונה }\הגדרה{-}\תהגדרה{ עריגת כל העולמים\mycircle{°} כולם לחי\hebrewmakaf \hebrewmakaf \hebrewmakaf העולמים\mycircle{°} }\תמקור{[מ״ר 494]. }

\ערך{אמונה }\הגדרה{- }\משנה{האמונה הרבה}\myfootnote{ \textbf{האמונה הרבה} - ע׳ י׳ מאמרות לרמ״ע, אכ״ח ח״ג יא, ביד יהודה ס״ק יד.\label{30}}\הגדרה{ - האמונה\hebrewmakaf האלהית\mycircle{°} (ה)חוקקת את התפקידים של כל כוחות ההויה, שיעשו את עבודתם בכל עז\mycircle{°} ומרץ, בכל איתניות של קיום ונצחון, שהכח הכביר של אור האמונה האלהית מאיר עליהם, למסור בידם תוכנים וענינים מתוקנים עומדים על מכונם, ולקלקל ולהפסיד את צורתם וערכם, כדי להוציא ע״י קלקול זה ערכים יותר נפלאים בחדוש פנים לעילוי\mycircle{°} ולשבח. מהסתעפות\mycircle{°} }\משנה{האמונה הרבה}\הגדרה{, הכוללת כל היקום, מתגלה הדר\mycircle{°} כח המחיה והמפריא, המעודד והמחדש, בכל תוכן שיוצאת ממנה ערות של חיים, אחרי שכבר צללי המות\mycircle{°} באו ופרשו עליו את אפלתם. האדם, כל כוחות החיים וההויה, מתעוררים אחרי בלותם, בכח אור\mycircle{°} האמונה\hebrewmakaf האלהית\hebrewmakaf העליונה, מיסוד אמונת\hebrewmakaf אומן\mycircle{°} של חכמת\mycircle{°} ודעת\mycircle{°}, אשר באמונת העתות, בכל מסיביהן ותמורותיהן }\מקור{[עפ״י ע״ר א ג]}\צהגדרה{. }

\הגדרה{האמונה המסדרת כוחות ההויה }\מקור{[עפ״י רצי״ה שם ב תלט]}\צהגדרה{. }

\הגדרה{ע׳ במדור תורה, ״אמון רבתא״. }

\ערך{האמונה הגדולה }\הגדרה{- אמונת עולמים. האמונה הגדולה השרויה בעומק האהבה, הפועלת ברוח חיים בשכל\hebrewmakaf עליון\mycircle{°} ומפואר, בסדר והתאמה בכללות המון היצורים והעולמים\mycircle{°} כולם. האמונה הפנימית, היודעת את כבודה, את אשרה וגבורתה המדושנת עונג פנימי, המכירה שהיא בכל מושלת, שהיא מחלקת במדה ובמשקל של צדק\mycircle{°} ויושר\mycircle{°}, אור\mycircle{°} וחיים\mycircle{°}, לכל המון יצורים לאין תכלית, על פי סדר וערך של יחושים נאמנים, ערוכים ברוח שלום\mycircle{°} ואמת\mycircle{°} }\מקור{[עפ״י א׳ מב\hebrewmakaf מג]}\צהגדרה{. }

\ערך{אמונה }\הגדרה{- }\משנה{״אֹמֶן״}\myfootnote{ \textbf{אומן} - עפ״י הקדמת ר״י הארוך בר קלונימוס האשכנזי (המיוחסת לראב״ד) לס״י, הנתיב הג׳.\label{31}}\משנה{ }\הגדרה{- החפש\mycircle{°}, אב\mycircle{°} האמונה\hebrewmakaf העליונה\mycircle{°}, השכל המקודש שהוא יסוד החכמה\mycircle{°} הקדומה, שמכוחו האמונה נאצלת }\מקור{[עפ״י א״א 128, שם 17, (77)]}\צהגדרה{. }

\הגדרה{ע׳ במדור שמות כינויים ותארים אלהיים, ״אהיה אשר אהיה״. ע׳ במדור הכרה והשכלה והפכן, השכלה העליונה. ע׳ במדור פסוקים ובטויי חז״ל, אמונות אמיתיות.}

\ערך{אמונה }\הגדרה{- }\משנה{״אמונת אומן\mycircle{°}״}\myfootnote{ \textbf{אמונת אומן} - עפ״י ישעיה כה א.\label{32}}\משנה{ }\הגדרה{- האמונה\mycircle{°} העליונה\mycircle{°} שגורל\mycircle{°} העתיד בידה הוא. המתק הגדול של הנועם\hebrewmakaf העליון\mycircle{°} אשר לאור עולם שלעתיד, נועם ד׳ המתבקש בכל שכלול של קדושה\mycircle{°} }\צהגדרה{[א״א }\צמקור{128, 80}\צהגדרה{]. }

\הגדרה{הנקודה הגורלית העליונה של כל האדם וכל היצור, בה מתגלה הערך היותר עצמי וטפוסי, היותר פנימי ומכוון אל הטוהר המהותי של האדם, שבה ספיגת כל החיים כולם, כל הזיו\mycircle{°} והזוהר\mycircle{°} שבעולם, כל השלום\mycircle{°} והאושר\mycircle{°} שבעולם, כל החיל\mycircle{°} והחוסן\mycircle{°} שבעולם, השושנה\hebrewmakaf העליונה\mycircle{°} חבצלת השרון }\מקור{[עפ״י שם 133]}\צהגדרה{. }

\הגדרה{האמונה הרוממה המושכלת, מעמק האמונה בטהרתה\mycircle{°} הפנימית\mycircle{°}, שהמחשבות הנובעות ממנה הן החפשיות\mycircle{°} באמת, שאין עליהן שום עול של הגבלה, והן המתדמות ביותר למקור היצירה האלהית, ״ואהיה\mycircle{°} אצלו אמון״. מקור כל השמחות והשעשועים\mycircle{°} האציליים, ״משחקת לפניו בכל עת ושעשועי את בני אדם״ }\צהגדרה{[עפ״י שם }\צמקור{55, 128, }\צהגדרה{80]. }

\הגדרה{האמונה\hebrewmakaf האלהית\mycircle{°} העליונה, שבישראל, מצד ערכה הפנימי, שאין לו שום הערכה בשום צד המתדמה לו בתואר חיצוני, הוא ענין סגולי\mycircle{°}, לא מצד בחירת נפשם בפרט אלא מצד הקדושה וסגולת ירושת אבות\mycircle{°} }\מקור{[עפ״י ע״א ד יא יג]}\צהגדרה{. }

\הגדרה{האמונה\hebrewmakaf האמיתית\mycircle{°}. האמונה העליונה }\מקור{[קובץ א קס, תקפט]}\צהגדרה{.}

\הגדרה{האמונה המחי׳ את בני׳ באור\hebrewmakaf ד׳\mycircle{°} ובשמירת התורה\mycircle{°} והמצוה\mycircle{°} באמת ובתמים }\מקור{[אג׳ א קטו]}\צהגדרה{.}

\משנה{האמונה הגדולה }\הגדרה{- הציור\mycircle{°} העליון\mycircle{°} של ההשכלה הרוחנית\mycircle{°} (ש)איננו מצטמצם בשכל הגיוני\mycircle{°}, (שאורו) הוא זיו\mycircle{°} החיים החזק; חוש\hebrewmakaf האמונה\hebrewmakaf האלקי\mycircle{°}, בכל גודל עזוזו\mycircle{°}, זהו החיים האמיתיים\mycircle{°}, החיים ששום מות\mycircle{°} אין עמם, החיים ששמחתם איננה נחלשת משום צרה יגון ואנחה, החיים שהטובה והברכה\mycircle{°} שרויים בם, עמם דבקים שמחת\mycircle{°} עולמים בלא עצב, חדוה\mycircle{°} עליונה רחבה ועשירה, בלא שום דלדול ורפיון }\מקור{[עפ״י א״א 131 (מ״ר 75)]}\צהגדרה{. }

\הגדרה{הכח הרוחני שבעומק קדושת הנשמה, רוח החיים הפועם במלוא הנשמה, רוח אלהים שבלב, החי בתוכיותה של הנשמה }\מקור{[עפ״י א״ת ב 105]}\צהגדרה{.}

\ערך{אמונה }\הגדרה{- }\משנה{היסוד העליון של האמונה }\הגדרה{- הארת\mycircle{°} הקודש\mycircle{°} בצורה העליונה שלמעלה מן הטבע\mycircle{°}, המושלת על הטבע ומשכללתו }\מקור{[א״א 128]}\צהגדרה{. }

\הגדרה{האמונה\hebrewmakaf העליונה, האמונה\hebrewmakaf הגדולה, אמונת\hebrewmakaf אומן }\צהגדרה{[עפ״י שם }\צמקור{131, 133}\צהגדרה{]. }

\הגדרה{ע״ע אמונה, האמונה הרבה. ע׳ במדור הכרה והשכלה והפכן, השכלה העליונה.}

\מעויןמרכזי{◊}

\ערך{אמונה }\הגדרה{- }\משנה{היסוד הטבעי של האמונה }\הגדרה{- עריגת הקודש שבנפש האדם }\מקור{[א״א 128]}\צהגדרה{. }

\משנה{טבע האמונה\hebrewmakaf האלהית\mycircle{°}}\הגדרה{ - עומק הטבע הנפשי, תאות הדבקות\hebrewmakaf האלהית\mycircle{°} ברעיון ובחפץ פנימי, (ש)היא תאוה וחמדה עליונה, חזקה מכל התאוות שבעולם }\מקור{[שם 116]}\צהגדרה{. }

\משנה{האמונה הטבעית }\הגדרה{- עריגת הקודש, החשק הפנימי, של הדבקות\hebrewmakaf האלהית }\מקור{[עפ״י שם 108 (קובץ ז קמא)]}\צהגדרה{. }

\משנה{הארת האמונה הטבעית }\הגדרה{- אור אלהים המפעם בנשמה בכחו הגדול מצד עצמו <חוץ ממה שהוא מואר באורה של תורה, של מורשת אבות וקבלה> }\מקור{[א״א 107 (קובץ ז פ)]}\צהגדרה{. }

\מעוין{◊ }\משנה{שני צדדים בטבע האמונה: צד החסד שלה }\הגדרה{- התוך הרוחני, הגרעין האידיאלי, שבאמונה, הצורה השכלית של החובה והמצוה האלהית העליונה שבה, הצורה הטבעית של האמונה, <בצורה הטבעית של הקודש שבכנסת\hebrewmakaf ישראל\mycircle{°} חקוקים הם כל תוכני הצורות של המצות כולן וסעיפיהן, כל התורה כולה>; ו}\משנה{צד הגבורה שבה }\הגדרה{- ההמשכה הטבעית, התאוה הנפשית הפנימית, החומר הרוחני של האמונה, עריגת הקודש של כל העולם, שאפשר לה להיות בישרי לב שבכל האדם כולו }\צהגדרה{[עפ״י שם }\צמקור{128, 117}\צהגדרה{]. }

\הגדרה{ע׳ במדור מועדים וחגים, פסח, יסוד חג הפסח. ושם, סוכות, יסוד חג הסוכות. ע״ע יהדות טבעית, ע״ע יהדות ניסית. ע׳ במדור פסוקים ובטויי חז״ל, ״עמוסי בטן״. ע׳ במדור אליליות ודתות, קליפת האגוז. ושם, ״יצרא דעבודה זרה״. ע״ע רליגיוזיות, הטבעיות הרליגיוזית. }

\ערך{אמונה }\הגדרה{- }\משנה{עבודת האמונה }\הגדרה{- שכלול האמונה ורעיית\hebrewmakaf האמונה}\myfootnote{ \textbf{רעיית האמונה} - עפ״י תהילים לז ג.\label{33}}\הגדרה{, במעשים טובים ובמדות טובות בתורה\mycircle{°} וחכמה\mycircle{°} עליונה }\צהגדרה{[עפ״י א״א }\צמקור{76, 77}\צהגדרה{]. }

\ערך{אמונה }\הגדרה{- }\משנה{עיקרי אמונה }\הגדרה{- ע״ע עקרים. }

\ערך{אמונה }\הגדרה{- }\משנה{רעיית האמונה }\הגדרה{- ע״ע אמונה, עבודת האמונה. }

\תערך{אמונה }\הגדרה{- }\תמשנה{מגמת האמונה הטהורה }\הגדרה{-}\תהגדרה{ השואת המעשה לתוכן המחשבה והבאת הרמוניה שלמה ביניהם }\תמקור{[מ״ר 493]. }

\ערך{אמונה }\הגדרה{- }\משנה{״אמונות אמיתיות״}\הגדרה{ - ע׳ במדור פסוקים ובטויי חז״ל, אמונות אמיתיות.}

\ערך{אמונה }\הגדרה{- }\משנה{״אמונות מוכרחות״ }\הגדרה{- ע׳ במדור פסוקים ובטויי חז״ל, אמונות מוכרחות.}

\ערך{״אמונה באמונה״ }\הגדרה{- ר׳ ״ללמוד אמונה באמונה״.}

\ערך{אמונה בברית\mycircle{°}}\הגדרה{ - הכונניות\mycircle{°} המונחות ביצירת הרגש, שכשיתפתח עפ״י מדתו יהיה תמיד מתאים אל התכונה הרוממה של הדעות\mycircle{°} הטהורות\mycircle{°} }\מקור{[ע״ר א שפה]}\צהגדרה{.}

\הגדרה{ע׳ במדור פסוקים ובטויי חז״ל, זכירת הברית. }

\ערך{אמונה עליונה }\הגדרה{- ע׳ במדור פסוקים ובטויי חז״ל, אמונות אמיתיות. }

\ערך{אמונה שלמה}\הגדרה{ - אמונה ביכולת ד׳ ובבריאת העולם יש מאין }\מקור{[פנק׳ א תי]}\צהגדרה{.}

\ערך{אמונה תחתונה }\הגדרה{- ע׳ במדור פסוקים ובטויי חז״ל, אמונות מוכרחות. }

\משנה{אמונת עתנו }\צהגדרה{- האמונה המתגלה מתוך כל מקוריותה ושלמותה של רציפות הדורות, בכל תקפה בממשות ההופעה האלהית של עתנו זאת }\צמקור{[ל״י ג קיז (מהדורת בית אל תשס״ג, ב תז)].}

\צהגדרה{ראיית יד\hebrewmakaf ד׳\mycircle{°} אלהי\hebrewmakaf ישראל\mycircle{°} קורא דורותינו אלה ומכונן פעליהם, המתגלה בהם, בחוסן ישועותינו, בתקומת עמו ונחלתו וחזרת שכינתו, בקיבוץ נדחיו וכינוסם ובבנין בית חייו }\צמקור{[ל״י א ריח\hebrewmakaf ט]. }

\ערך{אמירה }\הגדרה{- ההופעה המלולית, הבאה מסגולת כח הביטוי שבאדם, המתחלת מראשית הציור המחשבתי, שהוא מצויר אצל בעל המבטא. ומפני שראשית היסוד של צמיחת ההבטאה באה מתוך הרעיון\mycircle{°}, נקרא זה בשם }\משנה{אמירה}\הגדרה{; והמקור הוא אמור, כמו ראש אמיר, כלומר התגלות הדבור ביחסו להמדבר בעצמו, טרם שבא להתגלם בצורה המשפעת כבר על השומע }\מקור{[עפ״י ע״ר ב נד]}\צהגדרה{. }

\הגדרה{ע״ע אֹמר. ע״ע דבור, ע״ע קול. ע׳ בנספחות, מדור מחקרים, ״אֹמֶר״ לעומת ״דבור״. }

\ערך{אמיתי }\הגדרה{- פנימי\mycircle{°} }\מקור{[א״א 82]}\צהגדרה{. }

\הגדרה{כללי\mycircle{°} }\מקור{[ע״א ג ב קסו]}\צהגדרה{. }

\הגדרה{ע״ע אמת, באמת. ע״ע אמת, להיות חפץ באמת. }

\ערך{אמיתיות מקובלות}\הגדרה{ - דברים רבים, שלבד אמתתם ההגיונית הנשענת בדרך כללי מבירורה של תורה\mycircle{°} הגלוי והמבורר, ״אתה הראת לדעת כי ד׳ הוא האלהים״, עוד הם נתמכים ביסוד ההכרה הפנימית הכללית שבאומה}\myfootnote{ \textbf{ההכרה הפנימית הכללית שבאומה} - ע״ע במדור תורה, תורה שבעל פה, יסודה של תושבע״פ השמירה של קבלת אבותינו מה שהתנחלו באומה, ובהערה שם.\label{34}}\הגדרה{, שהעבר וההוה שלה מרוכס ומקושר בקשורים אמיצים גלויים ומוחשים בהכרה גלויה ואמתית לכל מי שהולך בדרכיה הטבעיים לה }\מקור{[ע״א ג ב קז]}\צהגדרה{.}

\הגדרה{ע׳ במדור פסוקים ובטויי חז״ל, הגוי כולו, קבלת ״הגוי כולו״. וע׳ בנספחות, מדור מחקרים, ודאות באמיתותה של התורה.}

\ערך{אֹֽמֶר}\הגדרה{ - הענף העליון של הבטוי. הבטוי הפנימי מצד המבטא }\מקור{[ע״ר ב נד]}\צהגדרה{.}

\הגדרה{ע״ע אמירה. ע״ע דבור. ע״ע קול.}

\ערך{אֹֽמֶר }\הגדרה{- }\משנה{א}\ערך{ֹ}\משנה{מֶר ההויה כולה }\הגדרה{- רוחה הפנימי של ההויה בסודה הכלול בקרבה, באותו המצב הראוי להתגלות ליצורים בעלי שכל והרגשה, המכירים כבר ציורים שכלים }\מקור{[ע״ר ב נד]}\צהגדרה{. }

\הגדרה{ע״ע קול, קול ההויה כולה. ע״ע דבור, דבור ההויה הכללית. }

\ערך{אמת }\הגדרה{- השלמות המוחלטת של האלהות }\מקור{[ע״א ד ח לה]}\צהגדרה{. }

\משנה{אור האמת }\הגדרה{- ברק העצם\mycircle{°} ממקור\hebrewmakaf המקורות\mycircle{°} עדי\hebrewmakaf עד\mycircle{°} }\מקור{[ע״ה קנה]}\צהגדרה{. }

\משנה{האמת כשהיא לעצמה }\הגדרה{- העליוניות\mycircle{°} המוחלטה }\מקור{[א״ק ג ל]}\צהגדרה{.}

\ערך{אמת }\הגדרה{- }\משנה{״אמת ד׳ לעולם״ }\הגדרה{- ההנהגה העליונה\mycircle{°}, שבמדת\hebrewmakaf הדין\mycircle{°}, קשר הקדש\hebrewmakaf העליון\mycircle{°} }\מקור{[עפ״י ע״ר א ריג]}\צהגדרה{. }

\הגדרה{מדת הדין כמו שהיה קודם בריאת\mycircle{°} העולם }\מקור{[מא״ה א קלו]}\צהגדרה{. }

\צמשנה{אור האמת }\צהגדרה{- מדת\hebrewmakaf הדין\hebrewmakaf העליונה\mycircle{°} }\צמקור{[ע״ר ב תפח]. }

\משנה{אמת }\הגדרה{- אור\hebrewmakaf ד׳\mycircle{°}, מחולל כל }\מקור{[א״ק א ג (מ״ר 402)]}\צהגדרה{. }

\הגדרה{היסוד של החיים השלמים, החיים המקיפים את כל ומלאים את הכל הנובעים מאור צור\hebrewmakaf העולמים\mycircle{°}, הכל\hebrewmakaf יוכל\hebrewmakaf וכוללם\hebrewmakaf יחד\mycircle{°} }\מקור{[ע״א ד יב לב]}\צהגדרה{. }

\משנה{האמת שלמעלה מכל גבולים }\הגדרה{- התענוג\mycircle{°} האצילי\mycircle{°} שלמעלה מכל פגמים, העז\mycircle{°} של העדן\mycircle{°} שכולו\hebrewmakaf אומר\hebrewmakaf כבוד\mycircle{°} }\מקור{[עפ״י א״א 77]}\צהגדרה{. }

\משנה{גבורת האמת }\הגדרה{- מציאות חיה וקיימת, עליונה, בחביון\hebrewmakaf העוז\mycircle{°} האלהי, באופן יותר עשיר, יותר קיים ויותר אמיץ בהוייתו, ויותר נעלה בהופעת רוממות אצילות שלמותו, מכל מה שכל רעיון יוכל לצייר ולתפוס }\מקור{[ע״ט נ]}\צהגדרה{. }

\הגדרה{סוד\mycircle{°} המציאות העצמית, שלמעלה מהזיו\mycircle{°} האידיאלי\mycircle{°} דלגבי דידן }\מקור{[שם שם]}\צהגדרה{. }

\משנה{זוהר האמת }\הגדרה{- ההוד של אור\hebrewmakaf החיים\mycircle{°} שבמקור הקודש\mycircle{°} }\מקור{[א״ק א ג (מ״ר 402)]}\צהגדרה{. }

\משנה{אמת אלהית }\הגדרה{- החפץ המרומם והנשגב של המגמה\mycircle{°} האידיאלית אשר בעליונות הקודש }\מקור{[א׳ יא]}\צהגדרה{. }

\משנה{האמת העליונה }\הגדרה{- אור\mycircle{°} הטוהר\mycircle{°} של שיגוב החכמה\mycircle{°} השוכן ברום חוסן\mycircle{°} אמונת\mycircle{°} ישראל\mycircle{°} }\מקור{[עפ״י א״ק ב רפה]}\צהגדרה{. }

\משנה{האמת העליונה }\הגדרה{- האמת\hebrewmakaf האלהית\mycircle{°}, שהיא נתונה מיד רבון כל המעשים ע״פ הסדר של אמיתת המציאות, ע״פ המהות העליון שלה }\מקור{[ע״ר א נד]}\צהגדרה{. }

\ערך{אמת }\הגדרה{- }\משנה{מדת האמת }\הגדרה{- הדין בלא צדקה }\מקור{[מא״ה קסד]}\צהגדרה{. }

\ערך{אמת }\הגדרה{- היסוד ההגיוני המופשט והקר, בעל המופתים והמשפטים המחייבים}\צהגדרה{ }\מקור{[א״ה (מהדורת תשס״ב) ב 81 (א״ב ג)]}\צהגדרה{.}

\ערך{אמת }\הגדרה{- }\משנה{(לעומת שלום\mycircle{°}) }\הגדרה{- שלמות הקיום של כל נמצא מצד עצמו ופרטיותו }\מקור{[עפ״י מ״ש קכ (מא״ה ב יד)]}\צהגדרה{. }

\משנה{כח האמת האלהית }\הגדרה{- מקור הקדושה\mycircle{°} הכללית של ישראל\mycircle{°}, (ה)נותן כחו בהם לכל יחיד פרטי, להתקדש בכח עצמי ולהעשות בזה מוכן להוסיף קדושה מיוחדת מצדו (לאיזה נושא או ענין) }\מקור{[עפ״י ע״ר ב רנה]}\צהגדרה{. }

\ערך{אמת }\הגדרה{- היסוד הנצחי המוציא את דבר המשפט\mycircle{°} בקו הצדק\mycircle{°} הגמור }\מקור{[עפ״י ע״א ד יב לח]}\צהגדרה{. }

\משנה{האמת הגדולה }\הגדרה{- מקור משפטי ד׳ }\מקור{[ע״ר ב ס]}\צהגדרה{. }

\משנה{אור האמת }\הגדרה{- נשמת הצדק\mycircle{°} העליון\mycircle{°}, אשר במשפטי ד׳ הישרים\mycircle{°} }\מקור{[שם שם]}\צהגדרה{. }

\משנה{האמת האלהית העליונה}\הגדרה{ - מגמת כל חמדת עולמים, הגנוזה במשפטי ד׳}\צהגדרה{ }\מקור{[עפ״י ע״ר ב נט]}\צהגדרה{.}

\ערך{אמת }\הגדרה{- }\מעוין{◊}\הגדרה{ דבר נצחי ומתקיים}\myfootnote{ \textbf{אמת דבר נצחי ומתקיים} - ע׳ של״ה ח״א, תולדות אדם, בית דוד, ז: ״אמת פירושו מציאות אמתיי נצחיי שלא יכזב רק יתד אשר לא ימוט״. מגן וצינה דף י.\label{35}}\הגדרה{ }\מקור{[ע״ר א רכז]}\צהגדרה{. }

\הגדרה{הנצח }\מקור{[מ״ר 159]}\צהגדרה{. }

\הגדרה{ע״ע שקר. ע״ע כזב. ע׳ בנספחות, מדור מחקרים, צדק ואמת. ע׳ במדור מונחי קבלה ונסתר, חסד, אמת (משפט), ורחמים. }

\ערך{אמת }\הגדרה{- }\משנה{גאולת האמת שלמעלה מכל גבולים }\הגדרה{- האידיאליות\mycircle{°} בתענוג\mycircle{°} האצילי\mycircle{°} שלמעלה מכל פגמים, העז\mycircle{°} של העדן\mycircle{°} שכולו\hebrewmakaf אומר\hebrewmakaf כבוד\mycircle{°} }\מקור{[עפ״י א״א 77]}\צהגדרה{. }

\ערך{אמת }\הגדרה{- }\משנה{באמת }\הגדרה{- }\משנה{(גישה אמיתית בשיפוט) }\הגדרה{- בלא נטיה של חפץ להטיב לשום צד }\מקור{[אג׳ א קנט]}\צהגדרה{. }

\ערך{אמת - }\משנה{״לכל אשר יקראוהו באמת״ }\הגדרה{- התכלית האמיתי, של עיקר קיום החיים <ולא הבלי עוה״ז הכלים> }\מקור{[עפ״י ע״ר א רכז, מא״ה, ענייני תפילה, שד]}\צהגדרה{. }

\ערך{אמת }\הגדרה{- }\משנה{להיות חפץ באמת }\הגדרה{- להיות חי ופועל לפי הדיעות היותר אמיתיות וההרגשות היותר קדושות\mycircle{°} לטוב\mycircle{°} ולחסד\mycircle{°}, כמעשה גדולי העולם אשר נגשו אל ד׳ במעשיהם הבהירים למלא את העולם חסד ואמת }\מקור{[ע״א ג ב קפד]}\צהגדרה{. }

\הגדרה{השאיפה לציורי\mycircle{°} המושכלות מצד עצמם ונצחיותם }\מקור{[א׳ לה]}\צהגדרה{.}

\הגדרה{ע״ע אמיתי. }

\ערך{אמת }\הגדרה{- }\משנה{חיי אמת }\הגדרה{- ע״ע חיים, חיי אמת. }

\ערך{אן }\הגדרה{- הוראת השאלה ביחס המקום של איזה מבוקש }\מקור{[ר״מ קכד]}\צהגדרה{.}

\ערך{אן }\הגדרה{- }\משנה{בצורה רוחנית }\הגדרה{- דרישת המטרה התכליתית ממחזה כללי המופיע בהמון פרקים, בתבנית ברקי אורות ונצוצי חיים מבריקים, בצורה זעירה ומעולמת. והשאלה חודרת היא, אן מונחת היא המטרה המרכזית של כל המון בריות הללו שהם בלי תכלית }\מקור{[שם]}\צהגדרה{.}

\ערך{״אנוֹש״ }\הגדרה{- יאמר (על האדם) ע״ש הכח היותר חלוש שבנפש\mycircle{°} והוא הרצון הסתמי בלא טעם דעת ובחירה ושכל, רק רצון לבד ונטי׳ דומה ממש לרצון כל בע״ח למיניהם }\מקור{[עפ״י פ״א קעז, קעה]}\צהגדרה{.}

\הגדרה{ע״ע ״אדם״.}

\ערך{אני }\הגדרה{- עצמיותי האנושית }\מקור{[ע״ר א מה]}\צהגדרה{.}

\ערך{אנרכיא }\הגדרה{- }\משנה{אנארכיזם הגשמי האינדיבידואלי }\הגדרה{- אהבה עצמית רבה וגדולה }\מקור{[אג׳ א קעד, קעה]}\צהגדרה{.}

\ערך{אנשים }\הגדרה{- }\משנה{(לעומת נשים\mycircle{°}) }\הגדרה{- הכח הפועל בעולם (בחברה) }\מקור{[ע״א ד ו כז]}\צהגדרה{.}

\הגדרה{ע״ע איש.}

\צמשנה{אסיפה}\צהגדרה{  - קיבוץ ואחדות }\מקור{[מ״ש קיד]}\צהגדרה{.}

\ערך{אסתטי }\הגדרה{- }\משנה{החוש האסתטי }\הגדרה{- הרגש של היופי\mycircle{°} וההידור }\מקור{[ע״א ג ב סז]}\צהגדרה{. }

\ערך{אף }\הגדרה{- הוראה לדבר נטפל שאינו עומד לעצמו, כ״א הוא מצטרף וטפל לדברים אחרים, גדולים ועקרים יותר ממנו }\מקור{[ר״מ קכד]}\צהגדרה{. }

\ערך{אף }\הגדרה{- ע׳ במדור נפשיות.}

\הגדרה{ }

\ערך{אף }\הגדרה{- }\משנה{(אלקי) }\הגדרה{- ע׳ במדור תיאורים אלהיים. }

\ערך{אף }\הגדרה{- }\משנה{(בתאור הפנים) }\הגדרה{- ע׳ במדור גוף האדם אבריו ותנועותיו.}

\ערך{אף }\הגדרה{- ע׳ במדור גוף האדם אבריו ותנועותיו.}

\תערך{אפיקורסות }\הגדרה{- }\תמשנה{תכן האפיקורסות }\הגדרה{-}\תהגדרה{ הסתלקות האדם מחבור של קדושה\mycircle{°}, מהתקשרות אלהית. המחשבה הגרועה, של הסתלקות מוחלטת מאלהות }\תמקור{[מ״ר 493]. }

\צהגדרה{מהלך מחשבה. מהלך רוח חומרני, מטריאליסטי, המנותק מעוה״ב\mycircle{°}, מנותק מקשר עם הנצח }\צמקור{[שי׳ ת״ת 138].}

\הגדרה{ע״ע כפירה (שלילת האמונה). ע״ע שלילה. ר׳ במדור מדרגות והערכות אישיותיות, אפיקורס. ושם, כופר. ע׳ במדור פסוקים ובטויי חז״ל, העושה תורתו עיתים הרי זה מיפר (תורה) }\צמקור{[ברית]}\הגדרה{. }

\ערך{אץ }\הגדרה{- }\מעוין{◊}\הגדרה{ מורה מהירות }\מקור{[ר״מ קכה]}\צהגדרה{. }

\ערך{אץ }\הגדרה{- לחיצה ודחיפה, מקושר עם צרות ביחש המקום\mycircle{°} }\מקור{[ר״מ קכה]}\צהגדרה{. }

\ערך{אצילות }\הגדרה{- }\משנה{אצילות רצונית }\הגדרה{- המוסריות\mycircle{°} העליונה המתגלה ע״י קדושה\mycircle{°} וחסידות\mycircle{°} טהורה\mycircle{°} ועליונה }\מקור{[ע״ט קכ]}\צהגדרה{. }

\הגדרה{הרצון התמים והבהיר של האדם התופס את קצה זיוה\mycircle{°} של האצילות\hebrewmakaf האלהית\mycircle{°} }\מקור{[עפ״י א״ק ב שמט]}\צהגדרה{. }

\ערך{אר }\הגדרה{- (מורה) קללה ומארה }\מקור{[ר״מ קכו]}\צהגדרה{. }

\ערך{אר }\הגדרה{- הוראת אור, הארה\mycircle{°} וזריחה\mycircle{°} }\מקור{[ר״מ קכו]}\צהגדרה{. }

\ערך{אר }\הגדרה{- (מורה) לקיטה ותלישת פירות }\מקור{[ר״מ קכו]}\צהגדרה{. }

\ערך{ארוכה }\הגדרה{- רפוי מתמיד וממושך למחלות כרוניות, מתוך קלקולים מתמידים }\מקור{[עפ״י מ״ר 473]}\צהגדרה{. }

\הגדרה{הרפואה המדרגת המשיבה את הכחות שנתרופפו }\מקור{[שם 371]}\צהגדרה{. }

\הגדרה{רפואה טבעית פנימית, לתקן הטבע, והוא לשון תקון כמו ״אריך לנא למחזא״}\myfootnote{ \textbf{אריך לנא למחזא} - עזרא ד יד.\label{36}}\הגדרה{. דרושה למחלות פנימיות. שבחה של דרך רפואה זו הוא להיות קרוב למזון יותר מלרפואה, כדי לחזק את טבע הגוף בעצמו מבלי להוסיף כח זר על הכח הטבעי }\מקור{[עפ״י משפט כהן, פתיחה טו]}\צהגדרה{. }

\הגדרה{ע״ע רפואה. ע״ע חולי. ע״ע מיחוש.}

\ערך{ארוסין }\הגדרה{- היסוד החוקי האצילי\mycircle{°} שבדבקות\mycircle{°} (שבין בני הזוג), שמתוך מעלתו אין בו התפסה לחקוי חמרי\mycircle{°} כלל }\מקור{[עפ״י ע״ר א לה]}\צהגדרה{. }

\ערך{ארץ }\הגדרה{- גמר כל תכליתם של הסבות\mycircle{°} הראשיות [של כל מגמה\mycircle{°} ותכלית] המסבבות כל המון המעשים }\מקור{[עפ״י ע״ר א רמה, ע״א א ב ד]}\צהגדרה{. }

\ערך{ארץ}\צהגדרה{ - }\משנה{הארץ בכלל }\הגדרה{- העולם בכלל }\מקור{[ע״א ד ו מו]}\צהגדרה{. }

\צהגדרה{המציאות }\מקור{[פנ׳ קלג]}\צהגדרה{.}

\הגדרה{המציאות המעשית }\מקור{[ע״ר א שכה]}\צהגדרה{.}

\הגדרה{גשמותה של המציאות }\מקור{[ע״א ב ט ל]}\צהגדרה{. }

\הגדרה{כלל כל העולמים\mycircle{°} החמריים\mycircle{°} כולם }\מקור{[עפ״י ע״ר א קטו]}\צהגדרה{. }

\הגדרה{העולם החומרי }\מקור{[ע״ר ב פ]}\צהגדרה{.}

\הגדרה{נבכי החומר ובמעמקי מסילותיו המסובכות }\מקור{[ע״א ד ט קד]}\צהגדרה{.}

\הגדרה{השטח התחתיתי }\מקור{[עפ״י ע״ר ב סז]}\צהגדרה{.}

\הגדרה{כחות החמריים, מחשבות האדם, סדרי החיים והחברה, וכל הנוגע לכל תהומות, עד מעמקי שפל }\מקור{[קובץ ה סח]}\צהגדרה{. }

\הגדרה{הענינים החומריים\mycircle{°} המדיניים האקונומיים }\מקור{[ע״א ג א לג]}\צהגדרה{. }

\הגדרה{ע״ע שמים. ע׳ במדור מונחי קבלה ונסתר, אחרית. ע׳ במדור מונחי קבלה ונסתר, ברתא. }

\ערך{ארץ }\הגדרה{- }\משנה{תכונה ארצית}\הגדרה{ - ברכת הטבע וכל כחותיו }\מקור{[ע״ר א רט]}\צהגדרה{. }

\צהגדרה{ }

\משנה{ארץ}\צהגדרה{ - }\מעוין{◊}\צהגדרה{ מכון הטבע של האומה\mycircle{°} }\צמקור{[צ״צ קא]}\צהגדרה{.}

\צהגדרה{ר׳ שפה.}

\משנה{ארץ}\צהגדרה{ - }\צמשנה{קדושת הארץ}\הגדרה{ - }\צהגדרה{קדושת הכלל\mycircle{°} }\צמקור{[שי׳ ה 212]. }

\צהגדרה{כלליות הקדושה של כל הקדושות }\צמקור{[רצי״ה ג״ר 133].}

\ערך{ארץ אשור }\הגדרה{- }\משנה{(׳האובדן בארץ אשור׳)}\myfootnote{ ישעיה כז יג.\label{37}}\הגדרה{ - <}\משנה{אשור}\הגדרה{ - לשון הבטה והשקפה> האובדן בארץ בדעות רעות ורוח שטות המביא לידי עבירה. גלות טעות השכל והתרופפות האמונה (בהשגחה וחסרון בטחון וכיו״ב, או זלזול הורים ומורים מפני מיעוט יקרת התורה בנפשו), מפני טרדות עוה״ז והבליו המשכחים את האמת }\מקור{[עפ״י מ״ש סו\hebrewmakaf ח]}\צהגדרה{.}

\הגדרה{ע״ע ארץ מצרים. ע׳ במדור מקומות וארצות, אשור. ע׳ במדור פסוקים ובטויי חז״ל, האבדים בארץ אשור והנדחים בארץ מצרים. }

\ערך{ארץ הרוחנית }\הגדרה{- הציורים\mycircle{°} ההולכים עם ההסברים של הידיעות האלהיות\mycircle{°}, רפידת המחשבה }\מקור{[פנ׳ קלו]}\צהגדרה{. }

\ערך{ארץ מצרים\mycircle{°} }\הגדרה{- }\משנה{(׳הנדחות בארץ מצרים׳)}\footref{37}\הגדרה{ - גלות תאוות עוה״ז, <כי }\משנה{מצרים}\הגדרה{ היא ערות הארץ ומקור התאוות והחומריות כולן> }\מקור{[מ״ש סו]}\צהגדרה{.}

\הגדרה{ע״ע ארץ אשור. ע׳ במדור מקומות וארצות, מצרים. ע׳ במדור פסוקים ובטויי חז״ל, האבדים בארץ אשור והנדחים בארץ מצרים. }

\ערך{אש }\הגדרה{- החומר השורף והמכלה, המחמם והמאיר, העושה את הפעולות ההפכיות בתכונתן, הכל לפי ערכם של מקבלי המפעלים }\מקור{[ר״מ קכז]}\צהגדרה{. }

\משנה{כח האש וחומו  }\הגדרה{- הפועל הגורם להמפעלים שיעשו }\מקור{[ע״ר א קכט]}\צהגדרה{. }

\ערך{אִשָּׁה }\הגדרה{-}\משנה{ יסוד השלמתה }\הגדרה{- עדינות הרגש\mycircle{°} הטהור\mycircle{°} והטוב\mycircle{°}, והשכל\mycircle{°} יעזר על ידו כפי המדה האפשרית }\מקור{[ע״א ג ב ריג]}\צהגדרה{.}

\הגדרה{ע״ע איש. ע׳ במדור הכרה והשכלה והפכן, בינה יתירה (באשה). ע״ע גברת.}

\ערך{את }\הגדרה{- מלת הצירוף, הוראת הטפלה, הערכת התוספת\mycircle{°} }\מקור{[ר״מ קכז]}\צהגדרה{. }

\ערך{את }\הגדרה{- הכלי היסודי לעבודת האדמה להוציא מחיה מן הארץ }\מקור{[ר״מ קכח]}\צהגדרה{. }

\משנה{א״ת ב״ש }\הגדרה{- אחדות החיצוניות והפנימיות <אופן הא״ב בסדר א״ת ב״ש הוא שהקצות הרחוקים מתאחדים, ואין שינוי והפרש, וכלם הולכים להתאחד ולהתקרב עד כ״ל, מפני שב״כל״ אין הרחק והבדל מדרגות> }\מקור{[עפ״י פנק׳ ג לה]}\צהגדרה{.}\mylettertitle{ב}

\ערך{בא }\הגדרה{- הוראת התכנסות הנושא אל המקום הראוי ע״י תנועה מוקדמת }\מקור{[ר״מ קכח]}\צהגדרה{.}

\ערך{באור }\הגדרה{- יחשו של כל מאמר בודד, לא רק לפי ערכו והדבר המבוצר בתוכו בלבד, כ״א עפ״י ערך כל אותן ההשפעות שאפשר לו להשפיע, לכשיתבאר, כאשר ״מעין ישיתוהו״ על עולם הרעיונות ההולכים בדרך ישרה, הוא פתוח בדרך מפולש לעולם הגדול של ההשכלות מלאות זיו\mycircle{°}, ומעורר בדרך פתחו להכניס אל תוכו ועל ידו תלי תלים של ידיעות והרחבות שכליות, שנותנות אומץ וגבורה לנפשות ההוגות בהם. ״מ״ם פתוחה - מאמר\hebrewmakaf פתוח\mycircle{°}״ }\מקור{[ע״א א, הקדמה, יד\hebrewmakaf טו]}\צהגדרה{.}

\הגדרה{ע״ע פרוש. ע׳ במדור תורה, דרש.}

\ערך{באור }\הגדרה{- }\משנה{דרך הביאור }\הגדרה{- הדרישה הבנויה ע״פ ערכי הכללים, שהם דומים לדרישת כל רעיון לא רק מצד עצמו, כ״א מצד הרעיונות שמטבעו להוליד ע״פ דרך ישרה, כן דרישת התורה שע״פ הכללים, אין הפרטים נולדים ומסתעפים זה מזה, כ״א כולם יחד יוצאים הם מהכללים הראשיים יסודי ועקרי התורה וסתרי טעמיה הגדולים }\מקור{[ע״א א, הקדמה, יז]}\צהגדרה{.}

\הגדרה{ע׳ במדור תורה, דרישת התורה בדרך הכהן. }

\ערך{בג }\הגדרה{- מזון }\מקור{[ר״מ קכח]}\צהגדרה{.}

\ערך{בד }\הגדרה{- מגזרת בדד, ההתבודדות הפרטית }\מקור{[ר״מ קכט]}\צהגדרה{.}

\צהגדרה{ }

\ערך{בד }\הגדרה{- שקר\mycircle{°} }\מקור{[ר״מ קכט]}\צהגדרה{.}

\ערך{בד }\הגדרה{- ענף של אילן }\מקור{[ר״מ קכט]}\צהגדרה{.}

\ערך{בד }\הגדרה{- מוט }\מקור{[ר״מ קכט]}\צהגדרה{.}

\ערך{בד }\הגדרה{- ע״ע בוץ.}

\ערך{בה }\הגדרה{- תואר הוראה של יחס פנימי, למושג הנקבי }\מקור{[ר״מ קל]}\צהגדרה{.}

\ערך{בהלה }\הגדרה{- טירוף של רצונות ומחשבות שכל אחת דוחה את חבירתה עד אפס מקום של חשבון נקי }\מקור{[ע״א ג ב רכג]}\צהגדרה{.}

\הגדרה{נפילת ערך וחרדה}\צהגדרה{ }\מקור{[ע״ר ב עא]}\צהגדרה{.}

\ערך{בו }\הגדרה{- מתאר את היחש הפנימי במושג הזכרי }\מקור{[ר״מ קלא]}\צהגדרה{.}

\ערך{בוץ }\הגדרה{- פשתן\mycircle{°}, <הנקרא ג״כ בד על שם שהוא עולה בד בבד> מורה על שמירת הגבולים, על כח צומח פורה, שעם זה הוא שומר חק וגבול, ואינו מתערב בחלקי חיים ותוכן של נושא אחר, שומר את המצר, ומגין על הצדק\mycircle{°} }\מקור{[ר״מ קלו]}\צהגדרה{.}

\ערך{בושה }\הגדרה{- }\משנה{אמיתתה במקורה }\הגדרה{- יראה\hebrewmakaf עליונה\mycircle{°}, יראת\hebrewmakaf ד׳\mycircle{°} }\מקור{[עפ״י ר״מ קמ, א״ש יד כד]}\צהגדרה{. }

\ערך{בחירה }\הגדרה{- }\משנה{״בחר בנו מכל העמים״ }\הגדרה{- נתן לנו יתרון ומעלה ודבקות\mycircle{°} בו ית׳ על כל עם ולשון }\מקור{[עפ״י מ״ש קמב (ה׳ קפד)]}\צהגדרה{.}

\צהגדרה{יצר אותנו להיות לו לעם\hebrewmakaf נחלה, להתגלות צלם\hebrewmakaf האלהים\mycircle{°} שבאדם בקרבנו בתור עם}\myfootnote{ ע׳ אור החיים בראשית א כז. ״״ויברא אלהים את האדם בצלמו״. כי ברא האדם בב׳ צלמים, הראשון צלם הניכר בכל אדם ואפילו בבני אדם הריקים מהקדושה אשר לא מבני ישראל המה, ועליהם אמר ״בצלמו״ פירוש: של הנברא; והב׳ הם בחינת המאושרים, עם ישראל נחלת שדי, כנגד אלו אמר ״בצלם אלהים בראו״, הרי זה בא ללמדנו כי יש בנבראים ב׳ צלמים צלם הניכר וצלם אלהים רוחני נעלם, והבן״.\label{38}}\צהגדרה{, באדם הצבורי המופיע בנו במלוא שיעור\hebrewmakaf קומתנו }\צמקור{[ל״י ג קיד\hebrewmakaf קטו (מהדורת בית אל תשס״ב ב תד)].}

\צהגדרה{הופעת קדושת עצמיותנו הצבורית, מתוך השראת\hebrewmakaf שכינתו\mycircle{°} על כלנו כאחד, בהוציאו אותנו מבית\hebrewmakaf עבדים ממצרים\mycircle{°}, ובקרבו\hebrewmakaf אותנו\hebrewmakaf לפני\hebrewmakaf הר\hebrewmakaf סיני\mycircle{°} }\צמקור{[ל״י א (מהדורת בית אל תשס״ב) רכו].}

\משנה{בחירתנו מכל העמים }\צהגדרה{- יצירת מהותנו. הוית עצמותנו הצבורית בכל ממשותה ושכלולה, תקפה ותפארתה}\צמקור{ [עפ״י שם, רכז].}

\הגדרה{ע״ע בחירת ישראל.}

\ערך{בחירה }\הגדרה{- }\משנה{(בעם ישראל, לעומת סגולה\mycircle{°}) }\הגדרה{- ההערכה הגלויה של קדושתם\mycircle{°} של ישראל\mycircle{°} }\מקור{[ע״ר ב פ]}\צהגדרה{.}

\מעוין{◊ }\צמשנה{הבחירה}\צהגדרה{ הגלויה, מתבררת ע״י המדות הקדושות והמובחרות, שבהן נתעטרו\mycircle{°} בני יעקב\mycircle{°} בכללם, עד שהכל מכירים שהבחירה האלהית ראויה להם }\צמקור{[ע״ר א רב].}

\הגדרה{ע׳ במדור מדתם ועניינם הרוחני של אישי התנ״ך, יעקב, מדת התאר יעקב (לעומת ישראל). ושם, ישראל, מדת התאר ישראל (לעומת יעקב). ע״ע ישראל לעומת ישורון. ע׳ במדור מונחי קבלה ונסתר, קוב״ה דרגא על דרגא סתים וגליא וכו׳. ע׳ במדור פסוקים ובטויי חז״ל, בית יעקב לעומת בני ישראל. ושם ממלכת כהנים וגוי קדוש. ושם בני בכורי לעומת בנים. ע׳ בנספחות, מדור מחקרים, ״בחרתי בכם ויחדתי שמי עליכם״ לעומת ״אני בכבודי מתהלך ביניכם״.}

\ערך{בחירה }\הגדרה{- }\משנה{בחירת ד׳ בכהנים\mycircle{°} }\הגדרה{- הטבעה טבעית רוחנית עליונה בסגולת\mycircle{°} נפשותם}\צהגדרה{ }\מקור{[ע״ר א קסא]}\צהגדרה{.}

\הגדרה{ע׳ במדור אישים, אהרן. ושם, ״אהרן ובניו״. ושם, ״בני אהרן״. ע״ע כהונה, קדושתה.}

\תערך{בחירה}\תהגדרה{ - חופש הפעולה של האדם }\תמקור{[נ״א ד 39].}

\ערך{בחירה גלויה }\הגדרה{- }\משנה{הבחירה הגלויה }\הגדרה{- הבחירה\hebrewmakaf החפשית\mycircle{°} הנמצאת בפועל, המתגלמת בבני אדם, שהמשפט המורגש מתראה על ידה, שבחיים המתגלים לפנינו לעולם לא נמצא אותה במילואה }\מקור{[עפ״י אג׳ א שיט, א״ק ג לד]}\צהגדרה{. }

\הגדרה{הבחירה שפרטי הכוחות והפרטיות הדקות שבמציאות ביחש לשכר ועונש נחלקים על פיה }\מקור{[עפ״י פנ׳ כג]}\צהגדרה{. }

\מעוין{◊}\הגדרה{ עקר הבחירה וראשית יסודה הוא בבחינת הרוח\mycircle{°}, שהוא מדרגת האדם }\מקור{[ע״ר א רמט]}\צהגדרה{. }

\הגדרה{ע״ע בחירה צפונה. ע״ע בחירה כמוסה. }

\ערך{בחירה  גנוזה }\הגדרה{- ע״ע בחירה כמוסה. }

\ערך{בחירה חפשית }\הגדרה{- החופש\mycircle{°} הגמור, המחולל בקרבו רצון\mycircle{°} שאין בו שום מועקה מבחוץ, שעל ידו מתגלה העצמות\mycircle{°} של מהות החיים, המקבלים את התכנית של הגורל\mycircle{°} הטוב\mycircle{°} או הרע\mycircle{°}, שרק החלק הקטן (ממנה), המתגלה בתור בחירה מעשית (בחירה\hebrewmakaf גלויה\mycircle{°}), מתוה לפנינו בגלוי את ארחות הטוב והרע }\מקור{[עפ״י אג׳ א שיט]}\צהגדרה{. }

\תהגדרה{חופש הבחירה, חופש הפעולה של האדם. חוקי האפשר\mycircle{°}, כח ואפשרות בחירת מעשים בזולת מעשים ותועלתם }\תמקור{[נ״א ד 39].}

\הגדרה{ע״ע התגלות המהות.}

\ערך{בחירה כמוסה }\הגדרה{- הבחירה שאיננה על פי התוכן המוסרי\mycircle{°} המתגלה, אלא על פי האידיאל\mycircle{°} העליון\mycircle{°}, שעל פי הצפיה\hebrewmakaf העליונה\mycircle{°}, למעלה מהתנאים שההויה נמצאת בהם כעת. ההזרחות\mycircle{°} שבאות מתוכן זה הם אורות\mycircle{°} הנשמה\mycircle{°} הפנימית\mycircle{°} של כל היש, והן כוללות את העבר ההוה והעתיד, למעלה מסדר זמנים וצורתם, וכל זה כלול בשם\hebrewmakaf ההויה\mycircle{°}, כסדרו ובכל אופני צירופיו }\מקור{[א״ק ג כג (ע״ט ב)]}\צהגדרה{. }

\הגדרה{הבחירה שכל מערכת המשפט\mycircle{°} של כל היש מתנהגת על ידה }\מקור{[א״ק ג לד]}\צהגדרה{.}

\הגדרה{הבחירה שפרטי הכוחות והפרטיות הדקות שבמציאות בכלל (להוציא מדרגות ה׳שכר ועונש׳) נחלקים למדריגותיהם על פיה ע״פ היסוד ד״הכל צפוי״\mycircle{°} }\מקור{[עפ״י פנ׳ כג]}\צהגדרה{. }

\משנה{בחירה צפונה }\הגדרה{- יסוד כל חק ומשפט\mycircle{°}. הבחירה השמה את המערכות לפי מדרגותיהן, מגדולי המציאות עד קטניהם }\מקור{[עפ״י א״ק ג לד]}\צהגדרה{. }

\הגדרה{גלגול (הזכות או החובה), האמצעות של המסבבים את הדברים הרשמיים בהם נעוץ כח החפץ הגמור, הנכלל בכחות המציאות שלא לחסר ממנה את מושגי המוסר והרשע והצדק וכל העלילות הגדולות המסובבות מהם ועל ידם, באין שום גרעון, ״כי כל אשר יעשה האלהים הוא יהיה לעולם עליו אין להוסיף וממנו אין לגרוע והאלהים עשה שיראו מלפניו״\mycircle{°}. וההכרה המחברת את מושג המשפט הקבוע, עם מושג החופש\mycircle{°}, להתאימם עם העז\mycircle{°} והמשפט המלא את כל היקום, כי ״אלהים שופט צדיק״, היא }\משנה{הבחירה הכמוסה}\הגדרה{ הגלויה רק ליוצר כל במקור החכמה\hebrewmakaf האלהית\mycircle{°} }\מקור{[ע״א ג ב רד]}\צהגדרה{. }

\משנה{הבחירה הגנוזה}\myfootnote{ ע״ע הערת הרצי״ה אג׳ א עמ׳ שפה-שפו, לעמ׳ שיט.\label{39}}\הגדרה{ - הבחירה\hebrewmakaf החפשית\mycircle{°} הגמורה, שהיא עצם המהותיות שלנו, המיטב והעיקרי שבהויתנו, המתגלה בכל מלואה ועשרה רק לצפיה\hebrewmakaf העליונה }\מקור{[עפ״י אג׳ א שיט, שם ב מב, ע״ר ב קנז]}\צהגדרה{.}

\הגדרה{ע׳ במדור מונחי קבלה ונסתר, יובל, עלמא דיובלא.  ע׳ בנספחות, מדור מחקרים, ידיעה ובחירה.}

\ערך{בחירת ישראל }\הגדרה{-}\משנה{ מטרת בחירת\mycircle{°} ישראל על פי ד׳ בהתגלות אלהות על ידי אותות ומופתים גלויים}\הגדרה{ - כדי שיהיו מוכנים לקרבת\hebrewmakaf אלהים\mycircle{°} היותר נעלה, שהיא יסוד העליון והתכליתי למין האנושי, ושמשפע מוסרם בצירוף הכח האלהי שכבר נגבל בהיסתוריה הברורה שלהם ובארץ הקודש, לכשתצא מן הכח אל הפועל גדולתם ותפארתם כמו שראוי להיות לעם הנושא את היסוד היותר מעולה וכולל לכל המין האנושי בידו, שהוא יסוד הרוחניות של הרחבת דעת השם בחיים האנושיים, אז מאיליה תצא הפעולה לכל העולם ברב הוד והדר }\מקור{[ל״ה 168]}\צהגדרה{.}

\ערך{בטול }\הגדרה{- התכללות }\מקור{[עפ״י קובץ ה צה]}\צהגדרה{. }

\ערך{בטול}\myfootnote{ \textbf{הטעות שיש במהותיות עצמותית, והתגברות החפץ לאשתאבה בגופא דמלכא} - ע׳ לקוטי תורה לרש״ז, שיר השירים א. ״והנה כלה יש בו ב׳ פירושים. הא׳ לשון כליון וכו׳, והב׳ מלשון כלתה נפשי והיינו תשוקת הנפש לידבק וליכלל באורו ית׳״. \label{40}}\ערך{ }\הגדרה{- }\משנה{בטול אל האור\hebrewmakaf האלהי\mycircle{°}}\הגדרה{ - להשתאב\hebrewmakaf בגופא\hebrewmakaf דמלכא\mycircle{°} }\מקור{[עפ״י א״ק ב שצח]}\צהגדרה{. }

\הגדרה{כלות\hebrewmakaf הנפש\hebrewmakaf לאלהים\mycircle{°} }\מקור{[עפ״י שם, ע״ר א מז, סז]}\צהגדרה{. }

\משנה{ביטול גמור של מהות עצמו }\הגדרה{- הכרת הנשמה את כל הטעות שיש במהותיות עצמותית, והתגברות החפץ לאשתאבה בגופא דמלכא, בשלמות אין סוף של נועם העליון. הענוה\mycircle{°} הגמורה, והשפלות העמוקה, שהעצמיות היא בה רק שירים\mycircle{°}, כלומר ענין של חסרון שנשאר בלתי כלול בשלמות העליונה }\מקור{[עפ״י קובץ א שיב]}\צהגדרה{. }

\הגדרה{ע״ע כניעה, ההכנעה מפני האלהות. ע׳ במדור פסוקים ובטויי חז״ל, ונחנו מה. ושם, ואנכי תולעת ולא איש. ושם, כלה שארי ולבבי צור לבבי וחלקי אלהים לעולם.}

\ערך{בטול}\myfootnote{ ע״ע פנק׳ א תכד סי׳ מח.\label{41}}\ערך{ }\הגדרה{- }\משנה{בטול פנימי עדין }\הגדרה{- (בטול עצמי) המשפיל את הצד המכוער שבנו ומרומם את כל מהות הטוב והעדין}\צהגדרה{ <ואינו מטשטש את אומץ החיים> }\מקור{[עפ״י א״ש יד כא]}\צהגדרה{. }

\ערך{בטחון }\הגדרה{- דעת בבינת\hebrewmakaf לב\mycircle{°} פנימית ואדירה }\מקור{[עפ״י ע״א ג ב צ]}\צהגדרה{. }

\מעוין{◊}\הגדרה{ בא מתוך הדבקות\hebrewmakaf האלהית\mycircle{°}, הבאה מתוך האמונה\hebrewmakaf השלמה\mycircle{°} }\מקור{[עפ״י ע״ר ב פו]}\צהגדרה{. }

\מעוין{◊ }\משנה{הבטחון}\הגדרה{ כולל בתוכו את הדבקות האמונית בשלמותה, וממשיך את אור\hebrewmakaf החיים\mycircle{°} ממקור\hebrewmakaf החיים\mycircle{°}, מחי\hebrewmakaf העולמים\mycircle{°} ברוך הוא, לכל מי שמתעטר בקדושת האמונה והדבקות האמיתית }\מקור{[עפ״י שם]}\צהגדרה{.}

\ערך{בטחון}\myfootnote{ \textbf{יסוד הבטחון} - \textbf{הוא לא שהאדם בטוח שמה שהוא דורש ימלא ד׳, כי אפשר שמה שהוא חושב, שהוא הטוב, הוא ההפך מהאמת. אלא שהוא בטוח בחסד עליון וכו׳} - ע״ע ע״א ג א מג. צבי לצדיק לרי״מ חרל״פ פרק ד. להבחנת מדרגות הבטחון השונות ע׳ ע״א א א קמג, ע״א ג א מו, ובע״א ב ט קעג. ובקבצ׳ ב עמ׳ יח סי׳ טז: ״שיטת הלאומיות שאומרים וכו׳ שאין להשתמש בבטחון לענין לאומי״.\textbf{אדם הנברא בצלם\hebrewmakaf אלהים הוא תמצית כל ואחוד הכל, ומצד הכל הלא אין אבוד ולא הירוס} - ע״ע ע״ר א שלב (א״ק ב תקג). ע״ע ע״ר א קעג ד״ה וטעונה ״לא אבדת ציורים ורשומים פרטיים הוא הענין של מלוי הקדש, אלא הרמתם העשירה עם כל רכוש ציוריהם לרום מעלות הקדש״. ושם שם א ד״ה אני, שם שם רטז ד״ה ד׳ צבאות, שם שם מט ד״ה וצור חבלי, ושם שם רכב ד״ה דעו. ובע״ר ב סב ד״ה דרשתי שם שם סג ד״ה הביטו, שם שם קנז ד״ה חלק, ושם שם רנה.\textbf{תכלית הבטחון} - \textbf{האסונות, הנכונים לבא על בני אדם, באופן כזה שאין הזהירות האנושית יכולה להגן הרי הם סרים מן הבוטח }- ע״ע שם בע״א ג ב קצב, ובמשפט כהן עמ׳ שכז\hebrewmakaf ח, שנט. ובעזרת כהן עמ׳ קמא. ע״א א א קא ״גם על ההשגחה האלקית ראוי שיקבע בנפשו שלא יאתה להיות סומך כ״א במה שאין ידו מגעת להשתדל בעצמו״. ושם ח״ב ט קכ ״הבטחון הוא מוגדר כשנשלים את חק ההשתדלות במה שהוא בידינו, ובמה שאין יכולת שלנו מגיע(ה) לזה, שם הוא מקום הבטחון. כי במקום שהיכולת מתגלה חלילה להשתמש בבטחון, שאין זה בטחון כ״א הוללות ומסה כלפי מעלה״. ע״ע שם שם כג, ושם ג ב קצו. ע״ע רמב״ם, פיהמ״ש פסחים נו., עקדה שער כו, בראשית דף רכא.\hebrewmakaf רכו:, וברבנו בחיי עה״ת, שמות יג יח. ובבאור הגר״א על משלי יד טז, (מהד׳ פיליפ) עמ׳ 173 ד״ה וכסיל מתעבר ובוטח ״הכסיל עובר במקום שיש לטעות או במקום סכנה ובוטח בה׳ שלא יבא לידי רע, והוא בטחון הכסילים כי מי מכריח אותו לילך במקום סכנה״. (אך ע׳ שם ג ה, עמ׳ 49 50 ובהערה 24 שם וצ״ע, ואולי י״ל עפ״י דבריו שם טז כ, עמ׳ 197, בחלוק בין עוה״ז ועוה״ב, ותורה ותפילה. מ״מ, אין שיטה זו עולה בקנה אחד עם שיטת האמונות ודעות לרס״ג י טו שאומר על מי שאומר שבטוח בד׳ על עניני עוה״ז בלא השתדלות, שהיא דעה זרה, דא״כ יאמר ג״כ על עניני עוה״ב, ומה תכלית התוהמ״צ. ואמנם, כבר נחלקו בענין זה אבות העולם ראשונים ואחרונים. וע׳ באלפי מנשה, ח״א, פרק צח. ובמערכתו של ר״ש מאלצאן, אבן שלמה, ליקוטים בסוף הספר דף סט.\hebrewmakaf עא. ואא״ל על פיהם). ע״ע אבן ישראל ח״ג, בהקדמה.\label{42}}\הגדרה{ - }\משנה{יסוד הבטחון המעלה את האדם לתכונת קדושה\mycircle{°} עליונה, רוממות נפש וגדולת קדש }\הגדרה{- <יסוד בטחון זה, הוא לא אותו הציור\mycircle{°}, שהאדם יצייר בעצמו שהוא בטוח, שמה שהוא דורש ומבקש, וחושב שדרוש לו, ימלא ד׳\mycircle{°}, כי אפשר שמה שהוא חושב, שהוא הטוב, הוא ההפך מהאמת. אלא> שהוא בטוח בחסד\mycircle{°} עליון\mycircle{°}, שברא את העולם ובנה אותו, וכוננו, ומשגיח עליו ברב חסד, ועל כן אין מקום לשום דאגה, לשום עצבון רוח, כי הלא יודעים אנו, שחסד אל נטוי על כל יצוריו, והננו נכנסים תחת כנפי חסדו בכל רגע }\מקור{[ע״ר א רכ]}\צהגדרה{.}

\משנה{יסוד (הבטחון ב)שם\hebrewmakaf ד׳\mycircle{°}}\הגדרה{ - לבטוח שההנהגה עוזרת לקנות השלמות האמיתית }\מקור{[עפ״י ע״א א א נג]}\צהגדרה{. }

\משנה{יסוד הבטחון והשמחה }\הגדרה{- נובע מהבירור הפנימי שאין לחפוץ, גם לעניני עצמו, כ״א את מה שהוא חפץ אדון כל ב״ה. ואז ימלא אדם שמחה ואומץ לב כפי מדת בירור דבר זה בלבבו, ולפי ערך התאמת כל ארחות חייו לזאת המדה העליונה }\מקור{[פנ׳ לד]}\צהגדרה{. }

\משנה{בטחון נשגב\mycircle{°} עליון\mycircle{°}}\הגדרה{ - החסיון\mycircle{°} האידיאלי\mycircle{°} הבא מתוך ההופעה\mycircle{°} העליונה, בלא שום מבט על הגורל\mycircle{°} הנופל בחלקה של האישיות הפרטית. כי מתוך השגוב העליון והזיו\hebrewmakaf האלהי\mycircle{°}, של מקור כל השלמות ושורש כל תענוג\mycircle{°} ואור\mycircle{°}, הכל מתבטל\mycircle{°} מרוב נועם\mycircle{°} }\מקור{[עפ״י ע״ר ב עד]}\צהגדרה{. }

\הגדרה{הבהירות של הידיעה האלהית העליונה וחשק הלב הפנימי בהתמלאותם של האידיאלים\hebrewmakaf האלהיים\mycircle{°} במלא כל היש, והבירור הגמור שכן הוא, ושהכל הולך לחפץ הטוב העליון, (המביאים) שמחת הנשמה הפנימית, וכל דאגה עצבית מתגרשת, וחדות\mycircle{°} ד׳ מתמלאת בכל מהותו של אדם }\מקור{[עפ״י קובץ ה קכה]}\צהגדרה{. }

\משנה{יסוד הבטחון }\הגדרה{- יסוד הבטחון בא מתוך החסן\mycircle{°} אשר לנו באלהים\mycircle{°} סלה\mycircle{°}. כשאדם הנברא בצלם\hebrewmakaf אלהים\mycircle{°} הלא הוא באמת תמצית כל ואחוד הכל, ומצד הכל הלא אין אבוד ולא הירוס, לא השפלה ולא ירידה\mycircle{°}, כ״א כולו אומר כבוד\mycircle{°} וחיים\mycircle{°}, וכשהאדם מכיר את עוזו\mycircle{°} באלהי\hebrewmakaf הצבאות, ד׳\hebrewmakaf צבאות\mycircle{°}, הלא הוא מלא בטחון }\מקור{[ע״ר א קנ]}\צהגדרה{. }

\משנה{תכלית הבטחון }\הגדרה{- קרבת\hebrewmakaf אלהים\mycircle{°} הנמשכת מן הבטחון, וגבורת הנפש בעז\hebrewmakaf ד׳\mycircle{°} הנמשכת ממנה בעת צר, שהאדם מוצא לו תמיד מחסה\mycircle{°} בשם\hebrewmakaf ד׳\mycircle{°}, וגם בעת אשר כל המסיבות הטבעיות\mycircle{°} כבר חדלו כח להצילו מרעתו, עז\hebrewmakaf ד׳\mycircle{°} ישגבהו\mycircle{°} תמיד, והאסונות, הנכונים לבא על בני אדם, באופן כזה שאין הזהירות האנושית יכולה להגן, הרי הם סרים מן הבוטח, כשם שרגשי הפחד הדמיוני סרים מפני האור\mycircle{°} של הבטחון, ונפשו מלאה אומץ, וסדר שלותי קבוע בה }\מקור{[עפ״י ע״א ג ב קצב, ע״ר ב עה]}\צהגדרה{. }

\משנה{הבטחון מדתו }\הגדרה{- הרחבת כח העז והגבורה, אפילו במה שהוא למעלה מגבולי כח היכולת הקבועה בכחות האדם הגלויים, כי אין מעצור לד׳ להושיע ולעזור גם לאין כח. }\צהגדרהמודגשת{מדת הבטחון }\צהגדרה{באה לאחר שיאזר האדם בגבורה בכל אשר תשיג ידו בכחותיו החומריים והרוחניים, ובבאו לגבול ששם יש לפניו מעצור כח המוגבל החלש, אל יפול לבבו. <וזאת היא }\צהגדרהמודגשת{מדת הבטחון}\צהגדרה{ שצריכה להתחבר תמיד עם מדת הגבורה, המועלת למלא את נפש האדם כבוד ועז. וכשהיא מתחברת עם הדעה השלמה והמוסר האמיתי, היא מדרכת את האדם בדרך ד׳ העליונה, ומכשרתו להיות בד׳ מבטחו. ומעלתו, שיהיה כבוד ד׳ חופף עליו לעשות לו ניסים, בין גלויים בין נסתרים, במערכות סדרי הטבע> }\מקור{[עפ״י ל״ה 177 (פנק׳ ב קכא)]}\צהגדרה{.}

\צהגדרה{יסוד החיים הלא הוא הכח לפעול ולעשות, כל איש לפי ערכו, וכל חברה לפי ערכה. החיים הטובים המה, שתהיינה הפעולות מסודרות יפה ועולות תמיד במעלה בהוספת ערך והשלמה. והנה האדם הוא איננו חפשי גמור, פעמים רבות יתיצבו לו כצר מונעים רבים שיעכבוהו שלא יוכל ללכת מהלך החיים שלו, שלא יוכל לפעול לפי תכונתו וערכו, ואז הוא צריך להתגבר עליהם בכל עז. ועל זה צריך }\משנה{שיבטח בד׳, }\הגדרה{שאם אפילו כחותיו לא יספיקו לו, מכל מקום ״אין מעצור לד׳ להושיע״, ותשועת ד׳ תשגבהו להסיר המניעות, למען יוכל לפעול ולעבוד ולחיות כראוי}\צהגדרה{ }\מקור{[עפ״י ל״ה 240]}\צהגדרה{. }

\הגדרה{ע׳ בנספחות, מדור מחקרים, בטחון לעומת אמונה.}

\ערך{בטלה - (הממיתה את הנשמה)}\myfootnote{ פנק׳ ד קלו: ״ע״י מה שנוטין לצד הגס של תאות החושים, מסתתמים כל הצינורות של ההארה הרוחנית, ואין האור של הרצון הטוב נקלט בתוך הלב״.\label{43}}\ערך{ - התעסקות בדברים חומריים\mycircle{°} וגסים\mycircle{°}}\צהגדרה{ }\מקור{[פנק׳ ד רז]}\צהגדרה{.}

\ערך{בי }\הגדרה{- מבטא, מבליט, את המהותיות הפנימית של הנושא, המכריז על עצמו את התגלותו העצמית, ומודיע את הגנוז בקרבו }\מקור{[ר״מ קלב]}\צהגדרה{.}

\ערך{בי }\הגדרה{- הבעה הבאה מתוך מעמקי הנשמה\mycircle{°}, הריכוז היותר פנימי\mycircle{°} ויותר כללי\mycircle{°}}\צהגדרה{ }\מקור{[ר״מ קלב]}\צהגדרה{.}

\ערך{בי }\הגדרה{- מורה על העצמיות\mycircle{°} המיוחדה של האדם ותוכן החיים הטבעי שלו, שהיא הבסיס לקבל עליה את האור\mycircle{°} העליון\mycircle{°} של הנשמה\mycircle{°} }\מקור{[ע״ר א ג]}\צהגדרה{. }

\הגדרה{התוכן המורגשי של האדם בהויתו הפרטית. אותו תוכן שיש עמו ג״כ חבור להצד הפרטי, המסמן את הפירוט היחידי של האדם באשר הוא מוגבל ומצומצם }\מקור{[שם סז]}\צהגדרה{. }

\ערך{ביזה }\הגדרה{- מה ששוללים דרך מלחמה }\מקור{[ר״מ קלא]}\צהגדרה{. }

\ערך{בית דין הגדול }\הגדרה{- מרכזנו הדתי, היסוד העיקרי לביאור התורה לפרטיה הנולדים בהמשך החיים, היושב ״במקום אשר יבחר ד׳״, שמשם הוראה צריכה לצאת לכל ישראל }\מקור{[א״ה ב (מהדורת תשס״ב) 127]}\צהגדרה{.}

\ערך{בית הגדול }\הגדרה{- }\משנה{הבית הגדול }\הגדרה{- כל העולמים כולם בכל הדר\mycircle{°} כונניותם\mycircle{°} }\מקור{[ר״מ קמג]}\צהגדרה{.}

\הגדרה{ע׳ במדור פסוקים ובטויי חז״ל, בית ד׳.}

\ערך{בית הכנסת }\הגדרה{- מקום הקיבוץ הציבורי לעבודת\mycircle{°} השי״ת\mycircle{°} }\מקור{[ע״א א א נו]}\צהגדרה{.}

\הגדרה{המכון לקיבוץ עבודת השי״ת והרמת כח האמונה\mycircle{°} ויראת\hebrewmakaf ד׳\mycircle{°} בלבבות }\מקור{[ע״א ג א כה]}\צהגדרה{.}

\הגדרה{(בית) שתעודתו היא עבודת השם ית׳, <שהוא המקום\mycircle{°} היותר גבוה שבחיים, שכל פינות החיים הפרטיים כולן אליו יפנו וע״י יתעלו ויתרוממו}\צהגדרה{> }\מקור{[עפ״י שם]}\צהגדרה{.}

\צהגדרה{הבית של ההתכנסות הפנימית לשם ד׳\hebrewmakaf אלהי\hebrewmakaf ישראל\mycircle{°}, להופעת רוחו ושייכות מצוותו\mycircle{°} }\צמקור{[ל״י א נו].}

\ערך{״אחורי בית הכנסת״}\myfootnote{ ברכות ו:.\label{44}}\ערך{ }\הגדרה{- }\משנה{באיכותו וערכו}\הגדרה{ - הצד הטפל של בית הכנסת - השגת המבוקש בתפילה\mycircle{°} }\מקור{[עפ״י פנק׳ ג ער]}\צהגדרה{. }

\ערך{פנים בית הכנסת}\הגדרה{ - }\משנה{באיכותו וערכו}\הגדרה{ - החלק העיקרי של בית הכנסת - תכליתו להרבות כבודו\mycircle{°} של השי״ת בלב כל הנכנסים בתוכו, ולתכלית זו באה התפילה\mycircle{°}}\צהגדרה{ }\מקור{[עפ״י פנק׳ ג ער]}\צהגדרה{.}

\הגדרה{ע׳ בנספחות, מדור מחקרים, רנה ותפילה. ע׳ במדור פסוקים ובטויי חז״ל, המתפלל אחורי בית הכנסת נקרא רשע. ושם, מסיר אזנו משמוע תורה גם תפילתו תועבה. }

\ערך{בית הכסא }\הגדרה{- מקום התגלות שפלות החומריות האנושית מצד עולמו הפנימי }\מקור{[ע״א ג א יג]}\צהגדרה{.}

\מעוין{◊}\הגדרה{ האמצעי לטהרה\mycircle{°} של הלכלוך הטבעי המחליא באי נקיונו בהשארתו בגויה }\מקור{[ע״א ג ב עג]}\צהגדרה{.}

\הגדרה{ע׳ במדור מלאכים ושדים, שעיר של בית הכסא.}

\ערך{בית המרחץ }\הגדרה{- מקום התגלות שפלות החומריות האנושית מצד עולמו החיצוני }\מקור{[ע״א ג א יג]}\צהגדרה{.}

\ערך{בך }\הגדרה{- מבטא את היחש החודר בפנימיותו של נושא חוצי, העומד לנכח הנושא העצמי, המביע את הרעיון }\מקור{[ר״מ קלב]}\צהגדרה{.}

\ערך{בכור }\הגדרה{- }\משנה{(ענינו) }\הגדרה{- היסוד הראשי של משך החיים}\צהגדרה{ }\מקור{[עפ״י ע״ר א מב]}\צהגדרה{.}

\ערך{בכורה }\הגדרה{- תכונה, שמחיבת להיות משפיע ופועל פעולה חנוכית על יתר הבנים והבנות, שבאים אחריו }\מקור{[ע״ר א קו]}\צהגדרה{.}

\הגדרה{ע׳ במדור מצוות, הלכות, מנהגים וטעמיהן, בכורות, קדושת הבכורות. }

\ערך{בל }\הגדרה{- הוראת שלילה }\מקור{[ר״מ קלג]}\צהגדרה{. }

\ערך{בם }\הגדרה{- מורה חדירת הנושא בתוך התוכנים הרבים העומדים בריחוק מקום מהנושא המתאר }\מקור{[ר״מ קלג]}\צהגדרה{. }

\ערך{בן }\הגדרה{- התולדה האיתנה, העובדת והמסדרת, היורשת את ההארות\mycircle{°} העליונות שהן הן הגורמות את החידוש התולדתי }\מקור{[ר״מ קלד]}\צהגדרה{. }

\ערך{בן}\הגדרה{ - }\מעוין{◊ }\הגדרה{המיוחש לאב\mycircle{°} ביחש הקשר הנשמתי היותר חזק }\מקור{[עפ״י ע״ר א פו]}\צהגדרה{.}

\הגדרה{ע׳ במדור פסוקים ובטויי חז״ל, נחלת ד׳ בנים. }

\ערך{בן}\הגדרה{ - }\משנה{״בנים״}\הגדרה{ - הצעירות, הצריכה לקבל את השפעתו\mycircle{°}, של האב\mycircle{°} הגדול }\מקור{[ע״ר ב סה]}\צהגדרה{.}

\ערך{בן }\הגדרה{- }\משנה{(בעבודת ד׳ לעומת עבד) }\הגדרה{- ע׳ במדור מדרגות והערכות אישיותיות. }

\ערך{בסום }\הגדרה{- }\משנה{התבסמות העולם}\myfootnote{ בש״ק, קובץ א קט: ״\textbf{התבסמות} העולם ע״י כל המשך הדורות, ע״י \textbf{ביסום} היותר עליון של גילויי השכינה בישראל וע״י נסיונות הזמנים, התגדלות היחש החברותי, והתרחבות המדעים, זיקקה הרבה את רוח האדם, עד שאע״פ שלא נגמרה עדיין טהרתו, מ״מ חלק גדול מהגיונותיו ושאיפת רצונו הטבעי הנם מכוונים מצד עצמם אל הטוב האלהי״. ושם שצד: ״העולם ב\textbf{התבסמותו} הולך הוא ומתעלה בתוכיותו. האדם מוצא את חפצו, הולך וטוב בערכו הפנימי״.\label{45}}\הגדרה{ - תיקון\mycircle{°} והתעלות }\מקור{[עפ״י פנק׳ ג שט, שי]}\צהגדרה{.}

\ערך{בסומה של הנשמה}\הגדרה{ - שאיבתה ממעין הקדושה\mycircle{°} האלהית הפנימית. סוד הקדושה האצילית\mycircle{°} הפנימית, הופעת הנשמה }\מקור{[עפ״י ע״ר א קנג]}\צהגדרה{.}

\הגדרה{ע״ע מתבסם.}

\ערך{בצבוץ }\הגדרה{- תיאור לכל רושם צמחני }\מקור{[עפ״י ר״מ קלו]}\צהגדרה{. }

\ערך{בצבוץ }\הגדרה{- ההפריה המתבודדת בחוגיה, וההתגלות הקולית }\מקור{[ר״מ קלו]}\צהגדרה{. }

\ערך{בִּצָה }\הגדרה{- מקום המוכשר לגידול, מסמל את הבסיס הדוגמתי בהתוכנים הרוחנים\mycircle{°}, בגליפת המושגים, במהות השכלתם וציור\mycircle{°} אמיתת צדקם\mycircle{°}, המשפיעים על היסוד המעשי, ערכי הצדק\mycircle{°} והמישרים\mycircle{°} }\מקור{[ר״מ קלו]}\צהגדרה{. }

\ערך{בק }\הגדרה{- ענין של התרוקנות }\מקור{[ר״מ קלו]}\צהגדרה{. }

\ערך{בֹּקֶר }\הגדרה{- עת\mycircle{°} ההזרחה\mycircle{°} של האורה\mycircle{°} האמיתית, אשר תביא להכרת החיים במהותם העצמית }\מקור{[ע״ר ב עג]}\צהגדרה{. }

\ערך{בקורת }\הגדרה{- }\משנה{מטרת הבקורת }\הגדרה{- להגיה אורות מאופל. לברר בחופש והרחבה, על צד השקר המועט, המוכרח להמצא בתוך האמת הגדולה והמרובה, ועל ניצוץ האמת המתגלה בתוך עומק החושך של השקר }\מקור{[מ״ר 288]}\צהגדרה{.}

\ערך{בקשת אלהים }\הגדרה{- דרישת חיים של אמת פנימית }\מקור{[עפ״י קובץ ד פז]}\צהגדרה{.}

\הגדרה{ע׳ במדור פסוקים ובטויי חז״ל, דרישת ד׳.}

\ערך{בר }\הגדרה{- המזון המבריא בכל הערכים }\מקור{[ר״מ קלז]}\צהגדרה{. }

\ערך{בר }\הגדרה{- הבנה של חוצה, העומד(ת) מחוץ להפרגוד\mycircle{°} אשר חביון\hebrewmakaf עז\mycircle{°} קודש הקדשים של הדממה העליו(נה) אצור שמה }\מקור{[עפ״י ר״מ קלח]}\צהגדרה{. }

\ערך{בר }\הגדרה{- תרגום בן\mycircle{°} }\מקור{[עפ״י ר״מ קלח]}\צהגדרה{. }

\ערך{ברה }\הגדרה{- מנצחת את כל צללי המחשכים }\מקור{[ע״ר ב נז]}\צהגדרה{. }

\ערך{ברוך }\הגדרה{- }\משנה{(משמעותו בברכת\hebrewmakaf המצוות\mycircle{°})}\הגדרה{ - מקור\hebrewmakaf חיי\hebrewmakaf החיים\mycircle{°}, אוצר הטוב והקודש, ששפעת כל ברכת ההויה שמה היא גנוזה, המוער בבאנו להוציא מן הכח אל הפועל את האור הקדוש של מצוה\mycircle{°} מעשית, בהתגלות המפעלית, ובהארת היפעה האלהית, הנובעת מראש מקור אור החיים העליונים של חי\hebrewmakaf העולמים\mycircle{°}, מתמשכת אז שפעת ברכה\mycircle{°}, ההולכת ומפלסת לה את נתיבה בהתחשפות האורה של החיים המעשיים, ומעין החיים מתברך\mycircle{°} במקורו, בהיותו מוכן להתגבר בשטף ברכותיו ע״י אותו השביל החדש, ההולך ומתבלט ע״י מפעלנו במעשה המצוה הבאה ממרום החפץ האלוהי העליון, מקור החיים והטוב, אל תחתית מעמקי העולם, הנמצר במצריו החמריים וכוחותיו המעשיים }\מקור{[עפ״י ע״ר א ז]}\צהגדרה{. }

\ערך{בריאה }\הגדרה{- }\משנה{הבריאה }\הגדרה{- הממשיות המוגלמת }\מקור{[א״א 125]}\צהגדרה{. }

\ערך{בריאה}\myfootnote{ הזכיר הרב מרדכי גלובמן בנ״א ה עמ׳ 22 מדברי אבן עזרא, בראשית א א ״רובי ממפרשים אמרו שהבריאה להוציא יש מאין וכן ״אם בריאה יברא ה׳״. והנה שכחו ״ויברא אלהים את התנינים״ ושלש בפסוק אחד ״ויברא אלהים את האדם״. ו״ברא חשך״ שהוא הפך האור שהוא יש. ויש דקדוק המלה ברא לשני טעמים: זה האחד, והשני ״לא ברה אתם לחם״. וזה השני ה״א תחת אלף כי כמוהו ״להברות את דוד״ כי הוא מהבנין הכבד הנוסף ואם היה באל״ף היה כמו ״להבריאכם״ ומצאנו מהבנין הכבד ״ובראת לך״. ואיננו כמו ״ברו לכם איש״ רק כמו  ״וברא אותהן בחרבותם״ (יחזקאל כג מז). וטעמו לגזור ולשום גבול נגזר והמשכיל יבין״. וכן כתב הרמ״ק בפרדס שער אבי״ע פרק א ״בריאה מלשון ברא [...] כונתו חוץ, או מלשון כריתה, כמו ״כי יער הוא ובראתו״ (יהושע יז)״. ובראש אמנה לאברבנאל: ״ברא״ הונח בהנחה ראשונה יש מאין, ומזה הושאל על כל בריאה ניסית או נשגבה היוצאת מגדר הטבע. אמנם גם פירושי ראשונים אלו לא יעמדו במבחן דברי האדרת אליהו, בראשית א א, ד״ה ברא ״הבינו כל מפרשי הדת שמורה על דבר מחודש יש מאין. אבל מה יאמרו ״ויברא אלהים התנינים הגדולים״ וכן ״ויברא אלהים את האדם בצלמו״ וכן מה שתקנו קדמונינו בכל ברכת הנהנין ״בורא פרי האדמה״, ״בורא פרי העץ״, ונשאר כללם הידוע מעל״. על כן פירש הגר״א שם ״מלת בריאה הונח להורות על חידוש העצם אשר אין בכח הנבראים אפי׳ כולם חכמים ונבונים לחדשו... וכן תיקנו ״בורא פרי״ כי אינו בכח כל הנבראים לחדשו בעבור שהוא עצם פועל ה׳״. ״ברא הוא עצם הדבר ואפילו יש מיש״. ובמטפחת ספרים ליעב״ץ, פרק ח ד ״לשון בריאה מורה על יש מיש על דרך האמת. ויתכן גם בריאה אין מיש״ וכו׳ עש״ע שהאריך, (ע״ע ע״ט לב ד״ה לא). וע׳ בדרך חיים למהר״ל, רי ושם שכא ״לשון בריאה נאמר על הצורה הנבדלת האלקית שדבק בנבראים, וזה כי האדם כתיב בפי׳ בצלם אלקים עשה את האדם, שתדע מזה כי דבק בצורת האדם ענין אלקי, וכן בשמים וארץ שהם כלל העולם, אין ספק שדבק בהם ענין אלקי ולכך כתיב לשון בריאה. וכן התנינים הגדולים שהכתוב מפרש שהם תנינים גדולים, ולפי גדלם עד שהם בריאה נפלאה דבק בהם ענין האלקי נאמר אצלם לשון בריאה. כי כל הנבראים יש בהם דבר זה כמו שיתבאר רק התורה הזכירה לשון בריאה באלו שלשה, כי באלו שלשה מפורסם ונראה לגמרי לעין ובשאר דברים אינו נראה״. והרש״ט גפן, בממדים, הנבואה והאדמתנות, תורת הנבואה הטהורה, מאמר שני, עיון בנבואה ובמופתים, פרק יג הגדיר בריאה: ״יציאת היש ממה שאיננו נופל תחת הציור באופן בלתי מובן ובלתי מושג לא לשכל ולא לחוש ומבלעדי כל הכרח״. עע״ש פרק יד. ושם, מעשה בראשית והאדמתנות, סוף דבר, א, הגדיר: ״הבריאה הוא השתכללות צורות הזמן והמקום על פי כוח נסתר ונעלם, בדעת האדם והכרתו״. עע״ש הערה 11. ובקסת הסופר לר״א מרקוס, בראשית א א ״ברא - הוציא יש מאין שלא כפי טבע הנברא״, עע״ש עמ׳ ב-ד בהרחבת דברים נפלאה. ע״ע מנֹפת צוף, למו״ר הרב יהונתן שמחה בלס, ח״ב עמ׳ 825 ״המונח ״ברא״ ראוי להמצאת מציאות ראשונית שלאחר מכן נותרה כבררת מחדל״. כדרכו של הרב ברב דבריו, על פי הסברו בסוגיה יעלו כל הפירושים בקנה אחד.\label{46}}\הגדרה{ - התהוות\mycircle{°} העולם וכל אשר לו מאותו החפץ\mycircle{°} הקדום\mycircle{°}, המלא עז\mycircle{°}, המעוטר בגבורה\mycircle{°} ובכל אור\mycircle{°} קדשי\hebrewmakaf קדשים\mycircle{°}, עדינות הטוב\mycircle{°}, החסדים\mycircle{°} הנאמנים עדי\hebrewmakaf עד\mycircle{°} }\מקור{[א״ק ג ע]}\צהגדרה{. }

\הגדרה{היש המצומצם שאנו פוגשים בציור\mycircle{°} של הויה, שבחופש\mycircle{°} ולמטרה ידועה, נלחצה בצמצומה}\צהגדרה{ }\מקור{[עפ״י קובץ ז קנא]}\צהגדרה{.}

\הגדרה{היצירה המוחלטה. היכולת הבלתי תנאית ממציאה הכל, על\hebrewmakaf פי היסוד החפצי}\צהגדרה{ }\מקור{[עפ״י קובץ ה קפה]}\צהגדרה{.}

\צהגדרה{הוצאת יש מאין}\צמקור{ [ק״ת עז].}

\משנה{בריאת העולם }\הגדרה{- הופעת\mycircle{°} האור של הקדושה\mycircle{°} העליונה\mycircle{°} התורנית\mycircle{°},  בתור אור\hebrewmakaf החיים\mycircle{°} של קבלת\hebrewmakaf מלכות\hebrewmakaf שמים\mycircle{°}, של כבוד\hebrewmakaf המלכות\mycircle{°}, המאיר בהויה והיצירה כולה }\מקור{[עפ״י ע״ר א קיא]}\צהגדרה{. }

\תערך{בריאה ראשונה }\הגדרה{- }\תמשנה{״בראשית ברא״ }\הגדרה{-}\תהגדרה{ (התהוות) שלא על דרך השתלשלות\mycircle{°} אלא בכונה ראשונה }\תמקור{[עפ״י נ״א ה 22-21]. }

\הגדרה{ע״ע נברא. ע״ע מחשבה אלהית על דבר העולם. ע׳ בנספחות, מדור מחקרים, תכלית הבריאה. ע׳ במדור שמות כינויים ותארים אלהיים, בורא. ר׳ יצירה. ר׳ עשיה.}

\ערך{בריאה }\הגדרה{- }\משנה{עולם הבריאה }\הגדרה{- ע׳ במדור מונחי קבלה ונסתר. }

\ערך{בריאות }\הגדרה{- המצב הטוב המסכים אל כלל המציאות, מבלי שיופרע הסדר ביציאת פרט אחד מהסכמתו אל הכלל }\מקור{[עפ״י ע״א א ה נח]}\צהגדרה{. }

\ערך{בריאות רוחנית}\הגדרה{ - }\משנה{הבריאות הרוחנית}\הגדרה{ - ההרגשות הנפשיות כולן, במצבן הנורמלי. הרגשת היופי, האהבה, נטיית הגבורה, חשק החיים הבריאים}\צהגדרה{ }\מקור{[עפ״י קבצ׳ ב קכז]}\צהגדרה{.}

\ערך{בריקה }\הגדרה{- פעולת התקפה (רוחנית) חזקה בפתאומיותה }\מקור{[רצי״ה א״ש ב הערה 3]}\צהגדרה{. }

\הגדרה{ע׳ בנספחות, מדור מחקרים, אור, זיו, ברק. }

\ערך{ברירות }\הגדרה{- הזיכוך המחשבי והמעשי }\מקור{[ר״מ קלח]}\צהגדרה{. }

\ערך{ברית }\הגדרה{- קשר שכלי, נמוסי או טבעי, בין שני נושאים }\מקור{[עפ״י ע״ר א שפד]}\צהגדרה{. }

\הגדרה{ע״ע אמונה בברית. ע׳ במדור פסוקים ובטויי חז״ל, זכירת הברית. }

\ערך{ברית }\הגדרה{- }\משנה{שורש ברית, וכריתת ברית במובן המוסרי }\הגדרה{- שהענין החיובי ואידיאלי\mycircle{°}, הנובע מתמצית המוסר\mycircle{°} היותר נעלה ונשגב, יהיה מוטבע עמוק וחזק בכל טבע הלב והנפש, עד שלא יצטרך לא זירוז ולא חזוק ולא סייג לשמירתו, כי\hebrewmakaf אם יהיה מוחש וקבוע, כמו שטבועה, למשל, בלב אדם ישר\mycircle{°} מניעת רציחה וכדומה מן השלילות [של] הרעות שכבר הספיק כח המוסר הכללי לקלטן יפה }\מקור{[מ״ה ברית א (פנ׳ ה)]}\צהגדרה{. }

\ערך{ברית }\הגדרה{- }\משנה{יסוד הברית }\הגדרה{- הפעולות המוסריות\mycircle{°}, ביחוד הדתיות, המכוונות לכבד את ד׳\mycircle{°} לפי ציור\mycircle{°} המדמה\mycircle{°}. השגחת\hebrewmakaf ד׳\mycircle{°}, שימצאו דתות לכל אומה, המחזקות את הצדק בעולם. ושתמצא בהן אחת יסודית, שמחזקת מעוז הציורים האמיתיים, ומקשרתם אל השכל המעשי }\צהגדרה{-}\הגדרה{ תורת\hebrewmakaf ישראל\mycircle{°} המאירה באורה הפנימי בבית ישראל ומפיצה קרנים ג״כ לבני נח. והוא אות ברית בין אלקים ובין האדם בכללו, שמונע עכ״פ מהשחתה }\מקור{[עפ״י פנק׳ א קמד (קבצ׳ א נז)]}\צהגדרה{. }

\הגדרה{ע׳ במדור מונחי קבלה ונסתר, קשת. }

\ערך{ברית }\הגדרה{- }\משנה{פגם הברית }\הגדרה{- ע׳ במדור הנטייה המינית.}

\ערך{ברית }\הגדרה{- הטבע של קדושת היהדות\mycircle{°}. עצם ההויה הנפשית והטבע הרוחני וגם הגופני, של הכלל\mycircle{°} כולו ושל כל אחד ואחד מישראל. הטבע היהדותי במעשה, ברעיון, ברגש ובמחשבה, ברצון ובמציאות }\מקור{[עפ״י א״ש יז ד]}\צהגדרה{.}

\משנה{קדושת הברית }\הגדרה{- אור הטבע הישראלי הנקי }\מקור{[א׳ מד]}\צהגדרה{.}

\הגדרה{ע׳ במדור פסוקים ובטויי חז״ל, הפרת ברית. ע׳ במדור מלאכים ושדים, אליהו. }

\ערך{ברית }\הגדרה{- המושג העצמי של התוכן אשר לנצח\mycircle{°} העומד למעלה מכל מושג מוסבר באיזה הגיון\mycircle{°} מוגבל }\מקור{[ע״ר א רב (ע״א ב ט קנז)]}\צהגדרה{. }

\מעוין{◊ }\הגדרה{הברית מיוסדת על תוכן קים, שאיננו נופל תחת שום שינוי }\מקור{[שם צז]}\צהגדרה{. }

\ערך{ברית }\הגדרה{- }\משנה{ תוכן הברית }\הגדרה{- הזכרון\mycircle{°} העולמי שאינו סובל שום הגבלה ציורית\mycircle{°} כלל }\מקור{[עפ״י שם רב]}\צהגדרה{. }

\הגדרה{ע׳ במדור מונחי קבלה ונסתר, רזא דברית.}

\ערך{ברית }\הגדרה{- }\משנה{(לעומת חסד\mycircle{°}) }\הגדרה{- מעלת בטחונו וחוזק מציאותו, של כל דבר נעלה בחיי\hebrewmakaf הרוח\mycircle{°} המתפשט במציאות }\מקור{[עפ״י ע״ר א פג]}\צהגדרה{. }

\הגדרה{הודאיות\hebrewmakaf המוחלטת\mycircle{°} }\מקור{[עפ״י א״ק א רז, ע״ר א פג\hebrewmakaf פד, רב]}\צהגדרה{. }

\הגדרה{ע׳ במדור מונחי קבלה ונסתר, אחרית, לעומת הראשית בחיי הרוח. ע׳ במדור פסוקים ובטויי חז״ל, ברית עולם. ושם, נתתי את תורתי בקרבם ועל לבם אכתבנה.}

\ערך{ברית }\הגדרה{- }\משנה{הברית שכרת ד׳ עם ישראל }\הגדרה{- שאי אפשר כלל שיהיה ח״ו כלל\hebrewmakaf ישראל נבדל ונפרד מקדושת שמו\mycircle{°} הגדול ב״ה }\מקור{[מ״ש שיד (מא״ה ג רג)]}\צהגדרה{. }

\משנה{כריתת ברית שכרת השי״ת עם ישראל }\הגדרה{- שאע״פ שהזמן פועל שינויים גדולים בעולם, ובני האדם הפועלים בזמן הם חפשים בבחירתם\mycircle{°} והענין ארוך מאד, א״כ היה נראה לכאורה שאפשר הדבר שיצאו הדברים בכללם חוץ למטרת החכמה\hebrewmakaf העליונה\mycircle{°} שכיון הבורא יתברך ח״ו, ע״י בני\hebrewmakaf אדם הפועלים שינויים רבים בבחירתם ע״י הזמן. ע״כ השי״ת בחר\hebrewmakaf בישראל\mycircle{°} וצוה אותם לקדש חדשים ושנים. פי׳ שע״י כחן של ישראל ופעולתן בעצמם ובעולם, תהי׳ ערובה בטוחה שכל הדברים יחזרו אל תכליתם, והזמן יפעול פעולה מקודשת\mycircle{°}, היינו פעולה המגעת אל התכלית העליונה שכיון השי״ת ולא פעולה של חול\mycircle{°} }\מקור{[מ״ש שנ]}\צהגדרה{. }

\הגדרה{ע׳ במדור פסוקים ובטויי חז״ל, ברית עולם. ע׳ במדור מועדים וחגים, קידוש הזמנים.}

\ערך{ברכה}\myfootnote{ רקאנאטי עה״ת עקב: ״הברכה היא אצילות תוספת המשכה מאפיסת המחשבה שהיא מקור החיים״. ובשל״ה עה״ת, וזאת הברכה, תורה אור, ד״ה וכבר כתבתי: ״ענין ברכה הוא התפשטות בשפע רב תמיד נצחי״. \label{47}}\הגדרה{ - תוספת אור\mycircle{°} ויתרון }\מקור{[עפ״י א״ק ב תקלד]}\צהגדרה{. }

\הגדרה{תוספת חיים עצמיים מקוריים }\מקור{[עפ״י שם רצד]}\צהגדרה{. }

\הגדרה{תוספת מעלה, הופעה\mycircle{°} ועליה\mycircle{°} }\מקור{[ע״ר א ריז]}\צהגדרה{. }

\הגדרה{ההוספה התמידית במעלה }\מקור{[שם]}\צהגדרה{. }

\הגדרה{התוספת התדירית, בשפעת\mycircle{°} אור הקדש\mycircle{°} וחיי האמת\mycircle{°} }\מקור{[שם]}\צהגדרה{. }

\הגדרה{התוספת וההגדלה }\מקור{[שם סב]}\צהגדרה{. }

\הגדרה{שפעת חידוש\mycircle{°} ומקור חיים }\מקור{[ע״א ד ט ק]}\צהגדרה{. }

\הגדרה{שפעת\mycircle{°} חיים טובים, נעימים ורעננים\mycircle{°} }\מקור{[עפ״י א״ק ג קפח]}\צהגדרה{. }

\הגדרה{שפע החיים, העז והעצמה }\מקור{[ע״ר א ריד]}\צהגדרה{. }

\הגדרה{השפעה והשלמה}\צהגדרה{ }\מקור{[עפ״י ע״א ג ב מט]}\צהגדרה{.}

\הגדרה{ענין הַבְרָכָה ובְּרֵכַת\hebrewmakaf מים המשקה את הארץ }\מקור{[מא״ה ד קסד]}\צהגדרה{.}

\הגדרה{ע״ע מתברך. ע״ע מברך. ע׳ במדור מונחי קבלה ונסתר, ״יחוד ברכה קדושה״. ע״ע קדושה.}

\ערך{ברכה }\הגדרה{- }\משנה{ברכה המושפעת מחסד\mycircle{°} אל עליון }\הגדרה{- האורות\hebrewmakaf העליונים\mycircle{°} כשהם מופיעים על הנשמה\mycircle{°}, על נשמת\hebrewmakaf הכלל\mycircle{°} ועל נשמת הפרט, המרחיבים את מהותה, מעצמים את הויתה, ומעלים אותה למרומי האושר\mycircle{°} הנצחי\mycircle{°} }\מקור{[עפ״י ע״ר א קנח]}\צהגדרה{. }

\ערך{ברכה }\הגדרה{- }\משנה{הברכה היסודית של העולם }\הגדרה{- ההתעלות\mycircle{°} התדירית של דעת\hebrewmakaf ד׳\mycircle{°}, בגדלה\mycircle{°}, ביפעתה\mycircle{°} ובטהרתה\mycircle{°} }\מקור{[ע״ר א נ]}\צהגדרה{. }

\ערך{ברכה }\הגדרה{- }\משנה{הברכה הכללית }\הגדרה{- הברכה הנכנסת בעומק הפנימי של החיים, ברכת שלום\mycircle{°} הפנימי שעל ידה ימצא האדם שהחיים המה טובים כשהם לעצמם, וממילא אין עמם מחסור כשהם מתמלאים עם הדרישות המעשיות }\מקור{[ע״א ג ב רכו]}\צהגדרה{.}

\הגדרה{ע׳ ברוך.}

\משנה{ברכה }\צהגדרה{- }\מעוין{◊ }\צהגדרה{הזכרת השם\mycircle{°} היא היסוד הפנימי השרשי של הברכה, והמלכות\mycircle{°} היא מהותה העצמית הממשית}\צמקור{ [א״ל קיט].}

\ערך{ברכה לעומת הודאה }\הגדרה{- ע׳ בנספחות, מדור מחקרים. }

\ערך{״ברכה לצורך״ }\הגדרה{- }\מעוין{◊}\הגדרה{ ההופעות\mycircle{°} העליונות\mycircle{°}, המופעות בנשמתנו מעולמי התעלומה, תפקידן הוא לרומם בנו את התוכן המהותי של כל עצמות חיינו אל רום הנצח\mycircle{°}, אל ההוד\mycircle{°} האלהי הנשגב ברוממות קדשו. ואם יופיעו המון הארות\mycircle{°}, ורכוז לא יהיה להם בעצמיות מהות החיים שלנו, הרי הן לנו כאבודות. על כן אין לנו רשות לברך ברכה כי אם לצורך, וברכה שאינה צריכה, ומה גם ברכה לבטלה\mycircle{°}, הרי היא לנו נשיאת עון וחלול שם שמים. הבהקת האורה הרוחנית מתגברת היא בתעצומתה על ידי הברכה, מתאדרת היא ההארה הרוחנית בנשמתנו מעולם הנעלם, בא לידי גלוי ע״י בטוי הברכה המון רב של מחשבות רוממות וציורים נאדרים בקודש. ומתי הם לברכה באמת - בזמן שיש להם רכוז בנטית החיים שלנו במהלך קדשם. }\משנה{ברכה לצורך }\הגדרה{- צורך גבוה וצורך הדיוט }\מקור{[ע״ר א לא]}\צהגדרה{. }

\הגדרה{ברכה שיצאה מפינו והשיגה את הרכוז בחיי המפעל ובהכרה המפורשה }\מקור{[עפ״י שם]}\צהגדרה{. }

\ערך{ברכת ד׳ }\הגדרה{- החיים המלאים צדק\mycircle{°} ע״פ תכונתם ואופיים }\מקור{[ע״א ב ט ער]}\צהגדרה{.}

\ערך{ברכת ד׳ }\הגדרה{- }\משנה{״לברך את שמך״ }\הגדרה{- ע׳ במדור פסוקים ובטויי חז״ל. }

\ערך{ברכת הדיוט הצריכה לגבוה}\צהגדרה{ - }\הגדרה{ע׳ במדור פסוקים ובטויי חז״ל. }

\ערך{בת }\הגדרה{- תיאור המין הנקבי שבכל נושא, ביחש להערך של המוליד והמחדש אותו, או המוציאו עכ״פ אל הפועל הגמור, כלומר ביחש ההורים האב\mycircle{°} או האם\mycircle{°}, או ביחש להנושא המיני הזכרי, שגם הוא מיוחש אל ההורים כלומר הבן\mycircle{°} }\מקור{[ר״מ קמ]}\צהגדרה{. }

\ערך{בת }\הגדרה{- מלת תואר, המדה }\מקור{[ר״מ קמ]}\צהגדרה{. }\mylettertitle{ג}

\ערך{גא }\הגדרה{- תאור הרוממות הגשמית }\מקור{[עפ״י ר״מ קמא]}\צהגדרה{. }

\ערך{גאוה עליונה }\הגדרה{- }\משנה{הגאוה העליונה }\הגדרה{- }\מעוין{◊}\הגדרה{ כשאור הקודש\hebrewmakaf העליון\mycircle{°} של המקור הראשי אשר להחכמה\mycircle{°} הקדומה מופיע בהארת אורה\mycircle{°} בתוך הגמול\mycircle{°} העולמי, ומשפט\mycircle{°} הצדק\mycircle{°} מבהיק את אורו בבהירותו, }\משנה{הגאוה העליונה}\הגדרה{ מתגלה בעולם }\מקור{[ר״מ קמא]}\צהגדרה{. }

\הגדרה{ע׳ במדור תיאורים אלהיים, גאות ד׳. }

\משנה{גאונות }\צהגדרה{- מקוריות יסודיות }\צמקור{[שי׳ א 67].}

\ערך{גאונות רוחנית }\הגדרה{- עצם כחות הנפש, קדושת הרגש וגדולת הכשרונות, בהתקבצם יחד, באיש אחד מיוחד ומצויין}\צהגדרה{ }\מקור{[עפ״י א״י כד]}\צהגדרה{.}

\ערך{גאונים }\הגדרה{-}\משנה{ תקופת הגאונים }\הגדרה{- התקופה הגדולה שאחר חתימת התלמוד\mycircle{°}, שהיתה תקופה הרת עולם בחיי הרוח הפנימיים של אומתנו. הנוגעת לעיקרה ויסודה של חכמת\hebrewmakaf ישראל\mycircle{°}. תורת ההלכה\mycircle{°} והאגדה\mycircle{°} של רבותינו הקדמונים בבבל, גאוני סורא ופומבדיתא, אשר מידם נמסר לנו המבצר הגדול לחומת אש דת התלמוד\mycircle{°} הבבלי כולו }\מקור{[מ״ר 315]}\צהגדרה{.}

\ערך{גאות עולמים }\הגדרה{- גאות קדש\mycircle{°} המתנשא מימות\hebrewmakaf עולם\mycircle{°}, המתעלה מכל עלוי\mycircle{°} ופאר\mycircle{°}, למעלה מכל תכן של שאיפה היותר נאצלה השיכת לכל דבר נברא, גם בהתאחד הכל למטרתו היותר עליונה, שהיא תשוקת קדשם של עם ד׳ }\מקור{[עפ״י ע״ר א קנט]}\צהגדרה{. }

\הגדרה{ע׳ במדור תיאורים אלהיים, גאות ד׳. ושם, מתנשא מימות עולם. ע׳ במדור שמות כינויים ותארים אלהיים, גאה.}

\ערך{גבהי גבוהים }\הגדרה{- מקור שרשו, חיי נשמתו (של האדם), אור חיי נשמת כל העולמים, אור אל\hebrewmakaf עליון\mycircle{°} טובו\mycircle{°} והדרו\mycircle{°} }\מקור{[עפ״י א״ת י ב]}\צהגדרה{.}

\ערך{גבוה }\הגדרה{- }\משנה{(לעומת נמוך\mycircle{°}}\הגדרה{) - כללי\mycircle{°} ומופשט\mycircle{°} }\מקור{[עפ״י קובץ ג קז]}\צהגדרה{. }

\ערך{גבוה }\הגדרה{- }\משנה{(לעומת רם\mycircle{°}) }\הגדרה{- }\מעוין{◊}\הגדרה{ מצטיין ג״כ ביחש להמשך הדבר, ההולך וגבה מתחתית מצבו עד הרום\hebrewmakaf העליון\mycircle{°} }\מקור{[ע״ר א קיב]}\צהגדרה{. }

\ערך{גבולים }\הגדרה{- זמנים ומקומות, מעשים ומחשבות, שאיפות ורצונות מוגבלים }\מקור{[ע״ר א קצג]}\צהגדרה{. }

\הגדרה{ע״ע מגביל. }

\ערך{גבולים }\הגדרה{- }\משנה{גבולים וגדרים}\הגדרה{ - המניעות }\מקור{[ע״ט טו]}\צהגדרה{. }

\הגדרה{ע״ע הגבלה. }

\ערך{גבורה }\הגדרה{- }\משנה{הגבורה המפעלית }\הגדרה{- היכולת להסיר בחוזק יד את כל המפריעים לציורים\mycircle{°} ההבנתיים והידיעתיים מהתגשמותם במעשים }\מקור{[עפ״י ע״א ד י ח]}\צהגדרה{. }

\הגדרה{היכולת להגשים, להוציא אל הפועל }\מקור{[עפ״י א״ק ג קו]}\צהגדרה{. }

\הגדרה{כח רצון כביר וכח מפעל עז, שיוכל להוציא מן הכח אל הפועל כל אשר ברוח עמו}\צהגדרה{ }\מקור{[עפ״י פנק׳ א שצט]}\צהגדרה{.}

\ערך{גבורה }\הגדרה{- היכולת לעמוד נגד כל מהרס המציאות ומחריבה, העמדה נגד כל העומד להחריב ולהפריע את הטוב\mycircle{°} ואת ההשפעה הראויה להביא תועלת לברואים }\מקור{[ע״ר א קג]}\צהגדרה{. }

\ערך{גבורה }\הגדרה{- }\משנה{כשרון הגבורה }\הגדרה{- כיבוש החיים, ישוב העולם, וההתכוננות המעשית לכל פרטי פרטיה }\מקור{[קובץ ו קנז]}\צהגדרה{. }

\ערך{גבורה }\הגדרה{- אומץ הנפש של שכלול הרצון, בתגבורת עזוזו, היראה\mycircle{°} }\מקור{[עפ״י מ״ר 100, וקבצ׳ ב קסו]}\צהגדרה{. }

\הגדרה{ע׳ במדור מונחי קבלה ונסתר, גבורות הגבורה.}

\ערך{גבורה }\הגדרה{- }\משנה{נאזר בגבורה }\הגדרה{- מלא עז\mycircle{°} חיים אמיצים ומכובדים }\מקור{[פנק׳ ד תמח]}\צהגדרה{.}

\ערך{גבורה }\הגדרה{- }\משנה{מדת הגבורה (האלהית) }\הגדרה{- ע׳ במדור מונחי קבלה ונסתר.}

\ערך{גבורה }\הגדרה{- }\משנה{(לעומת גדולה)}\הגדרה{ - ע׳ בנספחות, מדור מחקרים, גדולה וגבורה הבדל שביניהם. }

\ערך{גבורת האמת }\הגדרה{- ע״ע אמת, גבורת האמת. }

\ערך{גבישי }\הגדרה{- משוכלל במלוא מדתו ותאר חמרו\mycircle{°} }\מקור{[עפ״י רצי״ה א״ש יב הערה 16]}\צהגדרה{. }

\ערך{גבר }\הגדרה{- איש\mycircle{°} באנשים עם מילוי כל כחותיו בהקפה שלמה, חי ופועל }\מקור{[ע״א ג ב קעב]}\צהגדרה{. }

\הגדרה{ע״ע ״אנוש״. ע״ע ״אדם״.}

\ערך{גברת }\הגדרה{- אשה\mycircle{°} ראויה להנהגה }\מקור{[ע״א ב ז מז]}\צהגדרה{.}

\ערך{גג }\הגדרה{- הבניה העילית שבמושב האדם }\מקור{[ע״א ד ח טז]}\צהגדרה{.}

\הגדרה{למעלה מהתכונה הרגילה של המון החיים, יושבי שפל, בכל תביעות חייהם ורגשות נפשם }\מקור{[ע״א ד ח טז]}\צהגדרה{.}

\ערך{גד }\הגדרה{- }\משנה{יסוד הגד }\הגדרה{- הגדה מראש את גורל האדם, הנמשך מההמשכה\mycircle{°} הנמשכת בקביעות איתנית, כפי הטבע הקבוע, מבלי הבא בחשבון את אשר תחולל יד האדם בבחירתו לשנות בו }\מקור{[עפ״י ע״א ד ו צט]}\צהגדרה{. }

\משנה{גד }\הגדרה{- המשכה תדירית ההולכת ומגרת את השפעתה בהמשכה קבועה, שמתוך כך נקל להגיד גם כן את הגורל האישי למשתקעים בתכונת ההוויה על פי היסודות המוצקים והאיתנים שבה בחומר וברוח <שאין לה שום מקום במציאות לפי האמת> }\מקור{[עפ״י שם]}\צהגדרה{. }

\הגדרה{הסכמת חלקי העולם הכללי, הגשמי והרוחני\mycircle{°},  במבטם והתאחדם זה עם זה, להיות משפיעים את המסלול של צביון החיים המיוחד אשר לכל נושא, שתיאור ההתגמלות של הנושאים, בואם בגבול המפעל, והערכתם לגבי העולם החיצוני לערכם, והכשרון לשאוב מכל מקור המזדמן להם, לינק מכל מבוע של חיים ושפעת כח הבא בהתנגשות עמם, זה כולו מתמם את ערך המזל\mycircle{°}, העושה חטיבה קבועה על המהלכים של הפרטים מתוך המהות הכללית, שרק בהנשא הרוח למעלה מהעיבוי הגבולי, ובהתרוממו ממעל להדלות העולמית, הרי הוא מתנשא למעלה מן המזל, אין מזל לישראל. אבל בהיות ההגמלה מחוברת לשאיבה, ועם דלות ההקצבה, ופתיחת התפיסה המוחשית, או ההצטיירות ההבנית, כשמתאגדים יחד, נעשה המורד מוכן לערך מזלי, ובא גד }\מקור{[עפ״י ר״מ קמד]}\צהגדרה{. }

\הגדרה{ע״ע מזל, יסוד המזל. ע״ע כוכבים. }

\ערך{״גדול״ }\הגדרה{- מורה על רוממות מעלה }\מקור{[מ״ש קכח (מא״ה ג קמג)]}\צהגדרה{. }

\ערך{״גדול״ }\הגדרה{- }\משנה{אדם גדול }\הגדרה{- ע׳ במדור מדרגות והערכות אישיותיות, ״אדם גדול״. }

\ערך{״גדול״ }\הגדרה{- }\משנה{(כינוי לד׳) }\הגדרה{- ע׳ במדור שמות כינויים ותארים אלהיים.  }

\הגדרה{ }

\ערך{גדולה }\הגדרה{- ההצטירות\mycircle{°} הכללית של המעשה אשר עשה האלהים\mycircle{°} }\מקור{[ע״ר א רל]}\צהגדרה{. }

\ערך{גדולה }\הגדרה{- }\משנה{(לעומת גבורה)}\הגדרה{ - ע׳ בנספחות, מדור מחקרים, גדולה וגבורה הבדל שביניהם. עע״ש רבים, תאר הרבים לעומת תאר הגדולה. }

\ערך{גדולה }\הגדרה{- }\משנה{(גדולתו של אדון\hebrewmakaf עולם) }\הגדרה{- ע׳ במדור תיאורים אלהיים, גודל עליון. }

\ערך{גדולה לד׳ }\הגדרה{- ע׳ במדור פסוקים ובטויי חז״ל.}

\משנה{גדוף }\צהגדרה{- }\צמשנה{(כלפי שמיא) }\צהגדרה{- כמה מדות יש בגדוף, ומעקרו ומכללו הוא מיעוט יחס הכבוד\mycircle{°} וחרדתו - והזילותא שבזה }\צמקור{[להלכות צבור (מהדורת תשע״ט) רצג].}

\מעוין{◊}\הגדרה{ מכח הכרות חשוכות וציורי שקר בהענין\hebrewmakaf האלהי, שכל מיעוט יראה וכבוד הוא בא מיסודו, הולך הרע ומתגבר עד כדי התוכן החשוך של ה}\משנה{גידוף}\הגדרה{. אמנם הגידוף תוכן שלילי יש לו, לא קישור לאיזה ענין, כ״א ניתוק מהאור הטוב המחייה כל העולמים, אבל ההחשכה שבאה ע״י הריחוק, מחוללת תנועה רבה של רשעה בהכחות השפלים }\מקור{[קובץ ה ל]}\צהגדרה{. }

\הגדרה{ע״ע גדפנות.}

\ערך{גדלות }\הגדרה{- }\משנה{(לעומת קטנות\mycircle{°} באדם)}\myfootnote{ בא״ק א נג-ד ״תור הגדלות תובע מן האדם שלא תהיינה פעולותיו מצות אנשים מלומדה, אלא שכל פעולה וכל הרגל, כל עבודה וכל מצוה, כל רגש וכל רעיון, כל תורה וכל תפלה, תהיה מוארה באור הגנוז, באור הכללי, הגנוז בנשמה העליונה לפני הופעת פעולתם״.\label{48}}\הגדרה{ - כלליות\mycircle{°}. הכנסת האדם את עצמו בחיי הכלל, מתוך הכרת הטוב\mycircle{°} והאור\mycircle{°} באמתת עצמם, כשהאדם שוכח מעט את עצמו, את פרטיותו, והטוב הכללי לוקח את לבבו, בביכור המחשבה\hebrewmakaf האצילית\mycircle{°} העליונה\mycircle{°}, כאשר האמת של שכר ועונש איננה הגורם העיקרי בדחיפת החיים המוסריים\mycircle{°}, כי אם התשוקה\hebrewmakaf האידיאלית\mycircle{°}, לחיים שיש בהם תוכן מדעי ומוסרי במלא מובנו }\מקור{[עפ״י א״ק ב תקט, שם ג שכא\hebrewmakaf שכב]}\צהגדרה{. }

\ערך{גדלות }\הגדרה{- כלליות }\מקור{[עפ״י מ״ש ס]}\צהגדרה{.}

\הגדרה{ע״ע גודל, הגודל לעומת הקוטן. }

\ערך{גדלות }\הגדרה{- }\משנה{במעמד הבהירות שלה }\צהגדרה{- }\מעוין{◊}\צהגדרה{ אז אנו וכל עצמיותנו נעשים מובלעים בזוהר\mycircle{°} הכללי\mycircle{°} האלהי\mycircle{°} שממעל לכל הגבלת\mycircle{°} עולמים\mycircle{°} }\צמקור{[עפ״י ע״ר א יט].}

\ערך{גדפנות }\הגדרה{- המרדה גרועה על הטוב שלא תוכל לתן שום דבר ולא תאיר את השכל בשום דבר בינה, כ״א תוסיף עקשות על שרירות לב של הרשעות למלא את הנפש תמהון ושכרון }\מקור{[ע״א ד ז יד]}\צהגדרה{. }

\הגדרה{ר׳ גדוף. ע׳ במדור מדרגות והערכות אישיותיות, אמגושי, גדופי.}

\ערך{גדרים }\הגדרה{- ע״ע גבולים. }

\ערך{גודל }\הגדרה{- }\משנה{הגודל לעומת הקוטן\mycircle{°}}\הגדרה{ - הכלליות }\מקור{[א״ק א עט]}\צהגדרה{.}

\הגדרה{ע״ע גדלות. ע״ע גדולה.}

\ערך{גוון }\הגדרה{- }\משנה{צבע }\הגדרה{- הסברת איזה תכן של הבלטה ללבישת הצורה אשר לעשר הגדול של כלל\mycircle{°} ע״י פרט. כשהפרט מתגלה רק כדי לגוון על ידו איזה ציור\mycircle{°} תכונתי מוגבל\mycircle{°} להופעה גלויה של העשר הפנימי שצפון בכלל, המתגוון ע״י גילוייהם של מדות הגבול של הפרטים}\myfootnote{ \textbf{גוון} - ע״ע א״ק א מד, ע״ר ב רנה, קנז ד״ה חלק, א״ק ב תנו ושם ג פט, א׳ ט.\label{49}}\הגדרה{ }\מקור{[עפ״י ע״ר א קפא]}\צהגדרה{. }

\הגדרה{ציור פרטי }\מקור{[עפ״י ע״ר א קעג]}\צהגדרה{.}

\הגדרה{ע׳ במדור מונחי קבלה ונסתר,  חשמל. ע׳ במדור פסוקים ובטויי חז״ל, תחש. ר׳ צבע. }

\ערך{גוי }\הגדרה{- הקיבוץ הכללי, <שהוא מושפע מהפרט, מהאדם היחיד> }\מקור{[פנק׳ ג שסח (קבצ׳ ג קכג)]}\צהגדרה{.}

\צהגדרה{קיבוץ של אנשים. הערך הצבורי מצדו הממשי של רבוי האנשים הפרטיים הנמצאים בו, בחבור גזעם ותכונתם כשהם לעצמם }\צמקור{[רצי״ה ע״ר ב תא\hebrewmakaf תב, עפ״י שם א רד, רה\hebrewmakaf ו].}

\ערך{גוי }\הגדרה{- (}\ערך{״גויים״ לעומת ״עמים}\הגדרה{״) - אלה שיש להם הסתגלות לשמירת רוח עצמי פנימי, המתפתחים בהכרה עצמית, שומרי גזעם ותכונתם, בעלי נפש מרגשת הרגשה פנימית, בעלי ההכשרה של איזו רוחניות  שירית פנימית, שומרי הרוח המיוחד אשר להם ומאמצים באיזה אופן שהוא את סגולת מוצאם}\צהגדרה{ }\מקור{[עפ״י ע״ר א רה\hebrewmakaf ו]}\צהגדרה{.}

\משנה{גוי }\הגדרה{-}\צמשנה{ (לעומת עם\mycircle{°})}\צהגדרה{ - מהות התוכן הקיבוצי הפנימי. המצטרף בעובדת ההתקבצות }\צמקור{[עפ״י א״ל עו].}

\צהגדרה{צבור, כמות }\צמקור{[שי׳ פיקודי סדרה ב, תשל״ו 4].}

\הגדרה{ע״ע עם}\myfootnote{ ע״ע בע״ר ב תא\hebrewmakaf ב. ל״י ח״א ויחן ישראל נגד ההר. מלבי״ם, ישעיה א ד באור המלות.\label{50}}\הגדרה{, ע״ע אומה. ע׳ במדור פסוקים ובטויי חז״ל, משפחות עמים.}

\ערך{גולם }\הגדרה{- ערך\mycircle{°} מוגבל\mycircle{°} ותוכן נקצב }\מקור{[עפ״י ר״מ קסז]}\צהגדרה{. }

\ערך{גוף }\הגדרה{- הדבר הנדרש לחיים, לאורה\mycircle{°} ולהפרחה }\מקור{[א׳ מח]}\צהגדרה{.}

\משנה{גוף, תוכנו }\הגדרה{- ערך של קבלת חיים מאיזה רוחניות\mycircle{°} מציאותית }\מקור{[ע״ר א נב]}\צהגדרה{.}

\ערך{גוף }\הגדרה{- }\משנה{(בתאורי הקב״ה) }\הגדרה{- ע׳ במדור תיאורים אלהיים.}

\ערך{גוף}\הגדרה{ - }\משנה{הגוף הטבעי במצב תכונתו הגופנית }\הגדרה{- ע׳ במדור גוף האדם אבריו ותנועותיו.}

\ערך{גופני }\הגדרה{- }\משנה{(לעומת נפשי\mycircle{°}) }\הגדרה{- כמותי חיצוני\mycircle{°} (לעומת איכותי פנימי\mycircle{°}) }\מקור{[רצי״ה א״ש יד כא]}\צהגדרה{. }

\ערך{גורל }\הגדרה{- הזכיה הנעלמת, שאין ידועה סבתה המוסרית\mycircle{°} הנכונה. מזל\hebrewmakaf עליון\mycircle{°}, שאי\hebrewmakaf אפשר להעמידו באיזו בחינה הגיונית וצורה משפטית }\מקור{[ע״ר א קט]}\צהגדרה{. }

\הגדרה{שאין מתגלה מה טעם זה זכה, ומד׳\mycircle{°} כל משפטו בטעם גמור}\צהגדרה{ }\מקור{[פנק׳ ג צ]}\צהגדרה{.}

\ערך{גורל עליון }\הגדרה{- כל המהות התמציתית של חיי האדם }\מקור{[עפ״י א״א 133]}\צהגדרה{. }

\הגדרה{נקודת האמונה\mycircle{°} }\מקור{[שם 134]}\צהגדרה{. }

\ערך{גזירה}\myfootnote{ של״ה, תושב״כ, פרשת חקת, ד״ה במדרש רבות ״וכבר השיגוהו (לרמב״ם שאמר שגזרות הם בלא טעם) על זה. ואדרבה כל המצות בטעמיהן הם גזרות״. ושם שם ד״ה הענין ״לכך נקראים כל המצות גזירות, כי כמו שהנשמות חלק אלוה ממעל, לקוחה ממנו יתברך, כך המצות גזירות, ׳אוכלא דאפרת׳, חתוכות ממנו, מלשון ׳לגוזר ים סוף לגזרים׳, ׳והוא עבר בין הגזרים׳ וזהו כריתת הברית... וכן כל גזרה דרבנן, רצונו לומר, שהוא נגזר ונחתך מטעם איסור דאורייתא, כדי שלא יבא לידי איסור דאורייתא״. ושם, שם פרשת כי תצא, ד״ה ודע כי מצות שלוח הקן ״מלת גזירה היא מלפני, אינו כפי מה שמבינים העולם שענין גזירה הוא דבר שאין לו טעם, אדרבה ענין גזירה הוא דוקא שיש לו טעם והוא לקוח ונגזר ממה שלמעלה הימנו״. אמנם גם הרמב״ם עצמו כתב בסוף ה׳ תמורה ״אע״פ שכל חוקי התורה גזירות הם, כמו שביארנו בסוף מעילה, ראוי להתבונן בהן, וכל מה שאתה יכול ליתן לו טעם תן לו טעם. הרי אמרו חכמים הראשונים שהמלך שלמה הבין רוב הטעמים של כל חוקי התורה״.ע״ע במדור פסוקים ובטויי חז״ל, מדותיו של הקב״ה אינן רחמים אלא גזרות.\label{51}}\ערך{ }\הגדרה{- המשפט הגמור, החובה היותר גמורה }\מקור{[עפ״י קובץ ז קפג (ב״ר שכה)]}\צהגדרה{. }

\הגדרה{משפט ודין גמור, חק איתן. משפט צדק שיתגלה בכל שלמותו בבא עתו }\מקור{[עפ״י א״ה (מהדורת תשס״ב)  ב 93]}\צהגדרה{.}

\הגדרה{ע׳ במדור מצוות, הלכות, מנהגים וטעמיהן, בהגדרות המבוא, מצוות, כל מצוותיה של התורה, חקי התורה.}

\ערך{גזירה }\הגדרה{- }\מעוין{◊}\הגדרה{ עיקר הנחתה היא לפי ערך שלשלת הנצחיות שהסיבות מתיחסות למסובביהן לפי מהלך המציאות }\מקור{[ע״א א א קלח]}\צהגדרה{. }

\ערך{גזרה עליונה}\צהגדרה{ - }\הגדרה{חשבון של פרעות תוצאת חיסרון הדיוק, הויתורים והדברים שבני אדם דשים בעקביהם, היוצא אל הפועל בהריסות כלליות או פרטיות. השפעה של אי הדיוק המוטבע ברוב החיים הסוללת דרכה לגבות את חובה באופן ציורי או מוחשי.}

\ערך{הגזירות העליונות }\הגדרה{- יוצאות בזעפן כסדרי הוויה וחיים המושפעים לצורך הכלל ההולך במרוצת חייו על פי סדרי משקלות שיש בהם צדדי הכרעה שאינם מדוקדקים בכיוון גמור לפי המגמה האלוהית העליונה. }\ערך{הגזרות העליונות}\הגדרה{ באות לגבות את החובות של חיסרון הדיוק שבחיי המוסר שהם אינם ניכרים כל אחד לעצמו ולשעתו אבל מסתרגים בעידן ריתחא בהיקבצם }\מקור{[עפ״י ע״א ד ו סב]}\צהגדרה{.}

\ערך{גידוף }\הגדרה{- ע״ע גדפנות. }

\ערך{גיהנם }\הגדרה{- הכח הכללי, ים הכליון וההעדר, המתיך ומחדש חידוש אופי גמור }\מקור{[עפ״י ע״א ד יב נא]}\צהגדרה{. }

\הגדרה{הכח המעדיר והמהרס שמגמתו לשנות את הצורות הראשונות על ידי כחו המכלה, עד שיצאו בפנים חדשות }\מקור{[עפ״י שם מז]}\צהגדרה{. }

\הגדרה{כח המכלה שמגמתו השכלול הכללי שבא באחרית, על ידי ההירוס והכליון. <המגמה של מציאותו היא מגמת השכלול הבאה במקום שיש הכרח של הירוס ושברון, כדי להגיע אל הכונה האחרונה של השלמת ההויה, על ידי הסרתם של הסיגים המוחלטים, וחידוש יצירה חדשה, בצורה מחודשת, מתוך היסוד המקולקל שקדם> }\מקור{[עפ״י שם מט]}\צהגדרה{. }

\משנה{שאיפתו של הגיהנם }\הגדרה{- מגמת התיקון ע״י הרס וכליון הקדום }\מקור{[שם מז]}\צהגדרה{. }

\משנה{גיהנם }\הגדרה{- המירוק\mycircle{°} של הנשמות מזוהמתן\mycircle{°} (ב)צער העמוק הנחשולי }\מקור{[קובץ ח ק]}\צהגדרה{. }

\הגדרה{מירוק בדרך הפועלת לשנות את טבע הנפש עצמה }\מקור{[עפ״י ע״א א א קסט]}\צהגדרה{. }

\הגדרה{שינוי עצמיות הנפשות להכשירם לצד הטוב\mycircle{°} המתקן קלקולים הנדבקים בעצם טבעם }\מקור{[עפ״י ע״א ג ב רעד]}\צהגדרה{. }

\משנה{תפקידו של הגיהנם }\הגדרה{- לשנות את הצורה העצמית של הנשמה, נשמת החיים, הרוח והנפש, על ידי הכליון של החלקים העצמיים הגרועים שנתעצמו בקרבם, עד שאחר ההיתוך הצורי הזה, יצאו כחות החיים הפנימיים בצורה אחרת, לגמרי חדשה }\מקור{[ע״א ד יב נ]}\צהגדרה{. }

\משנה{גיהנם }\הגדרה{- הבוץ בו מרגשת הנשמה את צרתה הגדולה כאשר טבעה בו, בבא זמן האורה והנשמה מתנשאת למרום טבעה. הרשמים הטבעיים שישנם בנפש האדם להיותו יורד על ידם לשפל מדרגת הבהמות, כשהם פועלים את פעולתם ע״י כשלונו המוסרי\mycircle{°} של האדם שלא כהוגן, נמשך האדם על ידם במצב רוחני הפוך מטבע הנשמה הטהורה\mycircle{°} והאלהית שבקרבו; ומיעוט הכח, שנתדלדל מקרב הנשמה חילה הפנימי ע״י דרכיה הפרועים, הוא מועיל לא להחיותה כולה בשלמותה, כ״א להטעימה את מעמדה האומלל והאיום ולהבעיר את תשוקתה להחלץ מחשכת הדמיונות הגרועים שכסו את שמיה הבהירים }\מקור{[עפ״י ע״א ג ב רלד]}\צהגדרה{. }

\הגדרה{כור מצרף להרוחות\mycircle{°} והנשמות\mycircle{°} שבגלל חטאיהם\mycircle{°} יצאו מעולמם החומרי\mycircle{°} במצב של קלקול. הכח שעל ידו יבואו לידי יצירה חדשה ויותר מושלמת, יסורו הסיגים מהכסף הטהור של כוחות החיים ושל כל מה שיחובר להם. כח המכלה והמשבר\mycircle{°}, הים הגדול השוטף כליון על כל הצורות הקדומות, המעבר המשלים את צביונה של היצירה בכללותה }\מקור{[עפ״י ע״א ד יב מג]}\צהגדרה{. }

\הגדרה{מקום\mycircle{°} ששם מתגלה לעצם הנפש המרגשת הרגש האכזרי שפועלת עליה ההשקפה האמתית של אבדן הטוב היותר נחמד, האוצר היותר יקר שבכל אוצר החיים, שם מתגלה כח הצער\mycircle{°} בכל כחו, הצער המגיע מהכליון המוסרי\mycircle{°} }\מקור{[עפ״י ע״א ב ט קיב]}\צהגדרה{. }

\הגדרה{חוסר התורה\mycircle{°} }\מקור{[א״ת ז ו]}\צהגדרה{. }

\ערך{גיהנם }\הגדרה{- }\משנה{״שרה של גיהנם״}\myfootnote{ שבת קד. זוהר ח״ב יח.  \label{52}}\הגדרה{ - הכח המשבר את הצורה\mycircle{°} הרוחנית ביסודה, בכח ההרס האכזרי שלו }\מקור{[ע״א ד יב מה]}\צהגדרה{. }

\ערך{גיהנם }\הגדרה{- }\משנה{״אשו של גיהנם״ }\הגדרה{- כח הצורב של ההעדר הנורא }\מקור{[עפ״י שם מד, נ]}\צהגדרה{. }

\הגדרה{הצירוף והזיקוק, המהרס בזעפו }\מקור{[שם מח]}\צהגדרה{. }

\ערך{יסורי גיהנם}\myfootnote{ \textbf{יסורי גיהנם, הצער הפנימי על חסרון ההשלמה של הנשמה וכו׳. שטף הרצון החיצוני, שנגד הרצון הפנימי }- ע״ע אור השם, לר״ח קרשקש, מאמר ג ח״א כלל ג פרק ג ד״ה ואמנם השני והשלישי (מהדורת הר״ש פישר עמ׳ שלה).\textbf{הצער הפנימי [...] בקנין\hebrewmakaf תורה [...] מניעת אור\hebrewmakaf התורה [...] מעוט התורה }\textbf{-}\textbf{ }ע׳ נפה״ח שער ד פרק יז.\label{53}}\הגדרה{ - צער הנשמה שאינה מוציאה את מעלות רוחה מן הכח אל הפועל, המתענה בעינויים נוראים }\מקור{[עפ״י קבצ׳ ב קנז]}\צהגדרה{.}

\הגדרה{הצער הפנימי\mycircle{°} על חסרון ההשלמה של הנשמה\mycircle{°} במעשים, בידיעות, ובדעות, ביחוד בקנין\hebrewmakaf תורה\mycircle{°} }\מקור{[א״ת ז ה]}\צהגדרה{. }

\ערך{מצרי גיהנם }\הגדרה{- שטף הרצון החיצוני\mycircle{°}, שנגד הרצון הפנימי, הוא }\משנה{מצרי גיהנם}\הגדרה{ המתגברים לפי אותו הערך של מניעת אור\hebrewmakaf התורה\mycircle{°} }\מקור{[א״ת ז ו]}\צהגדרה{. }

\הגדרה{מצרים האוחזים בכל מי שריפה ידיו מן התורה; הצער הגדול על מעוט התורה, שבא מצד בטול תורה וצמצום הדעת, המאפיל על אורה הרוחני של תורה }\מקור{[עפ״י א״ת ז ח]}\צהגדרה{.}

\ערך{צער הגיהנם בעולם הזה}\myfootnote{ ע״ע ע״א ב ט לח.\label{54}}\הגדרה{ שחשים את כיעור הנפילה בעמקי הרע\mycircle{°}, ואי אפשר להושיע\mycircle{°} את עצמו }\מקור{[קבצ׳ ב קז (פנק׳ ד רנ)]}\צהגדרה{.}

\צהגדרה{המחשבה מוכרחת להתעלות כפי אותה המדה שגנוז בכחה, ואם אין מוציאים אותה מן הכח אל הפועל היא מתענה בעינויים נוראים, שרק המתמכרים אל החושים יכולים הם לשכחם. אבל מי שהוא איש רוחני\mycircle{°}, מרגיש את}\הגדרה{ צער נשמתו איך היא נתונה ממש בתוך אשה\hebrewmakaf של\hebrewmakaf גיהנם\mycircle{°}, כל זמן שאינה מוציאה את מעלות רוחה מן הכח אל הפועל}\צהגדרה{ }\מקור{[קבצ׳ ב קנז]}\צהגדרה{.}

\הגדרה{ע״ע שְאוֹל. ע׳ במדור מלאכים ושדים, ״דּוּמָה״. ע״ע ערבה. ע׳ במדור פסוקים ובטויי חז״ל, קילוסו של הקב״ה עולה מגיהנם כשם שעולה מגן עדן. ושם, עדן, דישון עדן. }

\ערך{גיור }\הגדרה{- }\משנה{ליהדות התורה של האומה }\הגדרה{- ע״ע גרות, התגירות. }

\ערך{גיור }\הגדרה{- }\משנה{הטרם\hebrewmakaf היסטורי במובנה של יהדות\hebrewmakaf התורה }\הגדרה{- ע״ע גרות.}

\ערך{גלוי }\הגדרה{- }\משנה{״פתוח״\mycircle{°}}\הגדרה{.}\משנה{ הדברים הגלויים }\הגדרה{- כל המציאות המוחשית והמושגת בכלל, וכל המעשים הנעשים במפעלות ידי אדם. כל התכניות הגלויות שבהויה וכל המעשים המושגים }\מקור{[ע״א ד יב ב (מא״ה ב קל)]}\צהגדרה{.}

\הגדרה{ע׳ במדור מונחי קבלה ונסתר, קוב״ה דרגא על דרגא סתים וגליא וכו׳. ע׳ במדור פסוקים ובטויי חז״ל, מאמר פתוח. ע״ע ״סתום״. }

\ערך{גלוי }\הגדרה{- מושג ומורגש }\מקור{[מא״ה ג (מהדורת תשס״ד) קכד]}\צהגדרה{.}

\הגדרה{הנתפס באיזה רעיון שכלי אנושי }\מקור{[ע״ר א קיא]}\צהגדרה{.}

\ערך{גלות }\הגדרה{- חרבן פנימי\mycircle{°} ופזור חצוני }\מקור{[א׳ קח]}\צהגדרה{.}

\משנה{חותם הגלות }\הגדרה{- דיכוי רוח עם ד׳ ודכדוך סדרי חייהם עד שלא יוכל לקום בהם רוח להרגיש יפה את צרכי החיים הטבעיים כולם באופן הראוי להיות מורגש לעם חי מלא אונים }\מקור{[עפ״י ע״א ג ב סה]}\צהגדרה{.}

\צהגדרה{הריסת חיי האומה\mycircle{°} בניתוקה ממקומה, ״גלינו מארצנו ונתרחקנו מעל אדמתנו״ }\צמקור{[ל״י א קצב]. }

\ערך{גלות }\הגדרה{-}\משנה{ (מטרת הגלות) }\הגדרה{- <לא בשביל איזה מירוק\mycircle{°} של חטא\mycircle{°} מיוחד, כ״א> כדי להסתגל להיות גם בגלות עומד חי וקיים בתור אומה\mycircle{°} מחוטבת\mycircle{°}}\צהגדרה{ }\מקור{[ע״א ד ט קמז]}\צהגדרה{.}

\הגדרה{כור ברזל והכשרה לזיכוכה של האומה, כדי שתהיה מוכנת באחרית הימים לשוב לארצה ולחוסן יקרה וכבודה }\מקור{[ע״א ד יד ז]}\צהגדרה{.}

\הגדרה{כור הברזל, (ש)זקק וצרף את האומה, הוציא מן הכח אל הפועל את דעת עצמותה, את נטיותיה הגנוזות, הטבעיות לה, לצדק\mycircle{°} לאורה\mycircle{°}, לחיי יושר\mycircle{°} וטהרה\mycircle{°}, באופן שבכל פינות שתהיה פונה בכל צורה שהחיים יראו לה על ידה, תכיר את האור ואת הטוב\mycircle{°} }\מקור{[ע״ה קכח]}\צהגדרה{.}

\צהגדרה{<ההתקשרות האלהית, העליונה והטהורה, אינה מנגדת כלל את העולם ואת החיים, לכל עמקי תוכניהם, אלא גם מכשרת אותם ומרחיבתם. }

\צהגדרה{אלמלא חטאו ישראל לא היו צריכים כלל לסגל להם איזה סגולות מן החוץ כדי להשלים את עצמם בכל ערכי החיים. אבל החטא גרם, והמחשבה העליונה הועמה, ומה שנשאר הוא אור של תולדה, שאין לו אותו הבוהק העליון וההכללה הגמורה של העליוניות המוחלטת, של המחשבה האלהית, וממילא אין לו אותה סגולת ההרחבה, עד שבא הדבר, שההתיחדות עם התכונה של ההתקשרות האלהית, החסירה את הכשרון לשארי כשרונות>, }\משנה{והוצרך הפיזור של הגלות}\הגדרה{ כדי להשלים את החסרונות הללו, לספג את כל היתרונות של כל הגויים אל תוכם כדי להשלים את צביונם, }\צהגדרה{<והשלמת הצביון והזיכוך הארוך של הנפש הלאומית בכור הברזל של הגלות גרם לאפשר את החזרת האורה העליונה> }\מקור{[א״ק ג שסז]}\צהגדרה{.}

\משנה{תכלית הגלות }\הגדרה{- כדי שיוכלו גם האומות להכיר כבוד\hebrewmakaf ד׳\mycircle{°} }\מקור{[ע״א א ה פג]}\צהגדרה{. }

\ערך{גמול }\הגדרה{- הגורל\mycircle{°} המוסרי\mycircle{°} }\מקור{[עפ״י ר״מ קכט]}\צהגדרה{.}

\הגדרה{קשור הגורל עם מהלך החיים הרצוניים }\מקור{[עפ״י שם צו]}\צהגדרה{.}

\הגדרה{משפט הצדק\mycircle{°} }\מקור{[שם קמא]}\צהגדרה{.}

\משנה{הגמול הגלוי }\הגדרה{- התגלות\mycircle{°} המשפט\mycircle{°} המעשי שבעולם ומלואו }\מקור{[שם קמג]}\צהגדרה{.}

\ערך{גמול }\הגדרה{- }\משנה{מדת הגמול העליונה האידיאלית }\הגדרה{- וחנותי את אשר אחון ורחמתי את אשר ארחם. הגמול האידיאלי העליון המופשט, הגנוז ביסוד החסד\hebrewmakaf העליון\mycircle{°} אשר בו עולם יבנה }\מקור{[ר״מ קמב\hebrewmakaf ג]}\צהגדרה{.}

\ערך{גמילות חסדים }\הגדרה{- }\משנה{יסוד גמילות חסדים }\הגדרה{- נטית החסד והאהבה של הבריות, היוצאה מכלל הרחמים על אומללים ונדכאים, אלא חפץ ההטבה ושפור החיים, להרבות טוב לכל }\מקור{[ע״ר א סג]}\צהגדרה{.}

\הגדרה{ע׳ במדור מדרגות והערכות אישיותיות, גומל החסדים.  }

\ערך{גמילות חסדים טובים }\הגדרה{- ע׳ במדור פסוקים ובטויי חז״ל, חסדים טובים.}

\ערך{גן החכמה}\הגדרה{ - ההכרה העליונה והבהירה, ההרגשה היפה והעדינה }\מקור{[קובץ ג קעח]}\צהגדרה{.}

\ערך{״גן עדן״}\myfootnote{ \textbf{גן עדן} - אדיר במרום ח״א עמ׳ מג, ״גן עדן הוא פנימיות העולם״. אור החיים עה״ת, בראשית ב טו ״גן עדן אשר היום אדמת הנפשות לבד״.\label{55}}\הגדרה{ - ההצמחה של כללות העדונים, מקום\mycircle{°} זיו\mycircle{°} חיי הנשמות\mycircle{°} והתענגותם\mycircle{°} האצילית\mycircle{°}, בטיסתם העליונה העולה למעלה למעלות, בכל חזות עולמי פאר\mycircle{°} נגוהות, בדליגות מעלות ע״ג מעלות, ובשובע שמחות\mycircle{°} של תענוגי רוית קדשים\mycircle{°}, ההולך וצומח בצמחי מעדניו, הוא התגלות העדן\mycircle{°} של גן\hebrewmakaf ד׳ }\מקור{[ע״ר א יז]}\צהגדרה{.}

\הגדרה{ע׳ במדור פסוקים ובטויי חז״ל, עדן, דישון עדן. ושם, אטייל עמכם בגן\hebrewmakaf עדן}\תהגדרה{. }\הגדרה{ע״ע, עדן העתיד. }

\ערך{גן עדן }\הגדרה{- }\מעוין{◊}\הגדרה{ מורה שישוב המין האנושי לגובה מעלתו בפרטיו, עד שלא יהיה צריך שום לימוד והדרכה, ולא עזרת קיבוץ, כ״א לחיות ב}\משנה{גן עדן, }\הגדרה{לעבדה ולשמרה, בתור טיול\mycircle{°} ועונג\mycircle{°} מורחב הממלא את החומר והרוח עדנים }\מקור{[קבצ׳ ב צז]}\צהגדרה{.}

\הגדרה{תיקון\mycircle{°} המלכות\mycircle{°}, שלעתיד ושקודם החטא\mycircle{°} (ה)מכוון לקבלת אור לנשמות ישראל בעצם. תיקון הנשמות ועילויים, נקרא בשם }\משנה{גן }\הגדרה{- המצמיח הנטיעות וארזים אשר נטע ד׳ }\מקור{[עפ״י פנק׳ ד תמ]}\צהגדרה{.}

\משנה{יסוד גן עדן}\הגדרה{ - השבת כללות האדם לטהרתו\mycircle{°} האלהית\mycircle{°}, שהיא למעלה מכל חילוק לאומים, שהוא עומד למעלה מעולם הזה }\צהגדרה{[קבצ׳ ב קנז]}\הגדרה{. }

\הגדרה{ע׳ במדור מונחי קבלה ונסתר, נקודת ציון. ע״ע רוח גן\hebrewmakaf עדן אלהים, המנשב בנשמה. ע׳ במדור פסוקים ובטויי חז״ל, אטייל עמכם בגן עדן. ע׳ במדור אדם הראשון, ״לַעֲבֹד את האדמה״, (מששולח האדם מגן\hebrewmakaf עדן).}

\ערך{גן עדן }\הגדרה{- }\משנה{בבחינת ״גן הדס״ }\הגדרה{- צורתו הרוחנית, הריחנית, של גן עדן, המוסיף ריח טוב ועדין שהנשמה\mycircle{°} נהנית ממנו }\מקור{[עפ״י ע״א ד יב מח, מט]}\צהגדרה{.}

\הגדרה{ע״ע ריח בושם של גן עדן. }

\ערך{גן עדן }\הגדרה{- }\משנה{סעודת אכילת הפירות של גן עדן }\הגדרה{- ההנאה מזיו\hebrewmakaf השכינה\mycircle{°}, היסוד העקרי של חיי\hebrewmakaf עולם\mycircle{°} }\מקור{[עפ״י שם]}\צהגדרה{.}

\ערך{גן עדן }\הגדרה{- }\משנה{(לאדם הראשון) }\הגדרה{- מקום\mycircle{°} הזרחת השכל על אדם\hebrewmakaf הראשון\mycircle{°} בפועל }\מקור{[עפ״י מא״ה ב רסט]}\צהגדרה{.}

\הגדרה{ע׳ במדור אדם הראשון, חטא אדם הראשון. ושם, אדם הראשון בגן עדן. ע״ע עונג, עונג הגן. }

\ערך{גן עדן }\הגדרה{- ההתיחדות המסותרת במעמקי הנשמה\mycircle{°} }\מקור{[עפ״י קבצ׳ ג קנה]}\צהגדרה{. }

\ערך{גנזי שחקים }\הגדרה{- הספירות\mycircle{°} העליונות, ששם חביון\hebrewmakaf העז\mycircle{°}, ושם כל התכונה הכללית של מערכי ההויה, מסבותיה ומגמותיה\mycircle{°}, צפויה היא ונערכת }\מקור{[ע״ר א ריד]}\צהגדרה{. }

\ערך{גס}\הגדרה{ - חומרי }\מקור{[פנק׳ ד תלא]}\צהגדרה{.}

\ערך{גסות }\הגדרה{- נטיה חמרית\mycircle{°} }\מקור{[עפ״י א״ש טז יב]}\צהגדרה{. }

\הגדרה{ע״ע דק.}

\ערך{געגוע }\הגדרה{- }\משנה{הגעגוע האלהי העליון }\הגדרה{- הצמאון\mycircle{°} האדיר להכלל באוצר האור בחיי החיים העליונים, מקור כל החיים ושרש כל ההויות }\מקור{[ע״ר א סז]}\צהגדרה{. }

\משנה{געגועים אלהיים }\הגדרה{- חפץ טמיר מלא חיים עזיזים מקוריים ממקור חיי עולמים }\מקור{[א׳ נט]}\צהגדרה{. }

\משנה{הגעגועים ההומים לזוהר\mycircle{°} הצחצחות\mycircle{°} הנשמתיות\mycircle{°}}\הגדרה{ - הצמאון\hebrewmakaf האלהי, החפיצה הפנימית\mycircle{°} להתעלות\mycircle{°}, להשאב\mycircle{°} בחיי חיי מקור חי כל חיי העולמים\mycircle{°} }\מקור{[א״ק ג ס]}\צהגדרה{. }

\ערך{״גר אמת״ }\הגדרה{- ע׳ במדור מדרגות והערכות אישיותיות. }

\ערך{״גר תושב״ }\הגדרה{- ע׳ במדור מדרגות והערכות אישיותיות. }

\ערך{גרות}\myfootnote{ השוה ל״גר תושב״, במדור מדרגות והערכות אישיותיות. לבירור החלוק בין שני מושגי הגרות, ע׳ ל״י (מהדורת בית אל) ח״ב, מאמר ששים, לבירור ענין הגיור מבחינת היהדות התורתית. ע״ע לבנת הספיר, ויחי, עה״פ אוסרי לגפן עירו. ובברית הלוי לר״ש אלקבץ, סוף פרק יד, כג:.\label{56}}\הגדרה{ - }\משנה{התגירות }\הגדרה{- התבטלות רוחנית\mycircle{°} גמורה לכללות\mycircle{°} האומה\mycircle{°} הישראלית\mycircle{°} }\מקור{[עפ״י ע״ר א שצח]}\צהגדרה{. }

\משנה{גיור }\צהגדרה{- }\צמשנה{ליהדות התורה של האומה }\צהגדרה{- קשור בסדרי חייה ובמערכת חוקיה ומשפטיה של התורה הזאת ושל האומה הזאת. התחברות והצטרפות אל אומה זו ואל ייחוד שכינתה עם ההתחייבות במשמרת התורה והמצווה }\צמקור{[ל״י ג קו].}

\מעוין{◊ }\משנה{גרות }\צמשנה{עיקרה}\צהגדרה{ החלטת קבלתו בדעתו להיות גר בישראל ונלוה על ד׳, המתחילה לצאת לפעל עם קבלת המצות, וידיעת אזהרותיהן ושכרן ועונשן, }\צהגדרהמודגשת{ונגמרת ונעשית}\צהגדרה{ על ידי המילה והטבילה, ואינו גר עד שימול ויטבול }\צמקור{[דעת כהן תמה].}

\הגדרה{ע׳ במדור מדרגות והערכות אישיותיות, גרי צדק.}

\ערך{גרות }\הגדרה{- הוראה של קבלת\hebrewmakaf מלכות\hebrewmakaf שמים\mycircle{°} באמצעות ישראל וע״י השפעתם }\מקור{[ע״א ב ט קיד]}\צהגדרה{. }

\משנה{גיור }\צהגדרה{- }\צמשנה{הטרם\hebrewmakaf היסטורי במובנה של יהדות\hebrewmakaf התורה }\צהגדרה{- ביטול האליליות\mycircle{°} ושחיתותיה. גיור במובנה של אגדה והשפעה רוחנית כללית }\צמקור{[ל״י ג קו].}

\צמשנה{״אברהם מגייר את האנשים ושרה מגיירת הנשים״}\myfootnote{ בר״ר פר׳ לט יד.\label{57}}\צהגדרה{ - מדריכים אותם רוחנית}\צמקור{ [שי׳ ת״ת 162].}

\הגדרה{ע׳ במדור מדרגות והערכות אישיותיות, גר תושב.}

\ערך{״גרים גרורים״}\הגדרה{ - ע׳ במדור מדרגות והערכות אישיותיות. }

\ערך{״גרי צדק״ }\הגדרה{- ע׳ במדור מדרגות והערכות אישיותיות. }

\ערך{גרים - }\משנה{״קשים גרים לישראל כספחת״}\הגדרה{ - ע׳ במדור פסוקים ובטויי חז״ל. }

\ערך{גשם }\הגדרה{- }\משנה{(לעומת טל\mycircle{°}) }\הגדרה{- ההשפעה\mycircle{°} שאנו גורמים בעבודתינו באתערותא\hebrewmakaf דלתתא\mycircle{°}, מכוון נגד קרבת\mycircle{°} השי״ת הבאה ע״י המחקר בגדולתו\mycircle{°} ית׳ }\מקור{[מא״ה א קנט, מ״ש נט]}\צהגדרה{.}

\ערך{גשמי }\הגדרה{- ע״ע התגשמות. }

\ערך{גשמים}\myfootnote{ אבות ה משנה ה.\label{58}}\ערך{, הבאים מיסוד המים }\הגדרה{- מורים על עצם ההויה הגופנית, שֶׂכֶל הגוף }\מקור{[ע״ר ב קעו]}\צהגדרה{. }

\הגדרה{ע׳ במדור משכן ומקדש, אש המערכה. ע״ע רוח, (הרוח שלא נצחה את עשן המערכה).}\mylettertitle{ד}

\ערך{דאגה}\הגדרה{ - עצב ופחד }\מקור{[פנק׳ ג רפט]}\צהגדרה{.}

\ערך{דאגה}\הגדרה{ - חיפוש עצה\mycircle{°} איך למלאת החסרון }\מקור{[פנק׳ ג רפט]}\צהגדרה{.}

\ערך{דבור }\הגדרה{- המחשבה\mycircle{°} הגופנית }\מקור{[אג׳ ב לט]}\צהגדרה{. }

\ערך{דבור }\הגדרה{- ע״ע לשון. ע״ע שפה. ע״ע פרי השפתים. ע״ע הבל.}

\ערך{דבור }\הגדרה{- }\משנה{תוכן של דבור המתלוה לאמירה\mycircle{°} }\הגדרה{- התגלות הדבור (ההופעה המילולית), כשבא להתגלם בצורה המשפעת על השומע, שמקבל הנהגה ע״י הצורה הדבורית, הבאה אל חוגו, והוא משותף עם דברות והנהגה }\מקור{[עפ״י ע״ר ב נד]}\צהגדרה{. }

\מעוין{◊}\הגדרה{ לשון דבור הוא לשון הנהגה ולשון קשה }\מקור{[ע״ר א קלו]}\צהגדרה{. }

\הגדרה{ע״ע אמירה. ע״ע אמר. ע״ע קול. ע׳ בנספחות, מדור מחקרים, ״אֹמֶר״ לעומת ״דבור״. }

\ערך{דבור }\הגדרה{- }\משנה{(לעומת ״אמרי פי״\mycircle{°}) }\הגדרה{- ההגיון הפנימי }\מקור{[א״ה ב 902]}\צהגדרה{. }

\משנה{הדבור הפנימי }\הגדרה{- הגיון הלב }\מקור{[עפ״י מ״א א א]}\צהגדרה{. }

\ערך{דבור החיצוני}\הגדרה{ - }\משנה{(לעומת דיבור\hebrewmakaf הפנימי\mycircle{°})}\myfootnote{ \textbf{דבור חיצוני ודבור פנימי} - מקבילים שני סוגי דבורים אלו לדיבור החול והקודש בע״ר א קצב: ״הדבור בכללו יש לו כח כפול: דבור קדש, ודבור חול. \textbf{דבור הקדש} מוצאו הוא ממקור הדעה העליונה, יוצרת כל היצורים, ״בדעתו תהומות נבקעו״, ו\textbf{דבור של חול }מקורו הוא למטה, מהציורים הבאים ומתרשמים בנפשו של אדם מכל מה שיש ביצירה״.\label{59}}\הגדרה{ - הדיבור אודות צרכיו הפרטיים והכלליים של האדם, בתור איש חברותי ובעל\hebrewmakaf חי נזון, ובעל צרכים רבים יותר מבעלי החיים האחרים. החלק החיצוני של הדיבור, שהם עלי הדבור ולא פריו}\צהגדרה{ [עפ״י ע״ר ב סו]}\הגדרה{.}

\הגדרה{ע׳ במדור גוף האדם אבריו ותנועותיו, שפתיים. ושם, שפה.}

\ערך{דבור הפנימי }\הגדרה{- }\משנה{(לעומת דיבור\hebrewmakaf החיצוני\mycircle{°})}\footref{1}\הגדרה{ - הדיבור השייך להערכים הרוחניים\mycircle{°}, שהם עומדים למעלה מצרכיו הפרטיים והכלליים של האדם. פריו של הדיבור}\צהגדרה{ }\מקור{[ע״ר ב סו]}\צהגדרה{.}

\הגדרה{ע׳ במדור גוף האדם אבריו ותנועותיו, לשון.}

\ערך{דבור }\הגדרה{- }\משנה{הדבור האלהי }\הגדרה{- יסוד אור\hebrewmakaf השכל\mycircle{°} במקורו }\מקור{[ע״ר א קיט]}\צהגדרה{.}

\ערך{דבור }\הגדרה{- }\משנה{דבור ההויה הכללית }\הגדרה{- ההשפעה הנהלית, הבאה ע״י השפעת כח המבטא על השומעים והמקשיבים לקול ההשפעה }\מקור{[ע״ר ב נד]}\צהגדרה{. }

\הגדרה{ע״ע קול, קול ההויה כולה. ע״ע אמר, אמר ההויה כולה. }

\ערך{דבור רע}\הגדרה{ - }\משנה{הדיבור הרע}\הגדרה{ - ארס שלוח והרחבת ענפי כח הזוהמא\mycircle{°} הבשרית והצד העכור שבחיי\hebrewmakaf הרוח\mycircle{°} }\מקור{[עפ״י קובץ ג רנו]}\צהגדרה{.}

\ערך{דבקות }\הגדרה{- יסוד הדבקות\hebrewmakaf האלהית\mycircle{°}. תשוקת הטוהר\mycircle{°} המתעלה\mycircle{°} ע״י טהרת מחשבה\mycircle{°}, עומק הגיון\mycircle{°}, חופש\mycircle{°} רוחני\mycircle{°}, והסתכלות בהירה. ההתהלכות\hebrewmakaf את\hebrewmakaf האלהים\mycircle{°} של הדורות הראשונים }\מקור{[א״ק ג קסט]}\צהגדרה{. }

\הגדרה{מעלת הערת\hebrewmakaf השכל\mycircle{°}, הארת השכל בעבודת השי״ת }\מקור{[עפ״י פנק׳ ג ערה]}\צהגדרה{.}

\מעוין{◊ }\הגדרה{תולדת הקשר של הקדושה\mycircle{°}, ברצון ובשכל, בחיים ובהכרה שכלית }\מקור{[ע״ר ב עז]}\צהגדרה{. }

\ערך{דבקות אלהית }\הגדרה{- אמונת\mycircle{°} אלהים\hebrewmakaf חיים\mycircle{°}, אהבתו\mycircle{°} ויראתו\mycircle{°} }\מקור{[ע״ט מח]}\צהגדרה{. }

\הגדרה{זריחת\mycircle{°} האור הבהיר של כבוד\mycircle{°} השי״ת, אמונתו ואהבתו, באופן נעלה מכל מופת וידיעה }\מקור{[עפ״י קבצ׳ א קלו]}\צהגדרה{.}

\הגדרה{הפניה אל ד׳, אמירת אלי אתה, בלב ושפה, זו היא האמרה המקנה את עירוב המהות}\צהגדרה{ }\מקור{[א״ק ד תנא]}\צהגדרה{.}

\הגדרה{התכונה המבוקשת של דעת\hebrewmakaf אלהים\mycircle{°} הטהורה\mycircle{°}, המלאה חוסן\mycircle{°} וחיל\mycircle{°} }\מקור{[קובץ ח רמט]}\צהגדרה{.}

\הגדרה{התגבורת של הטבעיות של הקודש\mycircle{°}, המחזקת את הנשמה באימוץ אדיר וקדוש מאד }\מקור{[פנק׳ ב קצה]}\צהגדרה{. }

\הגדרה{דבקות הנשמה בצרור\hebrewmakaf החיים\mycircle{°}, באור\hebrewmakaf ד׳\mycircle{°} }\מקור{[א״ק ג רמ]}\צהגדרה{. }

\הגדרה{קשור פנימי\mycircle{°} נשמתי במה שהוא הכל\mycircle{°} ומקור הכל ויותר כל מן הכל, והחפץ המלא טהרה\mycircle{°} הולך ומתגבר מבלי להשאיר שום חלק מבלי שאיבת תמצית לשדו העליון\mycircle{°} שבעליונים }\מקור{[אג׳ ג ד]}\צהגדרה{. }

\הגדרה{דבקות\mycircle{°} בחיי\hebrewmakaf החיים\mycircle{°}, באור ההויה העליונה שכל החמדות, כל הנצחיות, כל הענוגים, כל השלומים וכל הגבורות, הפארים, התהילות והזהרות ממנה הם באים }\מקור{[עפ״י קובץ א תרסג]}\צהגדרה{. }

\הגדרה{התשוקה האידיאלית\mycircle{°}, אשר היא יסוד הכל, חשק\mycircle{°} הטוב\hebrewmakaf העליון\mycircle{°}, העולה בטובו מכל שנקלט אצלנו במושג של תענוג\mycircle{°} ועידון\mycircle{°} }\מקור{[עפ״י א״ק ג קסט]}\צהגדרה{. }

\משנה{דבקות אלהית אמיתית }\הגדרה{- אור התענוג העליון, החיים האמתיים, מקור ההצלחה\mycircle{°} ומגמת\mycircle{°} החיים וההויה כולה }\מקור{[עפ״י א׳ צט]}\צהגדרה{. }

\משנה{דבקות אלהית פנימית}\הגדרה{ - יסוד כל הידיעות }\מקור{[א״ק א יב]}\צהגדרה{.}

\משנה{דבק בחיי\hebrewmakaf החיים\mycircle{°}}\הגדרה{ - דעת צור\hebrewmakaf העולמים\mycircle{°}, המתעלה על כל גבול ומחזה עין בשר }\מקור{[ע״א א מהדו״ב א ב]}\צהגדרה{. }

\משנה{תוכנה של הדבקות האלהית }\הגדרה{- להיות משוקה מטל חיי עד\mycircle{°}, משורש החיים האמיתיים שאין לו הגבלה }\מקור{[עפ״י ע״ר א סג]}\צהגדרה{. }

\הגדרה{שאיפת היש האישי, הגודל\mycircle{°} והשיגוב\mycircle{°} }\מקור{[קובץ ה כו]}\צהגדרה{. }

\משנה{חשקת הדבקות האלהית}\הגדרה{ - כלות הנפש ועריגה בלתי פוסקת, ההולכת ומתגברת בכל עת תוסיף ההכרה להתעמק בתהומות הנפשיים }\מקור{[קובץ ז קיח]}\צהגדרה{.}

\מעוין{◊ }\משנה{התוכן של הדבקות האלהית הטבעית }\הגדרה{- אהבת ישראל\mycircle{°} }\מקור{[א״ק ג שמ]}\צהגדרה{.}

\מעוין{◊}\משנה{ חיים של דבקות אלהית }\הגדרה{- דבקות שכלית}\צהגדרה{ }\מקור{[קבצ׳ א קסט]}\צהגדרה{.}

\ערך{דבקות אלהית }\הגדרה{- }\משנה{דבקות אלהית בהירה, מגמתה }\הגדרה{- שרוח\mycircle{°} האדם בעילויו\mycircle{°}, ע״י הכרה צלולה והתנשאות רצון מטוהר\mycircle{°} וגמור, יתנשא לבוא עד לידי השלטת רצונו על ההויה מפני עוזו וחשיבותו }\מקור{[עפ״י ע״ט סז]}\צהגדרה{. }

\הגדרה{ע״ע הצלחה אמיתית. ע״ע עונג. ע״ע בטול. ע״ע חשק. }

\משנה{דבקות בד׳ נותן התורה}\myfootnote{ ע׳ בהקדמת המהר״ל, תפארת ישראל, עמ׳ ב ג: שלא בירכו בתורה תחילה פירוש ״שלא היו דבקים בו יתברך באהבה במה שנתן תורה לישראל״. \label{60}}\צהגדרה{ - זכירת היצירה האלוהית של נשמת האומה אשר אל חי בקרבה, השכינה שורה בתוכה, וחיי עולם של תורה נטועים בתוכה }\צמקור{[שי׳ ה 46 מאגרת רבינו, מכ״ה א תשל״ז]. }

\ערך{דבר ד׳ }\הגדרה{- כל התיאור והבנין העולמי עם כל חוקותיו }\מקור{[קובץ ה קצב]}\צהגדרה{. }

\ערך{דבר ד׳ }\הגדרה{- סוד הידיעה\hebrewmakaf האלהית\mycircle{°} }\מקור{[קובץ ה קמג]}\צהגדרה{.}

\ערך{דבש }\הגדרה{- מורה על תכן של הנאה ומתיקות והחשת ענג מוחשי, הקשורות עמו }\מקור{[ע״ר א קמד]}\צהגדרה{. }

\הגדרה{ע׳ במדור נפשיות, גאוה, ב״שאור ודבש״. ע׳ במדור מצוות, הלכות, מנהגים וטעמיהן, דבש, סוד איסור הדבש בהקטרה ע״ג המזבח.}

\ערך{דגל }\הגדרה{- }\מעוין{◊}\הגדרה{ מסמן את ההכרה של הנקודה היסודית, שכל התוכן של הקבוץ תלוי בה. דגל הצבא מרכז את רוח המחנה }\מקור{[ע״ר א מט]}\צהגדרה{. }

\ערך{דין }\הגדרה{- יד שמאל\mycircle{°}, המכשיר לשפע הטוב\mycircle{°} (שהוא התכלית - יד ימין\mycircle{°}) בזמן המעשה, פחד העונש והיראה\mycircle{°} <שהיא תחילת הכניסה לדרכי\hebrewmakaf ה׳\mycircle{°} ית׳> }\מקור{[עפ״י מא״ה ב רפד (פנק׳ ג קפו)]}\צהגדרה{. }

\ערך{דין }\הגדרה{- }\משנה{״יכולני לפטור את כל העולם כולו מן הדין״}\myfootnote{ סוכה מה:.\label{61}}\הגדרה{ - מחובות מצרים וגבולים צרים }\מקור{[עפ״י א״ק ג ט, שלא]}\צהגדרה{. }

\ערך{דין }\הגדרה{- }\משנה{״משמים השמעת דין״}\myfootnote{ תהילים עו ט.\label{62}}\הגדרה{ - אושר המציאות במגמתה התכליתית, רום נשמת האדם ובהיקותה, אומצה וחוסנה הטהורים, במעמדם הנצחי החובק כל אושר, כל שאיפה וכל עדנה. הכבוד\hebrewmakaf העליון\mycircle{°} של הוד\mycircle{°} נשמת\mycircle{°} כל היקום }\מקור{[עפ״י ע״א ד ט ע]}\צהגדרה{. }

\ערך{דין }\הגדרה{- }\משנה{מדת הדין }\הגדרה{- כח המגביל את האור\mycircle{°} שלא יתרבה על הכלים\mycircle{°}, וישברו\mycircle{°}. הכח העוצר והמעכב, הפועל שלא יבקעו הדברים אל הפועל בעוד שאין הזמן גרמא, בעוד שאין הכשר לזה מצד האנושיות }\מקור{[עפ״י א״ב 65 (א״ה ב, מהדורת תשס״ב, 87)]}\צהגדרה{. }

\הגדרה{ע׳ במדור מונחי קבלה ונסתר, רחמים, מדת הרחמים. }

\ערך{דין }\הגדרה{- }\משנה{מדת הדין }\הגדרה{- ההנהגה העליונה המצמצמת, המודדת מדה כנגד מדה }\מקור{[עפ״י ע״ר א ריג]}\צהגדרה{. }

\הגדרה{ע׳ במדור מונחי קבלה ונסתר, חסד, מדת החסד. ושם, גבורה, מדת הגבורה (האלהית).}

\ערך{דין }\הגדרה{- }\משנה{מדת הדין האיתנה }\הגדרה{- הקו המחשבתי\mycircle{°} בהמצאת ההויה הנגלית שתהיה מפוארה כ״כ עד ההשתוות, בפעולה והופעה\mycircle{°} נאצלת\mycircle{°}, עם ההויה הגנוזה בסתר\hebrewmakaf עליון\mycircle{°} }\מקור{[עפ״י ע״א ד ט קלז]}\צהגדרה{. }

\משנה{מדת הדין העליונה }\צהגדרה{- אור האמת\mycircle{°} }\צמקור{[ע״ר ב תפח].}

\צהגדרה{תפיסת הבריאה מצד תוכן הרצון העליון, המתגלה לה בערכו העצמותי עד לחשבון צמצומה וגבוליותה }\צמקור{[ב״ר שעח].}

\משנה{מדת הדין של מעלה בעוצם חזקה }\הגדרה{- האידיאליות המאירה שברעיון היצירה, הקודמת לכל עולמים, הדורשת את האור בתכלית בהירותו }\מקור{[ר״מ צה]}\צהגדרה{. }

\משנה{הדין הקדוש }\הגדרה{- התוכן האידיאלי של כל היקום טרם הבראותו }\מקור{[עפ״י ר״מ לא]}\צהגדרה{. }

\הגדרה{ע׳ במדור מונחי קבלה ונסתר, גבורה גנוזה. ושם, גבורות, הגבורות. ע׳ במדור מדתם ועניינם הרוחני של אישי התנ״ך, יצחק, מדתו של יצחק. ר׳ במדור פסוקים ובטויי חז״ל, מדת הדין, שעלתה במחשבה, לפני בריאת העולם. }

\ערך{דין }\הגדרה{- }\משנה{מדת הדין של מעלה }\הגדרה{- הבינה\hebrewmakaf העליונה\mycircle{°}, שהדינים מתעוררים מצדה, כדי לבסם העולם ולשכללו }\מקור{[פנק׳ ג שדמ]}\צהגדרה{.}

\ערך{דין }\הגדרה{- }\משנה{מדת הדין המיוחסת לבית שמאי}\myfootnote{ זוהר ח״ג רמה.\label{63}}\משנה{ }\הגדרה{- העליה אל חקר עומק החכמה, היא החכמה הגדולה המופשטת ונעלה מכל רגש, בה כלולים כל השלמויות }\מקור{[עפ״י ע״א ב ח ד]}\צהגדרה{. }

\הגדרה{ע״ע חסד, מדת החסד המיוחסת לבית הלל. ע׳ קבלה ונסתר, הלכה כבית שמאי לעתיד לבוא. }

\ערך{דין }\הגדרה{- }\משנה{מדת הדין הקשה }\הגדרה{- }\מעוין{◊ }\הגדרה{מתיסדת ע״פ המבט של ענין היצורים כשהם מצד עצמם }\מקור{[ע״ר א צט]}\צהגדרה{. }

\ערך{דין }\הגדרה{- }\משנה{דינים }\הגדרה{- הגבורות\mycircle{°} והצמצומים\mycircle{°} }\מקור{[ח״פ 6]}\צהגדרה{.}

\צהגדרה{ }

\ערך{דין }\הגדרה{- }\משנה{דינים }\הגדרה{- הפעולות המרעישות, המחבלות והנראות כמהרסות ומחריבות את העולם }\מקור{[ע״ר א קלד]}\צהגדרה{. }

\משנה{דין }\הגדרה{- }\צמשנה{דינים קשים }\צהגדרה{- הדעות הרעות, שהן המביאות את המעשים הרעים }\צמקור{[ק״ו קנה].}

\הגדרה{ר׳ במדור מונחי קבלה ונסתר, מתוק דינים.}

\הגדרה{ }

\ערך{דין }\הגדרה{- }\משנה{כובד הדין }\הגדרה{- ע׳ במדור מונחי קבלה ונסתר, ״חבוט הקבר״. }

\ערך{דין }\הגדרה{- }\משנה{עומק הדין }\הגדרה{- הדקדוק היותר מכוון שבדין }\מקור{[קובץ ה קסד]}\צהגדרה{. }

\ערך{דין }\הגדרה{- }\משנה{שורת הדין}\הגדרה{ - כל מעשה וכל הנהגה בפרטיותה בצדק וביושר, כתורה  וכמצוה }\מקור{[עפ״י ע״ר א צח]}\צהגדרה{. }

\תערך{דליגה }\תמקור{- }\תהגדרה{פעולת הרצון\mycircle{°} החפשי\mycircle{°} ההחלטי מכל סטרא, ה״רצון\hebrewmakaf הפשוט״\mycircle{°}, שאינו בדרך השתלשלות\mycircle{°} מדרגא לדרגא }\תמקור{[נ״א ה 26]. }

\הגדרה{ע׳ במדור מונחי קבלה ונסתר, ״זריקה״. }

\ערך{דלת }\הגדרה{- מה שסותם את החלל הפתוח }\מקור{[ע״א א ב נה]}\צהגדרה{.}

\ערך{דם }\הגדרה{- }\משנה{(בביטוי ״דם יהודי״ וכדומה) }\הגדרה{- ״הדם הוא הנפש״ הטבעית בגלוייה המעשיים }\מקור{[רצי״ה אג׳ ב שמג]}\צהגדרה{. }

\מעוין{◊ }\הגדרה{בו טבוע כח החיים, <והוא מצד עצמו משולל השלמות ובהמי, אבל הוא בטבע נכנע אל השכל והקדושה> }\מקור{[מא״ה ב רסט-ע]}\צהגדרה{.}

\מעוין{◊ }\הגדרה{בו טבוע כח החיים והרצון, <שאמנם מצד עצמו הוא בהמי ומשולל השלמות, אבל כפוף הוא אל כח השכל והקדושה הפועל עליו ומטביע עליו את צורתו> }\מקור{[ע״ר א רנח]}\צהגדרה{.}

\ערך{דמות }\הגדרה{- תוכן המציין בצורה מיוחדת וידועה איזה תאר מוגבל }\מקור{[ע״ר א נב]}\צהגדרה{. }

\הגדרה{ע׳ בנספחות, מדור מחקרים, צלם, דמות, תבנית, צורה.}

\ערך{דמות האדם }\הגדרה{- התוכן של מרכז העולם, הרצון\mycircle{°} החפשי\mycircle{°}, שיתעלה\mycircle{°}, ויהיה, עם כל חופשו שהוא כשרון עצמי עדין, עזיז\mycircle{°} ומתמיד בפעולתו לטובה\mycircle{°} קבועה ההולכת ומתעלה }\מקור{[עפ״י א״ק ב תקס]}\צהגדרה{. }

\הגדרה{ע׳ במדור מונחי קבלה ונסתר, אדם עליון כללי. ע׳ במדור פסוקים ובטויי חז״ל, צלם אלהים. ע׳ במדור אדם הראשון, ״אדם הראשון״, נשמתו בכל מלואה.}

\ערך{דמיון }\הגדרה{- }\משנה{עולם הדמיון }\הגדרה{- האספקלריא\hebrewmakaf שאינה\hebrewmakaf מאירה\mycircle{°}, המלאה הדר\mycircle{°}, תבנית כל צורה מפוארה, כל חזון לב מרומם ומתעלה }\מקור{[א״ק א רמב]}\צהגדרה{. }

\ערך{דמיון }\הגדרה{- ע׳ במדור הכרה והשכלה והפכן.}

\ערך{דעת אלהות\mycircle{°}}\הגדרה{ - }\משנה{הבינה היותר שלמה שלה }\הגדרה{- הכרת היחש האלהי אל העולם הכללי ואל כל פרט ופרט מפרטיו, החמריים\mycircle{°} והרוחניים\mycircle{°}, (כ)יחש הנשמה\mycircle{°} אל הגוף\mycircle{°} }\מקור{[עפ״י א׳ מח]}\צהגדרה{. }

\הגדרה{ע׳ בנספחות, מדור מחקרים, אלהות, שני דרכים בהכרת האלהות. ע׳ במדור שמות כינויים ותארים אלהיים, הבנה האישית של האלהות. ושם, הבנה הכללית של האלהות.}

\ערך{דעת את ד׳ וההליכה בדרכיו - }\הגדרה{התרחבות הכח המוסרי\mycircle{°} }\מקור{[ע״א א ה פג]}\צהגדרה{.}

\ערך{דעת ד׳}\הגדרה{ - תלמוד של חלקי התורה הנוגעים לבירורן של המדות והדעות }\מקור{[ל״ה 178 (פנק׳ ב קכב)]}\צהגדרה{.}

\ערך{דעת ד׳ ויחודו}\הגדרה{ - }\מעוין{◊ }\הגדרה{דעת ד׳ ויחודו הוא מצד הנשמה\mycircle{°} לבד}\צהגדרה{ }\מקור{[פ״א קעח]}\צהגדרה{. }

\ערך{דעת צור\hebrewmakaf העולמים\mycircle{°} }\הגדרה{- דבקות בחיי\hebrewmakaf החיים\mycircle{°}, המתעלה על כל גבול ומחזה עין בשר }\מקור{[עפ״י ע״א א (מהדורא בתרא) א ב]}\צהגדרה{.}

\ערך{דק }\הגדרה{- רוחני }\מקור{[עפ״י פנק׳ ד תלא]}\צהגדרה{.}

\הגדרה{ע״ע גס.}

\ערך{דרור }\הגדרה{- }\משנה{הדרור האמיתי }\הגדרה{- הדרור המתאים להתכונה הפנימית הנטועה בנפש אשר עשה אלהים אותה ישרה\mycircle{°} }\מקור{[ע״א ד יג יא]}\צהגדרה{. }

\הגדרה{ע״ע חרות. ע״ע חפש. }

\ערך{דרישה}\myfootnote{ \textbf{דרישה} - הכוונה למחקר המדעי.\label{64}}\ערך{ }\הגדרה{- }\משנה{האימות היותר גדול בה}\הגדרה{ - הניסיון המוחשי }\מקור{[פנק׳ ג כד]}\צהגדרה{.}

\ערך{דשן }\הגדרה{- המחיה הגופנית של המאכל, המלא לשד ושמן, החומר המחיה, האוצר בתוכו כח חיים רב ועצום, החוזר לערך של חיים בהבלעו בגויה המצומצמה }\מקור{[ע״ר א כ]}\צהגדרה{. }

\ערך{דשן }\הגדרה{- }\משנה{״דשן ביתך\mycircle{°}״}\myfootnote{ תהילים לו ט.\label{65}}\הגדרה{ - התוכן התמציתי של הזוהר\mycircle{°} האלהי\mycircle{°} העליון\mycircle{°} בצורה מוגבלה, הראויה להיות המזון המבריא, והנותן את המחיה להנשמה הבריאה, השוקקת אל הנועם\mycircle{°} האלהי }\מקור{[ע״ר א כ]}\צהגדרה{. }

\ערך{דת }\הגדרה{-}\צהגדרה{ א}\הגדרה{מצעי עוזר להכשיר את המעשים, המדות, הרגשות, סדרי החברה החיצונים והפנימיים, באופן מתאים המכשיר את החיים ואת ההויה לדעת\hebrewmakaf אלהים\mycircle{°} }\מקור{[קבצ׳ ב נט]}\צהגדרה{.}

\משנה{הדת המעשית }\הגדרה{-}\משנה{ תוכנה, וכל הרגשות המתיחסים לה }\הגדרה{- רוח אלהים\mycircle{°} בצבע\mycircle{°} האומה }\מקור{[קבצ׳ ב פג]}\צהגדרה{.}

\ערך{דת }\הגדרה{-}\משנה{ ״אורתודוקסיה״, נשמתה }\הגדרה{- תכונת דרישת\hebrewmakaf ד׳\mycircle{°} }\מקור{[עפ״י אג׳ ב ח]}\צהגדרה{. }

\משנה{רגשי דת}\הגדרה{ - ציור\mycircle{°} קרבת\hebrewmakaf אלקים\mycircle{°} על דרך השכר והעונש }\מקור{[קבצ׳ ב ל]}\צהגדרה{.}

\ערך{דת }\הגדרה{- }\משנה{רגש דתי}\הגדרה{ - רגש נשגב נטוע בנפש האדם מיוצר נשמתו, כדי להטביע בקרבו את היחש האמיתי שבין האדם לקונו, בתור היחש הראוי להיות בין יציר ליוצרו. [רגש ה]חובק בזרוע עזו את כל יסודות החיים והמוסר\mycircle{°} הכללי והפרטי }\מקור{[עפ״י ל״ה 132]}\צהגדרה{.}

\משנה{עקרה של הדת }\הגדרה{- להרגיל את האדם בהכנעה ושיעבוד לאלהים\mycircle{°} לפי מובן האדם בהוה. הרגל הנפשות לדביקות\mycircle{°} ועבודת\hebrewmakaf ד׳\mycircle{°}, שממנו יתד ופינה לכל מוסר ומעגל טוב. מדה כללית מוסרית הנובעת ממקור ההכרה האמיתית של יחש האדם לקונו ולנשמתו הרוחנית }\מקור{[עפ״י ל״ה 81]}\צהגדרה{.}

\משנה{הכלל הכולל את הדתות }\הגדרה{- ההכשרה הרוחנית של האדם ביחשו לקונו על פי הרגש המוסרי הפנימי}\צהגדרה{ }\מקור{[ל״ה 82 (פנק׳ ב סג)]}\צהגדרה{.}

\משנה{רגש דת פשוט }\הגדרה{- }\מעוין{◊}\הגדרה{ מתבאר לאדם עפ״י השתדלותו להיות יותר קרוב לבוראו בדרכיו, מעשיו ודעותיו }\מקור{[עפ״י ע״א ב 384]}\צהגדרה{. }

\משנה{יסוד הרגש הדתי }\הגדרה{- הגעגועים אל הקדושה\mycircle{°} ורוממות הנפש לאהבת\hebrewmakaf ד׳\mycircle{°} וכבודו ופחד גאונו }\מקור{[מא״ה ב מד (קבצ׳ א מו)]}\צהגדרה{.}

\הגדרה{ע״ע עבודת ד׳, (המצויה גם בעמים). ע״ע רוח האמונה. ע״ע רליגיוזיות, הטבעיות הרליגיוזית. ע״ע דת, בישראל, רגש הדת בישראל. ע׳ במדור תורה, תורת חיים. ע׳ במדור אליליות ודתות, דת, אצל כל עם ולשון (חוץ מישראל). ע״ע אידיאה דתית.}

\ערך{דת }\הגדרה{- }\משנה{(לעומת מסורת\mycircle{°}) }\הגדרה{- יסודה הפנימי של המסורת }\מקור{[ב״א 11]}\צהגדרה{. }

\ערך{דת }\הגדרה{- }\משנה{בישראל }\הגדרה{- אור\hebrewmakaf ד׳\mycircle{°} שבנשמת\hebrewmakaf (ישראל\mycircle{°}), הבעת החיים היותר עצמיים והיותר פנימיים שלו, מה שנתן ונותן לו את הכל, שמעמידו על הרום העליון של במתי\hebrewmakaf ארץ, על המעמד של מורה האנושיות כולה }\מקור{[אג׳ ב רט]}\צהגדרה{. }

\הגדרה{״גופא דאורייתא״, ״נר אלהים בארץ״  }\מקור{[עפ״י א״ת ב 217 (פנק׳ ד עב)]}\צהגדרה{.}

\משנה{דת ישראל}\הגדרה{ - המבוע של אור\hebrewmakaf ד׳\mycircle{°} בעולם}\צהגדרה{ }\מקור{[ל״ה 104]}\צהגדרה{.}

\משנה{רגש הדת בישראל}\הגדרה{ - }\מעוין{◊ }\הגדרה{בא מהגעגועים אל הקדושה ורוממות הנפש לאהבת\hebrewmakaf ד׳\mycircle{°} וכבודו\mycircle{°} ופחד הדר\mycircle{°} גאונו; שכל התורה כולה היא הכנה לזכות את העולם (בעתיד) לדברים שעליהם מיוסד טבע הגעגועים האלה אל הטוב}\צהגדרה{ }\מקור{[עפ״י פנק׳ א קכג (מא״ה ב מד)]}\צהגדרה{. }

\הגדרה{ע״ע מוסר, רגש המוסר. ע״ע רליגיוזיות, הטבעיות הרליגיוזית.}

\משנה{דתיות }\צהגדרה{- הגילוי וההוצאה\hebrewmakaf לפועל, במחשבה ובמעשה, של התורה\mycircle{°} ומצוותיה\mycircle{°} }\צמקור{[ל״י א לג].}

\הגדרה{ע״ע עבודת אלהים}\צהגדרה{.}

\ערך{דת }\הגדרה{- }\משנה{אצל כל עם ולשון (חוץ מישראל) }\הגדרה{- ע׳ במדור אליליות ודתות.}\mylettertitle{ה}

\משנה{ }

\ערך{הא }\הגדרה{- מורה על המוכן ומוכשר להושטה }\מקור{[עפ״י ר״מ פה]}\צהגדרה{. }

\ערך{האדרת שם ד׳}\הגדרה{ - גלוי עז\mycircle{°} הגבורה\hebrewmakaf האלהית\mycircle{°} הפועלת את הכל למען הרוממות האצילית, המסוקרת אך לפני כסא\hebrewmakaf כבודו\mycircle{°} של בורא כל העולמים ברוך הוא. ההופעה העזיזה החודרת מרום הגובה העליון עד שפל המדרגה של אדם על הארץ\mycircle{°} }\מקור{[עפ״י ע״א ד ט קד]}\צהגדרה{. }

\ערך{האח }\הגדרה{- קריאת השמחה\mycircle{°} והחדוה\mycircle{°} }\מקור{[ר״מ קכ]}\צהגדרה{. }

\ערך{האצלה}\myfootnote{ בשו״ת הרדב״ז, ח״ג סי׳ תתקי ״בהיות האדם מתכוון אל רבו ונותן אליו לבו תתקשר נפשו בנפשו ויחול עליו מהשפע אשר עליו ויהיה לו נפש יתירה וזה נקרא אצלם סוד העיבור בחיי שניהם, וזה הוא שנאמר ״והיו עיניך רואות את מוריך״ וזהו ״והתיצבו... עמך שם ואצלתי מן הרוח״.\label{66}}\הגדרה{ - הזרחת\mycircle{°} אור\mycircle{°} ויפעת\mycircle{°} אצילות\mycircle{°}, המעדנת את הנשמה\mycircle{°} בהשפעה\mycircle{°} נעלה מעל מכל ערכי בינה\mycircle{°} וחקר }\מקור{[ע״ר א לג]}\צהגדרה{. }

\ערך{הארה }\הגדרה{- שביעה רוחנית\mycircle{°} }\מקור{[עפ״י ע״ר א קלד]}\צהגדרה{. }

\משנה{הארה גדולה }\הגדרה{- הופעה\mycircle{°} נשמתית אלהית, מגלה רזי\mycircle{°} עליון, מודיעה האגור בסתרי חושך }\מקור{[עפ״י א״ק ב שיז (ע״ט קיד)]}\צהגדרה{. }

\משנה{הארה רוחנית }\הגדרה{- רוממות נשמה\mycircle{°} וחדות\mycircle{°} קודש\mycircle{°} }\מקור{[שם ג שמב]}\צהגדרה{. }

\הגדרה{ע״ע מאיר, מאיר את העולמים. ע׳ בנספחות, מדור מחקרים, זריחה. }

\ערך{הארת הנשמה }\הגדרה{- ע׳ במדור נפשיות, נשמה. }

\ערך{הארת זיו שכינת אל }\הגדרה{- עלית אור עולמי עולמים, הדר כבוד אל הכבוד}\צהגדרה{ }\מקור{[א״ק ב תקל]}\צהגדרה{.}

\ערך{הבטה}\myfootnote{ מובחנת ההבטה מן ה״ראיה בעלמא״ דבה לא קני, ״דהבטה בהפקר קני״ לתוס׳ ב״מ ב. ד״ה דבראיה, ״היינו שעשה מעשה כל דהו״, שעל ידו קונה הראיה ועל כן היא קרויה - הבטה; ולרש״י שם קיח. ד״ה אתה אומר ״הבטה קניא הואיל ודבר טורח הוא ודעתו לכך״. של״ה, בי׳ מאמרות, מאמר ט, דף מא: ״יש חילוק בין הבטה לראיה כח הבטה יותר הסתכלות מראיה דעלמא כמו שפרש״י בפ׳ חוקת אצל נחש הנחשת שכתיב וראה וכתיב והביט ע״ש״. ובאור החיים עה״ת, בלק כג כא ״דקדק לומר לשון הבטה שהיא יותר מהראיה, פירוש כי הגם כי בראיה ראשונה הוא רואה מחשבה הרעה, כשהוא מביט בפנימיות מפעלה אין און״. וברוח חיים על פרקי אבות, פרק א לפני ההג״ה הראשונה ״ולשון הבטה הוא הסתכלות יתירה בעצם הענין״. ובהעמק דבר לנצי״ב במדבר כא ח, ושם כג כא: ״הסתכלות בתוך ופני הדבר״. ובאבי עזר על פי׳ הראב״ע במדבר כא ז ״גזרת הביט בנויה על השכלה זכה״. ובנפש\hebrewmakaf חיה לרר״מ סי׳ רכה ״הביט - מביט בו בכוונה״. וכן גם במלבי״ם ביאיר אור, הכרמל, ערך הבט: ״פעל הביט מציין קשורו עם המובט הוא הגיוני ומחשביי לא במציאות ובחוש הראות לבד רק משים לב על העצם לדעת מהותו וענינו״. ובתורה והמצוה, שמות ג ו ״ההבטה הוא שימת לב על הדבר ואינו בא על ראות העין״. ובבראשית, טו ה, ״ההבטה היא שימת לב על הדבר ועיון השכל... יבא לרוב על שימת הלב ועיון המחשבה... וכשבא על הראיה אמר במדרש (בר״ר פר׳ מד יב) ״אין הבטה אלא מלמעלה למטה״״. אמנם צ״ע האם עולים דבריו אלה האחרונים של המלבי״ם עם דברי המ״ר איכה פר׳ ה א ״הביטה וראה את חרפתנו. ר׳ יודן אמר הבטה מקרוב וראייה מרחוק. הבטה מקרוב שנאמר (מלכים א יט) ״ויבט והנה מראשותיו עוגת רצפים״ וראייה מרחוק שנאמר (בראשית כב) ״וירא את המקום מרחוק״. ר׳ פנחס אומר הבטה מרחוק שנאמר (תהלים פ) ״הבט משמים וראה״, וראייה מקרוב שנאמר (בראשית לב) ״וירא כי לא יכול לו ויגע בכף ירכו״, והי״ע. ואולי אפשר להבחין בין המושגים עפ״י ר״מ צז: ״מושג הראיה, וכו׳ מורה הסמנה מדויקת בפרטים, והארה כוללת בכללים״; לעומת ההבטה שהיא כניסת הרשמים מן החוץ אל הפנים. וצ״ע.\label{67}}\הגדרה{ - הסתכלות בהירה וחודרת, שמכנסת בתוך הנושא המקבל את הרושם, את כל הפרטים של הקוים והשרטוטים אשר להמושגים }\מקור{[ר״מ קלב]}\צהגדרה{. }

\הגדרה{הכרת הפרטים, אחרי שהסתמנו לפרטיהם והתקשרו אל בית קבולם }\מקור{[שם קלה]}\צהגדרה{. }

\הגדרה{ע׳ במדור גוף האדם אבריו ותנועותיו, עין. ע׳ במדור אותיות, עי״ן. }

\ערך{הבל }\הגדרה{- דבר כחני, שלא יצא אל הפועל }\מקור{[ע״ר א קה]}\צהגדרה{. }

\הגדרה{דבר שבכח שאיננו כלל בפועל }\מקור{[שם שם]}\צהגדרה{. }

\הגדרה{כל מציאות של כח, שאין לה ערך בפועל }\מקור{[שם שם]}\צהגדרה{. }

\משנה{הבל פה}\הגדרה{ - כח פנימי עצור שעתיד לצאת אל הישות המוחלטת }\מקור{[ע״א ד ט צו]}\צהגדרה{. }

\ערך{הבל }\הגדרה{- כח של הדיבור\mycircle{°} }\מקור{[מ״ש מד]}\צהגדרה{. }

\הגדרה{הדבור שבכח }\מקור{[שם נא]}\צהגדרה{. }

\הגדרה{הכנה לדיבור }\מקור{[מ״ר 423]}\צהגדרה{.}

\ערך{הגבלה }\הגדרה{- צמצום חמרי\mycircle{°} ומעצור }\מקור{[עפ״י ע״ר א קנד]}\צהגדרה{. }

\הגדרה{ע״ע מגביל. ע״ע גבולים. }

\ערך{הגשמה }\הגדרה{- הוצאה אל הפועל }\מקור{[עפ״י א״ק ג קו]}\צהגדרה{. }

\ערך{הוד }\הגדרה{- יופי, נוי\mycircle{°}, שאמתתו היא בהשלמת כל הפרטים וערכם\mycircle{°} הנעים והמדויק }\מקור{[עפ״י ח״פ לז.]}\צהגדרה{. }

\ערך{הוד }\הגדרה{- }\משנה{(בהויה\mycircle{°}) }\הגדרה{- ההתאמה הנאה של כל אורות\hebrewmakaf החיים\mycircle{°} וכל כוחות ההויה כולם בהסתדרותם הנפלאה }\מקור{[ע״ר א יב]}\צהגדרה{. }

\ערך{הוד תפארתו\mycircle{°}}\הגדרה{ - התגלותו ביקרו }\מקור{[א״ק ג שמג]}\צהגדרה{. }

\ערך{הודאה }\הגדרה{- הבעה של הכרת\hebrewmakaf טובה\mycircle{°} למטיב, ההולכת מתוך הרגש המלא, שדוחק את הלב, שלא ישאר תוכן הטובה מבלי הבעה של הכרה זו }\מקור{[עפ״י ע״ר א קז\hebrewmakaf ח]}\צהגדרה{. }

\הגדרה{מורה חובת תודה\mycircle{°} על קבלת הטוב\mycircle{°} }\מקור{[ע״א ב ט ד]}\צהגדרה{.}

\הגדרה{ע׳ בנספחות, מדור מחקרים, ברכה לעומת הודאה. }

\ערך{הודאה }\הגדרה{- }\משנה{תוכן ההודאה (לד׳) }\הגדרה{- הכרת\hebrewmakaf הטובה\mycircle{°}, שהאדם מוקף בים של השפעת\mycircle{°} הטוב\hebrewmakaf העליון\mycircle{°}, ונפשו בתכונתה העדינה מכריחה אותו, שלא יהיה כפוי\hebrewmakaf טובה וניב שפתיו יתפרץ בהבעה של הכרת הטובה למטיב העליון ב״ה }\מקור{[ע״ר א קצב\hebrewmakaf ג]}\צהגדרה{. }

\הגדרה{ע״ע תודה. }

\ערך{הויה }\הגדרה{- יציאה למציאות בפועל }\מקור{[עפ״י ע״א ד ו קו]}\צהגדרה{. }

\הגדרה{התגלות אל הפעל }\מקור{[עפ״י א״ש ו ו]}\צהגדרה{. }

\ערך{הויה }\הגדרה{- היקום וכל העולמים כולם }\מקור{[ע״ר א קנז]}\צהגדרה{. }

\משנה{כל ההויה כולה}\הגדרה{ - כל החיים, כל היופי, כל העז, כל הצדק, כל הטוב, כל הסדר, כל ההתעלות }\מקור{[א״ק ב שמט]}\צהגדרה{.}

\הגדרה{ע׳ בנספחות, מדור מחקרים, הויה. ושם, הויה, מציאות, חיים. }

\ערך{הופעה אלהית\mycircle{°}}\הגדרה{ - הסתעפות חיים וישות ממקור\hebrewmakaf החיים\mycircle{°} והיש }\מקור{[א״ק א ב (מ״ר 401)]}\צהגדרה{. }

\משנה{ההופעה האלהית הכוללת כל }\הגדרה{- אור הקודש\hebrewmakaf העליון\mycircle{°} }\מקור{[ע״ר ב עז]}\צהגדרה{. }

\ערך{הזרחה }\הגדרה{- ע׳ בנספחות, מדור מחקרים, זריחה. }

\הגדרה{ }

\ערך{הטבה }\הגדרה{- }\משנה{(הטבה רוחנית)}\הגדרה{ - הארה\mycircle{°} והרחבה, של החיים\hebrewmakaf הרוחניים\mycircle{°}, בהתפשטותם המלאה\mycircle{°} והכבירה }\מקור{[עפ״י ע״ר א קמו]}\צהגדרה{. }

\ערך{הטבה }\הגדרה{- }\משנה{ההטבה היותר עליונה\mycircle{°}}\הגדרה{ - ההתעלות\mycircle{°} הקדושה\mycircle{°} היותר טהורה\mycircle{°}, ואור\hebrewmakaf ד׳\mycircle{°} הבהיר בהגלותו בתועפות עזו }\מקור{[שם קסו]}\צהגדרה{. }

\הגדרה{ע״ע טוב. }

\הגדרה{ }

\ערך{הטבעה }\הגדרה{- קביעת צורה }\מקור{[רצי״ה א״ש יא הערה 23]}\צהגדרה{. }

\ערך{היכל}\הגדרה{ - ע׳ בנספחות, מדור מחקרים, בית לעומת היכל.}

\ערך{היכל }\הגדרה{- הזיו\mycircle{°} התוכי הפנימי\mycircle{°} של אל אלהי\mycircle{°} ישראל\mycircle{°}, המיחד אותנו בעליות\mycircle{°} הקודש\mycircle{°} של קדושת\mycircle{°} המצוות\mycircle{°} והאורות\hebrewmakaf האלהיות\mycircle{°} המיוחדות לעם סגולה\mycircle{°}. הטרקלין }\מקור{[ע״ר א ד]}\צהגדרה{. }

\הגדרה{ע׳ במדור פסוקים ובטויי חז״ל, חצרות ד׳. ע׳ שם, טרקלין מלכו של עולם.}

\ערך{״היכל ד׳״ }\הגדרה{- בית\hebrewmakaf המקדש\mycircle{°} מצד היותו מרכז הקדושה\mycircle{°} וע״י נשפע טהרה\mycircle{°} בלב והכשר לכל המדות היקרות. כנגד קדושת המדות המכינות יותר את לב האדם לדעת\hebrewmakaf את\hebrewmakaf השי״ת\mycircle{°} וכבודו\mycircle{°} }\מקור{[עפ״י ע״א א א ז]}\צהגדרה{.}

\הגדרה{ע׳ במדור משכן ומקדש, ״בית ד׳״.}

\ערך{״היכל מלך״ }\הגדרה{- }\משנה{סודו }\הגדרה{- ע׳ במדור מונחי קבלה ונסתר.  }

\ערך{היפנוטיזם }\הגדרה{-◊  מי שנפשו מסוגלת לזה, מושל על המושפע ממנו, עד שמה שרוצה שיצייר\mycircle{°}, כך מצוייר ממילא }\מקור{[פנק׳ ג פז]}\צהגדרה{.}

\ערך{הכנה }\הגדרה{- }\משנה{(לעבודת ד׳)}\הגדרה{ -  הכשרתו של האידיאל\mycircle{°} ההולך ומתהוה }\מקור{[ע״ר א קפו]}\צהגדרה{. }

\ערך{הכספה }\הגדרה{- כחישות ומיעוט ההבהקה }\מקור{[ע״א ב ז מט]}\צהגדרה{. }

\ערך{הכרת טובה}\myfootnote{ \textbf{הכרת טובה} - ע״ע פנק׳ א ו\hebrewmakaf ז ובהערה 1 שם. א״ה 728, 8\hebrewmakaf 757, א״ת פרק י יז.\label{68}}\ערך{ }\הגדרה{- }\מעוין{◊}\הגדרה{ מקור הרגש, הטבעי כ״כ לאדם, להתנשא אל הקדש\mycircle{°} ואל השאיפה האלהית\mycircle{°} הנעלה, גם במדה המתאימה לצמצום\mycircle{°} כוחות הנפש הטבעיים }\מקור{[ע״ר א קעה]}\צהגדרה{. }

\מעוין{◊ }\הגדרה{העמוד המוסרי\mycircle{°} היותר גדול ונשגב, שכשיתפתח כל צרכו בלבות בני אדם יהי׳ עוזר מאד אל התיקון הכללי }\מקור{[ע״א ג א יד]}\צהגדרה{.}

\הגדרה{ע״ע הודאה. ע״ע תודה. }

\ערך{הכתרה }\הגדרה{- }\משנה{ענין הכתרת מלך}\הגדרה{ - מורה תוספות גדולה והנהגה נסתרת שיש בשפע הרצון שהשי״ת משפיע על המלך\mycircle{°} }\מקור{[פנק׳ ג רמז]}\צהגדרה{.}

\הגדרה{ע״ע שליטה.}

\ערך{הִלוּל }\הגדרה{- }\משנה{ההילול האמיתי לשם ד׳ }\הגדרה{- הכרת הוד כבודו\mycircle{°} ושלמותו, משלמות מעשיו ופעולותיו. שיגלה בחיים תמיד צד יותר עליון מהשלמה, מה שלא היה לפני זה. <בזה יתואר הילול מגזרת ״יהל אור״\mycircle{°}>}\myfootnote{ לקוטי תורה לרש״ז, בשלח, דף א עמודה ב ס״ק ב. ״״כל הנשמה תהלל י״ה״, פי׳ תהלל לשון שבח ולשון הארה כמ״ש בהלו נרו היינו שהנשמה  תמשיך אור וגילוי שם י״ה וכו׳״.\label{69}}\הגדרה{, הארה חדשה, הברקה חדשה, הוספת שלמות חדשה }\מקור{[עפ״י ע״א ג ב לו]}\צהגדרה{. }

\הגדרה{מגזרת ״יהל אור״, הארה\mycircle{°} והופעה\mycircle{°} בהכרת החיים העליונים\mycircle{°} ברום ערכם, בהופיעם ממכון הטוב\mycircle{°} והעלוי\mycircle{°} הנשגב\mycircle{°}, להאיר על האופק של המציאות\mycircle{°} המוגבלה\mycircle{°} שפעת נהורים רחבי ידים וגדולי ערך }\מקור{[ע״ר א קצג]}\צהגדרה{. }

\מעוין{◊ }\משנה{ההילול}\הגדרה{ הבא מתוך מעמקי הנשמה לצור\hebrewmakaf כל\hebrewmakaf עולמים\mycircle{°} נובע הוא מההופעה\mycircle{°} הגדולה המקיפה באורה את נשמתנו, מההכרה של הטוב\mycircle{°}, של השלמות העליונה, שנשמת כל חי אליו עורגת, והצמאון\hebrewmakaf העליון\mycircle{°} והרוממות הבלתי משוערת בשיגוב קדושת\mycircle{°} אור\hebrewmakaf ד׳\mycircle{°} היא מפתחת בקרבנו את ההילול מעל לכל רגש וטעם }\מקור{[עפ״י ע״ר ב עט]}\צהגדרה{. }

\הגדרה{ע״ע תהילה. ע״ע שבח. ע״ע זמרה. ע׳ במדור פסוקים ובטויי חז״ל, התהללות, ״התהללו בשם קדשו״. }

\ערך{הליכה }\הגדרה{- יציאה מהכח אל הפועל והנהגה מפורטת }\מקור{[מ״ר 82]}\צהגדרה{. }

\ערך{הליכה }\הגדרה{- }\משנה{במהלך שלמות האדם }\ערך{- }\הגדרה{להוסיף לקנות מעלות מדתיות ושכליות }\מקור{[ע״ר א רסב]}\צהגדרה{.}

\צהגדרה{ע״ע עמידה.}

\ערך{הליכה בדרכי ד׳ }\הגדרה{- ע״ע דעת את ד׳ וההליכה בדרכיו.}

\ערך{הלכה }\הגדרה{- }\משנה{הלכות }\הגדרה{- פרטי הנהגות התורה\mycircle{°}. והמה באמת ״הליכות\hebrewmakaf עולם״\mycircle{°} }\מקור{[מ״ר 82]}\צהגדרה{. }

\הגדרה{עצות מרחוק}\myfootnote{ \textbf{עצות מרחוק} - ישעיה כה א. רמב״ם סוף ה׳ תמורה ״ורוב דיני התורה אינן אלא עצות מרחוק מגדול העצה לתקן הדעות וליישר כל המעשים״. ע״ע זוהר ח״ב פב:, ח״ג רב. \label{70}}\הגדרה{ שנתן האל הטוב ב״ה לבני אדם איך להגיע על ידי קיומם ולימודם לשלמותם, ענפים מחכמת אדון כל המעשים ב״ה, (ה)ערוכים ע״פ משפט חכמתו\hebrewmakaf העליונה\mycircle{°} }\מקור{[א״ה ב 306]}\צהגדרה{.}

\משנה{הלכה }\הגדרה{- }\משנה{ענינה }\הגדרה{- כיוון המעשה ע״פ פנימיות השכל ואחיזתו בה}\צהגדרה{ }\מקור{[א״ה ב (מהדורת תשס״ב) 214]}\צהגדרה{.}

\הגדרה{ע׳ במדור מצוות, הלכות, מנהגים וטעמיהן, בהגדרות מבוא, מצוות. ושם, הליכות של התורה. ע׳ במדור תורה, משנה. ע׳ במדור תורה, הלכה, ההלכה. ושם, הלכה, חכמת ההלכה.}

\ערך{המון }\הגדרה{- }\משנה{ההמון}\הגדרה{ - (בני אדם בעלי נטיה) ארצית ומעשית, (ש)כל מעייניהם לחיי הזמן והעולם, (ו)הקפה רוחנית עולמית לא תיווצר ברוחם }\מקור{[עפ״י א׳ עג]}\צהגדרה{.}

\ערך{המרה }\הגדרה{- }\משנה{ההמרה היסודית (המרת דת) }\הגדרה{- קציצת העיקר באמונה, והסרת הלב מאחרי ד׳\mycircle{°} }\מקור{[ע״א ד ט מד]}\צהגדרה{.}

\הגדרה{ר׳ מומרות.}

\ערך{המשכה }\צהגדרה{- גילוי ושכלול }\ערך{[עפ״י א״ל רלב].}

\ערך{הנהגה עליונה }\הגדרה{- }\משנה{מדת ההנהגה העליונה }\הגדרה{- דרכי\hebrewmakaf אלהים\hebrewmakaf חיים\mycircle{°}: החיים יוצאים מהרכבה ממוזגת של כוחות שכ״א מונח בטבעו לפעול בשפע רב יותר מקיומו של עולם, וממילא הוא מביא בכחו הריסה וחורבן, וכן השני; וע״י התחברם והתמזגם נמצא[ים] החיים וכל המונם נצבים לפנינו וקיימים ונהנים מזיו\mycircle{°} החיים }\מקור{[עפ״י ע״א ב ט קלא]}\צהגדרה{.}

\הגדרה{העולם כולו ביחוד עולם החיים, הוא משתכלל, לא ע״י חיבורי כחות מצומצמים שכ״א ואחד פועל על גבולו, כ״א ע״י התחברות של כחות, שכ״א שואף להתרחב יותר מגבולו, וחברו ג״כ שואף כן, וכשהם פוגשים זה בזה דוחק כ״א את חברו ועוצרו, וע״י התגוששויות כאלה מתהוה חזיון החיים. זה נוהג בין בטבע החמרי, בין אפילו בטבע השכלי והמוסרי. וזהו אור זיו חסד החיים, המתמלאים תמיד ממקור אין\hebrewmakaf סוף בכל צדדיהם, עד שהם מסבבים חיים, בין בהתגברם והתפרצם, בין בהדחקם והעצרם }\מקור{[עפ״י אג׳ א פד-פה וההגדרה הקודמת]}\צהגדרה{. }

\ערך{הנהגה צבורית }\הגדרה{- }\משנה{יסוד הגדולה שלה }\הגדרה{- (ההנהגה) המרכזת דעות שונות בודדות ומנוגדות זו מזו המפוזרות בכללות הצבור, ומביאה אותם לידי מערכה גדולה ומסודרת של אמת אחת גדולה וערוכה בכל }\מקור{[ע״א ד ו נג]}\צהגדרה{.}

\הגדרה{ע׳ בנספחות, מדור מחקרים, מנהיגים (סוגי מנהיגים). ע״ע שליטה. ע׳ במדור תורה, ״הלכות צבור״. }

\משנה{הסתפחות}\myfootnote{ ע׳ ישעיה יד א. \label{71}}\משנה{ }\הגדרה{- }\צהגדרה{התחברות לא ממשית הדוקה אלא חיצונית ורופפת, שסופה שנהפכת ל׳ספחת׳ }\צמקור{[עפ״י ק״ת סג].}

\הגדרה{ר׳ לויה. }

\ערך{הפנוזה }\הגדרה{- ע״ע היפנוטיזם.}

\ערך{הצלחה אמיתית }\הגדרה{- הישועה\hebrewmakaf האמיתית\mycircle{°}, הדבקות\hebrewmakaf האלהית\mycircle{°}, המושכלת, החיה, המורגשת, מצד גודל אחיזתה בכל תוכן החיים, בכל מהותו של האדם, בקדושתו\mycircle{°}. ״כל עצמותי תאמרנה ד׳ מי כמוך״ }\מקור{[ע״ר א רטז]}\צהגדרה{. }

\ערך{הצלחה פנימית }\הגדרה{- }\משנה{יסודה }\הגדרה{- שידע האדם שאשרו ימצא תמיד בעצמו בקרב נפשו ולבבו ולא ילך לבקשה אצל אחרים }\מקור{[ע״א ד ד ט]}\צהגדרה{. }

\ערך{הצצה}\myfootnote{ זוהר ח״ב רנ: ״מציץ, כמאן דאשגח מאתר דקיק, דחמי ולא חמי כל מה דאצטריך״.\label{72}}\הגדרה{ - הבטה\mycircle{°} שע״י הדחק, היינו בהשתדלות לראות דבר שלא היה ראוי להרגיש בו אם לא בהתכוונות יתירה }\מקור{[ע״א א ב כו]}\צהגדרה{. }

\ערך{הקשבה }\הגדרה{- השמעות אזנים, הקלטה את מה שבא מן החוץ }\מקור{[קובץ ח רי]}\צהגדרה{. }

\ערך{הרהור חטא\mycircle{°}}\הגדרה{ - }\מעוין{◊}\הגדרה{ הרהור של משגה הדעת\mycircle{°} וכהות ההשכלה }\מקור{[עפ״י ע״ר א כח]}\צהגדרה{. }

\ערך{הרמוניא }\הגדרה{- התאמה מאחדת }\מקור{[א״ש ח ז]}\צהגדרה{. }

\משנה{הרמוניה }\הגדרה{- הדרת\mycircle{°} האחדות\mycircle{°} ביקרתה }\מקור{[קובץ א תו]}\צהגדרה{. }

\ערך{השפעה }\הגדרה{- }\משנה{(השפעת חיים) }\הגדרה{- ערך החיים }\מקור{[עפ״י קובץ ו סג (א״א 132)]}\צהגדרה{. }

\הגדרה{ע״ע שפע. }

\ערך{השקפה שרשית }\הגדרה{- }\משנה{ההשקפה השרשית }\הגדרה{- ראיה בכל בריה את מסקנת החיים של שטח רחב מאד, של עומק ורום גדול ונעלה, המרוממת, לפי בהירותה, כל בריה למעלתה האידיאלית\mycircle{°} }\מקור{[עפ״י א״ק ב שנז]}\צהגדרה{. }

\הגדרה{ע׳ במדור מונחי קבלה ונסתר, העלאת מחשבות לשרשן. ושם, גונין דלא מתחזין ע״י גונין דמתחזין. ושם, בהגדרות מבוא, סוד, יסוד סוד ד׳ ליראיו. ע׳ במדור פסוקים ובטויי חז״ל, דימוי הצורה ליוצרה. }

\ערך{התבודדות}\myfootnote{ ע׳ חובות הלבבות, שער הפרישות פרק ג. ע״ע א״ל עמ׳ קפט.\label{73}}\הגדרה{ - (}\משנה{ענינה}\הגדרה{) - התעלות הרעיון, התעמקות המחשבה, השתחררות הדעה }\מקור{[א״ק ג רע]}\צהגדרה{.}

\צהגדרה{דבקות אלהית של המחשבה וההרגשה. יחוד המחשבה ורוממות הדעת של קדושת אמת }\צמקור{[א״ל קפט].}

\ערך{התבודדות עליונה }\הגדרה{- התקשרות פנימית למגמת החיים והעולם, למטרת ההויה ברזי\mycircle{°} רזיה }\מקור{[א ׳ מה]}\צהגדרה{.}

\ערך{התבוננות }\הגדרה{- }\משנה{עצם חיי התבוננות }\הגדרה{- סידור המחשבות\mycircle{°} }\מקור{[א״ק ג רד]}\צהגדרה{.}

\הגדרה{ע״ע פיוט ושירה.}

\ערך{התגלות }\הגדרה{- יציאה אל הפועל בעולם }\מקור{[עפ״י ע״ר א ריט]}\צהגדרה{.}

\ערך{התגלות המהות }\הגדרה{- }\משנה{באה (באדם)}\הגדרה{ - }\מעוין{◊ }\הגדרה{כפי אותה המדה שהבחירה\hebrewmakaf החפשית\mycircle{°} מגלה את תכנה }\מקור{[עפ״י אג׳ ב מא]}\צהגדרה{.}

\ערך{התגלמות }\הגדרה{-}\משנה{ (של אידיאל)}\הגדרה{ - התעטפותו (של האידיאל\mycircle{°}) בהגדרה מיוחדת }\מקור{[עפ״י א׳ קלג]}\צהגדרה{. }

\ערך{התגשמות }\הגדרה{- }\משנה{ההתגשמות }\הגדרה{- החמריות\mycircle{°}, המעשיות }\מקור{[א״ק א עט]}\צהגדרה{.}

\ערך{התהוות }\הגדרה{- התגלות\mycircle{°} הכח של היכולת\mycircle{°}, המלאה גבורת\mycircle{°} אין\hebrewmakaf קץ, בחוג המתואר חצוני לגבי עצמות הישות }\מקור{[ע״ר א ט]}\צהגדרה{.}

\ערך{התהוות }\הגדרה{- }\משנה{להוות }\הגדרה{- לחולל, להרבות חיים וישות }\מקור{[ר״מ קיז]}\צהגדרה{.}

\ערך{התנגדות }\הגדרה{- }\משנה{(עניינה של תנועת ההתנגדות הכללית) }\הגדרה{- הדאגה הרבה ליסוד המעשי שלא יתמוטט ע״י ההתגברות של הנטיה אל הרגש, ועל הפרטים, שהם מעמידי הכללים, שלא יתטשטשו ע״י הנטיה אל הכללים, ועל כח הדמיון, המתעורר גם ע״י התרגשות של הרגשות טובות, קדושות ואמיתיות, שלא יעבור את גבולו ויביא תוצאות רעות ומרות לכלל האומה לדורות הבאים}\צהגדרה{ }\מקור{[א״י כה]}\צהגדרה{.}

\הגדרה{ע״ע חסידות, (מגמת תנועת החסידות).}

\משנה{התנוצצות }\צהגדרה{- תחילת ההופעה\mycircle{°}, הברקה של ראשית האור }\צמקור{[ש׳ רד״ך 216].}

\ערך{התעלות }\הגדרה{- התקדשות\mycircle{°}, והתרוממות\mycircle{°} אל המרומים\mycircle{°} כולם }\מקור{[עפ״י א״ק א קפח (ע״ר ב ז)]}\צהגדרה{.}

\משנה{התעלות האדם }\הגדרה{- התפתחותו, בגילוי כל כשרונותיו הפנימיים }\מקור{[א״ק א צז]}\צהגדרה{.}

\הגדרה{להיות חפץ באמת\mycircle{°}, להיות חי ופועל לפי הדיעות היותר אמיתיות וההרגשות היותר קדושות לטוב ולחסד, כמעשה גדולי העולם אשר נגשו אל ד׳ במעשיהם הבהירים למלא את העולם חסד ואמת }\מקור{[ע״א ג ב קפד]}\צהגדרה{.}

\הגדרה{כל מה ששייכותו היא יותר גדולה לתוכן הפנימי של ההוויה והחיים }\מקור{[א״ק א יז]}\צהגדרה{.}

\משנה{עליה במעלה יותר שלמה ממה שהיה עומד עליה בתחילה - ששלטון השכל על חומרו הוא במעלה יותר שלמה ממה שהיתה אצלו בתחילה }\צהגדרהמודגשת{[ע״א ד ו כח].}

\משנה{ההתעלות התדירה במעלות הקודש}\הגדרה{ - הרצון הקבוע להליכה מתמידה בדרך הקדושה והטהרה }\מקור{[ע״ר ב סה]}\צהגדרה{. }

\משנה{התעלות הנפש למרומי שמי ד׳ }\צהגדרה{- ההכרה העליונה בהדר\mycircle{°} גאון\hebrewmakaf ד׳\hebrewmakaf ועוזו\mycircle{°}, ההכרה בכלליות\mycircle{°} המציאות והאהבה\mycircle{°}, היינו, הקרבה\hebrewmakaf אל\hebrewmakaf ד׳\mycircle{°} }\צמקור{[צ״צ א כד (א״ל ריב)].}

\הגדרה{ע״ע עלוי. ע׳ במדור פסוקים ובטויי חז״ל, עליה.}

\ערך{התעלות }\הגדרה{- }\משנה{ההתעלות הגמורה }\הגדרה{- התמזגותו של כל ערך\mycircle{°} ההויה בכל צורותיה, הקדש\mycircle{°} והחול\mycircle{°} שלה, לערך הקדש, ביסוד המכון של קדש\hebrewmakaf הקדשים\mycircle{°} }\מקור{[ע״ר א קעג]}\צהגדרה{.}

\משנה{ההתעלות העליונה }\הגדרה{- ע׳ במדור פסוקים ובטויי חז״ל, סוכת עורו של לויתן. ע״ע עלוי גמור.}

\ערך{התקטנות }\הגדרה{- העזבות ממקור השפע\mycircle{°} }\מקור{[עפ״י ע״ר א קנב]}\צהגדרה{.}

\הגדרה{ע״ע קטנות.}

\ערך{התרוממות }\הגדרה{- הגבהת\mycircle{°} הערך\mycircle{°} האמיתי של האדם }\מקור{[עפ״י ע״ר א קפו]}\צהגדרה{.}\mylettertitle{ו}

\ערך{ודאות }\הגדרה{- התרפאות הנשמה, בקבלתה צורה תקיפה, עדינה, מלאה תנחומות, ועזוז\mycircle{°} קודש בגבורה עליונה }\מקור{[עפ״י א״ק א רי (ע״ט קכח)]}\צהגדרה{. }

\משנה{יסוד הודאות }\הגדרה{- עדן\mycircle{°} }\מקור{[עפ״י שם שם רה]}\צהגדרה{. }

\ערך{ודאות באמיתותה של התורה }\הגדרה{- ע׳ בנספחות, מדור מחקרים. }

\ערך{ודאות מוחלטה }\הגדרה{- }\משנה{הודאות המוחלטה בצורתה האידיאלית }\הגדרה{- מקור כל הודאיות, ומקור כל הספקות, מקור שאיבת הודאות, שמשם כל הספקות שואבים, להחיותם, להעלותם\mycircle{°}, לרעננם\mycircle{°} ולפארם בפאר\mycircle{°} חי\hebrewmakaf העולם\mycircle{°}. האורה\mycircle{°} המופלגה, מקור ששון מלא עולם, מקור הגבורה\mycircle{°}, מקור החסד\mycircle{°}, מקור התפארת\mycircle{°}, הנצח\mycircle{°} וההוד\mycircle{°}, מקור כל יסוד\mycircle{°} עולמים\mycircle{°}, מקור כל מלכות\mycircle{°} כל אדנות\mycircle{°} כל שלטון\mycircle{°}, כל כח\hebrewmakaf ד׳\mycircle{°}, כל גבורת אצילות\mycircle{°} בריאה\mycircle{°} יצירה\mycircle{°} ועשיה }\מקור{[עפ״י א״ק א רה\hebrewmakaf ו]}\צהגדרה{.}

\הגדרה{הרז\mycircle{°} הפנימי\mycircle{°} של הערכים\mycircle{°}, מיסוד החכמה\hebrewmakaf העליונה\mycircle{°}. הראשית\mycircle{°}, המחשבה\hebrewmakaf האצילית\mycircle{°} }\מקור{[עפ״י שם רז]}\צהגדרה{.}

\הגדרה{ע׳ במדור מונחי קבלה ונסתר, אבא.}

\ערך{וו }\הגדרה{- קרס מחבר נושאים מפורדים }\מקור{[עפ״י ר״מ פה]}\צהגדרה{. }

\ערך{וידוי}\הגדרה{ - }\משנה{מגמת הנטיה הטבעית להתודות ששם אדון כל הנשמות בלב כל בנ״א }\הגדרה{- זרית הלאה כל רצון זר לכל עון\mycircle{°} ולכל חטאת, בהארת אור האמת והיושר בנפש פנימה וקביעת הרצון לפי טבע האלהי שלו }\מקור{[עפ״י ע״א ג ב רה]}\צהגדרה{.}

\הגדרה{ע״ע תשובה, באדם.}\mylettertitle{ז}

\ערך{זבול}\myfootnote{ מלכים א ח יג.\label{74}}\ערך{ - }\הגדרה{האמצעי בין (אוהל -) ארעי (לבית -) לקבע }\מקור{[פנק׳ ה קלב]}\צהגדרה{.}

\ערך{זהירות הפנימית -}\מקור{ חסימת התאוה הבהמית, (ופקיחת) [וסקירת] העינים על כל העשוי בכל חוג הבשר והחומר [א״ק ד תכט (קובץ ה לז)]}\צהגדרה{.}

\ערך{זוהר }\הגדרה{- הזריחה\mycircle{°} וההברקה\mycircle{°} של האור לכל שלל צבעיו\mycircle{°} הרבים המתנוצצים בבהירותם הנפלאה }\מקור{[עפ״י ע״א ד ט לח]}\צהגדרה{. }

\צהגדרה{חן\mycircle{°} העושר המיוחד של האור\mycircle{°} והברק }\צמקור{[צ״צ א עט]. }

\ערך{זוהר}\myfootnote{ \textbf{זוהר, הגודל והרום הקדוש, המופע מאור חיי כל עולמים} - אור החמה לרא״א, יתרו, ריש פרשת ואתה תחזה: ״בורא כל עלמין יורה על הא״ס, וסתם עלמין בבינה שהיא נקראת עולם״. וביונת אלם, פרק ד ״דע כי אור עדיף מזוהר״ ״אור צח זהו זוהר מאור, פי׳ גדולה הבאה מהתנוצצות הכתר והחכמה יחד״. ע״ע בנספחות, מדור מחקרים, אור, זוהר, זיו.\label{75}}\ערך{ }\הגדרה{- }\משנה{הזוהר הנצחי }\הגדרה{- הגודל\mycircle{°} והרום הקדוש\mycircle{°}, המופע מאור\mycircle{°} חיי כל עולמים\mycircle{°} }\מקור{[עפ״י ע״ר א יח]}\צהגדרה{. }

\משנה{זהר עליון }\הגדרה{- שם\mycircle{°} קודש\hebrewmakaf ד׳\mycircle{°} }\מקור{[עפ״י שם ר]}\צהגדרה{. }

\ערך{זוהר}\הגדרה{ - }\משנה{הזוהר הפנימי}\הגדרה{ - החיים שהם מגמת הכל, שכל היש חי בו בנשמתו פנימה. המאור האלהי של האותיות המאושרות של שם המפורש, הפועלות והמשפיעות בתכונתן האלהית }\מקור{[קובץ ז קמו]}\צהגדרה{.}

\הגדרה{ע׳ במדור מונחי קבלה ונסתר, טל. ע׳ במדור שמות כינויים ותארים אלהיים, יה״ו.}

\ערך{זוהר }\הגדרה{- }\משנה{הזוהר הנשמתי ביסודו }\הגדרה{- השכל\hebrewmakaf העליון\mycircle{°} }\מקור{[א״ק א יא]}\צהגדרה{. }

\משנה{״זוהר הרקיע״}\myfootnote{ ב״ב ח:.\label{76}}\משנה{ }\הגדרה{- מרחב ההתפשטות הנשמתית\mycircle{°} העליונה\mycircle{°} של האדם, ההוד\mycircle{°} העליון הנמצא בנשמתו בעצמיותה הגדולה לאין חקר, האורה\hebrewmakaf העליונה\mycircle{°} של הנשמה, ההיקף העליון של הנשמה העליונה בהתרחבותה המוחלטת }\מקור{[ע״ט קכה (א״ק ד תעה)]}\צהגדרה{. }

\הגדרה{ע׳ במדור פסוקים ובטויי חז״ל, כוכב. ושם, והמשכילים יזהירו כזהר הרקיע.}

\ערך{זוהר }\הגדרה{- ההוד\mycircle{°} הנעלם, המתנוצץ ומסתתר המבהיק ונגנז שהוא העמילן ויסוד המזין של ההשכלה בכלל <שהיא מתגלה בכל המדריגות בקודש\mycircle{°} ובחול\mycircle{°}, בכל יצירה רוחנית\mycircle{°} לפי ערכה, בהוד מפואר ובזיו\mycircle{°} כמוס> המתנוסס בתוכיות הרוח הטהור שבלב זכי הרוח, השואבים מאצילות החיים של רוח\hebrewmakaf הקודש\mycircle{°} הצפון במעמקי הנשמה\mycircle{°} }\מקור{[ר״מ קעד]}\צהגדרה{. }

\צהגדרה{ }

\ערך{זוהר }\הגדרה{- }\משנה{ספר הזוהר }\הגדרה{- ע׳ במדור תורה.}

\ערך{זיו אלהי }\הגדרה{- }\משנה{מעוז הזיו האלהי שממעל לכל גבולי\hebrewmakaf עולמים }\הגדרה{- הגודל\hebrewmakaf העליון\mycircle{°} }\מקור{[ע״ר א כ, ועפ״י שם ב עד]}\צהגדרה{.}

\ערך{זיו שכינת אל }\הגדרה{- אור\mycircle{°} עולמי\hebrewmakaf עולמים\mycircle{°} }\מקור{[א״ק ב תקל]}\צהגדרה{.}

\הגדרה{ע״ע הארת זיו שכינת אל. }

\ערך{הזיו העליון }\הגדרה{- ההארה\mycircle{°} המלאה\mycircle{°} }\מקור{[עפ״י א״ק ב תה\hebrewmakaf ו]}\צהגדרה{.}

\ערך{זיו}\myfootnote{ \textbf{זיו} - בדעת תבונות, עמ׳ עה (מהד׳ ר״ח פרידלנדר) כותב הרמח״ל כהקבלה לסדרי הבריאה של הקב״ה ודרכיה בבריאה עצמה: ״יש דבר אחד נמצא מהתחברות הנשמה והגוף - הוא זיו הפנים... ואין הזיו הזה נמצא לא לנשמה בפני עצמה, ולא לגוף בפני עצמו, אבל הוא הדבר הנולד מחיבור הנשמה והגוף ביחד״. ובדרך מצותיך נא. ״זיו הנפש הוא ענין חיוני מתפשט בגוף שהוא מקבל חיות זה וחי ממנו. וכך מהותו ועצמותו המחוייב המציאות, שהוא לבדו הוא, ואין זולתו, והוא חיי החיים, כשימשיך ויאיר ממנו ויגלה הארה בבחי׳ זיו ענינה הוא המשכת חיים בבחי׳ א״ס וז״ש הודו על ארץ ושמים כו׳ ופי׳ הוד וזיו״. ע׳ בנספחות, מדור מחקרים, אור, זוהר, הוד, זיו. ושם אור, זיו, ברק.\label{77}}\ערך{ אור אלהים }\הגדרה{- }\משנה{זיו החיים }\הגדרה{- שכינת\hebrewmakaf אל\mycircle{°}. אור\hebrewmakaf אלהים\mycircle{°}, הממלא את העולמים\mycircle{°} כולם, המחיה ומרוה אותם מדשן\mycircle{°} נועם\hebrewmakaf עליון\mycircle{°} של מקור\hebrewmakaf החיים\mycircle{°}, ונותן חיל\mycircle{°} בנשמות\mycircle{°}, במלאכים\mycircle{°}, ובכל יצור, לחוש את פנימיות\mycircle{°} תחושת החיים. יד\mycircle{°} אל\mycircle{°} עליון הפתוחה ומשביעה רצון\mycircle{°} לכל. מקורם של היצורים העליונים שביצירה, ברואי מעלה ומטה }\מקור{[עפ״י א״ק ב שכט\hebrewmakaf ל]}\צהגדרה{.}

\משנה{זיו השכינה }\צהגדרה{- }\הגדרה{הארת השכינה המודדת עולמי עד במדידת ההגבלה, הזמנית והמקומית, שרשיהם ושרשי שרשיהם, כדי להבליט את הודם ונצחם, את עליוניותם וחופש עצמתם, מרחבי טיסתם ואין סופיות הערצת פאר קדושתם }\מקור{[ע״א ד ט צז]}\צהגדרה{.}

\ערך{זיו אלהי }\הגדרה{- שכינתא\mycircle{°}}\צהגדרה{ }\מקור{[א״ק ג כ]}\צהגדרה{.}

\משנה{זיו השכינה }\צהגדרה{- השפע החיוני הכללי }\צמקור{[עפ״י א״ל רמז]. }

\מעוין{◊ }\משנה{הזיו}\הגדרה{ מתנוצץ מברק היפעה\mycircle{°} המתגלה על כל המון הברואים, היצורים, והנעשים, בכל הקצבת מהותיותם, השוכן עליהם ושומר את צביונם }\מקור{[ע״ר א יג]}\צהגדרה{.}

\הגדרה{ע׳ במדור פסוקים ובטויי חז״ל, שכינה, ״זיו השכינה״. }

\ערך{זיו של מעלה }\הגדרה{- אור עליון\mycircle{°} שמעל כל רעיון\mycircle{°} ומחשבה\mycircle{°} }\מקור{[א׳ קכ]}\צהגדרה{.}

\ערך{זיו החיים }\הגדרה{- טוהר\mycircle{°} הרוח\mycircle{°}, ברק השכל, זיו\mycircle{°} המוסר\mycircle{°}, נועם האהבה\mycircle{°} והאשר\mycircle{°} }\מקור{[א״א 73]}\צהגדרה{.}

\משנה{זיו החיים (ה)מתמשך על הכל }\הגדרה{- הוד\mycircle{°} השירה\mycircle{°} בסוד חייה הולך ומתפשט על כל החוגים, הקרובים והרחוקים}\צהגדרה{ }\מקור{[א״ק א פז]}\צהגדרה{.}

\ערך{זיכוך }\הגדרה{- }\משנה{זכוך האדם}\הגדרה{ - רוממות דעתו ושכלו, כינון תכונותיו ומעשיו לצד היותר נעלה וישר\mycircle{°} }\מקור{[עפ״י קובץ א קטו]}\צהגדרה{.}

\ערך{זין }\הגדרה{- כלי המלחמה להגן נגד כל מחריב }\מקור{[ר״מ פט]}\צהגדרה{.}

\הגדרה{הכלים שהמלחמה נאסרת על ידם נגד כל מפריע, נגד כל אויב ומתנקם, הנצחון המלחמתי, הנשק, הכח לכלות כל מפריע }\מקור{[עפ״י שם ט\hebrewmakaf י]}\צהגדרה{.}

\ערך{זית }\הגדרה{- }\מעוין{◊}\הגדרה{ מורה אורה }\מקור{[ע״א ב בכורים כט]}\צהגדרה{.}

\הגדרה{ע״ע שמן זית.}

\ערך{זכויות }\הגדרה{- ע״ע זכות.}

\ערך{זכות }\הגדרה{- }\משנה{ענינה }\הגדרה{- רצונו של איש ישראל כדי להתקרב לאביו\hebrewmakaf שבשמים\mycircle{°} }\מקור{[מ״ש קעז (מא״ה א קיט)]}\צהגדרה{.}

\משנה{זכויות }\הגדרה{- הפעולות הטובות\mycircle{°} מכל צד, שתהינה טובות כשהאדם הוא (ה)הולך ופועל ע״י המכשירים שמוצא לפניו ממעשה שמים, הרי הוא מתדבק\mycircle{°} בדרכי\hebrewmakaf ד׳\hebrewmakaf יתברך\mycircle{°} פועל כל }\מקור{[ע״א ג ב קצד]}\צהגדרה{. }

\משנה{זכירה }\צהגדרה{- דבקות ההכרה }\צמקור{[ל״י א י]. }

\הגדרה{ר׳ זכרון.}

\משנה{זכירה }\צהגדרה{- }\צמשנה{זכירה את ד׳ }\צהגדרה{- דבקות ההכרה האלהית התמידית. הזכירה הפנימית, הנשמתית הקבועה, החיה והקימת בישראל, של תוכן החיים ויסודו הפנימי הכולל, ממקור החיים וההויה ומעין ישעם וחפצם }\צמקור{[עפ״י ל״י א י]. }

\הגדרה{ר׳ במדור פסוקים ובטויי חז״ל, שכחי אלהים. ע״ע זכרון (לפני ד׳).}

\ערך{זכירה - }\משנה{זכירת הצור\mycircle{°} }\הגדרה{- העלאת עצמיותו (של האדם) אל העולם המגמתי העליון}\צהגדרה{ }\מקור{[א״ק ב תקנה]}\צהגדרה{.}

\ערך{זָכָר }\הגדרה{- היסוד העקרי שבתולדה והמשכת החיים, הסגולה\mycircle{°} המפעלית }\מקור{[עפ״י ע״ר א מ]}\צהגדרה{.}

\הגדרה{כח המפעלי שבחיים }\מקור{[ע״ר א מב]}\צהגדרה{.}

\מעוין{◊}\הגדרה{ המוכן לעבודה שכלית היותר רוממה וטהורה, השולטת ג״כ על עז הרגש }\מקור{[ע״א ג ב נ]}\צהגדרה{.}

\משנה{זכרי }\הגדרה{- תוכן פועל מפעלים חיים חזקים וגמורים}\צהגדרה{ }\מקור{[ר״מ קלא]}\צהגדרה{.}

\הגדרה{ע״ע נקבה, נקבי.}

\ערך{זֵכֶר }\הגדרה{- }\משנה{(השגת זכר ד׳ לעומת שם\hebrewmakaf ד׳\mycircle{°}) }\הגדרה{- התרשמות הנרשמת על הנשמות\mycircle{°}, מצד ההשגות הבאות מסבת ההסתכלות במעשים הנפלאים של דרכי ההשגחה\mycircle{°} העליונה על כל תקופה ותקופה, בתור חטיבה פרטית. הרישום של פליאת גדולת\mycircle{°} השם יתברך וצדקו\mycircle{°} העליון, היוצא מתוך כל הדורות ומסיבותיהם }\מקור{[עפ״י ע״ר ב פג]}\צהגדרה{.}

\משנה{זכרון }\צהגדרה{- קשר שייכות ודבקות }\צמקור{[שי׳ מועדים א 25].}

\הגדרה{ר׳ זכירה.}

\ערך{זכרון }\הגדרה{- }\משנה{״זכרו לעולם בריתו״, הזכרון העולמי }\הגדרה{- תכן הברית\mycircle{°} }\מקור{[עפ״י ע״ר א רב]}\צהגדרה{.}

\ערך{זכרון}\myfootnote{ תולעת יעקב, סוד ראש השנה סוד התפילה לג: ״מה שאנו אומרים זכרנו, הוא ענין בסוד הנהר המרוה צמאונים בסוד כל ענין, כי הוא סוד (ה׳,) שפע המערכה, ׳כי שם צוה ד׳ את הברכה׳, בסוד נעלם, ׳חיים עד העולם׳. כי בהתעלות הכבוד ממדרגה למדרגה ומסבה לסבה אז החיים יוצאים ממקור מוצאם ונמשכים אל הסבות כולם״. ע״ע עטרת ראש להרד״ב, שער ראש השנה, סי׳ טו. מחשבות חרוץ לר״צ, סה. ״כל זכירה הוא העלאת הדבר לשרשו והתחלתו כמו יום הזכרון דהרת עולם, כי הזכרון הוא שמזכיר ומצייר הדבר עתה ממש כמו שהי׳ אז״. וע׳ אלפי מנשה ח״א, סוף פרק צז.\label{78}}\הגדרה{ - }\משנה{(לפני ד׳) }\הגדרה{- כח הסגולה\mycircle{°} האורית\mycircle{°} כמו שהיא בעינה לפני ירידתה\mycircle{°} והתמעטותה במשך הדורות בירידתם, שאותה הבהירות, שעמדה הסגולה הקדושה\mycircle{°} המיוחדת, שממנה באה הנקודה הנשמתית\mycircle{°} הקדושה\mycircle{°} לכל פרט מפרטינו, נזכרת ומופעת מכח שורש מציאותה }\מקור{[עפ״י ע״ר א פג]}\צהגדרה{.}

\הגדרה{ע״ע פקידה. ר׳ זכירה, זכירה את ד׳. ע׳ בנספחות, מדור מחקרים, פקידה וזכרון, ההבדל בין פקידה לשאר דרכי זכרון. }

\ערך{זמן }\הגדרה{- }\משנה{(לעומת נצח\mycircle{°}) }\הגדרה{- ההוה התדירי }\מקור{[ע״ר א סד]}\צהגדרה{. }

\הגדרה{ע׳ במדור מועדים וחגים, קידוש הזמנים. }

\משנה{זמן (וערכיו) }\הגדרה{- ההתפתחות ההדרגית (ומדותיה) }\מקור{[ע׳׳א ד ט נא]}\צהגדרה{.}

\משנה{זמן }\צהגדרה{- סדר יחסו של האדם אל העולם }\צמקור{[עפ״י א״ל רמט]. }

\מעוין{◊}\צהגדרה{ יחסי האדם והעולם מתגלים הם במציאותו של ענין ה}\צהגדרהמודגשת{זמן,}\צהגדרה{ ומסתדרים בסדריו }\צמקור{[ל״י א קלב]. }

\צהגדרה{האדם בעולם. סדר החיים של האדם בעולם הזה. ענינה של הופעת הנשמה בגוף בסדר ההסטוריה. <}\צהגדרהמודגשת{הזמן}\צהגדרה{ אינו דבר שיש לו מציאות כשלעצמו, אבל בו מצוייה עובדת מציאותנו> }\צמקור{[עפ״י שי׳ ב 132]. }

\צמשנה{ענינו של הזמן בכלל, במהותו הטבעית}\צהגדרה{ - רציפות התנועה של השינויים החליפות והתמורות, אשר יוצר בראשית, המחדש מעשהו בטובו בכל יום תמיד, משנה עתים ומחליף את הזמנים, במערכת רצונו }\צמקור{[א״ל רמט].}

\ערך{זמר}\myfootnote{ תורה אור לרש״ז, בשלח דף סב: ד״ה אך ״ענין הנגון הוא שאין בו רק התפעלות הנפש מפני גילוי התנועות ולא מפני שיש בתנועות ההם מצד עצמם שום שכל והתחדשות אלא מפני גילוי התנועה מתעורר גילוי הלב... להיות בחי׳ התפעלות הנפש מצד עצמה ולא מצד השגה להוליד מבינתו כו׳... ולכן קוראין אותן פסוקי דזמרה שהם בחי׳ נגון... כשאומר הפסוקים ואינו נוגע לנקודת לבבו אין זה בחי׳ זמרה ונגון כו׳״.\label{79}}\הגדרה{ - (בטוי) עומק ההרגשה בנועם השתפכות הנפש }\מקור{[עפ״י ע״א ד ה עח]}\צהגדרה{. }

\הגדרה{התפרצותו הבטויית של הרגש הלבבי, המתעמק בעומק הנפש, הצפון בחגוי החיים, שאינה יכולה להתלבש בבטויים, כי אם בתנועות קוליות מסודרות }\מקור{[עפ״י ע״ר א קפו]}\צהגדרה{. }

\הגדרה{הבטוי של הרגש הנפשי\mycircle{°}, ההולך ומתעמק במעמקי החיים הרוחניים\mycircle{°}, (שיש והוא בא) בתור תוצאה רבת הכח מתוך ההסתכלות הבהירה, המצמחת את השירה\mycircle{°} בתחילה }\מקור{[עפ״י שם ר]}\צהגדרה{. }

\הגדרה{השגת\mycircle{°} הנפש דרכי חייה האמיתיים בעת המשכת הזכרון להמשיך אור החכמה, אחרי שמכסה כבר הענין באופן שאין הרקיע\hebrewmakaf בטהרתו\mycircle{°} }\מקור{[עפ״י מא״ה ב רסט\hebrewmakaf רע]}\צהגדרה{. }

\הגדרה{הרוח\mycircle{°} המתלבש בלב\mycircle{°} ע״פ השכלת השכל בציור אמיתי של }\משנה{ז}\הגדרה{כרון מעניני השכל, וה}\משנה{מ}\הגדרה{חשבה בעניני השלמות וה}\משנה{ר}\הגדרה{צון בהם באמת לפעול, הרוח המשמח את הנפש\mycircle{°} ומעירה לבקש דרכים לצאת ממאסרה כדי שתחת הזכרון יזרח\mycircle{°} עליה אור\hebrewmakaf השכל\mycircle{°} }\מקור{[שם רע]}\צהגדרה{. }

\הגדרה{ע׳ בנספחות, מדור מחקרים, שיר וזמר, ההפרש ביניהם. ע״ע שיח. ע״ע רנה.}

\תמקור{השירה בשעה שהיא מתפרצת מתוכיותה, מעצמיותה, משטף יצירתה וזרמה הנשמתי כשהיא באה לידי ביטוי [מא״ה ב רסא]. }

\ערך{זמרה }\הגדרה{- הכח המתמלא בלב\mycircle{°}, בהרצון וההסכם השכלי, המורכב מ}\משנה{ז}\הגדרה{כרון }\משנה{מ}\הגדרה{חשבה }\משנה{ר}\הגדרה{צון, אם כי עוד לא נשלמה שמחתו בפעל מ״מ בצפיתו יצפה לה }\מקור{[פנק׳ ג כב (מא״ה ב רע)]}\צהגדרה{. }

\הגדרה{ע״ע שיר.}

\הגדרה{ההבעות הנותנות דחיפה לביטויינו לקרא בשם\hebrewmakaf קדשו\mycircle{°}, בהם אנחנו מרגישים את הנועם\hebrewmakaf העליון\mycircle{°}. ההבעות הקשורות לענפים החודרים לתוך התוכיות הנפשיות שלנו, המשתרגים מתוך האור העליון, שלהופעה\mycircle{°} הגדולה המקיפה באורה\mycircle{°} את נשמתנו - ההכרה של הטוב\mycircle{°}, של השלמות העליונה }\מקור{[עפ״י ע״ר ב עט]}\צהגדרה{.}

\ערך{כח הזמר }\הגדרה{- הרגש, העולה מתוך רעותא\hebrewmakaf דליבא\mycircle{°}, הבוקע את האויר הנעלם, ומתרומם בהרחבת תועפות גדלו ממעל לחוג הצר של הבטוי והרעיון המחולל אותו }\מקור{[ע״ר א קצח]}\צהגדרה{.}

\הגדרה{ע״ע הלול. ע״ע רנה. }

\ערך{זקוק }\הגדרה{- בהיר\mycircle{°} ומלא יושר\mycircle{°} }\מקור{[א״ק ג נט]}\צהגדרה{.}

\ערך{זקיפה }\הגדרה{- התמתחות הכוחות והארכת החלקים והכוחות החיוניים כולם, להגלות בכל מלא מדתם }\מקור{[ע״ר א עג]}\צהגדרה{. }