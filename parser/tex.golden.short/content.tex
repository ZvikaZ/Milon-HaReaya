\mybookname{מילון הראיה}{מילון הראיה}
\mylettertitle{א}

\paragraphs

\ערך{אב }\הגדרה{- הנושא המוליד, המחולל את התולדות }\מקור{[ר״מ קיז]}\צהגדרה{. }

\paragraphs

\ערך{אב }\הגדרה{- המקים את הבית, המדריך את התולדות, המאיר את ארחות חייהם בהשפעתו הרוחנית }\מקור{[שם קיח]}\צהגדרה{. }

\paragraphs

\ערך{אב }\הגדרה{- הרועה הנאמן. מדריך, העומד במעלות נפשו הרבה יותר גבוה מהמעלה של הצעירות של הבנים\mycircle{°}, הצריכה לקבל את השפעתו\mycircle{°} }\מקור{[עפ״י ע״ר ב סה]}\צהגדרה{.}

\paragraphs

\ערך{אבנט}\הגדרה{ - מכוון בתור אמצעי, בין החלק העליון מקום הכחות הנפשיים, לבין החלק התחתון שבגוף, מקום הכחות הגופניים השפלים, שמורה אמנם על היחש החזק שיש לכחות השפלים אל הכחות הנפשיים, עד שהקדושה\mycircle{°} המעלה את הנטיות הנפשיות, פועלת להגביל יפה את סדרי הפעולות הטבעיות לצד המעלה והקדושה}\צהגדרה{ }\מקור{[ע״א ג ב ד]}\צהגדרה{.}

\הגדרה{ע״ע חגורה, יסוד הויתה.}

\paragraphs

\ערך{אבר }\הגדרה{- }\משנה{אבר הכנף\mycircle{°}}\הגדרה{ - הכח הפנימי המניע את העפיפה }\מקור{[עפ״י ע״א ב ח יב]}\צהגדרה{. }

\paragraphs

\משנה{אגרת רב שרירא גאון}\צהגדרה{ - מסמך\hebrewmakaf היסוד לסדר ההשתלשלות של כל התורה\hebrewmakaf שבעל\hebrewmakaf פה\mycircle{°}, שהוא כמגדל בנוי לתלפיות של היהדות, עליו תלוי אלף המגן כל שלטי הגבורים במלחמתה של תורה ואמתת דורותיה. בסיס האמונים לחתימת תורת אמת של ״חיי עולם הנטועים בתוכנו״\mycircle{°} מאז היותנו לעם ד׳ אלהינו, בקבלת מתנתה, ונמשכים לו באחרית חתימת התלמוד עם המשך דבריהם של הגאונים\mycircle{°} מוסרי עניניו }\צמקור{[ל״י ב (מהדורת בית אל תשס״ג) מט, נא].}

\הגדרה{ע״ע גאונים, תקופת הגאונים.}

\paragraphs

\ערך{אד }\הגדרה{- ענן }\מקור{[ר״מ קיח]}\צהגדרה{. }

\paragraphs

\ערך{אד }\הגדרה{- לישנא דתברא }\מקור{[ר״מ קיח]}\צהגדרה{. }

\paragraphs

\ערך{״אדם״ }\הגדרה{- כנוי לגויה. ציור האדם השפל והנבזה על שם האדמה אשר לוקח משם, להוראת היות חומרו שפל מאד, כי הוא המדרגה הפחותה מן הדצח״מ שהוא הדומם }\מקור{[עפ״י ע״א יבמות סג.]}\צהגדרה{.}

\ערך{״אדם״ }\הגדרה{- כנוי לנפש. ציור מדרגה גבוהה, כמו שכתוב ״בצלם\mycircle{°} אלקים עשה את האדם״, היינו מצד נשמתו\mycircle{°} הרוממה, אשר היא נאצלת מתחת כסא\hebrewmakaf הכבוד\mycircle{°}, ועל שם ״אדמה לעליון״}\myfootnote{ ישעיה יד יד.\label{1}}\הגדרה{ }\מקור{[עפ״י ע״א יבמות סג.]}\צהגדרה{.}

\הגדרה{ע״ע ״אנוש״. ע״ע גבר. ע״ע איש.}

\paragraphs

\ערך{אדם }\הגדרה{- נפש שכלית קשורה בחומר }\מקור{[ע״א ג ב קצט]}\צהגדרה{. }

\ערך{אדם }\הגדרה{- }\משנה{צורת\mycircle{°} האדם }\הגדרה{- המחשבה\hebrewmakaf העליונה\mycircle{°} העושה את האדם לאדם, התורה\mycircle{°} }\מקור{[ע״א ד ט יז]}\צהגדרה{. }

\משנה{צורת האדם הפנימית }\הגדרה{- שכלו ומוסרו }\מקור{[פנ׳ א]}\צהגדרה{. }

\ערך{אדם }\הגדרה{- }\משנה{כחו הרוחני }\הגדרה{- ע׳ במדור נפשיות, רוח, הכח הרוחני (של האדם).}

\ערך{אדם}\הגדרה{ - }\משנה{סגולת\mycircle{°} האדם}\myfootnote{ \textbf{כונס בקרבו את כל סגולת }\textbf{ההויה}\textbf{ וכו׳} - ש״ק קובץ א קעב: ״האדם הוא תמצית מלאה שההויה כולה משתקפת בו״.\label{2}}\הגדרה{ - מציינת את רוממותו הבאה בעקב שפלותו - יצור מושפל עד עמקי החומר, ועם זה כונס בקרבו את כל סגולת ההויה הרוחנית המלאה. שדוקא בהשתפלותו אל המורד הארצי הרי הוא רוכס את כל ההויה מראש היש עד סופו}\צהגדרה{ }\מקור{[ע״א ד ט קה]}\צהגדרה{.}

\ערך{אדם }\הגדרה{- }\משנה{נשמת האדם בכל חגויה השונים}\הגדרה{ - פרח רז עולם (של) החיבור הנעלה שממנו מתגלה הכבדות הארצית\mycircle{°} עם השאיפה השמימית המנצחתה, של שפעת החיים היציריים המשתפלת דרגה אחר דרגה, עד שיוצרת את החמריות\mycircle{°}, עם המאור\hebrewmakaf העליון\mycircle{°}, השפעה של הוית הישות, הרוחני, האצילי, השכלי, והמוסרי, הקדוש והמצוחצח }\מקור{[עפ״י א״ק ב תקכד]}\צהגדרה{.}

\הגדרה{יצירה שבה מתגלה האור ההויתי בכל עזו ותקפו. הכח המרכזי, שההויה חודרת באורה כולה אל הויתו, ומשלמת את תכונתה על ידו }\מקור{[פנק׳ ג של]}\צהגדרה{.}

\ערך{אדם }\הגדרה{- }\משנה{תעודת האדם שנוצר בגללה}\myfootnote{ \textbf{תעודת האדם} - ע״ע ע״ר א קפא ד״ה מכלל ופרט וכלל. א״ק ב תקלד. ע׳ במדור מונחי קבלה ונסתר, ״תוספת״.\label{3}}\הגדרה{ - להוסיף אור\mycircle{°} רצוני\mycircle{°} עליון\mycircle{°} בעזוז\mycircle{°} החיים הפרטיים, להעלותם\mycircle{°} אל עלוי\mycircle{°} הכלל\mycircle{°}, ולהוסיף בכלל זיו\mycircle{°} צביוני חדש ע״י עושר הבא ממשפלים. (לעסוק בתורה\hebrewmakaf לשמה\mycircle{°}) }\מקור{[א״ק א מד]}\צהגדרה{. }

\הגדרה{להשלים את מלכות\hebrewmakaf שמים\mycircle{°} }\צהגדרה{<שהיא מופיעה בכל היש בהדר גאונה, והולכת היא ומשתפלת בהעולמים המעשיים, בתהומות מאד עמוקים, בירידות מאד חשוכים>.}\הגדרה{ וברצונו הטוב והאיתן של האדם, שיצא אל הפועל בהיותו מתעלה להיות אוחז במשטר האלהי בהמון עולמים, }\צהגדרה{<שבפליאות נוראות נתגלה ע״י הבהקת אורם של אדירי הקודש שבדורות הקדמונים, ויצא במלא יקרתו בהתגלות האלהית שבאור התורה\mycircle{°} ונשמת\hebrewmakaf ישראל\mycircle{°}, הפרטית והכללית, בין כל עמי הארץ>. }\הגדרה{בכח חסון זה יתקן\mycircle{°} האדם ויעלה את החלק הירוד שבמלכות\mycircle{°} האצילות\hebrewmakaf האלהית\mycircle{°}, שירדה להיות מנהגת עולמי עד, בצורתם המוקצבה. ובזה יקשור נזר ועטרה למלכות שדי בכל העולמים כולם, והמגמה היצירתית תצא אל הפועל בכל יפעת אידיאליה, מראשית המחשבה עד סוף המעשה וכולה אומרת כבוד\mycircle{°} }\מקור{[עפ״י קובץ ח קעב]}\צהגדרה{.}

\משנה{מגמת היצירה האנושית }\הגדרה{- הנשמה החושבת, ההוגה דעה, המציירת\mycircle{°} ציורי קודש\mycircle{°} }\מקור{[א״ק ג שלד]}\צהגדרה{.}

\משנה{תעודת האדם }\הגדרה{- להיות משכיל ובן חורין, מתענג\hebrewmakaf על\hebrewmakaf ד׳\mycircle{°} ומתעלס בידיעת האמת ושמח בכבוד\mycircle{°} יוצר כל }\מקור{[עפ״י קבצ׳ א נח]}\צהגדרה{.}

\משנה{התפקיד האנושי }\הגדרה{- להיות איש חי מכיר ובעל השכלה }\מקור{[קובץ א קצה]}\צהגדרה{.}

\הגדרה{ע״ע חיי האדם. ע׳ במדור פסוקים ובטויי חז״ל, צלם אלהים, חותם צלם אלהים מוטבע באדם. ע״ע דמות האדם. ע׳ במדור אדם הראשון, תעודת האדם.}

\paragraphs

\ערך{אדנות מוחלטה }\הגדרה{- היכולת\mycircle{°} החפשית\mycircle{°} האין\hebrewmakaf סופית, המצויה תמיד בפועל בגבורה\hebrewmakaf של\hebrewmakaf מעלה\mycircle{°}, היא האדנות המוחלטה והמלוכה האמיתית שהיא עומדת למעלה מכל שם\mycircle{°}, מכל בטוי ומכל קריאה\mycircle{°}, שהרי האפשרות אין לה קץ ותכלית, והיכולת אין לה גבול והגדרה. מלכות\hebrewmakaf אין\hebrewmakaf סוף\mycircle{°} במובן העליון, המלוכה\hebrewmakaf העליונה\mycircle{°} }\מקור{[ע״ר א מו]}\צהגדרה{. }

\הגדרה{ע׳ במדור שמות כינויים ותארים אלהיים, ״אדון עולם״}\myfootnote{ ע׳ עטרת ראש להרד״ב, שער ראש השנה סי׳ ה.\label{4}}\הגדרה{. }

\paragraphs

\ערך{אדריכל }\הגדרה{- פועל (את) הבנין }\מקור{[א״ק ב שנ]}\צהגדרה{. }

\paragraphs

\משנה{אהבה }\הגדרה{- }\צהגדרה{ההתיחסות הנאמנה, הישרה, ההגונה, המתאימה אל האמת המציאותית, מתוך שייכות נכונה וזיקה רצויה, הכרה מלאה ושלמה של המציאות, של הענין שהיא מתייחסת אליו }\צמקור{[עפ״י ל״י ב רלד].}

\צהגדרה{מצב גדלותי, רוחני, אינטלקטואלי הכרתי נשמתי, שייכות חיונית, קישור התדבקות והזדהות, מתוך חכמה אמיתית והכרה אמיתית }\צמקור{[עפ״י שי׳ 63, 4\hebrewmakaf 5].}

\paragraphs

\ערך{אהבה }\הגדרה{- עדן החיים, התשוקה האלהית של העלאת נר החיים }\מקור{[מ״ר 24]}\צהגדרה{. }

\משנה{שלימות האהבה}\הגדרה{ - השמחה הגמורה ואור הנפש, שעמה כל טוב ואושר ובה כלולים נועם החכמה וההשגה ואהבתה }\מקור{[ע״א א ד לו]}\צהגדרה{.}

\מעוין{◊}\הגדרה{ האמונה\mycircle{°} }\צהגדרה{והאהבה}\הגדרה{ הן עצם החיים בעוה״ז ובעוה״ב\mycircle{°} }\מקור{[א׳ סט]}\צהגדרה{. }

\paragraphs

\ערך{אהבה }\הגדרה{- }\משנה{שמרי האהבה }\הגדרה{- ע״ע תאוות. }

\paragraphs

\ערך{אהבה }\הגדרה{- }\משנה{עבודת אהבה }\הגדרה{- זהירות בפרטי כל מצות ודקדוקי תורה מכח השפעת כללות התורה הדבקה בלב בחוזק והכרה ברורה }\מקור{[עפ״י א״ת ג ג]}\צהגדרה{.}

\משנה{עבודת ד׳ וכל מעגל טוב מאהבה }\הגדרה{- מידיעת הטוב\mycircle{°} הגנוז בהם }\מקור{[עפ״י ע״ר א רפו]}\צהגדרה{. }

\הגדרה{מהכרה אמיתית אל הטוב והשלימות }\מקור{[ע״ר א שסז\hebrewmakaf ח (ע״א א ג לב)]}\צהגדרה{.}

\משנה{באהבה}\הגדרה{ - בדרך חפץ פנימי והכרה עצמית }\מקור{[ל״ה 55]}\צהגדרה{. }

\משנה{כח העבודה מאהבה}\הגדרה{ - }\מעוין{◊ }\הגדרה{אינו בא כי אם לפי מדת הידיעה הבאה בלימוד של קביעות ועשירות רבה במקצעות השונים של תורת המוסר\mycircle{°} והיראה\mycircle{°}, שאי אפשר כלל להמצא מבלעדי לימוד בסדר נכון, למגרס תחילה בבקיאות מלמטה למעלה, ואחר כך למסבר בעומק עיון ודעה שלמה }\מקור{[ל״ה 188]}\צהגדרה{.}

\הגדרה{ע״ע עבודה מאהבה, עבודת ד׳ מאהבה ותלמוד תורה\hebrewmakaf לשמה.}

\paragraphs

\ערך{אהבה אלהית }\הגדרה{- }\משנה{האהבה האלהית העליונה, המבוסמת בבשמי הדעה העליונה }\הגדרה{- ההרגשה הנשמתית היותר חודרת ופנימית, אשר בכנסת\hebrewmakaf ישראל\mycircle{°} בכללותה, בנשמות אישיה היחידים, בחביון\hebrewmakaf עז\mycircle{°} נשמת כללותה, ובכל אשד הרוח המשתפך בכל פלגות תולדותיה }\מקור{[ע״א ד ט פח]}\צהגדרה{. }

\הגדרה{אהבת\hebrewmakaf ד׳\mycircle{°} אלהי\hebrewmakaf ישראל\mycircle{°}, עצם החיים (בישראל), נשמת\hebrewmakaf האומה\mycircle{°} ועצם חייה }\מקור{[אג׳ א מד]}\צהגדרה{.}

\משנה{אהבה אלהית }\הגדרה{- הנטיה היותר חפשית\mycircle{°} ונצחית\mycircle{°} של רוח החיים, שהופעתה באה מסקירת הגודל הבלתי מוקצב, של אור\mycircle{°} הקודש\mycircle{°} המקיף עולמי נצח ממעל לכל חק וקצב, שאור החסד\mycircle{°} הנאמן\mycircle{°} מתעלה שם, השופע ויורד בכל מלא חנו\mycircle{°}, ממעל לכל חק ומשפט\mycircle{°}, וכל פנות שהוא פונה הכל הוא רק לטובה\mycircle{°} ולברכה\mycircle{°} לאור ולחיים\mycircle{°}, וכל מעשה וכל תנועה מחוללת אך נועם\mycircle{°} והוד\mycircle{°} קודש }\מקור{[עפ״י א״י כט, ע״ר א יד]}\צהגדרה{.}

\משנה{האהבה העליונה }\צהגדרה{- אהבת\hebrewmakaf עולם\mycircle{°} ואהבה\hebrewmakaf רבה\mycircle{°}, אשר לישראל את ד׳ אלהיהם ואביהם\hebrewmakaf שבשמים\mycircle{°} מלך\hebrewmakaf עולמים, הבוחר בעמו ומלמדו ומדריכו }\צמקור{[ל״י א (מהדורת בית אל תשס״ב) צג]. }

\ערך{האהבה}\myfootnote{ \textbf{ההכרה האמיתית וכו׳ }\textbf{מכבוד\hebrewmakaf אל}\textbf{, הנשקף מהבריאה וכו׳ וכו׳ שומע קול ד׳ הקורא אליו וכו׳ ומרגיש שהוא וכו׳ שואף את חייו יחד עם }\textbf{מקור\hebrewmakaf החיים}\textbf{, וכל היצור כולו ניצב לו כאורגן שלם אדיר נחמד ואהוב, שהוא אחד מאבריו, המקבל מכולו ונותן לכולו, ויונק יחד עמו זיו חייו ממקור החיים} - ע׳ במדור שמות כינויים ותארים אלהיים, ״מלכנו״. ושם, ״אבינו״. ע״ע ע״ר א רמט, ד״ה ברוך. ושם, רפט ד״ה ברכנו. ושם ב ג ד״ה אמר ר׳ עקיבא. קבצ׳ ב קלז [87]. פנק׳ ב רד מט. ע״ע ״שמע״. ע׳ במדור פסוקים ובטויי חז״ל, ברוך שם כבוד מלכותו. (את ההבחנה בעניין האיר לי אהרן משה שיין).\label{5}}\הגדרה{ - }\צהגדרה{תכלית התעודה האנושית}\הגדרה{. ההכרה האמיתית כשמתגברת באדם כראוי, מכבוד\hebrewmakaf אל\mycircle{°} הכללי, הנשקף מכל הדר\mycircle{°} הבריאה וסדריה הגשמיים\mycircle{°} והרוחניים\mycircle{°}, בעבר, בהוה ובעתיד, }\צהגדרה{<שגם זה האחרון מוצץ הוא יפה למי שמבקש ודורש\hebrewmakaf את\hebrewmakaf אלהים\mycircle{°} באמת וחפץ שלם>}\הגדרה{ אותה ההכרה כשהיא מתעצמת יפה באדם, רק היא מטבעת עליו את חותמו האמיתי, את אופיו הטבעי להקרא בשם אדם\mycircle{°}. }\צהגדרה{רק אז הוא מרגיש שהוא חי חיים נצחיים ומכובדים. <הוא מכיר כי הדרכים שהחיים מתגלים בהם, לפי ערכנו ביחש מצבנו החומרי, שונים המה, ובכל השינויים ההווים והעתידים לבבו בוטח\hebrewmakaf בשם\hebrewmakaf ד׳\mycircle{°} אלהי עולם מחיה החיים וחי\hebrewmakaf העולמים\mycircle{°}> מצב נפש כזה כשהוא מתאים גם כן לכל סדרי החיים הפנימיים, הנפשיים והגופניים, חיי המשפחה והחברה, וכשהוא צועד בעוזו להיות גם כן מתפלש להיות המוסר הציבורי עומד על תילו ומכונו, אז הארץ מוכרחת להתמלא דעה, ותורת ד׳ היא נובעת ממעמקי הלב - כל }\הגדרה{אדם שומע קול ד׳ הקורא אליו ושש ושמח לעשות רצון קונו וחפץ צורו, שהוא צורו הפרטי וצור העולמים כולם; ומרגיש הוא אז, שהוא האדם, שואף את חייו יחד עם מקור\hebrewmakaf החיים\mycircle{°}, וכל היצור כולו ניצב לו כאורגן שלם אדיר נחמד ואהוב, שהוא אחד מאבריו, המקבל מכולו ונותן לכולו, ויונק יחד עמו זיו חייו ממקור החיים }\מקור{[עפ״י ל״ה 149]}\צהגדרה{.}

\משנה{מתק האהבה }\הגדרה{- רוחב הדעת\mycircle{°}, והנועם אשר לעדן\hebrewmakaf העליון\mycircle{°} }\מקור{[א״ק ג ראש דבר כט]}\צהגדרה{. }

\משנה{אהבת צור\hebrewmakaf העולמים\mycircle{°}}\הגדרה{ - זיו השכינה\mycircle{°}, הכרה שכלית והרגשית, ללכת בדרכי\hebrewmakaf ד׳\mycircle{°} באהבת אמת והכרה עמוקה פנימית\mycircle{°} }\מקור{[עפ״י ע״א ג ב נ]}\צהגדרה{. }

\משנה{זיקי אהבת אלהים }\הגדרה{- מציאת אור\hebrewmakaf ד׳\mycircle{°} בעומק רגש, בתוכן דעה }\מקור{[א״ק ג ריא]}\צהגדרה{. }

\הגדרה{ע״ע אהבת ד׳. ע״ע אהבת ד׳ העליונה. ע״ע יראת הגודל. }

\paragraphs

\ערך{אהבה לעומת טוב}\הגדרה{ - ע׳ בנספחות, מדור מחקרים.}

\paragraphs

\ערך{אהבה מינית }\הגדרה{- ע׳ במדור הנטייה המינית.}

\paragraphs

\ערך{אהבה קדושה }\הגדרה{- }\משנה{האהבה הקדושה}\הגדרה{ - אהבת\hebrewmakaf ד׳\mycircle{°} וכל העולמים, אהבת כל היקום וכל היצור }\מקור{[א״ק ג רעט\hebrewmakaf רפ]}\צהגדרה{.}

\paragraphs

\ערך{״אהבה רבה״ }\הגדרה{- ע׳ במדור פסוקים ובטויי חז״ל.}

\paragraphs

\ערך{אהבת אור ד׳}\הגדרה{ - אהבת החיים של הצדיק\hebrewmakaf האמיתי\mycircle{°}, <שאיננה כלל אותה הנטיה הגסה של אהבת החיים המרופדת בשכרון של נטיות החומר הגסים המצוי אצל רוב הבריות, כי אם> אהבת חיקוי לחסד\hebrewmakaf עליון\mycircle{°} בעולמו, המתפשטת על פני כל היצור }\מקור{[קבצ׳ א קעד]}\צהגדרה{.}

\paragraphs

\ערך{אהבת ד׳ }\הגדרה{- הרגשת השתוקקות תמיד לטוב\mycircle{°} ולאמת\mycircle{°} שהאדם מרגיש באמת בנקודת נשמתו\mycircle{°} הפנימית, <שהכל הוא בכלל טוב או בכלל אמת> }\מקור{[עפ״י קבצ׳ ב קלג (פנק׳ ד שסב)]}\צהגדרה{.}

\הגדרה{השגת המושכלות המופשטים שבענינים האלהיים ואהבה אמיתית את הטוב\mycircle{°} את היושר\mycircle{°}, והרדיפה ללכת בדרכי\hebrewmakaf ד׳\mycircle{°} }\מקור{[עפ״י פנק׳ א תקסג]}\צהגדרה{.}

\הגדרה{(האהבה) מצד כבודו\mycircle{°} וחסדיו\mycircle{°} שעשה }\מקור{[מא״ה א פט]}\צהגדרה{. }

\צהגדרה{גלויה היסודי של האמונה\mycircle{°} הגדולה }\צמקור{[נ״ה יא].}

\מעוין{◊ }\משנה{אהבת ד׳}\הגדרה{ באה כשישים האדם לבבו להדמות לדרכי\mycircle{°} השי״ת\mycircle{°}, אז ע״י ההדמות תולד האהבה, וכפי רוב הדמיון יהי׳ רוב האהבה }\מקור{[ע״א א ב לו (ע״ר ב קכג)]}\צהגדרה{. }

\מעוין{◊ }\משנה{האהבה}\הגדרה{ באה מצד השלמות שבנמצאים, שמצדה הם כולם נמצאים באמיתת מציאותו יתברך }\מקור{[ע״א ג ב קעא]}\צהגדרה{.}

\משנה{אהבת ד׳ }\הגדרה{- דעת\hebrewmakaf ד׳\mycircle{°} <שבכללה היא גם כן דעת כל המציאות לאמתתה לכל סעיפיה, כפי היכולת לאדם: דעת הטבע לכל סעיפיו גיאוגרפיה והתכונה, הרפואה וחכמת הנפש, תכונות העמים וכל הנלוה להם, המביאים גם כן לאהבה\hebrewmakaf העליונה\mycircle{°} הזכה בכללות האנושיות> }\מקור{[עפ״י קבצ׳ ב קלא]}\צהגדרה{.}

\הגדרה{ע״ע אהבה אלהית. ע״ע יראת ד׳. ע״ע בנספחות, מדור מחקרים, אהבה ויראה. }

\paragraphs

\ערך{אהבת ד׳ העליונה }\הגדרה{- אהבת השלמות המוחלטת והגמורה של סיבת\mycircle{°} הכל, מחולל כל ומחיה את כל }\מקור{[א״ק ב תמב]}\צהגדרה{. }

\משנה{אהבת ד׳ הבהירה }\הגדרה{- האהבה המרוממה והעדינה לאין\hebrewmakaf סוף\mycircle{°} }\מקור{[קובץ ה צה]}\צהגדרה{. }

\משנה{אהבת השי״ת}\myfootnote{ \textbf{בהכרת האמת של המציאות }\textbf{האלהית}\textbf{ מצד עצמה, המקור }\textbf{לשמו\hebrewmakaf הגדול}  - ע׳ א״ק ד ת.\label{6}}\הגדרה{ - }\מעוין{◊}\הגדרה{ בהכרת האמת של המציאות האלהית מצד עצמה, המקור לשמו\hebrewmakaf הגדול\mycircle{°} ב״ה }\מקור{[קבצ׳ א קלז]}\צהגדרה{. }

\משנה{אור אהבת ד׳ }\הגדרה{- עדן\mycircle{°} החיים, מגמת החיים, עצם החיים, בהירות החיים, ומעין חיי החיים, עליון מכל הגה, מכל רצון והסברה, מכל שאיפה פנימית\mycircle{°}, ומכל הזרחה\mycircle{°} יפעתית\mycircle{°}, הכל בה, והכל ממנה }\מקור{[קובץ ו רמא]}\צהגדרה{.  }

\הגדרה{ע׳ במדור פסוקים ובטויי חז״ל, אהבה רבה. ע״ע אהבת שם ד׳. }

\paragraphs

\משנה{״אהבת חנם״}\myfootnote{ ע׳ א״ק ג שכד.\label{7}}\הגדרה{ }\צהגדרה{- אהבה שגם כשיש במציאות דברים שכאילו מעכבים לה אעפ״כ תתגבר על כולם ותקבע\hebrewmakaf חנם }\צמקור{[ל״י א קיג]. }

\צהגדרה{אהבה שאינה תלויה בדבר <כאהבת ד׳ לישראל, ברית עולם> }\צמקור{[ק״ת נה].}

\paragraphs

\ערך{״אהבת חסד״ }\הגדרה{- ע׳ במדור פסוקים ובטויי חז״ל.}

\paragraphs

\ערך{״אהבת חסד״}\הגדרה{ - }\משנה{(לעומת ״תורת חיים״)}\הגדרה{ - ע׳ במדור פסוקים ובטויי חז״ל.}

\paragraphs

\ערך{״אהבת עולם״ }\הגדרה{- ע׳ במדור פסוקים ובטויי חז״ל.}

\paragraphs

\ערך{אהבת שם ד׳ }\הגדרה{- אהבת הלימוד והידיעה של מציאות השי״ת ודרכיו, וכל המכשירים המביאים לזה }\מקור{[קבצ׳ א קלו]}\צהגדרה{. }

\הגדרה{ע״ע אהבת ד׳. ע״ע אהבת ד׳ העליונה.}

\paragraphs

\ערך{אהבת תורה }\הגדרה{- ע׳ במדור תורה.}

\paragraphs

\ערך{אוביקטיבי }\הגדרה{- חיצוני}\myfootnote{ ע׳ בנספחות, מדור מחקרים, אוביקטיבי סוביקטיבי. ושם, חיצון, עולם חיצוני.\label{8}}\הגדרה{ }\מקור{[עפ״י א״ק ג צג]}\צהגדרה{.}

\הגדרה{ע׳ במדור הכרה והשכלה והפכן, סוביקטיבי. ע״ע חצון, עולם חיצון.}

\paragraphs

\ערך{אוהל }\הגדרה{- שם בית הדירה, העלול להיות מוכן למסעות, המרשם בתוכן הרוחני העליון (של האדם) את העליות הנכספות. האוהל מסמן את היסוד המטלטל, את הצביון של ההכנה אשר לתנועה, שכונתה היא תמיד השתנות ועליה לצד האושר\hebrewmakaf העליון\mycircle{°}, לקראת הזיו\mycircle{°} של מעלה}\צהגדרה{ }\מקור{[ע״ר א מג]}\צהגדרה{.}

\הגדרה{ע״ע משכן.}

\paragraphs

\ערך{״אוהל״ לעומת ״בית״ }\הגדרה{- ע׳ במדור מדרגות והערכות אישיותיות, ״יושב בבית״ לעומת ״יושב אוהל״.}

\paragraphs

\ערך{אויב }\הגדרה{- מי שהשנאה (אצלו) בכח לא בפועל }\מקור{[מ״ש שכז]}\צהגדרה{.}

\הגדרה{מבקש רעה בציורו ונטיתו הרוחנית }\מקור{[ע״ר א לד]}\צהגדרה{.}

\הגדרה{ע״ע קם להרע.}

\paragraphs

\ערך{אולפן }\הגדרה{- לימוד <בתרגום> }\מקור{[ר״מ ב]}\צהגדרה{. }

\paragraphs

\ערך{אומה }\הגדרה{- }\משנה{טבע האומה, הרוחני והחומרי }\הגדרה{- הטבע הפסיכולוגי של האומה, וטבע התולדה והמורשה של האבות והגזע, וטבע הגיאוגרפי של ארץ נחלתה }\מקור{[עפ״י קבצ׳ ב מה (ב״ר שכו\hebrewmakaf ז)]}\צהגדרה{. }

\ערך{אומה }\הגדרה{- }\משנה{צורת\mycircle{°} האומה }\הגדרה{- נשמתה ואורח חייה }\מקור{[עפ״י ע״א ד ה סא]}\צהגדרה{. }

\הגדרה{ע״ע עמים, הצד המהותי בחיי העמים. ע׳ במדור פסוקים ובטויי חז״ל, שבעים אומות.}

\paragraphs

\ערך{אומה }\הגדרה{- }\משנה{האומה (הישראלית) כולה בצרופה הכללי }\הגדרה{- אֵם החיים שלנו. האופן הכללי\mycircle{°} של כל ישראל\mycircle{°} בתור גוש אחד, המחבר את כל האישים הפרטיים להיות לעם\mycircle{°} אחד, הכולל ג״כ את כל הדורות כולם בהערכה אחת }\מקור{[עפ״י א׳ עו, ע״ר ב פד]}\צהגדרה{. }

\הגדרה{ע״ע עם. ע״ע גוי. }

\paragraphs

\ערך{אומה }\הגדרה{- }\משנה{רוח האומה היחידי}\myfootnote{ אולי צ״ל: יחודי. ע׳ בנספחות, מחקרים, כללים להבנה נכונה בקריאת כתבי הרב, יחידי.\label{9}}\הגדרה{ - השאיפה אל הטוב האלהי המונח בטבע נשמתה }\מקור{[א׳ נב]}\צהגדרה{.}

\הגדרה{ע״ע רוח ישראל. ע״ע רוח ד׳. ע׳ במדור תורה, תורה שבכתב, תורה שבכתב ברום תפארתה ותורה שבעל פה שניהם יחד.}

\paragraphs

\ערך{אומה }\הגדרה{- }\משנה{שכינת האומה }\הגדרה{- רוח החיים של השאיפה האלהית המקושרת בתוכן הסגנון הצבורי של הצורה הלאומית }\מקור{[א׳ קו]}\צהגדרה{. }

\הגדרה{ע׳ במדור מונחי קבלה ונסתר, ״שושנה עליונה״. ע״ע אידיאה לאומית.}

\paragraphs

\ערך{אומה }\הגדרה{- }\משנה{ישראל}\הגדרה{ - כנסת\hebrewmakaf ישראל\mycircle{°} המוגבלה בגבול נחלת ישראל }\מקור{[א׳ מב]}\צהגדרה{.}

\הגדרה{מקום מנוחתה של האידיאה\hebrewmakaf האלהית\mycircle{°} על המרחב ההיסתורי הכללי }\מקור{[א׳ קח]}\צהגדרה{.}

\ערך{אומה הישראלית}\הגדרה{ - }\משנה{התכלית הכללית של האומה הישראלית}\הגדרה{ - להודיע את שם\hebrewmakaf ד׳\mycircle{°} בעולם כולו ע״י מציאותה והנהגתה }\מקור{[ל״ה 119 (פנק׳ ב עו)]}\צהגדרה{.}

\הגדרה{חטיבה\mycircle{°} אחת בעולם, המצויינת בתקותה לעצמה לא בשביל עצמה, כ״א בשביל הטוב הכללי, שהוא חן השכל הטוב, המוסר\mycircle{°} והיושר\mycircle{°} האמיתי, שא״א שיבנה כ״א ע״י תיקון\hebrewmakaf עולם\hebrewmakaf במלכות\hebrewmakaf שדי\mycircle{°}}\צהגדרה{ }\מקור{[ע״ר א שפו (ע״א ב ט רצ)]}\צהגדרה{.}

\הגדרה{ע״ע ישראל, מהותם העצמית הנותנת להם את אופים המיוחד.}

\paragraphs

\ערך{אומה כללית }\הגדרה{- תמצית\mycircle{°} של המין האנושי הפועלת עליו בלי הרף בעיבוד צורתו\mycircle{°} הרוחנית\mycircle{°} }\מקור{[קובץ ה קצו]}\צהגדרה{.}

\הגדרה{ע׳ במדור פסוקים ובטויי חז״ל, עם לבדד.}

\paragraphs

\ערך{אוצר החיים }\הגדרה{- אורה\hebrewmakaf של\hebrewmakaf תורה\mycircle{°} במקורה }\מקור{[ע״ר א קמז]}\צהגדרה{. }

\הגדרה{הצד העליון של התורה, היקר בעצמו מכל החיים כולם, }\צהגדרה{אוצר חיים}\הגדרה{ עליונים נעלים ונשאים מכל חיי זמן ועולם}\צהגדרה{ }\מקור{[ע״א ד ט ז]}\צהגדרה{.}

\הגדרה{ע׳ במדור מונחי קבלה ונסתר, ״אורייתא מבינה נפקת״. ע׳ במדור תורה, תורה, שורש התורה. }

\paragraphs

\ערך{אוצר הטוב }\הגדרה{- מקור חי\hebrewmakaf העולמים\mycircle{°} }\מקור{[אג׳ א קי]}\צהגדרה{.}

\paragraphs

\ערך{אוצר עליון}\הגדרה{ - }\משנה{האוצר העליון}\הגדרה{ - מקור הברכות\mycircle{°}}\צהגדרה{ }\מקור{[א״ק א קיט]}\צהגדרה{.}

\paragraphs

\ערך{אור }\הגדרה{- יסוד ואומץ המשכת החיים }\מקור{[עפ״י א״ק ב רצז (ע״ט טז)]}\צהגדרה{. }

\הגדרה{כל יסוד החיים, חיי החיים, זיום\mycircle{°} ותפארתם\mycircle{°} }\מקור{[עפ״י ע״א ד יא יג]}\צהגדרה{. }

\הגדרה{כח הרוחניות\mycircle{°} של השכל הגדול, של החפץ הכביר, של המרץ הנשגב\mycircle{°} }\מקור{[מ״ר 296 (קבצ׳ ב עא)]}\צהגדרה{. }

\משנה{האור הגדול הכללי}\צהגדרה{ - שלמות החיים ובריאותם הנמשכת ממקור אמתתם, המתגלה על ידי כל פרטיותם של דברי התורה, טיפולם וקליטתם, במלא כל הנפש ובכל תפוצות חדריה }\צמקור{[א״ל מג].}

\הגדרה{ע״ע אור החיים. ע׳ במדור מונחי קבלה ונסתר, אורות. }

\ערך{אור }\הגדרה{- }\משנה{האור בעצם }\הגדרה{- אור\hebrewmakaf חדש\mycircle{°} של תשובה\hebrewmakaf עליונה\mycircle{°}, המ״ט שערי\mycircle{°} בינה\mycircle{°} [}\צהגדרה{ח״פ לב:].}

\צהגדרה{גילוי אמיתת המציאות המשוכללת בהופעת\mycircle{°} הקרנת הזרחתה\mycircle{°} }\צמקור{[עפ״י פנק׳ א תרלו (ב״א ד יא)]. }

\הגדרה{ע׳ בנספחות, מדור מחקרים, אור, זוהר, זיו. ושם, אור, זיו, ברק. }

\paragraphs

\ערך{אור }\הגדרה{- }\צמשנה{האור הנשגב }\הגדרה{- החיים המלאים הלאומיים הממולאים מטל חיים ממלכתיים אלהיים ממשיים וחזיוניים, העומדים אחר כותלנו }\מקור{[פנק׳ ד ריז-ח]}\צהגדרה{.}

\paragraphs

\ערך{אור }\הגדרה{- }\משנה{(לעומת כלי\mycircle{°}) }\הגדרה{- נשמתו הרוחנית של הכלי <שהוא לבושו המעשי החיצון> }\מקור{[עפ״י א׳ קנח]}\צהגדרה{.}

\הגדרה{התוכן (לעומת הסגנון)}\צהגדרה{ }\מקור{[ע״א ד יב ה]}\צהגדרה{.}

\הגדרה{החיים העצמיים של מחשבת ההויה (לעומת ההויה) }\מקור{[עפ״י ע״ר א כו, וא״ק ד ת (א״ה 1098)]}\צהגדרה{. }

\הגדרה{אצילות האלהות בתור נפש ההויה, (מבחינתה הפנימית), בדיבורים מצד הסתכלות השירית שברוח הקודש }\מקור{[עפ״י א״ק ב שמח]}\צהגדרה{. }

\paragraphs

\משנה{אור וכלים }\צהגדרה{- משמעות עמוקה, תוכן רוחני\mycircle{°}. }\צהגדרהמודגשת{כלים }\צהגדרה{- הגילויים של האור }\צמקור{[פנק׳ א תרלז (שי׳ 6, 25)]. }

\paragraphs

\ערך{אור }\הגדרה{- }\משנה{(לעומת חיים\mycircle{°}) }\הגדרה{- דעה\mycircle{°}, רוח\hebrewmakaf הבטה }\מקור{[עפ״י א׳ יא]}\צהגדרה{. }

\paragraphs

\ערך{׳אור׳ לעומת ׳מאור׳ }\הגדרה{- ע׳ בנספחות, מדור מחקרים. }

\paragraphs

\ערך{אור}\הגדרה{ - כללות הרגשה וידיעת מציאות. ערך ההשגה\mycircle{°} והרצון הגמור }\צהגדרה{<כי מה שלמעלה מההשגה האנושית אין לקרות כ״א בשם חושך\mycircle{°} מצד ההעלם, וכשיש העלם לחושך של מעלה, נגבל בגדר השגה ונעשה }\צהגדרהמודגשת{אור}\צהגדרה{> }\מקור{[עפ״י מא״ה ד כא-כב]}\צהגדרה{.}

\paragraphs

\ערך{אור }\הגדרה{- ההרגשה הנפשית וההבנה של הידיעה }\מקור{[ע״א ב ט קכד]}\צהגדרה{.}

\paragraphs

\ערך{אור }\הגדרה{- }\משנה{אוצר האור }\הגדרה{- חיי החיים העליונים, מקור כל החיים ושרש כל ההויות }\מקור{[ע״ר א סז]}\צהגדרה{. }

\ערך{אור }\הגדרה{- }\משנה{האור הפנימי (של המושג מאורו של אלקים\hebrewmakaf חיים, צור ישעינו, לגדולי המשיגים) }\הגדרה{- החיים האמיתיים שאין לנו מלה ליחסם, כמו שהם נמצאים במקור\hebrewmakaf החיים\mycircle{°} יתברך שמו }\מקור{[עפ״י ע״א ג ב נב]}\צהגדרה{. }

\paragraphs

\ערך{אור אין סוף }\הגדרה{- ע׳ במדור מונחי קבלה ונסתר. או במדור שמות כינויים ותארים אלהיים.}

\paragraphs

\ערך{אור אין סופי}\הגדרה{ - }\משנה{האור האין סופי}\הגדרה{ - ההארה האלהית\mycircle{°} המקיפה והממלאה את כל, את כל הנשמות\mycircle{°} ואת כל העולמים\mycircle{°} }\מקור{[פנק׳ א שצט]}\צהגדרה{.}

\paragraphs

\ערך{אור ״אל עליון קונה שמים וארץ״}\myfootnote{ בראשית יד יט.\label{10}}\הגדרה{ - החפץ\hebrewmakaf האלהי\mycircle{°}, המהוה את היש כולו, המעמידו ומחייהו, הדוחפו לעילוייו בכל קומתו המעשיית והרוחנית מריש דרגין עד סופם. הנבואה\hebrewmakaf העליונה\mycircle{°} של פה אל פה אדבר בו, הנשפעת לנאמן\mycircle{°} בית\mycircle{°}, להקים עדות ביעקב לעולמי עולמים, לקומם תבל ומלאה, בנשמת ד׳ יוצר כל }\מקור{[עפ״י ע״א ד ט טז]}\צהגדרה{. }

\paragraphs

\ערך{אור אלהי }\הגדרה{- אור\mycircle{°} האמת\mycircle{°} הצדק\mycircle{°} והדעת\mycircle{°} }\מקור{[ע״ה קכח]}\צהגדרה{. }

\הגדרה{זוהר\mycircle{°} גדול של שכל בהיר וחשק אדיר של רצון כביר מאד }\מקור{[א״ק ג רטז]}\צהגדרה{. }

\paragraphs

\ערך{אור אלהי }\הגדרה{- }\משנה{האור האלהי }\הגדרה{- המגמה השעשועית\mycircle{°} הפנימית\mycircle{°} של היצירה כולה, המזריחה\mycircle{°} באור\mycircle{°} יפעתה\mycircle{°} על פני כל היקום, מחייה הפנימיים }\מקור{[א״ק ג קפח]}\צהגדרה{. }

\משנה{נועם אור אלוה נורא הוד }\הגדרה{- מקור הנעימות ומעין העדנים\mycircle{°}, אוצר ההופעות\mycircle{°} ומקור מקורות החיים }\מקור{[ר״מ עו]}\צהגדרה{. }

\משנה{האור האלהי }\הגדרה{-  מקור\hebrewmakaf החיים\mycircle{°} ומקור כל העדן ורוממות כל אושר\hebrewmakaf עליון\mycircle{°}. אור חיי\hebrewmakaf החיים\mycircle{°} }\מקור{[עפ״י קבצ׳ א רטז (פנק׳ א תקיח, ג״ר 124)]}\צהגדרה{. }

\הגדרה{חיי החיים }\מקור{[ע״א ג ב רכו]}\צהגדרה{.}

\הגדרה{מקור החיים והשמחה\mycircle{°} }\מקור{[ע״א ג ב צט]}\צהגדרה{.}

\משנה{מקור האור האלהי}\הגדרה{ - נחל עדנים שאין לו סוף, ומקור עדן נצחי לכל נשמת חיים, המהפך את הכל לאור\hebrewmakaf חיים\mycircle{°}. המאור הפנימי, הכח הכמוס האלהי שיש במגמת הוייתה של האומה בעולם, שהוא הסוד של כל ההויה כולה }\מקור{[עפ״י קבצ׳ א קעה]}\צהגדרה{.}

\משנה{אור אלהי עליון }\הגדרה{- המרחב של אין סוף לבהירות והשלמת חיי עולמים בעד הכל }\מקור{[ע״א ד ט סה]}\צהגדרה{. }

\הגדרה{ע״ע אור עליון. ע״ע אור ד׳.}\myfootnote{ \textbf{אור }\textbf{אלהי}\textbf{, אור }\textbf{אלהים}\textbf{, אור ד׳, אור עליון }- בין מושגים אלה התקשתי למצוא הבדל, מכל מקום חולקו ההגדרות למחלקות שונות על פי המונחים השונים.\label{11}}\הגדרה{ ע׳ במדור שמות כינויים ותארים אלהיים, אלהי, המקור האלהי. }

\paragraphs

\ערך{אור אלהי }\הגדרה{- }\משנה{האור האלהי }\הגדרה{- נשמת\hebrewmakaf האומה\mycircle{°} השרשית }\מקור{[א׳ קנח]}\צהגדרה{. }

\הגדרה{הזיו\mycircle{°} הטהור\mycircle{°} הממלא נפשות טהורות }\מקור{[ע״א ג ב נ]}\צהגדרה{. }

\paragraphs

\ערך{אור\mycircle{°} אלהים\mycircle{°} }\הגדרה{- תעודת ההויה, מקור הנשמות\mycircle{°}, מלא\hebrewmakaf כל\mycircle{°}, רוח ישראל\mycircle{°} המופשט }\מקור{[עפ״י א״ת יב א]}\צהגדרה{. }

\הגדרה{אור החיים היותר יפים, היותר טהורים\mycircle{°} היותר מאירים\mycircle{°} }\מקור{[ע״א ד ו מ]}\צהגדרה{. }

\paragraphs

\ערך{אור ד׳\mycircle{°}}\הגדרה{ - העילוי\mycircle{°} העליון\mycircle{°}, שממעל למקור\hebrewmakaf החיים\mycircle{°}, יסוד המרחב העליון של הזוהר\mycircle{°} הבלתי מוגבל שכל עולמי\hebrewmakaf עולמים\mycircle{°} אינם כדאיים לו, שהוא מובדל מכל אורות עולמים, שכל תכונה של אורה בהם הרי היא מכוונת לראות על ידה גופים חשכים, שבעצמם אינם מערך מהות האורה, אבל האור בעצמו איננו דבר נראה, כי לא נתגלה בעולם לפי מדתו הכח הרואה את מהות האור. אמנם }\משנה{אור ד׳}\הגדרה{ במעלת הרחבת אצילות\mycircle{°} מקורו, הוא האור שאור נראה בו ועל ידו }\מקור{[עפ״י ע״ר א כא]}\צהגדרה{. }

\הגדרה{אור האורים, שאי\hebrewmakaf אפשר לנו לבטאו ואיננו יכול להתלבש באותיות של שום מבטא גם לא של שום רעיון }\מקור{[א׳ קלא]}\צהגדרה{. }

\הגדרה{מקור חיי\hebrewmakaf החיים\mycircle{°} ב״ה }\מקור{[ע״ר א קנה]}\צהגדרה{. }

\הגדרה{חיי\hebrewmakaf החיים\mycircle{°}, היסוד העליון מקור חיי אור העולמים\mycircle{°} }\מקור{[עפ״י א״ק ג צה]}\צהגדרה{. }

\הגדרה{יסוד כל היש, ויותר מכל היש באין קץ }\מקור{[קובץ א תתיא]}\צהגדרה{.}

\הגדרה{צרור\hebrewmakaf החיים\mycircle{°} }\מקור{[שם רמ]}\צהגדרה{. }

\הגדרה{אור האמת\mycircle{°} }\מקור{[עפ״י ע״א ב ט קנא]}\צהגדרה{. }

\הגדרה{אלהי עולם. הטוהר\mycircle{°}, הטוב\hebrewmakaf המוחלט\mycircle{°}, האמת המזהרת, הנצח\mycircle{°} בכל מלא הודו\mycircle{°} }\מקור{[עפ״י א״ק א קפב]}\צהגדרה{.}

\משנה{אור ד׳ מחולל כל }\הגדרה{- זוהר האמת, הוד\mycircle{°} אור\hebrewmakaf החיים\mycircle{°}, שבמקור\hebrewmakaf הקודש\mycircle{°} }\מקור{[עפ״י א״ק א ג (מ״ר 402)]}\צהגדרה{. }

\משנה{אור ד׳ וכבודו}\הגדרה{\mycircle{°} - הקודש\hebrewmakaf העליון }\מקור{[מ״ר 345]}\צהגדרה{. }

\משנה{אור ד׳ העליון }\הגדרה{- כולל הכל, ומקור הכל וחיי כל }\מקור{[ע״ר א רח]}\צהגדרה{. }

\הגדרה{ע״ע אור עליון. ע״ע אור אלהי.}\footref{11}

\ערך{אור ד׳ ממרומיו\mycircle{°}}\הגדרה{ - היש\hebrewmakaf העליון\mycircle{°}, הרוחניות\mycircle{°} והטוהר\mycircle{°} המעולה }\מקור{[עפ״י א״ק ג רפו]}\צהגדרה{. }

\משנה{אור ד׳ }\הגדרה{- אור פני המלך המתנשא לכל לראש מעל כל ענין העולמות }\מקור{[ע״ר א רפט]}\צהגדרה{.}

\הגדרה{הארת\mycircle{°} היש האמיתי וזיו\mycircle{°} החיים האלהיים }\מקור{[ע״א ד ט מז]}\צהגדרה{. }

\הגדרה{הקדושה השרשית העצמית, המצואה בפועל, הוד\mycircle{°} חיי הקודש\mycircle{°}, המרומם ונשא מכל שרעף ורעיון }\מקור{[עפ״י ע״ר א ט]}\צהגדרה{. }

\הגדרה{הנס\mycircle{°} המוחלט }\מקור{[ע״ר א מט]}\צהגדרה{.}

\הגדרה{האור העליון שהוא הרבה למעלה מן הטבעיות\mycircle{°}}\צהגדרה{ }\מקור{[ע״א ד ט מא]}\צהגדרה{.}

\הגדרה{טוב\hebrewmakaf העליון\mycircle{°} }\מקור{[ע״ר א שעב, רפט]}\צהגדרה{. }

\ערך{אור ד׳ המהוה הישות }\הגדרה{- זרוע\hebrewmakaf ד׳\mycircle{°} אשר נגלתה, יסוד ההשתלמות הבלתי פוסקת }\מקור{[עפ״י א״ק ב תקל]}\צהגדרה{. }

\משנה{אור ד׳ וכבודו }\הגדרה{- אמיתת הרצון\hebrewmakaf הכללי\mycircle{°} אשר בנשמת\mycircle{°} היקום כולו }\מקור{[שם ג לט]}\צהגדרה{. }

\משנה{אור ד׳ }\הגדרה{- המגמה האלהית היותר ברורה ותהומית לאין חקר }\מקור{[פנק׳ ג שלא]}\צהגדרה{.}

\הגדרה{נשמת\hebrewmakaf העולמים\mycircle{°} }\צהגדרה{[קבצ׳ ב קנה].}

\הגדרה{שפעת החיים הנובעים ושוטפים ממקור חיי העולמים }\מקור{[קבצ׳ א רכג (פנק׳ א תקכה)]}\צהגדרה{.}

\משנה{אור ד׳ בעולמו }\הגדרה{- אור השכינה\mycircle{°}, נשמת העולמים, הוד האידיאליות\mycircle{°} האלהית החיה בכל }\מקור{[א״ק ב שסח, א״ש יד ד]}\צהגדרה{. }

\הגדרה{אורו\hebrewmakaf של\hebrewmakaf משיח\mycircle{°} }\מקור{[א״ק ב תקסא]}\צהגדרה{. }

\משנה{אור ד׳ }\הגדרה{- גאולה\mycircle{°} רוחנית עליונה, נהירה אל ד׳\mycircle{°} ואל טובו }\מקור{[א׳ צא]}\צהגדרה{.}

\הגדרה{״אורן של ישראל״, רוח\hebrewmakaf ה׳\mycircle{°} השורה על כלל\mycircle{°}\hebrewmakaf ישראל ותפארתם\mycircle{°} הכללית }\מקור{[ע״א א ד לג]}\צהגדרה{. }

\הגדרה{אור\hebrewmakaf תורה\mycircle{°}, אור\hebrewmakaf חיים\mycircle{°} }\מקור{[קובץ ו קפח]}\צהגדרה{.}

\משנה{אור ד׳ אשר באומה\mycircle{°}}\הגדרה{ - השראת\hebrewmakaf השכינה\mycircle{°} וקדושת\mycircle{°} התורה\mycircle{°} והמצוה\mycircle{°} }\מקור{[עפ״י ע״ר א קעג]}\צהגדרה{. }

\משנה{אור ד׳ בעולם }\הגדרה{- האמת\mycircle{°} והצדק\mycircle{°} של דעת הקודש\mycircle{°} }\מקור{[ע״ר א רו]}\צהגדרה{. }

\משנה{אור ד׳ וטובו }\הגדרה{- הטוב\mycircle{°} והצדק }\מקור{[ל״ה 118]}\צהגדרה{. }

\ערך{אור ד׳ בארץ }\הגדרה{- התורה האמיתית והשכל הצלול\mycircle{°} והבהיר\mycircle{°} }\מקור{[אג׳ א קיז]}\צהגדרה{. }

\משנה{אור ד׳ }\הגדרה{- קדושת התורה והיהדות הנאמנה, מורשה קהלת יעקב }\מקור{[מ״ר 367]}\צהגדרה{.}

\הגדרה{אור צדק עולמים אשר בתורת\hebrewmakaf חיים\mycircle{°} }\מקור{[מ״ר 366]}\צהגדרה{.}

\הגדרה{המוסר\mycircle{°} האלהי\mycircle{°} המתגלה בתורה\mycircle{°}, במסורת, בשכל\mycircle{°} וביושר\mycircle{°} }\מקור{[א״ק ג א]}\צהגדרה{. }

\הגדרה{צמאון\hebrewmakaf אלהי\mycircle{°} }\מקור{[עפ״י קובץ ז רח]}\צהגדרה{.}

\הגדרה{נועם\mycircle{°} הקודש }\מקור{[א״ק ב שי]}\צהגדרה{. }

\הגדרה{זיו\mycircle{°} אור החכמה והשגת האמת }\מקור{[פנ׳ ח]}\צהגדרה{. }

\paragraphs

\ערך{אור ד׳ העליון }\הגדרה{- }\משנה{(לעומת אור\hebrewmakaf הדעת\hebrewmakaf התחתון\mycircle{°}) }\הגדרה{- אור\hebrewmakaf המקיף\mycircle{°} הגדול ורחב מרחבי שחקים }\מקור{[ע״א ד ט כט]}\צהגדרה{. }

\paragraphs

\ערך{אור האמת }\הגדרה{- ע״ע אמת. }

\paragraphs

\ערך{אור הגדול}\הגדרה{ -}\משנה{ האור הגדול}\הגדרה{ - התוכן האלהי\mycircle{°} שאי אפשר להגותו ולשערו\mycircle{°} }\מקור{[פנק׳ א שסב]}\צהגדרה{.}

\paragraphs

\ערך{אור הגלוי }\הגדרה{- }\משנה{האור הגלוי}\הגדרה{\mycircle{°} - האור הנראה של אור\hebrewmakaf התורה\mycircle{°} וחכמת\hebrewmakaf ישראל\mycircle{°} כולה בקדושתה\mycircle{°} וטהרתה\mycircle{°}, בבינתה והכרתה, בכבודה וישרותה, בעושר סעיפיה בעומק הגיונותיה ובאומץ מגמותיה. תלמודה של תורה בכל הרחבתה והסתעפו(יו)תיה, בדעת וכשרון, ברגש חי וקדוש, וברצון אדיר וחסון\mycircle{°} לחיות את אותם החיים הטהורים והקדושים אשר האור המלא הזה מתאר אותם לפנינו. (אור\hebrewmakaf הקודש\hebrewmakaf החבוי\mycircle{°}) בהיותו מתקרב מאד אל מושגינו, אל צרכינו הזמניים, ואל מאויינו הלאומיים }\מקור{[מא״ה ג (מהדורת תשס״ד) קכג, קכה]}\צהגדרה{. }

\הגדרה{ע״ע אור קודש חבוי. }

\paragraphs

\ערך{אור הדעת התחתון }\הגדרה{- }\משנה{(לעומת אור\hebrewmakaf ד׳\hebrewmakaf העליון\mycircle{°}) }\הגדרה{- אור\hebrewmakaf הפנימי\mycircle{°} המרוכז באוצר הדעת אשר לבן\hebrewmakaf אדם }\מקור{[ע״א ד ט כט]}\צהגדרה{. }

\paragraphs

\ערך{אור ההשואה }\הגדרה{- ע׳ במדור מונחי קבלה ונסתר.  }

\paragraphs

\ערך{אור החיים }\הגדרה{- מקור זיו\mycircle{°} החיים }\מקור{[עפ״י א״ק ב שכט]}\צהגדרה{. }

\הגדרה{שאיפת הגדלת כחותיהם }\מקור{[ע״א ד ו מא]}\צהגדרה{.}

\הגדרה{ע״ע אור. ע״ע אור חיים. }

\ערך{אור החיים }\הגדרה{- אור\hebrewmakaf ד׳\mycircle{°}, זיו\mycircle{°} החכמה\mycircle{°} האלהית, ואור פני מלך חוטר מגזע ישי }\מקור{[ע״א ב ט קנב]}\צהגדרה{. }

\paragraphs

\ערך{אור החיים העליונים }\הגדרה{- הנשגב\hebrewmakaf הכללי\mycircle{°} }\מקור{[א״ק ג רפ]}\צהגדרה{.}

\paragraphs

\ערך{אור העליון}\הגדרה{ - היושר האמיתי, העצמיות והקדושה בבירורה }\מקור{[קובץ ו רסט]}\צהגדרה{.}

\paragraphs

\ערך{אור העתיד}\הגדרה{ - הופעת\mycircle{°} כבוד\hebrewmakaf ד׳\mycircle{°} בעולם }\מקור{[א״ק ב קפב]}\צהגדרה{.}

\paragraphs

\ערך{אור הפנימי}\הגדרה{ - }\משנה{האור הפנימי (של המושג מאורו של אלקים\hebrewmakaf חיים, צור ישעינו, לגדולי המשיגים)}\הגדרה{ - ע״ע אור, האור הפנימי.}

\paragraphs

\ערך{אור השכינה }\הגדרה{- ע׳ במדור מונחי קבלה ונסתר, שכינה. }

\paragraphs

\ערך{אור התורה }\הגדרה{- ע׳ במדור תורה.}

\משנה{אורה של תורה }\הגדרה{- ע׳ שם. }

\paragraphs

\ערך{אור חדש }\הגדרה{- ע״ע אור קודש.}

\paragraphs

\ערך{״אור חדש״ }\הגדרה{- אוצר חיים חדש ומלא רעננות\mycircle{°}, נשמות\hebrewmakaf חדשות\mycircle{°}, מלאות הופעת חיים גאיוניים, ממשלת עולמי\hebrewmakaf עולמים\mycircle{°}, הפורחת ועולה, המשחקת בכל עת לפני הדר\mycircle{°} אל\mycircle{°} עליון, האצולות מזיו\mycircle{°} החכמה\mycircle{°} והגבורה\mycircle{°} של מעלה }\מקור{[א״ק ג שסח]}\צהגדרה{. }

\paragraphs

\ערך{אור חיים }\הגדרה{- אור קיום של הדר\mycircle{°} נצח נצחים }\מקור{[א״ק ג נח]}\צהגדרה{. }

\paragraphs

\ערך{אור חיים}\myfootnote{ \textbf{אור חיים} - לבירור ההבחנה בין ״\textbf{אור}״ ל״\textbf{חיים}״, ע׳ הוד הקרח הנורא פרק א סי׳ ג, ד, ובעיקר בעמ׳ לו. \label{12}}\הגדרה{ - דעה\mycircle{°} ורצון\mycircle{°}, רוח\hebrewmakaf הבטה ומציאות\hebrewmakaf מלאה\mycircle{°} }\מקור{[עפ״י א׳ יא]}\צהגדרה{. }

\הגדרה{ע״ע אור החיים. ע׳ בנספחות, מדור מחקרים, אור וחיים.}

\paragraphs

\ערך{אור חַי\hebrewmakaf העולמים\mycircle{°} }\הגדרה{- אור\hebrewmakaf עליון\mycircle{°}, מקור מקורות, חיי החיים }\מקור{[קובץ ה צט]}\צהגדרה{.}

\paragraphs

\ערך{אור חֵי\hebrewmakaf העולמים\mycircle{°} }\הגדרה{- הענין\hebrewmakaf האלהי\mycircle{°}}\צהגדרה{ }\מקור{[א׳ סו]}\צהגדרה{.}

\הגדרה{הטוב\hebrewmakaf הכללי\mycircle{°}, הטוב האלהי השורה בעולמות\mycircle{°} כולם. נשמת\hebrewmakaf כל, האצילית, בהודה\mycircle{°} וקדושתה\mycircle{°} }\מקור{[עפ״י א״ש פרק ב]}\צהגדרה{. }

\הגדרה{החיים האלהיים ההולכים ושופעים, המחיים כל חי, השולחים אורם מרום גובהם עד שפל תחתיות ארץ, המתפשטים על אדם ועל בהמה יחד }\מקור{[עפ״י ע״ט י]}\צהגדרה{. }

\הגדרה{הרצון הכללי, הרצון העולמי }\מקור{[א״ק ג נ]}\צהגדרה{. }

\paragraphs

\ערך{אור עליון }\הגדרה{- }\משנה{האור העליון }\הגדרה{- חייו ומקור שפעו, מחוללו ומהוהו של העולם }\מקור{[עפ״י ע״א ד ט נב]}\צהגדרה{. }

\הגדרה{יסוד הכל ומקורו }\מקור{[קובץ א תרלו]}\צהגדרה{.}

\הגדרה{הזיו\mycircle{°} האלהי\mycircle{°}, יוצר כל }\מקור{[א״ק א קצב]}\צהגדרה{. }

\משנה{האור העליון }\הגדרה{- זוהר\mycircle{°} הצחצחות\mycircle{°} של קדש\hebrewmakaf הקדשים\mycircle{°} }\מקור{[א״ק ג רח]}\צהגדרה{. }

\הגדרה{מקור מקוריות כל חיים וכל יש }\מקור{[קובץ ה נ]}\צהגדרה{. }

\הגדרה{מקור מקורות, חיי החיים, אור חי העולמים }\מקור{[שם צט]}\צהגדרה{. }

\הגדרה{מקור החיים והעונג }\מקור{[שם כה]}\צהגדרה{. }

\הגדרה{ע׳ במדור מונחי קבלה ונסתר, אור אין סוף. ע״ע אור אלהי. ע״ע אור ד׳.}\footref{11}

\paragraphs

\ערך{אור עליון }\הגדרה{- }\משנה{האור העליון שבהויה}\הגדרה{ - העילוי\mycircle{°} הרוחני\mycircle{°} }\מקור{[קובץ א קסח]}\צהגדרה{.}

\הגדרה{בהירות חיים והויה מלאה זיו קדש\mycircle{°}. החיים העליונים\mycircle{°} ברום ערכם, בהופיעם ממכון הטוב\mycircle{°} והעלוי\mycircle{°} הנשגב\mycircle{°} }\מקור{[עפ״י ע״ר א קצג]}\צהגדרה{. }

\הגדרה{לשד חיי העולמים הזולף בחסדי אבות ממקור הברכה\mycircle{°},  מיסוד עולם שהוא קודם ונעלה מכל הגבלה\mycircle{°} וחוקיות מוטבעה }\מקור{[אג׳ ג נח]}\צהגדרה{. }

\משנה{האור העליון הבלתי מוגבל }\הגדרה{- המוסר\mycircle{°} האלהי המוחלט }\מקור{[א״ת ד ד]}\צהגדרה{.}

\paragraphs

\ערך{אור קודש}\myfootnote{ \textbf{אור קודש} - בא״ק ג רפו הנוסח הוא: אור חדש.\label{13}}\הגדרה{ - טללי שפעת\mycircle{°} חכמה\mycircle{°} וציורים\mycircle{°} עליונים\mycircle{°}, נשגבים\mycircle{°} ונעימים\mycircle{°}, שהם משתפכים לתוך הנשמה\mycircle{°}, מודיעים לה זיוים\mycircle{°} עליונים, מנשאים אותה לרוממות\mycircle{°} מעלה\mycircle{°}, מקרבים לה את היש\hebrewmakaf העליון\mycircle{°}, את הרוחניות\mycircle{°} והטוהר\mycircle{°} המעולה, את אור\hebrewmakaf ד׳\hebrewmakaf ממרומיו\mycircle{°} }\מקור{[קובץ ו קנ]}\צהגדרה{.}

\משנה{אור קודש בו מלאה הנשמה }\הגדרה{- מאויי נצח והוד עליון ושאיפת חיים אצילית לאין חקר }\מקור{[ע״ר א סח]}\צהגדרה{.}

\הגדרה{ע״ע אור, האור בעצם.}

\paragraphs

\ערך{אור קודש חבוי }\הגדרה{- האור\mycircle{°} הקדוש\mycircle{°} הגנוז, (ה)מקור האלהי\mycircle{°} של התורה\mycircle{°}, החוסן\mycircle{°} של הנבואה\mycircle{°}, סגולתה\mycircle{°} של רוח\hebrewmakaf הקודש\mycircle{°} והמחזה העליון\mycircle{°}. האור הגנוז של אור הנבואה ורוח הקודש. מעין החיים של שורש התורה האלהית ומכון כל חזון ומראה עליון. אור הקודש של חמדת עולמים הגנוזה, שורש התורה האלהית ומקור הנבואה ורוח הקודש המיוחד לישראל }\מקור{[עפ״י מא״ה ג (מהדורת תשס״ד) קכב\hebrewmakaf ה]}\צהגדרה{.}

\הגדרה{ע״ע אור הגלוי. ע׳ במדור אליליות ודתות, חושך חבוי.}

\paragraphs

\ערך{אורגן }\הגדרה{- גוף חי, מסודר }\מקור{[רצי״ה א״ש ה הערה 1]}\צהגדרה{.}

\הגדרה{ע׳ בנספחות, מדור מחקרים, ארגון, מאורגן.}

\paragraphs

\ערך{אורגניסמוס }\הגדרה{- הקישור העצמי שיש להגוף עם הנשמה }\מקור{[קובץ ה קנה]}\צהגדרה{.}

\ערך{אורגניסמוס }\הגדרה{- }\משנה{(האורגניות הכללית שביצירה כולה) }\הגדרה{- קישור ושילוב\mycircle{°} החלקים זה בזה בכל צומח ובכל חי ובאדם. כל החלקים שביש (ה)צריכים זה לזה, ותהומות רבה והררי עד (ש)הם זה בזה משולבים ומצורפים }\מקור{[עפ״י א״ק ב תיז]}\צהגדרה{. }

\משנה{חק האורגניות}\הגדרה{ - היחש החי של השפעה ושל קבלה, (ה)הולך וחורז ומקיף את כל המצוי, את החומריות ואת הרוחניות, את הפעולות, המנהגים, ההרגשות ואת המחשבות}\צהגדרה{ }\מקור{[ע״א ד ו מא]}\צהגדרה{.}

\paragraphs

\ערך{״אורה״ }\הגדרה{- ע׳ במדור פסוקים ובטויי חז״ל. }

\ערך{אורה }\הגדרה{- }\משנה{האורה }\הגדרה{- העילוי\mycircle{°} הרוחני\mycircle{°} }\מקור{[עפ״י קובץ א קסח]}\צהגדרה{.}

\paragraphs

\ערך{אורה }\הגדרה{- }\משנה{אורה אלהית }\הגדרה{- שלמות הכל, ושלמות העדן\mycircle{°} של מקור הכל, שאין לנו שום מושג ממנה כי\hebrewmakaf אם מה שאנו חשים את מציאותה ומתענגים מזיוה בכל עומק נפש רוח ונשמה }\מקור{[עפ״י א״ק ג רצ, א׳ קיא]}\צהגדרה{. }

\הגדרה{הגודל והשיגוב האלהי }\מקור{[קובץ ו צ]}\צהגדרה{. }

\paragraphs

\ערך{אורה }\הגדרה{- }\משנה{אורה אלהית }\הגדרה{- החוסן\mycircle{°} המלא, האור העליון, המון החיים ומקור יממיהם }\מקור{[עפ״י אוה״ק ב תמז]}\צהגדרה{.}

\הגדרה{האורה האנושית בכללה המתגלה באורן\hebrewmakaf של\hebrewmakaf ישראל\mycircle{°}}\צהגדרה{ }\מקור{[אג׳ א מג]}\צהגדרה{.}

\משנה{האורה הכללית }\הגדרה{- המשך החיים הנובע מהתשוקה העליונה והכללית של קרבת\hebrewmakaf אלהים\mycircle{°}}\צהגדרה{ }\מקור{[מ״ר 38]}\צהגדרה{.}

\משנה{אורה עליונה }\הגדרה{- אור\mycircle{°} חכמת\mycircle{°} כל עולמים\mycircle{°} }\מקור{[א׳ כט]}\צהגדרה{. }

\paragraphs

\ערך{אורה }\הגדרה{- }\משנה{האורה הכללית }\הגדרה{- דעת היהדות\mycircle{°} בכל הדרת נשמתה הפנימית, העולה מעומקה של תורה\mycircle{°}, ומאוצר ההרגשה האלהית הבאה בהתמדת התלמוד והעיון בדברים שהם כבשונו של עולם, עם המכשירים המוסריים והעיוניים הדרושים לזה }\מקור{[עפ״י א״ה 913]}\צהגדרה{. }

\paragraphs

\ערך{אורה אלהית }\הגדרה{- החפץ הציורי\mycircle{°} והמעשי, לשלטון של עילוי\mycircle{°} כל עז\mycircle{°} של צדק\mycircle{°} ואור\mycircle{°} }\מקור{[ע״ה קלב]}\צהגדרה{. }

\הגדרה{השלמות, המעשית והשכלית, הרגשית והתכונית, השלמות במילואה }\מקור{[א״ק ב שעה]}\צהגדרה{. }

\משנה{אורה }\הגדרה{- שלמות בכל תיקונה. חיי קודש\mycircle{°} וטוהר\mycircle{°} }\מקור{[עפ״י שם רפז, שכט]}\צהגדרה{. }

\משנה{האורה העליונה }\הגדרה{- גדולת החיים של הכרת האלהות האמיתית, הכוללת את כל תענוגי הרוח וכל העדנים עמם ברום עוזם}\צהגדרה{ }\מקור{[קובץ ז עו]}\צהגדרה{.}

\paragraphs

\ערך{אורה חיצונית }\הגדרה{- נימוסים אנושיים טובים ויפים, תיקוני מדינה וממלכה נוחים ונעימים. התקדמות, סדרים, פאר\mycircle{°} ונעימות חיצונית הדורשים עמם חכמה מעשית רבה ללכת קוממיות ולהיות גוי איתן מלא חכמה מעשית וכליל יופי\mycircle{°} }\מקור{[עפ״י ע״א ג ב טז]}\צהגדרה{.}

\הגדרה{ע״ע אורה פנימית.}

\paragraphs

\ערך{אורה פנימית }\הגדרה{- אורה\hebrewmakaf של\hebrewmakaf תורה\mycircle{°}, רוח\hebrewmakaf הקודש\mycircle{°} והנבואה\mycircle{°}, ששופע בישראל\mycircle{°} ביחוד ממקום בית\hebrewmakaf המקדש\mycircle{°} יצאה האורה\mycircle{°}, היא האורה\hebrewmakaf האלהית\mycircle{°} שמאירה בישראל לבדם ואין לזרים חלק בו. כח הקדושה\mycircle{°} המיוחדת שהיא מעלה את ישראל למצב רם ברוח קדושה ודעת\hebrewmakaf אלהים\mycircle{°} ודרכיו\mycircle{°}, שכולה אומרת כבוד\hebrewmakaf אלהים\mycircle{°} }\מקור{[עפ״י ע״א ג ב טז]}\צהגדרה{.}

\הגדרה{ע״ע אורה חיצונית.}

\paragraphs

\ערך{אורה רוחנית }\הגדרה{- }\משנה{האורה הרוחנית }\הגדרה{- הגבורה\mycircle{°} הגמורה המנצחת את כל העולמים וכל כחותיהם }\מקור{[א׳ פד]}\צהגדרה{. }

\paragraphs

\ערך{אורה שכלית }\הגדרה{- }\משנה{האורה השכלית}\הגדרה{ - הדעות הקבועות וארחות הדעה\mycircle{°} }\מקור{[ע״ר א קסח]}\צהגדרה{.}

\מעוין{◊ }\משנה{האורה השכלית}\הגדרה{ באה מרוב תורה ודעת, מהרבה שימוש\hebrewmakaf של\hebrewmakaf חכמים\mycircle{°},  ומהרבה דעת העולם והחיים }\מקור{[א״ק א רמ]}\צהגדרה{. }

\paragraphs

\ערך{אורה של תורה }\הגדרה{- ע׳ במדור תורה.}

\paragraphs

\ערך{אורה של תורה }\הגדרה{- ע׳ במדור תורה, אור התורה. }

\paragraphs

\משנה{״אורות״ }\צהגדרה{- }\צמשנה{(עניינו של ספר אורות) }\צהגדרה{- שלמות גילוי אמתת קדושת עצמיותם של ישראל וערכם האלהי העליון הנצחי}\צמקור{ [א׳ קפז].}

\paragraphs

\ערך{אורות הקדש }\הגדרה{- החיים בחיים (ה)עליונים ברום עולמים בצחצחות\mycircle{°} אידיאליהם\mycircle{°}, ספוגי קדש\hebrewmakaf קדשים\mycircle{°} }\מקור{[ע״ר א קפ]}\צהגדרה{. }

\הגדרה{ע״ע מוסר הקודש. ר׳ חכמת הקודש.}

\paragraphs

\ערך{״אורך ימים״}\myfootnote{ תהילים כג ו, צג ה\label{14}}\הגדרה{ - כל הימים\mycircle{°} בעמדם בצביונם המלא, (בהיותם) בתוכן העליון\mycircle{°}, בקשר החיים עם הנצחיות\mycircle{°} האלהית\mycircle{°}, (ששם) אין הזמן\mycircle{°} עובר, הכל קיים, (כש)כל העשוי בהם עומד ומזהיר, ומשביע את הנשמה\mycircle{°} זיו\mycircle{°} וצחצחות\mycircle{°} ושובע נעימות. מלוא הימים, (כאשר) הצירוף של כל השיגוב, שנעשה מכל פרטי החיים בטוהר\mycircle{°} קדושתם\mycircle{°}, מתעלה בזיו ונהורא בהירה }\מקור{[עפ״י ע״ר ב עח]}\צהגדרה{. }

\משנה{באורך ימים}\צהגדרה{ כלולים הם כל הימים וכל ההשפעות\mycircle{°}, כל ההופעות\mycircle{°} וכל ההזרחות\mycircle{°}, כל המדעים וכל ההרגשות, כל צדדי ההסתכלות, וכל ארחות הדעה\mycircle{°} }\צמקור{[א״ק א סו]. }

\paragraphs

\ערך{״אורך ימים״}\myfootnote{ ברכת קריאת שמע שבערבית.\label{15}}\הגדרה{ - }\משנה{(תאר למעלה שבמצוות כ״אורך ימינו״ לעומת ״חיינו״) }\הגדרה{- התועלת המגיעה בשלמות הנשמה\mycircle{°} במה שנעלם ואינו נרגש כלל אבל הוא מקנה לה קנין נשגב }\מקור{[ע״ר א תיב (פנק׳ ג רסח)]}\צהגדרה{. }

\הגדרה{ע״ע ״חיים״, תאר למעלה שבמצוות (לעומת אורך ימים).}

\paragraphs

\ערך{אורך ימים }\הגדרה{- השלמת החיים היוצאת חוץ לגבול התעודה הפרטית. שמאריכים הם על המדה המוגבלת לפרטיותו ויספיק האדם תעודת החיים בעד העתיד בעד דור יבוא }\מקור{[עפ״י ע״א ג א נח (ח״פ מב.)]}\צהגדרה{. }

\הגדרה{הנביעה של האורה\mycircle{°} הרוחנית\mycircle{°} המתגברת ועולה על ידי הברכה\mycircle{°} הפנימית של הנשמה\mycircle{°}, שהיא באה ביחוד מהמקור של ההוקרה הפנימית והתוכית של הצד החיצוני המוכר בחכמה\mycircle{°}, שזהו התוכן הפרטי שבקניני הרוח, שהיא הכרה מפורדת לחלקים שונים, (העושה את) הימים מבורכים בפרטיותם }\מקור{[עפ״י שם ד יג ט]}\צהגדרה{.}

\הגדרה{ע״ע אורך שנים. }

\paragraphs

\ערך{אורך שנים }\הגדרה{- האורה הכללית של השנים. תפיסת חיים העולה בצורה כללית, מפני הוקרת התוכן האצילי\mycircle{°} של החכמה\mycircle{°} (ה)מביאה נהרה אחדותית בנפש האדם, ודחיפת החיים היוצאת ממנה היא משאת נפש לתוכן ההכללה של החכמה בצורתה הבהירה והמקפת }\מקור{[עפ״י ע״א ד יג ט]}\צהגדרה{. }

\הגדרה{ע״ע אורך ימים.}

\paragraphs

\ערך{אורן של צדיקים }\הגדרה{- האור\mycircle{°} הרענן\mycircle{°}, שהקודש\hebrewmakaf העליון\mycircle{°} חי במלא חפשו\mycircle{°} הנאדר בתוכו }\מקור{[א״ק ג ק]}\צהגדרה{. }

\paragraphs

\ערך{אושר }\הגדרה{- }\משנה{האושר}\הגדרה{ - }\מעוין{◊ }\הגדרה{מקור המנוחה\mycircle{°} והבטחה\mycircle{°} }\מקור{[ע״ר ב סד]}\צהגדרה{.}

\paragraphs

\ערך{אושר }\הגדרה{- }\משנה{(נשמתי) }\הגדרה{- תחושת הנשמה\mycircle{°} את עדונה הגדול, את זיו\mycircle{°} חייה המלא עדני עד, מהמון זרמי חיי עולם וישות אדירה, השוטפים בקרבה פנימה, מנחת דשן קרבת\mycircle{°} אלהים\hebrewmakaf חיים\mycircle{°}, ואור קדושתו המלאה על כל גדותיה }\מקור{[עפ״י ע״ר א קח\hebrewmakaf ט]}\צהגדרה{. }

\משנה{אושר נעלה }\הגדרה{- קדושת החיים במוסר\mycircle{°} מדות טובות ודעת\hebrewmakaf אלקים\mycircle{°} }\מקור{[פנ׳ פה]}\צהגדרה{. }

\משנה{אושר }\הגדרה{- שלמות אמיתית בדעת ובהנהגה }\מקור{[ע״א ג ב נה]}\צהגדרה{.}

\משנה{המצב המאושר }\הגדרה{- המצב הנפשי\mycircle{°}, שנועם\mycircle{°} ד׳, ועונג\mycircle{°} אהבה\mycircle{°} ושיקוק עליון\mycircle{°} מופיע בתוך הנשמה\mycircle{°} במצב של מנוחה וקביעות }\מקור{[א״ק ב תקח]}\צהגדרה{. }

\paragraphs

\ערך{אושר }\הגדרה{- }\משנה{קץ האושר }\הגדרה{- שיהפך העונג\mycircle{°} היותר חמרי ויותר מלוכלך בניוול לקודש\mycircle{°} אידיאלי עליון }\מקור{[פנק׳ ג שלט]}\צהגדרה{.}

\paragraphs

\ערך{אושר העולם }\הגדרה{- השמחה\mycircle{°} העדינה תולדתו של העדן\mycircle{°} האציל\mycircle{°} הבא מההארה\mycircle{°} של זיו\mycircle{°} הרעיונות של עומק האמונה\mycircle{°}, הרפודה באהבה\hebrewmakaf האלהית\mycircle{°} והדבקות\mycircle{°} הגדולה והרחבה שזיו שדי\mycircle{°} פרוש עליה }\מקור{[א״א 127]}\צהגדרה{. }

\משנה{תכלית האושר}\הגדרה{ - התכלית האידיאלית\mycircle{°} האחרונה, שהיא עצת\hebrewmakaf ד׳\mycircle{°} וברכתו\mycircle{°} לאדם ולעולם }\מקור{[ל״ה 158]}\צהגדרה{.}

\הגדרה{ע׳ במדור תיאורים אלהיים, שמחת ד׳ במעשיו.}

\ערך{אושר עליון\mycircle{°}}\הגדרה{ - שיהיה ״ד׳ אחד ושמו אחד״\mycircle{°} }\מקור{[א׳ קס]}\צהגדרה{. }

\הגדרה{ע״ע תֹּם. }

\paragraphs

\ערך{״אות מן התורה״}\myfootnote{ ע׳ מגלה עמוקות על ואתחנן אופן קצז. בית עולמים קלט.: ד״ה תיקונא ״נשמות ישראל הם אותיות התורה״.\label{16}}\הגדרה{ - נשמה מישראל }\מקור{[עפ״י א״ת יא ב]}\צהגדרה{.}

\paragraphs

\ערך{אותיות }\הגדרה{- }\משנה{כ״ב אתוון}\myfootnote{ \textbf{כ״ב אותיות} - בביאור ס׳ קהלת לרמ״ד וואלי, עמ׳ קעד ״א״ת, דהיינו כללות ההשפעה מא׳ ועד ת׳״. ושם קעו: ״בגין דאיהו עמודא דאמצעיתא, שהוא כלול מכל האורות, דהיינו מלת ״את״ שרומזת אל הכללות מא׳ ועד ת׳״. ובבאורו לאיכה, עמ׳ קמא: ״כי כ״ב אותיות הם רומזים אל הכללות כידוע״.\label{17}}\הגדרה{ - כלל ההנהגה}\צהגדרה{ }\מקור{[ג״ר 29]}\צהגדרה{. }

\הגדרה{סדרים לגילוי המאורות\mycircle{°}, }\צהגדרה{<ויש גילוי מצד השמיעה\mycircle{°} וגילוי מצד הראי׳\mycircle{°}, הרי ב׳ אותיות לכל ספי׳\mycircle{°}, ואחת הכוללת ענין חיבור האורות להאותיות בתחילת המחשבה, ואחת בסוף המעשה, הרי כ״ב> }\מקור{[פנק׳ ג צ]}\צהגדרה{.}

\paragraphs

\ערך{אז }\הגדרה{- מורה על העבר, אבל לא רק בדרך פרוזי\mycircle{°}, ספור של מאורע שאינו מרותק עם רגשי הנפש והתפעלו(יו)תיה השיריות, כ״א באורח שירי, ומצב נפשי מרומם }\מקור{[ר״מ קיט]}\צהגדרה{. }

\paragraphs

\ערך{אח }\הגדרה{- הקרוב היותר מקורב, המגובל בגבול האחדות }\מקור{[ר״מ קכ]}\צהגדרה{. }

\paragraphs

\ערך{אחדות }\הגדרה{- }\משנה{האחדות }\הגדרה{- יחוד שלטון השי״ת\mycircle{°} בהתחלת הסבות\mycircle{°} הראשיות, המסבבות כל המון המעשים, שהן בערך השמים\mycircle{°}, ובגמר כל תכליתם, שהן בערך הארץ\mycircle{°}, ובכל האמצעים הרבים השונים ומסובכים, אשר ביניהם, שהם בערך ד׳\hebrewmakaf רוחות\mycircle{°} העולם, שכאילו מחברים את השמים עם הארץ }\מקור{[ע״ר א רמה]}\צהגדרה{.}

\משנה{אחדות ד׳}\צהגדרה{ - }\הגדרה{הדעה היותר עליונה של מושג האלהות\mycircle{°} }\מקור{[ל״ה 224]}\צהגדרה{. }

\משנה{אחדות הרוחנית}\הגדרה{ - שם\hebrewmakaf ד׳\mycircle{°} אחד\mycircle{°} השוכן בישראל\mycircle{°} }\מקור{[ע״א ב ביכורים לט]}\צהגדרה{.}

\משנה{אחדות }\הגדרה{- מגמה\mycircle{°} אחת עשירה ואדירה, כוללת כל, וברוכה בכל - אור\hebrewmakaf החיים\mycircle{°} היותר מאירים ויותר שלמים. המקור האחד של כל האידיאלים\mycircle{°} היותר נשאים, שאנחנו מוצאים בנפשנו פנימה, שכל זמן שהם עולים\mycircle{°} ומתבכרים, הם באים אליו }\מקור{[עפ״י ע״ה קנב]}\צהגדרה{. }

\הגדרה{התוכן של השאיפה היותר נאצלה השיכת לכל הנברא בהתאחד הכל למטרתו היותר עליונה }\מקור{[עפ״י ע״ר א קנט]}\צהגדרה{. }

\הגדרה{השאיפה ותגבורת חיל\mycircle{°} החיים בהתרכזות עשירה של חטיבה כללית, בכל, וביחוד במציאות הרוחנית\mycircle{°} והאידיאלית\mycircle{°}, המתלבשת גם כן יפה בהחמרית\mycircle{°} והריאלית בכל מלא עולמים\mycircle{°} כולם }\מקור{[עפ״י מ״ר 16]}\צהגדרה{. }

\הגדרה{הכלליות\mycircle{°} הקדושה\mycircle{°} ברוממות קודש קדשה }\מקור{[ר״מ קסח]}\צהגדרה{. }

\משנה{יסוד האחדות\hebrewmakaf העליונה\mycircle{°}}\הגדרה{ - המציאות\mycircle{°} ההויתית\mycircle{°} המתגלה כחטיבה אחת }\מקור{[עפ״י א״ה 916]}\צהגדרה{. }

\הגדרה{ע׳ במדור שמות כינויים ותארים אלהיים, ״אחד״ (תאר כלפי מעלה). ע״ע יחוד ד׳ בעולם. }

\ערך{אחדות }\הגדרה{- }\משנה{האחדות האלהית }\הגדרה{- הרוח\mycircle{°} האצילי\mycircle{°} המקיף את כל הנטיות כולן ומאחדם עם כל המון הכוחות הגשמיים והרוחניים למטרה מוסרית\mycircle{°} עליונה }\מקור{[עפ״י א״ק ג ש]}\צהגדרה{. }

\הגדרה{הטהרה\mycircle{°} העליונה של הנקודה האמונית\mycircle{°}}\צהגדרה{ }\מקור{[קבצ׳ ב נ]}\צהגדרה{.}

\ערך{אחדות אלהים }\הגדרה{- הטוב\hebrewmakaf העליון\mycircle{°}, הטוב המגלה שאין כל רע\mycircle{°} מצוי }\מקור{[פנק׳ ד עה]}\צהגדרה{.}

\ערך{אחדות }\הגדרה{- }\משנה{האחדות המופעה בעולם }\הגדרה{- מתבארת ע״י הקשור שיש בין המצוה\mycircle{°} התורית\mycircle{°} בכלל ההתגלות של דבר\hebrewmakaf ד׳\mycircle{°} ובין כל הסדר העולמי במערכי הטבע\mycircle{°} וכל מוסדי ההויה כולם. זהו תוכן המברר את ה}\משנה{אחדות האלהית\mycircle{°} בעולם, }\הגדרה{שהכל מתאים לתוכן אחד והכל מתקשר לאגודה אחדותית אחת }\מקור{[ע״ר א כה]}\צהגדרה{. }

\הגדרה{ע׳ במדור מונחי קבלה ונסתר, ״יחוד תחתון״. ע׳ בנספחות, מדור מחקרים, אחדות ויחוד. }

\paragraphs

\ערך{אחדות }\הגדרה{- }\משנה{האחדות העולמית }\הגדרה{- הצד של השיווי שיש למצוא בהויה כולה, עד למעלה למעלה, לדימוי\hebrewmakaf הצורה\hebrewmakaf ליוצרה\mycircle{°}. הגשמיות והרוחניות, הציור והשכל, השפל והנישא, הם כולם תואמים, מתאחדים ומוקשים }\מקור{[עפ״י ע״ט סז]}\צהגדרה{. }

\paragraphs

\ערך{אחדות אין סופית }\הגדרה{- ע׳ במדור מונחי קבלה ונסתר, יחוד עליון. ושם, אור אין סוף. }

\paragraphs

\ערך{אחדות עליונה }\הגדרה{- המציאות ההוייתית (כ)חטיבה\mycircle{°} אחת }\מקור{[עפ״י פנק׳ ד רסז]}\צהגדרה{.}

\paragraphs

\ערך{אחדות עליונה }\הגדרה{- הדעת\mycircle{°}, אוצר\hebrewmakaf החיים\mycircle{°} אשר בנשמת חי\hebrewmakaf העולמים\mycircle{°} }\מקור{[א״ק א קעד]}\צהגדרה{. }

\הגדרה{אושר\mycircle{°} ותענוג\mycircle{°}, למעלה מכל אחדות\mycircle{°} ומכל צחצחות, שורש נשמתן\mycircle{°} של צדיקים אשר עם המלך ישבו במלאכתו }\מקור{[שם ג מד]}\צהגדרה{. }

\הגדרה{הסוד העליון של האורה\hebrewmakaf האלהית\mycircle{°} בראשית\mycircle{°} התחלת הופעתה, (אשר אין) יכולת בידי בן אדם להשכיל בכחו בשכלו ובאורח מדעו וחושיו, לצייר\mycircle{°} בהויתו, איזה הערכה מסוד האחדות העליונה מלמעלה\mycircle{°} למטה\mycircle{°} }\מקור{[עפ״י ח״פ מה.]}\צהגדרה{. }

\ערך{אחדות מוחלטה }\הגדרה{- אור הפשטות העליונה, יסוד העדן\hebrewmakaf העליון\mycircle{°} }\מקור{[ר״מ קג]}\צהגדרה{. }

\הגדרה{מקור כל הקדושה\mycircle{°}, מכון כל העונג\mycircle{°} הנצחי, ובסיס כל השלמות ההולכת ומתעלה עדי\hebrewmakaf עד\mycircle{°}}\צהגדרה{ }\מקור{[פנק׳ א תו]}\צהגדרה{.}

\משנה{מלא האחדות המוחלטה }\הגדרה{- מקור חיי כל החיים אור\hebrewmakaf אין\hebrewmakaf סוף\mycircle{°}, אדון כל היש ומלא כל ההויה, מקור כל הרחמים ואב כל החסדים וכל גבורות נעם, כל פאר ויפעה וכל תפארת קדש, שומע תפלה\mycircle{°} ומאזין עתירה, בלא קץ ותכלית }\מקור{[ע״ר א סה]}\צהגדרה{.}

\הגדרה{ע׳ במדור מונחי קבלה ונסתר, תפארה.}

\ערך{אחדות שלמה }\הגדרה{- }\משנה{האחדות השלמה }\הגדרה{- קודש\mycircle{°} ד׳\mycircle{°}, קדוש ישראל\mycircle{°} }\מקור{[א״ק ג ס]}\צהגדרה{. }

\paragraphs

\ערך{אחור }\הגדרה{- צד הטפל שבכל דבר הוא אחוריו }\מקור{[ע״א א ב מג, פנק׳ ג ער]}\צהגדרה{. }

\הגדרה{ע״ע פנים.}

\paragraphs

\ערך{אחור }\הגדרה{- הצד החיצוני\mycircle{°}, שהוא הרבה כהה והרבה חלוש מהצד שהחיים הפנימיים של רוח\hebrewmakaf ד׳\mycircle{°} אשר במלא עולמו יונקים ממנו }\מקור{[עפ״י מ״ר 249]}\צהגדרה{. }

\הגדרה{ע״ע פנים.}

\paragraphs

\ערך{אחור }\הגדרה{- }\משנה{ההנהגה האלהית שהיא לאחור }\הגדרה{- ההנהגה שהיא לצד ההתחסרות }\מקור{[עפ״י ע״ר ב סז]}\צהגדרה{. }

\הגדרה{ע״ע פנים.}

\paragraphs

\ערך{אחור ופנים במציאות הרוחניות }\הגדרה{- ע״ע פנים ואחור במציאות הרוחניות. }

\paragraphs

\ערך{אחוריים }\הגדרה{- השגה\mycircle{°} כללית סתומה, שאינה מתפרטת בפירוט האור בתכונה פרצופית\mycircle{°}, כ״א מתבזקת בהתבזקות כללית כמראה האחוריים שאין בו פירוט זיו\mycircle{°} הפנים בכל נתוח אבריו נושאי החושים העליונים }\מקור{[ר״מ קפ]}\צהגדרה{.}

\משנה{האחוריים של הפנים המחשביים }\הגדרה{- נחלים הנעשים מהאורים הגדולים המתפשטים מהפנים\hebrewmakaf המחשביים, נחלי חכמה מוקשבת, מחוללת אורה בינה והשכל לימודיים, המתלבשים בהמון התלמוד }\מקור{[עפ״י שם קפד]}\צהגדרה{. }

\הגדרה{ע״ע פנים, פנים מחשביים. ע׳ במדור משה, הראני נא את כבודך וגו׳. ע׳ במדור תיאורים אלהיים, אחוריים, ראיית אחוריים.}

\paragraphs

\ערך{אחוריים }\הגדרה{- }\משנה{מראה אחוריים }\הגדרה{- ע׳ במדור תיאורים אלהיים.  }

\paragraphs

\ערך{״אחסנתין״}\myfootnote{ באור הגר״א על משלי, ד ה, ח כא, יד יח, יט יד, כז כז. ממקור זוהר ח״ג רצא. (הערת מו״ר הרצב״י טאו).\label{18}}\הגדרה{ - קשר הקדושה\mycircle{°} שהוא ירושה מאבות שיש לאדם בענין עבודת\hebrewmakaf ד׳\mycircle{°} }\מקור{[עפ״י מא״ה ג קעה]}\צהגדרה{. }

\הגדרה{כח קדושת טבע הנפש שהיא מורשה לישראל }\מקור{[ה׳ רי]}\צהגדרה{. }

\הגדרה{ע״ע ״עטרין״. }

\paragraphs

\ערך{אט }\הגדרה{- ההוראה להליכה בנחת }\מקור{[ר״מ קכ]}\צהגדרה{. }

\paragraphs

\משנה{אידיאה }\צהגדרה{- מין נשמה\mycircle{°}, המשכת כח של צורה\mycircle{°} רוחנית\mycircle{°}, שיש לכל דבר שבעולם }\צמקור{[עפ״י א״ל רמד].}

\צהגדרה{הערך הרוחני האידיאלי של המציאות, פנימיות הדברים. מציאותיות רוחנית או הרוחניות המציאותית }\צמקור{[שי׳ ב 304, 303].}

\צהגדרה{מציאות התוכן הרוחני שהוא שורש המציאות }\צמקור{[שם 39, 6\hebrewmakaf 5].}

\צהגדרה{שורש רוחני\mycircle{°}, מציאותי, שעומד ביסוד המציאות של כל דבר. מציאות רוחנית, שהיא מקור המציאות החומרית\mycircle{°} }\צמקור{[עפ״י מה״ה ג רטז].}

\צמשנה{״למהלך האידיאות״ }\צהגדרה{- יסודי עולם רוחניים, מחשבתיים, מציאותיים, שמופיעים במהלך הדורות של קורא הדורות }\צמקור{[שם].}

\הגדרה{ר׳ בנספחות, מדור מחקרים, אידיאה.}

\paragraphs

\ערך{אידיאה אלהית }\הגדרה{- הרעיון האלהי שההכנה אליו הנמצאת באיזה אופן גלוי או נסתר ישר או מעוות, בכל הלבבות של האנושיות, לכל פלגותיה, משפחותיה וגוייה, ומחוללת דתות ורגשי\hebrewmakaf אמונה שונים סדרים ונמוסים }\מקור{[עפ״י א׳ קב]}\צהגדרה{. }

\מעוין{◊}\הגדרה{ סגנון המחשבה של הרעיון הרוחני\mycircle{°} בהתבררותו ביותר, בגימור קויו הרשמיים ברוח האומנות אשר להסתוריה מבטא את ה}\משנה{אידיאה האלהית }\מקור{[עפ״י א׳ קב]}\צהגדרה{. }

\ערך{האידיאה האלהית בישראל }\הגדרה{- הנטיה הרוחנית של כנסת\hebrewmakaf ישראל\mycircle{°} שהיא אורה ונשמתה של הנטיה הלאומית המעשית שלה }\מקור{[עפ״י א׳ קד, קנח]}\צהגדרה{. }

\paragraphs

\ערך{אידיאה האלהית המוחלטת }\הגדרה{- השכינה\hebrewmakaf העליונה\mycircle{°} }\מקור{[א׳ קיב]}\צהגדרה{. }

\הגדרה{הכשרון אל השכלול העליון והגמור המאיר את העולם כלו בכבודו }\מקור{[א׳ קה]}\צהגדרה{.}

\הגדרה{ע׳ בנספחות, מדור מחקרים, אידיאה אלהית ואידיאה לאומית. }

\paragraphs

\ערך{אידיאה דתית }\הגדרה{- }\משנה{האידיאה הדתית }\הגדרה{- ההופעה\mycircle{°} האלהית המוקטנת המיוחדת לצד הפרטיות המבססת את המוסר האישי הפרטי, הדאגה לחיי\hebrewmakaf הנצח האישיים הפרטיים, הדיוק הפרטי של כל מעשה בודד המקושר ברוח הכללי }\מקור{[עפ״י א׳ קי]}\צהגדרה{. }

\הגדרה{התוכן המוסרי\mycircle{°} הבא בתור תולדה מהכרת האחדות האלהית בתור גורם למעמד מוסרי יפה לכל יחיד, המביאו לחיי נצח טובים. ולמעולים וחשובים ננעץ בו גם\hebrewmakaf כן התעוררות מוסרית אדירה לשמה, המתנוצצת מהעולם האלהי, <המתגלה יפה כשהשיקוע החומרי ודרישת ההנאה הגסה, אפילו בצדדיה היפים והעדינים, המצוי בעולם האלילי המסוגל לו, סר מהם> }\מקור{[עפ״י קבצ׳ ג קיד-קטו]}\צהגדרה{.}

\הגדרה{ע״ע דת.}

\paragraphs

\ערך{אידיאה לאומית }\הגדרה{- תוכן הסגנון הצבורי של הצורה\mycircle{°} הלאומית }\מקור{[עפ״י א׳ קו]}\צהגדרה{. }

\הגדרה{הנטיה הקבוצית שבצורה הלאומית }\מקור{[עפ״י שם קב\hebrewmakaf ג]}\צהגדרה{. }

\מעוין{◊}\הגדרה{ סגנון החיים הסדרניים של החברה בהתבררותו ביותר, בגימור קויו הרשמיים ברוח האומנות אשר להסתוריה מבטא את ה}\משנה{אידיאה הלאומית }\מקור{[עפ״י שם קב]}\צהגדרה{. }

\הגדרה{ע׳ בנספחות, מדור מחקרים, אידיאה אלהית ואידיאה לאומית. }

\paragraphs

\משנה{אידיאה לאומית}\myfootnote{ ע׳ בנספחות, מדור מחקרים, אידיאה אלהית ואידיאה לאומית. \label{19}}\הגדרה{ }\צהגדרה{- כנסת\hebrewmakaf ישראל\mycircle{°} }\צמקור{[שי׳ ב 235].}

\הגדרה{שכינת\hebrewmakaf האומה\mycircle{°} }\מקור{[א׳ קו]}\צהגדרה{.}

\הגדרה{כנסת\hebrewmakaf ישראל – שהיא הלבוש\mycircle{°} לתשוקה האלהית, לדבקות קדושה, ושמחת אור עליון – החודרת בעצמת חייה בכל פרט ופרט מישראל, ובכל מעשיו ותנועותיו, שיחיו ושיגיו, שאיפותיו וקניניו הפרטיים }\מקור{[עפ״י קובץ ו קמג]}\צהגדרה{. }

\paragraphs

\ערך{אידיאה העליונה }\הגדרה{- }\משנה{האידיאה העליונה }\הגדרה{- הרצון\mycircle{°} (רצון ד׳\mycircle{°}) הקדוש\mycircle{°} והנשא }\מקור{[א״ק א קמב]}\צהגדרה{. }

\paragraphs

\משנה{אידיאל }\צהגדרה{- המשך של האידיאה\mycircle{°} }\צמקור{[שי׳ 39, 6\hebrewmakaf 5]. }

\משנה{אידיאל }\צהגדרה{- הצדק\hebrewmakaf העולמי\mycircle{°}, שהוא גם הטוב\mycircle{°}, האור\mycircle{°} הפיוט וכו׳ וכו׳, נשוא מאויי לבה של הכללות, }\צמקור{<המושגים האלה וכאלה יוכלו לבאר את מובני }\צהגדרהמודגשת{האידיאלים\hebrewmakaf האלהיים\mycircle{°} }\צמקור{ולהתיחס אליהם בהיותם מחוברים, נהגים ונאמרים אתם יחד, ולא אם אלה יתורגמו ויפורשו ויעתקו באלה בפני עצמם>}\צהגדרה{) }\צמקור{[ד״ל (מהדורת תשס״ה) אגרות יט, כ].}

\ערך{אידיאל }\הגדרה{- }\משנה{התוכן האידיאלי של העולם }\הגדרה{- סוד\mycircle{°} האלהי\mycircle{°} המוחלט שבהויה }\מקור{[א״א 18]}\צהגדרה{. }

\paragraphs

\ערך{אידיאל }\הגדרה{- }\משנה{האידיאל המוזער של העולם}\הגדרה{ - האידיאל הנופל בההויה המצומצמה }\מקור{[א״ק ב תנה]}\צהגדרה{.}

\הגדרה{ע׳ במדור מונחי קבלה ונסתר, זעיר.}

\paragraphs

\ערך{אידיאל }\הגדרה{- }\משנה{האידיאל המלא של העולם }\הגדרה{- האידיאל המרומם האלהי\mycircle{°} במלא מילואו, יסוד העולם היותר עתיק\mycircle{°}, מציאות העולם היותר ממשית. שהמציאות הזעירה\mycircle{°} (של האידיאל המוזער, הנופל בההויה המצומצמה) רק מיניקת לשדו העליון היא מתקיימת ומתברכת\mycircle{°} }\מקור{[א״ק ב תנה]}\צהגדרה{.}

\הגדרה{ע׳ במדור מונחי קבלה ונסתר, עתיק.}

\paragraphs

\ערך{אידיאל }\הגדרה{- }\משנה{התגלמות של אידיאל }\הגדרה{- ע״ע התגלמות. }

\paragraphs

\ערך{אידיאל }\הגדרה{-}\משנה{ הגילוי האידיאלי של כל דבר נעלה בחיי\hebrewmakaf הרוח\mycircle{°} המתפשט במציאות }\הגדרה{- חפשו\mycircle{°} ויסוד מציאותו מתוך רצון מלא נדבה }\מקור{[עפ״י ע״ר א פג\hebrewmakaf ד]}\צהגדרה{. }

\הגדרה{ע״ע חסד, (לעומת ברית). ע״ע ברית, (לעומת חסד). }

\paragraphs

\ערך{אידיאל לימודי}\הגדרה{ - ע״ע לימוד, האידיאל הלימודי.}

\paragraphs

\ערך{אידיאליות }\הגדרה{- הרעיון\mycircle{°}, ההתרוממות\mycircle{°} הרוחנית\mycircle{°} }\מקור{[עפ״י ע״ר א ז]}\צהגדרה{. }

\paragraphs

\ערך{אידיאליות }\הגדרה{- האהבה\mycircle{°} לדרכי\hebrewmakaf ד׳\mycircle{°} הנטועה בנפשו הלאומית (של ישראל\mycircle{°}) פנימה\mycircle{°}, והיא הולכת ועולה, פורחת ומתגדלת, לרגלי כל מה שמתגבר מקור\hebrewmakaf ישראל\mycircle{°} }\מקור{[ע״ה קלה]}\צהגדרה{. }

\ערך{אידיאליות העליונה }\הגדרה{- גילוי אור הקדש\mycircle{°} שבשאיפה הפנימית\mycircle{°} של הרוח\mycircle{°} }\מקור{[חד׳ תשס״ח קמה]}\צהגדרה{.}

\תערך{אידיאליות }\תהגדרה{- }\תמשנה{האידיאליות הנשמתית }\תהגדרה{- התשוקה העליונה והרוממה להתדבקות\mycircle{°} אלהית, תוכן האמונה\mycircle{°} }\תמקור{[עפ״י מ״ר 494]. }

\הגדרה{ע״ע תשוקה, התשוקה האידיאלית. }

\paragraphs

\ערך{אידיאליות }\הגדרה{- }\משנה{האידיאליות }\הגדרה{- האצילות\mycircle{°} }\מקור{[א״ק א עט]}\צהגדרה{. }

\משנה{אידיאליות אלהית }\הגדרה{- האידיאליות\mycircle{°} הגמורה, כל התפארת\mycircle{°}, כל ההוד\mycircle{°} שביסוד המציאות }\מקור{[א״ק ג קפד]}\צהגדרה{. }

\paragraphs

\ערך{אידיאלים }\הגדרה{- }\משנה{האידיאלים היותר נשאים, שהם הולכים ושואבים תמיד עלוי\mycircle{°} וצחצוח\mycircle{°} ממקור העצמיות\mycircle{°} העליונה\mycircle{°}}\הגדרה{ - פלגות נהרי אורותיה\mycircle{°} של עריגת הנשמה\mycircle{°} לעצמיות האלהית }\מקור{[עפ״י מ״ר 507]}\צהגדרה{. }

\משנה{האידיאלים האלהיים\mycircle{°}}\הגדרה{ - השמות\mycircle{°} האלהיים, דרכי\hebrewmakaf ד׳\mycircle{°}, חפציו\mycircle{°}, האצילות\mycircle{°}, הספירות\mycircle{°}, המדות\mycircle{°}, השבילים, הנתיבות, השערים\mycircle{°} והפרצופים\mycircle{°}, שמקצת תכנם האידיאלי\mycircle{°} חקוק וקבוע הוא גם כן בנפש\mycircle{°} האדם,}\myfootnote{ \textbf{האידיאלים }\textbf{האלהיים} - \textbf{שמקצת תכנם האידיאלי חקוק וקבוע הוא גם כן בנפש האדם, אשר עשהו }\textbf{האלהים}\textbf{ ישר} - ע׳ שיחות על אהבה לר״י אברבנאל, מוסד ביאליק, תשמ״ג, עמ׳ 450\hebrewmakaf 440. ע״ע מלבי״ם, איוב לו א\hebrewmakaf ד. ״האידעען הנטועים בנפשו, הם אמתיות מונחלות וטבועות בנפש נפש ממקור מחצבה, ירושה לה מאלהי הרוחות בעודה בחביון עוזה, והם קדושים וטהורים אלהיים; האידעען קראם בשם דעי, ודעים. (כן אצל רש״ט גפן בממדים, הנבואה והאדמתנות עמ׳ 103); ייחס מלין אלה לאלהים, כי הוא הטביע דעים אלה בנפשו להשיג על פיהם את סודותיו ואמתיותיו; אצל ד׳ הדעים האלה הם בתמימות ובשלמות, וא״כ תמים דעים נמצא עמך, והוא האלהות הנמצא טבוע בשורש נפשך״. ע״ע בנספחות, מדור מחקרים, אידיאה.\label{20}}\הגדרה{ אשר עשהו האלהים ישר\mycircle{°} }\מקור{[ע״ה קמה]}\צהגדרה{. }

\משנה{אידיאלים כלליים }\הגדרה{- מטע ד׳\mycircle{°}. המגמות\mycircle{°} העולמיות כולן, אצילות\mycircle{°} האורות\hebrewmakaf העליונים\mycircle{°} ברוממות תעודתם, שנשמתה של האומה הישראלית, ונשמתו של כל יחיד מישראל בפנימיות\mycircle{°} מהותה, מאור\mycircle{°} זה היא אצולה }\מקור{[עפ״י ע״ר ב קנח]}\צהגדרה{.}

\הגדרה{הכח הנסתר של האצילות\mycircle{°} שבנשמת\hebrewmakaf האומה\mycircle{°} }\מקור{[ע״ה קנב]}\צהגדרה{.}

\משנה{האידיאלים הנשאים }\צהגדרה{-}\הגדרה{ המגמות\mycircle{°} האחרונות של מעשה התורה\mycircle{°} והמצוה\mycircle{°} }\מקור{[א״ק ג שכב]}\צהגדרה{.}

\הגדרה{ע׳ במדור מונחי קבלה ונסתר, תפארת, התפארת האלהית. ע׳ במדור שמות כינויים ותארים אלהיים, אלהות. ע״ע בנספחות, בסוף מדור מחקרים, שני מכתבים לברור דברים בשיטת הרב. ע״ע רוממות האידיאלים. }

\paragraphs

\ערך{אילת השחר}\myfootnote{ תהילים כב א.\label{21}}\הגדרה{ - היופי העולמי שנאזר\mycircle{°} מגבורה וצמצומים }\מקור{[קובץ ה ר]}\צהגדרה{. }

\paragraphs

\ערך{איש }\הגדרה{- התוכן הממולא בכח המפעל וההשפעה, ששפעותיו עשירות הנה בכל הארחים, לטוב\mycircle{°} ולרע\mycircle{°}, לבנין ולסתירה, רק הוא בכללות קיבוץ כל חלקיו, ממלא הוא תוכן של אישיות\mycircle{°}, צורה\mycircle{°} ממולא(ה) בטפוס שלם, העומד הכן לפעול ולהשפיע, לשכלל ולהשלים}\צהגדרה{ }\מקור{[ר״מ קכז]}\צהגדרה{.}

\paragraphs

\ערך{איש }\משנה{- יסוד השלמתו }\הגדרה{- השכל\mycircle{°} העומד בראש והרגש\mycircle{°} עוזר על ידו }\מקור{[ע״א ג ב ריג]}\צהגדרה{.}

\הגדרה{ע״ע אשה. ע״ע אנשים. ע״ע גבר. ע״ע ״אדם״. ע״ע ״אנוש״.}

\paragraphs

\ערך{״איש״ לעומת ״בעל״}\הגדרה{ - }\משנה{בעל }\הגדרה{- יקרא על שם הבעילה, כמו ״האי מטרא בעלא דארעא״}\myfootnote{ תענית ו: \label{22}}\הגדרה{, ו}\צהגדרה{איש}\הגדרה{ - יקרא על שם המשפיע, יען כי הוא הנותן לה לחם לאכול ובגד ללבוש }\מקור{[ע״א יבמות סב:, סי׳ י]}\צהגדרה{.}

\paragraphs

\ערך{אישיות }\הגדרה{- צורה\mycircle{°} ממולא(ה) בטפוס שלם, העומד הכן לפעול ולהשפיע, לשכלל ולהשלים }\מקור{[ר״מ קכז]}\צהגדרה{. }

\paragraphs

\ערך{איתן העולם }\הגדרה{- ע׳ במדור מדרגות והערכות אישיותיות. }

\paragraphs

\ערך{איתנות }\הגדרה{- הקביעות העזיזה }\מקור{[ל״ה 162]}\צהגדרה{. }

\ערך{איתניות ברוח }\הגדרה{- עז\mycircle{°} הרצון הכביר המתגלה באופן בריא וחזק עם כל תנועה נפשית וגופית }\מקור{[ע״א ד ו פג]}\צהגדרה{. }

\paragraphs

\ערך{אך }\הגדרה{- מלת המיעוט\mycircle{°}, הצערת הנושא ממובנו הקדום, קציצת איזה סעיפים מתכונתו, <נרדף הוא עם התוכן של ההכאה הבא בתואר זה> }\מקור{[ר״מ קכא\hebrewmakaf ב]}\צהגדרה{. }

\paragraphs

\ערך{אכילה }\הגדרה{- }\משנה{(לעומת טעימה\mycircle{°})}\הגדרה{ - תתיחס מצד התועלת, התועלת של תכלית האכילה לחזק הגוף ולהמשיך החיים, הבאה אחר העיכול, שע״ז יבא פעל אכל, מאוּכַּל}\צהגדרה{ }\מקור{[עפ״י ע״א ג א לה]}\צהגדרה{.}

\הגדרה{ע״ע מזון, לקיחת מזון. ע״ע מאכל.}

\paragraphs

\ערך{אַל }\הגדרה{- הוראה שלילית }\מקור{[ר״מ קכב]}\צהגדרה{.}

\paragraphs

\ערך{אֵל }\הגדרה{- הוראת הכח }\מקור{[ר״מ קכב]}\צהגדרה{. }

\paragraphs

\ערך{אלהִי }\הגדרה{- }\משנה{הנקודה האלהית }\הגדרה{- מכון השלמות המוחלטה }\מקור{[ע״א ד ט קלד]}\צהגדרה{. }

\הגדרה{ע״ע אמונה, נקודת האמונה. }

\paragraphs

\ערך{אלהִי }\הגדרה{- }\משנה{עידון אלהי }\הגדרה{- (העידון) של החכמה\mycircle{°} והמישרים\mycircle{°}, של הצדק והאורה\mycircle{°} העדינה הרוחנית\mycircle{°} }\מקור{[ע״א ד ו מ]}\צהגדרה{. }

\paragraphs

\ערך{אלהים}\myfootnote{ ע׳ בהערה במדור שמות כינויים ותארים אלהיים, אלהים.\label{23}}\הגדרה{ - מנהיג ושולט }\מקור{[ע״ר א קיד]}\צהגדרה{. }

\הגדרה{כל כח שבנבראים שיש לו איזה השפעה, }\צהגדרה{<שמוכרחת היא להיות מוגבלת, מאחר שהכח המשפיע בעצמו הוא מוגבל, בבחינת התחלתו ובבחינת סופו, וכל מוגבל הרי מדת\hebrewmakaf הדין\mycircle{°} המצומצמת טבועה בתוכו> }\מקור{[עפ״י ע״ר ב פח]}\צהגדרה{.}

\paragraphs

\ערך{אַלף }\הגדרה{- תרגום\mycircle{°} של למד }\מקור{[ר״מ קיז]}\צהגדרה{. }

\הגדרה{תרגום של למוד, <בעברית הוא ג״כ ממקור הארמי> ובא על המדרגה הירודה של הלמוד, הצד המתנוצץ אל התלמיד משפעתו של הרב, כתכונת האחורים לעומת הפנים }\מקור{[שם פג]}\צהגדרה{. }

\הגדרה{הלומד מאחר, והוא בערך העתקה ואחורים של הפנים המאירים באור השכל המקורי. ההתלמדות האולפנית, מדת התרגום, שהוא הכשר וחינוך המביא אל המעמד המקורי העתיד }\מקור{[עפ״י שם קיג]}\צהגדרה{. }

\הגדרה{הלימוד המתורגם, לימוד חינוכי, לימוד מכשיר, <שמביא באחריתו ללמוד מקורי, להבנת הלב> }\מקור{[עפ״י שם קיח]}\צהגדרה{. }

\הגדרה{ע׳ במדור אותיות, למ״ד. }

\paragraphs

\ערך{אלף }\הגדרה{- שור\mycircle{°} }\מקור{[ר״מ פג]}\צהגדרה{. }

\paragraphs

\ערך{אֵם }\הגדרה{- הוראת האמהות }\מקור{[ר״מ קכג]}\צהגדרה{. }

\הגדרה{המשפעת שפע חיים ליצור בהולדו }\מקור{[ע״א ד ו עג]}\צהגדרה{.}

\הגדרה{ע׳ במדור פסוקים ובטויי חז״ל.}

\paragraphs

\ערך{אמונה}\myfootnote{ \textbf{אמונה} - הגדרות האמונה השונות חולקו לשתי מחלקות (שגבולן סומן ב ◊◊) : הקבוצה הראשונה עד ערך ׳תוכן האמונה׳ (ולא עד בכלל), מרכזת הגדרות למושג אמונה בסתם, ומקורותיה. ומערך ׳תוכן האמונה׳ ואילך הובאו הגדרות לבחינותיה השונות של האמונה, ענייניה, תכניה והשלכותיה.\textbf{גדר האמונה בכללה} - אמונות ודעות לרס״ג הקדמת המחבר, ״צריכים לבאר מה היא האמונה, ונאמר כי היא ענין עולה בלב לכל דבר ידוע בתכונה אשר הוא עליה, וכאשר תצא חמאת העיון יקבלנה השכל ויקיפנה ויכניסנה בלבבות ותמזג בהם, ויהיה בהם האדם מאמין בענין אשר הגיע אליו״. ובשומר אמונים הקדמון, ויכוח ראשון סי׳ לג (מיסוד המורה נבוכים, ח״א פ״נ) ״כי האמונה אינה האמירה בפה, כי אם התאמת הדבר במחשבת הלב והצטיירו בשכל״.\label{24}}\הגדרה{ - }\משנה{גדר האמונה בכללה }\הגדרה{- שאמיתת הענין הנודע קבועה בקרבו, לא מצד הידיעה לבדה כ״א מצד מנוחת הנפש הגמורה כשהוא מקבל אותה בקבלה שלמה, מבלי שיסתער בו מאומה נגד זה }\מקור{[ע״ר א שלז]}\צהגדרה{. }

\paragraphs

\ערך{אמונה }\הגדרה{- דת\mycircle{°}}\צהגדרה{ }\מקור{[קבצ׳ ב נג]}\צהגדרה{.}

\הגדרה{ע״ע אמונה, נשמת האמונה}\צהגדרה{.}

\paragraphs

\ערך{אמונה }\הגדרה{- }\משנה{מצות האמונה }\הגדרה{- מצוה מעשית, שיהי׳ הלב נמשך לאמונת התורה\mycircle{°} לעיון העבודה\mycircle{°} והמצות\mycircle{°}. שכל הדברים הברורים המאומתים שהודיענו השי״ת\mycircle{°} בתורתו\mycircle{°}, נראה להשתדל שכשם שהם מאומתים מצד אמתתם בשכל, כן יהיו הרהורי הלב מלאים מהם, והיינו על ידי שירבה המשא ומתן בענפיהם בעיון והשכל איש איש כפי רוחב לבו}\צהגדרה{ }\מקור{[פנק׳ ג כו (מא״ה ב רעג)]}\צהגדרה{.}

\משנה{צורת מצות האמונה האלהית }\הגדרה{- שהאלהים יתעלה, שהוא העושה כל המעשה הגדול הזה, אשר עינינו רואות אותו מסודר בעצה\mycircle{°} ובחכמה גדולה ועמוקה ומדוקדקת מאד, הוא שהוציא אותנו ממצרים\mycircle{°} מבית עבדים ונתן לנו את התורה\mycircle{°}, ולו שייכים ומיוחסים כל הדברים אשר הם שייכים ומיוחסים לאלהים <בין שמתבררים ע״פ אמתת הברור השכלי ובין שמתבררים ע״פ התורה והקבלה השלמה> }\מקור{[עפ״י ע״ר א שלו]}\צהגדרה{.}

\paragraphs

\משנה{אמונה }\צהגדרה{- התכּונות התפיסת אל השכליות האלהית העליונה המוחלטת הכוללת\hebrewmakaf כל, המושכלת ומוכרת ומבוררת בכל מלוא הדעת הלבב והנפש והחיות }\צמקור{[עפ״י שי׳ ה 40]}\הגדרה{. }

\צמשנה{אמת האמונה}\myfootnote{ \textbf{האמונה }- \textbf{הידיעה ההכרה} - מוסיף רבנו שם באגרת: לכן בלשון התורה שבכתב: אברהם אבינו ״האמין בד׳״; ובלשון חז״ל בתושבע״פ: ״הכיר את בוראו״. ובהמשך דברי רבנו שם: ״הכרה ברורה של הסתכלות האמונה הזאת, נמשכת ומתבהרת לנו, בשלשלת הדורות, מתוך זכירת ימות עולם אל בינת שנות דור ודור, המפגישה אותנו עם הופעתנו המיוחדת הגדולה, שאין דומה ודוגמה לה בכל מערכת האנושיות, ושלשלת דורותיה והשפעתנו בתוכה״. וע׳ בפרקי הקיום הכשרון וההשפעה, שבס׳ כארי יתנשא.\label{25}}\צהגדרה{ - הידיעה ההכרה ההבנה הברורה הפשוטה שכל הנמצאים שבעולם כולם נמשכים ומתגלים לנו מתוך מקור המציאות, וכלשונו של הרמב״ם: ׳שכל הנמצאים הם מתוך אמתת מציאותו׳ }\צמקור{[אגרת רבינו, מי״ז בכסלו תשל״ז. מובא בשי׳ ג 241, הערה 89].}

\צמשנה{רוממות\hebrewmakaf רוח\hebrewmakaf אמונה ואמתת\hebrewmakaf דעת\hebrewmakaf עליון }\צהגדרה{- תקף בהירותה של הכרת עולם ומלאו, שלמותו ומקוריותו }\צמקור{[ל״י ב (מהדורת בית אל תשס״ז) תסד].}

\paragraphs

\ערך{אמונה }\הגדרה{- }\משנה{עיקר מגמת הדרכת האמונה }\הגדרה{- לבסס את כח המדמה\mycircle{°} בטהרת\mycircle{°} השכל\mycircle{°} והמוסר\mycircle{°} }\מקור{[פנק׳ ד שסט]}\צהגדרה{.}

\paragraphs

\ערך{אמונה }\הגדרה{- }\משנה{כח האמונה }\הגדרה{- (הכח) להקשיב יפה ולקבל ממה שנמסר }\מקור{[א״א 66]}\צהגדרה{. }

\הגדרה{להיות מוכן לקבל בתורת ידיעה את המסור לנו מאבותינו }\מקור{[שם 94]}\צהגדרה{.}

\הגדרה{הקבלה את מה שנאמר מפי גדולי עולם}\צהגדרה{ }\מקור{[ע״א ב ט רעב]}\צהגדרה{. }

\paragraphs

\ערך{אמונה טבעית הסתכלותית }\הגדרה{- האמונה המופעה מהעולם }\מקור{[מ״ר 70]}\צהגדרה{. }

\ערך{אמונה ניסית מסורתית }\הגדרה{- האמונה המופעה מהתורה }\מקור{[מ״ר 70]}\צהגדרה{. }

\ערך{אמונה תוכית }\הגדרה{- האמונה המופעה ממעמקי הנשמה }\מקור{[מ״ר 70]}\צהגדרה{. }

\הגדרה{ע״ע תשובה טבעית. תשובה אמונית. תשובה שכלית. }

\paragraphs

\ערך{אמונה }\הגדרה{- שלמות המדעים המושכלים שהם שורשי התורה\mycircle{°} }\מקור{[א״ה א 702 (מהדורת תשס״ב ח״ב 30)]}\צהגדרה{. }

\paragraphs

\תערך{אמונה }\הגדרה{- }\תמשנה{רוח\mycircle{°} האמונה הקדושה\mycircle{°} }\הגדרה{- }\תהגדרה{הכרה עצמית\hebrewmakaf פנימית כ״טביעת\hebrewmakaf עין״\mycircle{°}, שהאדם מכיר את העולם מתוך עצמותו }\תמקור{[מ״ר 488]. }

\paragraphs

\תערך{אמונה }\הגדרה{- }\תמשנה{כח האמונה }\הגדרה{- }\תהגדרה{הצלם\hebrewmakaf האלהי\mycircle{°} המאיר מתוך תוכם של ברואיו של הקב״ה\mycircle{°}, והוא עצם מהותה של הנשמה, התשוקה להתדבקות\hebrewmakaf אלהית\mycircle{°} }\תמקור{[עפ״י מ״ר 487]. }

\paragraphs

\תערך{אמונה }\הגדרה{- }\תמשנה{עומק האמונה }\הגדרה{- }\תהגדרה{להתדבק\mycircle{°} בבורא כל העולמים\mycircle{°} ב״ה, בכל סדרי המחשבה והמעשה, והוא גילוי הכח השלם (של האמונה) }\תמקור{[מ״ר 488].}

\הגדרה{ע׳ במדור הכרה והשכלה והפכן, מחשבה, המחשבה היסודית. }

\משנה{אמונה }\צהגדרה{- הידיעה\hebrewmakaf ההכרה הבהירה והמלאה בד׳ אלהים, מקור חיי כל עולמים, הממלאה את כל חדרי הנפש, הרוח והנשמה, הכליות והלב והגוף כולו, ואשר לפיכך הם כולם ממולאים מתוכה אהבה\hebrewmakaf עליונה\mycircle{°} של דבקות\mycircle{°} חיונית לאבינו\hebrewmakaf שבשמים\mycircle{°}, אשר מלך בטרם כל יצור נברא - ואחריו }\צמקור{[נ״ה ט].}

\צהגדרה{ההשכלה וההכרה המבוררת בכל מלוא הדעת לבב ונפש וחיות, בהתכּונות התפיסה אל השכליות האלהית העליונה המוחלטת הכוללת כל }\צמקור{[עפ״י שי׳ ה 40 הערה 63].}

\צהגדרה{ענג רוממות ושבע נפש פנימי, (של) מלא ההכרה הברה והתמה ורווי עז הבטחון\mycircle{°} (ש)על ידי השקפת האחדות העליונה, שהנהגת ההשגחה\mycircle{°} העליונה מסבבת כל המעשים וכולם מלאים הם מזיו\mycircle{°} הטוב הכללי המקיף הכל וכוללם יחד, של ״מה דעבד רחמנא לטב עביד״, ״עושה שלום ובורא הכל״ }\צמקור{[עפ״י א״ל קצה].}

\צמשנה{כח האמונה }\צהגדרה{- ההכרה העליונה והדבקה באמתת צדיקו של עולם, עז מלכותו ומקור חיותו, אשר מראשית דרכו מאז ועד אחרית מופיעה המשכת כל המפעלים ושעשועי התולדה }\צמקור{[נ״ה יט].}

\צהגדרה{מציאות נפשית פנימית של זיקת האדם, הנברא בצלם\hebrewmakaf אלהים\mycircle{°}, אל מקורו, יוצרו, רבון\hebrewmakaf העולמים, שהיא ממלאה אותו כולו ומתוך\hebrewmakaf כך מתגלית בסידורי מעשיו ודורותיו היחידיים והצבוריים }\צמקור{[עפ״י ל״י ג קעז (מהדורת בית אל ב תשס״ג תה)].}

\צהגדרה{כִּווּנָה העצמי (שלא מתוך שטחי עניינים אחרים, של רגשיות או מוסריות או שכליות, של תועלתיות או חברתיות או הגיוניות) של תפיסת\hebrewmakaf עולם\hebrewmakaf ואדם שלמה וכוללת, בטבעיותו הרוחנית\mycircle{°} והחיונית }\צמקור{[עפ״י ל״י ב (מהדורת בית אל תשס״ג) תז].}

\paragraphs

\ערך{אמונה }\הגדרה{- ע״ע בקשת אלהים. ע׳ במדור פסוקים ובטויי חז״ל, דרישת ד׳. }

\paragraphs

\ערך{אמונה }\הגדרה{- ע׳ במדור הכרה והשכלה והפכן, הכרה אלהית. ע׳ במדור פסוקים ובטויי חז״ל, דעת אלהים.}

\paragraphs

\ערך{אמונה }\הגדרה{- }\משנה{כח האמונה }\הגדרה{- שלמות תמימות טבעו הרוחני של האדם }\מקור{[עפ״י ע״א ג ב קנג]}\צהגדרה{.}

\הגדרה{בבחי׳ נפש, הרגשה, כח המקבל שבהוי׳ }\מקור{[קבצ׳ ג קכג]}\צהגדרה{.}

\ערך{אמונה }\הגדרה{- }\מעוין{◊}\הגדרה{ היסוד הטבעי לכל טובה\mycircle{°} ולכל מוסר\mycircle{°} עליון\mycircle{°} הנעוץ בעומק הטבע הישר\mycircle{°}, האנושי. מקור חייו הטבעיים, הרוחניים\mycircle{°} (של האדם) הנותנים לו שלות לב ושמחת נפש בעוה״ז ואחרית ותקוה לעוה״ב\mycircle{°} }\מקור{[עפ״י מ״ר 225]}\צהגדרה{.}

\הגדרה{תוכן הנפש היותר עדין, והמקור לכל התרבות האידיאלית שהאנושיות כולה עורגת אליה}\צהגדרה{ }\מקור{[פנק׳ ב קצד]}\צהגדרה{. }

\paragraphs

\ערך{אמונה }\הגדרה{- }\משנה{הנטיה האמונית }\הגדרה{- מקור הקדושה\mycircle{°} בעולם כולו }\מקור{[קבצ׳ א נד]}\צהגדרה{.}

\ערך{אמונה }\הגדרה{- }\משנה{נשמת האמונה }\הגדרה{- אור\hebrewmakaf החיים\mycircle{°} האלהיים שבתוך הדת\mycircle{°} }\מקור{[עפ״י קבצ׳ ב נג]}\צהגדרה{.}

\paragraphs

\ערך{אמונה }\הגדרה{- }\משנה{עיקר האמונה }\הגדרה{- }\מעוין{◊}\הגדרה{ עיקר האמונה היא בגדולת\mycircle{°} שלמות אין\hebrewmakaf סוף\mycircle{°}. שכל מה שנכנס בתוך הלב הרי זה ניצוץ בטל לגמרי לגבי מה שראוי להיות משוער, ומה שראוי להיות משוער אינו עולה כלל בסוג של ביטול לגבי מה שהוא באמת }\מקור{[א׳ קכד]}\צהגדרה{. }

\משנה{שורש האמונה }\הגדרה{- להביע בפנימיות הנשמה את גדולת\hebrewmakaf א״ס }\מקור{[מ״ה אמונה לד]}\צהגדרה{. }

\הגדרה{מתוך החכמה\mycircle{°} הכמוסה בנשמה\mycircle{°}, שעל ידה היא מכירה בגודל האמת\hebrewmakaf האלהית\mycircle{°}, אלא שאינו יכול להוציא אל הפועל את הפרטים, מתוך כך הוא מתקשר הרבה להאמין בהפרטים שהוא מרגיש בפנימיות לבבו שהם הם מגלים את האמת הגדולה בכל האופנים שדרכה להגלות - במעשה, דיבור ומחשבה, ברגש, בדמיון, במזג ובנטיות נפשיות, ובמעוף חיים פנימי ועליון מכל הגיון לבב ומכל הקשבה מוגבלת}\צהגדרה{ }\מקור{[עפ״י קובץ א תתמט]}\צהגדרה{.}

\תמשנה{נקודת האמונה }\הגדרה{-}\תהגדרה{ יחס האדם להאין\hebrewmakaf סוף ברוך הוא, וכל העולמים כולם מתרכזים בנקודה אין\hebrewmakaf סופית זו }\תמקור{[מ״ר 487]. }

\הגדרה{ע״ע אלהי, הנקודה האלהית. ע׳ במדור מונחי קבלה ונסתר, לאשתאבא בגופא דמלכא.}

\paragraphs

\ערך{אמונה }\הגדרה{- }\משנה{חוש האמונה }\הגדרה{- החוש הטבעי היסודי של הנשמה, שנובע באדם מנשמת חֵי\hebrewmakaf העולמים\mycircle{°} מנשמת כל היקום, כל היש }\מקור{[עפ״י א״א 112]}\צהגדרה{. }

\מעוין{◊ }\צמשנה{היחש של האמונה בטהרתו}\צהגדרה{ בא מפני עצמיות הטבע הפנימי של הנפש האנושית שהיא קשורה בקשר הוייתה בשורש חיי כל החיים, במקור כל ההויה }\צמקור{[ע״א ד ט קכה]. }

\paragraphs

\ערך{אמונה}\myfootnote{ בע״א ד יא יג מציין הרב שתי בחינות באמונה. סגולית נשמתית - ״\textbf{האמונה }\textbf{האלהית}\textbf{ העליונה} שאינה מתדמה כלל לשום רישום של ידיעה והכרה בעולם, כי היא הסגולה של כל יסוד החיים, של אורם, של חיי חייהם, של זיום ותפארתם. והסגולה הזאת מצד ערכה הפנימי, שאין לו שום הערכה בשום צד המתדמה לו בתואר חצוני, הוא ענין סגולי בישראל, לא מצד בחירת נפשם בפרט אלא מצד מחצב הקדושה וסגולת ירושת אבות שלהם״. ולאידך גיסא, הכרתית חיצונית - ״\textbf{כח}\textbf{ האמונה מצד התגלותה }\textbf{החצונית}, שאפשר לו להגלות בפועל, בהכרה, ברגש, בביטוי ובמעשה, ששם אין האור הגנוז של שלמות חיי האמונה זורח״. בהתאם חולקו ההגדרות כאן לשתי קבוצות. \label{26}}\הגדרה{ - }\משנה{(בד׳) }\הגדרה{- שירת\hebrewmakaf החיים\mycircle{°}, שירת\mycircle{°} המציאות, שירת ההויה. שירת העולם העליונה\mycircle{°}}\myfootnote{ \textbf{האמונה היא שירת העולם העליונה} - ע׳ זוהר תרומה קלט: ״מ״י, רזא דעלמא עלאה איהו דהא מתמן נפקא שירותא לאתגליא רזא דמהימנותא״.\label{27}}\הגדרה{. ומקורה הוא הטבע האלהי\mycircle{°} שבעומק הנשמה\mycircle{°}, העונג\mycircle{°} של ההסתכלות הפנימית\mycircle{°} של אושר\mycircle{°} אין סוף }\צהגדרה{[א״א }\צמקור{66, 123}\צהגדרה{]. }

\הגדרה{שירת חיינו}\myfootnote{ \textbf{שירת חיינו} - ע״ע הסברת והגדרת האמונה בס׳ כארי יתנשא, פרק שלישי, האלהות, עמ׳ 30-32.\label{28}}\הגדרה{, (ה)מכון ששם האמת העליונה שוכנת בכל יפעתה\mycircle{°} והדרה\mycircle{°} }\מקור{[קבצ׳ ב קיב]}\צהגדרה{.}

\הגדרה{האמת הגדולה, התוכן העזיז, מלא הקודש, משך החיים האמתיים, וכל שיגוב ואידור במילואו }\מקור{[א״ק א עא]}\צהגדרה{.}

\משנה{האמונה העליונה }\הגדרה{- שירת העולם ואמת העולם }\מקור{[א׳ קכז]}\צהגדרה{. }

\משנה{הארת האמונה }\הגדרה{- הארה\mycircle{°} כללית למעלה מכל הערכים ובזה היא מבססת את הערכים כולם }\מקור{[שם קכה (א״א 68)]}\צהגדרה{. }

\משנה{אור האמונה }\הגדרה{- שורש כל הקדושות }\מקור{[א״א 141]}\צהגדרה{. }

\משנה{האמונה האלהית }\הגדרה{- המחשבה\mycircle{°} היותר עליונה\mycircle{°}, שמתוך גבהה היא משתפלת בכל השדרות גם היותר שפלות, ובכל דרגה ושדרה היא מתארת כפי ערכה }\צהגדרה{[מ״ה אמונה ג (א״א }\צמקור{7\hebrewmakaf 66}\צהגדרה{)]. }

\משנה{יסוד האמונה }\הגדרה{- התוכן המקיים את הצביון העליון הבלתי מדוד ושקול, הבלתי מצומצם ומתואר, בתוך כל מה שמתואר ומוגבל, המחיה את כל החוקים, בהתמשכם מראשית\mycircle{°} נביעתם עד אחרית\mycircle{°} המגמות\mycircle{°} כולן, עד אין קץ למורד }\מקור{[ע״ר א לה\hebrewmakaf ו]}\צהגדרה{.}

\משנה{שרש האמונה בטהרתה\mycircle{°}}\הגדרה{ - חיבור הקדושה הרוממה, של אור\hebrewmakaf אין\hebrewmakaf סוף\mycircle{°}, עם הקדושה החודרת בכל העולמים ובכל היצורים כולם }\מקור{[עפ״י ע״ר א קיד]}\צהגדרה{.}

\תמשנה{תוכן האמונה }\הגדרה{-}\תהגדרה{ האידיאליות\mycircle{°} הנשמתית\mycircle{°}, שהיא התשוקה העליונה והרוממה להתדבקות\hebrewmakaf אלהית\mycircle{°}, שתנאיה הם השואה גמורה ומוחלטת בין המעשה והמחשבה\hebrewmakaf האלהית\mycircle{°} }\תמקור{[מ״ר 494]. }

\משנה{אמונה, יסודה העצמי }\הגדרה{- נקודת הציור\mycircle{°} האלהי ברקמת החיים הפנימית (של) היחיד או הצבור, המשפחה או האומה\mycircle{°}, המפלגה או הסיעה, העושה את הרקמה הנפשית\mycircle{°} היותר חטיבית, יותר איתנה ויותר חודרת }\מקור{[עפ״י א״א 78 (קובץ ז עו)]}\צהגדרה{. }

\משנה{האמונה האלהית }\הגדרה{- הסגולה\mycircle{°} של כל יסוד החיים\mycircle{°}, של אורם\mycircle{°}, של חיי חייהם, של זיום\mycircle{°} ותפארתם\mycircle{°} }\מקור{[ע״א ד יא יג]}\צהגדרה{. }

\משנה{נקודת קדושת האמונה האלהית }\הגדרה{- יסוד יראת\hebrewmakaf ד׳\mycircle{°} באמת }\מקור{[א״א 95]}\צהגדרה{. }

\משנה{אמונה }\הגדרה{- היראה\mycircle{°}, התוכן האצילי\mycircle{°}, המקודש\mycircle{°}, האלהי, של האומה }\מקור{[א״ש טו יא]}\צהגדרה{. }

\הגדרה{היחס האמיתי\mycircle{°} הפנימי אל יסוד המציאות במקורו, המשיב לנשמה את צביונה }\צהגדרה{[עפ״י א״א 3\hebrewmakaf }\צמקור{82 (}\צהגדרה{מ״ר 74)]. }

\הגדרה{תהילת\hebrewmakaf ד׳\mycircle{°} וסקירת השלמות העליונה }\מקור{[א״ק ד תיא (קובץ ז קמג)]}\צהגדרה{.}

\משנה{העומק התמציתי של מהות האמונה }\הגדרה{- יסוד העלוי\mycircle{°} הנשמתי היותר בהיר של האדם }\מקור{[א״א 58]}\צהגדרה{. }

\משנה{מהות האמונה }\הגדרה{- התעמקות חיה בהענינים האלהיים }\מקור{[א״א 68 (קובץ א שלה)]}\צהגדרה{. }

\משנה{רז האמונה }\הגדרה{- אור האלהי שבנשמה, המבקש את הרוח\hebrewmakaf הטהור\mycircle{°}, את הקודש הנשגב בחיים, בהרגשה ובידיעה }\מקור{[קובץ ח ע]}\צהגדרה{. }

\מעוין{◊}\הגדרה{ האמונה והאהבה\mycircle{°} הן עצם החיים בעוה״ז ובעוה״ב\mycircle{°} }\מקור{[עפ״י א׳ סט]}\צהגדרה{. }

\משנה{האמונה }\הגדרה{- אינה לא שכל\mycircle{°} ולא רגש\mycircle{°}, אלא גלוי עצמי היותר יסודי של מהות הנשמה, <שצריך להדריך אותה בתכונתה, וכשאין משחיתים את דרכה הטבעי לה, איננה צריכה לשום תוכן אחר לסעדה, אלא היא מוצאה בעצמה את הכל> }\מקור{[מ״ר 70]}\צהגדרה{. }

\משנה{אמונה }\הגדרה{- }\משנה{הגיונה העליון }\הגדרה{- הגילוי האלהי שבנשמה שלמעלה מכל דעת\mycircle{°} }\מקור{[ע״ט יח (א״א 56)]}\צהגדרה{. }

\משנה{חוש האמונה האלהית }\הגדרה{- דעת הדעות והרגשת ההרגשות, שמחברת את ההויה הרוחנית\mycircle{°} של האדם, המציאותית בפועל, עם ההויה הרוחנית העליונה, ומערבת את החיים שלו עם החיים המציאותיים הרמים מכל גבול ונעדרי כל חולשה פיסית }\מקור{[מ״ר 1]}\צהגדרה{.}

\הגדרה{הרגש של הכלל, (ש)הכל אצור ב}\צהגדרה{ו <בעומק הרגש מונח הכל - כל הפרטים החבויים בגנזי השכל העליון, אשר רק חלקים ממנו הולכים ומתגלים על פי ארחות החיפוש והמחקר, הרגש הוא תופש בהם שלא בהדרגה, איננו צריך לעזר ההתחכמות בשביל להכיר את טובם> }\מקור{[קבצ׳ ב פח (פנק׳ ד סח)]}\צהגדרה{. }

\הגדרה{ע׳ בנספחות, מדור מחקרים, בטחון לעומת אמונה.}

\paragraphs

\ערך{אמונה}\footref{26}\הגדרה{ - המסקנה היותר מזהירה של הלימודים היותר רמים\mycircle{°} ונשאים המוארים באור\mycircle{°} עליון\mycircle{°} }\מקור{[א״י נ]}\צהגדרה{.}

\מעוין{◊}\הגדרה{ האמונה מקפת את כל הידיעות לעשות חטיבה כללית מכל הפרטים כולם, ובזה היא נותנת חיים נצחיים לכל הזוכים לאורה\mycircle{°}, והיא מחיה בלשד פנימיותה את המוסר\mycircle{°}, את חיי החברה ואת חיי היחיד, כשם שהיא מחיה את כל העולמים\mycircle{°} כולם, מראש\mycircle{°} ועד סוף }\מקור{[א״ק ג קז (א״א 68)]}\צהגדרה{. }

\מעויןמרכזי{◊◊}

\ערך{אמונה}\footref{24}\הגדרה{ }\myfootnote{ צריך להבחין בסדרת הגדרות האמונה הבאות (שגבולן סומן ב ◊) בין האמונה בבחינתה כממד אונטי, לבין האמונה כממד פסיכי; ובהגדרות המשתפות ביניהם.\label{29}}\הגדרה{ - }\משנה{תוכן האמונה }\הגדרה{- היסוד הקדמון לכל היש, החובק כל המציאות והממלא אותה עצמת הקיום וההמשכה ההוייתית. שורה היא האמונה בכל הנברא ונוצר ונעשה, שורה היא ברומי מרומים, משתפלת היא בשפלי שפלים. וקדמותה\mycircle{°} העליונה של אור האמונה וצחצחות טבעיותה האיתן, זהו המכריח את כל החיים האנושיים להיות מוטבעים על פיה }\מקור{[ע״א ד ט קכה]}\צהגדרה{. }

\מעוין{◊ }\תמשנה{תוכן האמונה}\תהגדרה{ אינו רעיון כ״א מציאות גמורה, הנמצאת בכל חלקי היצירה, גם בדומם, והוא סוד קיומו של עולם }\תמקור{[מ״ר 489].}

\תמשנה{כח האמונה }\תהגדרה{הוא כח כללי ולא פרטי, אור מקיף הנובע ממקור ההויה כולה ומתפשט על הכל }\תמקור{[מ״ר 489]. }

\תערך{אמונה }\הגדרה{-}\תהגדרה{ עריגת כל העולמים\mycircle{°} כולם לחי\hebrewmakaf העולמים\mycircle{°} }\תמקור{[מ״ר 494]. }

\paragraphs

\ערך{אמונה }\הגדרה{- }\משנה{האמונה הרבה}\myfootnote{ \textbf{האמונה הרבה} - ע׳ י׳ מאמרות לרמ״ע, אכ״ח ח״ג יא, ביד יהודה ס״ק יד.\label{30}}\הגדרה{ - האמונה\hebrewmakaf האלהית\mycircle{°} (ה)חוקקת את התפקידים של כל כוחות ההויה, שיעשו את עבודתם בכל עז\mycircle{°} ומרץ, בכל איתניות של קיום ונצחון, שהכח הכביר של אור האמונה האלהית מאיר עליהם, למסור בידם תוכנים וענינים מתוקנים עומדים על מכונם, ולקלקל ולהפסיד את צורתם וערכם, כדי להוציא ע״י קלקול זה ערכים יותר נפלאים בחדוש פנים לעילוי\mycircle{°} ולשבח. מהסתעפות\mycircle{°} }\משנה{האמונה הרבה}\הגדרה{, הכוללת כל היקום, מתגלה הדר\mycircle{°} כח המחיה והמפריא, המעודד והמחדש, בכל תוכן שיוצאת ממנה ערות של חיים, אחרי שכבר צללי המות\mycircle{°} באו ופרשו עליו את אפלתם. האדם, כל כוחות החיים וההויה, מתעוררים אחרי בלותם, בכח אור\mycircle{°} האמונה\hebrewmakaf האלהית\hebrewmakaf העליונה, מיסוד אמונת\hebrewmakaf אומן\mycircle{°} של חכמת\mycircle{°} ודעת\mycircle{°}, אשר באמונת העתות, בכל מסיביהן ותמורותיהן }\מקור{[עפ״י ע״ר א ג]}\צהגדרה{. }

\הגדרה{האמונה המסדרת כוחות ההויה }\מקור{[עפ״י רצי״ה שם ב תלט]}\צהגדרה{. }

\הגדרה{ע׳ במדור תורה, ״אמון רבתא״. }

\paragraphs

\ערך{האמונה הגדולה }\הגדרה{- אמונת עולמים. האמונה הגדולה השרויה בעומק האהבה, הפועלת ברוח חיים בשכל\hebrewmakaf עליון\mycircle{°} ומפואר, בסדר והתאמה בכללות המון היצורים והעולמים\mycircle{°} כולם. האמונה הפנימית, היודעת את כבודה, את אשרה וגבורתה המדושנת עונג פנימי, המכירה שהיא בכל מושלת, שהיא מחלקת במדה ובמשקל של צדק\mycircle{°} ויושר\mycircle{°}, אור\mycircle{°} וחיים\mycircle{°}, לכל המון יצורים לאין תכלית, על פי סדר וערך של יחושים נאמנים, ערוכים ברוח שלום\mycircle{°} ואמת\mycircle{°} }\מקור{[עפ״י א׳ מב\hebrewmakaf מג]}\צהגדרה{. }

\paragraphs

\ערך{אמונה }\הגדרה{- }\משנה{״אֹמֶן״}\myfootnote{ \textbf{אומן} - עפ״י הקדמת ר״י הארוך בר קלונימוס האשכנזי (המיוחסת לראב״ד) לס״י, הנתיב הג׳.\label{31}}\הגדרה{ - החפש\mycircle{°}, אב\mycircle{°} האמונה\hebrewmakaf העליונה\mycircle{°}, השכל המקודש שהוא יסוד החכמה\mycircle{°} הקדומה, שמכוחו האמונה נאצלת }\מקור{[עפ״י א״א 128, שם 17, (77)]}\צהגדרה{. }

\הגדרה{ע׳ במדור שמות כינויים ותארים אלהיים, ״אהיה אשר אהיה״. ע׳ במדור הכרה והשכלה והפכן, השכלה העליונה. ע׳ במדור פסוקים ובטויי חז״ל, אמונות אמיתיות.}

\paragraphs

\ערך{אמונה }\הגדרה{- }\משנה{״אמונת אומן\mycircle{°}״}\myfootnote{ \textbf{אמונת אומן} - עפ״י ישעיה כה א.\label{32}}\הגדרה{ - האמונה\mycircle{°} העליונה\mycircle{°} שגורל\mycircle{°} העתיד בידה הוא. המתק הגדול של הנועם\hebrewmakaf העליון\mycircle{°} אשר לאור עולם שלעתיד, נועם ד׳ המתבקש בכל שכלול של קדושה\mycircle{°} }\צהגדרה{[א״א }\צמקור{128, 80}\צהגדרה{]. }

\הגדרה{הנקודה הגורלית העליונה של כל האדם וכל היצור, בה מתגלה הערך היותר עצמי וטפוסי, היותר פנימי ומכוון אל הטוהר המהותי של האדם, שבה ספיגת כל החיים כולם, כל הזיו\mycircle{°} והזוהר\mycircle{°} שבעולם, כל השלום\mycircle{°} והאושר\mycircle{°} שבעולם, כל החיל\mycircle{°} והחוסן\mycircle{°} שבעולם, השושנה\hebrewmakaf העליונה\mycircle{°} חבצלת השרון }\מקור{[עפ״י שם 133]}\צהגדרה{. }

\הגדרה{האמונה הרוממה המושכלת, מעמק האמונה בטהרתה\mycircle{°} הפנימית\mycircle{°}, שהמחשבות הנובעות ממנה הן החפשיות\mycircle{°} באמת, שאין עליהן שום עול של הגבלה, והן המתדמות ביותר למקור היצירה האלהית, ״ואהיה\mycircle{°} אצלו אמון״. מקור כל השמחות והשעשועים\mycircle{°} האציליים, ״משחקת לפניו בכל עת ושעשועי את בני אדם״ }\צהגדרה{[עפ״י שם }\צמקור{55, 128, }\צהגדרה{80]. }

\הגדרה{האמונה\hebrewmakaf האלהית\mycircle{°} העליונה, שבישראל, מצד ערכה הפנימי, שאין לו שום הערכה בשום צד המתדמה לו בתואר חיצוני, הוא ענין סגולי\mycircle{°}, לא מצד בחירת נפשם בפרט אלא מצד הקדושה וסגולת ירושת אבות\mycircle{°} }\מקור{[עפ״י ע״א ד יא יג]}\צהגדרה{. }

\הגדרה{האמונה\hebrewmakaf האמיתית\mycircle{°}. האמונה העליונה }\מקור{[קובץ א קס, תקפט]}\צהגדרה{.}

\הגדרה{האמונה המחי׳ את בני׳ באור\hebrewmakaf ד׳\mycircle{°} ובשמירת התורה\mycircle{°} והמצוה\mycircle{°} באמת ובתמים }\מקור{[אג׳ א קטו]}\צהגדרה{.}

\משנה{האמונה הגדולה }\הגדרה{- הציור\mycircle{°} העליון\mycircle{°} של ההשכלה הרוחנית\mycircle{°} (ש)איננו מצטמצם בשכל הגיוני\mycircle{°}, (שאורו) הוא זיו\mycircle{°} החיים החזק; חוש\hebrewmakaf האמונה\hebrewmakaf האלקי\mycircle{°}, בכל גודל עזוזו\mycircle{°}, זהו החיים האמיתיים\mycircle{°}, החיים ששום מות\mycircle{°} אין עמם, החיים ששמחתם איננה נחלשת משום צרה יגון ואנחה, החיים שהטובה והברכה\mycircle{°} שרויים בם, עמם דבקים שמחת\mycircle{°} עולמים בלא עצב, חדוה\mycircle{°} עליונה רחבה ועשירה, בלא שום דלדול ורפיון }\מקור{[עפ״י א״א 131 (מ״ר 75)]}\צהגדרה{. }

\הגדרה{הכח הרוחני שבעומק קדושת הנשמה, רוח החיים הפועם במלוא הנשמה, רוח אלהים שבלב, החי בתוכיותה של הנשמה }\מקור{[עפ״י א״ת ב 105]}\צהגדרה{.}

\paragraphs

\ערך{אמונה }\הגדרה{- }\משנה{היסוד העליון של האמונה }\הגדרה{- הארת\mycircle{°} הקודש\mycircle{°} בצורה העליונה שלמעלה מן הטבע\mycircle{°}, המושלת על הטבע ומשכללתו }\מקור{[א״א 128]}\צהגדרה{. }

\הגדרה{האמונה\hebrewmakaf העליונה, האמונה\hebrewmakaf הגדולה, אמונת\hebrewmakaf אומן }\צהגדרה{[עפ״י שם }\צמקור{131, 133}\צהגדרה{]. }

\הגדרה{ע״ע אמונה, האמונה הרבה. ע׳ במדור הכרה והשכלה והפכן, השכלה העליונה.}

\מעויןמרכזי{◊}

\ערך{אמונה }\הגדרה{- }\משנה{היסוד הטבעי של האמונה }\הגדרה{- עריגת הקודש שבנפש האדם }\מקור{[א״א 128]}\צהגדרה{. }

\משנה{טבע האמונה\hebrewmakaf האלהית\mycircle{°}}\הגדרה{ - עומק הטבע הנפשי, תאות הדבקות\hebrewmakaf האלהית\mycircle{°} ברעיון ובחפץ פנימי, (ש)היא תאוה וחמדה עליונה, חזקה מכל התאוות שבעולם }\מקור{[שם 116]}\צהגדרה{. }

\משנה{האמונה הטבעית }\הגדרה{- עריגת הקודש, החשק הפנימי, של הדבקות\hebrewmakaf האלהית }\מקור{[עפ״י שם 108 (קובץ ז קמא)]}\צהגדרה{. }

\משנה{הארת האמונה הטבעית }\הגדרה{- אור אלהים המפעם בנשמה בכחו הגדול מצד עצמו <חוץ ממה שהוא מואר באורה של תורה, של מורשת אבות וקבלה> }\מקור{[א״א 107 (קובץ ז פ)]}\צהגדרה{. }

\מעוין{◊ }\משנה{שני צדדים בטבע האמונה: צד החסד שלה }\הגדרה{- התוך הרוחני, הגרעין האידיאלי, שבאמונה, הצורה השכלית של החובה והמצוה האלהית העליונה שבה, הצורה הטבעית של האמונה, <בצורה הטבעית של הקודש שבכנסת\hebrewmakaf ישראל\mycircle{°} חקוקים הם כל תוכני הצורות של המצות כולן וסעיפיהן, כל התורה כולה>; ו}\משנה{צד הגבורה שבה }\הגדרה{- ההמשכה הטבעית, התאוה הנפשית הפנימית, החומר הרוחני של האמונה, עריגת הקודש של כל העולם, שאפשר לה להיות בישרי לב שבכל האדם כולו }\צהגדרה{[עפ״י שם }\צמקור{128, 117}\צהגדרה{]. }

\הגדרה{ע׳ במדור מועדים וחגים, פסח, יסוד חג הפסח. ושם, סוכות, יסוד חג הסוכות. ע״ע יהדות טבעית, ע״ע יהדות ניסית. ע׳ במדור פסוקים ובטויי חז״ל, ״עמוסי בטן״. ע׳ במדור אליליות ודתות, קליפת האגוז. ושם, ״יצרא דעבודה זרה״. ע״ע רליגיוזיות, הטבעיות הרליגיוזית. }

\paragraphs

\ערך{אמונה }\הגדרה{- }\משנה{עבודת האמונה }\הגדרה{- שכלול האמונה ורעיית\hebrewmakaf האמונה}\myfootnote{ \textbf{רעיית האמונה} - עפ״י תהילים לז ג.\label{33}}\הגדרה{, במעשים טובים ובמדות טובות בתורה\mycircle{°} וחכמה\mycircle{°} עליונה }\צהגדרה{[עפ״י א״א }\צמקור{76, 77}\צהגדרה{]. }

\paragraphs

\ערך{אמונה }\הגדרה{- }\משנה{עיקרי אמונה }\הגדרה{- ע״ע עקרים. }

\paragraphs

\ערך{אמונה }\הגדרה{- }\משנה{רעיית האמונה }\הגדרה{- ע״ע אמונה, עבודת האמונה. }

\paragraphs

\תערך{אמונה }\הגדרה{- }\תמשנה{מגמת האמונה הטהורה }\הגדרה{- }\תהגדרה{השואת המעשה לתוכן המחשבה והבאת הרמוניה שלמה ביניהם }\תמקור{[מ״ר 493]. }

\paragraphs

\ערך{אמונה }\הגדרה{- }\משנה{״אמונות אמיתיות״}\הגדרה{ - ע׳ במדור פסוקים ובטויי חז״ל, אמונות אמיתיות.}

\paragraphs

\ערך{אמונה }\הגדרה{- }\משנה{״אמונות מוכרחות״ }\הגדרה{- ע׳ במדור פסוקים ובטויי חז״ל, אמונות מוכרחות.}

\paragraphs

\ערך{״אמונה באמונה״ }\הגדרה{- ר׳ ״ללמוד אמונה באמונה״.}

\paragraphs

\ערך{אמונה בברית\mycircle{°}}\הגדרה{ - הכונניות\mycircle{°} המונחות ביצירת הרגש, שכשיתפתח עפ״י מדתו יהיה תמיד מתאים אל התכונה הרוממה של הדעות\mycircle{°} הטהורות\mycircle{°} }\מקור{[ע״ר א שפה]}\צהגדרה{.}

\הגדרה{ע׳ במדור פסוקים ובטויי חז״ל, זכירת הברית. }

\paragraphs

\ערך{אמונה עליונה }\הגדרה{- ע׳ במדור פסוקים ובטויי חז״ל, אמונות אמיתיות. }

\paragraphs

\ערך{אמונה שלמה}\הגדרה{ - אמונה ביכולת ד׳ ובבריאת העולם יש מאין }\מקור{[פנק׳ א תי]}\צהגדרה{.}

\paragraphs

\ערך{אמונה תחתונה }\הגדרה{- ע׳ במדור פסוקים ובטויי חז״ל, אמונות מוכרחות. }

\paragraphs

\משנה{אמונת עתנו }\צהגדרה{- האמונה המתגלה מתוך כל מקוריותה ושלמותה של רציפות הדורות, בכל תקפה בממשות ההופעה האלהית של עתנו זאת }\צמקור{[ל״י ג קיז (מהדורת בית אל תשס״ג, ב תז)].}

\צהגדרה{ראיית יד\hebrewmakaf ד׳\mycircle{°} אלהי\hebrewmakaf ישראל\mycircle{°} קורא דורותינו אלה ומכונן פעליהם, המתגלה בהם, בחוסן ישועותינו, בתקומת עמו ונחלתו וחזרת שכינתו, בקיבוץ נדחיו וכינוסם ובבנין בית חייו }\צמקור{[ל״י א ריח\hebrewmakaf ט]. }

\paragraphs

\ערך{אמירה }\הגדרה{- ההופעה המלולית, הבאה מסגולת כח הביטוי שבאדם, המתחלת מראשית הציור המחשבתי, שהוא מצויר אצל בעל המבטא. ומפני שראשית היסוד של צמיחת ההבטאה באה מתוך הרעיון\mycircle{°}, נקרא זה בשם }\משנה{אמירה}\הגדרה{; והמקור הוא אמור, כמו ראש אמיר, כלומר התגלות הדבור ביחסו להמדבר בעצמו, טרם שבא להתגלם בצורה המשפעת כבר על השומע }\מקור{[עפ״י ע״ר ב נד]}\צהגדרה{. }

\הגדרה{ע״ע אֹמר. ע״ע דבור, ע״ע קול. ע׳ בנספחות, מדור מחקרים, ״אֹמֶר״ לעומת ״דבור״. }

\paragraphs

\ערך{אמיתי }\הגדרה{- פנימי\mycircle{°} }\מקור{[א״א 82]}\צהגדרה{. }

\הגדרה{כללי\mycircle{°} }\מקור{[ע״א ג ב קסו]}\צהגדרה{. }

\הגדרה{ע״ע אמת, באמת. ע״ע אמת, להיות חפץ באמת. }

\paragraphs

\ערך{אמיתיות מקובלות}\הגדרה{ - דברים רבים, שלבד אמתתם ההגיונית הנשענת בדרך כללי מבירורה של תורה\mycircle{°} הגלוי והמבורר, ״אתה הראת לדעת כי ד׳ הוא האלהים״, עוד הם נתמכים ביסוד ההכרה הפנימית הכללית שבאומה}\myfootnote{ \textbf{ההכרה הפנימית הכללית שבאומה} - ע״ע במדור תורה, תורה שבעל פה, יסודה של תושבע״פ השמירה של קבלת אבותינו מה שהתנחלו באומה, ובהערה שם.\label{34}}\הגדרה{, שהעבר וההוה שלה מרוכס ומקושר בקשורים אמיצים גלויים ומוחשים בהכרה גלויה ואמתית לכל מי שהולך בדרכיה הטבעיים לה }\מקור{[ע״א ג ב קז]}\צהגדרה{.}

\הגדרה{ע׳ במדור פסוקים ובטויי חז״ל, הגוי כולו, קבלת ״הגוי כולו״. וע׳ בנספחות, מדור מחקרים, ודאות באמיתותה של התורה.}

\paragraphs

\ערך{אֹֽמֶר}\הגדרה{ - הענף העליון של הבטוי. הבטוי הפנימי מצד המבטא }\מקור{[ע״ר ב נד]}\צהגדרה{.}

\הגדרה{ע״ע אמירה. ע״ע דבור. ע״ע קול.}

\paragraphs

\ערך{אֹֽמֶר }\הגדרה{- }\משנה{אֹמֶר ההויה כולה }\הגדרה{- רוחה הפנימי של ההויה בסודה הכלול בקרבה, באותו המצב הראוי להתגלות ליצורים בעלי שכל והרגשה, המכירים כבר ציורים שכלים }\מקור{[ע״ר ב נד]}\צהגדרה{. }

\הגדרה{ע״ע קול, קול ההויה כולה. ע״ע דבור, דבור ההויה הכללית. }

\paragraphs

\ערך{אמת }\הגדרה{- השלמות המוחלטת של האלהות }\מקור{[ע״א ד ח לה]}\צהגדרה{. }

\משנה{אור האמת }\הגדרה{- ברק העצם\mycircle{°} ממקור\hebrewmakaf המקורות\mycircle{°} עדי\hebrewmakaf עד\mycircle{°} }\מקור{[ע״ה קנה]}\צהגדרה{. }

\משנה{האמת כשהיא לעצמה }\הגדרה{- העליוניות\mycircle{°} המוחלטה }\מקור{[א״ק ג ל]}\צהגדרה{.}

\ערך{אמת }\הגדרה{- }\משנה{״אמת ד׳ לעולם״ }\הגדרה{- ההנהגה העליונה\mycircle{°}, שבמדת\hebrewmakaf הדין\mycircle{°}, קשר הקדש\hebrewmakaf העליון\mycircle{°} }\מקור{[עפ״י ע״ר א ריג]}\צהגדרה{. }

\הגדרה{מדת הדין כמו שהיה קודם בריאת\mycircle{°} העולם }\מקור{[מא״ה א קלו]}\צהגדרה{. }

\צמשנה{אור האמת }\צהגדרה{- מדת\hebrewmakaf הדין\hebrewmakaf העליונה\mycircle{°} }\צמקור{[ע״ר ב תפח]. }

\משנה{אמת }\הגדרה{- אור\hebrewmakaf ד׳\mycircle{°}, מחולל כל }\מקור{[א״ק א ג (מ״ר 402)]}\צהגדרה{. }

\הגדרה{היסוד של החיים השלמים, החיים המקיפים את כל ומלאים את הכל הנובעים מאור צור\hebrewmakaf העולמים\mycircle{°}, הכל\hebrewmakaf יוכל\hebrewmakaf וכוללם\hebrewmakaf יחד\mycircle{°} }\מקור{[ע״א ד יב לב]}\צהגדרה{. }

\משנה{האמת שלמעלה מכל גבולים }\הגדרה{- התענוג\mycircle{°} האצילי\mycircle{°} שלמעלה מכל פגמים, העז\mycircle{°} של העדן\mycircle{°} שכולו\hebrewmakaf אומר\hebrewmakaf כבוד\mycircle{°} }\מקור{[עפ״י א״א 77]}\צהגדרה{. }

\משנה{גבורת האמת }\הגדרה{- מציאות חיה וקיימת, עליונה, בחביון\hebrewmakaf העוז\mycircle{°} האלהי, באופן יותר עשיר, יותר קיים ויותר אמיץ בהוייתו, ויותר נעלה בהופעת רוממות אצילות שלמותו, מכל מה שכל רעיון יוכל לצייר ולתפוס }\מקור{[ע״ט נ]}\צהגדרה{. }

\הגדרה{סוד\mycircle{°} המציאות העצמית, שלמעלה מהזיו\mycircle{°} האידיאלי\mycircle{°} דלגבי דידן }\מקור{[שם שם]}\צהגדרה{. }

\משנה{זוהר האמת }\הגדרה{- ההוד של אור\hebrewmakaf החיים\mycircle{°} שבמקור הקודש\mycircle{°} }\מקור{[א״ק א ג (מ״ר 402)]}\צהגדרה{. }

\משנה{אמת אלהית }\הגדרה{- החפץ המרומם והנשגב של המגמה\mycircle{°} האידיאלית אשר בעליונות הקודש }\מקור{[א׳ יא]}\צהגדרה{. }

\משנה{האמת העליונה }\הגדרה{- אור\mycircle{°} הטוהר\mycircle{°} של שיגוב החכמה\mycircle{°} השוכן ברום חוסן\mycircle{°} אמונת\mycircle{°} ישראל\mycircle{°} }\מקור{[עפ״י א״ק ב רפה]}\צהגדרה{. }

\משנה{האמת העליונה }\הגדרה{- האמת\hebrewmakaf האלהית\mycircle{°}, שהיא נתונה מיד רבון כל המעשים ע״פ הסדר של אמיתת המציאות, ע״פ המהות העליון שלה }\מקור{[ע״ר א נד]}\צהגדרה{. }

\paragraphs

\ערך{אמת }\הגדרה{- }\משנה{מדת האמת }\הגדרה{- הדין בלא צדקה }\מקור{[מא״ה קסד]}\צהגדרה{. }

\paragraphs

\ערך{אמת }\הגדרה{- היסוד ההגיוני המופשט והקר, בעל המופתים והמשפטים המחייבים}\צהגדרה{ }\מקור{[א״ה (מהדורת תשס״ב) ב 81 (א״ב ג)]}\צהגדרה{.}

\paragraphs

\ערך{אמת }\הגדרה{- }\משנה{(לעומת שלום\mycircle{°}) }\הגדרה{- שלמות הקיום של כל נמצא מצד עצמו ופרטיותו }\מקור{[עפ״י מ״ש קכ (מא״ה ב יד)]}\צהגדרה{. }

\משנה{כח האמת האלהית }\הגדרה{- מקור הקדושה\mycircle{°} הכללית של ישראל\mycircle{°}, (ה)נותן כחו בהם לכל יחיד פרטי, להתקדש בכח עצמי ולהעשות בזה מוכן להוסיף קדושה מיוחדת מצדו (לאיזה נושא או ענין) }\מקור{[עפ״י ע״ר ב רנה]}\צהגדרה{. }

\paragraphs

\ערך{אמת }\הגדרה{- היסוד הנצחי המוציא את דבר המשפט\mycircle{°} בקו הצדק\mycircle{°} הגמור }\מקור{[עפ״י ע״א ד יב לח]}\צהגדרה{. }

\משנה{האמת הגדולה }\הגדרה{- מקור משפטי ד׳ }\מקור{[ע״ר ב ס]}\צהגדרה{. }

\משנה{אור האמת }\הגדרה{- נשמת הצדק\mycircle{°} העליון\mycircle{°}, אשר במשפטי ד׳ הישרים\mycircle{°} }\מקור{[שם שם]}\צהגדרה{. }

\משנה{האמת האלהית העליונה}\הגדרה{ - מגמת כל חמדת עולמים, הגנוזה במשפטי ד׳}\צהגדרה{ }\מקור{[עפ״י ע״ר ב נט]}\צהגדרה{.}

\paragraphs

\ערך{אמת }\הגדרה{- }\מעוין{◊}\הגדרה{ דבר נצחי ומתקיים}\myfootnote{ \textbf{אמת דבר נצחי ומתקיים} - ע׳ של״ה ח״א, תולדות אדם, בית דוד, ז: ״אמת פירושו מציאות אמתיי נצחיי שלא יכזב רק יתד אשר לא ימוט״. מגן וצינה דף י.\label{35}}\הגדרה{ }\מקור{[ע״ר א רכז]}\צהגדרה{. }

\הגדרה{הנצח }\מקור{[מ״ר 159]}\צהגדרה{. }

\הגדרה{ע״ע שקר. ע״ע כזב. ע׳ בנספחות, מדור מחקרים, צדק ואמת. ע׳ במדור מונחי קבלה ונסתר, חסד, אמת (משפט), ורחמים. }

\paragraphs

\ערך{אמת }\הגדרה{- }\משנה{גאולת האמת שלמעלה מכל גבולים }\הגדרה{- האידיאליות\mycircle{°} בתענוג\mycircle{°} האצילי\mycircle{°} שלמעלה מכל פגמים, העז\mycircle{°} של העדן\mycircle{°} שכולו\hebrewmakaf אומר\hebrewmakaf כבוד\mycircle{°} }\מקור{[עפ״י א״א 77]}\צהגדרה{. }

\paragraphs

\ערך{אמת }\הגדרה{- }\משנה{באמת }\הגדרה{- }\משנה{(גישה אמיתית בשיפוט) }\הגדרה{- בלא נטיה של חפץ להטיב לשום צד }\מקור{[אג׳ א קנט]}\צהגדרה{. }

\paragraphs

\ערך{אמת - }\משנה{״לכל אשר יקראוהו באמת״ }\הגדרה{- התכלית האמיתי, של עיקר קיום החיים <ולא הבלי עוה״ז הכלים> }\מקור{[עפ״י ע״ר א רכז, מא״ה, ענייני תפילה, שד]}\צהגדרה{. }

\paragraphs

\ערך{אמת }\הגדרה{- }\משנה{להיות חפץ באמת }\הגדרה{- להיות חי ופועל לפי הדיעות היותר אמיתיות וההרגשות היותר קדושות\mycircle{°} לטוב\mycircle{°} ולחסד\mycircle{°}, כמעשה גדולי העולם אשר נגשו אל ד׳ במעשיהם הבהירים למלא את העולם חסד ואמת }\מקור{[ע״א ג ב קפד]}\צהגדרה{. }

\הגדרה{השאיפה לציורי\mycircle{°} המושכלות מצד עצמם ונצחיותם }\מקור{[א׳ לה]}\צהגדרה{.}

\הגדרה{ע״ע אמיתי. }

\paragraphs

\ערך{אמת }\הגדרה{- }\משנה{חיי אמת }\הגדרה{- ע״ע חיים, חיי אמת. }

\paragraphs

\ערך{אן }\הגדרה{- הוראת השאלה ביחס המקום של איזה מבוקש }\מקור{[ר״מ קכד]}\צהגדרה{.}

\ערך{אן }\הגדרה{- }\משנה{בצורה רוחנית }\הגדרה{- דרישת המטרה התכליתית ממחזה כללי המופיע בהמון פרקים, בתבנית ברקי אורות ונצוצי חיים מבריקים, בצורה זעירה ומעולמת. והשאלה חודרת היא, אן מונחת היא המטרה המרכזית של כל המון בריות הללו שהם בלי תכלית }\מקור{[שם]}\צהגדרה{.}

\paragraphs

\ערך{״אנוֹש״ }\הגדרה{- יאמר (על האדם) ע״ש הכח היותר חלוש שבנפש\mycircle{°} והוא הרצון הסתמי בלא טעם דעת ובחירה ושכל, רק רצון לבד ונטי׳ דומה ממש לרצון כל בע״ח למיניהם }\מקור{[עפ״י פ״א קעז, קעה]}\צהגדרה{.}

\הגדרה{ע״ע ״אדם״.}

\paragraphs

\ערך{אני }\הגדרה{- עצמיותי האנושית }\מקור{[ע״ר א מה]}\צהגדרה{.}

\paragraphs

\ערך{אנרכיא }\הגדרה{- }\משנה{אנארכיזם הגשמי האינדיבידואלי }\הגדרה{- אהבה עצמית רבה וגדולה }\מקור{[אג׳ א קעד, קעה]}\צהגדרה{.}

\paragraphs

\ערך{אנשים }\הגדרה{- }\משנה{(לעומת נשים\mycircle{°}) }\הגדרה{- הכח הפועל בעולם (בחברה) }\מקור{[ע״א ד ו כז]}\צהגדרה{.}

\הגדרה{ע״ע איש.}

\paragraphs

\צמשנה{אסיפה}\הגדרה{  - קיבוץ ואחדות }\מקור{[מ״ש קיד]}\צהגדרה{.}

\paragraphs

\ערך{אסתטי }\הגדרה{- }\משנה{החוש האסתטי }\הגדרה{- הרגש של היופי\mycircle{°} וההידור }\מקור{[ע״א ג ב סז]}\צהגדרה{. }

\paragraphs

\ערך{אף }\הגדרה{- הוראה לדבר נטפל שאינו עומד לעצמו, כ״א הוא מצטרף וטפל לדברים אחרים, גדולים ועקרים יותר ממנו }\מקור{[ר״מ קכד]}\צהגדרה{. }

\paragraphs

\ערך{אף }\הגדרה{- ע׳ במדור נפשיות.}

\paragraphs

\ערך{אף }\הגדרה{- }\משנה{(אלקי) }\הגדרה{- ע׳ במדור תיאורים אלהיים. }

\paragraphs

\ערך{אף }\הגדרה{- }\משנה{(בתאור הפנים) }\הגדרה{- ע׳ במדור גוף האדם אבריו ותנועותיו.}

\paragraphs

\ערך{אף }\הגדרה{- ע׳ במדור גוף האדם אבריו ותנועותיו.}

\paragraphs

\תערך{אפיקורסות }\הגדרה{- }\תמשנה{תכן האפיקורסות }\הגדרה{-}\תהגדרה{ הסתלקות האדם מחבור של קדושה\mycircle{°}, מהתקשרות אלהית. המחשבה הגרועה, של הסתלקות מוחלטת מאלהות }\תמקור{[מ״ר 493]. }

\צהגדרה{מהלך מחשבה. מהלך רוח חומרני, מטריאליסטי, המנותק מעוה״ב\mycircle{°}, מנותק מקשר עם הנצח }\צמקור{[שי׳ ת״ת 138].}

\הגדרה{ע״ע כפירה (שלילת האמונה). ע״ע שלילה. ר׳ במדור מדרגות והערכות אישיותיות, אפיקורס. ושם, כופר. ע׳ במדור פסוקים ובטויי חז״ל, העושה תורתו עיתים הרי זה מיפר (תורה) }\צמקור{[ברית]}\הגדרה{. }

\paragraphs

\ערך{אץ }\הגדרה{- }\מעוין{◊}\הגדרה{ מורה מהירות }\מקור{[ר״מ קכה]}\צהגדרה{. }

\paragraphs

\ערך{אץ }\הגדרה{- לחיצה ודחיפה, מקושר עם צרות ביחש המקום\mycircle{°} }\מקור{[ר״מ קכה]}\צהגדרה{. }

\paragraphs

\ערך{אצילות }\הגדרה{- }\משנה{אצילות רצונית }\הגדרה{- המוסריות\mycircle{°} העליונה המתגלה ע״י קדושה\mycircle{°} וחסידות\mycircle{°} טהורה\mycircle{°} ועליונה }\מקור{[ע״ט קכ]}\צהגדרה{. }

\הגדרה{הרצון התמים והבהיר של האדם התופס את קצה זיוה\mycircle{°} של האצילות\hebrewmakaf האלהית\mycircle{°} }\מקור{[עפ״י א״ק ב שמט]}\צהגדרה{. }

\paragraphs

\ערך{אר }\הגדרה{- (מורה) קללה ומארה }\מקור{[ר״מ קכו]}\צהגדרה{. }

\paragraphs

\ערך{אר }\הגדרה{- הוראת אור, הארה\mycircle{°} וזריחה\mycircle{°} }\מקור{[ר״מ קכו]}\צהגדרה{. }

\paragraphs

\ערך{אר }\הגדרה{- (מורה) לקיטה ותלישת פירות }\מקור{[ר״מ קכו]}\צהגדרה{. }

\paragraphs

\ערך{ארוכה }\הגדרה{- רפוי מתמיד וממושך למחלות כרוניות, מתוך קלקולים מתמידים }\מקור{[עפ״י מ״ר 473]}\צהגדרה{. }

\הגדרה{הרפואה המדרגת המשיבה את הכחות שנתרופפו }\מקור{[שם 371]}\צהגדרה{. }

\הגדרה{רפואה טבעית פנימית, לתקן הטבע, והוא לשון תקון כמו ״אריך לנא למחזא״}\myfootnote{ \textbf{אריך לנא }\textbf{למחזא} - עזרא ד יד.\label{36}}\הגדרה{. דרושה למחלות פנימיות. שבחה של דרך רפואה זו הוא להיות קרוב למזון יותר מלרפואה, כדי לחזק את טבע הגוף בעצמו מבלי להוסיף כח זר על הכח הטבעי }\מקור{[עפ״י משפט כהן, פתיחה טו]}\צהגדרה{. }

\הגדרה{ע״ע רפואה. ע״ע חולי. ע״ע מיחוש.}

\paragraphs

\ערך{ארוסין }\הגדרה{- היסוד החוקי האצילי\mycircle{°} שבדבקות\mycircle{°} (שבין בני הזוג), שמתוך מעלתו אין בו התפסה לחקוי חמרי\mycircle{°} כלל }\מקור{[עפ״י ע״ר א לה]}\צהגדרה{. }

\paragraphs

\ערך{ארץ }\הגדרה{- גמר כל תכליתם של הסבות\mycircle{°} הראשיות [של כל מגמה\mycircle{°} ותכלית] המסבבות כל המון המעשים }\מקור{[עפ״י ע״ר א רמה, ע״א א ב ד]}\צהגדרה{. }

\ערך{ארץ}\צהגדרה{ - }\משנה{הארץ בכלל }\הגדרה{- העולם בכלל }\מקור{[ע״א ד ו מו]}\צהגדרה{. }

\הגדרה{המציאות }\מקור{[פנ׳ קלג]}\צהגדרה{.}

\הגדרה{המציאות המעשית }\מקור{[ע״ר א שכה]}\צהגדרה{.}

\הגדרה{גשמותה של המציאות }\מקור{[ע״א ב ט ל]}\צהגדרה{. }

\הגדרה{כלל כל העולמים\mycircle{°} החמריים\mycircle{°} כולם }\מקור{[עפ״י ע״ר א קטו]}\צהגדרה{. }

\הגדרה{העולם החומרי }\מקור{[ע״ר ב פ]}\צהגדרה{.}

\הגדרה{נבכי החומר ובמעמקי מסילותיו המסובכות }\מקור{[ע״א ד ט קד]}\צהגדרה{.}

\הגדרה{השטח התחתיתי }\מקור{[עפ״י ע״ר ב סז]}\צהגדרה{.}

\הגדרה{כחות החמריים, מחשבות האדם, סדרי החיים והחברה, וכל הנוגע לכל תהומות, עד מעמקי שפל }\מקור{[קובץ ה סח]}\צהגדרה{. }

\הגדרה{הענינים החומריים\mycircle{°} המדיניים האקונומיים }\מקור{[ע״א ג א לג]}\צהגדרה{. }

\הגדרה{ע״ע שמים. ע׳ במדור מונחי קבלה ונסתר, אחרית. ע׳ במדור מונחי קבלה ונסתר, ברתא. }

\paragraphs

\ערך{ארץ }\הגדרה{- }\משנה{תכונה ארצית}\הגדרה{ - ברכת הטבע וכל כחותיו }\מקור{[ע״ר א רט]}\צהגדרה{. }

\paragraphs

\משנה{ארץ}\צהגדרה{ - }\מעוין{◊}\צהגדרה{ מכון הטבע של האומה\mycircle{°} }\צמקור{[צ״צ קא]}\צהגדרה{.}

\צהגדרה{ר׳ שפה.}

\paragraphs

\משנה{ארץ}\צהגדרה{ - }\צמשנה{קדושת הארץ}\הגדרה{ - }\צהגדרה{קדושת הכלל\mycircle{°} }\צמקור{[שי׳ ה 212]. }

\צהגדרה{כלליות הקדושה של כל הקדושות }\צמקור{[רצי״ה ג״ר 133].}

\paragraphs

\ערך{ארץ אשור }\הגדרה{- }\משנה{(׳האובדן בארץ אשור׳)}\myfootnote{ ישעיה כז יג.\label{37}}\הגדרה{ - <}\צהגדרה{אשור}\הגדרה{ - לשון הבטה והשקפה> האובדן בארץ בדעות רעות ורוח שטות המביא לידי עבירה. גלות טעות השכל והתרופפות האמונה (בהשגחה וחסרון בטחון וכיו״ב, או זלזול הורים ומורים מפני מיעוט יקרת התורה בנפשו), מפני טרדות עוה״ז והבליו המשכחים את האמת }\מקור{[עפ״י מ״ש סו\hebrewmakaf ח]}\צהגדרה{.}

\הגדרה{ע״ע ארץ מצרים. ע׳ במדור מקומות וארצות, אשור. ע׳ במדור פסוקים ובטויי חז״ל, האבדים בארץ אשור והנדחים בארץ מצרים. }

\paragraphs

\ערך{ארץ הרוחנית }\הגדרה{- הציורים\mycircle{°} ההולכים עם ההסברים של הידיעות האלהיות\mycircle{°}, רפידת המחשבה }\מקור{[פנ׳ קלו]}\צהגדרה{. }

\paragraphs

\ערך{ארץ מצרים\mycircle{°} }\הגדרה{- }\משנה{(׳הנדחות בארץ מצרים׳)}\footref{37}\הגדרה{ - גלות תאוות עוה״ז, <כי }\צהגדרה{מצרים}\הגדרה{ היא ערות הארץ ומקור התאוות והחומריות כולן> }\מקור{[מ״ש סו]}\צהגדרה{.}

\הגדרה{ע״ע ארץ אשור. ע׳ במדור מקומות וארצות, מצרים. ע׳ במדור פסוקים ובטויי חז״ל, האבדים בארץ אשור והנדחים בארץ מצרים. }

\paragraphs

\ערך{אש }\הגדרה{- החומר השורף והמכלה, המחמם והמאיר, העושה את הפעולות ההפכיות בתכונתן, הכל לפי ערכם של מקבלי המפעלים }\מקור{[ר״מ קכז]}\צהגדרה{. }

\משנה{כח האש וחומו  }\הגדרה{- הפועל הגורם להמפעלים שיעשו }\מקור{[ע״ר א קכט]}\צהגדרה{. }

\paragraphs

\ערך{אִשָּׁה }\הגדרה{-}\משנה{ יסוד השלמתה }\הגדרה{- עדינות הרגש\mycircle{°} הטהור\mycircle{°} והטוב\mycircle{°}, והשכל\mycircle{°} יעזר על ידו כפי המדה האפשרית }\מקור{[ע״א ג ב ריג]}\צהגדרה{.}

\הגדרה{ע״ע איש. ע׳ במדור הכרה והשכלה והפכן, בינה יתירה (באשה). ע״ע גברת.}

\paragraphs

\ערך{את }\הגדרה{- מלת הצירוף, הוראת הטפלה, הערכת התוספת\mycircle{°} }\מקור{[ר״מ קכז]}\צהגדרה{. }

\paragraphs

\ערך{את }\הגדרה{- הכלי היסודי לעבודת האדמה להוציא מחיה מן הארץ }\מקור{[ר״מ קכח]}\צהגדרה{. }

\paragraphs

\משנה{א״ת ב״ש }\הגדרה{- אחדות החיצוניות והפנימיות <אופן הא״ב בסדר א״ת ב״ש הוא שהקצות הרחוקים מתאחדים, ואין שינוי והפרש, וכלם הולכים להתאחד ולהתקרב עד כ״ל, מפני שב״כל״ אין הרחק והבדל מדרגות> }\מקור{[עפ״י פנק׳ ג לה]}\צהגדרה{.}\mylettertitle{ב}

\paragraphs

\ערך{בא }\הגדרה{- הוראת התכנסות הנושא אל המקום הראוי ע״י תנועה מוקדמת }\מקור{[ר״מ קכח]}\צהגדרה{.}

\paragraphs

\ערך{באור }\הגדרה{- יחשו של כל מאמר בודד, לא רק לפי ערכו והדבר המבוצר בתוכו בלבד, כ״א עפ״י ערך כל אותן ההשפעות שאפשר לו להשפיע, לכשיתבאר, כאשר ״מעין ישיתוהו״ על עולם הרעיונות ההולכים בדרך ישרה, הוא פתוח בדרך מפולש לעולם הגדול של ההשכלות מלאות זיו\mycircle{°}, ומעורר בדרך פתחו להכניס אל תוכו ועל ידו תלי תלים של ידיעות והרחבות שכליות, שנותנות אומץ וגבורה לנפשות ההוגות בהם. ״מ״ם פתוחה - מאמר\hebrewmakaf פתוח\mycircle{°}״ }\מקור{[ע״א א, הקדמה, יד\hebrewmakaf טו]}\צהגדרה{.}

\הגדרה{ע״ע פרוש. ע׳ במדור תורה, דרש.}

\paragraphs

\ערך{באור }\הגדרה{- }\משנה{דרך הביאור }\הגדרה{- הדרישה הבנויה ע״פ ערכי הכללים, שהם דומים לדרישת כל רעיון לא רק מצד עצמו, כ״א מצד הרעיונות שמטבעו להוליד ע״פ דרך ישרה, כן דרישת התורה שע״פ הכללים, אין הפרטים נולדים ומסתעפים זה מזה, כ״א כולם יחד יוצאים הם מהכללים הראשיים יסודי ועקרי התורה וסתרי טעמיה הגדולים }\מקור{[ע״א א, הקדמה, יז]}\צהגדרה{.}

\הגדרה{ע׳ במדור תורה, דרישת התורה בדרך הכהן. }

\paragraphs

\ערך{בג }\הגדרה{- מזון }\מקור{[ר״מ קכח]}\צהגדרה{.}

\paragraphs

\ערך{בד }\הגדרה{- מגזרת בדד, ההתבודדות הפרטית }\מקור{[ר״מ קכט]}\צהגדרה{.}

\paragraphs

\ערך{בד }\הגדרה{- שקר\mycircle{°} }\מקור{[ר״מ קכט]}\צהגדרה{.}

\paragraphs

\ערך{בד }\הגדרה{- ענף של אילן }\מקור{[ר״מ קכט]}\צהגדרה{.}

\paragraphs

\ערך{בד }\הגדרה{- מוט }\מקור{[ר״מ קכט]}\צהגדרה{.}

\paragraphs

\ערך{בד }\הגדרה{- ע״ע בוץ.}

\paragraphs

\ערך{בה }\הגדרה{- תואר הוראה של יחס פנימי, למושג הנקבי }\מקור{[ר״מ קל]}\צהגדרה{.}

\paragraphs

\ערך{בהלה }\הגדרה{- טירוף של רצונות ומחשבות שכל אחת דוחה את חבירתה עד אפס מקום של חשבון נקי }\מקור{[ע״א ג ב רכג]}\צהגדרה{.}

\הגדרה{נפילת ערך וחרדה}\צהגדרה{ }\מקור{[ע״ר ב עא]}\צהגדרה{.}

\paragraphs

\ערך{בו }\הגדרה{- מתאר את היחש הפנימי במושג הזכרי }\מקור{[ר״מ קלא]}\צהגדרה{.}

\paragraphs

\ערך{בוץ }\הגדרה{- פשתן\mycircle{°}, <הנקרא ג״כ בד על שם שהוא עולה בד בבד> מורה על שמירת הגבולים, על כח צומח פורה, שעם זה הוא שומר חק וגבול, ואינו מתערב בחלקי חיים ותוכן של נושא אחר, שומר את המצר, ומגין על הצדק\mycircle{°} }\מקור{[ר״מ קלו]}\צהגדרה{.}

\paragraphs

\ערך{בושה }\הגדרה{- }\משנה{אמיתתה במקורה }\הגדרה{- יראה\hebrewmakaf עליונה\mycircle{°}, יראת\hebrewmakaf ד׳\mycircle{°} }\מקור{[עפ״י ר״מ קמ, א״ש יד כד]}\צהגדרה{. }

\paragraphs

\ערך{בחירה }\הגדרה{- }\משנה{״בחר בנו מכל העמים״ }\הגדרה{- נתן לנו יתרון ומעלה ודבקות\mycircle{°} בו ית׳ על כל עם ולשון }\מקור{[עפ״י מ״ש קמב (ה׳ קפד)]}\צהגדרה{.}

\צהגדרה{יצר אותנו להיות לו לעם\hebrewmakaf נחלה, להתגלות צלם\hebrewmakaf האלהים\mycircle{°} שבאדם בקרבנו בתור עם}\myfootnote{ ע׳ אור החיים בראשית א כז. ״״ויברא אלהים את האדם בצלמו״. כי ברא האדם בב׳ צלמים, הראשון צלם הניכר בכל אדם ואפילו בבני אדם הריקים מהקדושה אשר לא מבני ישראל המה, ועליהם אמר ״בצלמו״ פירוש: של הנברא; והב׳ הם בחינת המאושרים, עם ישראל נחלת שדי, כנגד אלו אמר ״בצלם אלהים בראו״, הרי זה בא ללמדנו כי יש בנבראים ב׳ צלמים צלם הניכר וצלם אלהים רוחני נעלם, והבן״.\label{1}}\צהגדרה{, באדם הצבורי המופיע בנו במלוא שיעור\hebrewmakaf קומתנו }\צמקור{[ל״י ג קיד\hebrewmakaf קטו (מהדורת בית אל תשס״ב ב תד)].}

\צהגדרה{הופעת קדושת עצמיותנו הצבורית, מתוך השראת\hebrewmakaf שכינתו\mycircle{°} על כלנו כאחד, בהוציאו אותנו מבית\hebrewmakaf עבדים ממצרים\mycircle{°}, ובקרבו\hebrewmakaf אותנו\hebrewmakaf לפני\hebrewmakaf הר\hebrewmakaf סיני\mycircle{°} }\צמקור{[ל״י א (מהדורת בית אל תשס״ב) רכו].}

\משנה{בחירתנו מכל העמים }\צהגדרה{- יצירת מהותנו. הוית עצמותנו הצבורית בכל ממשותה ושכלולה, תקפה ותפארתה}\צמקור{ [עפ״י שם, רכז].}

\הגדרה{ע״ע בחירת ישראל.}

\paragraphs

\ערך{בחירה }\הגדרה{- }\משנה{(בעם ישראל, לעומת סגולה\mycircle{°}) }\הגדרה{- ההערכה הגלויה של קדושתם\mycircle{°} של ישראל\mycircle{°} }\מקור{[ע״ר ב פ]}\צהגדרה{.}

\מעוין{◊ }\צמשנה{הבחירה}\צהגדרה{ הגלויה, מתבררת ע״י המדות הקדושות והמובחרות, שבהן נתעטרו\mycircle{°} בני יעקב\mycircle{°} בכללם, עד שהכל מכירים שהבחירה האלהית ראויה להם }\צמקור{[ע״ר א רב].}

\הגדרה{ע׳ במדור מדתם ועניינם הרוחני של אישי התנ״ך, יעקב, מדת התאר יעקב (לעומת ישראל). ושם, ישראל, מדת התאר ישראל (לעומת יעקב). ע״ע ישראל לעומת ישורון. ע׳ במדור מונחי קבלה ונסתר, קוב״ה דרגא על דרגא סתים וגליא וכו׳. ע׳ במדור פסוקים ובטויי חז״ל, בית יעקב לעומת בני ישראל. ושם ממלכת כהנים וגוי קדוש. ושם בני בכורי לעומת בנים. ע׳ בנספחות, מדור מחקרים, ״בחרתי בכם ויחדתי שמי עליכם״ לעומת ״אני בכבודי מתהלך ביניכם״.}

\paragraphs

\ערך{בחירה }\הגדרה{- }\משנה{בחירת ד׳ בכהנים\mycircle{°} }\הגדרה{- הטבעה טבעית רוחנית עליונה בסגולת\mycircle{°} נפשותם}\צהגדרה{ }\מקור{[ע״ר א קסא]}\צהגדרה{.}

\הגדרה{ע׳ במדור אישים, אהרן. ושם, ״אהרן ובניו״. ושם, ״בני אהרן״. ע״ע כהונה, קדושתה.}

\paragraphs

\תערך{בחירה}\תהגדרה{ - חופש הפעולה של האדם }\תמקור{[נ״א ד 39].}

\paragraphs

\ערך{בחירה גלויה }\הגדרה{- }\משנה{הבחירה הגלויה }\הגדרה{- הבחירה\hebrewmakaf החפשית\mycircle{°} הנמצאת בפועל, המתגלמת בבני אדם, שהמשפט המורגש מתראה על ידה, שבחיים המתגלים לפנינו לעולם לא נמצא אותה במילואה }\מקור{[עפ״י אג׳ א שיט, א״ק ג לד]}\צהגדרה{. }

\הגדרה{הבחירה שפרטי הכוחות והפרטיות הדקות שבמציאות ביחש לשכר ועונש נחלקים על פיה }\מקור{[עפ״י פנ׳ כג]}\צהגדרה{. }

\מעוין{◊}\הגדרה{ עקר הבחירה וראשית יסודה הוא בבחינת הרוח\mycircle{°}, שהוא מדרגת האדם }\מקור{[ע״ר א רמט]}\צהגדרה{. }

\הגדרה{ע״ע בחירה צפונה. ע״ע בחירה כמוסה. }

\paragraphs

\ערך{בחירה  גנוזה }\הגדרה{- ע״ע בחירה כמוסה. }

\paragraphs

\ערך{בחירה חפשית }\הגדרה{- החופש\mycircle{°} הגמור, המחולל בקרבו רצון\mycircle{°} שאין בו שום מועקה מבחוץ, שעל ידו מתגלה העצמות\mycircle{°} של מהות החיים, המקבלים את התכנית של הגורל\mycircle{°} הטוב\mycircle{°} או הרע\mycircle{°}, שרק החלק הקטן (ממנה), המתגלה בתור בחירה מעשית (בחירה\hebrewmakaf גלויה\mycircle{°}), מתוה לפנינו בגלוי את ארחות הטוב והרע }\מקור{[עפ״י אג׳ א שיט]}\צהגדרה{. }

\תהגדרה{חופש הבחירה, חופש הפעולה של האדם. חוקי האפשר\mycircle{°}, כח ואפשרות בחירת מעשים בזולת מעשים ותועלתם }\תמקור{[נ״א ד 39].}

\הגדרה{ע״ע התגלות המהות.}

\paragraphs

\ערך{בחירה כמוסה }\הגדרה{- הבחירה שאיננה על פי התוכן המוסרי\mycircle{°} המתגלה, אלא על פי האידיאל\mycircle{°} העליון\mycircle{°}, שעל פי הצפיה\hebrewmakaf העליונה\mycircle{°}, למעלה מהתנאים שההויה נמצאת בהם כעת. ההזרחות\mycircle{°} שבאות מתוכן זה הם אורות\mycircle{°} הנשמה\mycircle{°} הפנימית\mycircle{°} של כל היש, והן כוללות את העבר ההוה והעתיד, למעלה מסדר זמנים וצורתם, וכל זה כלול בשם\hebrewmakaf ההויה\mycircle{°}, כסדרו ובכל אופני צירופיו }\מקור{[א״ק ג כג (ע״ט ב)]}\צהגדרה{. }

\הגדרה{הבחירה שכל מערכת המשפט\mycircle{°} של כל היש מתנהגת על ידה }\מקור{[א״ק ג לד]}\צהגדרה{.}

\הגדרה{הבחירה שפרטי הכוחות והפרטיות הדקות שבמציאות בכלל (להוציא מדרגות ה׳שכר ועונש׳) נחלקים למדריגותיהם על פיה ע״פ היסוד ד״הכל צפוי״\mycircle{°} }\מקור{[עפ״י פנ׳ כג]}\צהגדרה{. }

\משנה{בחירה צפונה }\הגדרה{- יסוד כל חק ומשפט\mycircle{°}. הבחירה השמה את המערכות לפי מדרגותיהן, מגדולי המציאות עד קטניהם }\מקור{[עפ״י א״ק ג לד]}\צהגדרה{. }

\הגדרה{גלגול (הזכות או החובה), האמצעות של המסבבים את הדברים הרשמיים בהם נעוץ כח החפץ הגמור, הנכלל בכחות המציאות שלא לחסר ממנה את מושגי המוסר והרשע והצדק וכל העלילות הגדולות המסובבות מהם ועל ידם, באין שום גרעון, ״כי כל אשר יעשה האלהים הוא יהיה לעולם עליו אין להוסיף וממנו אין לגרוע והאלהים עשה שיראו מלפניו״\mycircle{°}. וההכרה המחברת את מושג המשפט הקבוע, עם מושג החופש\mycircle{°}, להתאימם עם העז\mycircle{°} והמשפט המלא את כל היקום, כי ״אלהים שופט צדיק״, היא }\משנה{הבחירה הכמוסה}\הגדרה{ הגלויה רק ליוצר כל במקור החכמה\hebrewmakaf האלהית\mycircle{°} }\מקור{[ע״א ג ב רד]}\צהגדרה{. }

\משנה{הבחירה הגנוזה}\myfootnote{ ע״ע הערת הרצי״ה אג׳ א עמ׳ שפה-שפו, לעמ׳ שיט.\label{2}}\הגדרה{ - הבחירה\hebrewmakaf החפשית\mycircle{°} הגמורה, שהיא עצם המהותיות שלנו, המיטב והעיקרי שבהויתנו, המתגלה בכל מלואה ועשרה רק לצפיה\hebrewmakaf העליונה }\מקור{[עפ״י אג׳ א שיט, שם ב מב, ע״ר ב קנז]}\צהגדרה{.}

\הגדרה{ע׳ במדור מונחי קבלה ונסתר, יובל, עלמא דיובלא.  ע׳ בנספחות, מדור מחקרים, ידיעה ובחירה.}

\paragraphs

\ערך{בחירת ישראל }\הגדרה{-}\משנה{ מטרת בחירת\mycircle{°} ישראל על פי ד׳ בהתגלות אלהות על ידי אותות ומופתים גלויים}\הגדרה{ - כדי שיהיו מוכנים לקרבת\hebrewmakaf אלהים\mycircle{°} היותר נעלה, שהיא יסוד העליון והתכליתי למין האנושי, ושמשפע מוסרם בצירוף הכח האלהי שכבר נגבל בהיסתוריה הברורה שלהם ובארץ הקודש, לכשתצא מן הכח אל הפועל גדולתם ותפארתם כמו שראוי להיות לעם הנושא את היסוד היותר מעולה וכולל לכל המין האנושי בידו, שהוא יסוד הרוחניות של הרחבת דעת השם בחיים האנושיים, אז מאיליה תצא הפעולה לכל העולם ברב הוד והדר }\מקור{[ל״ה 168]}\צהגדרה{.}

\paragraphs

\ערך{בטול }\הגדרה{- התכללות }\מקור{[עפ״י קובץ ה צה]}\צהגדרה{. }

\ערך{בטול}\myfootnote{ \textbf{הטעות שיש במהותיות עצמותית, והתגברות החפץ }\textbf{לאשתאבה}\textbf{ }\textbf{בגופא}\textbf{ }\textbf{דמלכא} - ע׳ לקוטי תורה לרש״ז, שיר השירים א. ״והנה כלה יש בו ב׳ פירושים. הא׳ לשון כליון וכו׳, והב׳ מלשון כלתה נפשי והיינו תשוקת הנפש לידבק וליכלל באורו ית׳״. \label{3}}\הגדרה{ - }\משנה{בטול אל האור\hebrewmakaf האלהי\mycircle{°}}\הגדרה{ - להשתאב\hebrewmakaf בגופא\hebrewmakaf דמלכא\mycircle{°} }\מקור{[עפ״י א״ק ב שצח]}\צהגדרה{. }

\הגדרה{כלות\hebrewmakaf הנפש\hebrewmakaf לאלהים\mycircle{°} }\מקור{[עפ״י שם, ע״ר א מז, סז]}\צהגדרה{. }

\משנה{ביטול גמור של מהות עצמו }\הגדרה{- הכרת הנשמה את כל הטעות שיש במהותיות עצמותית, והתגברות החפץ לאשתאבה בגופא דמלכא, בשלמות אין סוף של נועם העליון. הענוה\mycircle{°} הגמורה, והשפלות העמוקה, שהעצמיות היא בה רק שירים\mycircle{°}, כלומר ענין של חסרון שנשאר בלתי כלול בשלמות העליונה }\מקור{[עפ״י קובץ א שיב]}\צהגדרה{. }

\הגדרה{ע״ע כניעה, ההכנעה מפני האלהות. ע׳ במדור פסוקים ובטויי חז״ל, ונחנו מה. ושם, ואנכי תולעת ולא איש. ושם, כלה שארי ולבבי צור לבבי וחלקי אלהים לעולם.}

\paragraphs

\ערך{בטול}\myfootnote{ ע״ע פנק׳ א תכד סי׳ מח.\label{4}}\הגדרה{ - }\משנה{בטול פנימי עדין }\הגדרה{- (בטול עצמי) המשפיל את הצד המכוער שבנו ומרומם את כל מהות הטוב והעדין}\צהגדרה{ <ואינו מטשטש את אומץ החיים> }\מקור{[עפ״י א״ש יד כא]}\צהגדרה{. }

\paragraphs

\ערך{בטחון }\הגדרה{- דעת בבינת\hebrewmakaf לב\mycircle{°} פנימית ואדירה }\מקור{[עפ״י ע״א ג ב צ]}\צהגדרה{. }

\מעוין{◊}\הגדרה{ בא מתוך הדבקות\hebrewmakaf האלהית\mycircle{°}, הבאה מתוך האמונה\hebrewmakaf השלמה\mycircle{°} }\מקור{[עפ״י ע״ר ב פו]}\צהגדרה{. }

\מעוין{◊ }\משנה{הבטחון}\הגדרה{ כולל בתוכו את הדבקות האמונית בשלמותה, וממשיך את אור\hebrewmakaf החיים\mycircle{°} ממקור\hebrewmakaf החיים\mycircle{°}, מחי\hebrewmakaf העולמים\mycircle{°} ברוך הוא, לכל מי שמתעטר בקדושת האמונה והדבקות האמיתית }\מקור{[עפ״י שם]}\צהגדרה{.}

\ערך{בטחון}\myfootnote{ \textbf{יסוד הבטחון} - \textbf{הוא לא שהאדם בטוח שמה שהוא דורש ימלא ד׳, כי אפשר שמה שהוא חושב, שהוא הטוב, הוא ההפך מהאמת. אלא שהוא בטוח בחסד עליון וכו׳} - ע״ע ע״א ג א מג. צבי לצדיק לרי״מ חרל״פ פרק ד. להבחנת מדרגות הבטחון השונות ע׳ ע״א א א קמג, ע״א ג א מו, ובע״א ב ט קעג. ובקבצ׳ ב עמ׳ יח סי׳ טז: ״שיטת הלאומיות שאומרים וכו׳ שאין להשתמש בבטחון לענין לאומי״.\textbf{אדם הנברא }\textbf{בצלם\hebrewmakaf אלהים}\textbf{ הוא תמצית כל ואחוד }\textbf{הכל}\textbf{, ומצד }\textbf{הכל}\textbf{ הלא אין אבוד ולא }\textbf{הירוס} - ע״ע ע״ר א שלב (א״ק ב תקג). ע״ע ע״ר א קעג ד״ה וטעונה ״לא אבדת ציורים ורשומים פרטיים הוא הענין של מלוי הקדש, אלא הרמתם העשירה עם כל רכוש ציוריהם לרום מעלות הקדש״. ושם שם א ד״ה אני, שם שם רטז ד״ה ד׳ צבאות, שם שם מט ד״ה וצור חבלי, ושם שם רכב ד״ה דעו. ובע״ר ב סב ד״ה דרשתי שם שם סג ד״ה הביטו, שם שם קנז ד״ה חלק, ושם שם רנה.\textbf{תכלית הבטחון} - \textbf{האסונות, הנכונים לבא על בני אדם, באופן כזה שאין הזהירות האנושית יכולה להגן הרי הם סרים מן הבוטח }- ע״ע שם בע״א ג ב קצב, ובמשפט כהן עמ׳ שכז\hebrewmakaf ח, שנט. ובעזרת כהן עמ׳ קמא. ע״א א א קא ״גם על ההשגחה האלקית ראוי שיקבע בנפשו שלא יאתה להיות סומך כ״א במה שאין ידו מגעת להשתדל בעצמו״. ושם ח״ב ט קכ ״הבטחון הוא מוגדר כשנשלים את חק ההשתדלות במה שהוא בידינו, ובמה שאין יכולת שלנו מגיע(ה) לזה, שם הוא מקום הבטחון. כי במקום שהיכולת מתגלה חלילה להשתמש בבטחון, שאין זה בטחון כ״א הוללות ומסה כלפי מעלה״. ע״ע שם שם כג, ושם ג ב קצו. ע״ע רמב״ם, פיהמ״ש פסחים נו., עקדה שער כו, בראשית דף רכא.\hebrewmakaf רכו:, וברבנו בחיי עה״ת, שמות יג יח. ובבאור הגר״א על משלי יד טז, (מהד׳ פיליפ) עמ׳ 173 ד״ה וכסיל מתעבר ובוטח ״הכסיל עובר במקום שיש לטעות או במקום סכנה ובוטח בה׳ שלא יבא לידי רע, והוא בטחון הכסילים כי מי מכריח אותו לילך במקום סכנה״. (אך ע׳ שם ג ה, עמ׳ 49 50 ובהערה 24 שם וצ״ע, ואולי י״ל עפ״י דבריו שם טז כ, עמ׳ 197, בחלוק בין עוה״ז ועוה״ב, ותורה ותפילה. מ״מ, אין שיטה זו עולה בקנה אחד עם שיטת האמונות ודעות לרס״ג י טו שאומר על מי שאומר שבטוח בד׳ על עניני עוה״ז בלא השתדלות, שהיא דעה זרה, דא״כ יאמר ג״כ על עניני עוה״ב, ומה תכלית התוהמ״צ. ואמנם, כבר נחלקו בענין זה אבות העולם ראשונים ואחרונים. וע׳ באלפי מנשה, ח״א, פרק צח. ובמערכתו של ר״ש מאלצאן, אבן שלמה, ליקוטים בסוף הספר דף סט.\hebrewmakaf עא. ואא״ל על פיהם). ע״ע אבן ישראל ח״ג, בהקדמה.\label{5}}\הגדרה{ - }\משנה{יסוד הבטחון המעלה את האדם לתכונת קדושה\mycircle{°} עליונה, רוממות נפש וגדולת קדש }\הגדרה{- <יסוד בטחון זה, הוא לא אותו הציור\mycircle{°}, שהאדם יצייר בעצמו שהוא בטוח, שמה שהוא דורש ומבקש, וחושב שדרוש לו, ימלא ד׳\mycircle{°}, כי אפשר שמה שהוא חושב, שהוא הטוב, הוא ההפך מהאמת. אלא> שהוא בטוח בחסד\mycircle{°} עליון\mycircle{°}, שברא את העולם ובנה אותו, וכוננו, ומשגיח עליו ברב חסד, ועל כן אין מקום לשום דאגה, לשום עצבון רוח, כי הלא יודעים אנו, שחסד אל נטוי על כל יצוריו, והננו נכנסים תחת כנפי חסדו בכל רגע }\מקור{[ע״ר א רכ]}\צהגדרה{.}

\משנה{יסוד (הבטחון ב)שם\hebrewmakaf ד׳\mycircle{°}}\הגדרה{ - לבטוח שההנהגה עוזרת לקנות השלמות האמיתית }\מקור{[עפ״י ע״א א א נג]}\צהגדרה{. }

\משנה{יסוד הבטחון והשמחה }\הגדרה{- נובע מהבירור הפנימי שאין לחפוץ, גם לעניני עצמו, כ״א את מה שהוא חפץ אדון כל ב״ה. ואז ימלא אדם שמחה ואומץ לב כפי מדת בירור דבר זה בלבבו, ולפי ערך התאמת כל ארחות חייו לזאת המדה העליונה }\מקור{[פנ׳ לד]}\צהגדרה{. }

\משנה{בטחון נשגב\mycircle{°} עליון\mycircle{°}}\הגדרה{ - החסיון\mycircle{°} האידיאלי\mycircle{°} הבא מתוך ההופעה\mycircle{°} העליונה, בלא שום מבט על הגורל\mycircle{°} הנופל בחלקה של האישיות הפרטית. כי מתוך השגוב העליון והזיו\hebrewmakaf האלהי\mycircle{°}, של מקור כל השלמות ושורש כל תענוג\mycircle{°} ואור\mycircle{°}, הכל מתבטל\mycircle{°} מרוב נועם\mycircle{°} }\מקור{[עפ״י ע״ר ב עד]}\צהגדרה{. }

\הגדרה{הבהירות של הידיעה האלהית העליונה וחשק הלב הפנימי בהתמלאותם של האידיאלים\hebrewmakaf האלהיים\mycircle{°} במלא כל היש, והבירור הגמור שכן הוא, ושהכל הולך לחפץ הטוב העליון, (המביאים) שמחת הנשמה הפנימית, וכל דאגה עצבית מתגרשת, וחדות\mycircle{°} ד׳ מתמלאת בכל מהותו של אדם }\מקור{[עפ״י קובץ ה קכה]}\צהגדרה{. }

\משנה{יסוד הבטחון }\הגדרה{- יסוד הבטחון בא מתוך החסן\mycircle{°} אשר לנו באלהים\mycircle{°} סלה\mycircle{°}. כשאדם הנברא בצלם\hebrewmakaf אלהים\mycircle{°} הלא הוא באמת תמצית כל ואחוד הכל, ומצד הכל הלא אין אבוד ולא הירוס, לא השפלה ולא ירידה\mycircle{°}, כ״א כולו אומר כבוד\mycircle{°} וחיים\mycircle{°}, וכשהאדם מכיר את עוזו\mycircle{°} באלהי\hebrewmakaf הצבאות, ד׳\hebrewmakaf צבאות\mycircle{°}, הלא הוא מלא בטחון }\מקור{[ע״ר א קנ]}\צהגדרה{. }

\משנה{תכלית הבטחון }\הגדרה{- קרבת\hebrewmakaf אלהים\mycircle{°} הנמשכת מן הבטחון, וגבורת הנפש בעז\hebrewmakaf ד׳\mycircle{°} הנמשכת ממנה בעת צר, שהאדם מוצא לו תמיד מחסה\mycircle{°} בשם\hebrewmakaf ד׳\mycircle{°}, וגם בעת אשר כל המסיבות הטבעיות\mycircle{°} כבר חדלו כח להצילו מרעתו, עז\hebrewmakaf ד׳\mycircle{°} ישגבהו\mycircle{°} תמיד, והאסונות, הנכונים לבא על בני אדם, באופן כזה שאין הזהירות האנושית יכולה להגן, הרי הם סרים מן הבוטח, כשם שרגשי הפחד הדמיוני סרים מפני האור\mycircle{°} של הבטחון, ונפשו מלאה אומץ, וסדר שלותי קבוע בה }\מקור{[עפ״י ע״א ג ב קצב, ע״ר ב עה]}\צהגדרה{. }

\משנה{הבטחון מדתו }\הגדרה{- הרחבת כח העז והגבורה, אפילו במה שהוא למעלה מגבולי כח היכולת הקבועה בכחות האדם הגלויים, כי אין מעצור לד׳ להושיע ולעזור גם לאין כח. }\צהגדרהמודגשת{מדת הבטחון }\צהגדרה{באה לאחר שיאזר האדם בגבורה בכל אשר תשיג ידו בכחותיו החומריים והרוחניים, ובבאו לגבול ששם יש לפניו מעצור כח המוגבל החלש, אל יפול לבבו. <וזאת היא }\צהגדרהמודגשת{מדת הבטחון}\צהגדרה{ שצריכה להתחבר תמיד עם מדת הגבורה, המועלת למלא את נפש האדם כבוד ועז. וכשהיא מתחברת עם הדעה השלמה והמוסר האמיתי, היא מדרכת את האדם בדרך ד׳ העליונה, ומכשרתו להיות בד׳ מבטחו. ומעלתו, שיהיה כבוד ד׳ חופף עליו לעשות לו ניסים, בין גלויים בין נסתרים, במערכות סדרי הטבע> }\מקור{[עפ״י ל״ה 177 (פנק׳ ב קכא)]}\צהגדרה{.}

\צהגדרה{יסוד החיים הלא הוא הכח לפעול ולעשות, כל איש לפי ערכו, וכל חברה לפי ערכה. החיים הטובים המה, שתהיינה הפעולות מסודרות יפה ועולות תמיד במעלה בהוספת ערך והשלמה. והנה האדם הוא איננו חפשי גמור, פעמים רבות יתיצבו לו כצר מונעים רבים שיעכבוהו שלא יוכל ללכת מהלך החיים שלו, שלא יוכל לפעול לפי תכונתו וערכו, ואז הוא צריך להתגבר עליהם בכל עז. ועל זה צריך }\צהגדרה{שיבטח בד׳, }\הגדרה{שאם אפילו כחותיו לא יספיקו לו, מכל מקום ״אין מעצור לד׳ להושיע״, ותשועת ד׳ תשגבהו להסיר המניעות, למען יוכל לפעול ולעבוד ולחיות כראוי}\צהגדרה{ }\מקור{[עפ״י ל״ה 240]}\צהגדרה{. }

\הגדרה{ע׳ בנספחות, מדור מחקרים, בטחון לעומת אמונה.}

\paragraphs

\ערך{בטלה - }\צהגדרה{(הממיתה את הנשמה)}\myfootnote{ \textbf{פנק׳ ד קלו: ״}\textbf{ע״י מה }\textbf{שנוטין}\textbf{ לצד הגס של }\textbf{תאות}\textbf{ החושים, מסתתמים כל הצינורות של ההארה הרוחנית, ואין האור של הרצון הטוב נקלט בתוך הלב}\textbf{״.}\label{6}}\הגדרה{ - התעסקות בדברים חומריים\mycircle{°} וגסים\mycircle{°}}\צהגדרה{ }\מקור{[פנק׳ ד רז]}\צהגדרה{.}

\paragraphs

\ערך{בי }\הגדרה{- מבטא, מבליט, את המהותיות הפנימית של הנושא, המכריז על עצמו את התגלותו העצמית, ומודיע את הגנוז בקרבו }\מקור{[ר״מ קלב]}\צהגדרה{.}

\ערך{בי }\הגדרה{- הבעה הבאה מתוך מעמקי הנשמה\mycircle{°}, הריכוז היותר פנימי\mycircle{°} ויותר כללי\mycircle{°}}\צהגדרה{ }\מקור{[ר״מ קלב]}\צהגדרה{.}

\ערך{בי }\הגדרה{- מורה על העצמיות\mycircle{°} המיוחדה של האדם ותוכן החיים הטבעי שלו, שהיא הבסיס לקבל עליה את האור\mycircle{°} העליון\mycircle{°} של הנשמה\mycircle{°} }\מקור{[ע״ר א ג]}\צהגדרה{. }

\הגדרה{התוכן המורגשי של האדם בהויתו הפרטית. אותו תוכן שיש עמו ג״כ חבור להצד הפרטי, המסמן את הפירוט היחידי של האדם באשר הוא מוגבל ומצומצם }\מקור{[שם סז]}\צהגדרה{. }

\paragraphs

\ערך{ביזה }\הגדרה{- מה ששוללים דרך מלחמה }\מקור{[ר״מ קלא]}\צהגדרה{. }

\paragraphs

\ערך{בית דין הגדול }\הגדרה{- מרכזנו הדתי, היסוד העיקרי לביאור התורה לפרטיה הנולדים בהמשך החיים, היושב ״במקום אשר יבחר ד׳״, שמשם הוראה צריכה לצאת לכל ישראל }\מקור{[א״ה ב (מהדורת תשס״ב) 127]}\צהגדרה{.}

\paragraphs

\ערך{בית הגדול }\הגדרה{- }\משנה{הבית הגדול }\הגדרה{- כל העולמים כולם בכל הדר\mycircle{°} כונניותם\mycircle{°} }\מקור{[ר״מ קמג]}\צהגדרה{.}

\הגדרה{ע׳ במדור פסוקים ובטויי חז״ל, בית ד׳.}

\paragraphs

\ערך{בית הכנסת }\הגדרה{- מקום הקיבוץ הציבורי לעבודת\mycircle{°} השי״ת\mycircle{°} }\מקור{[ע״א א א נו]}\צהגדרה{.}

\הגדרה{המכון לקיבוץ עבודת השי״ת והרמת כח האמונה\mycircle{°} ויראת\hebrewmakaf ד׳\mycircle{°} בלבבות }\מקור{[ע״א ג א כה]}\צהגדרה{.}

\הגדרה{(בית) שתעודתו היא עבודת השם ית׳, <שהוא המקום\mycircle{°} היותר גבוה שבחיים, שכל פינות החיים הפרטיים כולן אליו יפנו וע״י יתעלו ויתרוממו}\צהגדרה{> }\מקור{[עפ״י שם]}\צהגדרה{.}

\צהגדרה{הבית של ההתכנסות הפנימית לשם ד׳\hebrewmakaf אלהי\hebrewmakaf ישראל\mycircle{°}, להופעת רוחו ושייכות מצוותו\mycircle{°} }\צמקור{[ל״י א נו].}

\ערך{״אחורי בית הכנסת״}\myfootnote{ ברכות ו:.\label{7}}\הגדרה{ - }\צהגדרה{באיכותו וערכו}\הגדרה{ - הצד הטפל של בית הכנסת - השגת המבוקש בתפילה\mycircle{°} }\מקור{[עפ״י פנק׳ ג ער]}\צהגדרה{. }

\ערך{פנים בית הכנסת}\הגדרה{ - }\משנה{באיכותו וערכו}\הגדרה{ - החלק העיקרי של בית הכנסת - תכליתו להרבות כבודו\mycircle{°} של השי״ת בלב כל הנכנסים בתוכו, ולתכלית זו באה התפילה\mycircle{°}}\צהגדרה{ }\מקור{[עפ״י פנק׳ ג ער]}\צהגדרה{.}

\הגדרה{ע׳ בנספחות, מדור מחקרים, רנה ותפילה. ע׳ במדור פסוקים ובטויי חז״ל, המתפלל אחורי בית הכנסת נקרא רשע. ושם, מסיר אזנו משמוע תורה גם תפילתו תועבה. }

\paragraphs

\ערך{בית הכסא }\הגדרה{- מקום התגלות שפלות החומריות האנושית מצד עולמו הפנימי }\מקור{[ע״א ג א יג]}\צהגדרה{.}

\מעוין{◊}\הגדרה{ האמצעי לטהרה\mycircle{°} של הלכלוך הטבעי המחליא באי נקיונו בהשארתו בגויה }\מקור{[ע״א ג ב עג]}\צהגדרה{.}

\הגדרה{ע׳ במדור מלאכים ושדים, שעיר של בית הכסא.}

\paragraphs

\ערך{בית המרחץ }\הגדרה{- מקום התגלות שפלות החומריות האנושית מצד עולמו החיצוני }\מקור{[ע״א ג א יג]}\צהגדרה{.}

\paragraphs

\ערך{בך }\הגדרה{- מבטא את היחש החודר בפנימיותו של נושא חוצי, העומד לנכח הנושא העצמי, המביע את הרעיון }\מקור{[ר״מ קלב]}\צהגדרה{.}

\paragraphs

\ערך{בכור }\הגדרה{- }\משנה{(ענינו) }\הגדרה{- היסוד הראשי של משך החיים}\צהגדרה{ }\מקור{[עפ״י ע״ר א מב]}\צהגדרה{.}

\paragraphs

\ערך{בכורה }\הגדרה{- תכונה, שמחיבת להיות משפיע ופועל פעולה חנוכית על יתר הבנים והבנות, שבאים אחריו }\מקור{[ע״ר א קו]}\צהגדרה{.}

\הגדרה{ע׳ במדור מצוות, הלכות, מנהגים וטעמיהן, בכורות, קדושת הבכורות. }

\paragraphs

\ערך{בל }\הגדרה{- הוראת שלילה }\מקור{[ר״מ קלג]}\צהגדרה{. }

\paragraphs

\ערך{בם }\הגדרה{- מורה חדירת הנושא בתוך התוכנים הרבים העומדים בריחוק מקום מהנושא המתאר }\מקור{[ר״מ קלג]}\צהגדרה{. }

\paragraphs

\ערך{בן }\הגדרה{- התולדה האיתנה, העובדת והמסדרת, היורשת את ההארות\mycircle{°} העליונות שהן הן הגורמות את החידוש התולדתי }\מקור{[ר״מ קלד]}\צהגדרה{. }

\paragraphs

\ערך{בן}\הגדרה{ - }\מעוין{◊ }\הגדרה{המיוחש לאב\mycircle{°} ביחש הקשר הנשמתי היותר חזק }\מקור{[עפ״י ע״ר א פו]}\צהגדרה{.}

\הגדרה{ע׳ במדור פסוקים ובטויי חז״ל, נחלת ד׳ בנים. }

\paragraphs

\ערך{בן}\הגדרה{ - }\משנה{״בנים״}\הגדרה{ - הצעירות, הצריכה לקבל את השפעתו\mycircle{°}, של האב\mycircle{°} הגדול }\מקור{[ע״ר ב סה]}\צהגדרה{.}

\paragraphs

\ערך{בן }\הגדרה{- }\משנה{(בעבודת ד׳ לעומת עבד) }\הגדרה{- ע׳ במדור מדרגות והערכות אישיותיות. }

\paragraphs

\ערך{בסום }\הגדרה{- }\משנה{התבסמות העולם}\myfootnote{ בש״ק, קובץ א קט: ״\textbf{התבסמות} העולם ע״י כל המשך הדורות, ע״י \textbf{ביסום} היותר עליון של גילויי השכינה בישראל וע״י נסיונות הזמנים, התגדלות היחש החברותי, והתרחבות המדעים, זיקקה הרבה את רוח האדם, עד שאע״פ שלא נגמרה עדיין טהרתו, מ״מ חלק גדול מהגיונותיו ושאיפת רצונו הטבעי הנם מכוונים מצד עצמם אל הטוב האלהי״. ושם שצד: ״העולם ב\textbf{התבסמותו} הולך הוא ומתעלה בתוכיותו. האדם מוצא את חפצו, הולך וטוב בערכו הפנימי״.\label{8}}\הגדרה{ - תיקון\mycircle{°} והתעלות }\מקור{[עפ״י פנק׳ ג שט, שי]}\צהגדרה{.}

\ערך{בסומה של הנשמה}\הגדרה{ - שאיבתה ממעין הקדושה\mycircle{°} האלהית הפנימית. סוד הקדושה האצילית\mycircle{°} הפנימית, הופעת הנשמה }\מקור{[עפ״י ע״ר א קנג]}\צהגדרה{.}

\הגדרה{ע״ע מתבסם.}

\paragraphs

\ערך{בצבוץ }\הגדרה{- תיאור לכל רושם צמחני }\מקור{[עפ״י ר״מ קלו]}\צהגדרה{. }

\paragraphs

\ערך{בצבוץ }\הגדרה{- ההפריה המתבודדת בחוגיה, וההתגלות הקולית }\מקור{[ר״מ קלו]}\צהגדרה{. }

\paragraphs

\ערך{בִּצָה }\הגדרה{- מקום המוכשר לגידול, מסמל את הבסיס הדוגמתי בהתוכנים הרוחנים\mycircle{°}, בגליפת המושגים, במהות השכלתם וציור\mycircle{°} אמיתת צדקם\mycircle{°}, המשפיעים על היסוד המעשי, ערכי הצדק\mycircle{°} והמישרים\mycircle{°} }\מקור{[ר״מ קלו]}\צהגדרה{. }

\paragraphs

\ערך{בק }\הגדרה{- ענין של התרוקנות }\מקור{[ר״מ קלו]}\צהגדרה{. }

\paragraphs

\ערך{בֹּקֶר }\הגדרה{- עת\mycircle{°} ההזרחה\mycircle{°} של האורה\mycircle{°} האמיתית, אשר תביא להכרת החיים במהותם העצמית }\מקור{[ע״ר ב עג]}\צהגדרה{. }

\paragraphs

\ערך{בקורת }\הגדרה{- }\משנה{מטרת הבקורת }\הגדרה{- להגיה אורות מאופל. לברר בחופש והרחבה, על צד השקר המועט, המוכרח להמצא בתוך האמת הגדולה והמרובה, ועל ניצוץ האמת המתגלה בתוך עומק החושך של השקר }\מקור{[מ״ר 288]}\צהגדרה{.}

\paragraphs

\ערך{בקשת אלהים }\הגדרה{- דרישת חיים של אמת פנימית }\מקור{[עפ״י קובץ ד פז]}\צהגדרה{.}

\הגדרה{ע׳ במדור פסוקים ובטויי חז״ל, דרישת ד׳.}

\paragraphs

\ערך{בר }\הגדרה{- המזון המבריא בכל הערכים }\מקור{[ר״מ קלז]}\צהגדרה{. }

\paragraphs

\ערך{בר }\הגדרה{- הבנה של חוצה, העומד(ת) מחוץ להפרגוד\mycircle{°} אשר חביון\hebrewmakaf עז\mycircle{°} קודש הקדשים של הדממה העליו(נה) אצור שמה }\מקור{[עפ״י ר״מ קלח]}\צהגדרה{. }

\paragraphs

\ערך{בר }\הגדרה{- תרגום בן\mycircle{°} }\מקור{[עפ״י ר״מ קלח]}\צהגדרה{. }

\paragraphs

\ערך{ברה }\הגדרה{- מנצחת את כל צללי המחשכים }\מקור{[ע״ר ב נז]}\צהגדרה{. }

\paragraphs

\ערך{ברוך }\הגדרה{- }\משנה{(משמעותו בברכת\hebrewmakaf המצוות\mycircle{°})}\הגדרה{ - מקור\hebrewmakaf חיי\hebrewmakaf החיים\mycircle{°}, אוצר הטוב והקודש, ששפעת כל ברכת ההויה שמה היא גנוזה, המוער בבאנו להוציא מן הכח אל הפועל את האור הקדוש של מצוה\mycircle{°} מעשית, בהתגלות המפעלית, ובהארת היפעה האלהית, הנובעת מראש מקור אור החיים העליונים של חי\hebrewmakaf העולמים\mycircle{°}, מתמשכת אז שפעת ברכה\mycircle{°}, ההולכת ומפלסת לה את נתיבה בהתחשפות האורה של החיים המעשיים, ומעין החיים מתברך\mycircle{°} במקורו, בהיותו מוכן להתגבר בשטף ברכותיו ע״י אותו השביל החדש, ההולך ומתבלט ע״י מפעלנו במעשה המצוה הבאה ממרום החפץ האלוהי העליון, מקור החיים והטוב, אל תחתית מעמקי העולם, הנמצר במצריו החמריים וכוחותיו המעשיים }\מקור{[עפ״י ע״ר א ז]}\צהגדרה{. }

\paragraphs

\ערך{בריאה }\הגדרה{- }\משנה{הבריאה }\הגדרה{- הממשיות המוגלמת }\מקור{[א״א 125]}\צהגדרה{. }

\paragraphs

\ערך{בריאה}\myfootnote{ הזכיר הרב מרדכי גלובמן בנ״א ה עמ׳ 22 מדברי אבן עזרא, בראשית א א ״רובי ממפרשים אמרו שהבריאה להוציא יש מאין וכן ״אם בריאה יברא ה׳״. והנה שכחו ״ויברא אלהים את התנינים״ ושלש בפסוק אחד ״ויברא אלהים את האדם״. ו״ברא חשך״ שהוא הפך האור שהוא יש. ויש דקדוק המלה ברא לשני טעמים: זה האחד, והשני ״לא ברה אתם לחם״. וזה השני ה״א תחת אלף כי כמוהו ״להברות את דוד״ כי הוא מהבנין הכבד הנוסף ואם היה באל״ף היה כמו ״להבריאכם״ ומצאנו מהבנין הכבד ״ובראת לך״. ואיננו כמו ״ברו לכם איש״ רק כמו  ״וברא אותהן בחרבותם״ (יחזקאל כג מז). וטעמו לגזור ולשום גבול נגזר והמשכיל יבין״. וכן כתב הרמ״ק בפרדס שער אבי״ע פרק א ״בריאה מלשון ברא [...] כונתו חוץ, או מלשון כריתה, כמו ״כי יער הוא ובראתו״ (יהושע יז)״. ובראש אמנה לאברבנאל: ״ברא״ הונח בהנחה ראשונה יש מאין, ומזה הושאל על כל בריאה ניסית או נשגבה היוצאת מגדר הטבע. אמנם גם פירושי ראשונים אלו לא יעמדו במבחן דברי האדרת אליהו, בראשית א א, ד״ה ברא ״הבינו כל מפרשי הדת שמורה על דבר מחודש יש מאין. אבל מה יאמרו ״ויברא אלהים התנינים הגדולים״ וכן ״ויברא אלהים את האדם בצלמו״ וכן מה שתקנו קדמונינו בכל ברכת הנהנין ״בורא פרי האדמה״, ״בורא פרי העץ״, ונשאר כללם הידוע מעל״. על כן פירש הגר״א שם ״מלת בריאה הונח להורות על חידוש העצם אשר אין בכח הנבראים אפי׳ כולם חכמים ונבונים לחדשו... וכן תיקנו ״בורא פרי״ כי אינו בכח כל הנבראים לחדשו בעבור שהוא עצם פועל ה׳״. ״ברא הוא עצם הדבר ואפילו יש מיש״. ובמטפחת ספרים ליעב״ץ, פרק ח ד ״לשון בריאה מורה על יש מיש על דרך האמת. ויתכן גם בריאה אין מיש״ וכו׳ עש״ע שהאריך, (ע״ע ע״ט לב ד״ה לא). וע׳ בדרך חיים למהר״ל, רי ושם שכא ״לשון בריאה נאמר על הצורה הנבדלת האלקית שדבק בנבראים, וזה כי האדם כתיב בפי׳ בצלם אלקים עשה את האדם, שתדע מזה כי דבק בצורת האדם ענין אלקי, וכן בשמים וארץ שהם כלל העולם, אין ספק שדבק בהם ענין אלקי ולכך כתיב לשון בריאה. וכן התנינים הגדולים שהכתוב מפרש שהם תנינים גדולים, ולפי גדלם עד שהם בריאה נפלאה דבק בהם ענין האלקי נאמר אצלם לשון בריאה. כי כל הנבראים יש בהם דבר זה כמו שיתבאר רק התורה הזכירה לשון בריאה באלו שלשה, כי באלו שלשה מפורסם ונראה לגמרי לעין ובשאר דברים אינו נראה״. והרש״ט גפן, בממדים, הנבואה והאדמתנות, תורת הנבואה הטהורה, מאמר שני, עיון בנבואה ובמופתים, פרק יג הגדיר בריאה: ״יציאת היש ממה שאיננו נופל תחת הציור באופן בלתי מובן ובלתי מושג לא לשכל ולא לחוש ומבלעדי כל הכרח״. עע״ש פרק יד. ושם, מעשה בראשית והאדמתנות, סוף דבר, א, הגדיר: ״הבריאה הוא השתכללות צורות הזמן והמקום על פי כוח נסתר ונעלם, בדעת האדם והכרתו״. עע״ש הערה 11. ובקסת הסופר לר״א מרקוס, בראשית א א ״ברא - הוציא יש מאין שלא כפי טבע הנברא״, עע״ש עמ׳ ב-ד בהרחבת דברים נפלאה. ע״ע מנֹפת צוף, למו״ר הרב יהונתן שמחה בלס, ח״ב עמ׳ 825 ״המונח ״ברא״ ראוי להמצאת מציאות ראשונית שלאחר מכן נותרה כבררת מחדל״. כדרכו של הרב ברב דבריו, על פי הסברו בסוגיה יעלו כל הפירושים בקנה אחד.\label{9}}\הגדרה{ - התהוות\mycircle{°} העולם וכל אשר לו מאותו החפץ\mycircle{°} הקדום\mycircle{°}, המלא עז\mycircle{°}, המעוטר בגבורה\mycircle{°} ובכל אור\mycircle{°} קדשי\hebrewmakaf קדשים\mycircle{°}, עדינות הטוב\mycircle{°}, החסדים\mycircle{°} הנאמנים עדי\hebrewmakaf עד\mycircle{°} }\מקור{[א״ק ג ע]}\צהגדרה{. }

\הגדרה{היש המצומצם שאנו פוגשים בציור\mycircle{°} של הויה, שבחופש\mycircle{°} ולמטרה ידועה, נלחצה בצמצומה}\צהגדרה{ }\מקור{[עפ״י קובץ ז קנא]}\צהגדרה{.}

\הגדרה{היצירה המוחלטה. היכולת הבלתי תנאית ממציאה הכל, על\hebrewmakaf פי היסוד החפצי}\צהגדרה{ }\מקור{[עפ״י קובץ ה קפה]}\צהגדרה{.}

\צהגדרה{הוצאת יש מאין}\צמקור{ [ק״ת עז].}

\משנה{בריאת העולם }\הגדרה{- הופעת\mycircle{°} האור של הקדושה\mycircle{°} העליונה\mycircle{°} התורנית\mycircle{°},  בתור אור\hebrewmakaf החיים\mycircle{°} של קבלת\hebrewmakaf מלכות\hebrewmakaf שמים\mycircle{°}, של כבוד\hebrewmakaf המלכות\mycircle{°}, המאיר בהויה והיצירה כולה }\מקור{[עפ״י ע״ר א קיא]}\צהגדרה{. }

\תערך{בריאה ראשונה }\הגדרה{- }\תמשנה{״בראשית ברא״ }\הגדרה{-}\תהגדרה{ (התהוות) שלא על דרך השתלשלות\mycircle{°} אלא בכונה ראשונה }\תמקור{[עפ״י נ״א ה 22-21]. }

\הגדרה{ע״ע נברא. ע״ע מחשבה אלהית על דבר העולם. ע׳ בנספחות, מדור מחקרים, תכלית הבריאה. ע׳ במדור שמות כינויים ותארים אלהיים, בורא. ר׳ יצירה. ר׳ עשיה.}

\paragraphs

\ערך{בריאה }\הגדרה{- }\משנה{עולם הבריאה }\הגדרה{- ע׳ במדור מונחי קבלה ונסתר. }

\paragraphs

\ערך{בריאות }\הגדרה{- המצב הטוב המסכים אל כלל המציאות, מבלי שיופרע הסדר ביציאת פרט אחד מהסכמתו אל הכלל }\מקור{[עפ״י ע״א א ה נח]}\צהגדרה{. }

\paragraphs

\ערך{בריאות רוחנית}\הגדרה{ - }\משנה{הבריאות הרוחנית}\הגדרה{ - ההרגשות הנפשיות כולן, במצבן הנורמלי. הרגשת היופי, האהבה, נטיית הגבורה, חשק החיים הבריאים}\צהגדרה{ }\מקור{[עפ״י קבצ׳ ב קכז]}\צהגדרה{.}

\paragraphs

\ערך{בריקה }\הגדרה{- פעולת התקפה (רוחנית) חזקה בפתאומיותה }\מקור{[רצי״ה א״ש ב הערה 3]}\צהגדרה{. }

\הגדרה{ע׳ בנספחות, מדור מחקרים, אור, זיו, ברק. }

\paragraphs

\ערך{ברירות }\הגדרה{- הזיכוך המחשבי והמעשי }\מקור{[ר״מ קלח]}\צהגדרה{. }

\paragraphs

\ערך{ברית }\הגדרה{- קשר שכלי, נמוסי או טבעי, בין שני נושאים }\מקור{[עפ״י ע״ר א שפד]}\צהגדרה{. }

\הגדרה{ע״ע אמונה בברית. ע׳ במדור פסוקים ובטויי חז״ל, זכירת הברית. }

\paragraphs

\ערך{ברית }\הגדרה{- }\משנה{שורש ברית, וכריתת ברית במובן המוסרי }\הגדרה{- שהענין החיובי ואידיאלי\mycircle{°}, הנובע מתמצית המוסר\mycircle{°} היותר נעלה ונשגב, יהיה מוטבע עמוק וחזק בכל טבע הלב והנפש, עד שלא יצטרך לא זירוז ולא חזוק ולא סייג לשמירתו, כי\hebrewmakaf אם יהיה מוחש וקבוע, כמו שטבועה, למשל, בלב אדם ישר\mycircle{°} מניעת רציחה וכדומה מן השלילות [של] הרעות שכבר הספיק כח המוסר הכללי לקלטן יפה }\מקור{[מ״ה ברית א (פנ׳ ה)]}\צהגדרה{. }

\paragraphs

\ערך{ברית }\הגדרה{- }\משנה{יסוד הברית }\הגדרה{- הפעולות המוסריות\mycircle{°}, ביחוד הדתיות, המכוונות לכבד את ד׳\mycircle{°} לפי ציור\mycircle{°} המדמה\mycircle{°}. השגחת\hebrewmakaf ד׳\mycircle{°}, שימצאו דתות לכל אומה, המחזקות את הצדק בעולם. ושתמצא בהן אחת יסודית, שמחזקת מעוז הציורים האמיתיים, ומקשרתם אל השכל המעשי }\צהגדרה{- }\הגדרה{תורת\hebrewmakaf ישראל\mycircle{°} המאירה באורה הפנימי בבית ישראל ומפיצה קרנים ג״כ לבני נח. והוא אות ברית בין אלקים ובין האדם בכללו, שמונע עכ״פ מהשחתה }\מקור{[עפ״י פנק׳ א קמד (קבצ׳ א נז)]}\צהגדרה{. }

\הגדרה{ע׳ במדור מונחי קבלה ונסתר, קשת. }

\paragraphs

\ערך{ברית }\הגדרה{- }\משנה{פגם הברית }\הגדרה{- ע׳ במדור הנטייה המינית.}

\paragraphs

\ערך{ברית }\הגדרה{- הטבע של קדושת היהדות\mycircle{°}. עצם ההויה הנפשית והטבע הרוחני וגם הגופני, של הכלל\mycircle{°} כולו ושל כל אחד ואחד מישראל. הטבע היהדותי במעשה, ברעיון, ברגש ובמחשבה, ברצון ובמציאות }\מקור{[עפ״י א״ש יז ד]}\צהגדרה{.}

\משנה{קדושת הברית }\הגדרה{- אור הטבע הישראלי הנקי }\מקור{[א׳ מד]}\צהגדרה{.}

\הגדרה{ע׳ במדור פסוקים ובטויי חז״ל, הפרת ברית. ע׳ במדור מלאכים ושדים, אליהו. }

\paragraphs

\ערך{ברית }\הגדרה{- המושג העצמי של התוכן אשר לנצח\mycircle{°} העומד למעלה מכל מושג מוסבר באיזה הגיון\mycircle{°} מוגבל }\מקור{[ע״ר א רב (ע״א ב ט קנז)]}\צהגדרה{. }

\מעוין{◊ }\הגדרה{הברית מיוסדת על תוכן קים, שאיננו נופל תחת שום שינוי }\מקור{[שם צז]}\צהגדרה{. }

\ערך{ברית }\הגדרה{- }\משנה{ תוכן הברית }\הגדרה{- הזכרון\mycircle{°} העולמי שאינו סובל שום הגבלה ציורית\mycircle{°} כלל }\מקור{[עפ״י שם רב]}\צהגדרה{. }

\הגדרה{ע׳ במדור מונחי קבלה ונסתר, רזא דברית.}

\paragraphs

\ערך{ברית }\הגדרה{- }\משנה{(לעומת חסד\mycircle{°}) }\הגדרה{- מעלת בטחונו וחוזק מציאותו, של כל דבר נעלה בחיי\hebrewmakaf הרוח\mycircle{°} המתפשט במציאות }\מקור{[עפ״י ע״ר א פג]}\צהגדרה{. }

\הגדרה{הודאיות\hebrewmakaf המוחלטת\mycircle{°} }\מקור{[עפ״י א״ק א רז, ע״ר א פג\hebrewmakaf פד, רב]}\צהגדרה{. }

\הגדרה{ע׳ במדור מונחי קבלה ונסתר, אחרית, לעומת הראשית בחיי הרוח. ע׳ במדור פסוקים ובטויי חז״ל, ברית עולם. ושם, נתתי את תורתי בקרבם ועל לבם אכתבנה.}

\paragraphs

\ערך{ברית }\הגדרה{- }\משנה{הברית שכרת ד׳ עם ישראל }\הגדרה{- שאי אפשר כלל שיהיה ח״ו כלל\hebrewmakaf ישראל נבדל ונפרד מקדושת שמו\mycircle{°} הגדול ב״ה }\מקור{[מ״ש שיד (מא״ה ג רג)]}\צהגדרה{. }

\משנה{כריתת ברית שכרת השי״ת עם ישראל }\הגדרה{- שאע״פ שהזמן פועל שינויים גדולים בעולם, ובני האדם הפועלים בזמן הם חפשים בבחירתם\mycircle{°} והענין ארוך מאד, א״כ היה נראה לכאורה שאפשר הדבר שיצאו הדברים בכללם חוץ למטרת החכמה\hebrewmakaf העליונה\mycircle{°} שכיון הבורא יתברך ח״ו, ע״י בני\hebrewmakaf אדם הפועלים שינויים רבים בבחירתם ע״י הזמן. ע״כ השי״ת בחר\hebrewmakaf בישראל\mycircle{°} וצוה אותם לקדש חדשים ושנים. פי׳ שע״י כחן של ישראל ופעולתן בעצמם ובעולם, תהי׳ ערובה בטוחה שכל הדברים יחזרו אל תכליתם, והזמן יפעול פעולה מקודשת\mycircle{°}, היינו פעולה המגעת אל התכלית העליונה שכיון השי״ת ולא פעולה של חול\mycircle{°} }\מקור{[מ״ש שנ]}\צהגדרה{. }

\הגדרה{ע׳ במדור פסוקים ובטויי חז״ל, ברית עולם. ע׳ במדור מועדים וחגים, קידוש הזמנים.}

\paragraphs

\ערך{ברכה}\myfootnote{ רקאנאטי עה״ת עקב: ״הברכה היא אצילות תוספת המשכה מאפיסת המחשבה שהיא מקור החיים״. ובשל״ה עה״ת, וזאת הברכה, תורה אור, ד״ה וכבר כתבתי: ״ענין ברכה הוא התפשטות בשפע רב תמיד נצחי״. \label{10}}\הגדרה{ - תוספת אור\mycircle{°} ויתרון }\מקור{[עפ״י א״ק ב תקלד]}\צהגדרה{. }

\הגדרה{תוספת חיים עצמיים מקוריים }\מקור{[עפ״י שם רצד]}\צהגדרה{. }

\הגדרה{תוספת מעלה, הופעה\mycircle{°} ועליה\mycircle{°} }\מקור{[ע״ר א ריז]}\צהגדרה{. }

\הגדרה{ההוספה התמידית במעלה }\מקור{[שם]}\צהגדרה{. }

\הגדרה{התוספת התדירית, בשפעת\mycircle{°} אור הקדש\mycircle{°} וחיי האמת\mycircle{°} }\מקור{[שם]}\צהגדרה{. }

\הגדרה{התוספת וההגדלה }\מקור{[שם סב]}\צהגדרה{. }

\הגדרה{שפעת חידוש\mycircle{°} ומקור חיים }\מקור{[ע״א ד ט ק]}\צהגדרה{. }

\הגדרה{שפעת\mycircle{°} חיים טובים, נעימים ורעננים\mycircle{°} }\מקור{[עפ״י א״ק ג קפח]}\צהגדרה{. }

\הגדרה{שפע החיים, העז והעצמה }\מקור{[ע״ר א ריד]}\צהגדרה{. }

\הגדרה{השפעה והשלמה}\צהגדרה{ }\מקור{[עפ״י ע״א ג ב מט]}\צהגדרה{.}

\הגדרה{ענין הַבְרָכָה ובְּרֵכַת\hebrewmakaf מים המשקה את הארץ }\מקור{[מא״ה ד קסד]}\צהגדרה{.}

\הגדרה{ע״ע מתברך. ע״ע מברך. ע׳ במדור מונחי קבלה ונסתר, ״יחוד ברכה קדושה״. ע״ע קדושה.}

\ערך{ברכה }\הגדרה{- }\משנה{ברכה המושפעת מחסד\mycircle{°} אל עליון }\הגדרה{- האורות\hebrewmakaf העליונים\mycircle{°} כשהם מופיעים על הנשמה\mycircle{°}, על נשמת\hebrewmakaf הכלל\mycircle{°} ועל נשמת הפרט, המרחיבים את מהותה, מעצמים את הויתה, ומעלים אותה למרומי האושר\mycircle{°} הנצחי\mycircle{°} }\מקור{[עפ״י ע״ר א קנח]}\צהגדרה{. }

\ערך{ברכה }\הגדרה{- }\משנה{הברכה היסודית של העולם }\הגדרה{- ההתעלות\mycircle{°} התדירית של דעת\hebrewmakaf ד׳\mycircle{°}, בגדלה\mycircle{°}, ביפעתה\mycircle{°} ובטהרתה\mycircle{°} }\מקור{[ע״ר א נ]}\צהגדרה{. }

\ערך{ברכה }\הגדרה{- }\משנה{הברכה הכללית }\הגדרה{- הברכה הנכנסת בעומק הפנימי של החיים, ברכת שלום\mycircle{°} הפנימי שעל ידה ימצא האדם שהחיים המה טובים כשהם לעצמם, וממילא אין עמם מחסור כשהם מתמלאים עם הדרישות המעשיות }\מקור{[ע״א ג ב רכו]}\צהגדרה{.}

\הגדרה{ע׳ ברוך.}

\paragraphs

\משנה{ברכה }\צהגדרה{- }\מעוין{◊ }\צהגדרה{הזכרת השם\mycircle{°} היא היסוד הפנימי השרשי של הברכה, והמלכות\mycircle{°} היא מהותה העצמית הממשית}\צמקור{ [א״ל קיט].}

\paragraphs

\ערך{ברכה לעומת הודאה }\הגדרה{- ע׳ בנספחות, מדור מחקרים. }

\paragraphs

\ערך{״ברכה לצורך״ }\הגדרה{- }\מעוין{◊}\הגדרה{ ההופעות\mycircle{°} העליונות\mycircle{°}, המופעות בנשמתנו מעולמי התעלומה, תפקידן הוא לרומם בנו את התוכן המהותי של כל עצמות חיינו אל רום הנצח\mycircle{°}, אל ההוד\mycircle{°} האלהי הנשגב ברוממות קדשו. ואם יופיעו המון הארות\mycircle{°}, ורכוז לא יהיה להם בעצמיות מהות החיים שלנו, הרי הן לנו כאבודות. על כן אין לנו רשות לברך ברכה כי אם לצורך, וברכה שאינה צריכה, ומה גם ברכה לבטלה\mycircle{°}, הרי היא לנו נשיאת עון וחלול שם שמים. הבהקת האורה הרוחנית מתגברת היא בתעצומתה על ידי הברכה, מתאדרת היא ההארה הרוחנית בנשמתנו מעולם הנעלם, בא לידי גלוי ע״י בטוי הברכה המון רב של מחשבות רוממות וציורים נאדרים בקודש. ומתי הם לברכה באמת - בזמן שיש להם רכוז בנטית החיים שלנו במהלך קדשם. }\צהגדרה{ברכה לצורך }\הגדרה{- צורך גבוה וצורך הדיוט }\מקור{[ע״ר א לא]}\צהגדרה{. }

\הגדרה{ברכה שיצאה מפינו והשיגה את הרכוז בחיי המפעל ובהכרה המפורשה }\מקור{[עפ״י שם]}\צהגדרה{. }

\paragraphs

\ערך{ברכת ד׳ }\הגדרה{- החיים המלאים צדק\mycircle{°} ע״פ תכונתם ואופיים }\מקור{[ע״א ב ט ער]}\צהגדרה{.}

\paragraphs

\ערך{ברכת ד׳ }\הגדרה{- }\משנה{״לברך את שמך״ }\הגדרה{- ע׳ במדור פסוקים ובטויי חז״ל. }

\paragraphs

\ערך{ברכת הדיוט הצריכה לגבוה}\צהגדרה{ - }\הגדרה{ע׳ במדור פסוקים ובטויי חז״ל. }

\paragraphs

\ערך{בת }\הגדרה{- תיאור המין הנקבי שבכל נושא, ביחש להערך של המוליד והמחדש אותו, או המוציאו עכ״פ אל הפועל הגמור, כלומר ביחש ההורים האב\mycircle{°} או האם\mycircle{°}, או ביחש להנושא המיני הזכרי, שגם הוא מיוחש אל ההורים כלומר הבן\mycircle{°} }\מקור{[ר״מ קמ]}\צהגדרה{. }

\paragraphs

\ערך{בת }\הגדרה{- מלת תואר, המדה }\מקור{[ר״מ קמ]}\צהגדרה{. }\mylettertitle{ג}

\ערך{גא }\הגדרה{- תאור הרוממות הגשמית }\מקור{[עפ״י ר״מ קמא]}\צהגדרה{. }

\paragraphs

\ערך{גאוה עליונה }\הגדרה{- }\משנה{הגאוה העליונה }\הגדרה{- }\מעוין{◊}\הגדרה{ כשאור הקודש\hebrewmakaf העליון\mycircle{°} של המקור הראשי אשר להחכמה\mycircle{°} הקדומה מופיע בהארת אורה\mycircle{°} בתוך הגמול\mycircle{°} העולמי, ומשפט\mycircle{°} הצדק\mycircle{°} מבהיק את אורו בבהירותו, }\משנה{הגאוה העליונה}\הגדרה{ מתגלה בעולם }\מקור{[ר״מ קמא]}\צהגדרה{. }

\הגדרה{ע׳ במדור תיאורים אלהיים, גאות ד׳. }

\paragraphs

\משנה{גאונות }\צהגדרה{- מקוריות יסודיות }\צמקור{[שי׳ א 67].}

\paragraphs

\ערך{גאונות רוחנית }\הגדרה{- עצם כחות הנפש, קדושת הרגש וגדולת הכשרונות, בהתקבצם יחד, באיש אחד מיוחד ומצויין}\צהגדרה{ }\מקור{[עפ״י א״י כד]}\צהגדרה{.}

\paragraphs

\ערך{גאונים }\הגדרה{-}\משנה{ תקופת הגאונים }\הגדרה{- התקופה הגדולה שאחר חתימת התלמוד\mycircle{°}, שהיתה תקופה הרת עולם בחיי הרוח הפנימיים של אומתנו. הנוגעת לעיקרה ויסודה של חכמת\hebrewmakaf ישראל\mycircle{°}. תורת ההלכה\mycircle{°} והאגדה\mycircle{°} של רבותינו הקדמונים בבבל, גאוני סורא ופומבדיתא, אשר מידם נמסר לנו המבצר הגדול לחומת אש דת התלמוד\mycircle{°} הבבלי כולו }\מקור{[מ״ר 315]}\צהגדרה{.}

\paragraphs

\ערך{גאות עולמים }\הגדרה{- גאות קדש\mycircle{°} המתנשא מימות\hebrewmakaf עולם\mycircle{°}, המתעלה מכל עלוי\mycircle{°} ופאר\mycircle{°}, למעלה מכל תכן של שאיפה היותר נאצלה השיכת לכל דבר נברא, גם בהתאחד הכל למטרתו היותר עליונה, שהיא תשוקת קדשם של עם ד׳ }\מקור{[עפ״י ע״ר א קנט]}\צהגדרה{. }

\הגדרה{ע׳ במדור תיאורים אלהיים, גאות ד׳. ושם, מתנשא מימות עולם. ע׳ במדור שמות כינויים ותארים אלהיים, גאה.}

\paragraphs

\ערך{גבהי גבוהים }\הגדרה{- מקור שרשו, חיי נשמתו (של האדם), אור חיי נשמת כל העולמים, אור אל\hebrewmakaf עליון\mycircle{°} טובו\mycircle{°} והדרו\mycircle{°} }\מקור{[עפ״י א״ת י ב]}\צהגדרה{.}

\paragraphs

\ערך{גבוה }\הגדרה{- }\משנה{(לעומת נמוך\mycircle{°}}\הגדרה{) - כללי\mycircle{°} ומופשט\mycircle{°} }\מקור{[עפ״י קובץ ג קז]}\צהגדרה{. }

\paragraphs

\ערך{גבוה }\הגדרה{- }\משנה{(לעומת רם\mycircle{°}) }\הגדרה{- }\מעוין{◊}\הגדרה{ מצטיין ג״כ ביחש להמשך הדבר, ההולך וגבה מתחתית מצבו עד הרום\hebrewmakaf העליון\mycircle{°} }\מקור{[ע״ר א קיב]}\צהגדרה{. }

\paragraphs

\ערך{גבולים }\הגדרה{- זמנים ומקומות, מעשים ומחשבות, שאיפות ורצונות מוגבלים }\מקור{[ע״ר א קצג]}\צהגדרה{. }

\הגדרה{ע״ע מגביל. }

\paragraphs

\ערך{גבולים }\הגדרה{- }\משנה{גבולים וגדרים}\הגדרה{ - המניעות }\מקור{[ע״ט טו]}\צהגדרה{. }

\הגדרה{ע״ע הגבלה. }

\paragraphs

\ערך{גבורה }\הגדרה{- }\משנה{הגבורה המפעלית }\הגדרה{- היכולת להסיר בחוזק יד את כל המפריעים לציורים\mycircle{°} ההבנתיים והידיעתיים מהתגשמותם במעשים }\מקור{[עפ״י ע״א ד י ח]}\צהגדרה{. }

\הגדרה{היכולת להגשים, להוציא אל הפועל }\מקור{[עפ״י א״ק ג קו]}\צהגדרה{. }

\הגדרה{כח רצון כביר וכח מפעל עז, שיוכל להוציא מן הכח אל הפועל כל אשר ברוח עמו}\צהגדרה{ }\מקור{[עפ״י פנק׳ א שצט]}\צהגדרה{.}

\ערך{גבורה }\הגדרה{- היכולת לעמוד נגד כל מהרס המציאות ומחריבה, העמדה נגד כל העומד להחריב ולהפריע את הטוב\mycircle{°} ואת ההשפעה הראויה להביא תועלת לברואים }\מקור{[ע״ר א קג]}\צהגדרה{. }

\paragraphs

\ערך{גבורה }\הגדרה{- }\משנה{כשרון הגבורה }\הגדרה{- כיבוש החיים, ישוב העולם, וההתכוננות המעשית לכל פרטי פרטיה }\מקור{[קובץ ו קנז]}\צהגדרה{. }

\paragraphs

\ערך{גבורה }\הגדרה{- אומץ הנפש של שכלול הרצון, בתגבורת עזוזו, היראה\mycircle{°} }\מקור{[עפ״י מ״ר 100, וקבצ׳ ב קסו]}\צהגדרה{. }

\הגדרה{ע׳ במדור מונחי קבלה ונסתר, גבורות הגבורה.}

\paragraphs

\ערך{גבורה }\הגדרה{- }\משנה{נאזר בגבורה }\הגדרה{- מלא עז\mycircle{°} חיים אמיצים ומכובדים }\מקור{[פנק׳ ד תמח]}\צהגדרה{.}

\paragraphs

\ערך{גבורה }\הגדרה{- }\משנה{מדת הגבורה (האלהית) }\הגדרה{- ע׳ במדור מונחי קבלה ונסתר.}

\paragraphs

\ערך{גבורה }\הגדרה{- }\משנה{(לעומת גדולה)}\הגדרה{ - ע׳ בנספחות, מדור מחקרים, גדולה וגבורה הבדל שביניהם. }

\paragraphs

\ערך{גבורת האמת }\הגדרה{- ע״ע אמת, גבורת האמת. }

\paragraphs

\ערך{גבישי }\הגדרה{- משוכלל במלוא מדתו ותאר חמרו\mycircle{°} }\מקור{[עפ״י רצי״ה א״ש יב הערה 16]}\צהגדרה{. }

\paragraphs

\ערך{גבר }\הגדרה{- איש\mycircle{°} באנשים עם מילוי כל כחותיו בהקפה שלמה, חי ופועל }\מקור{[ע״א ג ב קעב]}\צהגדרה{. }

\הגדרה{ע״ע ״אנוש״. ע״ע ״אדם״.}

\paragraphs

\ערך{גברת }\הגדרה{- אשה\mycircle{°} ראויה להנהגה }\מקור{[ע״א ב ז מז]}\צהגדרה{.}

\paragraphs

\ערך{גג }\הגדרה{- הבניה העילית שבמושב האדם }\מקור{[ע״א ד ח טז]}\צהגדרה{.}

\הגדרה{למעלה מהתכונה הרגילה של המון החיים, יושבי שפל, בכל תביעות חייהם ורגשות נפשם }\מקור{[ע״א ד ח טז]}\צהגדרה{.}

\paragraphs

\ערך{גד }\הגדרה{- }\משנה{יסוד הגד }\הגדרה{- הגדה מראש את גורל האדם, הנמשך מההמשכה\mycircle{°} הנמשכת בקביעות איתנית, כפי הטבע הקבוע, מבלי הבא בחשבון את אשר תחולל יד האדם בבחירתו לשנות בו }\מקור{[עפ״י ע״א ד ו צט]}\צהגדרה{. }

\משנה{גד }\הגדרה{- המשכה תדירית ההולכת ומגרת את השפעתה בהמשכה קבועה, שמתוך כך נקל להגיד גם כן את הגורל האישי למשתקעים בתכונת ההוויה על פי היסודות המוצקים והאיתנים שבה בחומר וברוח <שאין לה שום מקום במציאות לפי האמת> }\מקור{[עפ״י שם]}\צהגדרה{. }

\הגדרה{הסכמת חלקי העולם הכללי, הגשמי והרוחני\mycircle{°},  במבטם והתאחדם זה עם זה, להיות משפיעים את המסלול של צביון החיים המיוחד אשר לכל נושא, שתיאור ההתגמלות של הנושאים, בואם בגבול המפעל, והערכתם לגבי העולם החיצוני לערכם, והכשרון לשאוב מכל מקור המזדמן להם, לינק מכל מבוע של חיים ושפעת כח הבא בהתנגשות עמם, זה כולו מתמם את ערך המזל\mycircle{°}, העושה חטיבה קבועה על המהלכים של הפרטים מתוך המהות הכללית, שרק בהנשא הרוח למעלה מהעיבוי הגבולי, ובהתרוממו ממעל להדלות העולמית, הרי הוא מתנשא למעלה מן המזל, אין מזל לישראל. אבל בהיות ההגמלה מחוברת לשאיבה, ועם דלות ההקצבה, ופתיחת התפיסה המוחשית, או ההצטיירות ההבנית, כשמתאגדים יחד, נעשה המורד מוכן לערך מזלי, ובא גד }\מקור{[עפ״י ר״מ קמד]}\צהגדרה{. }

\הגדרה{ע״ע מזל, יסוד המזל. ע״ע כוכבים. }

\paragraphs

\ערך{״גדול״ }\הגדרה{- מורה על רוממות מעלה }\מקור{[מ״ש קכח (מא״ה ג קמג)]}\צהגדרה{. }

\paragraphs

\ערך{״גדול״ }\הגדרה{- }\משנה{אדם גדול }\הגדרה{- ע׳ במדור מדרגות והערכות אישיותיות, ״אדם גדול״. }

\paragraphs

\ערך{״גדול״ }\הגדרה{- }\משנה{(כינוי לד׳) }\הגדרה{- ע׳ במדור שמות כינויים ותארים אלהיים.  }

\paragraphs

\ערך{גדולה }\הגדרה{- ההצטירות\mycircle{°} הכללית של המעשה אשר עשה האלהים\mycircle{°} }\מקור{[ע״ר א רל]}\צהגדרה{. }

\paragraphs

\ערך{גדולה }\הגדרה{- }\משנה{(לעומת גבורה)}\הגדרה{ - ע׳ בנספחות, מדור מחקרים, גדולה וגבורה הבדל שביניהם. עע״ש רבים, תאר הרבים לעומת תאר הגדולה. }

\paragraphs

\ערך{גדולה }\הגדרה{- }\משנה{(גדולתו של אדון\hebrewmakaf עולם) }\הגדרה{- ע׳ במדור תיאורים אלהיים, גודל עליון. }

\paragraphs

\ערך{גדולה לד׳ }\הגדרה{- ע׳ במדור פסוקים ובטויי חז״ל.}

\paragraphs

\משנה{גדוף }\צהגדרה{- }\צמשנה{(כלפי שמיא) }\צהגדרה{- כמה מדות יש בגדוף, ומעקרו ומכללו הוא מיעוט יחס הכבוד\mycircle{°} וחרדתו - והזילותא שבזה }\צמקור{[להלכות צבור (מהדורת תשע״ט) רצג].}

\מעוין{◊ }\הגדרה{מכח הכרות חשוכות וציורי שקר בהענין\hebrewmakaf האלהי, שכל מיעוט יראה וכבוד הוא בא מיסודו, הולך הרע ומתגבר עד כדי התוכן החשוך של ה}\משנה{גידוף}\הגדרה{. אמנם הגידוף תוכן שלילי יש לו, לא קישור לאיזה ענין, כ״א ניתוק מהאור הטוב המחייה כל העולמים, אבל ההחשכה שבאה ע״י הריחוק, מחוללת תנועה רבה של רשעה בהכחות השפלים }\מקור{[קובץ ה ל]}\צהגדרה{. }

\הגדרה{ע״ע גדפנות.}

\paragraphs

\ערך{גדלות }\הגדרה{- }\משנה{(לעומת קטנות\mycircle{°} באדם)}\myfootnote{ בא״ק א נג-ד ״תור הגדלות תובע מן האדם שלא תהיינה פעולותיו מצות אנשים מלומדה, אלא שכל פעולה וכל הרגל, כל עבודה וכל מצוה, כל רגש וכל רעיון, כל תורה וכל תפלה, תהיה מוארה באור הגנוז, באור הכללי, הגנוז בנשמה העליונה לפני הופעת פעולתם״.\label{1}}\הגדרה{ - כלליות\mycircle{°}. הכנסת האדם את עצמו בחיי הכלל, מתוך הכרת הטוב\mycircle{°} והאור\mycircle{°} באמתת עצמם, כשהאדם שוכח מעט את עצמו, את פרטיותו, והטוב הכללי לוקח את לבבו, בביכור המחשבה\hebrewmakaf האצילית\mycircle{°} העליונה\mycircle{°}, כאשר האמת של שכר ועונש איננה הגורם העיקרי בדחיפת החיים המוסריים\mycircle{°}, כי אם התשוקה\hebrewmakaf האידיאלית\mycircle{°}, לחיים שיש בהם תוכן מדעי ומוסרי במלא מובנו }\מקור{[עפ״י א״ק ב תקט, שם ג שכא\hebrewmakaf שכב]}\צהגדרה{. }

\paragraphs

\ערך{גדלות }\הגדרה{- כלליות }\מקור{[עפ״י מ״ש ס]}\צהגדרה{.}

\הגדרה{ע״ע גודל, הגודל לעומת הקוטן. }

\paragraphs

\משנה{גדלות }\הגדרה{- }\משנה{במעמד הבהירות שלה }\צהגדרה{- }\מעוין{◊}\צהגדרה{ אז אנו וכל עצמיותנו נעשים מובלעים בזוהר\mycircle{°} הכללי\mycircle{°} האלהי\mycircle{°} שממעל לכל הגבלת\mycircle{°} עולמים\mycircle{°} }\צמקור{[עפ״י ע״ר א יט].}

\paragraphs

\ערך{גדפנות }\הגדרה{- המרדה גרועה על הטוב שלא תוכל לתן שום דבר ולא תאיר את השכל בשום דבר בינה, כ״א תוסיף עקשות על שרירות לב של הרשעות למלא את הנפש תמהון ושכרון }\מקור{[ע״א ד ז יד]}\צהגדרה{. }

\הגדרה{ר׳ גדוף. ע׳ במדור מדרגות והערכות אישיותיות, אמגושי, גדופי.}

\paragraphs

\ערך{גדרים }\הגדרה{- ע״ע גבולים. }

\paragraphs

\ערך{גודל }\הגדרה{- }\משנה{הגודל לעומת הקוטן\mycircle{°}}\הגדרה{ - הכלליות }\מקור{[א״ק א עט]}\צהגדרה{.}

\הגדרה{ע״ע גדלות. ע״ע גדולה.}

\paragraphs

\ערך{גוון }\הגדרה{- }\משנה{צבע }\הגדרה{- הסברת איזה תכן של הבלטה ללבישת הצורה אשר לעשר הגדול של כלל\mycircle{°} ע״י פרט. כשהפרט מתגלה רק כדי לגוון על ידו איזה ציור\mycircle{°} תכונתי מוגבל\mycircle{°} להופעה גלויה של העשר הפנימי שצפון בכלל, המתגוון ע״י גילוייהם של מדות הגבול של הפרטים}\myfootnote{ \textbf{גוון} - ע״ע א״ק א מד, ע״ר ב רנה, קנז ד״ה חלק, א״ק ב תנו ושם ג פט, א׳ ט.\label{2}}\הגדרה{ }\מקור{[עפ״י ע״ר א קפא]}\צהגדרה{. }

\הגדרה{ציור פרטי }\מקור{[עפ״י ע״ר א קעג]}\צהגדרה{.}

\הגדרה{ע׳ במדור מונחי קבלה ונסתר,  חשמל. ע׳ במדור פסוקים ובטויי חז״ל, תחש. ר׳ צבע. }

\paragraphs

\ערך{גוי }\הגדרה{- הקיבוץ הכללי, <שהוא מושפע מהפרט, מהאדם היחיד> }\מקור{[פנק׳ ג שסח (קבצ׳ ג קכג)]}\צהגדרה{.}

\צהגדרה{קיבוץ של אנשים. הערך הצבורי מצדו הממשי של רבוי האנשים הפרטיים הנמצאים בו, בחבור גזעם ותכונתם כשהם לעצמם }\צמקור{[רצי״ה ע״ר ב תא\hebrewmakaf תב, עפ״י שם א רד, רה\hebrewmakaf ו].}

\ערך{גוי }\הגדרה{- (}\צהגדרה{״גויים״ לעומת ״עמים}\הגדרה{״) - אלה שיש להם הסתגלות לשמירת רוח עצמי פנימי, המתפתחים בהכרה עצמית, שומרי גזעם ותכונתם, בעלי נפש מרגשת הרגשה פנימית, בעלי ההכשרה של איזו רוחניות  שירית פנימית, שומרי הרוח המיוחד אשר להם ומאמצים באיזה אופן שהוא את סגולת מוצאם}\צהגדרה{ }\מקור{[עפ״י ע״ר א רה\hebrewmakaf ו]}\צהגדרה{.}

\משנה{גוי }\הגדרה{-}\צמשנה{ (לעומת עם\mycircle{°})}\צהגדרה{ - מהות התוכן הקיבוצי הפנימי. המצטרף בעובדת ההתקבצות }\צמקור{[עפ״י א״ל עו].}

\צהגדרה{צבור, כמות }\צמקור{[שי׳ פיקודי סדרה ב, תשל״ו 4].}

\הגדרה{ע״ע עם}\myfootnote{ ע״ע בע״ר ב תא\hebrewmakaf ב. ל״י ח״א ויחן ישראל נגד ההר. מלבי״ם, ישעיה א ד באור המלות.\label{3}}\הגדרה{, ע״ע אומה. ע׳ במדור פסוקים ובטויי חז״ל, משפחות עמים.}

\paragraphs

\ערך{גולם }\הגדרה{- ערך\mycircle{°} מוגבל\mycircle{°} ותוכן נקצב }\מקור{[עפ״י ר״מ קסז]}\צהגדרה{. }

\paragraphs

\ערך{גוף }\הגדרה{- הדבר הנדרש לחיים, לאורה\mycircle{°} ולהפרחה }\מקור{[א׳ מח]}\צהגדרה{.}

\משנה{גוף, תוכנו }\הגדרה{- ערך של קבלת חיים מאיזה רוחניות\mycircle{°} מציאותית }\מקור{[ע״ר א נב]}\צהגדרה{.}

\paragraphs

\ערך{גוף }\הגדרה{- }\משנה{(בתאורי הקב״ה) }\הגדרה{- ע׳ במדור תיאורים אלהיים.}

\paragraphs

\ערך{גוף}\הגדרה{ - }\משנה{הגוף הטבעי במצב תכונתו הגופנית }\הגדרה{- ע׳ במדור גוף האדם אבריו ותנועותיו.}

\paragraphs

\ערך{גופני }\הגדרה{- }\משנה{(לעומת נפשי\mycircle{°}) }\הגדרה{- כמותי חיצוני\mycircle{°} (לעומת איכותי פנימי\mycircle{°}) }\מקור{[רצי״ה א״ש יד כא]}\צהגדרה{. }

\paragraphs

\ערך{גורל }\הגדרה{- הזכיה הנעלמת, שאין ידועה סבתה המוסרית\mycircle{°} הנכונה. מזל\hebrewmakaf עליון\mycircle{°}, שאי\hebrewmakaf אפשר להעמידו באיזו בחינה הגיונית וצורה משפטית }\מקור{[ע״ר א קט]}\צהגדרה{. }

\הגדרה{שאין מתגלה מה טעם זה זכה, ומד׳\mycircle{°} כל משפטו בטעם גמור}\צהגדרה{ }\מקור{[פנק׳ ג צ]}\צהגדרה{.}

\paragraphs

\ערך{גורל עליון }\הגדרה{- כל המהות התמציתית של חיי האדם }\מקור{[עפ״י א״א 133]}\צהגדרה{. }

\הגדרה{נקודת האמונה\mycircle{°} }\מקור{[שם 134]}\צהגדרה{. }

\paragraphs

\ערך{גזירה}\myfootnote{ של״ה, תושב״כ, פרשת חקת, ד״ה במדרש רבות ״וכבר השיגוהו (לרמב״ם שאמר שגזרות הם בלא טעם) על זה. ואדרבה כל המצות בטעמיהן הם גזרות״. ושם שם ד״ה הענין ״לכך נקראים כל המצות גזירות, כי כמו שהנשמות חלק אלוה ממעל, לקוחה ממנו יתברך, כך המצות גזירות, ׳אוכלא דאפרת׳, חתוכות ממנו, מלשון ׳לגוזר ים סוף לגזרים׳, ׳והוא עבר בין הגזרים׳ וזהו כריתת הברית... וכן כל גזרה דרבנן, רצונו לומר, שהוא נגזר ונחתך מטעם איסור דאורייתא, כדי שלא יבא לידי איסור דאורייתא״. ושם, שם פרשת כי תצא, ד״ה ודע כי מצות שלוח הקן ״מלת גזירה היא מלפני, אינו כפי מה שמבינים העולם שענין גזירה הוא דבר שאין לו טעם, אדרבה ענין גזירה הוא דוקא שיש לו טעם והוא לקוח ונגזר ממה שלמעלה הימנו״. אמנם גם הרמב״ם עצמו כתב בסוף ה׳ תמורה ״אע״פ שכל חוקי התורה גזירות הם, כמו שביארנו בסוף מעילה, ראוי להתבונן בהן, וכל מה שאתה יכול ליתן לו טעם תן לו טעם. הרי אמרו חכמים הראשונים שהמלך שלמה הבין רוב הטעמים של כל חוקי התורה״.ע״ע במדור פסוקים ובטויי חז״ל, מדותיו של הקב״ה אינן רחמים אלא גזרות.\label{4}}\הגדרה{ - המשפט הגמור, החובה היותר גמורה }\מקור{[עפ״י קובץ ז קפג (ב״ר שכה)]}\צהגדרה{. }

\הגדרה{משפט ודין גמור, חק איתן. משפט צדק שיתגלה בכל שלמותו בבא עתו }\מקור{[עפ״י א״ה (מהדורת תשס״ב)  ב 93]}\צהגדרה{.}

\הגדרה{ע׳ במדור מצוות, הלכות, מנהגים וטעמיהן, בהגדרות המבוא, מצוות, כל מצוותיה של התורה, חקי התורה.}

\paragraphs

\ערך{גזירה }\הגדרה{- }\מעוין{◊}\הגדרה{ עיקר הנחתה היא לפי ערך שלשלת הנצחיות שהסיבות מתיחסות למסובביהן לפי מהלך המציאות }\מקור{[ע״א א א קלח]}\צהגדרה{. }

\ערך{גזרה עליונה}\צהגדרה{ - }\הגדרה{חשבון של פרעות תוצאת חיסרון הדיוק, הויתורים והדברים שבני אדם דשים בעקביהם, היוצא אל הפועל בהריסות כלליות או פרטיות. השפעה של אי הדיוק המוטבע ברוב החיים הסוללת דרכה לגבות את חובה באופן ציורי או מוחשי.}

\ערך{הגזירות העליונות }\הגדרה{- יוצאות בזעפן כסדרי הוויה וחיים המושפעים לצורך הכלל ההולך במרוצת חייו על פי סדרי משקלות שיש בהם צדדי הכרעה שאינם מדוקדקים בכיוון גמור לפי המגמה האלוהית העליונה. }\הגדרה{הגזרות העליונות}\הגדרה{ באות לגבות את החובות של חיסרון הדיוק שבחיי המוסר שהם אינם ניכרים כל אחד לעצמו ולשעתו אבל מסתרגים בעידן ריתחא בהיקבצם }\מקור{[עפ״י ע״א ד ו סב]}\צהגדרה{.}

\paragraphs

\ערך{גידוף }\הגדרה{- ע״ע גדפנות. }

\paragraphs

\ערך{גיהנם }\הגדרה{- הכח הכללי, ים הכליון וההעדר, המתיך ומחדש חידוש אופי גמור }\מקור{[עפ״י ע״א ד יב נא]}\צהגדרה{. }

\הגדרה{הכח המעדיר והמהרס שמגמתו לשנות את הצורות הראשונות על ידי כחו המכלה, עד שיצאו בפנים חדשות }\מקור{[עפ״י שם מז]}\צהגדרה{. }

\הגדרה{כח המכלה שמגמתו השכלול הכללי שבא באחרית, על ידי ההירוס והכליון. <המגמה של מציאותו היא מגמת השכלול הבאה במקום שיש הכרח של הירוס ושברון, כדי להגיע אל הכונה האחרונה של השלמת ההויה, על ידי הסרתם של הסיגים המוחלטים, וחידוש יצירה חדשה, בצורה מחודשת, מתוך היסוד המקולקל שקדם> }\מקור{[עפ״י שם מט]}\צהגדרה{. }

\משנה{שאיפתו של הגיהנם }\הגדרה{- מגמת התיקון ע״י הרס וכליון הקדום }\מקור{[שם מז]}\צהגדרה{. }

\משנה{גיהנם }\הגדרה{- המירוק\mycircle{°} של הנשמות מזוהמתן\mycircle{°} (ב)צער העמוק הנחשולי }\מקור{[קובץ ח ק]}\צהגדרה{. }

\הגדרה{מירוק בדרך הפועלת לשנות את טבע הנפש עצמה }\מקור{[עפ״י ע״א א א קסט]}\צהגדרה{. }

\הגדרה{שינוי עצמיות הנפשות להכשירם לצד הטוב\mycircle{°} המתקן קלקולים הנדבקים בעצם טבעם }\מקור{[עפ״י ע״א ג ב רעד]}\צהגדרה{. }

\משנה{תפקידו של הגיהנם }\הגדרה{- לשנות את הצורה העצמית של הנשמה, נשמת החיים, הרוח והנפש, על ידי הכליון של החלקים העצמיים הגרועים שנתעצמו בקרבם, עד שאחר ההיתוך הצורי הזה, יצאו כחות החיים הפנימיים בצורה אחרת, לגמרי חדשה }\מקור{[ע״א ד יב נ]}\צהגדרה{. }

\משנה{גיהנם }\הגדרה{- הבוץ בו מרגשת הנשמה את צרתה הגדולה כאשר טבעה בו, בבא זמן האורה והנשמה מתנשאת למרום טבעה. הרשמים הטבעיים שישנם בנפש האדם להיותו יורד על ידם לשפל מדרגת הבהמות, כשהם פועלים את פעולתם ע״י כשלונו המוסרי\mycircle{°} של האדם שלא כהוגן, נמשך האדם על ידם במצב רוחני הפוך מטבע הנשמה הטהורה\mycircle{°} והאלהית שבקרבו; ומיעוט הכח, שנתדלדל מקרב הנשמה חילה הפנימי ע״י דרכיה הפרועים, הוא מועיל לא להחיותה כולה בשלמותה, כ״א להטעימה את מעמדה האומלל והאיום ולהבעיר את תשוקתה להחלץ מחשכת הדמיונות הגרועים שכסו את שמיה הבהירים }\מקור{[עפ״י ע״א ג ב רלד]}\צהגדרה{. }

\הגדרה{כור מצרף להרוחות\mycircle{°} והנשמות\mycircle{°} שבגלל חטאיהם\mycircle{°} יצאו מעולמם החומרי\mycircle{°} במצב של קלקול. הכח שעל ידו יבואו לידי יצירה חדשה ויותר מושלמת, יסורו הסיגים מהכסף הטהור של כוחות החיים ושל כל מה שיחובר להם. כח המכלה והמשבר\mycircle{°}, הים הגדול השוטף כליון על כל הצורות הקדומות, המעבר המשלים את צביונה של היצירה בכללותה }\מקור{[עפ״י ע״א ד יב מג]}\צהגדרה{. }

\הגדרה{מקום\mycircle{°} ששם מתגלה לעצם הנפש המרגשת הרגש האכזרי שפועלת עליה ההשקפה האמתית של אבדן הטוב היותר נחמד, האוצר היותר יקר שבכל אוצר החיים, שם מתגלה כח הצער\mycircle{°} בכל כחו, הצער המגיע מהכליון המוסרי\mycircle{°} }\מקור{[עפ״י ע״א ב ט קיב]}\צהגדרה{. }

\הגדרה{חוסר התורה\mycircle{°} }\מקור{[א״ת ז ו]}\צהגדרה{. }

\ערך{גיהנם }\הגדרה{- }\משנה{״שרה של גיהנם״}\myfootnote{ שבת קד. זוהר ח״ב יח.  \label{5}}\הגדרה{ - הכח המשבר את הצורה\mycircle{°} הרוחנית ביסודה, בכח ההרס האכזרי שלו }\מקור{[ע״א ד יב מה]}\צהגדרה{. }

\ערך{גיהנם }\הגדרה{- }\משנה{״אשו של גיהנם״ }\הגדרה{- כח הצורב של ההעדר הנורא }\מקור{[עפ״י שם מד, נ]}\צהגדרה{. }

\הגדרה{הצירוף והזיקוק, המהרס בזעפו }\מקור{[שם מח]}\צהגדרה{. }

\ערך{יסורי גיהנם}\myfootnote{ \textbf{יסורי}\textbf{ }\textbf{גיהנם}\textbf{, הצער הפנימי על חסרון ההשלמה של הנשמה }\textbf{וכו}\textbf{׳. שטף הרצון החיצוני, שנגד הרצון הפנימי }- ע״ע אור השם, לר״ח קרשקש, מאמר ג ח״א כלל ג פרק ג ד״ה ואמנם השני והשלישי (מהדורת הר״ש פישר עמ׳ שלה).\textbf{הצער הפנימי [...] }\textbf{בקנין\hebrewmakaf תורה}\textbf{ [...] מניעת }\textbf{אור\hebrewmakaf התורה}\textbf{ [...] מעוט התורה }\textbf{-}\textbf{ }ע׳ נפה״ח שער ד פרק יז.\label{6}}\הגדרה{ - צער הנשמה שאינה מוציאה את מעלות רוחה מן הכח אל הפועל, המתענה בעינויים נוראים }\מקור{[עפ״י קבצ׳ ב קנז]}\צהגדרה{.}

\הגדרה{הצער הפנימי\mycircle{°} על חסרון ההשלמה של הנשמה\mycircle{°} במעשים, בידיעות, ובדעות, ביחוד בקנין\hebrewmakaf תורה\mycircle{°} }\מקור{[א״ת ז ה]}\צהגדרה{. }

\ערך{מצרי גיהנם }\הגדרה{- שטף הרצון החיצוני\mycircle{°}, שנגד הרצון הפנימי, הוא }\צהגדרה{מצרי גיהנם}\הגדרה{ המתגברים לפי אותו הערך של מניעת אור\hebrewmakaf התורה\mycircle{°} }\מקור{[א״ת ז ו]}\צהגדרה{. }

\הגדרה{מצרים האוחזים בכל מי שריפה ידיו מן התורה; הצער הגדול על מעוט התורה, שבא מצד בטול תורה וצמצום הדעת, המאפיל על אורה הרוחני של תורה }\מקור{[עפ״י א״ת ז ח]}\צהגדרה{.}

\ערך{צער הגיהנם בעולם הזה}\myfootnote{ ע״ע ע״א ב ט לח.\label{7}}\הגדרה{ שחשים את כיעור הנפילה בעמקי הרע\mycircle{°}, ואי אפשר להושיע\mycircle{°} את עצמו }\מקור{[קבצ׳ ב קז (פנק׳ ד רנ)]}\צהגדרה{.}

\צהגדרה{המחשבה מוכרחת להתעלות כפי אותה המדה שגנוז בכחה, ואם אין מוציאים אותה מן הכח אל הפועל היא מתענה בעינויים נוראים, שרק המתמכרים אל החושים יכולים הם לשכחם. אבל מי שהוא איש רוחני\mycircle{°}, מרגיש את}\הגדרה{ צער נשמתו איך היא נתונה ממש בתוך אשה\hebrewmakaf של\hebrewmakaf גיהנם\mycircle{°}, כל זמן שאינה מוציאה את מעלות רוחה מן הכח אל הפועל}\צהגדרה{ }\מקור{[קבצ׳ ב קנז]}\צהגדרה{.}

\הגדרה{ע״ע שְאוֹל. ע׳ במדור מלאכים ושדים, ״דּוּמָה״. ע״ע ערבה. ע׳ במדור פסוקים ובטויי חז״ל, קילוסו של הקב״ה עולה מגיהנם כשם שעולה מגן עדן. ושם, עדן, דישון עדן. }

\paragraphs

\ערך{גיור }\הגדרה{- }\משנה{ליהדות התורה של האומה }\הגדרה{- ע״ע גרות, התגירות. }

\paragraphs

\ערך{גיור }\הגדרה{- }\משנה{הטרם\hebrewmakaf היסטורי במובנה של יהדות\hebrewmakaf התורה }\הגדרה{- ע״ע גרות.}

\paragraphs

\ערך{גלוי }\הגדרה{- }\משנה{״פתוח״\mycircle{°}}\הגדרה{.}\משנה{ הדברים הגלויים }\הגדרה{- כל המציאות המוחשית והמושגת בכלל, וכל המעשים הנעשים במפעלות ידי אדם. כל התכניות הגלויות שבהויה וכל המעשים המושגים }\מקור{[ע״א ד יב ב (מא״ה ב קל)]}\צהגדרה{.}

\הגדרה{ע׳ במדור מונחי קבלה ונסתר, קוב״ה דרגא על דרגא סתים וגליא וכו׳. ע׳ במדור פסוקים ובטויי חז״ל, מאמר פתוח. ע״ע ״סתום״. }

\ערך{גלוי }\הגדרה{- מושג ומורגש }\מקור{[מא״ה ג (מהדורת תשס״ד) קכד]}\צהגדרה{.}

\הגדרה{הנתפס באיזה רעיון שכלי אנושי }\מקור{[ע״ר א קיא]}\צהגדרה{.}

\paragraphs

\ערך{גלות }\הגדרה{- חרבן פנימי\mycircle{°} ופזור חצוני }\מקור{[א׳ קח]}\צהגדרה{.}

\משנה{חותם הגלות }\הגדרה{- דיכוי רוח עם ד׳ ודכדוך סדרי חייהם עד שלא יוכל לקום בהם רוח להרגיש יפה את צרכי החיים הטבעיים כולם באופן הראוי להיות מורגש לעם חי מלא אונים }\מקור{[עפ״י ע״א ג ב סה]}\צהגדרה{.}

\צהגדרה{הריסת חיי האומה\mycircle{°} בניתוקה ממקומה, ״גלינו מארצנו ונתרחקנו מעל אדמתנו״ }\צמקור{[ל״י א קצב]. }

\ערך{גלות }\הגדרה{-}\משנה{ (מטרת הגלות) }\הגדרה{- <לא בשביל איזה מירוק\mycircle{°} של חטא\mycircle{°} מיוחד, כ״א> כדי להסתגל להיות גם בגלות עומד חי וקיים בתור אומה\mycircle{°} מחוטבת\mycircle{°}}\צהגדרה{ }\מקור{[ע״א ד ט קמז]}\צהגדרה{.}

\הגדרה{כור ברזל והכשרה לזיכוכה של האומה, כדי שתהיה מוכנת באחרית הימים לשוב לארצה ולחוסן יקרה וכבודה }\מקור{[ע״א ד יד ז]}\צהגדרה{.}

\הגדרה{כור הברזל, (ש)זקק וצרף את האומה, הוציא מן הכח אל הפועל את דעת עצמותה, את נטיותיה הגנוזות, הטבעיות לה, לצדק\mycircle{°} לאורה\mycircle{°}, לחיי יושר\mycircle{°} וטהרה\mycircle{°}, באופן שבכל פינות שתהיה פונה בכל צורה שהחיים יראו לה על ידה, תכיר את האור ואת הטוב\mycircle{°} }\מקור{[ע״ה קכח]}\צהגדרה{.}

\צהגדרה{<ההתקשרות האלהית, העליונה והטהורה, אינה מנגדת כלל את העולם ואת החיים, לכל עמקי תוכניהם, אלא גם מכשרת אותם ומרחיבתם. }

\צהגדרה{אלמלא חטאו ישראל לא היו צריכים כלל לסגל להם איזה סגולות מן החוץ כדי להשלים את עצמם בכל ערכי החיים. אבל החטא גרם, והמחשבה העליונה הועמה, ומה שנשאר הוא אור של תולדה, שאין לו אותו הבוהק העליון וההכללה הגמורה של העליוניות המוחלטת, של המחשבה האלהית, וממילא אין לו אותה סגולת ההרחבה, עד שבא הדבר, שההתיחדות עם התכונה של ההתקשרות האלהית, החסירה את הכשרון לשארי כשרונות>, }\צהגדרה{והוצרך הפיזור של הגלות}\הגדרה{ כדי להשלים את החסרונות הללו, לספג את כל היתרונות של כל הגויים אל תוכם כדי להשלים את צביונם, }\צהגדרה{<והשלמת הצביון והזיכוך הארוך של הנפש הלאומית בכור הברזל של הגלות גרם לאפשר את החזרת האורה העליונה> }\מקור{[א״ק ג שסז]}\צהגדרה{.}

\משנה{תכלית הגלות }\הגדרה{- כדי שיוכלו גם האומות להכיר כבוד\hebrewmakaf ד׳\mycircle{°} }\מקור{[ע״א א ה פג]}\צהגדרה{. }

\paragraphs

\ערך{גמול }\הגדרה{- הגורל\mycircle{°} המוסרי\mycircle{°} }\מקור{[עפ״י ר״מ קכט]}\צהגדרה{.}

\הגדרה{קשור הגורל עם מהלך החיים הרצוניים }\מקור{[עפ״י שם צו]}\צהגדרה{.}

\הגדרה{משפט הצדק\mycircle{°} }\מקור{[שם קמא]}\צהגדרה{.}

\משנה{הגמול הגלוי }\הגדרה{- התגלות\mycircle{°} המשפט\mycircle{°} המעשי שבעולם ומלואו }\מקור{[שם קמג]}\צהגדרה{.}

\paragraphs

\ערך{גמול }\הגדרה{- }\משנה{מדת הגמול העליונה האידיאלית }\הגדרה{- וחנותי את אשר אחון ורחמתי את אשר ארחם. הגמול האידיאלי העליון המופשט, הגנוז ביסוד החסד\hebrewmakaf העליון\mycircle{°} אשר בו עולם יבנה }\מקור{[ר״מ קמב\hebrewmakaf ג]}\צהגדרה{.}

\paragraphs

\ערך{גמילות חסדים }\הגדרה{- }\משנה{יסוד גמילות חסדים }\הגדרה{- נטית החסד והאהבה של הבריות, היוצאה מכלל הרחמים על אומללים ונדכאים, אלא חפץ ההטבה ושפור החיים, להרבות טוב לכל }\מקור{[ע״ר א סג]}\צהגדרה{.}

\הגדרה{ע׳ במדור מדרגות והערכות אישיותיות, גומל החסדים.  }

\paragraphs

\ערך{גמילות חסדים טובים }\הגדרה{- ע׳ במדור פסוקים ובטויי חז״ל, חסדים טובים.}

\paragraphs

\ערך{גן החכמה}\הגדרה{ - ההכרה העליונה והבהירה, ההרגשה היפה והעדינה }\מקור{[קובץ ג קעח]}\צהגדרה{.}

\paragraphs

\ערך{״גן עדן״}\myfootnote{ \textbf{גן עדן} - אדיר במרום ח״א עמ׳ מג, ״גן עדן הוא פנימיות העולם״. אור החיים עה״ת, בראשית ב טו ״גן עדן אשר היום אדמת הנפשות לבד״.\label{8}}\הגדרה{ - ההצמחה של כללות העדונים, מקום\mycircle{°} זיו\mycircle{°} חיי הנשמות\mycircle{°} והתענגותם\mycircle{°} האצילית\mycircle{°}, בטיסתם העליונה העולה למעלה למעלות, בכל חזות עולמי פאר\mycircle{°} נגוהות, בדליגות מעלות ע״ג מעלות, ובשובע שמחות\mycircle{°} של תענוגי רוית קדשים\mycircle{°}, ההולך וצומח בצמחי מעדניו, הוא התגלות העדן\mycircle{°} של גן\hebrewmakaf ד׳ }\מקור{[ע״ר א יז]}\צהגדרה{.}

\הגדרה{ע׳ במדור פסוקים ובטויי חז״ל, עדן, דישון עדן. ושם, אטייל עמכם בגן\hebrewmakaf עדן}\תהגדרה{. }\הגדרה{ע״ע, עדן העתיד. }

\ערך{גן עדן }\הגדרה{- }\מעוין{◊}\הגדרה{ מורה שישוב המין האנושי לגובה מעלתו בפרטיו, עד שלא יהיה צריך שום לימוד והדרכה, ולא עזרת קיבוץ, כ״א לחיות ב}\צהגדרה{גן עדן, }\הגדרה{לעבדה ולשמרה, בתור טיול\mycircle{°} ועונג\mycircle{°} מורחב הממלא את החומר והרוח עדנים }\מקור{[קבצ׳ ב צז]}\צהגדרה{.}

\הגדרה{תיקון\mycircle{°} המלכות\mycircle{°}, שלעתיד ושקודם החטא\mycircle{°} (ה)מכוון לקבלת אור לנשמות ישראל בעצם. תיקון הנשמות ועילויים, נקרא בשם }\משנה{גן }\הגדרה{- המצמיח הנטיעות וארזים אשר נטע ד׳ }\מקור{[עפ״י פנק׳ ד תמ]}\צהגדרה{.}

\משנה{יסוד גן עדן}\הגדרה{ - השבת כללות האדם לטהרתו\mycircle{°} האלהית\mycircle{°}, שהיא למעלה מכל חילוק לאומים, שהוא עומד למעלה מעולם הזה }\צהגדרה{[קבצ׳ ב קנז]}\הגדרה{. }

\הגדרה{ע׳ במדור מונחי קבלה ונסתר, נקודת ציון. ע״ע רוח גן\hebrewmakaf עדן אלהים, המנשב בנשמה. ע׳ במדור פסוקים ובטויי חז״ל, אטייל עמכם בגן עדן. ע׳ במדור אדם הראשון, ״לַעֲבֹד את האדמה״, (מששולח האדם מגן\hebrewmakaf עדן).}

\ערך{גן עדן }\הגדרה{- }\משנה{בבחינת ״גן הדס״ }\הגדרה{- צורתו הרוחנית, הריחנית, של גן עדן, המוסיף ריח טוב ועדין שהנשמה\mycircle{°} נהנית ממנו }\מקור{[עפ״י ע״א ד יב מח, מט]}\צהגדרה{.}

\הגדרה{ע״ע ריח בושם של גן עדן. }

\ערך{גן עדן }\הגדרה{- }\משנה{סעודת אכילת הפירות של גן עדן }\הגדרה{- ההנאה מזיו\hebrewmakaf השכינה\mycircle{°}, היסוד העקרי של חיי\hebrewmakaf עולם\mycircle{°} }\מקור{[עפ״י שם]}\צהגדרה{.}

\ערך{גן עדן }\הגדרה{- }\משנה{(לאדם הראשון) }\הגדרה{- מקום\mycircle{°} הזרחת השכל על אדם\hebrewmakaf הראשון\mycircle{°} בפועל }\מקור{[עפ״י מא״ה ב רסט]}\צהגדרה{.}

\הגדרה{ע׳ במדור אדם הראשון, חטא אדם הראשון. ושם, אדם הראשון בגן עדן. ע״ע עונג, עונג הגן. }

\ערך{גן עדן }\הגדרה{- ההתיחדות המסותרת במעמקי הנשמה\mycircle{°} }\מקור{[עפ״י קבצ׳ ג קנה]}\צהגדרה{. }

\paragraphs

\ערך{גנזי שחקים }\הגדרה{- הספירות\mycircle{°} העליונות, ששם חביון\hebrewmakaf העז\mycircle{°}, ושם כל התכונה הכללית של מערכי ההויה, מסבותיה ומגמותיה\mycircle{°}, צפויה היא ונערכת }\מקור{[ע״ר א ריד]}\צהגדרה{. }

\paragraphs

\ערך{גס}\הגדרה{ - חומרי }\מקור{[פנק׳ ד תלא]}\צהגדרה{.}

\ערך{גסות }\הגדרה{- נטיה חמרית\mycircle{°} }\מקור{[עפ״י א״ש טז יב]}\צהגדרה{. }

\הגדרה{ע״ע דק.}

\paragraphs

\ערך{געגוע }\הגדרה{- }\משנה{הגעגוע האלהי העליון }\הגדרה{- הצמאון\mycircle{°} האדיר להכלל באוצר האור בחיי החיים העליונים, מקור כל החיים ושרש כל ההויות }\מקור{[ע״ר א סז]}\צהגדרה{. }

\משנה{געגועים אלהיים }\הגדרה{- חפץ טמיר מלא חיים עזיזים מקוריים ממקור חיי עולמים }\מקור{[א׳ נט]}\צהגדרה{. }

\משנה{הגעגועים ההומים לזוהר\mycircle{°} הצחצחות\mycircle{°} הנשמתיות\mycircle{°}}\הגדרה{ - הצמאון\hebrewmakaf האלהי, החפיצה הפנימית\mycircle{°} להתעלות\mycircle{°}, להשאב\mycircle{°} בחיי חיי מקור חי כל חיי העולמים\mycircle{°} }\מקור{[א״ק ג ס]}\צהגדרה{. }

\paragraphs

\ערך{״גר אמת״ }\הגדרה{- ע׳ במדור מדרגות והערכות אישיותיות. }

\paragraphs

\ערך{״גר תושב״ }\הגדרה{- ע׳ במדור מדרגות והערכות אישיותיות. }

\paragraphs

\ערך{גרות}\myfootnote{ השוה ל״גר תושב״, במדור מדרגות והערכות אישיותיות. לבירור החלוק בין שני מושגי הגרות, ע׳ ל״י (מהדורת בית אל) ח״ב, מאמר ששים, לבירור ענין הגיור מבחינת היהדות התורתית. ע״ע לבנת הספיר, ויחי, עה״פ אוסרי לגפן עירו. ובברית הלוי לר״ש אלקבץ, סוף פרק יד, כג:.\label{9}}\הגדרה{ - }\משנה{התגירות }\הגדרה{- התבטלות רוחנית\mycircle{°} גמורה לכללות\mycircle{°} האומה\mycircle{°} הישראלית\mycircle{°} }\מקור{[עפ״י ע״ר א שצח]}\צהגדרה{. }

\משנה{גיור }\צהגדרה{- }\צמשנה{ליהדות התורה של האומה }\צהגדרה{- קשור בסדרי חייה ובמערכת חוקיה ומשפטיה של התורה הזאת ושל האומה הזאת. התחברות והצטרפות אל אומה זו ואל ייחוד שכינתה עם ההתחייבות במשמרת התורה והמצווה }\צמקור{[ל״י ג קו].}

\מעוין{◊ }\משנה{גרות }\צמשנה{עיקרה}\צהגדרה{ החלטת קבלתו בדעתו להיות גר בישראל ונלוה על ד׳, המתחילה לצאת לפעל עם קבלת המצות, וידיעת אזהרותיהן ושכרן ועונשן, }\צהגדרהמודגשת{ונגמרת ונעשית}\צהגדרה{ על ידי המילה והטבילה, ואינו גר עד שימול ויטבול }\צמקור{[דעת כהן תמה].}

\הגדרה{ע׳ במדור מדרגות והערכות אישיותיות, גרי צדק.}

\paragraphs

\ערך{גרות }\הגדרה{- הוראה של קבלת\hebrewmakaf מלכות\hebrewmakaf שמים\mycircle{°} באמצעות ישראל וע״י השפעתם }\מקור{[ע״א ב ט קיד]}\צהגדרה{. }

\משנה{גיור }\צהגדרה{- }\צמשנה{הטרם\hebrewmakaf היסטורי במובנה של יהדות\hebrewmakaf התורה }\צהגדרה{- ביטול האליליות\mycircle{°} ושחיתותיה. גיור במובנה של אגדה והשפעה רוחנית כללית }\צמקור{[ל״י ג קו].}

\צמשנה{״אברהם מגייר את האנשים ושרה מגיירת הנשים״}\myfootnote{ בר״ר פר׳ לט יד.\label{10}}\צהגדרה{ - מדריכים אותם רוחנית}\צמקור{ [שי׳ ת״ת 162].}

\הגדרה{ע׳ במדור מדרגות והערכות אישיותיות, גר תושב.}

\paragraphs

\ערך{״גרים גרורים״}\הגדרה{ - ע׳ במדור מדרגות והערכות אישיותיות. }

\paragraphs

\ערך{״גרי צדק״ }\הגדרה{- ע׳ במדור מדרגות והערכות אישיותיות. }

\paragraphs

\ערך{גרים - }\משנה{״קשים גרים לישראל כספחת״}\הגדרה{ - ע׳ במדור פסוקים ובטויי חז״ל. }

\paragraphs

\ערך{גשם }\הגדרה{- }\משנה{(לעומת טל\mycircle{°}) }\הגדרה{- ההשפעה\mycircle{°} שאנו גורמים בעבודתינו באתערותא\hebrewmakaf דלתתא\mycircle{°}, מכוון נגד קרבת\mycircle{°} השי״ת הבאה ע״י המחקר בגדולתו\mycircle{°} ית׳ }\מקור{[מא״ה א קנט, מ״ש נט]}\צהגדרה{.}

\paragraphs

\ערך{גשמי }\הגדרה{- ע״ע התגשמות. }

\paragraphs

\ערך{גשמים}\myfootnote{ אבות ה משנה ה.\label{11}}\ערך{, הבאים מיסוד המים }\הגדרה{- מורים על עצם ההויה הגופנית, שֶׂכֶל הגוף }\מקור{[ע״ר ב קעו]}\צהגדרה{. }

\הגדרה{ע׳ במדור משכן ומקדש, אש המערכה. ע״ע רוח, (הרוח שלא נצחה את עשן המערכה).}\mylettertitle{ד}

\ערך{דאגה}\הגדרה{ - עצב ופחד }\מקור{[פנק׳ ג רפט]}\צהגדרה{.}

\paragraphs

\ערך{דאגה}\הגדרה{ - חיפוש עצה\mycircle{°} איך למלאת החסרון }\מקור{[פנק׳ ג רפט]}\צהגדרה{.}

\paragraphs

\ערך{דבור }\הגדרה{- המחשבה\mycircle{°} הגופנית }\מקור{[אג׳ ב לט]}\צהגדרה{. }

\ערך{דבור }\הגדרה{- ע״ע לשון. ע״ע שפה. ע״ע פרי השפתים. ע״ע הבל.}

\paragraphs

\ערך{דבור }\הגדרה{- }\משנה{תוכן של דבור המתלוה לאמירה\mycircle{°} }\הגדרה{- התגלות הדבור (ההופעה המילולית), כשבא להתגלם בצורה המשפעת על השומע, שמקבל הנהגה ע״י הצורה הדבורית, הבאה אל חוגו, והוא משותף עם דברות והנהגה }\מקור{[עפ״י ע״ר ב נד]}\צהגדרה{. }

\מעוין{◊}\הגדרה{ לשון דבור הוא לשון הנהגה ולשון קשה }\מקור{[ע״ר א קלו]}\צהגדרה{. }

\הגדרה{ע״ע אמירה. ע״ע אמר. ע״ע קול. ע׳ בנספחות, מדור מחקרים, ״אֹמֶר״ לעומת ״דבור״. }

\paragraphs

\ערך{דבור }\הגדרה{- }\משנה{(לעומת ״אמרי פי״\mycircle{°}) }\הגדרה{- ההגיון הפנימי }\מקור{[א״ה ב 902]}\צהגדרה{. }

\משנה{הדבור הפנימי }\הגדרה{- הגיון הלב }\מקור{[עפ״י מ״א א א]}\צהגדרה{. }

\paragraphs

\ערך{דבור החיצוני}\הגדרה{ - }\משנה{(לעומת דיבור\hebrewmakaf הפנימי\mycircle{°})}\myfootnote{ \textbf{דבור חיצוני ודבור פנימי} - מקבילים שני סוגי דבורים אלו לדיבור החול והקודש בע״ר א קצב: ״הדבור בכללו יש לו כח כפול: דבור קדש, ודבור חול. \textbf{דבור הקדש} מוצאו הוא ממקור הדעה העליונה, יוצרת כל היצורים, ״בדעתו תהומות נבקעו״, ו\textbf{דבור של חול }מקורו הוא למטה, מהציורים הבאים ומתרשמים בנפשו של אדם מכל מה שיש ביצירה״.\label{1}}\הגדרה{ - הדיבור אודות צרכיו הפרטיים והכלליים של האדם, בתור איש חברותי ובעל\hebrewmakaf חי נזון, ובעל צרכים רבים יותר מבעלי החיים האחרים. החלק החיצוני של הדיבור, שהם עלי הדבור ולא פריו}\צהגדרה{ [עפ״י ע״ר ב סו]}\הגדרה{.}

\הגדרה{ע׳ במדור גוף האדם אבריו ותנועותיו, שפתיים. ושם, שפה.}

\paragraphs

\ערך{דבור הפנימי }\הגדרה{- }\משנה{(לעומת דיבור\hebrewmakaf החיצוני\mycircle{°})}\footref{1}\הגדרה{ - הדיבור השייך להערכים הרוחניים\mycircle{°}, שהם עומדים למעלה מצרכיו הפרטיים והכלליים של האדם. פריו של הדיבור}\צהגדרה{ }\מקור{[ע״ר ב סו]}\צהגדרה{.}

\הגדרה{ע׳ במדור גוף האדם אבריו ותנועותיו, לשון.}

\paragraphs

\ערך{דבור }\הגדרה{- }\משנה{הדבור האלהי }\הגדרה{- יסוד אור\hebrewmakaf השכל\mycircle{°} במקורו }\מקור{[ע״ר א קיט]}\צהגדרה{.}

\paragraphs

\ערך{דבור }\הגדרה{- }\משנה{דבור ההויה הכללית }\הגדרה{- ההשפעה הנהלית, הבאה ע״י השפעת כח המבטא על השומעים והמקשיבים לקול ההשפעה }\מקור{[ע״ר ב נד]}\צהגדרה{. }

\הגדרה{ע״ע קול, קול ההויה כולה. ע״ע אמר, אמר ההויה כולה. }

\paragraphs

\ערך{דבור רע}\הגדרה{ - }\משנה{הדיבור הרע}\הגדרה{ - ארס שלוח והרחבת ענפי כח הזוהמא\mycircle{°} הבשרית והצד העכור שבחיי\hebrewmakaf הרוח\mycircle{°} }\מקור{[עפ״י קובץ ג רנו]}\צהגדרה{.}

\paragraphs

\ערך{דבקות }\הגדרה{- יסוד הדבקות\hebrewmakaf האלהית\mycircle{°}. תשוקת הטוהר\mycircle{°} המתעלה\mycircle{°} ע״י טהרת מחשבה\mycircle{°}, עומק הגיון\mycircle{°}, חופש\mycircle{°} רוחני\mycircle{°}, והסתכלות בהירה. ההתהלכות\hebrewmakaf את\hebrewmakaf האלהים\mycircle{°} של הדורות הראשונים }\מקור{[א״ק ג קסט]}\צהגדרה{. }

\הגדרה{מעלת הערת\hebrewmakaf השכל\mycircle{°}, הארת השכל בעבודת השי״ת }\מקור{[עפ״י פנק׳ ג ערה]}\צהגדרה{.}

\מעוין{◊ }\הגדרה{תולדת הקשר של הקדושה\mycircle{°}, ברצון ובשכל, בחיים ובהכרה שכלית }\מקור{[ע״ר ב עז]}\צהגדרה{. }

\paragraphs

\ערך{דבקות אלהית }\הגדרה{- אמונת\mycircle{°} אלהים\hebrewmakaf חיים\mycircle{°}, אהבתו\mycircle{°} ויראתו\mycircle{°} }\מקור{[ע״ט מח]}\צהגדרה{. }

\הגדרה{זריחת\mycircle{°} האור הבהיר של כבוד\mycircle{°} השי״ת, אמונתו ואהבתו, באופן נעלה מכל מופת וידיעה }\מקור{[עפ״י קבצ׳ א קלו]}\צהגדרה{.}

\הגדרה{הפניה אל ד׳, אמירת אלי אתה, בלב ושפה, זו היא האמרה המקנה את עירוב המהות}\צהגדרה{ }\מקור{[א״ק ד תנא]}\צהגדרה{.}

\הגדרה{התכונה המבוקשת של דעת\hebrewmakaf אלהים\mycircle{°} הטהורה\mycircle{°}, המלאה חוסן\mycircle{°} וחיל\mycircle{°} }\מקור{[קובץ ח רמט]}\צהגדרה{.}

\הגדרה{התגבורת של הטבעיות של הקודש\mycircle{°}, המחזקת את הנשמה באימוץ אדיר וקדוש מאד }\מקור{[פנק׳ ב קצה]}\צהגדרה{. }

\הגדרה{דבקות הנשמה בצרור\hebrewmakaf החיים\mycircle{°}, באור\hebrewmakaf ד׳\mycircle{°} }\מקור{[א״ק ג רמ]}\צהגדרה{. }

\הגדרה{קשור פנימי\mycircle{°} נשמתי במה שהוא הכל\mycircle{°} ומקור הכל ויותר כל מן הכל, והחפץ המלא טהרה\mycircle{°} הולך ומתגבר מבלי להשאיר שום חלק מבלי שאיבת תמצית לשדו העליון\mycircle{°} שבעליונים }\מקור{[אג׳ ג ד]}\צהגדרה{. }

\הגדרה{דבקות\mycircle{°} בחיי\hebrewmakaf החיים\mycircle{°}, באור ההויה העליונה שכל החמדות, כל הנצחיות, כל הענוגים, כל השלומים וכל הגבורות, הפארים, התהילות והזהרות ממנה הם באים }\מקור{[עפ״י קובץ א תרסג]}\צהגדרה{. }

\הגדרה{התשוקה האידיאלית\mycircle{°}, אשר היא יסוד הכל, חשק\mycircle{°} הטוב\hebrewmakaf העליון\mycircle{°}, העולה בטובו מכל שנקלט אצלנו במושג של תענוג\mycircle{°} ועידון\mycircle{°} }\מקור{[עפ״י א״ק ג קסט]}\צהגדרה{. }

\משנה{דבקות אלהית אמיתית }\הגדרה{- אור התענוג העליון, החיים האמתיים, מקור ההצלחה\mycircle{°} ומגמת\mycircle{°} החיים וההויה כולה }\מקור{[עפ״י א׳ צט]}\צהגדרה{. }

\משנה{דבקות אלהית פנימית}\הגדרה{ - יסוד כל הידיעות }\מקור{[א״ק א יב]}\צהגדרה{.}

\משנה{דבק בחיי\hebrewmakaf החיים\mycircle{°}}\הגדרה{ - דעת צור\hebrewmakaf העולמים\mycircle{°}, המתעלה על כל גבול ומחזה עין בשר }\מקור{[ע״א א מהדו״ב א ב]}\צהגדרה{. }

\משנה{תוכנה של הדבקות האלהית }\הגדרה{- להיות משוקה מטל חיי עד\mycircle{°}, משורש החיים האמיתיים שאין לו הגבלה }\מקור{[עפ״י ע״ר א סג]}\צהגדרה{. }

\הגדרה{שאיפת היש האישי, הגודל\mycircle{°} והשיגוב\mycircle{°} }\מקור{[קובץ ה כו]}\צהגדרה{. }

\משנה{חשקת הדבקות האלהית}\הגדרה{ - כלות הנפש ועריגה בלתי פוסקת, ההולכת ומתגברת בכל עת תוסיף ההכרה להתעמק בתהומות הנפשיים }\מקור{[קובץ ז קיח]}\צהגדרה{.}

\מעוין{◊ }\משנה{התוכן של הדבקות האלהית הטבעית }\הגדרה{- אהבת ישראל\mycircle{°} }\מקור{[א״ק ג שמ]}\צהגדרה{.}

\מעוין{◊}\משנה{ חיים של דבקות אלהית }\הגדרה{- דבקות שכלית}\צהגדרה{ }\מקור{[קבצ׳ א קסט]}\צהגדרה{.}

\ערך{דבקות אלהית }\הגדרה{- }\משנה{דבקות אלהית בהירה, מגמתה }\הגדרה{- שרוח\mycircle{°} האדם בעילויו\mycircle{°}, ע״י הכרה צלולה והתנשאות רצון מטוהר\mycircle{°} וגמור, יתנשא לבוא עד לידי השלטת רצונו על ההויה מפני עוזו וחשיבותו }\מקור{[עפ״י ע״ט סז]}\צהגדרה{. }

\הגדרה{ע״ע הצלחה אמיתית. ע״ע עונג. ע״ע בטול. ע״ע חשק. }

\paragraphs

\משנה{דבקות בד׳ נותן התורה}\myfootnote{ ע׳ בהקדמת המהר״ל, תפארת ישראל, עמ׳ ב ג: שלא בירכו בתורה תחילה פירוש ״שלא היו דבקים בו יתברך באהבה במה שנתן תורה לישראל״. \label{2}}\צהגדרה{ - זכירת היצירה האלוהית של נשמת האומה אשר אל חי בקרבה, השכינה שורה בתוכה, וחיי עולם של תורה נטועים בתוכה }\צמקור{[שי׳ ה 46 מאגרת רבינו, מכ״ה א תשל״ז]. }

\paragraphs

\ערך{דבר ד׳ }\הגדרה{- כל התיאור והבנין העולמי עם כל חוקותיו }\מקור{[קובץ ה קצב]}\צהגדרה{. }

\paragraphs

\ערך{דבר ד׳ }\הגדרה{- סוד הידיעה\hebrewmakaf האלהית\mycircle{°} }\מקור{[קובץ ה קמג]}\צהגדרה{.}

\paragraphs

\ערך{דבש }\הגדרה{- מורה על תכן של הנאה ומתיקות והחשת ענג מוחשי, הקשורות עמו }\מקור{[ע״ר א קמד]}\צהגדרה{. }

\הגדרה{ע׳ במדור נפשיות, גאוה, ב״שאור ודבש״. ע׳ במדור מצוות, הלכות, מנהגים וטעמיהן, דבש, סוד איסור הדבש בהקטרה ע״ג המזבח.}

\paragraphs

\ערך{דגל }\הגדרה{- }\מעוין{◊}\הגדרה{ מסמן את ההכרה של הנקודה היסודית, שכל התוכן של הקבוץ תלוי בה. דגל הצבא מרכז את רוח המחנה }\מקור{[ע״ר א מט]}\צהגדרה{. }

\paragraphs

\ערך{דין }\הגדרה{- יד שמאל\mycircle{°}, המכשיר לשפע הטוב\mycircle{°} (שהוא התכלית - יד ימין\mycircle{°}) בזמן המעשה, פחד העונש והיראה\mycircle{°} <שהיא תחילת הכניסה לדרכי\hebrewmakaf ה׳\mycircle{°} ית׳> }\מקור{[עפ״י מא״ה ב רפד (פנק׳ ג קפו)]}\צהגדרה{. }

\paragraphs

\ערך{דין }\הגדרה{- }\משנה{״יכולני לפטור את כל העולם כולו מן הדין״}\myfootnote{ סוכה מה:.\label{3}}\הגדרה{ - מחובות מצרים וגבולים צרים }\מקור{[עפ״י א״ק ג ט, שלא]}\צהגדרה{. }

\paragraphs

\ערך{דין }\הגדרה{- }\משנה{״משמים השמעת דין״}\myfootnote{ תהילים עו ט.\label{4}}\הגדרה{ - אושר המציאות במגמתה התכליתית, רום נשמת האדם ובהיקותה, אומצה וחוסנה הטהורים, במעמדם הנצחי החובק כל אושר, כל שאיפה וכל עדנה. הכבוד\hebrewmakaf העליון\mycircle{°} של הוד\mycircle{°} נשמת\mycircle{°} כל היקום }\מקור{[עפ״י ע״א ד ט ע]}\צהגדרה{. }

\paragraphs

\ערך{דין }\הגדרה{- }\משנה{מדת הדין }\הגדרה{- כח המגביל את האור\mycircle{°} שלא יתרבה על הכלים\mycircle{°}, וישברו\mycircle{°}. הכח העוצר והמעכב, הפועל שלא יבקעו הדברים אל הפועל בעוד שאין הזמן גרמא, בעוד שאין הכשר לזה מצד האנושיות }\מקור{[עפ״י א״ב 65 (א״ה ב, מהדורת תשס״ב, 87)]}\צהגדרה{. }

\הגדרה{ע׳ במדור מונחי קבלה ונסתר, רחמים, מדת הרחמים. }

\paragraphs

\ערך{דין }\הגדרה{- }\משנה{מדת הדין }\הגדרה{- ההנהגה העליונה המצמצמת, המודדת מדה כנגד מדה }\מקור{[עפ״י ע״ר א ריג]}\צהגדרה{. }

\הגדרה{ע׳ במדור מונחי קבלה ונסתר, חסד, מדת החסד. ושם, גבורה, מדת הגבורה (האלהית).}

\paragraphs

\ערך{דין }\הגדרה{- }\משנה{מדת הדין האיתנה }\הגדרה{- הקו המחשבתי\mycircle{°} בהמצאת ההויה הנגלית שתהיה מפוארה כ״כ עד ההשתוות, בפעולה והופעה\mycircle{°} נאצלת\mycircle{°}, עם ההויה הגנוזה בסתר\hebrewmakaf עליון\mycircle{°} }\מקור{[עפ״י ע״א ד ט קלז]}\צהגדרה{. }

\משנה{מדת הדין העליונה }\צהגדרה{- אור האמת\mycircle{°} }\צמקור{[ע״ר ב תפח].}

\צהגדרה{תפיסת הבריאה מצד תוכן הרצון העליון, המתגלה לה בערכו העצמותי עד לחשבון צמצומה וגבוליותה }\צמקור{[ב״ר שעח].}

\משנה{מדת הדין של מעלה בעוצם חזקה }\הגדרה{- האידיאליות המאירה שברעיון היצירה, הקודמת לכל עולמים, הדורשת את האור בתכלית בהירותו }\מקור{[ר״מ צה]}\צהגדרה{. }

\משנה{הדין הקדוש }\הגדרה{- התוכן האידיאלי של כל היקום טרם הבראותו }\מקור{[עפ״י ר״מ לא]}\צהגדרה{. }

\הגדרה{ע׳ במדור מונחי קבלה ונסתר, גבורה גנוזה. ושם, גבורות, הגבורות. ע׳ במדור מדתם ועניינם הרוחני של אישי התנ״ך, יצחק, מדתו של יצחק. ר׳ במדור פסוקים ובטויי חז״ל, מדת הדין, שעלתה במחשבה, לפני בריאת העולם. }

\paragraphs

\ערך{דין }\הגדרה{- }\משנה{מדת הדין של מעלה }\הגדרה{- הבינה\hebrewmakaf העליונה\mycircle{°}, שהדינים מתעוררים מצדה, כדי לבסם העולם ולשכללו }\מקור{[פנק׳ ג שדמ]}\צהגדרה{.}

\paragraphs

\ערך{דין }\הגדרה{- }\משנה{מדת הדין המיוחסת לבית שמאי}\myfootnote{ זוהר ח״ג רמה.\label{5}}\הגדרה{ - העליה אל חקר עומק החכמה, היא החכמה הגדולה המופשטת ונעלה מכל רגש, בה כלולים כל השלמויות }\מקור{[עפ״י ע״א ב ח ד]}\צהגדרה{. }

\הגדרה{ע״ע חסד, מדת החסד המיוחסת לבית הלל. ע׳ קבלה ונסתר, הלכה כבית שמאי לעתיד לבוא. }

\paragraphs

\ערך{דין }\הגדרה{- }\משנה{מדת הדין הקשה }\הגדרה{- }\מעוין{◊ }\הגדרה{מתיסדת ע״פ המבט של ענין היצורים כשהם מצד עצמם }\מקור{[ע״ר א צט]}\צהגדרה{. }

\paragraphs

\ערך{דין }\הגדרה{- }\משנה{דינים }\הגדרה{- הגבורות\mycircle{°} והצמצומים\mycircle{°} }\מקור{[ח״פ 6]}\צהגדרה{.}

\paragraphs

\ערך{דין }\הגדרה{- }\משנה{דינים }\הגדרה{- הפעולות המרעישות, המחבלות והנראות כמהרסות ומחריבות את העולם }\מקור{[ע״ר א קלד]}\צהגדרה{. }

\paragraphs

\משנה{דין }\הגדרה{- }\צמשנה{דינים קשים }\צהגדרה{- הדעות הרעות, שהן המביאות את המעשים הרעים }\צמקור{[ק״ו קנה].}

\הגדרה{ר׳ במדור מונחי קבלה ונסתר, מתוק דינים.}

\paragraphs

\ערך{דין }\הגדרה{- }\משנה{כובד הדין }\הגדרה{- ע׳ במדור מונחי קבלה ונסתר, ״חבוט הקבר״. }

\paragraphs

\ערך{דין }\הגדרה{- }\משנה{עומק הדין }\הגדרה{- הדקדוק היותר מכוון שבדין }\מקור{[קובץ ה קסד]}\צהגדרה{. }

\paragraphs

\ערך{דין }\הגדרה{- }\משנה{שורת הדין}\הגדרה{ - כל מעשה וכל הנהגה בפרטיותה בצדק וביושר, כתורה  וכמצוה }\מקור{[עפ״י ע״ר א צח]}\צהגדרה{. }

\paragraphs

\תערך{דליגה }\תמקור{- }\תהגדרה{פעולת הרצון\mycircle{°} החפשי\mycircle{°} ההחלטי מכל סטרא, ה״רצון\hebrewmakaf הפשוט״\mycircle{°}, שאינו בדרך השתלשלות\mycircle{°} מדרגא לדרגא }\תמקור{[נ״א ה 26]. }

\הגדרה{ע׳ במדור מונחי קבלה ונסתר, ״זריקה״. }

\paragraphs

\ערך{דלת }\הגדרה{- מה שסותם את החלל הפתוח }\מקור{[ע״א א ב נה]}\צהגדרה{.}

\paragraphs

\ערך{דם }\הגדרה{- }\משנה{(בביטוי ״דם יהודי״ וכדומה) }\הגדרה{- ״הדם הוא הנפש״ הטבעית בגלוייה המעשיים }\מקור{[רצי״ה אג׳ ב שמג]}\צהגדרה{. }

\מעוין{◊ }\הגדרה{בו טבוע כח החיים, <והוא מצד עצמו משולל השלמות ובהמי, אבל הוא בטבע נכנע אל השכל והקדושה> }\מקור{[מא״ה ב רסט-ע]}\צהגדרה{.}

\מעוין{◊ }\הגדרה{בו טבוע כח החיים והרצון, <שאמנם מצד עצמו הוא בהמי ומשולל השלמות, אבל כפוף הוא אל כח השכל והקדושה הפועל עליו ומטביע עליו את צורתו> }\מקור{[ע״ר א רנח]}\צהגדרה{.}

\paragraphs

\ערך{דמות }\הגדרה{- תוכן המציין בצורה מיוחדת וידועה איזה תאר מוגבל }\מקור{[ע״ר א נב]}\צהגדרה{. }

\הגדרה{ע׳ בנספחות, מדור מחקרים, צלם, דמות, תבנית, צורה.}

\paragraphs

\ערך{דמות האדם }\הגדרה{- התוכן של מרכז העולם, הרצון\mycircle{°} החפשי\mycircle{°}, שיתעלה\mycircle{°}, ויהיה, עם כל חופשו שהוא כשרון עצמי עדין, עזיז\mycircle{°} ומתמיד בפעולתו לטובה\mycircle{°} קבועה ההולכת ומתעלה }\מקור{[עפ״י א״ק ב תקס]}\צהגדרה{. }

\הגדרה{ע׳ במדור מונחי קבלה ונסתר, אדם עליון כללי. ע׳ במדור פסוקים ובטויי חז״ל, צלם אלהים. ע׳ במדור אדם הראשון, ״אדם הראשון״, נשמתו בכל מלואה.}

\paragraphs

\ערך{דמיון }\הגדרה{- }\משנה{עולם הדמיון }\הגדרה{- האספקלריא\hebrewmakaf שאינה\hebrewmakaf מאירה\mycircle{°}, המלאה הדר\mycircle{°}, תבנית כל צורה מפוארה, כל חזון לב מרומם ומתעלה }\מקור{[א״ק א רמב]}\צהגדרה{. }

\paragraphs

\ערך{דמיון }\הגדרה{- ע׳ במדור הכרה והשכלה והפכן.}

\paragraphs

\ערך{דעת אלהות\mycircle{°}}\הגדרה{ - }\משנה{הבינה היותר שלמה שלה }\הגדרה{- הכרת היחש האלהי אל העולם הכללי ואל כל פרט ופרט מפרטיו, החמריים\mycircle{°} והרוחניים\mycircle{°}, (כ)יחש הנשמה\mycircle{°} אל הגוף\mycircle{°} }\מקור{[עפ״י א׳ מח]}\צהגדרה{. }

\הגדרה{ע׳ בנספחות, מדור מחקרים, אלהות, שני דרכים בהכרת האלהות. ע׳ במדור שמות כינויים ותארים אלהיים, הבנה האישית של האלהות. ושם, הבנה הכללית של האלהות.}

\paragraphs

\ערך{דעת את ד׳ וההליכה בדרכיו }\הגדרה{- התרחבות הכח המוסרי\mycircle{°} }\מקור{[ע״א א ה פג]}\צהגדרה{.}

\paragraphs

\ערך{דעת ד׳}\הגדרה{ - תלמוד של חלקי התורה הנוגעים לבירורן של המדות והדעות }\מקור{[ל״ה 178 (פנק׳ ב קכב)]}\צהגדרה{.}

\paragraphs

\ערך{דעת ד׳ ויחודו}\הגדרה{ - }\מעוין{◊ }\הגדרה{דעת ד׳ ויחודו הוא מצד הנשמה\mycircle{°} לבד}\צהגדרה{ }\מקור{[פ״א קעח]}\צהגדרה{. }

\paragraphs

\ערך{דעת צור\hebrewmakaf העולמים\mycircle{°} }\הגדרה{- דבקות בחיי\hebrewmakaf החיים\mycircle{°}, המתעלה על כל גבול ומחזה עין בשר }\מקור{[עפ״י ע״א א (מהדורא בתרא) א ב]}\צהגדרה{.}

\paragraphs

\ערך{דק }\הגדרה{- רוחני }\מקור{[עפ״י פנק׳ ד תלא]}\צהגדרה{.}

\הגדרה{ע״ע גס.}

\paragraphs

\ערך{דרור }\הגדרה{- }\משנה{הדרור האמיתי }\הגדרה{- הדרור המתאים להתכונה הפנימית הנטועה בנפש אשר עשה אלהים אותה ישרה\mycircle{°} }\מקור{[ע״א ד יג יא]}\צהגדרה{. }

\הגדרה{ע״ע חרות. ע״ע חפש. }

\paragraphs

\ערך{דרישה}\myfootnote{ \textbf{דרישה} - הכוונה למחקר המדעי.\label{6}}\הגדרה{ - }\צהגדרה{האימות היותר גדול בה}\הגדרה{ - הניסיון המוחשי }\מקור{[פנק׳ ג כד]}\צהגדרה{.}

\paragraphs

\ערך{דשן }\הגדרה{- המחיה הגופנית של המאכל, המלא לשד ושמן, החומר המחיה, האוצר בתוכו כח חיים רב ועצום, החוזר לערך של חיים בהבלעו בגויה המצומצמה }\מקור{[ע״ר א כ]}\צהגדרה{. }

\paragraphs

\ערך{דשן }\הגדרה{- }\משנה{״דשן ביתך\mycircle{°}״}\myfootnote{ תהילים לו ט.\label{7}}\הגדרה{ - התוכן התמציתי של הזוהר\mycircle{°} האלהי\mycircle{°} העליון\mycircle{°} בצורה מוגבלה, הראויה להיות המזון המבריא, והנותן את המחיה להנשמה הבריאה, השוקקת אל הנועם\mycircle{°} האלהי }\מקור{[ע״ר א כ]}\צהגדרה{. }

\paragraphs

\ערך{דת }\הגדרה{-}\צהגדרה{ א}\הגדרה{מצעי עוזר להכשיר את המעשים, המדות, הרגשות, סדרי החברה החיצונים והפנימיים, באופן מתאים המכשיר את החיים ואת ההויה לדעת\hebrewmakaf אלהים\mycircle{°} }\מקור{[קבצ׳ ב נט]}\צהגדרה{.}

\משנה{הדת המעשית }\הגדרה{-}\משנה{ תוכנה, וכל הרגשות המתיחסים לה }\הגדרה{- רוח אלהים\mycircle{°} בצבע\mycircle{°} האומה }\מקור{[קבצ׳ ב פג]}\צהגדרה{.}

\ערך{דת }\הגדרה{-}\משנה{ ״אורתודוקסיה״, נשמתה }\הגדרה{- תכונת דרישת\hebrewmakaf ד׳\mycircle{°} }\מקור{[עפ״י אג׳ ב ח]}\צהגדרה{. }

\משנה{רגשי דת}\הגדרה{ - ציור\mycircle{°} קרבת\hebrewmakaf אלקים\mycircle{°} על דרך השכר והעונש }\מקור{[קבצ׳ ב ל]}\צהגדרה{.}

\ערך{דת }\הגדרה{- }\משנה{רגש דתי}\הגדרה{ - רגש נשגב נטוע בנפש האדם מיוצר נשמתו, כדי להטביע בקרבו את היחש האמיתי שבין האדם לקונו, בתור היחש הראוי להיות בין יציר ליוצרו. [רגש ה]חובק בזרוע עזו את כל יסודות החיים והמוסר\mycircle{°} הכללי והפרטי }\מקור{[עפ״י ל״ה 132]}\צהגדרה{.}

\משנה{עקרה של הדת }\הגדרה{- להרגיל את האדם בהכנעה ושיעבוד לאלהים\mycircle{°} לפי מובן האדם בהוה. הרגל הנפשות לדביקות\mycircle{°} ועבודת\hebrewmakaf ד׳\mycircle{°}, שממנו יתד ופינה לכל מוסר ומעגל טוב. מדה כללית מוסרית הנובעת ממקור ההכרה האמיתית של יחש האדם לקונו ולנשמתו הרוחנית }\מקור{[עפ״י ל״ה 81]}\צהגדרה{.}

\משנה{הכלל הכולל את הדתות }\הגדרה{- ההכשרה הרוחנית של האדם ביחשו לקונו על פי הרגש המוסרי הפנימי}\צהגדרה{ }\מקור{[ל״ה 82 (פנק׳ ב סג)]}\צהגדרה{.}

\משנה{רגש דת פשוט }\הגדרה{- }\מעוין{◊}\הגדרה{ מתבאר לאדם עפ״י השתדלותו להיות יותר קרוב לבוראו בדרכיו, מעשיו ודעותיו }\מקור{[עפ״י ע״א ב 384]}\צהגדרה{. }

\משנה{יסוד הרגש הדתי }\הגדרה{- הגעגועים אל הקדושה\mycircle{°} ורוממות הנפש לאהבת\hebrewmakaf ד׳\mycircle{°} וכבודו ופחד גאונו }\מקור{[מא״ה ב מד (קבצ׳ א מו)]}\צהגדרה{.}

\הגדרה{ע״ע עבודת ד׳, (המצויה גם בעמים). ע״ע רוח האמונה. ע״ע רליגיוזיות, הטבעיות הרליגיוזית. ע״ע דת, בישראל, רגש הדת בישראל. ע׳ במדור תורה, תורת חיים. ע׳ במדור אליליות ודתות, דת, אצל כל עם ולשון (חוץ מישראל). ע״ע אידיאה דתית.}

\paragraphs

\ערך{דת }\הגדרה{- }\משנה{(לעומת מסורת\mycircle{°}) }\הגדרה{- יסודה הפנימי של המסורת }\מקור{[ב״א 11]}\צהגדרה{. }

\paragraphs

\ערך{דת }\הגדרה{- }\משנה{בישראל }\הגדרה{- אור\hebrewmakaf ד׳\mycircle{°} שבנשמת\hebrewmakaf (ישראל\mycircle{°}), הבעת החיים היותר עצמיים והיותר פנימיים שלו, מה שנתן ונותן לו את הכל, שמעמידו על הרום העליון של במתי\hebrewmakaf ארץ, על המעמד של מורה האנושיות כולה }\מקור{[אג׳ ב רט]}\צהגדרה{. }

\הגדרה{״גופא דאורייתא״, ״נר אלהים בארץ״  }\מקור{[עפ״י א״ת ב 217 (פנק׳ ד עב)]}\צהגדרה{.}

\משנה{דת ישראל}\הגדרה{ - המבוע של אור\hebrewmakaf ד׳\mycircle{°} בעולם}\צהגדרה{ }\מקור{[ל״ה 104]}\צהגדרה{.}

\משנה{רגש הדת בישראל}\הגדרה{ - }\מעוין{◊ }\הגדרה{בא מהגעגועים אל הקדושה ורוממות הנפש לאהבת\hebrewmakaf ד׳\mycircle{°} וכבודו\mycircle{°} ופחד הדר\mycircle{°} גאונו; שכל התורה כולה היא הכנה לזכות את העולם (בעתיד) לדברים שעליהם מיוסד טבע הגעגועים האלה אל הטוב}\צהגדרה{ }\מקור{[עפ״י פנק׳ א קכג (מא״ה ב מד)]}\צהגדרה{. }

\הגדרה{ע״ע מוסר, רגש המוסר. ע״ע רליגיוזיות, הטבעיות הרליגיוזית.}

\משנה{דתיות }\צהגדרה{- הגילוי וההוצאה\hebrewmakaf לפועל, במחשבה ובמעשה, של התורה\mycircle{°} ומצוותיה\mycircle{°} }\צמקור{[ל״י א לג].}

\הגדרה{ע״ע עבודת אלהים}\צהגדרה{.}

\paragraphs

\ערך{דת }\הגדרה{- }\משנה{אצל כל עם ולשון (חוץ מישראל) }\הגדרה{- ע׳ במדור אליליות ודתות.}\mylettertitle{ה}

\paragraphs

\ערך{הא }\הגדרה{- מורה על המוכן ומוכשר להושטה }\מקור{[עפ״י ר״מ פה]}\צהגדרה{. }

\paragraphs

\ערך{האדרת שם ד׳}\הגדרה{ - גלוי עז\mycircle{°} הגבורה\hebrewmakaf האלהית\mycircle{°} הפועלת את הכל למען הרוממות האצילית, המסוקרת אך לפני כסא\hebrewmakaf כבודו\mycircle{°} של בורא כל העולמים ברוך הוא. ההופעה העזיזה החודרת מרום הגובה העליון עד שפל המדרגה של אדם על הארץ\mycircle{°} }\מקור{[עפ״י ע״א ד ט קד]}\צהגדרה{. }

\paragraphs

\ערך{האח }\הגדרה{- קריאת השמחה\mycircle{°} והחדוה\mycircle{°} }\מקור{[ר״מ קכ]}\צהגדרה{. }

\paragraphs

\ערך{האצלה}\myfootnote{ בשו״ת הרדב״ז, ח״ג סי׳ תתקי ״בהיות האדם מתכוון אל רבו ונותן אליו לבו תתקשר נפשו בנפשו ויחול עליו מהשפע אשר עליו ויהיה לו נפש יתירה וזה נקרא אצלם סוד העיבור בחיי שניהם, וזה הוא שנאמר ״והיו עיניך רואות את מוריך״ וזהו ״והתיצבו... עמך שם ואצלתי מן הרוח״.\label{1}}\הגדרה{ - הזרחת\mycircle{°} אור\mycircle{°} ויפעת\mycircle{°} אצילות\mycircle{°}, המעדנת את הנשמה\mycircle{°} בהשפעה\mycircle{°} נעלה מעל מכל ערכי בינה\mycircle{°} וחקר }\מקור{[ע״ר א לג]}\צהגדרה{. }

\paragraphs

\ערך{הארה }\הגדרה{- שביעה רוחנית\mycircle{°} }\מקור{[עפ״י ע״ר א קלד]}\צהגדרה{. }

\משנה{הארה גדולה }\הגדרה{- הופעה\mycircle{°} נשמתית אלהית, מגלה רזי\mycircle{°} עליון, מודיעה האגור בסתרי חושך }\מקור{[עפ״י א״ק ב שיז (ע״ט קיד)]}\צהגדרה{. }

\משנה{הארה רוחנית }\הגדרה{- רוממות נשמה\mycircle{°} וחדות\mycircle{°} קודש\mycircle{°} }\מקור{[שם ג שמב]}\צהגדרה{. }

\הגדרה{ע״ע מאיר, מאיר את העולמים. ע׳ בנספחות, מדור מחקרים, זריחה. }

\paragraphs

\ערך{הארת הנשמה }\הגדרה{- ע׳ במדור נפשיות, נשמה. }

\paragraphs

\ערך{הארת זיו שכינת אל }\הגדרה{- עלית אור עולמי עולמים, הדר כבוד אל הכבוד}\צהגדרה{ }\מקור{[א״ק ב תקל]}\צהגדרה{.}

\paragraphs

\ערך{הבטה}\myfootnote{ מובחנת ההבטה מן ה״ראיה בעלמא״ דבה לא קני, ״דהבטה בהפקר קני״ לתוס׳ ב״מ ב. ד״ה דבראיה, ״היינו שעשה מעשה כל דהו״, שעל ידו קונה הראיה ועל כן היא קרויה - הבטה; ולרש״י שם קיח. ד״ה אתה אומר ״הבטה קניא הואיל ודבר טורח הוא ודעתו לכך״. של״ה, בי׳ מאמרות, מאמר ט, דף מא: ״יש חילוק בין הבטה לראיה כח הבטה יותר הסתכלות מראיה דעלמא כמו שפרש״י בפ׳ חוקת אצל נחש הנחשת שכתיב וראה וכתיב והביט ע״ש״. ובאור החיים עה״ת, בלק כג כא ״דקדק לומר לשון הבטה שהיא יותר מהראיה, פירוש כי הגם כי בראיה ראשונה הוא רואה מחשבה הרעה, כשהוא מביט בפנימיות מפעלה אין און״. וברוח חיים על פרקי אבות, פרק א לפני ההג״ה הראשונה ״ולשון הבטה הוא הסתכלות יתירה בעצם הענין״. ובהעמק דבר לנצי״ב במדבר כא ח, ושם כג כא: ״הסתכלות בתוך ופני הדבר״. ובאבי עזר על פי׳ הראב״ע במדבר כא ז ״גזרת הביט בנויה על השכלה זכה״. ובנפש\hebrewmakaf חיה לרר״מ סי׳ רכה ״הביט - מביט בו בכוונה״. וכן גם במלבי״ם ביאיר אור, הכרמל, ערך הבט: ״פעל הביט מציין קשורו עם המובט הוא הגיוני ומחשביי לא במציאות ובחוש הראות לבד רק משים לב על העצם לדעת מהותו וענינו״. ובתורה והמצוה, שמות ג ו ״ההבטה הוא שימת לב על הדבר ואינו בא על ראות העין״. ובבראשית, טו ה, ״ההבטה היא שימת לב על הדבר ועיון השכל... יבא לרוב על שימת הלב ועיון המחשבה... וכשבא על הראיה אמר במדרש (בר״ר פר׳ מד יב) ״אין הבטה אלא מלמעלה למטה״״. אמנם צ״ע האם עולים דבריו אלה האחרונים של המלבי״ם עם דברי המ״ר איכה פר׳ ה א ״הביטה וראה את חרפתנו. ר׳ יודן אמר הבטה מקרוב וראייה מרחוק. הבטה מקרוב שנאמר (מלכים א יט) ״ויבט והנה מראשותיו עוגת רצפים״ וראייה מרחוק שנאמר (בראשית כב) ״וירא את המקום מרחוק״. ר׳ פנחס אומר הבטה מרחוק שנאמר (תהלים פ) ״הבט משמים וראה״, וראייה מקרוב שנאמר (בראשית לב) ״וירא כי לא יכול לו ויגע בכף ירכו״, והי״ע. ואולי אפשר להבחין בין המושגים עפ״י ר״מ צז: ״מושג הראיה, וכו׳ מורה הסמנה מדויקת בפרטים, והארה כוללת בכללים״; לעומת ההבטה שהיא כניסת הרשמים מן החוץ אל הפנים. וצ״ע.\label{2}}\הגדרה{ - הסתכלות בהירה וחודרת, שמכנסת בתוך הנושא המקבל את הרושם, את כל הפרטים של הקוים והשרטוטים אשר להמושגים }\מקור{[ר״מ קלב]}\צהגדרה{. }

\הגדרה{הכרת הפרטים, אחרי שהסתמנו לפרטיהם והתקשרו אל בית קבולם }\מקור{[שם קלה]}\צהגדרה{. }

\הגדרה{ע׳ במדור גוף האדם אבריו ותנועותיו, עין. ע׳ במדור אותיות, עי״ן. }

\paragraphs

\ערך{הבל }\הגדרה{- דבר כחני, שלא יצא אל הפועל }\מקור{[ע״ר א קה]}\צהגדרה{. }

\הגדרה{דבר שבכח שאיננו כלל בפועל }\מקור{[שם שם]}\צהגדרה{. }

\הגדרה{כל מציאות של כח, שאין לה ערך בפועל }\מקור{[שם שם]}\צהגדרה{. }

\משנה{הבל פה}\הגדרה{ - כח פנימי עצור שעתיד לצאת אל הישות המוחלטת }\מקור{[ע״א ד ט צו]}\צהגדרה{. }

\ערך{הבל }\הגדרה{- כח של הדיבור\mycircle{°} }\מקור{[מ״ש מד]}\צהגדרה{. }

\הגדרה{הדבור שבכח }\מקור{[שם נא]}\צהגדרה{. }

\הגדרה{הכנה לדיבור }\מקור{[מ״ר 423]}\צהגדרה{.}

\paragraphs

\ערך{הגבלה }\הגדרה{- צמצום חמרי\mycircle{°} ומעצור }\מקור{[עפ״י ע״ר א קנד]}\צהגדרה{. }

\הגדרה{ע״ע מגביל. ע״ע גבולים. }

\paragraphs

\ערך{הגשמה }\הגדרה{- הוצאה אל הפועל }\מקור{[עפ״י א״ק ג קו]}\צהגדרה{. }

\paragraphs

\ערך{הוד }\הגדרה{- יופי, נוי\mycircle{°}, שאמתתו היא בהשלמת כל הפרטים וערכם\mycircle{°} הנעים והמדויק }\מקור{[עפ״י ח״פ לז.]}\צהגדרה{. }

\ערך{הוד }\הגדרה{- }\משנה{(בהויה\mycircle{°}) }\הגדרה{- ההתאמה הנאה של כל אורות\hebrewmakaf החיים\mycircle{°} וכל כוחות ההויה כולם בהסתדרותם הנפלאה }\מקור{[ע״ר א יב]}\צהגדרה{. }

\paragraphs

\ערך{הוד תפארתו\mycircle{°}}\הגדרה{ - התגלותו ביקרו }\מקור{[א״ק ג שמג]}\צהגדרה{. }

\paragraphs

\ערך{הודאה }\הגדרה{- הבעה של הכרת\hebrewmakaf טובה\mycircle{°} למטיב, ההולכת מתוך הרגש המלא, שדוחק את הלב, שלא ישאר תוכן הטובה מבלי הבעה של הכרה זו }\מקור{[עפ״י ע״ר א קז\hebrewmakaf ח]}\צהגדרה{. }

\הגדרה{מורה חובת תודה\mycircle{°} על קבלת הטוב\mycircle{°} }\מקור{[ע״א ב ט ד]}\צהגדרה{.}

\הגדרה{ע׳ בנספחות, מדור מחקרים, ברכה לעומת הודאה. }

\paragraphs

\ערך{הודאה }\הגדרה{- }\משנה{תוכן ההודאה (לד׳) }\הגדרה{- הכרת\hebrewmakaf הטובה\mycircle{°}, שהאדם מוקף בים של השפעת\mycircle{°} הטוב\hebrewmakaf העליון\mycircle{°}, ונפשו בתכונתה העדינה מכריחה אותו, שלא יהיה כפוי\hebrewmakaf טובה וניב שפתיו יתפרץ בהבעה של הכרת הטובה למטיב העליון ב״ה }\מקור{[ע״ר א קצב\hebrewmakaf ג]}\צהגדרה{. }

\הגדרה{ע״ע תודה. }

\paragraphs

\ערך{הויה }\הגדרה{- יציאה למציאות בפועל }\מקור{[עפ״י ע״א ד ו קו]}\צהגדרה{. }

\הגדרה{התגלות אל הפעל }\מקור{[עפ״י א״ש ו ו]}\צהגדרה{. }

\paragraphs

\ערך{הויה }\הגדרה{- היקום וכל העולמים כולם }\מקור{[ע״ר א קנז]}\צהגדרה{. }

\משנה{כל ההויה כולה}\הגדרה{ - כל החיים, כל היופי, כל העז, כל הצדק, כל הטוב, כל הסדר, כל ההתעלות }\מקור{[א״ק ב שמט]}\צהגדרה{.}

\הגדרה{ע׳ בנספחות, מדור מחקרים, הויה. ושם, הויה, מציאות, חיים. }

\paragraphs

\ערך{הופעה אלהית\mycircle{°}}\הגדרה{ - הסתעפות חיים וישות ממקור\hebrewmakaf החיים\mycircle{°} והיש }\מקור{[א״ק א ב (מ״ר 401)]}\צהגדרה{. }

\משנה{ההופעה האלהית הכוללת כל }\הגדרה{- אור הקודש\hebrewmakaf העליון\mycircle{°} }\מקור{[ע״ר ב עז]}\צהגדרה{. }

\paragraphs

\ערך{הזרחה }\הגדרה{- ע׳ בנספחות, מדור מחקרים, זריחה. }

\paragraphs

\ערך{הטבה }\הגדרה{- }\משנה{(הטבה רוחנית)}\הגדרה{ - הארה\mycircle{°} והרחבה, של החיים\hebrewmakaf הרוחניים\mycircle{°}, בהתפשטותם המלאה\mycircle{°} והכבירה }\מקור{[עפ״י ע״ר א קמו]}\צהגדרה{. }

\ערך{הטבה }\הגדרה{- }\משנה{ההטבה היותר עליונה\mycircle{°}}\הגדרה{ - ההתעלות\mycircle{°} הקדושה\mycircle{°} היותר טהורה\mycircle{°}, ואור\hebrewmakaf ד׳\mycircle{°} הבהיר בהגלותו בתועפות עזו }\מקור{[שם קסו]}\צהגדרה{. }

\הגדרה{ע״ע טוב. }

\paragraphs

\ערך{הטבעה }\הגדרה{- קביעת צורה }\מקור{[רצי״ה א״ש יא הערה 23]}\צהגדרה{. }

\paragraphs

\ערך{היכל}\הגדרה{ - ע׳ בנספחות, מדור מחקרים, בית לעומת היכל.}

\paragraphs

\ערך{היכל }\הגדרה{- הזיו\mycircle{°} התוכי הפנימי\mycircle{°} של אל אלהי\mycircle{°} ישראל\mycircle{°}, המיחד אותנו בעליות\mycircle{°} הקודש\mycircle{°} של קדושת\mycircle{°} המצוות\mycircle{°} והאורות\hebrewmakaf האלהיות\mycircle{°} המיוחדות לעם סגולה\mycircle{°}. הטרקלין }\מקור{[ע״ר א ד]}\צהגדרה{. }

\הגדרה{ע׳ במדור פסוקים ובטויי חז״ל, חצרות ד׳. ע׳ שם, טרקלין מלכו של עולם.}

\paragraphs

\ערך{״היכל ד׳״ }\הגדרה{- בית\hebrewmakaf המקדש\mycircle{°} מצד היותו מרכז הקדושה\mycircle{°} וע״י נשפע טהרה\mycircle{°} בלב והכשר לכל המדות היקרות. כנגד קדושת המדות המכינות יותר את לב האדם לדעת\hebrewmakaf את\hebrewmakaf השי״ת\mycircle{°} וכבודו\mycircle{°} }\מקור{[עפ״י ע״א א א ז]}\צהגדרה{.}

\הגדרה{ע׳ במדור משכן ומקדש, ״בית ד׳״.}

\paragraphs

\ערך{״היכל מלך״ }\הגדרה{- }\משנה{סודו }\הגדרה{- ע׳ במדור מונחי קבלה ונסתר.  }

\paragraphs

\ערך{היפנוטיזם }\הגדרה{-◊  מי שנפשו מסוגלת לזה, מושל על המושפע ממנו, עד שמה שרוצה שיצייר\mycircle{°}, כך מצוייר ממילא }\מקור{[פנק׳ ג פז]}\צהגדרה{.}

\paragraphs

\ערך{הכנה }\הגדרה{- }\משנה{(לעבודת ד׳)}\הגדרה{ -  הכשרתו של האידיאל\mycircle{°} ההולך ומתהוה }\מקור{[ע״ר א קפו]}\צהגדרה{. }

\paragraphs

\ערך{הכספה }\הגדרה{- כחישות ומיעוט ההבהקה }\מקור{[ע״א ב ז מט]}\צהגדרה{. }

\paragraphs

\ערך{הכרת טובה}\myfootnote{ \textbf{הכרת טובה} - ע״ע פנק׳ א ו\hebrewmakaf ז ובהערה 1 שם. א״ה 728, 8\hebrewmakaf 757, א״ת פרק י יז.\label{3}}\הגדרה{ - }\מעוין{◊}\הגדרה{ מקור הרגש, הטבעי כ״כ לאדם, להתנשא אל הקדש\mycircle{°} ואל השאיפה האלהית\mycircle{°} הנעלה, גם במדה המתאימה לצמצום\mycircle{°} כוחות הנפש הטבעיים }\מקור{[ע״ר א קעה]}\צהגדרה{. }

\מעוין{◊ }\הגדרה{העמוד המוסרי\mycircle{°} היותר גדול ונשגב, שכשיתפתח כל צרכו בלבות בני אדם יהי׳ עוזר מאד אל התיקון הכללי }\מקור{[ע״א ג א יד]}\צהגדרה{.}

\הגדרה{ע״ע הודאה. ע״ע תודה. }

\paragraphs

\ערך{הכתרה }\הגדרה{- }\משנה{ענין הכתרת מלך}\הגדרה{ - מורה תוספות גדולה והנהגה נסתרת שיש בשפע הרצון שהשי״ת משפיע על המלך\mycircle{°} }\מקור{[פנק׳ ג רמז]}\צהגדרה{.}

\הגדרה{ע״ע שליטה.}

\paragraphs

\ערך{הִלוּל }\הגדרה{- }\משנה{ההילול האמיתי לשם ד׳ }\הגדרה{- הכרת הוד כבודו\mycircle{°} ושלמותו, משלמות מעשיו ופעולותיו. שיגלה בחיים תמיד צד יותר עליון מהשלמה, מה שלא היה לפני זה. <בזה יתואר הילול מגזרת ״יהל אור״\mycircle{°}>}\myfootnote{ לקוטי תורה לרש״ז, בשלח, דף א עמודה ב ס״ק ב. ״״כל הנשמה תהלל י״ה״, פי׳ תהלל לשון שבח ולשון הארה כמ״ש בהלו נרו היינו שהנשמה  תמשיך אור וגילוי שם י״ה וכו׳״.\label{4}}\הגדרה{, הארה חדשה, הברקה חדשה, הוספת שלמות חדשה }\מקור{[עפ״י ע״א ג ב לו]}\צהגדרה{. }

\הגדרה{מגזרת ״יהל אור״, הארה\mycircle{°} והופעה\mycircle{°} בהכרת החיים העליונים\mycircle{°} ברום ערכם, בהופיעם ממכון הטוב\mycircle{°} והעלוי\mycircle{°} הנשגב\mycircle{°}, להאיר על האופק של המציאות\mycircle{°} המוגבלה\mycircle{°} שפעת נהורים רחבי ידים וגדולי ערך }\מקור{[ע״ר א קצג]}\צהגדרה{. }

\מעוין{◊ }\משנה{ההילול}\הגדרה{ הבא מתוך מעמקי הנשמה לצור\hebrewmakaf כל\hebrewmakaf עולמים\mycircle{°} נובע הוא מההופעה\mycircle{°} הגדולה המקיפה באורה את נשמתנו, מההכרה של הטוב\mycircle{°}, של השלמות העליונה, שנשמת כל חי אליו עורגת, והצמאון\hebrewmakaf העליון\mycircle{°} והרוממות הבלתי משוערת בשיגוב קדושת\mycircle{°} אור\hebrewmakaf ד׳\mycircle{°} היא מפתחת בקרבנו את ההילול מעל לכל רגש וטעם }\מקור{[עפ״י ע״ר ב עט]}\צהגדרה{. }

\הגדרה{ע״ע תהילה. ע״ע שבח. ע״ע זמרה. ע׳ במדור פסוקים ובטויי חז״ל, התהללות, ״התהללו בשם קדשו״. }

\paragraphs

\ערך{הליכה }\הגדרה{- יציאה מהכח אל הפועל והנהגה מפורטת }\מקור{[מ״ר 82]}\צהגדרה{. }

\ערך{הליכה }\הגדרה{- }\משנה{במהלך שלמות האדם }\הגדרה{- להוסיף לקנות מעלות מדתיות ושכליות }\מקור{[ע״ר א רסב]}\צהגדרה{.}

\הגדרה{ע״ע עמידה.}

\paragraphs

\ערך{הליכה בדרכי ד׳ }\הגדרה{- ע״ע דעת את ד׳ וההליכה בדרכיו.}

\paragraphs

\ערך{הלכה }\הגדרה{- }\משנה{הלכות }\הגדרה{- פרטי הנהגות התורה\mycircle{°}. והמה באמת ״הליכות\hebrewmakaf עולם״\mycircle{°} }\מקור{[מ״ר 82]}\צהגדרה{. }

\הגדרה{עצות מרחוק}\myfootnote{ \textbf{עצות מרחוק} - ישעיה כה א. רמב״ם סוף ה׳ תמורה ״ורוב דיני התורה אינן אלא עצות מרחוק מגדול העצה לתקן הדעות וליישר כל המעשים״. ע״ע זוהר ח״ב פב:, ח״ג רב. \label{5}}\הגדרה{ שנתן האל הטוב ב״ה לבני אדם איך להגיע על ידי קיומם ולימודם לשלמותם, ענפים מחכמת אדון כל המעשים ב״ה, (ה)ערוכים ע״פ משפט חכמתו\hebrewmakaf העליונה\mycircle{°} }\מקור{[א״ה ב 306]}\צהגדרה{.}

\משנה{הלכה }\הגדרה{- }\משנה{ענינה }\הגדרה{- כיוון המעשה ע״פ פנימיות השכל ואחיזתו בה}\צהגדרה{ }\מקור{[א״ה ב (מהדורת תשס״ב) 214]}\צהגדרה{.}

\הגדרה{ע׳ במדור מצוות, הלכות, מנהגים וטעמיהן, בהגדרות מבוא, מצוות. ושם, הליכות של התורה. ע׳ במדור תורה, משנה. ע׳ במדור תורה, הלכה, ההלכה. ושם, הלכה, חכמת ההלכה.}

\paragraphs

\ערך{המון }\הגדרה{- }\משנה{ההמון}\הגדרה{ - (בני אדם בעלי נטיה) ארצית ומעשית, (ש)כל מעייניהם לחיי הזמן והעולם, (ו)הקפה רוחנית עולמית לא תיווצר ברוחם }\מקור{[עפ״י א׳ עג]}\צהגדרה{.}

\paragraphs

\ערך{המרה }\הגדרה{- }\משנה{ההמרה היסודית (המרת דת) }\הגדרה{- קציצת העיקר באמונה, והסרת הלב מאחרי ד׳\mycircle{°} }\מקור{[ע״א ד ט מד]}\צהגדרה{.}

\הגדרה{ר׳ מומרות.}

\paragraphs

\משנה{המשכה }\צהגדרה{- גילוי ושכלול }\מקור{[עפ״י א״ל רלב]}\הגדרה{.}

\paragraphs

\ערך{הנהגה עליונה }\הגדרה{- }\משנה{מדת ההנהגה העליונה }\הגדרה{- דרכי\hebrewmakaf אלהים\hebrewmakaf חיים\mycircle{°}: החיים יוצאים מהרכבה ממוזגת של כוחות שכ״א מונח בטבעו לפעול בשפע רב יותר מקיומו של עולם, וממילא הוא מביא בכחו הריסה וחורבן, וכן השני; וע״י התחברם והתמזגם נמצא[ים] החיים וכל המונם נצבים לפנינו וקיימים ונהנים מזיו\mycircle{°} החיים }\מקור{[עפ״י ע״א ב ט קלא]}\צהגדרה{.}

\הגדרה{העולם כולו ביחוד עולם החיים, הוא משתכלל, לא ע״י חיבורי כחות מצומצמים שכ״א ואחד פועל על גבולו, כ״א ע״י התחברות של כחות, שכ״א שואף להתרחב יותר מגבולו, וחברו ג״כ שואף כן, וכשהם פוגשים זה בזה דוחק כ״א את חברו ועוצרו, וע״י התגוששויות כאלה מתהוה חזיון החיים. זה נוהג בין בטבע החמרי, בין אפילו בטבע השכלי והמוסרי. וזהו אור זיו חסד החיים, המתמלאים תמיד ממקור אין\hebrewmakaf סוף בכל צדדיהם, עד שהם מסבבים חיים, בין בהתגברם והתפרצם, בין בהדחקם והעצרם }\מקור{[עפ״י אג׳ א פד-פה וההגדרה הקודמת]}\צהגדרה{. }

\paragraphs

\ערך{הנהגה צבורית }\הגדרה{- }\משנה{יסוד הגדולה שלה }\הגדרה{- (ההנהגה) המרכזת דעות שונות בודדות ומנוגדות זו מזו המפוזרות בכללות הצבור, ומביאה אותם לידי מערכה גדולה ומסודרת של אמת אחת גדולה וערוכה בכל }\מקור{[ע״א ד ו נג]}\צהגדרה{.}

\הגדרה{ע׳ בנספחות, מדור מחקרים, מנהיגים (סוגי מנהיגים). ע״ע שליטה. ע׳ במדור תורה, ״הלכות צבור״. }

\paragraphs

\משנה{הסתפחות}\myfootnote{ ע׳ ישעיה יד א. \label{6}}\הגדרה{ - }\צהגדרה{התחברות לא ממשית הדוקה אלא חיצונית ורופפת, שסופה שנהפכת ל׳ספחת׳ }\צמקור{[עפ״י ק״ת סג].}

\הגדרה{ר׳ לויה. }

\paragraphs

\ערך{הפנוזה }\הגדרה{- ע״ע היפנוטיזם.}

\paragraphs

\ערך{הצלחה אמיתית }\הגדרה{- הישועה\hebrewmakaf האמיתית\mycircle{°}, הדבקות\hebrewmakaf האלהית\mycircle{°}, המושכלת, החיה, המורגשת, מצד גודל אחיזתה בכל תוכן החיים, בכל מהותו של האדם, בקדושתו\mycircle{°}. ״כל עצמותי תאמרנה ד׳ מי כמוך״ }\מקור{[ע״ר א רטז]}\צהגדרה{. }

\paragraphs

\ערך{הצלחה פנימית }\הגדרה{- }\משנה{יסודה }\הגדרה{- שידע האדם שאשרו ימצא תמיד בעצמו בקרב נפשו ולבבו ולא ילך לבקשה אצל אחרים }\מקור{[ע״א ד ד ט]}\צהגדרה{. }

\paragraphs

\ערך{הצצה}\myfootnote{ זוהר ח״ב רנ: ״מציץ, כמאן דאשגח מאתר דקיק, דחמי ולא חמי כל מה דאצטריך״.\label{7}}\הגדרה{ - הבטה\mycircle{°} שע״י הדחק, היינו בהשתדלות לראות דבר שלא היה ראוי להרגיש בו אם לא בהתכוונות יתירה }\מקור{[ע״א א ב כו]}\צהגדרה{. }

\paragraphs

\ערך{הקשבה }\הגדרה{- השמעות אזנים, הקלטה את מה שבא מן החוץ }\מקור{[קובץ ח רי]}\צהגדרה{. }

\paragraphs

\ערך{הרהור חטא\mycircle{°}}\הגדרה{ - }\מעוין{◊}\הגדרה{ הרהור של משגה הדעת\mycircle{°} וכהות ההשכלה }\מקור{[עפ״י ע״ר א כח]}\צהגדרה{. }

\paragraphs

\ערך{הרמוניא }\הגדרה{- התאמה מאחדת }\מקור{[א״ש ח ז]}\צהגדרה{. }

\משנה{הרמוניה }\הגדרה{- הדרת\mycircle{°} האחדות\mycircle{°} ביקרתה }\מקור{[קובץ א תו]}\צהגדרה{. }

\paragraphs

\ערך{השפעה }\הגדרה{- }\משנה{(השפעת חיים) }\הגדרה{- ערך החיים }\מקור{[עפ״י קובץ ו סג (א״א 132)]}\צהגדרה{. }

\הגדרה{ע״ע שפע. }

\paragraphs

\ערך{השקפה שרשית }\הגדרה{- }\משנה{ההשקפה השרשית }\הגדרה{- ראיה בכל בריה את מסקנת החיים של שטח רחב מאד, של עומק ורום גדול ונעלה, המרוממת, לפי בהירותה, כל בריה למעלתה האידיאלית\mycircle{°} }\מקור{[עפ״י א״ק ב שנז]}\צהגדרה{. }

\הגדרה{ע׳ במדור מונחי קבלה ונסתר, העלאת מחשבות לשרשן. ושם, גונין דלא מתחזין ע״י גונין דמתחזין. ושם, בהגדרות מבוא, סוד, יסוד סוד ד׳ ליראיו. ע׳ במדור פסוקים ובטויי חז״ל, דימוי הצורה ליוצרה. }

\paragraphs

\ערך{התבודדות}\myfootnote{ ע׳ חובות הלבבות, שער הפרישות פרק ג. ע״ע א״ל עמ׳ קפט.\label{8}}\הגדרה{ - (}\צהגדרה{ענינה}\הגדרה{) - התעלות הרעיון, התעמקות המחשבה, השתחררות הדעה }\מקור{[א״ק ג רע]}\צהגדרה{.}

\צהגדרה{דבקות אלהית של המחשבה וההרגשה. יחוד המחשבה ורוממות הדעת של קדושת אמת }\צמקור{[א״ל קפט].}

\ערך{התבודדות עליונה }\הגדרה{- התקשרות פנימית למגמת החיים והעולם, למטרת ההויה ברזי\mycircle{°} רזיה }\מקור{[א ׳ מה]}\צהגדרה{.}

\paragraphs

\ערך{התבוננות }\הגדרה{- }\משנה{עצם חיי התבוננות }\הגדרה{- סידור המחשבות\mycircle{°} }\מקור{[א״ק ג רד]}\צהגדרה{.}

\הגדרה{ע״ע פיוט ושירה.}

\paragraphs

\ערך{התגלות }\הגדרה{- יציאה אל הפועל בעולם }\מקור{[עפ״י ע״ר א ריט]}\צהגדרה{.}

\paragraphs

\ערך{התגלות המהות }\הגדרה{- }\משנה{באה (באדם)}\הגדרה{ - }\מעוין{◊ }\הגדרה{כפי אותה המדה שהבחירה\hebrewmakaf החפשית\mycircle{°} מגלה את תכנה }\מקור{[עפ״י אג׳ ב מא]}\צהגדרה{.}

\paragraphs

\ערך{התגלמות }\הגדרה{-}\משנה{ (של אידיאל)}\הגדרה{ - התעטפותו (של האידיאל\mycircle{°}) בהגדרה מיוחדת }\מקור{[עפ״י א׳ קלג]}\צהגדרה{. }

\paragraphs

\ערך{התגשמות }\הגדרה{- }\משנה{ההתגשמות }\הגדרה{- החמריות\mycircle{°}, המעשיות }\מקור{[א״ק א עט]}\צהגדרה{.}

\paragraphs

\ערך{התהוות }\הגדרה{- התגלות\mycircle{°} הכח של היכולת\mycircle{°}, המלאה גבורת\mycircle{°} אין\hebrewmakaf קץ, בחוג המתואר חצוני לגבי עצמות הישות }\מקור{[ע״ר א ט]}\צהגדרה{.}

\paragraphs

\ערך{התהוות }\הגדרה{- }\משנה{להוות }\הגדרה{- לחולל, להרבות חיים וישות }\מקור{[ר״מ קיז]}\צהגדרה{.}

\paragraphs

\ערך{התנגדות }\הגדרה{- }\משנה{(עניינה של תנועת ההתנגדות הכללית) }\הגדרה{- הדאגה הרבה ליסוד המעשי שלא יתמוטט ע״י ההתגברות של הנטיה אל הרגש, ועל הפרטים, שהם מעמידי הכללים, שלא יתטשטשו ע״י הנטיה אל הכללים, ועל כח הדמיון, המתעורר גם ע״י התרגשות של הרגשות טובות, קדושות ואמיתיות, שלא יעבור את גבולו ויביא תוצאות רעות ומרות לכלל האומה לדורות הבאים}\צהגדרה{ }\מקור{[א״י כה]}\צהגדרה{.}

\הגדרה{ע״ע חסידות, (מגמת תנועת החסידות).}

\paragraphs

\משנה{התנוצצות }\צהגדרה{- תחילת ההופעה\mycircle{°}, הברקה של ראשית האור }\צמקור{[ש׳ רד״ך 216].}

\paragraphs

\ערך{התעלות }\הגדרה{- התקדשות\mycircle{°}, והתרוממות\mycircle{°} אל המרומים\mycircle{°} כולם }\מקור{[עפ״י א״ק א קפח (ע״ר ב ז)]}\צהגדרה{.}

\משנה{התעלות האדם }\הגדרה{- התפתחותו, בגילוי כל כשרונותיו הפנימיים }\מקור{[א״ק א צז]}\צהגדרה{.}

\הגדרה{להיות חפץ באמת\mycircle{°}, להיות חי ופועל לפי הדיעות היותר אמיתיות וההרגשות היותר קדושות לטוב ולחסד, כמעשה גדולי העולם אשר נגשו אל ד׳ במעשיהם הבהירים למלא את העולם חסד ואמת }\מקור{[ע״א ג ב קפד]}\צהגדרה{.}

\הגדרה{כל מה ששייכותו היא יותר גדולה לתוכן הפנימי של ההוויה והחיים }\מקור{[א״ק א יז]}\צהגדרה{.}

\משנה{עליה במעלה יותר שלמה ממה שהיה עומד עליה בתחילה }\הגדרה{- ששלטון השכל על חומרו הוא במעלה יותר שלמה ממה שהיתה אצלו בתחילה }\מקור{[ע״א ד ו כח]}\צהגדרהמודגשת{.}

\משנה{ההתעלות התדירה במעלות הקודש}\הגדרה{ - הרצון הקבוע להליכה מתמידה בדרך הקדושה והטהרה }\מקור{[ע״ר ב סה]}\צהגדרה{. }

\משנה{התעלות הנפש למרומי שמי ד׳ }\צהגדרה{- ההכרה העליונה בהדר\mycircle{°} גאון\hebrewmakaf ד׳\hebrewmakaf ועוזו\mycircle{°}, ההכרה בכלליות\mycircle{°} המציאות והאהבה\mycircle{°}, היינו, הקרבה\hebrewmakaf אל\hebrewmakaf ד׳\mycircle{°} }\צמקור{[צ״צ א כד (א״ל ריב)].}

\הגדרה{ע״ע עלוי. ע׳ במדור פסוקים ובטויי חז״ל, עליה.}

\paragraphs

\ערך{התעלות }\הגדרה{- }\משנה{ההתעלות הגמורה }\הגדרה{- התמזגותו של כל ערך\mycircle{°} ההויה בכל צורותיה, הקדש\mycircle{°} והחול\mycircle{°} שלה, לערך הקדש, ביסוד המכון של קדש\hebrewmakaf הקדשים\mycircle{°} }\מקור{[ע״ר א קעג]}\צהגדרה{.}

\משנה{ההתעלות העליונה }\הגדרה{- ע׳ במדור פסוקים ובטויי חז״ל, סוכת עורו של לויתן. ע״ע עלוי גמור.}

\paragraphs

\ערך{התקטנות }\הגדרה{- העזבות ממקור השפע\mycircle{°} }\מקור{[עפ״י ע״ר א קנב]}\צהגדרה{.}

\הגדרה{ע״ע קטנות.}

\paragraphs

\ערך{התרוממות }\הגדרה{- הגבהת\mycircle{°} הערך\mycircle{°} האמיתי של האדם }\מקור{[עפ״י ע״ר א קפו]}\צהגדרה{.}\mylettertitle{ו}

\paragraphs

\ערך{ודאות }\הגדרה{- התרפאות הנשמה, בקבלתה צורה תקיפה, עדינה, מלאה תנחומות, ועזוז\mycircle{°} קודש בגבורה עליונה }\מקור{[עפ״י א״ק א רי (ע״ט קכח)]}\צהגדרה{. }

\משנה{יסוד הודאות }\הגדרה{- עדן\mycircle{°} }\מקור{[עפ״י שם שם רה]}\צהגדרה{. }

\paragraphs

\ערך{ודאות באמיתותה של התורה }\הגדרה{- ע׳ בנספחות, מדור מחקרים. }

\paragraphs

\ערך{ודאות מוחלטה }\הגדרה{- }\משנה{הודאות המוחלטה בצורתה האידיאלית }\הגדרה{- מקור כל הודאיות, ומקור כל הספקות, מקור שאיבת הודאות, שמשם כל הספקות שואבים, להחיותם, להעלותם\mycircle{°}, לרעננם\mycircle{°} ולפארם בפאר\mycircle{°} חי\hebrewmakaf העולם\mycircle{°}. האורה\mycircle{°} המופלגה, מקור ששון מלא עולם, מקור הגבורה\mycircle{°}, מקור החסד\mycircle{°}, מקור התפארת\mycircle{°}, הנצח\mycircle{°} וההוד\mycircle{°}, מקור כל יסוד\mycircle{°} עולמים\mycircle{°}, מקור כל מלכות\mycircle{°} כל אדנות\mycircle{°} כל שלטון\mycircle{°}, כל כח\hebrewmakaf ד׳\mycircle{°}, כל גבורת אצילות\mycircle{°} בריאה\mycircle{°} יצירה\mycircle{°} ועשיה }\מקור{[עפ״י א״ק א רה\hebrewmakaf ו]}\צהגדרה{.}

\הגדרה{הרז\mycircle{°} הפנימי\mycircle{°} של הערכים\mycircle{°}, מיסוד החכמה\hebrewmakaf העליונה\mycircle{°}. הראשית\mycircle{°}, המחשבה\hebrewmakaf האצילית\mycircle{°} }\מקור{[עפ״י שם רז]}\צהגדרה{.}

\הגדרה{ע׳ במדור מונחי קבלה ונסתר, אבא.}

\paragraphs

\ערך{וו }\הגדרה{- קרס מחבר נושאים מפורדים }\מקור{[עפ״י ר״מ פה]}\צהגדרה{. }

\paragraphs

\ערך{וידוי}\הגדרה{ - }\משנה{מגמת הנטיה הטבעית להתודות ששם אדון כל הנשמות בלב כל בנ״א }\הגדרה{- זרית הלאה כל רצון זר לכל עון\mycircle{°} ולכל חטאת, בהארת אור האמת והיושר בנפש פנימה וקביעת הרצון לפי טבע האלהי שלו }\מקור{[עפ״י ע״א ג ב רה]}\צהגדרה{.}

\הגדרה{ע״ע תשובה, באדם.}\mylettertitle{ז}

\paragraphs

\ערך{זבול}\myfootnote{ מלכים א ח יג.\label{1}}\הגדרה{ - האמצעי בין (אוהל -) ארעי (לבית -) לקבע }\מקור{[פנק׳ ה קלב]}\צהגדרה{.}

\paragraphs

\ערך{זהירות הפנימית -}\הגדרה{ חסימת התאוה הבהמית, (ופקיחת) [וסקירת] העינים על כל העשוי בכל חוג הבשר והחומר }\מקור{[א״ק ד תכט (קובץ ה לז)]}\צהגדרה{.}

\paragraphs

\ערך{זוהר }\הגדרה{- הזריחה\mycircle{°} וההברקה\mycircle{°} של האור לכל שלל צבעיו\mycircle{°} הרבים המתנוצצים בבהירותם הנפלאה }\מקור{[עפ״י ע״א ד ט לח]}\צהגדרה{. }

\צהגדרה{חן\mycircle{°} העושר המיוחד של האור\mycircle{°} והברק }\צמקור{[צ״צ א עט]. }

\paragraphs

\ערך{זוהר}\myfootnote{ \textbf{זוהר, הגודל והרום הקדוש, המופע מאור חיי כל עולמים} - אור החמה לרא״א, יתרו, ריש פרשת ואתה תחזה: ״בורא כל עלמין יורה על הא״ס, וסתם עלמין בבינה שהיא נקראת עולם״. וביונת אלם, פרק ד ״דע כי אור עדיף מזוהר״ ״אור צח זהו זוהר מאור, פי׳ גדולה הבאה מהתנוצצות הכתר והחכמה יחד״. ע״ע בנספחות, מדור מחקרים, אור, זוהר, זיו.\label{2}}\הגדרה{ - }\משנה{הזוהר הנצחי }\הגדרה{- הגודל\mycircle{°} והרום הקדוש\mycircle{°}, המופע מאור\mycircle{°} חיי כל עולמים\mycircle{°} }\מקור{[עפ״י ע״ר א יח]}\צהגדרה{. }

\משנה{זהר עליון }\הגדרה{- שם\mycircle{°} קודש\hebrewmakaf ד׳\mycircle{°} }\מקור{[עפ״י שם ר]}\צהגדרה{. }

\paragraphs

\ערך{זוהר}\הגדרה{ - }\משנה{הזוהר הפנימי}\הגדרה{ - החיים שהם מגמת הכל, שכל היש חי בו בנשמתו פנימה. המאור האלהי של האותיות המאושרות של שם המפורש, הפועלות והמשפיעות בתכונתן האלהית }\מקור{[קובץ ז קמו]}\צהגדרה{.}

\הגדרה{ע׳ במדור מונחי קבלה ונסתר, טל. ע׳ במדור שמות כינויים ותארים אלהיים, יה״ו.}

\paragraphs

\ערך{זוהר }\הגדרה{- }\משנה{הזוהר הנשמתי ביסודו }\הגדרה{- השכל\hebrewmakaf העליון\mycircle{°} }\מקור{[א״ק א יא]}\צהגדרה{. }

\משנה{״זוהר הרקיע״}\myfootnote{ ב״ב ח:.\label{3}}\הגדרה{ - מרחב ההתפשטות הנשמתית\mycircle{°} העליונה\mycircle{°} של האדם, ההוד\mycircle{°} העליון הנמצא בנשמתו בעצמיותה הגדולה לאין חקר, האורה\hebrewmakaf העליונה\mycircle{°} של הנשמה, ההיקף העליון של הנשמה העליונה בהתרחבותה המוחלטת }\מקור{[ע״ט קכה (א״ק ד תעה)]}\צהגדרה{. }

\הגדרה{ע׳ במדור פסוקים ובטויי חז״ל, כוכב. ושם, והמשכילים יזהירו כזהר הרקיע.}

\paragraphs

\ערך{זוהר }\הגדרה{- ההוד\mycircle{°} הנעלם, המתנוצץ ומסתתר המבהיק ונגנז שהוא העמילן ויסוד המזין של ההשכלה בכלל <שהיא מתגלה בכל המדריגות בקודש\mycircle{°} ובחול\mycircle{°}, בכל יצירה רוחנית\mycircle{°} לפי ערכה, בהוד מפואר ובזיו\mycircle{°} כמוס> המתנוסס בתוכיות הרוח הטהור שבלב זכי הרוח, השואבים מאצילות החיים של רוח\hebrewmakaf הקודש\mycircle{°} הצפון במעמקי הנשמה\mycircle{°} }\מקור{[ר״מ קעד]}\צהגדרה{. }

\paragraphs

\ערך{זוהר }\הגדרה{- }\משנה{ספר הזוהר }\הגדרה{- ע׳ במדור תורה.}

\paragraphs

\ערך{זיו אלהי }\הגדרה{- }\משנה{מעוז הזיו האלהי שממעל לכל גבולי\hebrewmakaf עולמים }\הגדרה{- הגודל\hebrewmakaf העליון\mycircle{°} }\מקור{[ע״ר א כ, ועפ״י שם ב עד]}\צהגדרה{.}

\ערך{זיו שכינת אל }\הגדרה{- אור\mycircle{°} עולמי\hebrewmakaf עולמים\mycircle{°} }\מקור{[א״ק ב תקל]}\צהגדרה{.}

\הגדרה{ע״ע הארת זיו שכינת אל. }

\ערך{הזיו העליון }\הגדרה{- ההארה\mycircle{°} המלאה\mycircle{°} }\מקור{[עפ״י א״ק ב תה\hebrewmakaf ו]}\צהגדרה{.}

\paragraphs

\ערך{זיו}\myfootnote{ \textbf{זיו} - בדעת תבונות, עמ׳ עה (מהד׳ ר״ח פרידלנדר) כותב הרמח״ל כהקבלה לסדרי הבריאה של הקב״ה ודרכיה בבריאה עצמה: ״יש דבר אחד נמצא מהתחברות הנשמה והגוף - הוא זיו הפנים... ואין הזיו הזה נמצא לא לנשמה בפני עצמה, ולא לגוף בפני עצמו, אבל הוא הדבר הנולד מחיבור הנשמה והגוף ביחד״. ובדרך מצותיך נא. ״זיו הנפש הוא ענין חיוני מתפשט בגוף שהוא מקבל חיות זה וחי ממנו. וכך מהותו ועצמותו המחוייב המציאות, שהוא לבדו הוא, ואין זולתו, והוא חיי החיים, כשימשיך ויאיר ממנו ויגלה הארה בבחי׳ זיו ענינה הוא המשכת חיים בבחי׳ א״ס וז״ש הודו על ארץ ושמים כו׳ ופי׳ הוד וזיו״. ע׳ בנספחות, מדור מחקרים, אור, זוהר, הוד, זיו. ושם אור, זיו, ברק.\label{4}}\ערך{ אור אלהים }\הגדרה{- }\משנה{זיו החיים }\הגדרה{- שכינת\hebrewmakaf אל\mycircle{°}. אור\hebrewmakaf אלהים\mycircle{°}, הממלא את העולמים\mycircle{°} כולם, המחיה ומרוה אותם מדשן\mycircle{°} נועם\hebrewmakaf עליון\mycircle{°} של מקור\hebrewmakaf החיים\mycircle{°}, ונותן חיל\mycircle{°} בנשמות\mycircle{°}, במלאכים\mycircle{°}, ובכל יצור, לחוש את פנימיות\mycircle{°} תחושת החיים. יד\mycircle{°} אל\mycircle{°} עליון הפתוחה ומשביעה רצון\mycircle{°} לכל. מקורם של היצורים העליונים שביצירה, ברואי מעלה ומטה }\מקור{[עפ״י א״ק ב שכט\hebrewmakaf ל]}\צהגדרה{.}

\משנה{זיו השכינה }\צהגדרה{- }\הגדרה{הארת השכינה המודדת עולמי עד במדידת ההגבלה, הזמנית והמקומית, שרשיהם ושרשי שרשיהם, כדי להבליט את הודם ונצחם, את עליוניותם וחופש עצמתם, מרחבי טיסתם ואין סופיות הערצת פאר קדושתם }\מקור{[ע״א ד ט צז]}\צהגדרה{.}

\paragraphs

\ערך{זיו אלהי }\הגדרה{- שכינתא\mycircle{°}}\צהגדרה{ }\מקור{[א״ק ג כ]}\צהגדרה{.}

\משנה{זיו השכינה }\צהגדרה{- השפע החיוני הכללי }\צמקור{[עפ״י א״ל רמז]. }

\מעוין{◊ }\משנה{הזיו}\הגדרה{ מתנוצץ מברק היפעה\mycircle{°} המתגלה על כל המון הברואים, היצורים, והנעשים, בכל הקצבת מהותיותם, השוכן עליהם ושומר את צביונם }\מקור{[ע״ר א יג]}\צהגדרה{.}

\הגדרה{ע׳ במדור פסוקים ובטויי חז״ל, שכינה, ״זיו השכינה״. }

\ערך{זיו של מעלה }\הגדרה{- אור עליון\mycircle{°} שמעל כל רעיון\mycircle{°} ומחשבה\mycircle{°} }\מקור{[א׳ קכ]}\צהגדרה{.}

\ערך{זיו החיים }\הגדרה{- טוהר\mycircle{°} הרוח\mycircle{°}, ברק השכל, זיו\mycircle{°} המוסר\mycircle{°}, נועם האהבה\mycircle{°} והאשר\mycircle{°} }\מקור{[א״א 73]}\צהגדרה{.}

\משנה{זיו החיים (ה)מתמשך על הכל }\הגדרה{- הוד\mycircle{°} השירה\mycircle{°} בסוד חייה הולך ומתפשט על כל החוגים, הקרובים והרחוקים}\צהגדרה{ }\מקור{[א״ק א פז]}\צהגדרה{.}

\paragraphs

\ערך{זיכוך }\הגדרה{- }\משנה{זכוך האדם}\הגדרה{ - רוממות דעתו ושכלו, כינון תכונותיו ומעשיו לצד היותר נעלה וישר\mycircle{°} }\מקור{[עפ״י קובץ א קטו]}\צהגדרה{.}

\paragraphs

\ערך{זין }\הגדרה{- כלי המלחמה להגן נגד כל מחריב }\מקור{[ר״מ פט]}\צהגדרה{.}

\הגדרה{הכלים שהמלחמה נאסרת על ידם נגד כל מפריע, נגד כל אויב ומתנקם, הנצחון המלחמתי, הנשק, הכח לכלות כל מפריע }\מקור{[עפ״י שם ט\hebrewmakaf י]}\צהגדרה{.}

\paragraphs

\ערך{זית }\הגדרה{- }\מעוין{◊}\הגדרה{ מורה אורה }\מקור{[ע״א ב בכורים כט]}\צהגדרה{.}

\הגדרה{ע״ע שמן זית.}

\paragraphs

\ערך{זכויות }\הגדרה{- ע״ע זכות.}

\paragraphs

\ערך{זכות }\הגדרה{- }\משנה{ענינה }\הגדרה{- רצונו של איש ישראל כדי להתקרב לאביו\hebrewmakaf שבשמים\mycircle{°} }\מקור{[מ״ש קעז (מא״ה א קיט)]}\צהגדרה{.}

\משנה{זכויות }\הגדרה{- הפעולות הטובות\mycircle{°} מכל צד, שתהינה טובות כשהאדם הוא (ה)הולך ופועל ע״י המכשירים שמוצא לפניו ממעשה שמים, הרי הוא מתדבק\mycircle{°} בדרכי\hebrewmakaf ד׳\hebrewmakaf יתברך\mycircle{°} פועל כל }\מקור{[ע״א ג ב קצד]}\צהגדרה{. }

\paragraphs

\משנה{זכירה }\צהגדרה{- דבקות ההכרה }\צמקור{[ל״י א י]. }

\הגדרה{ר׳ זכרון.}

\paragraphs

\משנה{זכירה }\צהגדרה{- }\צמשנה{זכירה את ד׳ }\צהגדרה{- דבקות ההכרה האלהית התמידית. הזכירה הפנימית, הנשמתית הקבועה, החיה והקימת בישראל, של תוכן החיים ויסודו הפנימי הכולל, ממקור החיים וההויה ומעין ישעם וחפצם }\צמקור{[עפ״י ל״י א י]. }

\הגדרה{ר׳ במדור פסוקים ובטויי חז״ל, שכחי אלהים. ע״ע זכרון (לפני ד׳).}

\paragraphs

\ערך{זכירה }\הגדרה{- }\משנה{זכירת הצור\mycircle{°} }\הגדרה{- העלאת עצמיותו (של האדם) אל העולם המגמתי העליון}\צהגדרה{ }\מקור{[א״ק ב תקנה]}\צהגדרה{.}

\paragraphs

\ערך{זָכָר }\הגדרה{- היסוד העקרי שבתולדה והמשכת החיים, הסגולה\mycircle{°} המפעלית }\מקור{[עפ״י ע״ר א מ]}\צהגדרה{.}

\הגדרה{כח המפעלי שבחיים }\מקור{[ע״ר א מב]}\צהגדרה{.}

\מעוין{◊}\הגדרה{ המוכן לעבודה שכלית היותר רוממה וטהורה, השולטת ג״כ על עז הרגש }\מקור{[ע״א ג ב נ]}\צהגדרה{.}

\משנה{זכרי }\הגדרה{- תוכן פועל מפעלים חיים חזקים וגמורים}\צהגדרה{ }\מקור{[ר״מ קלא]}\צהגדרה{.}

\הגדרה{ע״ע נקבה, נקבי.}

\paragraphs

\ערך{זֵכֶר }\הגדרה{- }\משנה{(השגת זכר ד׳ לעומת שם\hebrewmakaf ד׳\mycircle{°}) }\הגדרה{- התרשמות הנרשמת על הנשמות\mycircle{°}, מצד ההשגות הבאות מסבת ההסתכלות במעשים הנפלאים של דרכי ההשגחה\mycircle{°} העליונה על כל תקופה ותקופה, בתור חטיבה פרטית. הרישום של פליאת גדולת\mycircle{°} השם יתברך וצדקו\mycircle{°} העליון, היוצא מתוך כל הדורות ומסיבותיהם }\מקור{[עפ״י ע״ר ב פג]}\צהגדרה{.}

\paragraphs

\משנה{זכרון }\צהגדרה{- קשר שייכות ודבקות }\צמקור{[שי׳ מועדים א 25].}

\הגדרה{ר׳ זכירה.}

\paragraphs

\ערך{זכרון }\הגדרה{- }\משנה{״זכרו לעולם בריתו״, הזכרון העולמי }\הגדרה{- תכן הברית\mycircle{°} }\מקור{[עפ״י ע״ר א רב]}\צהגדרה{.}

\paragraphs

\ערך{זכרון}\myfootnote{ תולעת יעקב, סוד ראש השנה סוד התפילה לג: ״מה שאנו אומרים זכרנו, הוא ענין בסוד הנהר המרוה צמאונים בסוד כל ענין, כי הוא סוד (ה׳,) שפע המערכה, ׳כי שם צוה ד׳ את הברכה׳, בסוד נעלם, ׳חיים עד העולם׳. כי בהתעלות הכבוד ממדרגה למדרגה ומסבה לסבה אז החיים יוצאים ממקור מוצאם ונמשכים אל הסבות כולם״. ע״ע עטרת ראש להרד״ב, שער ראש השנה, סי׳ טו. מחשבות חרוץ לר״צ, סה. ״כל זכירה הוא העלאת הדבר לשרשו והתחלתו כמו יום הזכרון דהרת עולם, כי הזכרון הוא שמזכיר ומצייר הדבר עתה ממש כמו שהי׳ אז״. וע׳ אלפי מנשה ח״א, סוף פרק צז.\label{5}}\הגדרה{ - }\משנה{(לפני ד׳) }\הגדרה{- כח הסגולה\mycircle{°} האורית\mycircle{°} כמו שהיא בעינה לפני ירידתה\mycircle{°} והתמעטותה במשך הדורות בירידתם, שאותה הבהירות, שעמדה הסגולה הקדושה\mycircle{°} המיוחדת, שממנה באה הנקודה הנשמתית\mycircle{°} הקדושה\mycircle{°} לכל פרט מפרטינו, נזכרת ומופעת מכח שורש מציאותה }\מקור{[עפ״י ע״ר א פג]}\צהגדרה{.}

\הגדרה{ע״ע פקידה. ר׳ זכירה, זכירה את ד׳. ע׳ בנספחות, מדור מחקרים, פקידה וזכרון, ההבדל בין פקידה לשאר דרכי זכרון. }

\paragraphs

\ערך{זמן }\הגדרה{- }\משנה{(לעומת נצח\mycircle{°}) }\הגדרה{- ההוה התדירי }\מקור{[ע״ר א סד]}\צהגדרה{. }

\הגדרה{ע׳ במדור מועדים וחגים, קידוש הזמנים. }

\משנה{זמן (וערכיו) }\הגדרה{- ההתפתחות ההדרגית (ומדותיה) }\מקור{[ע׳׳א ד ט נא]}\צהגדרה{.}

\משנה{זמן }\צהגדרה{- סדר יחסו של האדם אל העולם }\צמקור{[עפ״י א״ל רמט]. }

\מעוין{◊}\צהגדרה{ יחסי האדם והעולם מתגלים הם במציאותו של ענין ה}\צהגדרהמודגשת{זמן,}\צהגדרה{ ומסתדרים בסדריו }\צמקור{[ל״י א קלב]. }

\צהגדרה{האדם בעולם. סדר החיים של האדם בעולם הזה. ענינה של הופעת הנשמה בגוף בסדר ההסטוריה. <}\צהגדרהמודגשת{הזמן}\צהגדרה{ אינו דבר שיש לו מציאות כשלעצמו, אבל בו מצוייה עובדת מציאותנו> }\צמקור{[עפ״י שי׳ ב 132]. }

\צמשנה{ענינו של הזמן בכלל, במהותו הטבעית}\צהגדרה{ - רציפות התנועה של השינויים החליפות והתמורות, אשר יוצר בראשית, המחדש מעשהו בטובו בכל יום תמיד, משנה עתים ומחליף את הזמנים, במערכת רצונו }\צמקור{[א״ל רמט].}

\paragraphs

\ערך{זמר}\myfootnote{ תורה אור לרש״ז, בשלח דף סב: ד״ה אך ״ענין הנגון הוא שאין בו רק התפעלות הנפש מפני גילוי התנועות ולא מפני שיש בתנועות ההם מצד עצמם שום שכל והתחדשות אלא מפני גילוי התנועה מתעורר גילוי הלב... להיות בחי׳ התפעלות הנפש מצד עצמה ולא מצד השגה להוליד מבינתו כו׳... ולכן קוראין אותן פסוקי דזמרה שהם בחי׳ נגון... כשאומר הפסוקים ואינו נוגע לנקודת לבבו אין זה בחי׳ זמרה ונגון כו׳״.\label{6}}\הגדרה{ - (בטוי) עומק ההרגשה בנועם השתפכות הנפש }\מקור{[עפ״י ע״א ד ה עח]}\צהגדרה{. }

\הגדרה{התפרצותו הבטויית של הרגש הלבבי, המתעמק בעומק הנפש, הצפון בחגוי החיים, שאינה יכולה להתלבש בבטויים, כי אם בתנועות קוליות מסודרות }\מקור{[עפ״י ע״ר א קפו]}\צהגדרה{. }

\הגדרה{הבטוי של הרגש הנפשי\mycircle{°}, ההולך ומתעמק במעמקי החיים הרוחניים\mycircle{°}, (שיש והוא בא) בתור תוצאה רבת הכח מתוך ההסתכלות הבהירה, המצמחת את השירה\mycircle{°} בתחילה }\מקור{[עפ״י שם ר]}\צהגדרה{. }

\הגדרה{השגת\mycircle{°} הנפש דרכי חייה האמיתיים בעת המשכת הזכרון להמשיך אור החכמה, אחרי שמכסה כבר הענין באופן שאין הרקיע\hebrewmakaf בטהרתו\mycircle{°} }\מקור{[עפ״י מא״ה ב רסט\hebrewmakaf רע]}\צהגדרה{. }

\הגדרה{הרוח\mycircle{°} המתלבש בלב\mycircle{°} ע״פ השכלת השכל בציור אמיתי של }\צהגדרה{ז}\הגדרה{כרון מעניני השכל, וה}\צהגדרה{מ}\הגדרה{חשבה בעניני השלמות וה}\צהגדרה{ר}\הגדרה{צון בהם באמת לפעול, הרוח המשמח את הנפש\mycircle{°} ומעירה לבקש דרכים לצאת ממאסרה כדי שתחת הזכרון יזרח\mycircle{°} עליה אור\hebrewmakaf השכל\mycircle{°} }\מקור{[שם רע]}\צהגדרה{. }

\הגדרה{ע׳ בנספחות, מדור מחקרים, שיר וזמר, ההפרש ביניהם. ע״ע שיח. ע״ע רנה.}

\תמקור{השירה בשעה שהיא מתפרצת מתוכיותה, מעצמיותה, משטף יצירתה וזרמה הנשמתי כשהיא באה לידי ביטוי [מא״ה ב רסא]. }

\ערך{זמרה }\הגדרה{- הכח המתמלא בלב\mycircle{°}, בהרצון וההסכם השכלי, המורכב מ}\משנה{ז}\הגדרה{כרון }\משנה{מ}\הגדרה{חשבה }\משנה{ר}\הגדרה{צון, אם כי עוד לא נשלמה שמחתו בפעל מ״מ בצפיתו יצפה לה }\מקור{[פנק׳ ג כב (מא״ה ב רע)]}\צהגדרה{. }

\הגדרה{ע״ע שיר.}

\הגדרה{ההבעות הנותנות דחיפה לביטויינו לקרא בשם\hebrewmakaf קדשו\mycircle{°}, בהם אנחנו מרגישים את הנועם\hebrewmakaf העליון\mycircle{°}. ההבעות הקשורות לענפים החודרים לתוך התוכיות הנפשיות שלנו, המשתרגים מתוך האור העליון, שלהופעה\mycircle{°} הגדולה המקיפה באורה\mycircle{°} את נשמתנו - ההכרה של הטוב\mycircle{°}, של השלמות העליונה }\מקור{[עפ״י ע״ר ב עט]}\צהגדרה{.}

\ערך{כח הזמר }\הגדרה{- הרגש, העולה מתוך רעותא\hebrewmakaf דליבא\mycircle{°}, הבוקע את האויר הנעלם, ומתרומם בהרחבת תועפות גדלו ממעל לחוג הצר של הבטוי והרעיון המחולל אותו }\מקור{[ע״ר א קצח]}\צהגדרה{.}

\הגדרה{ע״ע הלול. ע״ע רנה. }

\paragraphs

\ערך{זקוק }\הגדרה{- בהיר\mycircle{°} ומלא יושר\mycircle{°} }\מקור{[א״ק ג נט]}\צהגדרה{.}

\paragraphs

\ערך{זקיפה }\הגדרה{- התמתחות הכוחות והארכת החלקים והכוחות החיוניים כולם, להגלות בכל מלא מדתם }\מקור{[ע״ר א עג]}\צהגדרה{. }\mylettertitle{ח}

\paragraphs

\ערך{חביון }\הגדרה{- פנימיות, גנז }\מקור{[רצי״ה א״ש ד הערה 9]}\צהגדרה{. }

\paragraphs

\ערך{חביון עז }\הגדרה{- צורה\mycircle{°} פנימית\mycircle{°} }\מקור{[מ״ר 257]}\צהגדרה{. }

\ערך{״חביון עז״ }\הגדרה{- ע׳ במדור פסוקים ובטויי חז״ל.}

\paragraphs

\ערך{חגויה }\הגדרה{- בינות חלקיה }\מקור{[רצי״ה א״ש ח הערה 16]}\צהגדרה{.}

\paragraphs

\ערך{חגורה }\הגדרה{- }\משנה{יסוד הויתה }\הגדרה{- התאמצות כל שרירי הגוף לחטיבה אחת. קישור כונניות\mycircle{°} הגויה בכללה }\מקור{[עפ״י ע״א ד ו לג]}\צהגדרה{. }

\הגדרה{ע״ע אבנט.}

\paragraphs

\ערך{חדוש }\הגדרה{- }\משנה{החידוש של המחשבה}\הגדרה{ - פנינים רוחניים שבאים מנטיפות גן\hebrewmakaf העדן\mycircle{°} אשר לנשמה}\צהגדרה{ }\מקור{[פנק׳ ג שכט]}\צהגדרה{.}

\ערך{חדוש }\הגדרה{- }\משנה{(חידושי תורה)}\הגדרה{ - פיתוח שפעת החיים של האורה החיה האלהית הכללית של תורה\hebrewmakaf שבע״פ\mycircle{°} }\מקור{[עפ״י א״ת א א]}\צהגדרה{.}

\הגדרה{לימוד דבר שלא היה אצל האדם מצד מדרגתו אפילו בכח }\מקור{[הגדרת מו״ר הר״ש אבינר, עפ״י א״ת ו ד]}\צהגדרה{. }

\מעוין{◊ }\צמשנה{מוצא החדוש האמיתי }\צהגדרה{-  נלקח מתמצית כל המהות של היש הקדום לו. כל הבעה מחודשת באמת אינה יוצאת מהסעיף האחד הקרוב, אותה החוליה שההגיון תופסה, אלא מכל המהותיות, המשכלת והחיה, הפועלת וההוגה של המחדש, ״ערוכה בכל ושמורה״}\myfootnote{ \textbf{ערוכה בכל ושמורה} - שמואל ב כג ה.\label{1}}\צהגדרה{, וכל חדוש שהוא מזדלף רק מאיזה חלק ומשאיר חלקים בלא תנועה רוחנית, איננו חדוש באמת כי אם סדור מכני במערכת ההשכלה. }

\צהגדרה{שאיפת החדוש הוא הזלת התמצית מתנובת הכל\mycircle{°} }\צמקור{[עפ״י אג׳ ג ד].}

\paragraphs

\ערך{חול }\הגדרה{- }\משנה{(לעומת קודש) }\הגדרה{- החומר\mycircle{°} של הקודש\mycircle{°} }\מקור{[א״ק א קמה (מ״ר 400)]}\צהגדרה{. }

\הגדרה{הנמצא בדרך סיבוב והתעסקות, המעשה הבלתי מכוון רק מסובב ומתפשט }\מקור{[עפ״י אג׳ ג מא, מב]}\צהגדרה{. }

\הגדרה{כל מה שעודנו רחוק מתכליתו ולא נקשר בקשר התקון שהיתה אליו הכונה התכליתית }\מקור{[מ״ש שמט (ה׳ קעה)]}\צהגדרה{. }

\מעוין{◊}\הגדרה{ דברים בהנהגת העולם שמועילים לחזק את עולם החיים וחוזק הברואים והטבעים, אלה הם בכלל דברים של חול שעוד אינם עיקר בבריאה }\מקור{[ע״א א ה סב]}\צהגדרה{. }

\הגדרה{דרכי החיים המעשיים. המכשיר את החיים לתעודתם, הנותן חומר כח והכנה להם למען יתרוממו לתכליתם האמיתית שהוא הקודש }\מקור{[עפ״י שם ג ב סט]}\צהגדרה{. }

\משנה{האור הבא מימות החול }\הגדרה{- שפעת החיים, יושר השכל הפשוט הבא מהופעת החושים, החברותיות וכל תוכניה }\מקור{[שם ד ט נד (ג״ר 51)]}\צהגדרה{. }

\ערך{חול }\הגדרה{- }\משנה{השגת החול }\הגדרה{- השגה מחוברת ברשומה גופנית }\מקור{[קבצ׳ א נט]}\צהגדרה{. }

\הגדרה{ע׳ במדור מועדים וחגים, שבת, השגת השבת. }

\משנה{חיים חלוניים }\הגדרה{- החיים המעשיים, הטבעיים, צרכי האדם וחייו. החיים העסוקים בצרכי החמר הפעוטים, אחר שנתפשטו מתוך מרכז הקודש\mycircle{°}, שירדו מרום מעלת הנצח, לעסוק בצרכי הזמן של יום יום, של חיי\hebrewmakaf שעה }\מקור{[עפ״י ע״ר א קכה, קכד]}\צהגדרה{. }

\משנה{כח חלוני }\הגדרה{- כח אשר רק התוצאות המעשיות ביסודי חיי החברה הן מגמתו, שגם להם (כלקודש) יד ושם לשאיפה של עדינות הכוחות וטהרת\mycircle{°} המושגים }\מקור{[שם קכב]}\צהגדרה{. }

\הגדרה{ע״ע קודש, לעומת חול. ע״ע כנסת ישראל הטבעית. }

\paragraphs

\ערך{חול}\הגדרה{ - }\משנה{נטיית חולין }\הגדרה{- גופניות, בהמיות }\מקור{[א״ש יד ה]}\צהגדרה{.}

\paragraphs

\ערך{חולוני, חולני }\הגדרה{- ע׳ בנספחות, מדור מחקרים, חילוני, חולוני, חולני. }

\paragraphs

\ערך{חוֹלי }\הגדרה{- בלבול סדר החיים בכללם במערכת הגוף }\מקור{[ע״א ג א לא]}\צהגדרה{. }

\מעוין{◊}\הגדרה{ הפרעת הסדר מיציאת פרט מהסכמתו אל הכלל בגוף עושה }\משנה{חולי }\מקור{[שם א ה נח]}\צהגדרה{. }

\הגדרה{ע״ע מיחוש. ע״ע ארוכה. ע״ע רפואה.}

\paragraphs

\ערך{חולשה }\הגדרה{- רשלנות החיים ומיעוט צביונם }\מקור{[ע״א ד ט כב]}\צהגדרה{. }

\paragraphs

\ערך{חומר }\הגדרה{- }\משנה{(תכונת ההחמרה, להחמיר) }\הגדרה{- הדיוק, העול, הזהירות וההרחקה מכל כעור ותעוב, גם נגד הנטיה, ונגד הזרמת האורה, כפי מה שהיא מצטיירת\mycircle{°} לפעמים בנפש. ולפעמים יצא גם בזעף לשם הבאה למגמתה העליונה של השאיפה הקדושה\mycircle{°} }\מקור{[עפ״י ע״ר א קפג]}\צהגדרה{. }

\paragraphs

\ערך{חומר }\הגדרה{-}\משנה{ (לעומת כח\mycircle{°}, ענינו) }\הגדרה{- שלא להיות פועל כי אם מקבל פעולות של הכוחות }\מקור{[מ״ש קיז (מא״ה ב יא)]}\צהגדרה{. }

\משנה{חומר (לעומת צורה\mycircle{°}) }\הגדרה{- כח מקבל פעולה }\מקור{[ע״א ב ט רמ]}\צהגדרה{. }

\paragraphs

\ערך{חומר }\הגדרה{- }\משנה{(לעומת רוח) }\הגדרה{- ע״ע חמר. }

\paragraphs

\ערך{חוסן }\הגדרה{- עז\mycircle{°} ואיתנות\mycircle{°} }\מקור{[ע״א ג ב קנה]}\צהגדרה{. }

\הגדרה{ע׳ במדור פסוקים ובטויי חז״ל, חסין. }

\paragraphs

\ערך{חופש }\הגדרה{- ע״ע חפש. }

\paragraphs

\ערך{חוץ }\הגדרה{- ע״ע סקירה חצונית. ע״ע חצוני. }

\paragraphs

\ערך{חושב }\הגדרה{- התופר ציורים\mycircle{°} זרים על יריעה}\myfootnote{ ע׳ רש״י שמות כח, ו.\label{2}}\מקור{ [פנק׳ ו קעב]}\צהגדרה{.}

\paragraphs

\ערך{חושב }\הגדרה{- מצייר\mycircle{°} ציורי המוח על לוח הלב בשכל }\מקור{[פנק׳ ו קעב]}\צהגדרה{. }

\paragraphs

\ערך{חושב }\הגדרה{- }\משנה{החושב }\הגדרה{- העומד על עמדתו העצמית}\צהגדרה{ }\מקור{[א״ק ג קכ]}\צהגדרה{.}

\הגדרה{ע׳ במדור הכרה והשכלה והפכן, המחשבה היסודית.}

\paragraphs

\משנה{חושך }\הגדרה{- הניגוד המסובך והסתירה המבולבלת }\מקור{[קבצ׳ ג טז]}\צהגדרה{.}

\paragraphs

\משנה{חושך - }\הגדרה{התשוקות הגסות\mycircle{°} וכל המדות הרעות, הגאוה\mycircle{°} והשנאה, הרפיון ושויון הרוח לכל נאצל וכל מרום וקדוש }\מקור{[פנק׳ ה סג]}\צהגדרה{.}

\הגדרה{התשוקות הגסות והמדות הרעות, התגעשות חית האדם, ושויון\hebrewmakaf הרוח ורפיון\hebrewmakaf החפץ לכל נשא ונאצל, לכל מרום וקדוש, לכל מה שלמעלה מהחושים החמריים והדמיון הגס הנשען רק עליהם }\מקור{[עפ״י א׳ קו]}\צהגדרה{.}

\paragraphs

\ערך{חושך }\הגדרה{- }\משנה{החשכה פנימית }\הגדרה{- ע׳ במדור נפשיות, החשכה פנימית.}

\paragraphs

\משנה{חזון }\הגדרה{- }\צהגדרה{ר׳ ראיה לעומת חזון.}

\paragraphs

\ערך{חטא }\הגדרה{- תוכן של חסרון }\מקור{[ע״ר א עז]}\צהגדרה{. }

\הגדרה{הפסד והורדה\mycircle{°} }\מקור{[ע״א ג ב נא]}\צהגדרה{. }

\ערך{חטא }\הגדרה{- הירידה המוסרית\mycircle{°}, מצד שעדיין לא בא השכלול האנושי לאותה המדרגה העליונה שאין החטא\mycircle{°} נמצא עמה }\מקור{[ע״א ד ה לו]}\צהגדרה{. }

\ערך{חטא }\הגדרה{- }\משנה{פגם החטא}\הגדרה{ - פגם התכונה הנפשית חסרון מוסרי\mycircle{°} בטבע הנפש\mycircle{°} ותכונת המדות הנוטות לאיזה כיעור וקלקול }\מקור{[ע״ר א עז]}\צהגדרה{. }

\הגדרה{הפחיתות שבנפש }\מקור{[מ״א א ו (ע״ר ב קכב)]}\צהגדרה{. }

\הגדרה{מעמד הנפש שלא נשלמה בכל בכללותה }\מקור{[ע״א ד ה לו]}\צהגדרה{. }

\ערך{חטא }\הגדרה{- }\משנה{המובן הפשוט של החטא }\הגדרה{- השגגה\mycircle{°} }\מקור{[ע״ר א קכו]}\צהגדרה{. }

\הגדרה{ריקות הפעולה מתכנה הראוי }\מקור{[ע״ר ב קסב]}\צהגדרה{. }

\הגדרה{פעולה בלתי הוגנת המונחת ברוב בתועלת הבאה מצד סכלות }\מקור{[עפ״י ע״א ב 194]}\צהגדרה{.}

\מעוין{◊}\הגדרה{ קלקולו של כל חטא בא מתוך מעוט האור\mycircle{°} של הנשמה\mycircle{°} שלא האירה יפה בתוכיות החיים }\מקור{[ע״ר א קכו]}\צהגדרה{.}

\הגדרה{ע״ע עברה. ע״ע עוון. ע״ע פשע. ע״ע הרהור חטא. ע״ע מחילה. ע׳ במדור מדרגות והערכות אישיותיות, פושע. ע׳ בנספחות, מדור מחקרים, מחילה סליחה וכפרה. }

\paragraphs

\משנה{חטא}\הגדרה{ - כל דבר שהוא אמצעי להפריד ממנו השכל }\מקור{[פנק׳ ג קעד]}\צהגדרה{.}

\הגדרה{ע׳ בהגדרות מבוא למדור מצוות, הלכות, מנהגים וטעמיהן, מצוה.}

\paragraphs

\ערך{חטא }\הגדרה{- אי התאמה להרמוניה נצחית }\מקור{[עפ״י א״ק ג ז]}\צהגדרה{.}

\paragraphs

\ערך{חטא }\הגדרה{- }\משנה{(לעומת מעילה\mycircle{°}) }\הגדרה{- החסרון ו(ה)השפלה שנעשה בנפשו, שהיא עלולה כעת לחטאים אחרים הבאים מסיבות שונות לידי נסיון יקראו חטא, מלשון חסרון, כמו קולע אל השערה ולא יחטיא, כי חסר נפשו מטובה ומדת הזהירות שקנתה הפסידה לפעמים ע״י חטא אחד המאבד טובה הרבה }\מקור{[ע״ס 34]}\צהגדרה{. }

\paragraphs

\ערך{חטובה }\הגדרה{- מאורגנת}\myfootnote{ \textbf{חטובה} - \textbf{מאורגנת} - כלומר: אורגנית. ע׳ בנספחות, מדור מחקרים, ארגון, מאורגן.\label{3}}\הגדרה{ }\מקור{[קובץ ו רכט]}\צהגדרה{. }

\paragraphs

\ערך{חטיבה}\הגדרה{ - צורה\mycircle{°} אורגנית\mycircle{°} }\מקור{[עפ״י א״ק א יד]}\צהגדרה{.}

\הגדרה{אחדות גמורה }\מקור{[ע״א ד ו מה]}\צהגדרה{.}

\הגדרה{קשור כללי }\מקור{[א״ש ה]}\צהגדרה{. }

\משנה{חטיבה פנימית}\הגדרה{ - מאור מאחד במהותיותו העצמית }\מקור{[עפ״י א״ק ב תנז]}\צהגדרה{.}

\paragraphs

\ערך{חיות טבעי }\הגדרה{- }\משנה{התוכן החיוני }\הגדרה{- המזגים והתכונות הגופניות }\מקור{[ע״ר א קכב]}\צהגדרה{. }

\paragraphs

\ערך{חיי האדם }\הגדרה{- חיבור נטית החפץ לקיום ההרגשה העצמית - אשר לכל חי - עם רגש שאיפת ההשתלמות, הנובעת מההשתקפות הפנימית\mycircle{°} העמוקה אל השלמות האלהית\mycircle{°}, הנשקפת ביסודה הפנימי של הנשמה\mycircle{°} ממקור השלמות המוחלטה והנצחית, מקור אין\hebrewmakaf סוף\mycircle{°} }\מקור{[מ״ר 34]}\צהגדרה{. }

\משנה{חיים }\הגדרה{- אור הקודש\mycircle{°} וההערצה המביאה לפתוח בנו את כל מקורי השירה\mycircle{°} }\מקור{[עפ״י ר״מ קעח]}\צהגדרה{. }

\הגדרה{ע״ע חיים, תביעתם השלמה. ע״ע חיים, תוכן החיים. ע׳ בנספחות, מדור מחקרים, אהבה ויראה. ע׳ במדור פסוקים ובטויי חז״ל, חיים, ״חפץ חיים״. ושם, ימים, ״אוהב ימים לראות טוב״. }

\paragraphs

\ערך{חיי החיים }\הגדרה{- הרוח הגדול השוכן בשמי שמי עז\mycircle{°}, בהיכל הצדק\mycircle{°} והבינה\mycircle{°}, הנותן עז וגבורה\mycircle{°}, שהוא העז והגבורה, הוא נותן חיים, והוא החיים\mycircle{°} }\מקור{[עפ״י א״ק ב שסו]}\צהגדרה{.}

\paragraphs

\ערך{חיים }\הגדרה{- שלמות עליונה של סגולת הרכבת החלקים וקשורם }\מקור{[ע״ר א רצ]}\צהגדרה{. }

\משנה{כח החיים }\הגדרה{- התאחדות כל פרטי האברים והחלקים שבגוף למרכז אחד, <שע״י שליטת השכל\mycircle{°} שולטים החיים בכל פרטי פרטיות הקטנים שבגוף> }\מקור{[ע״א א ה ע]}\צהגדרה{.}

\paragraphs

\ערך{חיים }\הגדרה{- }\משנה{החיים }\הגדרה{- שורש הרצון השופע על התנועות וההרגשה}\צהגדרה{ }\מקור{[פ״א רכז]}\צהגדרה{.}

\ערך{חיים }\הגדרה{- }\משנה{״החיים״}\הגדרה{ -  תוכן הפנימי של החיים, שהוא מלכות\hebrewmakaf שמים\mycircle{°} ויראת\hebrewmakaf שמים\mycircle{°} }\מקור{[מ״ר 390]}\צהגדרה{.}

\הגדרה{היחש האלהי\mycircle{°} של ההויה }\מקור{[מ״ה יראה ב (א״ק ד תלא)]}\צהגדרה{. }

\משנה{חיי אמת }\הגדרה{- חיים של נועם\hebrewmakaf ד׳\mycircle{°} וההתענג\mycircle{°} מזיו\mycircle{°} כבודו\mycircle{°} }\מקור{[ע״א ג ב סט]}\צהגדרה{.}

\משנה{החיים האמיתים }\הגדרה{- קדושת\mycircle{°} הנפש }\מקור{[מ״ש קנב (ה׳ קצא)]}\צהגדרה{. }

\הגדרה{(חיים) המתאימים למקור\hebrewmakaf הנשמה ואמתת הישות }\מקור{[עפ״י אג׳ ב רסג]}\צהגדרה{.}

\הגדרה{חיי גדולה\mycircle{°} גבורה\mycircle{°} וקדושה\mycircle{°} }\מקור{[אג׳ א מט]}\צהגדרה{.}

\משנה{עיקר החיים (אפילו בחיי הזמן) }\הגדרה{- עוצם הרגשת הכוחות הנפשיים בהוד החיים }\מקור{[ע״א א ג י]}\צהגדרה{. }

\הגדרה{ע״ע חיים עצמיים. ע״ע חיי החיים.}

\paragraphs

\ערך{חיים }\הגדרה{- פעולותיו הבחיריות\mycircle{°} (של האדם) }\מקור{[ע״א א ב לד]}\צהגדרה{. }

\paragraphs

\ערך{חיים }\הגדרה{- }\משנה{תביעתם השלמה }\הגדרה{- התביעה המרחקת את האדם מן החטא\mycircle{°}, המעמידתו ישר\mycircle{°} כאשר עשהו אלהים\mycircle{°} }\מקור{[א״ש י ה]}\צהגדרה{. }

\הגדרה{ע״ע חיי האדם. }

\ערך{חיים }\הגדרה{- }\משנה{יסוד החיים }\הגדרה{- רוח\mycircle{°} החיים. יסוד רוחני, בעל תוכן מתעלה. הכח המעלה, המרומם, הנושא את נושאיו החומריים\mycircle{°} }\מקור{[ע״א ד י, יב יג]}\צהגדרה{. }

\משנה{יסוד החיים במהותו}\הגדרה{ - הכח - הכח המעשי, כח העבודה שבחיים ובמציאות, השואב את עז המפעל ממקורו ומשלח את מעינותיו ופלגיו לכל מקצעות המפעלים כולם; ההתעוררות - כח המתעורר שבחיים, המעורר את המפעלים, ביתרון התעצומה שבהמשך החיים; וההמשך של החיים - כל המשכת החיים של תכן הזמן }\מקור{[עפ״י ע״ר א קסד, קסג]}\צהגדרה{.}

\ערך{חיים }\הגדרה{- }\משנה{תוכן החיים }\הגדרה{- רצון החיים לחיות, המעריך את ערכם }\מקור{[עפ״י ע״ר א לה]}\צהגדרה{. }

\הגדרה{ע״ע חיי האדם. }

\ערך{חיים - }\משנה{תכליתם }\הגדרה{- קרבת\hebrewmakaf אלהות\mycircle{°} }\מקור{[ע״א א ה מג]}\צהגדרה{. }

\ערך{חיים - }\צהגדרה{חיים של אמת }\הגדרה{- חיי\hebrewmakaf עולם\hebrewmakaf הבא\mycircle{°} מסבת ההטבה הכללית\mycircle{°} בחיי\hebrewmakaf שעה\mycircle{°}, והמשכת צורתם\mycircle{°} הפנימית\mycircle{°} }\מקור{[קבצ׳ ב סד]}\צהגדרה{. }

\paragraphs

\ערך{״חיים״ }\הגדרה{- }\משנה{״חפץ חיים״ }\הגדרה{- ע׳ במדור פסוקים ובטויי חז״ל. }

\paragraphs

\ערך{חיים }\הגדרה{- }\משנה{(לעומת אור\mycircle{°}) }\הגדרה{- רצון\mycircle{°}, מציאות\hebrewmakaf מלאה\mycircle{°} }\מקור{[עפ״י א׳ יא, ע״ר ב עז]}\צהגדרה{. }

\paragraphs

\ערך{״חיים״ }\הגדרה{- }\משנה{תאר למעלה שבמצוות\mycircle{°} (לעומת אורך\hebrewmakaf ימים\mycircle{°}) }\הגדרה{- ההרגשה הקדושה\mycircle{°}, שמרגיש (אדם) שנעשה לבבו יותר קרוב אל השי״ת\mycircle{°}, ומדותיו יותר מזוקקות וצרופות }\מקור{[ע״ר א תיא\hebrewmakaf ב (פנק׳ ג רסח)]}\צהגדרה{. }

\paragraphs

\ערך{״חיים״ }\הגדרה{- }\משנה{״חלקם בחיים״ }\הגדרה{- ע׳ במדור פסוקים ובטויי חז״ל, חלקם בחיים.}

\paragraphs

\ערך{חיים עליונים }\הגדרה{- }\משנה{החיים העליונים }\הגדרה{- החיים הקדושים השמימיים חיי החכמה, חיי העדן הקדוש של נועם\hebrewmakaf ד׳\mycircle{°}  }\מקור{[א״ק ב תקסא\hebrewmakaf ב]}\צהגדרה{. }

\paragraphs

\ערך{חיים עצמיים }\הגדרה{- החיים השופעים ממקור האמת\mycircle{°} המוחלטת }\מקור{[עפ״י מ״ר 148]}\צהגדרה{. }

\הגדרה{ע״ע חיים, חיי אמת. }

\משנה{תהום החיים העצמיים }\הגדרה{- אותו המקום\mycircle{°} העליון\mycircle{°} ששרשי הכחות, הרעיונות\mycircle{°}, הצביונות והאידיאלים\mycircle{°}, מתהוים שם }\מקור{[ר״מ לג]}\צהגדרה{. }

\paragraphs

\ערך{חיים פנימיים\mycircle{°}}\הגדרה{ - החיים השכליים והמוסריים\mycircle{°} של האדם מצד צורתו\mycircle{°} וצלמו צלם\hebrewmakaf אלהים\mycircle{°} }\מקור{[ע״א ג א מב]}\צהגדרה{.}

\paragraphs

\ערך{חיל }\הגדרה{- }\משנה{החיל }\הגדרה{- רכישת הכוחות וריכוזם למטרת האושר\mycircle{°} העולמי הכללי, שכל אושר של כל נקודה בה תלוי }\מקור{[א״ק ג עא]}\צהגדרה{. }

\הגדרה{ע׳ במדור פסוקים ובטויי חז״ל, איש חיל.}

\paragraphs

\ערך{חיל }\הגדרה{- }\משנה{עז החיל }\הגדרה{- הצדק\mycircle{°} המחדש ומהווה }\מקור{[א״ק ג ע]}\צהגדרה{. }

\paragraphs

\ערך{חילה }\הגדרה{- הידיעה במה שאין ראוי לצייר\mycircle{°}, כמו חלל פנוי במשפט הכרתו מיראת\hebrewmakaf ד׳\mycircle{°} }\מקור{[ע״ר א רנו]}\צהגדרה{. }

\paragraphs

\ערך{חילוי }\הגדרה{- תפילה\mycircle{°} שבאה לצרכי הכלל }\צהגדרה{<מגזרת חולי. שצרת הרבים מסעירה מאד את הלב, עד שהוא במצב חולי מרוב כאב ועקת נפש, כי לא תוכל נפש היחיד לשאת רגש צרת הכלל הנוגע עד לבבה כ״א ברגש חזק מאד>. }\הגדרה{(ו)מצייר הרעידה וההגבלה לעמוד בתחום השגתו שלא להרוס יותר במה שלא הורשה. להמנע מלהלך בגדולות במה שהוא מופלא משכל האדם. שבתפילת (היחיד על) הרבים ההכרח לשלח המחשבות בדברים יותר נעלים ונשגבים, עד שראוי להשמר עמהם ג״כ מהרוס אל המופלא }\מקור{[עפ״י ע״א א ד נז]}\צהגדרה{.}

\paragraphs

\תערך{חילוני }\תהגדרה{- }\תמשנה{נשמה חילונית }\תהגדרה{- זרם ההויה הבא מצד המשך ההויה שאנחנו קוראים טבע\mycircle{°}, ההשגה של אוה״ע, הזרם הבא מהאמצעים מהסיבות\mycircle{°} }\תמקור{[נ״א ה 31].}

\הגדרה{ע׳ במדור פסוקים ובטויי חז״ל, שכינה, גילוי שכינה.}

\paragraphs

\ערך{חילוני, חולוני, חולני }\הגדרה{- ע׳ בנספחות, מדור מחקרים. }

\paragraphs

\ערך{חילוף }\הגדרה{- ע״ע חלוף. }

\paragraphs

\ערך{חינוך }\הגדרה{- ע״ע חנוך. }

\paragraphs

\ערך{חיצוני }\הגדרה{- ע״ע חצוני. }

\paragraphs

\ערך{חיקוי }\הגדרה{- }\משנה{תכונת החיקוי (הטוב) }\הגדרה{- הנטיה של ההתיצבות בתואר המעמד המושג ממעל להגבול המוגבל של היש המפורט הנערך בערכיו המיוחדים. החיקוי המשכלל את הישות הוא מקור הפאר\mycircle{°} וההתהדרות\mycircle{°}, המעלה את כל שפל אל רום, מכון המנהגים הטובים והקבלות היקרות שהן הולכות משרים וקולעות אל מטרת רוממות צביונו }\מקור{[ר״מ כב]}\צהגדרה{. }

\תהגדרה{הכשרון להתקרב יותר ויותר למדרגה העליונה שעליו }\תמקור{[נ״א ה 31].}

\משנה{התחקות }\הגדרה{- יסוד ההעתקה של הציורים\mycircle{°}, המושגים, הכחות, המפעלים, והתנועות, המשפטים, השאיפות והאידיאלים, ממרומי מקוריותם לתחתית המשכת החיים והמציאות בחוקיהם }\מקור{[ר״מ כד]}\צהגדרה{. }

\הגדרה{ע״ע קופיות. }

\paragraphs

\ערך{חירות }\הגדרה{- ע״ע חרות.}

\paragraphs

\ערך{חכמה }\הגדרה{- מה שצריך עיון }\מקור{[מ״א א ב]}\צהגדרה{.}

\paragraphs

\ערך{חכמה ישראלית }\הגדרה{- }\משנה{החכמה הישראלית יסודה}\הגדרה{ - ההתגברות של העולם המאוחד על העולם המפורד}\צהגדרה{ }\מקור{[א״ק ב תכג]}\צהגדרה{.}

\הגדרה{ע׳ במדור תורה, חכמת התורה.}

\paragraphs

\משנה{חכמת הקודש }\צהגדרה{- חכמת\hebrewmakaf האמת\mycircle{°}, הכוללת הדעות כולן, בשלום\mycircle{°} פנימי }\צמקור{[א״ק א, מבוא הרד״ך 21, ובשער הפתיחה]. }

\הגדרה{ע״ע אורות הקֹדש. ע״ע מוסר הקודש.  }

\paragraphs

\ערך{חכמת ישראל }\הגדרה{- חכמת ההסתוריה והבקורת, הגיון הדעות, הפיוט וכל ענפיהם }\מקור{[עפ״י אג׳ א קמח]}\צהגדרה{. }

\paragraphs

\ערך{חכמת ישראל }\הגדרה{- כללות חכמת\hebrewmakaf התורה\mycircle{°} ומדע האמונה\mycircle{°}, דעת\hebrewmakaf אלהים\mycircle{°}, הבנת המוסר הנשגב וחינוכו וקנייתו, שבכללם נכנס ממילא ג״כ כל מה שנקרא בדרך הרגיל בפי הדור בשם זה}\myfootnote{ \textbf{כל מה שנקרא בדרך הרגיל בפי הדור בשם זה }- כלומר: חכמת ההסתוריה והבקורת, הגיון הדעות, הפיוט וכל ענפיהם.\label{4}}\הגדרה{ }\מקור{[עפ״י פנק׳ ג שיח (קבצ׳ ג כג)]}\צהגדרה{.}

\הגדרה{הדעות והמחשבות של היהדות\mycircle{°} }\מקור{[פנק׳ ד תצב]}\צהגדרה{.}

\משנה{חכמת ישראל המקורית והאמתית }\הגדרה{- חכמת הרזים\mycircle{°} }\מקור{[עפ״י א״ק א קיב]}\צהגדרה{.}

\הגדרה{קבלה\mycircle{°}, שיסודה הוא ההכרה בערך האלהי\mycircle{°} של נשמת\hebrewmakaf האומה }\מקור{[עפ״י מ״ר 280]}\צהגדרה{.}

\ערך{חכמת ישראל - }\צהגדרה{שבבירורה לאמיתתה מיוחסת לאליהו }\הגדרה{- הידיעה איך כל הדברים הגדולים רבי הפרטים של הקדושות המיוחדות לישראל בקנאתו לשמירת טהרת לאומיותו, הם דברים רבי ערך מאד, נצרכים ומוכרחים לכלל ישראל, וכל הפורש מהם פורש מן החיים וחותר מחתרת במעוזם של ישראל. ההגעה עד עומקם של דברים בהצביון המיוחד לישראל, להיות עם לבדד ישכן, שומר טהרתו ויחוסו}\צהגדרה{ }\מקור{[ע״א ב ט רחצ]}\צהגדרה{.}

\הגדרה{ע׳ במדור תורה, חכמת התורה. }

\paragraphs

\ערך{חכמת נשמת אל בעולם }\הגדרה{- החיים הרוחניים\mycircle{°} העליונים\mycircle{°} הכוללים. החכמה\mycircle{°} הכללית העליונה, החיה ומחיה את ההויה }\מקור{[עפ״י ע״ט סב]}\צהגדרה{. }

\הגדרה{ע׳ במדור מונחי קבלה ונסתר, ספירות, חכמה.}

\paragraphs

\ערך{חלום }\הגדרה{- ציור\mycircle{°} בחפץ כביר, המשולל מכל התחשבות עם איזה מניעה, עם הגבלת יכולת, מאיזה צד ופינה }\מקור{[עפ״י ר״מ מח]}\צהגדרה{. }

\משנה{חזיון החלום }\הגדרה{- החזיון הפנימי שהנפש מתבודדת בעצמותה כשמתנשאת מעל לכל הגבולים הצרים של העולם המעשי, ומתנשאת הנפש לעולם אחר לגמרי }\מקור{[ע״א ד ה פד]}\צהגדרה{. }

\ערך{חלום }\הגדרה{- }\משנה{ענינו }\הגדרה{- התעוררות מהכח והסגולה\mycircle{°} הפנימית שיש בנפש החולם לסבב את גלגל מעשיו ומקריו }\מקור{[מ״ש רלא (מא״ה ג קא)]}\צהגדרה{. }

\הגדרה{התחלת ההתפתחות שהסגולה הנעלמה שבנפש מתפתחת לצאת מן הכח אל הפועל }\מקור{[מ״ש רכה (מא״ה ג צד)]}\צהגדרה{. }

\משנה{הוראת החלום }\הגדרה{- שיש סגולה צפונה בנפש החולם שכאשר תצא מן הכח אל הפועל תסבב אותן הפעולות הנוגעות לו כפי מה שבא רמיזתן בחלום }\מקור{[עפ״י מ״ש רכד (מא״ה ג צד)]}\צהגדרה{.}

\ערך{חלומות ופתרונם }\הגדרה{- }\משנה{עניינם}\הגדרה{ - עיקר הכח אל הפעולות הנוגעות לאדם שם השי״ת בסגולות נפשו ובציוריו\mycircle{°}, והציורים באים ג״כ בהשגחת\mycircle{°} השי״ת ע״פ המעשים, והם מתחילים לצאת מן הכח אל הפועל ע״י החלומות\mycircle{°}, רק שאותן ההוראות שצריכין חלומות לצאתן אל הפועל עדיין חלושות הן ועוד לא יצאו כראוי אל הפועל. והחלום כשם שהוא בא מכח עומק הסגולה שבנפש שבא ע״י מעשיו ומדותיו ודרכיו של אדם, ה״נ ע״י החלום מתחזקת הסגולה הצפונה שהרי מתחזקת בציורה, מיהו צריכה עוד להתחזק ע״י הפותר ע״פ דרך האחת, והפותר הוא מפתח את הסגולה. ע״כ אם יפתור בכל ההוראות הצודקות שבו, אע״פ שסותרות זא״ז כי זו תהי׳ לטוב וזו להפכו, כולן נכונות להתקיים }\מקור{[מ״ש רכה]}\צהגדרה{. }

\הגדרה{ע׳ במדור פסוקים ובטויי חז״ל, כל החלומות הולכים אחר הפה}\צהגדרה{.}

\paragraphs

\ערך{חלוני}\הגדרה{ - ע״ע חילוני. }

\paragraphs

\ערך{חלוף }\הגדרה{- שלילת ערך קיים מתואר ב}\משנה{חליפין}\הגדרה{, כמו ״יציץ וחלף״ }\מקור{[ע״ר א נד]}\צהגדרה{. }

\הגדרה{ע״ע תמורה. }

\paragraphs

\ערך{חלל }\הגדרה{- המקום הריק המשולל ממציאות }\מקור{[ע״א א ד נז]}\צהגדרה{.}

\paragraphs

\ערך{חַמָּה }\הגדרה{- }\משנה{(בחינתה שבאדם) }\הגדרה{- ענין פעולתו שפועל (אדם) עליו, בתורה ומצות - לטוב, ובעבירות - להפכו, מצד עצמו ומעשיו וחכמתו }\מקור{[עפ״י מ״ש קעט (מא״ה א קב)]}\צהגדרה{. }

\הגדרה{ע״ע לבנה, (בחינתה שבאדם).}

\paragraphs

\ערך{חמר }\הגדרה{- }\משנה{החמריות שבעולם }\הגדרה{- הכוחות הפועלים המכניים }\מקור{[ע״ט סו]}\צהגדרה{.}

\הגדרה{הטבע הקשה, החשוך, הקשור לחק עפר אבן וברזל }\מקור{[עפ״י א״ק ב תקכד]}\צהגדרה{.}

\ערך{חמרי }\הגדרה{- }\משנה{החזיון החמרי של ההויה }\הגדרה{- ממשותה הראלית }\צהגדרה{<שהיא המסקנה של החזיון הרוחני של ההויה, של האידיאל שלה, ועולה עד ראשה> }\מקור{[א״ת ד ב (ע״ט נג)]}\צהגדרה{.}

\משנה{הענינים החמריים }\הגדרה{- (הענינים) המדיניים אקונומיים }\מקור{[ע״א ג א לג]}\צהגדרה{.}

\ערך{חמרי }\הגדרה{- }\משנה{(הכח החמרי בנפש) }\הגדרה{- התעודדות החיים, חשק העבודה ומרץ הפעולה }\מקור{[א״ק ג ראש דבר ל (מ״ה יראה ט)]}\צהגדרה{.}

\הגדרה{ע׳ במדור נפשיות, רוחני, הכח הרוחני. ע״ע רוחניות.}

\paragraphs

\ערך{חן }\הגדרה{- הגודל\mycircle{°} והפאר\mycircle{°} הנפשי, המתגדל ע״י ההכרה לטובה, שהאדם מכיר איזה חזון, איזו מציאות, אע״פ שהרושם הזה, של הגודל והפאר, אין בו כלל משום נטיה מעשית לטובה ממשית ולכבוד להנושא ההוא }\מקור{[ע״ר א עט]}\צהגדרה{. }

\הגדרה{ע״ע חסד. }

\משנה{נתינה לחן }\הגדרה{- מציאות המעוררת רגשי כבוד, גדולה ותפארת, בכל הנפגש עמה, ובכל הסביבה כולה }\מקור{[עפ״י שם שם]}\צהגדרה{. }

\משנה{חן }\צהגדרה{- }\מעוין{◊ }\צהגדרה{יסודו בכללות האדם ולא בפרט מיוחד שבו, בכלל פרצופו, בכל הנשמה כולה. מגלה באופן היותר עמוק את ענין החיים }\צמקור{[עפ״י ק״ת סג].}

\משנה{כח החן ואור הנשמה }\הגדרה{- קרני ההוד\mycircle{°} היוצא מזוהר\mycircle{°} אור חכמת אדם אשר תאיר פניו }\מקור{[ר״מ קעד\hebrewmakaf ה]}\צהגדרה{. }

\paragraphs

\ערך{חן }\הגדרה{- }\משנה{עניינו }\הגדרה{- השפעת\mycircle{°} דבר }\מקור{[ע״ר א רנו]}\צהגדרה{. }

\paragraphs

\ערך{חן }\הגדרה{- }\משנה{החן הפנימי של המחשבה }\הגדרה{- הקדש\mycircle{°} היסודי שבתוכה }\מקור{[א״ש יד יד]}\צהגדרה{. }

\paragraphs

\ערך{חִנּוּך }\הגדרה{- הכנסה בענין מחודש מתמיד }\מקור{[ע״ר א תלג]}\צהגדרה{. }

\paragraphs

\ערך{חִנּוּך }\הגדרה{- הטית התוכן הנפשי אל המבוקש מכאן ולהבא }\מקור{[עפ״י ע״א ד ט ל]}\צהגדרה{. }

\הגדרה{עבודת רוח האדם. עסוק בפתוח הכוחות הנפשיים, להשלימם, להוציאם מן הכח אל הפועל, לעדנם, להרחיבם, לחדדם ולשכללם, ולהעמידם על מצב החופש\mycircle{°} }\מקור{[עפ״י קבצ׳ ג יא-יב]}\צהגדרה{. }

\הגדרה{פקוח והוצאה מן ההעלם אל הגלוי ומן הכח אל הפועל מה שהוא גנוז בנשמת המתחנך ושהוא גלוי בתור אופי כללי ומרכזי בכללות האנושיות וקבוציה ההיסטוריים והלאומיים. התאמת היחיד עם התרבות הכללית בעומק טבעה }\מקור{[עפ״י מ״ר 33]}\צהגדרה{.}

\הגדרה{הדרכה, להסתגלות לההויה הכללית ולמקוריותה }\מקור{[שם 100]}\צהגדרה{. }

\משנה{חינוך }\צהגדרה{- }\צמשנה{של האיש הפרטי }\צהגדרה{- ההוצאה מכח לפועל, מהעלם לגילוי, את הכוחות והכשרונות הנמצאים בנפש הילד בטבעה, או הראויים לפי טבעה להתקבל בה ביותר מתוך ההשפעות החינוכיות השונות }\צמקור{[ל״י א ו]. }

\משנה{יסוד החינוך הטוב }\הגדרה{- לבסס יפה את הרגש הטבעי לטובה\mycircle{°}, לצדק\mycircle{°}, לקדושה\mycircle{°} ולמישרים\mycircle{°}, באופן שכשיגדיל האדם כבר ימצא דרך החיים מוכנת לפניו ואיננו צריך כ״א להוסיף לקח ושכל\hebrewmakaf טוב\mycircle{°}, לעלות מעלה\hebrewmakaf מעלה\mycircle{°} }\מקור{[ע״א ג ב ריד]}\צהגדרה{. }

\משנה{מטרת החינוך }\הגדרה{- להכשיר את האדם לצורתו המתוקנת, שהנקודה המרכזית שבה היא לעשותו טוב\mycircle{°} וישר\mycircle{°} לפני ד׳\mycircle{°} ואדם }\מקור{[עפ״י אג׳ א ריח-ריט]}\צהגדרה{.}

\משנה{תפקיד החנוך }\צהגדרה{- }\הגדרה{למצוא את הדרך לטרקלין\hebrewmakaf מלכו\hebrewmakaf של\hebrewmakaf עולם\mycircle{°}, אל הנקודה הפנימית של דרך החיים, מלכות\hebrewmakaf שמים\mycircle{°}  ויראת\hebrewmakaf שמים\mycircle{°} }\מקור{[עפ״י מ״ר 390]}\צהגדרה{.}

\משנה{תעודת החנוך בישראל }\הגדרה{- להביא את הילדות בעצם תומה וטהרתה אל גבול הבחרות והשחרות, הזקנה והשיבה, באופן שתוסד עוז, שלעולם לא ינוס לחה}\צהגדרה{ }\מקור{[מ״ר 230]}\צהגדרה{.}

\הגדרה{ר׳ תרבות. }

\paragraphs

\ערך{חנוך}\הגדרה{ - }\משנה{החנוך הכללי אצל העמים}\הגדרה{ - צורת חינוך המכרת את הילדות רק למעבר בעלמא, לזיין על ידה את האיש העתיד בכלי קרב למלחמת קיומו }\מקור{[עפ״י מ״ר 230]}\צהגדרה{.}

\paragraphs

\ערך{חנוכת הבית }\הגדרה{- }\משנה{(חנוכת בית\hebrewmakaf המקדש\mycircle{°}) }\הגדרה{- קביעת הבית הגדול והקדוש שממנו תצא תורה\mycircle{°} ואורה לישראל ולעולם עדי\hebrewmakaf עד\mycircle{°}. קביעות מקום האורה\mycircle{°} לישראל ולעולם }\מקור{[ע״א ג ב נ]}\צהגדרה{. }

\הגדרה{ההכנה, הכשרתו של האידיאל ההולך ומתהוה, הולכת ונמשכת ע״י עבודות תדיריות, הקבועות במקום היותר מוכשר להתעלות היותר מקודשת }\מקור{[ע״ר א קפו]}\צהגדרה{. }

\paragraphs

\ערך{חניה }\הגדרה{- קביעת מקום והתפסו }\מקור{[ע״ר א רנו]}\צהגדרה{. }

\paragraphs

\ערך{חנינה }\הגדרה{- }\משנה{מדת החנינה }\הגדרה{- בוחן לבבות הוא ידע את הטוב העצמי שגם בהמנעו מצאת אל הפועל הוא חפץ יקר ונעלה שראוי למצא חן\mycircle{°} מצד עצמיותו, ע״כ גם הנחשלים שבטובים, שטובם לא יצא אל הפועל, אבל כ״ז שלא הושחת יסודם והזיק האחרון קיים בקרבם מנחלת אבות, ראויים הנם לחן, יגיעו למצבים כאלה שיוכל הטוב הגנוז ג״כ לצאת לפועל }\מקור{[ע״א ד ה לב]}\צהגדרה{. }

\paragraphs

\ערך{חסד }\הגדרה{- ענין שירושם בנטיה מעשית, להגדיל הטוב של איזה חזון, או איזו מציאות, בכל מלוי הרצון }\מקור{[עפ״י ע״ר א עט]}\צהגדרה{. }

\הגדרה{ע״ע חן. ע׳ במדור פסוקים ובטויי חז״ל, אהבת חסד. }

\ערך{חסד }\הגדרה{- }\משנה{כח החסד }\הגדרה{- הרצון להשפיע טוב\mycircle{°}, שאיפת הטוב\mycircle{°} מצד עצמה }\מקור{[עפ״י ע״ר א קג]}\צהגדרה{. }

\ערך{חסד }\הגדרה{- הנהגה טובה של צדק\mycircle{°} ומשרים\mycircle{°}}\צהגדרה{ }\מקור{[פנק׳ א לט]}\צהגדרה{. }

\צהגדרה{כח הצדק\mycircle{°} והיושר\mycircle{°}, החסד\mycircle{°} והרחמים\mycircle{°}, המוסרי\mycircle{°} הפנימי\mycircle{°} שבנפש האדם }\צמקור{[א״ל רג].}

\הגדרה{ע״ע צדק, כח הצדק. ע׳ בנספחות, מדור מחקרים, חסד לעומת צדקה. }

\paragraphs

\משנה{״חסד״}\myfootnote{ סוכה מט: ״אין צדקה משתלמת אלא לפי חסד שבה, שנאמר: ״זרעו לכם לצדקה וקצרו לפי חסד״״.\label{5}}\הגדרה{ - }\צהגדרה{הרגש הפנימי הנעלה }\צמקור{[שי׳ ה 57]}\הגדרה{.}

\paragraphs

\ערך{חסד }\הגדרה{- }\מעוין{◊ }\משנה{מדת החסד }\הגדרה{- כוללת כל המדות הטובות כולם }\מקור{[ע״א א א ה]}\צהגדרה{.}

\paragraphs

\ערך{חסד }\הגדרה{- }\משנה{אור החסד של אברהם אבינו }\הגדרה{- ע׳ במדור מדתם ועניינם הרוחני של אישי התנ״ך, אברהם. }

\paragraphs

\ערך{חסד }\הגדרה{- }\משנה{(לעומת ברית\mycircle{°}) }\הגדרה{- האידיאליות\mycircle{°} המנודבת. מעלת חפשו\mycircle{°} ויסוד מציאותו מתוך רצון מלא נדבה, שהוא גילויו האידיאלי של כל דבר נעלה בחיי\hebrewmakaf הרוח\mycircle{°} המתפשט במציאות }\מקור{[עפ״י ע״ר א פג, פד]}\צהגדרה{. }

\הגדרה{ע׳ במדור מונחי קבלה ונסתר, אחרית, לעומת הראשית בחיי הרוח. ע׳ במדור פסוקים ובטויי חז״ל, נתתי את תורתי בקרבם ועל לבם אכתבנה.}

\paragraphs

\ערך{חסד }\הגדרה{- }\משנה{מדת החסד המיוחסת לבית הלל}\myfootnote{ זוהר ח״ג רמה.\label{6}}\הגדרה{ - הכשרת הרגש לפי המדה האנושית ועשית דרך סלולה לכל העם כולו, יעלו בה כקטן כגדול }\מקור{[עפ״י ע״א ב ח ד]}\צהגדרה{. }

\הגדרה{ע״ע דין, מדת הדין המיוחסת לבית שמאי. }

\paragraphs

\ערך{חסד מוחלט }\הגדרה{- }\משנה{״רב חסד״ }\הגדרה{- ההטבה\mycircle{°} שאין לה קץ ותכלית }\מקור{[ע״ר א קכח]}\צהגדרה{. }

\הגדרה{השטף הגדול של החסד\mycircle{°} }\מקור{[שם נו]}\צהגדרה{. }

\paragraphs

\ערך{חסחון}\myfootnote{ ע׳ מנחות עו:, וברש״י זבחים ו:, ד״ה אלא שחיסך הכתוב. ועוד י״ל מלשון - חס וחן (גיא מונטנר).\label{7}}\הגדרה{ - ההבטה בעין יפה, שלא יקטן בעיניו כל פרט}\צהגדרה{ }\מקור{[עפ״י פנק׳ ב קסז (ל״ה 247)]}\צהגדרה{.}

\paragraphs

\ערך{חסחון }\הגדרה{- }\משנה{החסחון הרוחני }\הגדרה{(}\צהגדרה{בישראל}\הגדרה{) - ההבטה בעין יפה על כל מה שהוא משלנו, ממוסר, ממִדות, מדעות, מדינים ומנהגים, לאהוב הכל ולחבב הכל}\צהגדרה{ }\מקור{[עפ״י פנק׳ ב קסז (ל״ה 247)]}\צהגדרה{.}

\paragraphs

\ערך{חסידות }\הגדרה{- כשרון אמנותי נפלא המבסס את רוח האדם בכללו ומעמידו על עמדתו העליונה, שהוא שואף אליה בפנימיות תכנו }\מקור{[עפ״י א״ק ג שמו]}\צהגדרה{. }

\הגדרה{דבקות\hebrewmakaf אלהית\mycircle{°}, (ושייכות ל)חכמות רוחניות עליונות מופשטות }\מקור{[עפ״י שם רז]}\צהגדרה{. }

\הגדרה{התעמקות קדושה\mycircle{°} בהכרה בהירה של הבינה והמדע הרוחני המזהיר. יסודה אהבת\hebrewmakaf החסד\mycircle{°}, שאיפת ההטבה המוחלטת כשהיא מתגברת בלב, עד שהיא נטבעת בטבע הנפש }\מקור{[עפ״י ע״ר א קפז\hebrewmakaf ח, קצז]}\צהגדרה{. }

\הגדרה{התעמקות קדושה המביאה לידי הבעה גלויה את יסוד ההארה הצפונה, המתגלה בהתגלות מצערה לפי הערך, בפי עמו (של התהלה\mycircle{°}) }\מקור{[עפ״י ע״ר א קצז]}\צהגדרה{.}

\הגדרה{הנטייה אל הטוב\mycircle{°} והצדק\mycircle{°} }\מקור{[עפ״י פנק׳ ב סז (ל״ה 106)]}\צהגדרה{.}

\משנה{החסידות הטהורה }\הגדרה{- אהבת חסד במלואה }\מקור{[א״י סד]}\צהגדרה{.}

\משנה{החסידות ענינה }\הגדרה{- לעשות נחת\hebrewmakaf רוח\hebrewmakaf ליוצרו\mycircle{°} }\מקור{[מ״ר 275]}\צהגדרה{. }

\משנה{מדת החסידות הגדולה }\הגדרה{- }\מעוין{◊}\הגדרה{ יוצאת מתוך בינה\mycircle{°} אלהית\mycircle{°} עליונה ותכופה }\מקור{[א׳ נג]}\צהגדרה{. }

\משנה{שורש החסידות האמיתית }\הגדרה{- ההנחה האמיתית של כל דבר ע״פ תכונתו האמיתית בתכלית היושר\mycircle{°} והתיקון, עד שלא תמצא הפרעת סדר מפני כל רוח ונטיה אף כחוט השערה }\מקור{[ע״א א ה צא]}\צהגדרה{. }

\משנה{יסוד תמימות העבודה והחסידות האמיתית }\הגדרה{- שהשלמות הפרטית ותשוקתו אליה לא תתפוש מקום כלל לעומת חשקו בשלימות הכלל}\צהגדרה{<ואיש כזה שקנה לו השלמות הזו ביושר שכלו וטבעו, הוא החסיד\hebrewmakaf האמיתי\mycircle{°} ההולך בעצת השכל ולא ע״פ הרגש של אהבת עצמו> }\מקור{[עפ״י ע״א א ג נא]}\צהגדרה{. }

\הגדרה{ע׳ במדור מדרגות והערכות אישיותיות, חסיד.}

\ערך{חסידות }\הגדרה{- }\משנה{החסידות המעשית }\הגדרה{- ההתגשמות של קרבת\hebrewmakaf אלהים\mycircle{°} בחיים בפועל }\מקור{[א״ק ג שיא]}\צהגדרה{. }

\הגדרה{ע״ע מעשים של קדושה, שהחסידות המעשית מתלבשת בהם וכו׳. }

\paragraphs

\ערך{חסידות }\הגדרה{- }\משנה{(מגמת תנועת החסידות) }\הגדרה{- לרומם קדושת המדות, קדושת האמונה, (ו)ההכרה הכללית בקדושתם של ישראל ומעלתם, (ו)להגדיל את רגשי הקדש, הנרדמים בלב בטבע, ע״י רוממות ערך התפלה}\צהגדרה{ }\מקור{[א״י כה]}\צהגדרה{.}

\הגדרה{להאיר את האור\hebrewmakaf האלקי\mycircle{°} בעשירות ובזריחה\mycircle{°} מבהקת בכל לב\mycircle{°} וכל מח}\צהגדרה{ }\מקור{[אג׳ א קלב]}\צהגדרה{.}

\הגדרה{ע״ע התנגדות.}

\paragraphs

\ערך{חסיון אלהי }\הגדרה{- }\משנה{הבא מתוך הדבקות\mycircle{°}}\הגדרה{ - צרור\hebrewmakaf החיים\mycircle{°} }\מקור{[עפ״י ע״ר ב עז]}\צהגדרה{. }

\הגדרה{ע״ע בטחון נשגב עליון. }

\paragraphs

\ערך{חפש }\הגדרה{- }\משנה{החפש היסודי }\הגדרה{- התפשטות הנפש וכחותיה לאותם הצדדים שהיא נוטה, שוקקת ושואפת אליהם מעצמות טבעה }\מקור{[מ״ר 33]}\צהגדרה{. }

\הגדרה{מעמד טבעי כזה שמצד עצמו יחשק כבר בהנהגה היותר שלמה ומתוקנת.}\צהגדרה{ <זה בא אחר עמל גדול וידיעה שלמה בהשכלת החיים, שהציור הפנימי התרומם, להתרחב ולהתעמק בעומק היותר נעלה עד כדי להרגיש יפה את כל הקוים היותר דקים הנמצאים בענפי החיים וכל סכסוכים שנמצאים בהם, ולדעת שעכ״ז מטרתו חזקה ובינתו ברורה שיבחור תמיד את הטוב> }\מקור{[עפ״י פנ׳ קה]}\צהגדרה{.}

\הגדרה{כשתנועות החיים הולכות אחרי ההכרה הברורה בטוב\mycircle{°} שנמצא בנטיה שהן חפצות לנטות לה }\מקור{[קבצ׳ ב מז (ב״ר שכח)]}\צהגדרה{.}

\משנה{מצב החופש}\הגדרה{ - שיהיו הכוחות הנפשיים עושים את תפקידם ברוח האדם לא על\hebrewmakaf פי מועקה של עבדות\mycircle{°}, כי\hebrewmakaf אם על\hebrewmakaf פי הכרה והתנדבות הרוח המלא אורה\mycircle{°}, ודעת מלאה בערכו ותוכן פעולתו }\מקור{[עפ״י קבצ׳ ג יא-יב]}\צהגדרה{.}

\הגדרה{האידיאליות\mycircle{°} העליונה שהיא גלוי אור הקדש\mycircle{°} שבשאיפה הפנימית של הרוח\mycircle{°}}\צהגדרה{ }\מקור{[חד׳ קמ]}\צהגדרה{.}

\הגדרה{הרגשת שורש נטית הנשמה ממקור החיים שלה }\מקור{[עפ״י א״ק א ה]}\צהגדרה{.}

\משנה{חפשי }\הגדרה{- שיסודו ביסוד הנשמה\mycircle{°} במקוריותה }\מקור{[קובץ א תקיב]}\צהגדרה{. }

\משנה{חופש, עליתו עד מרום פסגתו }\הגדרה{- חיים ברוחו הוא, כפי חפץ הטבע האביר של נשמתו החיה }\מקור{[א׳ ק]}\צהגדרה{. }

\הגדרה{כשאין הנשמה\mycircle{°} סובלת משום מוטה ועול של כל דעה מוסכמת שיוצאת שלא ממקור\hebrewmakaf ישראל\mycircle{°} }\מקור{[עפ״י א״ק א קלה]}\צהגדרה{. }

\ערך{החופש הישראלי }\הגדרה{- ההגיון הרזי\mycircle{°}, כלומר, הנשמה\mycircle{°} הישראלית בהיותה, בתור נשמה יחידה יונקת מטל החיים של כנסת\hebrewmakaf ישראל\mycircle{°}, חושבת ומציירת על פי טבעה, ועל פי כל אותם הגורמים שגרמו לה לחשוב על דבר כל אותם המאורעות הגדולים, המיוחדים לישראל ותולדתו, על פי אותו היחש האלהי, העליון, המובלט, אשר הופע באומה נפלאה זו }\מקור{[א״ק א קלה]}\צהגדרה{. }

\משנה{חפש גמור}\הגדרה{ - בלא כח צביוני מיוחד}\צהגדרה{ }\מקור{[עפ״י א״ק ג לו]}\צהגדרה{.}

\הגדרה{בלא שום העקה}\צהגדרה{ }\מקור{[פנק׳ ג שלג]}\צהגדרה{. }

\הגדרה{חירות הרצון }\מקור{[א״ק ג לו]}\צהגדרה{.}

\הגדרה{ע״ע חרות. ע״ע דרור. }

\paragraphs

\ערך{חפש דעות }\הגדרה{- }\משנה{תעודתה של שאיפת חופש הדעות }\הגדרה{- סילוק הפחדנות הרפויה מכל התחום השכלי, כדי שיתגלה אור האלהים במלוי אורו, במלא חופש של שם\hebrewmakaf מלא\mycircle{°}}\צהגדרה{ }\מקור{[מ״ה פחדנות ג]}\צהגדרה{. }

\paragraphs

\ערך{חפש העליון }\הגדרה{- חופש רוח טהרה\mycircle{°} וקדושה\mycircle{°} }\מקור{[אג׳ ב רכו]}\צהגדרה{. }

\מעוין{◊}\הגדרה{ המבוקש העולמי היותר מגמתי\mycircle{°} }\מקור{[שם]}\צהגדרה{. }

\paragraphs

\ערך{חפש }\הגדרה{- }\משנה{עולם של חפש }\הגדרה{- כשהעולם המוסרי\mycircle{°} מרומם את העולם החוקי הסבתי כולו ומושכו אליו, משפיע עליו מאורו ונמצא שהוא כולו טבוע בים של אור\hebrewmakaf חיים\mycircle{°} זה של החוקים המוסריים שהם הרבה עליונים ונשגבים מהחוקים הסבתיים }\מקור{[עפ״י א׳ יט]}\צהגדרה{. }

\הגדרה{ע׳ במדור מונחי קבלה ונסתר, יושר (לעומת עיגולים). }

\paragraphs

\ערך{חפש מחשבה }\הגדרה{- התעלות\mycircle{°} ושכלול המחשבה\mycircle{°}, כשיותן לה כל היקף גדלה, כשיוכנו אשיותיה על פי הגודל הכמותי והאיכותי, בכל אשר תעיף שם את עיני רוחה\mycircle{°} }\מקור{[עפ״י א״ק א רטז]}\צהגדרה{. }

\paragraphs

\ערך{חפש עליון\mycircle{°}}\הגדרה{ - דבר\hebrewmakaf ד׳\mycircle{°} }\מקור{[ע״ר א מט]}\צהגדרה{. }

\הגדרה{הגבורה\mycircle{°} והחסד\mycircle{°} המוחלטים, באין גבול מדה והערכה }\מקור{[שם מו]}\צהגדרה{. }

\משנה{החפש האלהי }\הגדרה{- היכולת המלאה באין מעצור, הכוללת את האידיאליות\mycircle{°} המלאה השלמה ומכוללה מכל חמדה, מכל נצח\mycircle{°} והוד\mycircle{°}, מכל נשגב\mycircle{°} ומפואר\mycircle{°} }\מקור{[עפ״י א״א 140]}\צהגדרה{. }

\משנה{החפש העליון }\הגדרה{- היכולת\hebrewmakaf העליונה\mycircle{°} האוצרת בקרבה המון עולמים\mycircle{°} לאין תכלית }\מקור{[א״ק ג כח]}\צהגדרה{. }

\משנה{החפש הגמור }\הגדרה{- אמיתת השלמות, ההוראה ליכולת הגמורה }\מקור{[עפ״י ע״א א ה קז]}\צהגדרה{. }

\משנה{החופש}\הגדרה{ - אפס הגבלה, כל יכול(ת}\מקור{[עפ״י פנק׳ א רנד]}\צהגדרה{.}

\משנה{חפש גמור }\הגדרה{- אפס מצרים וגבולים\mycircle{°} מצמצמים\mycircle{°} }\מקור{[ע״ר א קס]}\צהגדרה{.}

\משנה{מרומי האור\mycircle{°} והחפש העליון }\הגדרה{- שם עטרת המלוכה צפונה, מקום\mycircle{°} ההפך הגמור של העבדות\mycircle{°} וההכרח - מקור מלכות ישראל }\מקור{[עפ״י מ״ר 38]}\צהגדרה{. }

\משנה{חפש עליון }\הגדרה{- תשובה\hebrewmakaf עילאה\mycircle{°}, עליצות דרור\mycircle{°} }\מקור{[א״ק ג ז (א״ש יב ט)]}\צהגדרה{. }

\משנה{החפש }\הגדרה{- שכל מקודש, יסוד החכמה\hebrewmakaf הקדומה\mycircle{°}, אומן\mycircle{°}, אב האמונה\hebrewmakaf העליונה - אמונת\hebrewmakaf אומן\mycircle{°}, שמכוחו האמונה נאצלת }\צהגדרה{[עפ״י א״א }\צמקור{128, 77}\צהגדרה{]. }

\משנה{החפש העליון }\הגדרה{- ההתעודדות המאושרה המתעלה מכל רעיון\mycircle{°} ומכל הקצבה מחשבתית, וק״ו מכל סדר מדותי ומכל תכונה מעשית }\מקור{[ע״ר א יג]}\צהגדרה{. }

\paragraphs

\ערך{חצון }\הגדרה{- }\משנה{עולם חיצון }\הגדרה{- המציאות הרחבה של הרוחניות ממעל להויתנו }\מקור{[א״ק ב שמח]}\צהגדרה{. }

\הגדרה{ע׳ בנספחות, מדור מחקרים, חיצון, עולם חיצוני. }

\paragraphs

\ערך{חצוני }\הגדרה{- }\משנה{ההכרה החצונית (לעומת פנימית\mycircle{°}) }\הגדרה{- ההכרה השכלית\mycircle{°} שבאה מתוך הכרת העולם והמציאות\mycircle{°} }\מקור{[א״ק א נח]}\צהגדרה{.}

\משנה{מושג חיצוני }\הגדרה{- מושג הבא לאדם מחוץ לפנים }\מקור{[עפ״י שם שם]}\צהגדרה{.}

\paragraphs

\ערך{חצוני }\הגדרה{- }\משנה{בנין חצוני ברוחו\mycircle{°} של אדם (לעומת בנין פנימי\mycircle{°}) }\הגדרה{- התורה, החכמה וההכרה, בלא - היראה\mycircle{°}, התפילה\mycircle{°} והעבודה\mycircle{°}. כל מה שהחפץ הפנימי\mycircle{°} עומד מרחוק, או אינו משתתף בזה. <אפילו ההכרה המדעית כשאיננה משתכללת כל כך עד לכדי הטבעת הרצון היותר עמוק על פיה. כל זמן שתהיה מציירת\mycircle{°} והוגה, וחפץ הלב בחייו אינו נוטל חלק במעמק החיים הציוריים הללו> }\מקור{[עפ״י א״ק ג פח]}\צהגדרה{.}

\paragraphs

\ערך{חצוני }\הגדרה{- ע״ע סקירה חצונית. }

\paragraphs

\ערך{חצוני }\הגדרה{- }\משנה{חצוניות באדם, הצד החצוני }\הגדרה{- הכלים\mycircle{°} החמריים\mycircle{°} וההמשכה הטבעית אחרי נטיותיהם }\מקור{[ע״א ג ב קא]}\צהגדרה{.}

\הגדרה{ע״ע פנימי, פנימיות האדם.}

\paragraphs

\ערך{חצוניות העולם }\הגדרה{- }\משנה{(לעומת פנימיותו) }\הגדרה{- ע׳ במדור מונחי קבלה ונסתר.  }

\paragraphs

\ערך{חצות לילה }\הגדרה{- ע״ע לילה, חצי הלילה הראשון. ע״ע לילה, מחצית האחרונה של הלילה.}

\paragraphs

\ערך{חרדת קדש }\הגדרה{- החרדה והרגשת כח הקדושה ע״י ההתנגדות שיש בטבעה הבהמי של הנפש הטבעית אל השלמות }\מקור{[עפ״י ע״א א ה ה]}\צהגדרה{. }

\paragraphs

\ערך{חרוב }\הגדרה{- אילן שתכונת מטעו בעולם בא ע״י דעה מיושבת של אהבת הכלל, באופן מוסרי עליון, בתור חובה של היושר המושכל לנטע בשביל שיהנו הדורות הבאים, ״כי היכי דנטעי לי אבהתי״}\myfootnote{ תענית כג. ״כי היכי דשתלי לי אבהתי שתלי נמי לבראי״.\label{8}}\הגדרה{. ע״כ הוא יותר קרוב להשכליות הנשאות של גדולי עולם המקפלים את הטובה היותר עליונה של הדורות והזמנים בסקירתם העליונה והקדושה }\מקור{[ע״א ג ב רסז]}\צהגדרה{.}

\paragraphs

\ערך{חרות }\הגדרה{- הרוח הנשאה שהאדם וכן העם בכלל מתרומם על ידה להיות נאמן להעצמיות הפנימית שלו, להתכונה הנפשית של צלם\hebrewmakaf אלהים\mycircle{°} אשר בקרבו, ובתכונה כזאת אפשר לו להרגיש את חייו בתור חיים מגמתיים, שהם שוים את ערכם }\מקור{[ע״ר ב רמה]}\צהגדרה{. }

\הגדרה{כשהעמיד האדם את שכלו במעלה כזאת עד שהוא הולך ועולה מאיליו במסילה ישרה העולה למעלה, וכן מדותיו הן כ״כ מזוקקות עד שכל מה שהן מתגברות הן מוסיפות טוב וישר, ואז אינו צריך לשום הדרכה מחוץ לעצמותו }\מקור{[עפ״י פנ׳ קכד]}\צהגדרה{.}

\משנה{חרות עצמית }\הגדרה{- חרות הגוף מכל שעבוד זר, מכל שעבוד הכופה את צלם אלהים אשר באדם להיות משועבד לכל כח אשר הוא מוריד את ערכו, את תפארת\mycircle{°} גדולתו\mycircle{°} והדרת\mycircle{°} קדושתו\mycircle{°}, הנקנית ע״י חרותה של הנשמה, חרות הרוח מכל מה שהוא מטה אותה ממסילתה הישרה והאיתנה היצוקה במהותו העצמית }\מקור{[שם רמד\hebrewmakaf ה]}\צהגדרה{. }

\משנה{חירות אמיתית }\הגדרה{- גאולה\mycircle{°} מכבלים עבדותיים\mycircle{°} }\מקור{[א״ק ג לה]}\צהגדרה{.}

\הגדרה{להתפתח על פי הטבע הפנימי בלא עירוב יסודות זרים המעיקים }\מקור{[ע״ר ב רפח]}\צהגדרה{. }

\הגדרה{מצב המתאים לנטיות הטבעיות שביצור ההוא, שיצאו אל הפועל השלם }\מקור{[עפ״י ע״א ג ה ז]}\צהגדרה{.}

\הגדרה{כשהאדם מתפשט בהארת נשמתו לכל הצדדים, לעומק\hebrewmakaf רום\mycircle{°} ועומק\hebrewmakaf תחת\mycircle{°}, עומק מזרח ועומק מערב, עומק צפון ועומק דרום ואינו מניח מקום בשכל ובהרגשה שלא יחדור שם במלוא חופשו\mycircle{°} }\מקור{[א״ה ב (מהדורת תשס״ב) ב 493]}\צהגדרה{.}

\משנה{יסוד החירות}\הגדרה{ - החיים על פי הטבע הטהור בלא שום מעיק ומכריח, לא העקה חומרית, לא מוסרית, ולא שכלית }\מקור{[עפ״י ל״ה 222]}\צהגדרה{.}

\משנה{עיקר החירות}\הגדרה{ - התגלות הצביון הטהור של הנטיה הפנימית }\מקור{[עפ״י ל״ה 222]}\צהגדרה{.}

\משנה{חרות עליונה }\הגדרה{- שחרור הרצון\mycircle{°}, חפש האופי. התכונה היותר עליונה\mycircle{°} הראויה להיות מתממת את כל העולם באור החרות העליון  }\מקור{[א״ק ג לה]}\צהגדרה{. }

\משנה{חרות גמורה }\הגדרה{- השתלמות שלטון רוח האדם על עצמו ועל העולם. ״אי בעו צדיקי ברו עלמא״\mycircle{°}}\myfootnote{ סנהדרין סה:\label{9}}\הגדרה{ }\צהגדרה{[מ״ר }\צמקור{35\hebrewmakaf 34}\צהגדרה{]. }

\משנה{חרות }\צהגדרה{- התגלות התביעה הפנימית במציאות המעשית}\צמקור{ [שי׳ מועדים ב 129].}

\הגדרה{ע״ע חפש. ע״ע דרור. ר׳ עבדות. ר׳ ״חרות עולם״. ע׳ במדור פסוקים ובטויי חז״ל, אין לך בן חורין אלא מי שעוסק בתלמוד תורה. }

\paragraphs

\ערך{חרות }\הגדרה{- }\משנה{אור החירות\hebrewmakaf העליון}\הגדרה{ - ההתודעות האלהית\mycircle{°} הבהירה, הבאה רק מתוך הכרה ואהבה\mycircle{°}, מתוך רוחב לב, מדושן עונג\mycircle{°} רוחני }\מקור{[עפ״י קובץ א תרצג]}\צהגדרה{. }

\paragraphs

\ערך{חרות }\הגדרה{- }\משנה{יסודה בישראל }\הגדרה{- ההכרה שכל דרך\hebrewmakaf ד׳\mycircle{°} כתורה היא תולדה נאמנה מכוונת לפי אמתת טבעיותנו הכללית, מצד כללות כנסת\hebrewmakaf ישראל\mycircle{°} }\מקור{[ע״ר ב רפח]}\צהגדרה{. }

\paragraphs

\ערך{חרות }\הגדרה{- }\משנה{״עלמא דחירו״ }\הגדרה{- ע׳ במדור מונחי קבלה ונסתר.  }

\paragraphs

\משנה{״חרות עולם״ }\צהגדרה{- חירות עולמית, כללית, קוסמולוגית, של שייכות לרבונו של עולם. החירות הפנימית }\צמקור{[שי׳ ב 38].}

\הגדרה{ע״ע חרות עליונה.}

\paragraphs

\ערך{חשבון הנפש }\הגדרה{- הלמוד של תורת המוסר\mycircle{°}, והבקורת העצמית של האדם }\מקור{[מ״ר 121]}\צהגדרה{. }

\הגדרה{ההשתדלות בינו לבין שכלו במה שיש עליו מהחובות }\מקור{[מ״א ד ב]}\צהגדרה{. }

\משנה{חשבון הנפש, מהותו }\הגדרה{- שיהיה האדם נושא ונותן בשכל על כל עניני נפשו וחיוב עבודתו לאלהים\mycircle{°} ית׳ }\מקור{[מ״א ד א]}\צהגדרה{. }

\paragraphs

\צמשנה{חשך }\הגדרה{- ע״ע חושך.}

\paragraphs

\ערך{חשק }\הגדרה{- }\משנה{(הדבקות\hebrewmakaf האלהית\mycircle{°} המתבטאת בשם חשק) }\הגדרה{- קשר מקיף\mycircle{°} שעומד מבחוץ (לנפש), ומקרין קרני זוהר\mycircle{°} ממרחק, עד שהוא מקיף את כל שעור\hebrewmakaf קומתו\mycircle{°} הרוחנית של האדם. <דומה להביטוי חשוקים, וחשוקיהם כסף, החשוק המקיף את העמוד. תוכן זה מתגלה ע״י מפעלי חיים ומעשים נבלטים, שעל ידם בא החבור של אור הקודש\mycircle{°} עם נפש האדם> }\מקור{[עפ״י ע״א ד יב לה]}\צהגדרה{. }

\משנה{החשק הפנימי }\הגדרה{- המחשבה\hebrewmakaf הרוחנית\mycircle{°}, הצמאון\hebrewmakaf האלהי\mycircle{°} }\מקור{[ע״א ד ט צ (ג״ר 93)]}\צהגדרה{. }\mylettertitle{ט}

\paragraphs

\ערך{טבע }\הגדרה{- }\משנה{הטבע }\הגדרה{- החיים והעולם }\מקור{[עפ״י א״א 113]}\צהגדרה{. }

\paragraphs

\ערך{טבע }\הגדרה{- }\משנה{הטבע }\הגדרה{- הגופניות\mycircle{°} והחברתיות }\מקור{[א״ק ב שיז]}\צהגדרה{. }

\paragraphs

\ערך{טבע }\הגדרה{- }\משנה{תביעות הטבע }\הגדרה{- הפשטות, הבריאות, הנורמאליות שבחיים, בהרגשה, בשכל, בהתפעלות }\מקור{[שם (ע״ט קיד)]}\צהגדרה{. }

\paragraphs

\תערך{טבע }\הגדרה{-}\תהגדרה{ החוקיות הכוללת את האדם והעולם. חוקיות החורזת את כל סדרי המציאות }\צמקור{-}\תהגדרה{ האדם, הדורות, ההויה וההסתוריה, שמים וארץ. חוקיות המארגנת את כל הברואים והיצורים לחטיבה אחת }\תמקור{[עפ״י ב״ר שצז]. }

\ערך{טבע }\הגדרה{- החק הכללי התדירי שבו חקק צור\hebrewmakaf עולמים\mycircle{°} ב״ה את עולמו לעדי\hebrewmakaf עד\mycircle{°} }\מקור{[ע״א ד ה ג]}\צהגדרה{.}

\הגדרה{אחיזת הסבות\mycircle{°} במסובבים, באופנים השונים ובתכניות הרבות שהנן עומדות ומסודרות, שוטפות והולכות }\מקור{[עפ״י א״ק ג כח]}\צהגדרה{. }

\הגדרה{קשרי הסיבות בעלילותיהן }\מקור{[קובץ ז לה]}\צהגדרה{. }

\משנה{מערכה טבעית }\הגדרה{- חוליות של שלשלת גדולה, ארוכה ומסודרת, שכל אחת נאחזת בחברתה }\מקור{[א״ק א קמד]}\צהגדרה{. }

\משנה{סדרי הטבע }\הגדרה{- השלשלאות שהן טבועות כטבעות אחוזות זו בזו באופן קבוע ובלתי נהרס ונדלג }\מקור{[ע״ר א מט]}\צהגדרה{. }

\משנה{התהלוכה עפ״י טבע }\הגדרה{- בלא השגחה\mycircle{°} פרטית }\מקור{[עפ״י מא״ה ג קפח]}\צהגדרה{. }

\משנה{טבע }\הגדרה{- מהלך ההשלמה לאט }\מקור{[ה׳ רכ]}\צהגדרה{. }

\הגדרה{הסדר וההדרגה המניחים מקום לאדם להיות פועל לא נפעל }\מקור{[עפ״י ע״א ג ב קצד]}\צהגדרה{. }

\תמשנה{טבע }\תהגדרה{- זרם מצד המשך ההויה, מהאמצעים מהסיבות\mycircle{°} }\תמקור{[עפ״י נ״א ה 31]. }

\תהגדרה{הסיבות והמסובבים}\תמקור{ [נ״א ה 29].}

\תערך{טבע }\תהגדרה{- }\תמשנה{(במובן המדעי הרגיל) }\תהגדרה{- חוקי ברזל, החוקיות העיוורת\mycircle{°} השולטת במציאות (שהקדושה\mycircle{°} לוחמת כנגדה) }\תמקור{[עפ״י ב״ר שצז]. }

\משנה{טבעיות }\צהגדרה{- היציבות המסודרת של התופעות }\צמקור{[ב״א 10]. }

\ערך{טבע }\הגדרה{- }\משנה{עולם הטבע }\הגדרה{- העולם\hebrewmakaf התחתיתי\mycircle{°}, המוגבל ומצומצם בטבעיותו }\מקור{[עפ״י א׳ כז\hebrewmakaf ח]}\צהגדרה{. }

\הגדרה{ע״ע נס, עולם הנס. ע׳ בנספחות, מדור מחקרים, הנהגת הטבע והנהגה נסיית. }

\paragraphs

\ערך{טבע הניסי }\הגדרה{- }\משנה{הטבע הניסי\mycircle{°} }\הגדרה{- אותם החוקים הניסיים והגדרים הנפלאים, המקשרים את עלילותיהם, והכל עשוי בעצה\mycircle{°} של החירות\hebrewmakaf העליונה\mycircle{°} שאין עמה שיעבוד והכרח }\מקור{[קובץ ז לה]}\צהגדרה{. }

\paragraphs

\ערך{טהור }\הגדרה{- }\משנה{במלא טהרתו }\הגדרה{- בתוכן המסולק מכל עכירות וזוהם\mycircle{°} רצוני משופל }\מקור{[ע״ר א ח]}\צהגדרה{. }

\הגדרה{בלא שום פניה גסה\mycircle{°} }\מקור{[קובץ א מז]}\צהגדרה{.}

\הגדרה{לא בתערובות תאוות הגופניות, חלאה וסיג}\צהגדרה{ }\מקור{[עפ״י פנק׳ ג רעז]}\צהגדרה{.}

\הגדרה{מנושא מכל נטיות החומר\mycircle{°} וזוהמותיו\mycircle{°} }\מקור{[עפ״י ע״ר א כח]}\צהגדרה{.}

\משנה{טהורה}\הגדרה{ - שאינה סופגת\mycircle{°} בקרבה את הזוהמא\mycircle{°} של ערלת\mycircle{°} הלב וטומאת\mycircle{°} יצרי האדם הרעים }\מקור{[פנק׳ א שסב]}\צהגדרה{.}

\הגדרה{ע׳ בנספחות, מדור מחקרים, טהרה וקדושה ההבדל בינהן, בהפרש שבין טהור לקדוש. }

\ערך{טהור }\הגדרה{- }\משנה{שם טהור }\הגדרה{- השגה\mycircle{°} צחה בלתי מעורבת בדמיונות שוא }\מקור{[לא׳ קפב]}\צהגדרה{. }

\הגדרה{ע״ע ״יראת ד׳ טהורה״. }

\paragraphs

\ערך{טהור }\הגדרה{- ע׳ בנספחות, מדור מחקרים, קדוש. טהור. טמא.}

\paragraphs

\ערך{טהרה }\הגדרה{- העתקה מן הרע והכיעור, מן הרפיון והשפלות }\מקור{[ע״א ד ט נא]}\צהגדרה{. }

\הגדרה{העלאה משפלות ומאפשרות עיוות לדרכי חשך }\מקור{[עפ״י שם ג ב נ]}\צהגדרה{. }

\הגדרה{טהרת הלב מתאוות רעות }\מקור{[עפ״י פנק׳ ג רעז]}\צהגדרה{. }

\הגדרה{הפעולה המזקקת ומעלה }\מקור{[קובץ ה רט]}\צהגדרה{. }

\הגדרה{שיהיה הלב טהור\mycircle{°} וכל פניותיו במעשה חובה ורשות לא יהיה כ״א לשם השי״ת }\מקור{[מ״ר 274]}\צהגדרה{. }

\משנה{מעלת הטהרה }\הגדרה{- שתהי׳ כל הגשמיות לכה״פ הכנה לקדושה\mycircle{°} }\מקור{[מא״ה ב שטו (ב״ר שמא)]}\צהגדרה{. }

\משנה{תכלית הטהרה }\הגדרה{- שלא יתערב אפילו משהו מנטיה לצד הרע\mycircle{°} והכעור מצד עצמו }\מקור{[עפ״י ע״ר א קלז\hebrewmakaf ח]}\צהגדרה{. }

\משנה{הטהרה האמיתית }\הגדרה{- קרבתו ית׳ והשגת רצונו }\מקור{[מא״ה ג רעו]}\צהגדרה{. }

\הגדרה{הקירוב בערך אור\hebrewmakaf פני\hebrewmakaf השי״ת\mycircle{°}, קרבתו\mycircle{°} ית׳ והשגת רצונו\mycircle{°} }\צמקור{[מ״ש שיט]}\הגדרה{.}

\הגדרה{ר׳ טוהר. ע׳ בנספחות, מדור מחקרים, טהרה וקדושה ההבדל בינהן. }

\paragraphs

\ערך{טהרה }\הגדרה{- }\משנה{הסבה של צורך הטהרה }\הגדרה{- ע״ע טומאה. }

\paragraphs

\ערך{טהרה }\הגדרה{- עצמיות הטוב. יסודו העליון\mycircle{°} של תוכן הטוב\mycircle{°} המתגלה בחיים }\מקור{[עפ״י א״ק ב תעז]}\צהגדרה{. }

\משנה{טהרה עליונה }\הגדרה{- שפעת\mycircle{°} החיים העליונים, המקודשים בקדושה\mycircle{°}  האצילית\mycircle{°}. שפעת החיים ההולכת ממקור\hebrewmakaf החיים\mycircle{°} מהטוב\hebrewmakaf העליון\mycircle{°}. שפעת עז\mycircle{°} החיים\hebrewmakaf החיים בטהרת\mycircle{°} טובם}\צהגדרה{ }\מקור{[עפ״י ע״ר א לד]}\צהגדרה{. }

\הגדרה{ע״ע טומאה. }

\paragraphs

\ערך{טהרה }\הגדרה{- }\משנה{(באדם) }\הגדרה{- שיהיה האדם טהור\mycircle{°} בכל דבריו ודרכיו ומחשבתו שלא יהיה בהם שום תערובת דבר שהוא נגד רצון\hebrewmakaf עליון\mycircle{°} ית״ש. <ואפי׳ אם לא יהיה לו חטא\mycircle{°} ועבירה\mycircle{°} בפועל רק שהוא נוטה לחטא ג״כ אין זה טהור> }\מקור{[מא״ה א קנ]}\צהגדרה{. }

\משנה{מדת הטהור }\הגדרה{- שימחו מקרב לבבו כל הנטיות של החטאים ותאוות חומריות }\מקור{[שם]}\צהגדרה{. }

\משנה{הטהרה הגופנית }\הגדרה{- }\משנה{(הטהרה) במזג ומדות }\הגדרה{- שקיטת התאוה וזיכוך החומר }\מקור{[א״ק א רמ]}\צהגדרה{. }

\משנה{טהרה }\הגדרה{- }\משנה{״הבא ליטהר״ }\הגדרה{- המתאים את הויית מציאותו עם המציאות הכללית שלכל ההויה כולה לכל הרום\mycircle{°} ולכל העומק שלה }\מקור{[עפ״י ע״א ד יב ל]}\צהגדרה{. }

\paragraphs

\ערך{טוב }\הגדרה{- הסדור הקיים בתנאיו המורכבים המתקבלים על רצונו ודעתו של אדם }\מקור{[ע״ר א קפט]}\צהגדרה{. }

\הגדרה{המושג המתאים}\צהגדרה{ }\מקור{[א״ק ב תנט]}\צהגדרה{.}

\ערך{טוב }\הגדרה{- המשכלל את הישות }\מקור{[ר״מ כב]}\צהגדרה{. }

\הגדרה{התכונה שהיא מכוננת לשכלולה של ההויה, לכללותיה ולפרטיותיה, לרוחה ולגויתה }\מקור{[א״ק ג קסג]}\צהגדרה{. }

\הגדרה{שאיפת בנין, קוממות, הארה\mycircle{°} והתרוממות\mycircle{°} }\מקור{[שם ב תעו]}\צהגדרה{. }

\הגדרה{מוביל לתכלית השלמות הנצחית\mycircle{°} הרצויה }\מקור{[עפ״י ע״ר א פ]}\צהגדרה{. }

\הגדרה{בנין העולם שכלולו ופארו }\מקור{[פנק׳ א שכט]}\צהגדרה{.}

\הגדרה{ע״ע רע. }

\paragraphs

\ערך{טוב}\הגדרה{ -  שמה שראוי להשפיע ולהתלבש באיברים ופעולות משפיע\mycircle{°} לזולתו }\מקור{[מא״ה ד קסג]}\צהגדרה{.}

\הגדרה{ע״ע ״רע״.}

\paragraphs

\ערך{טוב }\הגדרה{- }\משנה{הטוב האמיתי }\הגדרה{- הטוב\hebrewmakaf הכללי\mycircle{°}, חן\mycircle{°} השכל הטוב, המוסר\mycircle{°} והיושר\mycircle{°} האמיתי, שא״א שיבנה כי אם על ידי תיקון\hebrewmakaf עולם\hebrewmakaf במלכות\hebrewmakaf שדי\mycircle{°} }\מקור{[עפ״י ע״ר א שפו (ע״א ב ט רצ. א״ה 721)]}\צהגדרה{. }

\הגדרה{חפץ\mycircle{°} השם יתברך }\מקור{[מ״ה ברית א (פנ׳ ה)]}\צהגדרה{.}

\הגדרה{השגתו\mycircle{°} יתברך}\צהגדרה{ }\מקור{[מא״ה, ענייני תפילה, תקט]}\צהגדרה{.  }

\הגדרה{הטוב ההולם עם כל הכחות הפנימיים והחיצוניים של הנפש ועם כל ההויה כולה }\מקור{[קבצ׳ ב קח (פנק׳ ד רנ. מא״ה א (מהדורת תשס״ה) עג)]}\צהגדרה{.}

\paragraphs

\ערך{טוּב }\הגדרה{- }\משנה{הטוּב הגדול }\הגדרה{- הטוב העולה מהכללות המאוחדת של כל הפרטים, בין פרטי הזמן, בין פרטי המקום, בין פרטי המציאות בכללותה }\מקור{[ע״ר א צט]}\צהגדרה{. }

\משנה{״טוּב ד׳״ }\הגדרה{- המציאות כולה }\מקור{[ע״א ב ז לט]}\צהגדרה{.}

\משנה{״טוּב ד׳״ }\הגדרה{- מדת טובו המוחלטה, אשר לא תגור עמו כל רעה}\צהגדרה{ }\מקור{[ע״ר ב פח]}\צהגדרה{.}

\הגדרה{ע׳ במדור פסוקים ובטויי חז״ל, אעביר כל טובי על פניך. }

\paragraphs

\ערך{טוב }\הגדרה{- }\משנה{הטוב הכללי }\הגדרה{- הטוב\hebrewmakaf האלהי\mycircle{°} השורה בעולמות\mycircle{°} כולם, אור חֵי\hebrewmakaf העולמים\mycircle{°}. נשמת\hebrewmakaf כל, האצילית, בהודה וקדושתה }\מקור{[א״ש פרק ב]}\צהגדרה{.}

\הגדרה{כבוד\hebrewmakaf שם\hebrewmakaf ד׳\mycircle{°}, להועיל לכלל המציאות }\מקור{[ע״א ב ט שכא]}\צהגדרה{. }

\הגדרה{הטוב\hebrewmakaf המוחלט\mycircle{°} הסובב כל מצבי ההויה}\צהגדרה{ }\מקור{[מ״ר 4]}\צהגדרה{.}

\צהגדרה{הטוב האידיאלי, המתגלה בהשקפת האחדות\mycircle{°}, הוא הטוב\hebrewmakaf המוחלט\mycircle{°}, הנשאף בטוב המתעלה }\צמקור{[א״ק ב תנא]. }

\משנה{הטוב המוחלט }\הגדרה{- הטוב האלהי הנצחי שבמציאות כולה. הטוב אל הכל. הטוב האלהי הכללי. מלכות\hebrewmakaf שמים\mycircle{°} }\מקור{[א׳ נב]}\צהגדרה{. }

\צהגדרה{הטוב הכל\hebrewmakaf כללי, הנצחי, האינסופי, האלהי }\צמקור{[צ״צ כג]. }

\ערך{טוב אלהי }\הגדרה{- }\משנה{הטוב האלהי }\הגדרה{- עוצם היכולת\mycircle{°} ואדירות החפץ, להיטיב\mycircle{°}, להאיר\mycircle{°} ולהחיות }\מקור{[א״ק ג יא]}\צהגדרה{. }

\הגדרה{הטוב הגמור, היסוד של המגמה\mycircle{°} לכל התגלות אלהות\mycircle{°}, והשפעת התורה\mycircle{°} והנבואה\mycircle{°} בעולם }\מקור{[א״ק ב תקמד]}\צהגדרה{.}

\הגדרה{הטוב לכל }\מקור{[ע״ר ב רסד]}\צהגדרה{.}

\משנה{הטוב הגמור }\הגדרה{- עוצם הרצון לתיקון\mycircle{°}, לשכלול, להויה }\מקור{[א״ק ב תקכו]}\צהגדרה{. }

\תהגדרה{הטוב\hebrewmakaf העליון\mycircle{°} }\תמקור{[נ״א ה 30]. }

\מעוין{◊ }\צהגדרה{בעולם הגדול של הטוב המוחלט, ערכי ההבדלה שבין טוב לרע מתבטלים, מרב טובו ותגבורת שפעת חייו האדירים }\צמקור{[עפ״י א״ק ב תצט]. }

\ערך{טוב }\הגדרה{- }\משנה{״רוב טוּבך״}\myfootnote{ תהילים קמה ז.\label{1}}\הגדרה{ - הטוב הרב, הטוב\hebrewmakaf המוחלט\mycircle{°}. מלא החיים, בכל עז\mycircle{°} ואומץ, בקדושתו ושאיפת טובו, שרק רצון\mycircle{°} הבורא כל עולמים יעשה בעולמים כולם, ורק החפץ\mycircle{°} הכביר והקדוש העליון יהיה הרודה והדוחף את כל מערכות המעשים בכל היקום, בטבע, ובכל משטרי המציאות }\מקור{[ע״ר א קנט]}\צהגדרה{. }

\ערך{טוב העליון }\הגדרה{- זיו\hebrewmakaf העליון\mycircle{°} של אור פני המלך המתנשא לכל לראש מעל כל ענין העולמות\mycircle{°} }\מקור{[ע״ר א רפט]}\צהגדרה{.}

\הגדרה{החכמה\hebrewmakaf העליונה\mycircle{°} }\מקור{[ע״ר ב סא]}\צהגדרה{. }

\הגדרה{מקור\hebrewmakaf החיים\mycircle{°} }\מקור{[עפ״י ע״ר א לד]}\צהגדרה{. }

\משנה{תכלית הטוב }\הגדרה{- מקומו\hebrewmakaf של\hebrewmakaf עולם\mycircle{°}, החכמה העליונה השלמה }\מקור{[עפ״י מ״א א ו]}\צהגדרה{. }

\משנה{הטוב }\הגדרה{- התוכן של הקודש\mycircle{°} }\מקור{[א״ק א צ, ע״ר א שעב, א״ש ו ו]}\צהגדרה{. }

\הגדרה{השלמות העליונה שנשמת כל חי אליה עורגת }\מקור{[ע״ר ב עט]}\צהגדרה{. }

\הגדרה{האור\mycircle{°} של החסד\mycircle{°} הנאמן\mycircle{°}, מקור\hebrewmakaf החיים\mycircle{°} והשלום\mycircle{°}, האמת\mycircle{°}  והצדק\mycircle{°}, שכל ההפכים יושלמו אליו }\מקור{[א״ק ג קפג]}\צהגדרה{. }

\הגדרה{האלהות\mycircle{°} - }\מקור{[קבצ׳ ב עא]}\צהגדרה{.}

\משנה{זרם הטוב}\הגדרה{ - מקור האושר\mycircle{°}, הצדקה\mycircle{°} והישרנות, השלום והברכה\mycircle{°}}\צהגדרה{ }\מקור{[א״ק ג קפט]}\צהגדרה{.}

\הגדרה{הטוב\mycircle{°} שהוא למעלה מכל מדה ושיעור. התעלות\mycircle{°} עצמנו והתעלות כל העולם, כל היש, כל ההויה למעלת יסוד כל המרחבים, לאור\mycircle{°} כל האורות\mycircle{°}, לנחלה בלא מצרים }\מקור{[שם שמז]}\צהגדרה{. }

\משנה{הטוב האמיתי }\הגדרה{- אור\hebrewmakaf ד׳\mycircle{°} והשכלת אמיתו, נעם\mycircle{°} טובו, ענג\mycircle{°} יפעת\mycircle{°} תפארת\mycircle{°} גדולתו\mycircle{°} }\מקור{[ע״ר א קנז]}\צהגדרה{. }

\משנה{הטוב האידיאלי האלהי }\הגדרה{- מקור\hebrewmakaf ישראל\mycircle{°}, המקור שכל הטוב של המציאות נובע ממנו. המקור האמיתי של הצד האידיאלי ביסוד המוסרי שבקרב עם ישראל }\מקור{[עפ״י ע״ה קלה]}\צהגדרה{. }

\הגדרה{ע״ע טהרה. }

\paragraphs

\ערך{טוב }\הגדרה{- }\משנה{הטוב והיושר\mycircle{°}}\הגדרה{ - מרומי המגמות\mycircle{°} היותר נשאות, שהם הנם החפצים\mycircle{°} האלהיים\mycircle{°} }\מקור{[ע״ה קיב]}\צהגדרה{. }

\paragraphs

\ערך{טוב }\הגדרה{- }\משנה{כל טוב }\הגדרה{- חיים שלום\mycircle{°} ועדנים\mycircle{°} }\מקור{[א״ק ב שסט]}\צהגדרה{. }

\paragraphs

\ערך{טוב }\הגדרה{- }\משנה{״רב טוב״ }\הגדרה{- ע״ע טוב אלהי, ״רוב טובך״. }

\paragraphs

\ערך{טוּב ד׳ }\הגדרה{- ע׳ טוב, הטוב הגדול. }

\paragraphs

\ערך{טוב לב }\הגדרה{- }\מעוין{◊}\הגדרה{ טוב הלב יבוא כשהנטיות הטבעיות הן קולעות את מטרתן }\מקור{[ע״א ג ב נ]}\צהגדרה{. }

\הגדרה{ע״ע לב טוב.}

\paragraphs

\משנה{טוהר }\צהגדרה{- בהירותה של הערפליות וחסרון לקויה }\צמקור{[צ״צ א עט]. }

\paragraphs

\ערך{טומאה }\הגדרה{- עצמיות הרע\mycircle{°}. יסודו העליון של תוכן הרע המתגלה בחיים }\מקור{[עפ״י א״ק ב תעז]}\צהגדרה{. }

\הגדרה{הרשעה\mycircle{°} הסרה בעיקר עומק מהותה משכלול הטוב\mycircle{°} }\מקור{[עפ״י שם תעה]}\צהגדרה{. }

\הגדרה{רגש יחש מציאות הרע והבחנתו }\מקור{[ע״א ג ב עח]}\צהגדרה{. }

\הגדרה{הריחוק בערך אור פני השי״ת }\מקור{[מ״ש שיט]}\צהגדרה{.}

\הגדרה{ע״ע טהרה. ע״ע טמא. }

\paragraphs

\ערך{טל }\הגדרה{- }\משנה{(לעומת גשם\mycircle{°}) }\הגדרה{- ההשפעה\mycircle{°} שבאה מצד חסד\hebrewmakaf עליון\mycircle{°} לבדו, מכוון כנגד קרבת\mycircle{°} השי״ת הבאה ע״י הכנעת\mycircle{°} הלב, שע״י שהוא מכניע את לבו השי״ת מרחם\mycircle{°} עליו, ובלא השתדלותו מתקרבת אליו השגחתו\mycircle{°} ית׳ ואור\hebrewmakaf פניו\mycircle{°} }\מקור{[עפ״י מא״ה א קנט, מ״ש נט]}\צהגדרה{. }

\הגדרה{שפע קדושת הנפש, שאינו נמנע לעולם וטוב לכל כטל }\צהגדרה{[עפ״י פנק׳ ה קיט]}

\paragraphs

\ערך{טל חיים }\הגדרה{- }\משנה{טל החיים היסודי }\הגדרה{- הרצון\mycircle{°} וההרגשה, ההכרה והתשוקה ממהותה הפנימית\mycircle{°} של הנשמה\mycircle{°} }\מקור{[א״ק ג קלז]}\צהגדרה{. }

\הגדרה{חכמה ורעננות\mycircle{°} של קדושה\mycircle{°}}\צהגדרה{ }\מקור{[פנק׳ א תלז]}\צהגדרה{.}

\paragraphs

\ערך{טללי אורות\mycircle{°}}\הגדרה{ - חיי קודש\mycircle{°} וטוהר\mycircle{°} }\מקור{[א״ק ב שכט]}\צהגדרה{. }

\paragraphs

\ערך{טללי חיים }\הגדרה{- לשדי אורות\mycircle{°} }\מקור{[א״ק ג שמג]}\צהגדרה{. }

\paragraphs

\ערך{טמא }\הגדרה{- שהמהותיות העצמית שלו הוא רע }\מקור{[קובץ ח א]}\צהגדרה{. }

\הגדרה{ע״ע טומאה. }

\paragraphs

\ערך{טמא }\הגדרה{- ע׳ בנספחות, מדור מחקרים, קדוש. טהור. טמא.}

\paragraphs

\ערך{טמטום}\myfootnote{ \textbf{החיים, התביעה השלמה שלהם} - ע״ע חיים, תביעתם השלמה.  \label{2}}\הגדרה{  - כשנעשה האדם מדרס לצד העור\mycircle{°} שבו, לאותו הצד החלש, שאינו יכול להתגבר עד כדי האומץ של אחיזת החיים\mycircle{°} בסדורם כפי התביעה השלמה שלהם }\מקור{[א״ש י ה]}\צהגדרה{. }

\משנה{טמטום הלב }\הגדרה{- איבוד הרושם של ציור\mycircle{°} החיים המיוחדים לישראל, טשטוש הקדושה הישראלית, (עד ש)שוב אין כחות הנפש מתקשרים בתכונה ישראלית, ותוכן החיים הפנימיים של האומה, בעומק הרגשותיה, איננו נכנס בתוך (הלב). פגימת הנשמה }\מקור{[עפ״י קובץ א קסא]}\צהגדרה{. }

\הגדרה{שלא יכנס אור הדעת שבמוח אל תוך הלב וזה גורם ממילא קלקול המדות }\מקור{[פנק׳ ג טז (מא״ה ב רסה)]}\צהגדרה{. }

\משנה{טמטום הלב המרגיש והמח החושב}\צהגדרה{ - }\הגדרה{(שלא) להיות עלול להרגיש עדינות ולחשב גדולות\mycircle{°} ואמתיות}\צהגדרה{ }\מקור{[עפ״י קבצ׳ ג קלג (קבצ׳ ב קמז)]}\צהגדרה{.}

\הגדרה{ע׳ במדור פסוקים ובטויי חז״ל, עברה מטמטמת לבו של אדם.}

\paragraphs

\ערך{טמיר }\הגדרה{- ספון וגנוז }\מקור{[פנק׳ ב רט]}\צהגדרה{.}

\paragraphs

\ערך{טעימה }\הגדרה{- }\משנה{(לעומת אכילה\mycircle{°})}\הגדרה{ - תתיחס לההנאה של שעת העסק באכילה, ההנאה בעת האוכל }\מקור{[ע״א ג א לה]}\צהגדרה{.}

\paragraphs

\ערך{טעם }\הגדרה{- }\משנה{(במובנו הרוחני\mycircle{°} לעומת ריח\mycircle{°}) }\הגדרה{- מסמן את השאיפה האידיאלית\mycircle{°} לפי ערכה אחרי ההתגשמות המעשית שבפועל }\מקור{[עפ״י מ״ר 237]}\צהגדרה{. }\mylettertitle{י}

\paragraphs

\ערך{ידיעה }\הגדרה{- רעיון\mycircle{°} ומחשבה, המכריעים את דרכי החיים כולם }\מקור{[עפ״י ע״ר א קיד]}\צהגדרה{. }

\הגדרה{המסקנא המוסרית\mycircle{°} המתגלמת }\מקור{[קובץ ד מו]}\צהגדרה{. }

\ערך{ידיעה }\הגדרה{- }\משנה{(מדד איכותה) }\הגדרה{- עפ״י עומק ההבנה וחריפות השימוש בה לכל חפץ }\מקור{[עפ״י ע״א ב ט שמד]}\צהגדרה{. }

\paragraphs

\ערך{ידיעה }\הגדרה{- }\משנה{הידיעה המוחלטת (האלהית, לעומת צפיה אלהית) }\הגדרה{- ע׳ במדור השגחה.}

\paragraphs

\ערך{ידיעת האלהות }\הגדרה{- }\משנה{ידיעת האלהות הטהורה בעולם }\הגדרה{- אותה ההכרה השלמה, שגוררת אחריה ועמה כל המעשים הטובים בכלל האנושיות, להאיר את העולם כולו באור האמת\mycircle{°} והשלום\mycircle{°} }\מקור{[ע״ר א שפו]}\צהגדרה{.}

\paragraphs

\ערך{ידיעת שם ד׳ }\הגדרה{- התנשאות אל רוממות התכלית הכללית }\צהגדרה{[עפ״י ע״א ב }\צמקור{211\hebrewmakaf 210}\צהגדרה{]. }

\מעוין{◊ }\משנה{ידיעת השם ית׳ כוללת שתי אלה }\הגדרה{- הידיעה האמיתית הדרושה להישיר המעשים ולרוממם. והשנית - הידיעה שהיא לרומם הנפש עצמה בידיעות האמת והדרת\mycircle{°} נעם\mycircle{°} הנשגבות\mycircle{°} בדעת\hebrewmakaf האלקים }\מקור{[ע״א א ב ה]}\צהגדרה{. }

\הגדרה{ע׳ במדור פסוקים ובטויי חז״ל, דעת אלהים. ושם, יודעי שמך. ע״ע דעת ד׳ העליונה. }

\paragraphs

\ערך{יהדות }\הגדרה{- הוקרת האמונה\mycircle{°} ושמירת התורה\mycircle{°} והדת\mycircle{°} בכנסת\hebrewmakaf ישראל\mycircle{°} ביתרון של קדש\mycircle{°} עליון }\מקור{[מ״ר 30]}\צהגדרה{. }

\הגדרה{ארחות החיים הנובעים ממקור האחדות של יחיד\mycircle{°} חֵי\hebrewmakaf העולמים\mycircle{°} }\מקור{[אג׳ א קעה]}\צהגדרה{. }

\תמקור{האידיאל\mycircle{°} של האומה\mycircle{°} כולה. הטבע הנפשי, שמשתף את הרצונות להרגשה אלהית\mycircle{°} קרובה וגם נקיה וטהורה\mycircle{°} [מ״ר 457, 493]. }

\הגדרה{התמצית הפנימית\mycircle{°} של האנושיות ושל המציאות במובן הרחב, מצד תוכנה הפנימי של היהדות ומקורה שהוא כנסת\hebrewmakaf ישראל\mycircle{°} }\מקור{[עפ״י אג׳ א קעה, ב סה, א׳ קמב]}\צהגדרה{. }

\ערך{יהדות }\הגדרה{- }\משנה{מגמתה }\הגדרה{- ההארה של האלהות\mycircle{°} בצורה היותר טהורה\mycircle{°} ומאירה בקרבה פנימה ובעולם ובחיים בכלל, והשלמת טפוסה הלאומי הזה עפ״י רוחה העצמי בארצה ההסתורית }\מקור{[א׳ נ]}\צהגדרה{. }

\משנה{דמותה העליונה של היהדות}\צהגדרה{ - גלוי ערכיה הנצחיים והקדושים של האומה הישראלית }\צמקור{[ל״י א (מהדורת בית אל תשס״ב) כה].}

\הגדרה{ע״ע רוח הזמן, רוח הזמן האמיתי של היהדות.}

\paragraphs

\ערך{יהדות טבעית }\הגדרה{- ההוד\mycircle{°} שבקומתה הרוחנית\mycircle{°} של היהדות\mycircle{°}, הצורה הטבעית\mycircle{°} אשר לכנסת\hebrewmakaf ישראל\mycircle{°} בהסתעפותה }\מקור{[עפ״י א׳ קמב]}\צהגדרה{. }

\הגדרה{הדבקות בכנסת\hebrewmakaf ישראל ובעול מלכות שמים ברגש אמונה\mycircle{°} שלמה ואהבה\mycircle{°} טבעית נאמנה\mycircle{°} של עם ד׳\mycircle{°}, <שעל זה נאמר ״העמוסים מיני בטן הנשואים מיני רחם״> }\מקור{[עפ״י ע״א ב 235]}\צהגדרה{. }

\הגדרה{ע״ע יהדות ניסית. ע״ע אמונה, היסוד הטבעי של האמונה. ע׳ במדור פסוקים ובטויי חז״ל, עמוסי בטן. ושם, ירח (בעולם השכלי וכו׳). ע׳ בנספחות, מדור מחקרים, כנסת ישראל ניסית וטבעית וכו׳. }

\paragraphs

\ערך{יהדות ניסית }\הגדרה{- הנצח\mycircle{°} שבקומתה הרוחנית\mycircle{°} של היהדות\mycircle{°}, הניסיות שבה והלך מחשבותיה, עומק רגשותיה וחותם המעשי שלה. הצורה הניסית\mycircle{°} אשר לכנסת\hebrewmakaf ישראל\mycircle{°} בהסתעפותה }\מקור{[עפ״י א׳ קמב]}\צהגדרה{. }

\הגדרה{ע״ע יהדות טבעית. ע״ע אמונה, היסוד הטבעי של האמונה. ע׳ בנספחות, מדור מחקרים, כנסת ישראל ניסית וטבעית וכו׳. }

\paragraphs

\ערך{יום }\הגדרה{- }\משנה{היום }\הגדרה{- זמן המוכשר לחיי החברה, והתפזרות החיים ברעות וידידות קבוצית }\מקור{[ע״ר א קעב]}\צהגדרה{. }

\הגדרה{ע״ע לילה. }

\paragraphs

\ערך{יום }\הגדרה{- }\משנה{״מדת יום״ }\הגדרה{- ע׳ במדור פסוקים ובטויי חז״ל.    }

\paragraphs

\ערך{יום הגדול }\הגדרה{- }\משנה{היום הגדול }\הגדרה{- היום אשר כל הימים השוטפים ועוברים אליו ישובו ועמו יתאחדו בעלייתו יתעלו וירוממו ברומו }\מקור{[ר״מ קפג]}\צהגדרה{. }

\paragraphs

\ערך{יופי }\הגדרה{- הסדר המתוקן}\myfootnote{ \textbf{יופי, הסדר המתוקן} - על היופי\hebrewmakaf העליון כותב רמח״ל בקל״ח פתחי חכמה, פתח נח: ״יופי הוא הסידור הנאה של התיקונים, שאין שום עירבוב של חשך, אלא הכל מסודר בדרך הנהגה שלמה, לפי התיקונים הצריכים להשפעת הטוב״.\label{1}}\הגדרה{ }\מקור{[ע״א ב ו מו (מ״ר 212)]}\צהגדרה{. }

\משנה{משפט היופי }\הגדרה{- קבוץ חוקים נפרדים אל תוך מערכה אחת וסדור מקביל }\מקור{[שם]}\צהגדרה{. }

\משנה{הערכת היופי }\הגדרה{- חק הדברים המתאימים זה לזה כשהם יחד, שבהם תוכן ההדר והפאר\mycircle{°} }\מקור{[עפ״י ע״ר א קט]}\צהגדרה{. }

\הגדרה{ע״ע נוי. ע׳ בנספחות, מדור מחקרים, נוי לעומת יופי. }

\paragraphs

\ערך{יופי עליון }\הגדרה{- }\משנה{היופי העליון }\הגדרה{- התפארת\hebrewmakaf האלהית\mycircle{°} }\מקור{[קובץ א תרו]}\צהגדרה{. }

\paragraphs

\ערך{יופי שכלי }\הגדרה{- }\משנה{היופי השכלי }\הגדרה{- הצדקה\mycircle{°} והמשרים\mycircle{°}, מעשים טובים ומדות טובות }\מקור{[ע״א ב ט רסט]}\צהגדרה{.}

\paragraphs

\ערך{יוצר}\הגדרה{ - }\משנה{(כדוגמא של מעלה) }\הגדרה{- פועל בציורו\mycircle{°} על כל היקום }\מקור{[עפ״י א״ק ג סח]}\צהגדרה{.}

\paragraphs

\ערך{יושר }\הגדרה{- האור\mycircle{°} והצדק\mycircle{°} }\מקור{[עפ״י ע״ר א קעו]}\צהגדרה{. }

\הגדרה{המוסר\mycircle{°} השלם ודעת\hebrewmakaf ד׳\mycircle{°} וכל המדות העליונות הנמשכות מהם }\מקור{[ע״א ג ב צא]}\צהגדרה{. }

\הגדרה{להיות הנפש דבקה בעצם טבעה בשיווי קדושת התורה }\מקור{[עפ״י ע״א א ה פח]}\צהגדרה{. }

\משנה{יושר האמיתי}\הגדרה{ - העצמיות והקדושה\mycircle{°} בבירורה }\מקור{[קובץ ו רסט]}\צהגדרה{.}

\משנה{מרום היושר }\הגדרה{- הקושט והאור, הקודש\mycircle{°} והטוב\mycircle{°}, להיות מבלט את החותם הנצחי, את האמת, חותמו של הקב״ה }\מקור{[ר״מ כו]}\צהגדרה{. }

\משנה{יושר וצדק }\הגדרה{- הסדר הטוב\mycircle{°} והמעולה המסבב טוב לכל, שהוא קשור יפה באמתת המציאות וברעיון הלב ביחש המעולה שלפי גודלו ותפארתו אנו קוראים אותו ומתגלה בו חפץ\hebrewmakaf ד׳\mycircle{°} }\מקור{[מ״ה אהבה ג]}\צהגדרה{. }

\הגדרה{ע׳ במדור מדרגות והערכות אישיותיות, ישר. ע״ע טוב, הטוב והיושר. }

\צמשנה{יושר פנימי }\צהגדרה{- }\מעוין{◊ }\צהגדרה{אדם מישראל ימצא אותו לפי אותה המדה שאורה\hebrewmakaf של\hebrewmakaf תורה\mycircle{°} במעשה ובעיון התרחבה בלבבו, ולפי הערך שהמדות הטובות נקבעו עמוק בנפשו }\צמקור{[אג׳ א כט]. }

\paragraphs

\ערך{יחוד ד׳ }\הגדרה{-}\משנה{ דעת ד׳ ויחודו }\הגדרה{- }\מעוין{◊ }\הגדרה{היא מצד הנשמה\mycircle{°} בלבד }\מקור{[פ״א קעח]}\צהגדרה{.}

\paragraphs

\ערך{יחוד ד׳ בעולם }\הגדרה{- ליחד כל הברואים שיעשו כולם אגודה אחת, לעשות רצונו בלבב שלם }\מקור{[עפ״י ע״ר א קנח]}\צהגדרה{. }

\paragraphs

\ערך{יחידוּת אלהית }\הגדרה{- }\משנה{היחידות המוחלטה }\הגדרה{- }\מעוין{◊}\הגדרה{ היחידות המוחלטה למעלה מכל ציור\mycircle{°} של עולם\mycircle{°}, אין כח בשום שפה לבטאה, ולא בשום שכל ולב להשכילה, אבל מתוך האחדות\hebrewmakaf המוחלטה\mycircle{°}, שהיא מבארת את היחידות\hebrewmakaf האלהית\hebrewmakaf בעולם\mycircle{°}, שברב זהרה היא באה לנו ע״י אור\hebrewmakaf התורה\mycircle{°} דוקא, אנו באים לקצת הכרה בערך היחידות המוחלטה, שאינה נהגת בשום אופן, מאפס ציור ודוגמא אותיותית איך לתארה }\מקור{[ע״ר א כה]}\צהגדרה{. }

\הגדרה{ע׳ במדור שמות כינויים ותארים אלהיים, ״יחיד״. ע׳ בנספחות, מדור מחקרים, אחדות ויחוד. }

\paragraphs

\ערך{יחס }\הגדרה{- }\משנה{יחוס בשלשלת שלו, יסודו }\הגדרה{- הנושא הפנימי של הכוחות הטובים המתנחלים והולכים, המשתלשלים מראשי הדורות עד תולדותיהם. <ובכל אדם גנוז בקרבו אוצר הטוב לתולדותיו, כל זמן שלא יחלל זרעו ולא ישפיל את הוד תעודת האוצר אשר אור חיי עולמים בתוכו, ברוח החיים וקדושת כל הטוב, (שאז) כל הנחלה של רכוש המדות הטובות והידיעות היקרות שנמזגו במשך הזמן בדם ובבשר של הדורות ונעשו לנחלה בטוחה וקבועה, (ו)הבאים אחריהם מתחילים את בנינם מאותו המקום שכבר עלו עליו דורותיהם הקודמים והולכים למעלה למשכיל> }\מקור{[עפ״י ע״א ד ה נד]}\צהגדרה{. }

\paragraphs

\ערך{יכולת }\הגדרה{- }\משנה{(לעומת רצון}\הגדרה{\mycircle{°}, }\צהגדרה{חפץ\mycircle{°})}\הגדרה{ - החכמה והכשרון }\מקור{[א״י לו]}\צהגדרה{. }

\משנה{שלמות היכולת }\הגדרה{- כשיוציא האדם אל הפועל את כשרונו המדעי בכל חכמה עיונית ומעשית וכל מלאכה, תשתלם }\צהגדרה{יכלתו }\מקור{[עפ״י ל״ה 191]}\צהגדרה{. }

\paragraphs

\ערך{יכולת עליונה }\הגדרה{- }\משנה{היכולת העליונה }\הגדרה{- המציאות היותר נבחרה, יותר פאורה, יותר חזקה ויותר אמיתית. היכולת האין סופית, כלל כל האפשרויות האלהיות, שאין להן קץ ותכלית. הים האוירי\hebrewmakaf העליון}\myfootnote{ \textbf{הים }\textbf{האוירי}\textbf{ העליון} - בתפילת היחוד לר׳ נחוניא בן הקנה, המובאת בתחילת הלשב״ו הקדו״ש ״ד׳ שדי הטהור בטוהר המציאות ואדיר באחדות השוה המתעלה והמתרומם באויר הקדמון המתמצע לז ללז בקודש הדר נועם טובך״ וכו׳ ע״ש. ובס״י פ״ב מ״ו ״וחצב עמודים גדולים מאויר שאינו נתפס״. ובז״ח שיר השירים ע: ״ובא אליה מלך גדול, דא את ל׳, דאיהי מגדלא דפרח באוירא דכיא דאתפס. בגין דאית אוירא אחרא דלא אתפס כלל, ולא ידיע״. ובעמק המלך שער א, פרק נג: ״אוירא דאין סוף, דהיינו העצמות״. ושם שער ב, פרק ב, ובנובלות חכמה ליש״ר מקנדיא עלה יז: ״המקום שנתפנה מעליית חצי המלבוש נקרא ׳אויר קדמון׳ ו׳רשימו׳. ואמר שהמקום שנעשה מפניית חצי המלבוש נקרא ׳אויר קדמון׳ ובספר הזוהר ׳אוירא קדמאה׳, ונקרא גם כן ׳רשימו׳ בפי חכמי הקבלה, שבאויר זה נשאר רושם הן מאור עצמות הא״ס שנסתלק, הן מאור הלבוש שנתצמצם ונתעלה. ובתוך האויר קדמון יש אדם קדמון״. וביונת אלם, פרק א ״טהירו עילאה בגי׳ שמ״ו... ונקרא אויר קדמון והוא נפש העולם״. ובביאורים לס׳ אוצרות חיים לרמח״ל, סי׳ ג: ״חלל נקרא מה שהוסר ממנו הבלתי תכליות. והרשימו הנשאר הוא אויר פנוי, ואין חלל בלא רשימו, כדלקמן. אך בבחינת מה שנסתלק נקרא חלל, ובבחינה שנשאר נקרא רשימו, והיינו אויר קדמון, כאויר שבין עצם לעצם בעה״ז״. ובליקוטי הגר״א סוף ספדצ״נ, לח: (מהד׳ שערי זוהר) ד״ה וכ״א ״אוירא הוא בלא גבול״. וברד״ל פדר״א ד ס״ק טז ״האויר הקדמון הכתר״. ובלקוטי תורה לרש״ז, שיר השירים, דף ב עמודה ד, ודף ג עמודה א: ״אויר שהוא מקור האור וכו׳״. ובלשב״ו הקדו״ש, שער ה פרק א ״הצמצום הוא כעין אויר ומקום פנוי אשר בתוכו נמצאו ונעשו כל העולמות כולם ולכן נקרא אויר הקדמון״. ע״ע במדור מונחי קבלה ונסתר, טהירו עילאה, ובהערה שם.\label{2}}\הגדרה{ של כל האפשרויות האין\hebrewmakaf סופיות }\מקור{[עפ״י א״ק ג כח, ע״ר א מז]}\צהגדרה{. }

\הגדרה{ע״ע אדנות מוחלטה. }

\paragraphs

\ערך{ימין }\הגדרה{- }\משנה{(לעומת שמאל\mycircle{°})}\הגדרה{ - מתיחש יחוש של ערך ורושם אל הענינים הרוחניים\mycircle{°}, }\צהגדרה{<מול הענינים הגשמיים שראוי להכירם בתור חלושים מהם, כערך הימין אל השמאל>.}\הגדרה{ צד השכל והמעלה }\מקור{[ע״א ב ט רלח, רלט (ח״פ לח:)]}\צהגדרה{. }

\משנה{כח הימין}\myfootnote{ \textbf{מורה על יתרונות} - מתאים לדברי הרלב״ג יהושע א ז, ושם כג ב. (הערת הרב שלום הימן).\label{3}}\הגדרה{ - מורה על יתרונות }\מקור{[פנ׳ יא]}\צהגדרה{. }

\משנה{ימין}\הגדרה{ - הנטיה התכליתית של כללות המטרה של החיים. המטרה העליונה והמסומנת שהחיים מתאימים לה הניכרת ביותר, ע״פ היסוד האלהי בהם, בצד הימין, שעילוי כח החיים ניכר בו}\צהגדרה{ }\מקור{[עפ״י ע״א ד ו טו]}\צהגדרה{.}

\הגדרה{המטרות העקריות }\מקור{[ע״ר א נו]}\צהגדרה{.}

\הגדרה{המגמה העיקרית }\מקור{[עפ״י ע״א ד ו נו]}\צהגדרה{.}

\הגדרה{המגמות העליונות }\מקור{[ע״א ד ו נח]}\צהגדרה{. }

\הגדרה{הצד החשוב }\מקור{[עפ״י מ״ש נז]}\צהגדרה{.}

\הגדרה{העיקר }\מקור{[ע״א ד ו יז]}\צהגדרה{.}

\הגדרה{הכח הפועל, כערך צורה\mycircle{°} }\מקור{[עפ״י ע״א ב ט רמ]}\צהגדרה{.  }

\הגדרה{הפועל, המסדר, הבונה והעוסק התדירי. מקום הרחמים והחסד, הגבורה והאמונה האלמותית }\מקור{[עפ״י קובץ ז רג]}\צהגדרה{.     }

\הגדרה{ע׳ במדור גוף האדם אבריו ותנועותיו, יד ימין. ושם, יד שמאל. ושם, הצד הימני שבאדם. ושם, הצד השמאלי שבאדם. ע׳ במדור תיאורים האלהיים, ימין. ע׳ במדור תורה, ״מיימין בתורה״. ושם, ״ימינה של תורה״.}

\paragraphs

\ערך{ימין }\הגדרה{- }\משנה{מַימין }\הגדרה{- יותר מתגלה ברב עצמה }\מקור{[עפ״י ר״מ ג]}\צהגדרה{. }

\משנה{פונה אל הימין }\הגדרה{- להרבות עצמת חיים חיוביים\mycircle{°} ועדונים מסומנים }\מקור{[עפ״י ר״מ סז]}\צהגדרה{. }

\משנה{להימין }\הגדרה{- לצד הזיכוך והעילוי }\מקור{[קבצ׳ ב עו (פנק׳ ד קעט)]}\צהגדרה{.}

\משנה{ההויה בתחתית ימינה }\הגדרה{- תחתית הצד היותר בהיר שבהשגות היותר קטנות וקרובות }\מקור{[ר״מ קנב]}\צהגדרה{. }

\הגדרה{ע״ע שמאל, להשמאיל.}

\paragraphs

\ערך{ימין ושמאל }\הגדרה{- הקודש\mycircle{°} והחול\mycircle{°} }\מקור{[עפ״י מ״ר 45]}\צהגדרה{.}

\paragraphs

\ערך{ימין ושמאל }\הגדרה{- רחמים\mycircle{°} ודין\mycircle{°} }\מקור{[פנק׳ ג קפו]}\צהגדרה{.}

\paragraphs

\ערך{ימין }\הגדרה{-}\משנה{ הליכה מימין לשמאל }\הגדרה{- הכח השואב ממעינות הקודש\mycircle{°} ומשקה את מטעי החול\mycircle{°}. הנטיה משפעת הקודש אל התבונה הקצובה, המאירה באדם, ברוחו ושכלו, בציוריו ובכח יצירתו. כשמסבירים עמקי הקודש בדעת השכל, בראית התבונה האנושית. <}\צהגדרה{מדתו של אהליאב}\הגדרה{> }\מקור{[א״ק א סט]}\צהגדרה{.}

\הגדרה{ע״ע שמאל, הליכה משמאל לימין.}

\paragraphs

\ערך{יסוד }\הגדרה{- }\משנה{יסוד העולם }\הגדרה{- השלום\mycircle{°} החברתי, ועוצם הטוב להשלמת הנשמות בחיי עולם }\מקור{[א״א 65]}\צהגדרה{. }

\paragraphs

\ערך{יסוד הכללי }\הגדרה{- }\משנה{היסוד הכללי }\הגדרה{- ע״ע כללי. }

\paragraphs

\ערך{יפעה }\הגדרה{- יפי הופעה\mycircle{°} }\צהגדרה{[רצי״ה א״ש יב הערה }\צמקור{24}\צהגדרה{]. }

\paragraphs

\ערך{יצירה }\הגדרה{- }\משנה{היצירה }\הגדרה{- ההתהוות\mycircle{°} העולמית\mycircle{°} }\מקור{[א״ק ב תקלב]}\צהגדרה{. }

\הגדרה{ההויה בכללותה }\מקור{[ע״ר א רכא]}\צהגדרה{.}

\paragraphs

\ערך{יצירה}\הגדרה{ - }\משנה{(כדוגמא של מעלה) }\הגדרה{- פעולה בציור\mycircle{°} על כל היקום }\מקור{[עפ״י א״ק ג סח]}\צהגדרה{.}

\ערך{יצירה }\הגדרה{- הנתקת האדם מהעולם העכור, שכחות החיים החומריים כולם כל כך מושרשים בו, אל עולם האצילות והטוהר, ששם הנשמה האנושית היא אזרחית ופועלת בחופש כחותיה }\מקור{[א״ק א קצה]}\צהגדרה{. }

\הגדרה{הסתכלות בבהירות הציורים\mycircle{°} של רוממות\mycircle{°} מוסרית\mycircle{°}, ששם מעורבים הם יחד תאורי המוסר, המדע, וההשקפות העדינות שבמרומי הקודש\mycircle{°} }\מקור{[עפ״י א״ש יג י]}\צהגדרה{. }

\paragraphs

\משנה{יצירה }\צהגדרה{- כוח אלקי, המחבר ומקשר בין חלקים שונים הקיימים במציאות והופכם לגוף אחד }\צמקור{[ק״ת עז].}

\הגדרה{ר׳ בריאה. ר׳ עשיה.}

\paragraphs

\ערך{יצירה }\הגדרה{- }\משנה{עולם היצירה }\הגדרה{- ע׳ במדור מונחי קבלה ונסתר.  }

\paragraphs

\ערך{יקום }\הגדרה{- היצירה\mycircle{°} כולה בקישוריה }\מקור{[ע׳׳א ד ט עא]}\צהגדרה{.}

\paragraphs

\ערך{יראה }\הגדרה{- המוסריות במעשה וברגש, המאגדת את כל הפרטים אל הכלל והכלל אליהם }\מקור{[ע״א ב 288]}\צהגדרה{. }

\הגדרה{הטבת המעשים בהנהגה }\מקור{[עפ״י מ״ר 395]}\צהגדרה{. }

\משנה{יראת ד׳\mycircle{°}}\הגדרה{ - ״ראשית\mycircle{°} דעת״, כלל כל דרך\hebrewmakaf ד׳\mycircle{°}, המתאים עם יושר\mycircle{°} השכל והמוסר\mycircle{°} הטוב }\מקור{[עפ״י א״ת ט ה (פנק׳ א רנג)]}\צהגדרה{. }

\הגדרה{האמונה בהשפעתה המעשית, התוכן האצילי\mycircle{°}, המקודש, האלהי\mycircle{°}, של האומה (והאדם) }\צהגדרה{[עפ״י א״א }\צמקור{81, 90, 95, 138}\צהגדרה{ א״ש טו יא]. }

\משנה{יראת ד׳ מטרתה }\הגדרה{- לתקן כל דרכיו על דרך התיקון היותר שלם ויותר נאה }\מקור{[ע״א א ב ס]}\צהגדרה{. }

\הגדרה{שמירת העמדה על יסוד הטוב\mycircle{°} וישר\mycircle{°} }\מקור{[עפ״י א״ת ב 78]}\צהגדרה{.}

\הגדרה{ע״ע יראת שמים. ע״ע ״יראת שמים בגלוי״. ע׳ במדור פסוקים ובטויי חז״ל, והאלהים עשה שיראו מלפניו. }

\paragraphs

\ערך{יראה חיצונית }\הגדרה{- יראה\hebrewmakaf תתאה\mycircle{°} לבדה מצד עונשי עוה״ז או עונשי עוה״ב }\מקור{[אג׳ ב קפז]}\צהגדרה{.}

\הגדרה{ע״ע יראה פסולה. ע׳ במדור מדרגות והערכות אישיותיות, בעלי יראה חיצונית.}

\paragraphs

\ערך{״יראה עליונה״}\myfootnote{ \textbf{יראה }\textbf{עילאה}\textbf{ יראה }\textbf{תתאה} - זוהר ח״א יא: יב. ת״ז תי׳ ל, (מהדורת וילנא דף פא:). עפ״י ע״ר א מג שני צדדים יש לה ליראת\hebrewmakaf ד׳ ה\textbf{יראה} \textbf{התחתונה והיראה העליונה. התחתונה} היא מתיראת שלא לרדת מטה מטה, ומתוך כך היא מתאמצת בעליותיה התדירות, ו\textbf{העליונה} היא מתיראת שלא תהרוס אל על יותר מהחוג המתאים לה, שלא תאבד בזה את המנוחה המבססת כל טוב.\label{4}}\הגדרה{ - יראה הבאה מתוך ההופעה\mycircle{°} של הגדולה\hebrewmakaf האלהית\mycircle{°}, המתרוממת ממעל לכל הערכים ולכל ההשערות\mycircle{°}, לכל המדות ולכל האפשרויות ועם כל זה היא שייכת לנו, ואנו אחוזים קשורים ודבוקים בה, בכל עומק הויתנו, מתוך השאיפה הבלתי\hebrewmakaf פוסקת להקלט בהדרת\mycircle{°} הקודש\mycircle{°} הנוראה, המתעלה מכל מדה וקצב, מכל ערכי זמנים ומקומות }\מקור{[עפ״י ע״ר א קפה]}\צהגדרה{. }

\מעוין{◊}\ערך{ מקור יראת ד׳ }\הגדרה{- השערת\mycircle{°} הגודל והשיגוב המוחלט <שהאדם נהנה מזה בעונג אסתתי נורא, שהוא דומה לההנאה של טעם חריף מוטעם מאד, המאמץ את הלב ומעורר את הכח> }\מקור{[קבצ׳ א קצ]}\צהגדרה{.}

\מעוין{◊ }\משנה{מקור יראת ד׳ }\הגדרה{הוא מצטייר בעומק הנשמה מפני ההרכבה הנפלאה של שני ההפכים אשר בההשגה האלהית: העדר הידיעה המוחלטת במהות האלהות, וודאות הידיעה במציאותה השלמה. והתמזגות נפלאה של שני ערכים אדירים הפכיים מאיימת היא מאד }\מקור{[מ״ה יראה ב (א״ק ד תלא)]}\צהגדרה{.}

\מעוין{◊ }\משנה{מקור היראה }\הגדרה{- המחשבות המקודשות, בציור\mycircle{°} האמתות האלהית, (שמתוכם) מתפתח מרום רוממות והערצה אין\hebrewmakaf סופית }\מקור{[ע״ר א ריד]}\צהגדרה{. }

\משנה{יראה עליונה }\הגדרה{- יראה הבאה מתוך הכרת הכבוד והשלמות האלהית\mycircle{°}, הממלאה אותנו הוד נורא, שתוכה רצוף אהבה\mycircle{°}, העומדת למעלה מכל ציור\mycircle{°} אהבה ורגש, רק במרום השכל\mycircle{°} וההכרה העליונה בעז\mycircle{°} טהרת\mycircle{°} גבורתה }\מקור{[עפ״י ע״ר א מד, צג]}\צהגדרה{. }

\הגדרה{אמתת הבושה במקורה, בתעודתה התמימה. והבושה הזאת היא הבגרות העולמית, הכוללת את כל הבריות כולם, בהיותם מכירים את יסוד התפארת\mycircle{°}, מקור הכבוד וההוד\mycircle{°}, וכל נקוד(ו)ת חייהם (תהיינה) נערכ(ות) על פיו }\מקור{[עפ״י ר״מ קמ]}\צהגדרה{. }

\משנה{יראה עילאה }\הגדרה{- התכונה הפנימית\mycircle{°} המקננת בסתר נשמתם של בני\hebrewmakaf אדם, שהיא העצמיות\mycircle{°} היותר טבעית של נשמת כל\hebrewmakaf חי }\מקור{[עפ״י ע״ר ב נז]}\צהגדרה{. }

\משנה{נקודת יראת אלהים }\הגדרה{- הסגולה\mycircle{°} הפנימית של החיים החובקת ביוקר מהותה את כל המציאות הגדולה }\מקור{[קובץ ה רט]}\צהגדרה{. }

\תמקור{(היראה) שלא יעלה בעבודתו בקודש יותר ממדרגתו [מא״ה ג פג]. }

\הגדרה{ע״ע יראת שמים. ע״ע יראת הכבוד והרוממות. ע״ע ״יראה תתאה״. ע״ע יראת ד׳. ע׳ בנספחות, מדור מחקרים, אהבה ויראה. ע׳ במדור מונחי קבלה ונסתר, אתרא דיראה עילאה. ע׳ במדור שמות כינויים ותארים אלהיים, ״נורא״. }

\paragraphs

\ערך{יראה פסולה }\הגדרה{- יראה\hebrewmakaf חיצונית ומשפלת רוח ונשמה }\מקור{[אג׳ א ריד]}\צהגדרה{. }

\הגדרה{יראת\hebrewmakaf שמים\mycircle{°} בתכונה כזאת שבלא השפעתה על החיים היו החיים יותר נוטים לפעול טוב\mycircle{°}, ולהוציא אל הפועל דברים מועילים לפרט ולכלל, ועל פי השפעתה מתמעט כח הפועל ההוא }\מקור{[א״ק ג, ראש דבר כז]}\צהגדרה{. }

\משנה{יראה רעה}\הגדרה{ - פחדנות}\צהגדרה{ }\מקור{[קבצ׳ ב קסו]}\צהגדרה{.}

\הגדרה{יראת\hebrewmakaf ד׳\mycircle{°} כפי שהיא מתיצבת לפתאים, סמל של הבהלה, הגורמת רפיון ויאוש וחסרון אונים}\myfootnote{ \textbf{יראת\hebrewmakaf ד}\textbf{׳ }\textbf{מתיצבת}\textbf{ לפתאים, סמל של הבהלה, הגורמת רפיון }\textbf{ויאוש}\textbf{ וחסרון אונים} - ע״ע ש״ק, קובץ ב קכו. קובץ א תרכח. ע״ע ״יראה תתאה״, הנתוקה ממקורה שהוא היראה\hebrewmakaf העילאה.\label{5}}\הגדרה{ }\מקור{[עפ״י פנק׳ א עא]}\צהגדרה{.}

\הגדרה{ע׳ במדור מדרגות והערכות אישיותיות, בעלי יראה חיצונית.}

\paragraphs

\תערך{״יראה תתאה״}\footref{4}\הגדרה{ }\צמקור{-}\תמקור{ שירא תמיד שלא יקלקל מעשיו בעבודת הקודש [מא״ה ג פג].}

\הגדרה{ע״ע יראת ד׳. }

\ערך{״יראה תתאה״}\myfootnote{ \textbf{יראה }\textbf{תתאה}\textbf{ }\textbf{הנתוקה}\textbf{ ממקורה שהיא }\textbf{היראה\hebrewmakaf העילאה} - \textbf{המצב הריקן וכו׳ מהשכלה ותורה שבו המהות }\textbf{האלהית}\textbf{ כלולה בתור }\textbf{כח}\textbf{ תקיף שאין }\textbf{להמלט}\textbf{ ממנו וכו׳} - ע״ע א׳ מט, ע״ה קמו, קנ, ע״ט מו ד״ה כשההשכלה, א״ה 919 ד״ה התפילה כשהיא מתגשמת, מ״ה, כבוד א, ב, ג. א״ק ד תז, תיז\hebrewmakaf תכט. פנ׳ לז, לט, סא, ע. אג׳ א מב. פנק׳ ד תנ: ״יראת העונש הקטנה והפחותה, שלעמוד תחת השפעתה לבד אין רשות כ״א לקטני דעה, לעמי הארץ וגסי המוסר והשכל״. ע״ע זוהר בראשית יא:, ת״ז תי׳ ל (מהדורת וילנא דף פא:), ושם סוף תי׳ לו (מהדורת וילנא דף פו:). רש״י סוטה כב: ד״ה פרוש מיראה - ״של עונשין״. רמב״ם, תשובה, פ״י, א, ה. סמ״ג מ״ע ג. ובספר הישר, עמ׳ יט (הוצאת אשכול) ״ודע כי עבודת הבורא ית׳ מיראה איננה עבודת החסידים אלא עבודת הרשעים או אומות העולם״. בספר חסידים (מהד׳ רר״מ) סי׳ קסד בסופו ״ואם מפחד עושים זכויות ומיראתו, שהם יראים מפורעניות בעוה״ז: שלא ימית אותם ואת בניהם, או שלא ילקו ביסורים, או שלא ידלדלו, או שלא יביא עליהם את התוכחות ואת האלות, ויראים מן הגיהנם, או שלא ייטיב להם בגן עדן, או בעוה״ז או בעוה״ב, ולכך יראים מן הדין, אין להחזיק טובה להם בשביל מה שעושין״ וכו׳. ובמקור חסד ציין שם מדרש תדשא פי״ב, ורוקח שרש היראה ואו״ז א״ב סי׳ מ״ד. ע״ע בתומר דבורה פרק תשיעי ״היראה מסוכנת מאד ליפגם ולהכנס בה החיצונים. שהרי אם הוא ירא מן היסורים או מן המיתה או מגיהנם הרי זו יראת החיצונים״. וע׳ י׳ מאמרות לרמ״ע, חקו״ד ח״א כד כה. ובשומר אמונים הקדמון, ויכוח ראשון, סי׳ לג: ״״בפיו ובשפתיו כבדוני, ולבו רחק ממני, ותהי יראתם אותי מצות אנשים מלומדה״. והלא אם יאמר אדם כל היום, אני מאמין שהשי״ת אחד, מה מועיל לו, אם אינו מצייר בלבו איך הוא אחדותו, כי האמונה אינה האמירה בפה, כי אם היתאמת הדבר במחשבת הלב והצטיירו בשכל״. ושם, סי׳ לז: ״כי הפשטנים שאינם יודעים גודל רוממות אלהותו, והבדלו משאר הנמצאים, אינם יכולים להשיג כי אם יראת העונש שהיא יראה גרועה״. ובשו״ת מהרי״ל ס י׳ קנט: ״עובד מיראת עונש אינו בכלל צדיק״. ובאדיר במרום ח״ב, מאמר יחוד היראה ״והנה אי אפשר להגיע לזה אלא בכח היראה האמיתית - יראת ה׳ שהיא לחיים כנ״ל. שאם לא כן, שאפילו הוא ירא יראת העונש, הנה אין מעשיו אלא להינצל מן הרע, אך לא לתקן המציאות כולו, שהוא עיקר הכוונה... אלא שאם הוא יראה רעה אינו פועל אלא דרך המדות להביא השכר ולהרחיק העונש״. ע״ע אגרות פתחי חכמה ודעת סי׳ מ, (שערי רמח״ל עמ׳ שצד-שצה). ע״ע ליקוטי מוהר״ן תנינא סי׳ כב, ובהיכל הברכה, שמות דף רעב: ״יראה של עצבות, שהירא מעונש הכאה לחייבא בלא שום יראה ממלך עליון הרי הוא ירא מהקליפות והוא כעובד ע״ז, ואין הבחינה זאת יכולה לצאת מהקליפות לגמרי עד זמן התחיה״. וכן שם דף סב. וע׳ מלבי״ם מלאכי ג טז, יח. איוב כב, פתיחה למענה הארבעה עשר מענה אליפז.\label{6}}\הגדרה{ - }\משנה{הנתוקה ממקורה שהוא היראה\hebrewmakaf העילאה\mycircle{°}}\הגדרה{ - המצב הריקן של הציור\mycircle{°} האפל המלא תהו\mycircle{°} ובהו\mycircle{°} שמתהוה ברעיון כשחושבים על\hebrewmakaf דבר האלהים\mycircle{°} בלא השכלה ובלא תורה\mycircle{°}, שבו המהות האלהית כלולה רק בתור כח תקיף שאין להמלט ממנו והכרח הוא להשתעבד אליו }\מקור{[עפ״י א׳ קכו]}\צהגדרה{. }

\הגדרה{ע״ע יראת העונש. ע״ע יראת אלהי האמורי. ע״ע יראה חיצונית. ע״ע יראה רעה.}

\paragraphs

\ערך{יראה }\הגדרה{- }\משנה{זעזוע היראה }\הגדרה{- דכאות\mycircle{°}, שנוטה לעצבון, לבוש ערפלי המלביש את היראה מצד החסרון של ההגבלה שלנו}\צהגדרה{ }\מקור{[עפ״י קבצ׳ א רכד]}\צהגדרה{.}

\הגדרה{היראה שמשיג האדם ומרגיש ממנה הרגש מדאיב ועצב }\מקור{[פניני ראיה 432]}\צהגדרה{.}

\paragraphs

\ערך{יראה }\הגדרה{- }\משנה{מורא היראה }\הגדרה{- יראת\hebrewmakaf הכבוד\hebrewmakaf והרוממות\mycircle{°} שמטלת על האדם הרגש עז והדר\mycircle{°} נועם\mycircle{°} }\מקור{[פניני ראיה 432]}\צהגדרה{. }

\paragraphs

\ערך{יראת אלהי האמורי}\הגדרה{ - }\משנה{״לא תיראו את אלהי האמורי״}\myfootnote{ שופטים ו י.\label{7}}\הגדרה{  - יראת\hebrewmakaf העונש\mycircle{°} כשהיא אינה משמשת לתפקידה <שהוא, לעורר את המהות העצמית של האידיאליות העליונה, שביסוד יראת\hebrewmakaf הכבוד\hebrewmakaf והרוממות\mycircle{°}. הזיקוק האידיאלי שהוא מעורר את כח האהבה\mycircle{°} העדינה, ויראת\hebrewmakaf ההוד\mycircle{°} המפוארה>, כשהזיקים האידיאליים נכבים, שאז כל עצמה של יראה זו אינה כ״א מפלצת, ועליה נאמר }\צהגדרה{״לא תיראו את אלהי האמורי״}\הגדרה{, וי ליה למאן דדחיל מרצועה בישא}\myfootnote{ \textbf{וי ליה למאן }\textbf{דדחיל}\textbf{ מרצועה בישא }- עפ״י ת״ז יב. אמנם שם: ״בראשית, יר״א שב״ת, אזהרה דיליה דלא לחללא ברתא דמלכא בט״ל מלאכות, דאינון ארבעים מלאכות חסר א׳, לקבל ארבעים מלקיות חסר חד, רצועה לאלקאה איהי שפחה בישא, שעטנז, כלילא משור וחמור, דאמר יעקב ויהי לי שור וחמור, על כל דבר פשע על שור על חמור, דאין מלקין בשבת, דלא שלטא שפחה בישא על עלמא, ווי ליה למאן דאשליט לה על עלמא״.\label{8}}\הגדרה{, בהיותה נטולה מיסוד עץ\hebrewmakaf החיים\mycircle{°}, של תפארת\mycircle{°} הצחצחות\mycircle{°}, המוסריות\hebrewmakaf העליונה\mycircle{°} בהוד\mycircle{°} קדשה }\מקור{[עפ״י קובץ ד כג]}\צהגדרה{.}

\הגדרה{ע״ע ״יראה תתאה״, הנתוקה ממקורה שהוא היראה\hebrewmakaf העילאה.}

\paragraphs

\ערך{יראת אלהים האמיתית }\הגדרה{- יראת הדעת, החכמה והתבונה, הנאחזת במושגים של כבוד ורוממות נפש }\מקור{[פנק׳ ג דש]}\צהגדרה{.}

\משנה{יראת אלהים }\הגדרה{- אומץ הנפש של שכלול הרצון, בתגבורת עזוזו }\מקור{[עפ״י מ״ר 100 (א״א 91)]}\צהגדרה{. }

\משנה{יראה}\הגדרה{ - גבורה\mycircle{°} }\מקור{[קבצ׳ ב קסו]}\צהגדרה{.}

\הגדרה{ע׳ במדור מונחי קבלה ונסתר, גבורות הגבורה. ע׳ במדור מדתם ועניינם הרוחני של אישי התנ״ך, יצחק, מדתו של יצחק. }

\paragraphs

\ערך{יראת ד׳ טבעית}\הגדרה{ - אשר אם יאמר לכל אדם, אפילו יהיה ערל לב, השי״ת ברא השמים והארץ וכל צבאם והוא משגיח\mycircle{°} על כל הברואים הרבים עד אין קץ וחקר והוא מעניש לעוברי מצותיו, בודאי יתנוצץ בלבו יראת ד׳, לכה״פ במדה כזאת שעל ידה יעשה המצות המוטלות עליו }\מקור{[עפ״י פנק׳ ה סד]}\צהגדרה{.}

\הגדרה{ע״ע יראת העונש. ע׳ בנספחות, מדור מחקרים, יראת הרוממות והמעלה, יראת הטבע, יראת העונש.}

\paragraphs

\ערך{יראת ד׳\mycircle{°}}\הגדרה{ - }\משנה{יראה מד׳ }\הגדרה{- ההשקפה הקדושה\mycircle{°} שאין לשום בריה לחשוב שיש בה השלמות המוחלטה, שהיא אינה מצויה כי אם אצל מקור\hebrewmakaf חיי\hebrewmakaf כל\hebrewmakaf החיים\mycircle{°} כולם, בורא כל העולמים\mycircle{°} ב״ה }\מקור{[ע״ר ב סד\hebrewmakaf ה]}\צהגדרה{. }

\משנה{יראה }\הגדרה{- }\מעוין{◊ }\הגדרה{היראה באה ע״י הערכת ציור\mycircle{°} החסרון שבברואים לעומת המעלה המוחלטת של שלימות השם\hebrewmakaf יתברך\mycircle{°} }\מקור{[עפ״י ע״א ג ב קעא]}\צהגדרה{. }

\משנה{יראת ד׳ }\הגדרה{- יראה הבאה מתוך ההכרה הבהירה של הזיקה הגדולה, שהמסובב זקוק אל סבתו\mycircle{°}, ומזה שכל ההויה כולה זקוקה היא לעילת כל העילות וסבת כל הסבות ברוך הוא }\מקור{[עפ״י ע״ר ב סג]}\צהגדרה{. }

\הגדרה{יראה הבאה מכח ההכרה שהטוב והאמת הוא שוכן עדי\hebrewmakaf עד\mycircle{°} עם מקור האמת והטוב, ואמיתת ערכו של כל נשגב איננו נערך כלל כ״א לפי הדעה\hebrewmakaf האלהית\mycircle{°}, שרק ביראה\hebrewmakaf טהורה\mycircle{°} אפשר להגיש לקצה ציורה }\מקור{[ע״א ג ב קנט]}\צהגדרה{. }

\הגדרה{פנית התמצית של כל הציורים\mycircle{°} הפרטיים לרבבותיהם וכל הרחבתם אל המושג האמיתי\mycircle{°} הכללי, שהוא חובק את כל המציאות החמרית\mycircle{°} והרוחנית\mycircle{°} }\מקור{[עפ״י שם שם קסו]}\צהגדרה{.}

\הגדרה{הראשית והאחרית של תמצית כל המחשבות והנטיות של האדם ושל כל העולמים }\מקור{[קובץ ד קכו]}\צהגדרה{.}

\משנה{יראת ד׳ האמיתית }\הגדרה{- יראת\hebrewmakaf הרוממות\mycircle{°} }\מקור{[אג׳ א עח]}\צהגדרה{. }

\משנה{יראת ד׳ }\הגדרה{- יראה הבאה מצד ההתבוננות בגודל הדרכים\mycircle{°} העליונים, שאין ערך ומבוא לאדם ללכת בהם, מצד רום\mycircle{°} שלמותם הרחוקה מחוקנו. יראת\hebrewmakaf רוממתו\mycircle{°} מצד הנשגב מהשגתנו ויכולתנו }\מקור{[עפ״י ע״א א 79 (ע״ר ב קכג)]}\צהגדרה{. }

\הגדרה{היראה הנולדת לאדם מצד שכלו בחקרו בפרטות פלאי ד׳ ית׳, מדרגה גדולה הרבה יותר מאד (מיראת\hebrewmakaf ד׳\hebrewmakaf הטבעית\mycircle{°}) שממנה תולד הבושת מהשי״ת והדר\mycircle{°} גאונו\mycircle{°} }\מקור{[עפ״י פנק׳ ה סד]}\צהגדרה{.}

\הגדרה{הנשמה השרויה בתוכיותה של היראה המוגשמת והגלומה, יראת\hebrewmakaf העונש\mycircle{°} וההפחדה הרגילה, הנשמה המתגלה לפי בהירות השגתו של אדם ולפי קדושת מעשיו ורוחו, עטרת כל אידיאלי הרוח, הוד הכבוד, ויראת הרוממות, מעוז האהבה, ומקור נחלי כל השאיפות האידיאליות, כבירות החשק, המתרוממים עדי\hebrewmakaf עד\mycircle{°} ברום עולמים }\מקור{[עפ״י קובץ ה סה]}\צהגדרה{.}

\משנה{יסוד היראה }\הגדרה{- מה שמתיראים להפסיק את הדבקות\hebrewmakaf האלהית\mycircle{°}, המלאה אורות\hebrewmakaf עליונים\mycircle{°} וחיים עד העולם, ותוספות הארה בכל העולמים. ונחת רוח, ועונג\hebrewmakaf עליון\mycircle{°}, ביסוד חיי החיים. ומיד בהפסק החשק, ונטיית התאוה הבהמית, הכל מסתתר, והאורה מטשטשת, ומרגישים מיד מיעוט השגה, וכהות רגש, הסרת רוח השמחה\mycircle{°} העליונה, פחדות של שפלות. וזה נוהג בכל נטיה לצד החומריות, ותמהון לבב, גם כשהיא של רשות, וק״ו בעניני איסור, שאין שיעור לנפילה והריסה לחשכת המאורות, וסיתומי צינורות החיים, התמעטות השפעים והעינוגים }\מקור{[א״ק ד תלג]}\צהגדרה{. }

\הגדרה{ההכרה השלילית שהיא יסוד האמת, המונחת על יסוד הערך האמיתי שיש לכל חמדה מעשית ועיונית בעומק ורום המדע האלהי והגבורה האמיתית של אין\hebrewmakaf סוף\mycircle{°} }\מקור{[ע״א ג ב קנט]}\צהגדרה{. }

\הגדרה{הרגש\mycircle{°} הטהור\mycircle{°} ההולך למשרים\mycircle{°}, רגש הלב הפנימי\mycircle{°}, השמיעה הפנימית להבין את הערך של השכל\mycircle{°} הטהור אל הרגש הקדוש החי בלב האדם ההולך למשרים בדרכי\hebrewmakaf ד׳\mycircle{°} וטובו\mycircle{°} }\מקור{[עפ״י שם שם קסט]}\צהגדרה{. }

\הגדרה{ע״ע יראת שמים. ע״ע אהבת ד׳. }

\ערך{״יראת ד׳ היא אוצרו״}\myfootnote{ ישעיה לג ו.\label{9}}\הגדרה{ - יראה הבאה מכח ההכרה שהטוב והאמת הוא שוכן עדי\hebrewmakaf עד\mycircle{°} עם מקור האמת והטוב, ואמיתת ערכו של כל נשגב איננו נערך כלל כ״א לפי הדעה\hebrewmakaf האלהית\mycircle{°}, שרק ביראה\hebrewmakaf טהורה\mycircle{°} אפשר להגיש לקצה ציורה\mycircle{°}, היא האוצר הבטוח על כל אלה הקנינים הגדולים, שיהיו לעד מתקיימים ועומדים בערכם הראוי להם ומשפיעים בכחם הגדול את כל הטוב שהם צריכים להשפיע לטובת הכלל והפרט עדי עד }\מקור{[ע״א ג ב קנט]}\צהגדרה{. }

\הגדרה{הרגש הטוב והנעים של כניעת הלב, של תמימות הרוח ושחות הנפש, שהוא מעטר את האדם כשימצא בו כפי אותה המדה שנדרשת לו לעומת שארי רגשותיו ותכונותיו, ״כקב חומטין בכור של חיטים שבעליה״ }\מקור{[עפ״י ע״ה קמד]}\צהגדרה{. }

\הגדרה{ע׳ במדור פסוקים ובטויי חז״ל, והיה אמונת עתיך חוסן ישועות חכמת ודעת יראת ד׳ היא אוצרו.}

\paragraphs

\ערך{יראת ד׳ }\הגדרה{- }\משנה{ביראת ד׳ }\הגדרה{- בדרכי חכמה ושכל }\מקור{[ע״א ג ב קסח]}\צהגדרה{.}

\משנה{יראת ד׳ }\הגדרה{- כלל כל יושר\mycircle{°} השכל\mycircle{°} והמוסר\mycircle{°} הטוב\mycircle{°} }\מקור{[עפ״י פנק׳ א רנג]}\צהגדרה{.}

\paragraphs

\ערך{״יראת ד׳ טהורה״}\myfootnote{ תהילים יט י.\label{10}}\הגדרה{ - יראת\hebrewmakaf ד׳\mycircle{°} המובנת לאמיתתה }\מקור{[ע״א ב ט כג 258]}\צהגדרה{. }

\הגדרה{העליה וההעמקה הרוחנית בזיקוק האלהי של האדם בעומק מילואו, שהוא היסוד שכל המעשים נסמכים אליו ויוצאים ממקורו }\מקור{[קובץ ה רכה]}\צהגדרה{. }

\הגדרה{ראשית כל חכמה, גברת כל החכמות, ומקור חכמת החיים }\מקור{[פנק׳ ג שיז (קבצ׳ ג כב)]}\צהגדרה{.}

\צהגדרה{ודאיות האמונה באשר בחר\hebrewmakaf בנו\hebrewmakaf מכל\hebrewmakaf העמים\mycircle{°} ונתן לנו את תורתו }\צמקור{[ל״י א (מהדורת בית אל תשע״א) שיט].}

\מעוין{◊ }\משנה{היראה הטהורה}\הגדרה{ נובעת מהשגת הסוד\mycircle{°}, עצמותה (של המחשבה\mycircle{°}) שהיא הכרת האמת\mycircle{°}, הכרת השלמות המוחלטת של האלהות\mycircle{°}, השגת התהילה\mycircle{°} האלהית בשיקוף פנימי }\מקור{[ע״א ד ח לה]}\צהגדרה{. }

\משנה{יראת אלהים טהורה }\הגדרה{- אמונת\mycircle{°} אלהים אמיצה, השתולה דוקא בתוך הכנסיה\hebrewmakaf הישראלית\mycircle{°} }\מקור{[מ״ר 3]}\צהגדרה{. }

\הגדרה{ע׳ במדור שמות כינויים ותארים אלהיים, טהור, שם טהור.}

\paragraphs

\ערך{יראת הגודל }\הגדרה{- איום שמימי עליון\mycircle{°}, המעמיק עד למעמקי שרשי הנשמה\mycircle{°}, המרככת את התכונה הקשה של הגסות\mycircle{°} החמרית\mycircle{°}, ומכשרת את החיים לספוג אל תוכם את הנועם\mycircle{°} העדין של אור\mycircle{°}\hebrewmakaf הקודש\mycircle{°} בכל מלא הודו כפי מדת הנפש ואפשרות קבולה }\מקור{[ע״ר א יד]}\צהגדרה{. }

\הגדרה{ע״ע אהבה אלהית. }

\paragraphs

\ערך{יראת הכבוד והרוממות }\הגדרה{- היראה המתמלאת לפי אותו הגודל של שיקוף האידיאלים\mycircle{°} המוסריים\mycircle{°} והשכליים הגנוזים, שההוגה בם מתיחש אליהם בכל הוד\mycircle{°}, כבוד וגדולה\mycircle{°}, ועל ידם מתעלה ומתעדן אותו הרגש הפנימי של אור האמת\mycircle{°}, של ברק העצם\mycircle{°}, מקור המקורות עדי\hebrewmakaf עד\mycircle{°} }\מקור{[ע״ה קנה]}\צהגדרה{. }

\משנה{יראת הכבוד וההוד }\הגדרה{- יחושה השכלי והמגמתי\mycircle{°} של הנפש למה שהוא מכובד למעלה בהרגשתה היותר פנימית\mycircle{°}, המגדל אותה, מעשירה ומרהיבה בגאון פנימי }\מקור{[עפ״י ע״ה קמז]}\צהגדרה{. }

\משנה{יראת רוממות }\הגדרה{- יראה של כבוד\mycircle{°}, של הדר\mycircle{°} מוסרי\mycircle{°} ושכלי, של שלמות אין קץ }\מקור{[אג׳ א מז]}\צהגדרה{. }

\הגדרה{דעת\hebrewmakaf ד׳\hebrewmakaf ועוזו\mycircle{°} }\מקור{[ע״א ג ב קסז]}\צהגדרה{. }

\הגדרה{השגת עצם רוממות כבודו ב״ה, ע״י האוהב\mycircle{°} האמיתי\mycircle{°}, והראיה שההוד והיראה ראויים רק לו לבדו }\מקור{[עפ״י מא״ה א קפג]}\צהגדרה{. }

\הגדרה{היראה הבאה מהכרת ערך מה שהוא למעלה ומרומם <זהו הגרעין שממנו צומח יסוד יראת\hebrewmakaf שמים\mycircle{°} בנפש> }\מקור{[ע״א ד ו סח]}\צהגדרה{. }

\הגדרה{יראת\hebrewmakaf ד׳\mycircle{°} בציור שמרומם ומחיה את כל הכוחות הנפשיים}\צהגדרה{ }\מקור{[עפ״י פנק׳ א עא\hebrewmakaf ב]}\צהגדרה{.}

\משנה{יראה רוממה }\הגדרה{- היראה הצורתית\mycircle{°}, האידיאלית, הבאה ומופיעה\mycircle{°} על הנשמה המסתכלת, מאותה ההברקה המזרחת\mycircle{°} בהשגת השלמות האלהית הכוללת כל, המנצחת את כל ההפכים, שהם כולם מתכללים בצורה עליונה של תפארת\mycircle{°} מלאה וקדושה\mycircle{°} }\מקור{[עפ״י קובץ ז קיח]}\צהגדרה{.}

\מעוין{◊ }\משנה{יראת הרוממות והמעלה }\הגדרה{- היראה השלמה, המחוברת מהאהבה\mycircle{°} והכבוד\mycircle{°} בהתקבצותם יחד בנפש עד שיהיו לרגש אחד }\מקור{[עפ״י ע״א ג ב לב]}\צהגדרה{. }

\הגדרה{ע״ע יראת ד׳. ע״ע יראה עליונה. }

\paragraphs

\ערך{יראת העונש }\הגדרה{- פחד הדין ואימת המשפט בענינים הגשמיים דבני חיי ומזוני }\מקור{[עפ״י מא״ה א פט]}\צהגדרה{. }

\מעוין{◊ }\הגדרה{המדה השפלה שבצורת המוסר הרוחני <שהיא גם כן המדה העליונה - כשהיא אידיאלית (ובאה ליראת\hebrewmakaf החטא\mycircle{°})>}\צהגדרה{ }\מקור{[עפ״י א״ק ד תכד]}\צהגדרה{.}

\משנה{יראת העונש וההפחדה הרגילה}\myfootnote{ \textbf{היראה המוגשמת }\textbf{וכו}\textbf{׳, שהיא קרקע המשכן של האמונה המתפשטת} - ע״ע פנק׳ ג קפא-קפג, ושם: ״בכל מדרגות השלמים, יראת העונש אינה סרה לגמרי להיות מועילה כל זמן ששוכן בגוף החמרי״. ובע״ר א קצה: ״היראה מוכרחת היא להיות מתחלת, בראשית גידולה, מציור של רטט, הממולא פחד, שתכונתו היא תכן מר המקושר עם צער וכאב נפשי. אמנם זהו רק ראשית הצמח של היראה מצד התכונה החמרית של האדם, הפועלת את רשומה על הנפש״. באגרות פתחי חכמה ודעת, לרמח״ל, סי׳ מ (שערי רמח״ל שצד-שצה): ״יראת העונש סוף סוף אינו רע. ועוד שאפילו יהיה רע, הוא רק אם שעובדים להקב״ה רק בעבור זאת. אבל לעבוד לעשות נחת רוח ליוצרו, ואח״כ להיות מפחדים מן העונש - אדרבה זה טוב, ועל זה נאמר ״אשרי אדם מפחד תמיד״״.\label{11}}\הגדרה{ - היראה המוגשמת והגלומה, שהיא קרקע המשכן של האמונה המתפשטת לתקון העולם }\מקור{[קובץ ה סה]}\צהגדרה{.}

\משנה{תעודתה של יראת העונש}\myfootnote{ \textbf{הצתת }\textbf{אליתא} - ע׳ יומא כד: וברש״י שם מה. ד״ה אליתא - קיסמים דקים.\textbf{לעורר את האידיאלים הגנוזים של הקודש והטוב} - ע׳ רש״י שם מה. ד״ה להצתת האליתא - להצית בהן אור מערכה גדולה.\textbf{ללבותם יותר כשהם עמומים} - רש״י שם כד: ד״ה הצתת אליתא - הצתת קיסמים דקים, אם הוצרך להצית על המזבח באחת מן המערכות, כגון שקרבה מערכה לכלות, שנשרפה כל המדורה.\label{12}}\הגדרה{ - היא רק תעודת הצתת\hebrewmakaf אליתא\mycircle{°}, לעורר את המהות העצמית של האידיאליות העליונה, שביסוד יראת\hebrewmakaf הכבוד\hebrewmakaf והרוממות\mycircle{°}. לעורר את האידיאלים הגנוזים של הקודש והטוב, או ללבותם יותר כשהם עמומים}\צהגדרה{ }\מקור{[קובץ ד כג]}\צהגדרה{.}

\הגדרה{ע״ע יראת ד׳ טבעית. ע״ע יראה תתאה הנתוקה ממקורה שהיא היראה\hebrewmakaf העילאה. ע״ע מוסר נמוך. ע״ע יראת אלהי האמורי. ע״ע מלח. ע׳ בנספחות, מדור מחקרים, אוהב לעומת ירא. ע״ע יראת שמים, ״מורא שמים״. }

\paragraphs

\ערך{יראת הרוממות }\הגדרה{- ע״ע יראת הכבוד והרוממות. }

\paragraphs

\ערך{יראת הרוממות והמעלה, יראת הטבע, יראת העונש }\הגדרה{- ע׳ בנספחות, מדור מחקרים.}

\paragraphs

\ערך{יראת הרוממות }\הגדרה{- }\משנה{בהשגת הענינים האלהיים}\הגדרה{ - להתירא מלהרוס בגנזי עולם ושלא להשען בהם על בינתו }\מקור{[פנק׳ א תקסד-ה (קבצ׳ א ס)]}\צהגדרה{.}

\paragraphs

\ערך{יראת חטא\mycircle{°}}\הגדרה{ - היראה להשמר מן הפחיתות שבנפש }\מקור{[עפ״י מ״א א ו (ע״ר ב קכב)]}\צהגדרה{. }

\הגדרה{הזהירות גם במצוה קלה ובריחה מן העבירה, מן הכיעור והדומה לו, מצד ההכרה של שלמות מצב הנפש בשמחת דעת\hebrewmakaf ד׳\mycircle{°} הטהורה, והיראה, שכל חטא, כל רפיון ועצלות משקידה וזהירות, מאפיל את חלונות האורה, ועושה חיץ שלא תוכל הנפש לבא למצב דעת\hebrewmakaf אלהים\mycircle{°} בטהרתה\mycircle{°} }\מקור{[עפ״י פנק׳ א נב]}\צהגדרה{.}

\הגדרה{דאגה שלא יצא מתחת ידו דבר שהוא נגד כבודו ית׳ }\מקור{[מ״ר 275]}\צהגדרה{. }

\הגדרה{הפרישה מדבר רע}\צהגדרה{, שתבוא (מאחת) משתי סיבות: הסיבה הטובה היא מתגבורת כח היושר\mycircle{°} בנפש, הבא מכח קדושת התורה, עד שיהיה מכיר בתכלית ההכרה בגנותו של כל חטא ויתיירא ממנו מצד עצמו, חוץ מיראת\hebrewmakaf עונשו\mycircle{°}; ויש יראת חטא הבאה מצד רפיון הנפש, שאם היה אותו האיש אדם חזק ובריא לא היה בא למדה זו }\מקור{[ע״א ג ב קעד]}\צהגדרה{. }

\משנה{יראת חטא הטבעית }\הגדרה{- הסילוד מן החטא ותביעת התשובה הקבועה אם נזדמן חלילה לידו איזה דבר עברה\mycircle{°} ועוון\mycircle{°}}\צהגדרה{, <(שהוא) הטבע האנושי הבריא ביחס למוסר\mycircle{°} הכללי, והטבע הישראלי המיוחד ביחש לכל חטא ועוון מצד התורה\mycircle{°}  והמצוה\mycircle{°} מורשה קהלת יעקב>  }\מקור{[א״ש ו ג]}\צהגדרה{. }

\משנה{יראה מעשית }\הגדרה{- }\משנה{מקורה }\הגדרה{- השכלת ההגבלה, גזרת החק והמשפט, ההולכים עפ״י תכונת הקצב העולמי, שעל ידם הכל מובל אל מגמתו\mycircle{°} ותכליתו המיועדות. וההבנה הזאת הולכת היא ועולה ומסתכלת בכבוד יוצר בראשית בכל מלא עולמים\mycircle{°}, ובבאה להכיר את תכונת חופש הרצון האישי של האדם, והאפשרות שיש לו להיות הולך ותועה במערכות המעשה, באופן מהרס ומכער את כל חקי התפארת\mycircle{°} וההוד\mycircle{°} של כל מלא קצבת העולמים, מיד יראה מעשית באה ומתמשכת על האדם }\מקור{[ע״ר א יד]}\צהגדרה{. }

\משנה{היראה מכל חטא}\הגדרה{ - }\משנה{יסודה }\הגדרה{- סילוק הדעת\mycircle{°}, שהחטא גורם בהכרח הרחקת האור\hebrewmakaf האלהי\mycircle{°} מנשמת\mycircle{°} האדם, ולפי זה מכל העולם לפי הערך, שהוא דבר מכאיב יותר מכל מיני ההשחתות שיכולות להימצא בעולם, <כי מניעת הטוב הגמור הוא הרע היותר נורא,}\myfootnote{ \textbf{מניעת הטוב הגמור הוא הרע היותר נורא} - באג׳ א קסו ״י״ל ד״אסונים״ כולל ג״כ האסון היותר גדול של הריחוק מקרבת ד׳ ב״ה שהוא הטוב האמיתי, א״כ הוי בכלל יראת אמת הבאה מאהבה״.  \label{13}}\הגדרה{ ו״טוב חסדך מחיים״}\צהגדרה{> }\מקור{[קבצ׳ א רטז]}\צהגדרה{.}

\הגדרה{המדה העליונה שבצורת המוסר\mycircle{°} הרוחני, הבאה כאשר מכיר האדם את החטא לכיעור, והכיעור למתועב להנפש, ומתועב אל ההויה. והתיעוב מצער הוא צער רוחני אמיץ מאד. <ולפי גודל האור שבנשמה, ככה הכתם של החטא נראה, וככה הוא מונע את ההרמוניה\mycircle{°} היפה שלה עם ההויה, עם יפיה ותפארתה, עם כבודה וקדושתה, והיראה ממנו באה>}\צהגדרה{ }\מקור{[עפ״י א״ק ד תכד]}\צהגדרה{.}

\paragraphs

\משנה{יראת שמים }\צהגדרה{- הקשור הפנימי והכולל למעלה }\צמקור{[א״ל כה].}

\paragraphs

\ערך{יראת שמים }\הגדרה{- יראת (האדם) מצד ריחוק האור\hebrewmakaf האלהי\mycircle{°} ממנו, ובושת פניו וצערו }\מקור{[עפ״י מ״א א ו (ע״ר ב קכב)]}\צהגדרה{. }

\הגדרה{יראת\hebrewmakaf ד׳\mycircle{°}, היא יראת\hebrewmakaf רוממותו\mycircle{°}, מצד הנשגב מהשגתנו ויכולתנו, כענין ״גבהי שמים מה תפעל״ }\צהגדרה{[עפ״י ע״א א }\צמקור{79\hebrewmakaf 80}\צהגדרה{ (ע״ר ב קכג)]. }

\מעוין{◊ }\משנה{כח יראת שמים }\הגדרה{בא מציור\mycircle{°} האמת העיונית, המשגת את גבורת החיובים מכח אור ההגבלה\mycircle{°} והשלילה }\מקור{[ע״א ג ב קעה]}\צהגדרה{. }

\paragraphs

\ערך{יראת שמים }\הגדרה{- היראה המושפעת מהרגש\mycircle{°} הכללי המתמלא בלב מההשקפה הגדולה על פליאותיו של צור\hebrewmakaf העולמים\mycircle{°} ב״ה, ״כי אראה שמיך מעשה אצבעותיך״ }\מקור{[ע״א ג ב קסט]}\צהגדרה{. }

\הגדרה{מכון האמונה\hebrewmakaf האלהית\mycircle{°} העליונה }\מקור{[א״ק ג, ראש דבר כו (א״א 138)]}\צהגדרה{. }

\משנה{יראת ד׳ }\הגדרה{- תמצית הטהור\mycircle{°} של האמונה הפשוטה }\מקור{[שם כז]}\צהגדרה{. }

\הגדרה{מהותו, צורתו\mycircle{°} העצמית, של האדם }\מקור{[עפ״י ע״ר א קא]}\צהגדרה{. }

\מעוין{◊}\הגדרה{ תוכן רוח יראת\hebrewmakaf שמים\mycircle{°} שבלב מבונה הוא ממניות שונות. המנה היסודית והעקרית היא חוש העמוק של האמונה\hebrewmakaf האלהית\mycircle{°} ברום טהרתה\mycircle{°} }\מקור{[א״א 81]}\צהגדרה{.}

\מעוין{◊ }\משנה{מרכזה הפנימי של יראת שמים }\הגדרה{- ההכרה האלהית }\מקור{[א״ק ג קנה]}\צהגדרה{. }

\הגדרה{ע״ע יראה עליונה. ע״ע יראת ד׳. }

\paragraphs

\ערך{יראת שמים }\הגדרה{- }\משנה{״מורא שמים״}\myfootnote{ אבות א ג.\label{14}}\הגדרה{ - מורא הדין\mycircle{°} ועונש הרשעים\mycircle{°}}\צהגדרה{ }\מקור{[ע״א ב ט יג]}\צהגדרה{.}

\paragraphs

\ערך{״יראת שמים שבגלוי״ }\הגדרה{- יראת שמים המבליטה את תכונתה על כל מפעלי חייו של האדם, בשמירת המצות. יראת שמים מלולית, תנועתית, מעשית }\מקור{[ע״ר א קא]}\צהגדרה{. }

\paragraphs

\ערך{״יראת שמים שבסתר״ }\הגדרה{- יראת שמים החדורה בכל מהותו הפנימית של האדם. יראת שמים אצילית, רעיונית, שכלית מופשטה }\מקור{[ע״ר א קא]}\צהגדרה{. }

\paragraphs

\ערך{ירד }\הגדרה{- נעשה גס\mycircle{°} ומגושם }\מקור{[ע״ר א קנו]}\צהגדרה{. }

\משנה{ירידה, תקלתה}\הגדרה{ - שקיעה של חמריות\mycircle{°} }\מקור{[שם יח]}\צהגדרה{. }

\הגדרה{ההשתקעות בתהום התוהו\mycircle{°} והכליון, בתהום ההתפזרות והאבקיות }\מקור{[ר״מ יט]}\צהגדרה{. }

\הגדרה{ע״ע עלוי. ע׳ במדור פסוקים ובטויי חז״ל, ירידת בור. }

\paragraphs

\ערך{יש }\הגדרה{- }\משנה{היש העליון\mycircle{°}}\הגדרה{ - הרוחניות\mycircle{°} והטוהר\mycircle{°} המעולה, אור\hebrewmakaf ד׳\mycircle{°}  ממרומיו\mycircle{°} }\מקור{[א״ק ג רפו]}\צהגדרה{. }

\משנה{היש המוחלט }\הגדרה{- אור הקודש\mycircle{°}, החכמה\mycircle{°}, מקום\mycircle{°} השאיבה המקורית, מקור\hebrewmakaf החיים\mycircle{°} }\מקור{[עפ״י שם ב רפד\hebrewmakaf ה]}\צהגדרה{. }

\מעוין{◊ }\משנה{היש האמיתי }\הגדרה{- מכונו הוא מקור\hebrewmakaf החיים\mycircle{°} }\מקור{[עפ״י ע״ר ב נז]}\צהגדרה{. }

\משנה{לשד היש}\הגדרה{ - הופעת העצמיות\mycircle{°}, מעמק ההויה משרש פנימיותה, המקור שמשם ההתהוות הנשמתית, אושר הנשמות }\מקור{[א״ק ב תצה]}\צהגדרה{.}

\הגדרה{ע׳ במדור פסוקים ובטויי חז״ל, ש״י עולמות.}

\paragraphs

\משנה{ישורון}\הגדרה{ - }\צהגדרה{השם המיוחד לישראל בהופעת קדושת אחדותם הכללית <בתחילת פרשת הברכה של משה איש האלהים ״ויהי בישורון מלך וגו׳ יחד שבט ישראל״> }\צמקור{[רצי״ה ע״ר ב רכט].}

\הגדרה{ע״ע ישראל לעומת ישורון.}

\paragraphs

\ערך{ישיבות }\הגדרה{- }\משנה{הישיבות }\הגדרה{- מכונות התורה\mycircle{°} אשר מעולם, מבצרי הנשמה אשר לישראל }\צהגדרה{[מ״ר }\צמקור{306-307}\צהגדרה{].}

\הגדרה{בתים גדולים שמגדילין בהן תורה, הרבצת התורה באופן במלא, להעמיד על ידן לפחות אחוזים הגונים של יהדות בריאה, רטובה מלשד החיים}\צהגדרה{ }\מקור{[פנק׳ א תקכז]}\צהגדרה{.}

\משנה{ישיבה}\צהגדרה{ - בית חרושת לגידול תלמידי\hebrewmakaf חכמים\mycircle{°}}\צמקור{ [שי׳ ת״ת 169].}

\צהגדרה{בית\hebrewmakaf מדרש גבוה ללימוד תורה, אשר מטרתו לגדל גדולי רוח, גדולי תורה וקדושה\mycircle{°} אנשי נשמה של כלל\hebrewmakaf ישראל\mycircle{°} }\צמקור{[שי׳ ת״ת 162].}

\משנה{המגמה הראשית בכונניותן של הישיבות בישראל}\הגדרה{ - תחיית התורה, המעמדת קדושתם של ישראל בכלל, ע״י גדולי תורה באמת, מיחידי\hebrewmakaf סגולה\mycircle{°}, וע״י המון רב כפי האפשרי של בעלי\hebrewmakaf בתים תלמידי\hebrewmakaf חכמים\mycircle{°} הגונים, }\צהגדרה{<ולא רק להעמיד רבנים> }\מקור{[אג׳ א קפד]}\צהגדרה{.}

\צהגדרה{להרבות כוחות רוחניים אדירים באומה }\צמקור{[שי׳ ת״ת 170].}

\paragraphs

\משנה{ישיבות }\צהגדרה{- }\צמשנה{תלמידי הישיבות }\הגדרה{- }\צהגדרה{צבא\hebrewmakaf הקדש של השתולים בבית ד׳ הגדול, שמגדלים תורה\mycircle{°} ותפילה\mycircle{°}, וכל מלא דבקותנו בדברי אלהים\hebrewmakaf חיים\mycircle{°} ומלך עולם, מגדילים ומאדרים למודה וקיומה וכל סגולות נצחה, הנושאים את דגלה, אחריותה ומעמסתה, וממשיכים מתוך כך את כל הצלחת בטחוננו הרוחני והחמרי, ההלכתי והמצותי, החברתי והמדיני }\צמקור{[ל״י ב קסה (מהדורת בית אל שעט)].}

\paragraphs

\משנה{ישיבת מרכז הרב}\צהגדרה{ - }\צמשנה{(עניינה עפ״י השאיפה העומדת ביסודה)}\צהגדרה{ - הישיבה הגדולה המרכזית לישראל ולתורתו הקדושה בעיר בירתו הק׳, שתהיה, עם גדל ואדר תכנה הפנימי הרוחני, בתור מכון יצירת רוח ישראל המקורית והתעוררות השפעתה ופעולתה, בארץ, בעם ובעולם, בעֹשר וגדל התורה וכל מקצעותיה ועמק חסידות וקדושה וכל השייך לה, עם קדושת התחיה הלאומית ואָמצה הרענן ודעת\hebrewmakaf העולם\hebrewmakaf והחיים השלמה והבריאה, משוכללת במבחר כחותינו הרוחניים ובמיטב הסדרים והתכסיסים החדשים וביפי פאר הבנין כראוי לישיבה הישראלית הגדולה הבאה להמשיך ולחדש את עבודת רוח ישראל ועז חייו בקדש נחלתו }\צמקור{[צ״צ קצח\hebrewmakaf ט].}

\משנה{ישיבה מרכזית לישוב החדש }\הגדרה{- ישיבה שלמה בצביונה, ממולאת בתורה במובן רחב, ומותמכת בתכסיסי החיים כפי הצורך, (ש)תכריע את כל הישוב כלפי זכות, ותכין יסוד לחיי ישראל אמתיים על אדמת\hebrewmakaf הקודש\mycircle{°}. מכון שיתן ת״ח שיהיו מקובלים על הבריות, ושידעו פרק הגון בכל ענפי התורה, הסובבים את מערכות ההלכה והאגדה, כדי שידעו לכלכל דבר במשפט, נגד המון העוקרים אשר סבונו כדבורים }\מקור{[אג׳ א רפז]}\צהגדרה{.}

\צמשנה{ערכה המיוחד המרכזי האמיתי}\צהגדרה{ - }\מעוין{◊}\צהגדרה{ [גם מצד מרכזיות המקום: עיקו״ת, המוכרחת בדבר\hebrewmakaf ה׳ מירושלים, והיא הראשונה והיחידה בה לחנוך וגידול שטתי של צעירים מובחרים תופשי התורה], גם מצד הכוון הפנימי הרוחני של רכוז דרך הלמוד ההלכותי המקורי, היסודי של בירורי עניני התורה ממקורות תושבע״פ הראשונים, כדרך רבותינו הקדמונים ודרך הגר״א ז״ל, ושמוש בתורת גדולי האחרונים, וגם הברקת הפלפול וחדושיו על יסודה של דרך זו ובתור תבלין לה. ושל הקפת כל צדדי התורה ומקצעותיה, בהלכה ואגדה, לכל גוניהן ודרגותיהן, בקביעות שעורים עיוניים בחכמת רוח היהדות הפנימית, וקביעות למודים בתושב״כ בהדרכה להסתכלות ובירור וקליטת עניניה מצד פנים, וקביעות הדרכה לשימת לב ועיון וקליטה של מדרשי חז״ל וספרי הדעות והאמונות לחכמי ישראל המובהקים, מספרי המוסר לכוזרי, שמונה פרקים, מהר״ל ומחקר אלהי ועיון, וחסידות וקבלה. ועם שימת הלב לבירור ועיון של דרכי המדות בתושבע״פ על יסוד מדרשי ההלכה, ומתוך כך של הקפת והכללת כל זרמי וגוני היהדות כולה, בעבר ובהוה, בעיון ובחיים; וקשור מרכזיות המקום ומרכזיות הכוון הרוחני הזה עם מרכזיות הזמן של תקופת התחיה והבנין והגאולה, והריסת מרכזי היהדות והתורה בגולה המחייבת והמבססת את קביעות מרכז התורה במקומה האמיתי, את רכוז הכוחות התורניים הצעירים וגדולם במקומם האמיתי, את רכוז כל אותה הקפה וההכללה הרוחנית המכוונת לעמידה על טבעיותה האלהית השלמותית האחדותית של היהדות, בהדרכה והשפעה של סגולת קדושת האומה והארץ וקביעות לשון הקודש; גם מצד מרכזיות האישיות של כקאאמו״ר הגאון שליט״א, ומתוך כך העוזרים על ידו בעבוה״ק. גם מצד מרכזיות החומר האנושי המובחר והמוצלח של בני הישיבה הנוהרים ומתלקטים מכל הישיבות הגדולות בגולה ובארץ }\צמקור{[צ״צ ח״ב אגרת מה׳ אדר ב תרפ״ט].}

\paragraphs

\ערך{ישראל }\הגדרה{- העם היחיד בין העמים אשר תחת כל השמים, החי במילואו חיים רעננים\mycircle{°}, הדבק\mycircle{°} בכל עומק נשמתו\mycircle{°} באלהים\hebrewmakaf חיים\mycircle{°}, בתביעה טבעית עמוקה מתהום הויתו. וכשרון חיים זה, שהוא כח החיים המתמיד יותר הוא ההולך עמו בדרך הנוראה של גלויותיו השונות, הנוראות, הוא הנותן לו כח לישא ולסבול, ולצאת בדימוס, בנזר נצחון מכל אשר מררוהו ורובו }\מקור{[עפ״י א״א 83 (מ״ר 74)]}\צהגדרה{. }

\הגדרה{העם האחד שהאידיאה שלו היא להחיות בקרבו את היסוד המוסרי\mycircle{°}, לא רק בצדו המעשי לבדו, כי\hebrewmakaf אם גם בצדו האידיאלי\mycircle{°}, שהוא מטרה בפני עצמה בתור התכנית העליונה\mycircle{°} של החיים. והצד האידיאלי לא יושלם לעולם בעז\mycircle{°} גבורתו, עד שרק הוא יהיה הכח המניע את הגלגל הקולטורי לכל צדדיו המרובים, כי אם כשיהיה נובע ממקורו האמיתי, מאותו המקור שכל הטוב של המציאות נובע ממנו, הטוב האידיאלי האלהי, והוא סובל גורל נורא בחפץ פנימי לעמוד בחיים מיוחדים, כדי שעל ידו ״יתגדל ויתקדש שמיה רבא בעלמא די ברא כרעותיה״ }\מקור{[ע״ה קלה]}\צהגדרה{.}

\הגדרה{עם עולם, גוי איתן\mycircle{°} של שורש המחשבה האידיאלית האלהית ביצירת כל מוסדי הבריאה\mycircle{°}}\צהגדרה{ }\מקור{[ע״ר א יד]}\צהגדרה{.}

\הגדרה{״גוי גדול, אשר לו אלהים קרובים אליו״\mycircle{°}, הנקודה האחת של האנושיות, שהטבעיות של התביעה האלהית נמצאה בה בעצם רעננותה\mycircle{°} ובהירותה. המדע והרגש האלהי נוגעים אליו בכל מציאותו לכל חייו, לא יוכל רגע להסיח מהם את דעתו, ואי אפשר לו למצוא אושר בנקודה אחת מהחיים מבלעדם }\מקור{[עפ״י מ״ר 11]}\צהגדרה{.}

\הגדרה{גוי\hebrewmakaf קדוש\mycircle{°}, נושאי האמת האלהית העליונה, אשר נתגלתה פנים בפנים במעמד אשר לא היה כן לכל גוי }\מקור{[קובץ ה קמט]}\צהגדרה{.}

\הגדרה{גוי קדוש שומר אמונים, סגולה מכל העמים, זקן העמים ומורם, גואל התבל ומושהו מים השקר והרשעה }\מקור{[אג׳ ב שכה]}\צהגדרה{.}

\הגדרה{הופעת דבר\hebrewmakaf ד׳\mycircle{°} באנושיות ובכל ההויה }\מקור{[ע״ר א צו]}\צהגדרה{. }

\הגדרה{העם הנפלא, המפליא את כל העולם כולו בהדרת שיבתו ובהוד קדושת אמתת תורת אלהים אשר בלבבו}\צהגדרה{ }\מקור{[ע״א ב ט רעד]}\צהגדרה{. }

\הגדרה{החיה השמימית, המתהלכת על הארץ, בתבנית גוי, ששמה }\צהגדרה{ישראל }\מקור{[א״ק ד תקכ]}\צהגדרה{.}

\הגדרה{תמצית האנושיות כולה ותמצית היש בכללותו, מאוסף בצורה מרכזית לרומם ולשגב הכל, שבשלמותו הצורית אוצרות האורה\mycircle{°} הכוללים כל החמודות האישיות, החברותיות, ההיסתוריות והעולמיות }\מקור{[עפ״י ע״א ד ט נז]}\צהגדרה{. }

\הגדרה{האומה\mycircle{°} הכוללת את הצביון של כללות הרוחניות\mycircle{°} והקדושה של כלל האדם, ״אתם קרויים אדם״\mycircle{°}, עם ד׳ ועם קדוש, שכח חיים אדיר, איום ונפלא, אצור בקרבה מראש מקדם מתחילת מטעה }\מקור{[עפ״י ע״ר ב רסג]}\צהגדרה{.}

\הגדרה{עם שלם שהתקין עצמו בתחילה לשמע מפי\hebrewmakaf הגבורה\mycircle{°} ״אנכי״ ו״לא יהיה לך״, ונשא ברמה ובאומץ, בחרוף נפש ובעז, את הדגל של ד׳ אחד, ״ואהבת את ד׳ אלהיך בכל לבבך ובכל נפשך ובכל מאדך״, ההווה לעמוד אור לעצמו ולאנושות כולה, המביא את ברכת אחדות החיים והמציאות בעולם}\צהגדרה{ }\מקור{[עפ״י ל״ה 153]}\צהגדרה{. }

\הגדרה{האומה האחת שידיעת\hebrewmakaf האלהות\mycircle{°} הטהורה\mycircle{°} בעולם תלויה בגורלה }\מקור{[פנ׳ קכט]}\צהגדרה{. }

\הגדרה{עם קדוש, עם מפואר, עם עולמים, עם אור עולם חי ומאיר בלבו, עם יסוד הרוממות והגודל, הזוהר האידיאלי, לכל האדם, עם יסוד עמים ולאומים, עם זיו העולם וכבודו, הן עם כלביא יקום וכארי יתנשא }\מקור{[עפ״י קובץ ו קמג]}\צהגדרה{.}

\משנה{ישראל }\צהגדרה{- פלא\hebrewmakaf החיים המיוחד של האדם בשלמות מטבעו הצבורית }\צמקור{[עפ״י ל״י ב רפט].}

\צהגדרה{צור המחצב ועם הסגולה\mycircle{°} האלהית במערכות ימות עולם ושנות דור ודור }\צמקור{[נ״ה י].}

\צהגדרה{מכונת גלויו העצמית של הקב״ה\mycircle{°} בצלם\hebrewmakaf אלהים\mycircle{°} שבאדם לדורותיו, בקדושתה\mycircle{°} שקדמה\hebrewmakaf לעולם\mycircle{°} בהתפרטות בריאתו\mycircle{°} }\צמקור{[עפ״י שם ט].}

\צהגדרה{תמצית צלם אלהים שבאדם, אשר בגויים לא יתחשב, בשנוי חליפותיהם ותמורותיהם }\צמקור{[שם יט].}

\הגדרה{ר׳ במדור תורה, תורה.}

\ערך{ישראל }\הגדרה{- }\משנה{רום המעלה המהותית של קדושת ישראל }\הגדרה{- יסוד כל העולמים כולם בתוכנם האידיאלי העליון}\צהגדרה{ }\מקור{[ע״א ד ט קמ]}\צהגדרה{.}

\ערך{ישראל }\הגדרה{- }\משנה{מהותם העצמית הנותנת להם את אופים המיוחד }\הגדרה{- הדעה\mycircle{°} הגדולה של הטהרה\mycircle{°} העליונה של ההכרה הברורה בקדושת\mycircle{°} אמונת\mycircle{°} האחדות\hebrewmakaf האלהית\mycircle{°} }\מקור{[שם טו]}\צהגדרה{. }

\משנה{יסוד חיי האומה\mycircle{°}}\הגדרה{ - הדעה\hebrewmakaf העליונה\mycircle{°} של הכרה של האחדות\mycircle{°} האלהית\mycircle{°} וטהרתה\mycircle{°} }\מקור{[א״ק ב ש]}\צהגדרה{. }

\משנה{היסוד האמיתי של תכונת האומה }\הגדרה{- שאיפת הצדקה\hebrewmakaf העליונה\mycircle{°}, צדקת ד׳ בעולם }\מקור{[א״ש יג א]}\צהגדרה{. }

\ערך{ישראל }\הגדרה{- }\משנה{חיי ישראל האמיתיים}\הגדרה{ - החטבתן של המצוות\mycircle{°} בציור\mycircle{°} וברעיון, בשירה\mycircle{°} ובפועל }\מקור{[עפ״י א׳ יג]}\צהגדרה{.}

\ערך{ישראל }\הגדרה{- }\משנה{מקור חיי ישראל }\הגדרה{- אור שם\hebrewmakaf ד׳\mycircle{°} }\מקור{[אג׳ ב ריא]}\צהגדרה{. }

\ערך{ישראל }\הגדרה{- }\משנה{נטית האומה }\הגדרה{- הנטיה של החיים האלהיים\mycircle{°} כמו שהם }\מקור{[ש״ה, הקדמה, ח]}\צהגדרה{. }

\משנה{אור החיים הפנימיים של ישראל, דוקא מצד קיבוצם, דוקא מצד נקודת הגובה של המרכזיות שבהם }\הגדרה{- אור\hebrewmakaf אלהים\mycircle{°}, אור החיים היותר יפים, היותר טהורים, היותר מאירים }\מקור{[ע״א ד ו מ]}\צהגדרה{. }

\ערך{ישראל }\הגדרה{- }\משנה{אור ישראל }\הגדרה{- אור עליון של בהירות תהלת\mycircle{°} שם\hebrewmakaf ד׳\mycircle{°}, של עם זו יצר לו אל לספר\hebrewmakaf תהלתו\mycircle{°}, הידיעה המנוחלת מברכת\hebrewmakaf אברהם\mycircle{°}, הברוך לאל עליון קונה שמים וארץ, ל״עם לבדד ישכון ובגוים לא יתחשב״\mycircle{°} }\מקור{[א׳ כב]}\צהגדרה{.}

\הגדרה{אור העולם, העדין והנערץ}\צהגדרה{ }\מקור{[קובץ ה ח]}\צהגדרה{. }

\משנה{האור הישראלי }\הגדרה{- רוח\hebrewmakaf ה׳\mycircle{°} השורה על כלל ישראל ותפארתם\mycircle{°} הכללית. אור\hebrewmakaf ה׳\mycircle{°} באמת }\מקור{[ע״א א ד לג]}\צהגדרה{. }

\ערך{ישראל }\הגדרה{- }\משנה{שורש נשמת האומה }\הגדרה{- ההתעלות לצד החפץ האלהי, <שהוא עילוי שאין שום גבול וסוף לרוממותו, כי התכונה האלהית לא תדע גבול וסוף לעילויה, להשלמתה והתפתחותה> }\מקור{[עפ״י מ״ר 280]}\צהגדרה{. }

\ערך{ישראל}\צהגדרה{ - }\משנה{נשמתו הפנימית של ישראל }\הגדרה{- הצורה\mycircle{°} המזוקקת של האומה, חפץ ואמץ העליונות האלהית בעולם }\מקור{[קבצ׳ ג צד]}\צהגדרה{.}

\ערך{ישראל }\הגדרה{- }\משנה{תעודת ישראל }\הגדרה{- להאיר את עצמו ואת העולם, באור החכמה והדעת האמתית, אור ד׳ אלהי עולם }\צהגדרה{[ע״א ב בכורים כט (מ״ר}\צמקור{ 4\hebrewmakaf 183}\צהגדרה{)]. }

\הגדרה{העדות על גאות\hebrewmakaf ד׳\mycircle{°} ע״י כנסת\hebrewmakaf ישראל\mycircle{°}, המבררת במציאותה את היסוד האלהי של הנהגת העמים הממלכות וממילא של כל המציאות כולה }\מקור{[פנק׳ ב רז]}\צהגדרה{.}

\משנה{מגמתם ותעודתם של ישראל }\הגדרה{- }\צהגדרה{ל}\הגדרה{למד סוד אחדות ד׳ בעולם, להודיע שהוא ״עושה שלום ובורא רע״. ״הנה אנכי בראתי חרש נופח באש פחם ומוציא כלי למעשהו ואנכי בראתי משחית לחבל״ }\מקור{[ע״א ג א כ]}\צהגדרה{.}

\ערך{ישראל }\הגדרה{- }\משנה{ההשקפה הישראלית }\הגדרה{- ההשקפה של קודש\mycircle{°} על המציאות כולה }\מקור{[א׳ קסד]}\צהגדרה{.}

\ערך{ישראל }\הגדרה{- }\משנה{שאיפת ישראל }\הגדרה{- שאיפה ברוח האדם, הנשפעת מרוח העולם, העומדת להציל את הכל, מבלי להשאיר גם צרור, לחשוב מחשבות לבל ידח ממנו נדח, להציל את הגוף כמו הנשמה, את חיצוניות ההויה כמו פנימיותה, את הרע בעצמו כמו את הטוב, ולא עוד אלא להפך את הרע לטוב גמור, ולהעלות את העולם ומלואו בכל צדדיו ותכסיסיו, את העולם היחידי בכל ערכיו החומריים, ואת העולם החברותי בכל סדריו, להעמיד את הכל על בסיס הטוב. להסיר את כל חושך, ולהעביר מן הארץ כל ענן מקדיר, להסיר את פני הלוט על כל העמים והמסכה הנסוכה על כל הגוים}\צהגדרה{ }\מקור{[עפ״י א״ק ב תפח, תפט]}\צהגדרה{.}

\הגדרה{ע׳ במדור אליליות ודתות, אליליות, נצריות. ושם, בודאיות. ע״ע כנסת\hebrewmakaf ישראל. ע״ע רוח ישראל. ע״ע סגולת ישראל. ע׳ במדור מונחי קבלה ונסתר, ישראל, ״שם כל ישראל״. ע״ע אומה, האומה (הישראלית) כולה בצרופה הכללי. ע״ע אומה, רוח האומה היחידי. ע״ע אומה, ישראל. ע״ע אומה הישראלית, התכלית הכללית של האומה הישראלית. ע״ע אומה כללית. ע״ע ספור תהילת ד׳, כשרון ספור תהילת ד׳ אשר לישראל. }

\paragraphs

\ערך{ישראל בעמים}\הגדרה{ - כח של תסיסה\mycircle{°} תמידית, כח שלא יחדל את עבודתו ואת סגולת\hebrewmakaf הפריתו, לעולם יתבע את תביעתו ויביע ברמה את תקותו לנצחון מוחלט ושלם של דעתו, הכוללת, העליונה והקבועה }\מקור{[אג׳ ב רט]}\צהגדרה{.}

\הגדרה{ע׳ במדור האבות, ״נחלת יעקב״. ע׳ במדור פסוקים ובטויי חז״ל, עם לבדד. ע׳ במדור תורה, חכמת התורה.}

\paragraphs

\ערך{ישראל }\הגדרה{- }\משנה{סדרי עם ישראל בארצו}\הגדרה{ - מקדש ומלכות, כהונה ונבואה, שופטים ושוטרים וכל תכסיסיהם }\מקור{[א״ת א ג]}\צהגדרה{.}

\paragraphs

\ערך{ישראל }\הגדרה{- }\משנה{(לעומת יהודה) }\הגדרה{- ע״ע מלכות ישראל. }

\paragraphs

\ערך{ישראל }\הגדרה{- ע׳ במדור מדתם ועניינם הרוחני של אישי התנ״ך.}

\paragraphs

\ערך{ישראל }\הגדרה{- }\משנה{מדת התאר ישראל (לעומת יעקב) }\הגדרה{- ע׳ במדור מדתם ועניינם הרוחני של אישי התנ״ך.}

\paragraphs

\ערך{ישראל לעומת ישורון\mycircle{°}}\הגדרה{ - }\משנה{ישראל}\הגדרה{ - קדושה\mycircle{°} קבוצית חברותית פנימית\mycircle{°}, בחיי הכלל כשהם לעצמם, הסגולה\mycircle{°} הקבוצית המיוחדת לנו בישראל פנימה. ישראל הפנימי בתור ״הן עם לבדד ישכון ובגויים לא יתחשב״\mycircle{°}. <נחלת האהבה\mycircle{°} האלהית העליונה המאירה בישראל>. }\צהגדרה{ישורון}\הגדרה{ - השם החיצוני (של עדת יעקב), שהכל מביטים עליו, ״וישורו יראו בו״, איך ללכת באור\mycircle{°} נגהו\mycircle{°} וזיו\mycircle{°} זרחו\mycircle{°}. הכח המיוחד להופיע בעולם כולו בהופעה אלהית. קדושה קבוצית חברותית פועלת על המון עמים רבים, <שהיא תנובת השמחה\mycircle{°} האלהית, הזורחת\mycircle{°} למרחקים בקרני יפעתה\mycircle{°}> }\מקור{[עפ״י ע״ר א קז]}\צהגדרה{. }

\מעוין{◊ }\משנה{ישראל וישורון}\הגדרה{ כוללים ההיקף\mycircle{°} (}\צהגדרה{ישראל}\הגדרה{) והפנים\mycircle{°} (}\צהגדרה{ישורון}\הגדרה{) }\מקור{[א׳ קמ]}\צהגדרה{. }

\הגדרה{ע׳ במדור פסוקים ובטויי חז״ל, בני בכורי לעומת בנים. ושם בית יעקב לעומת בני ישראל. ושם ממלכת כהנים וגוי קדוש. ע׳ במדור מונחי קבלה ונסתר, קוב״ה דרגא על דרגא סתים וגליא וכו׳. ע׳ במדור מדתם ועניינם הרוחני של אישי התנ״ך, יעקב, מדת התאר יעקב (לעומת ישראל). ושם, ישראל, מדת התאר ישראל (לעומת יעקב). }\mylettertitle{כ}

\paragraphs

\משנה{כבוד }\צהגדרה{- ההתיחסות הראויה וההערכה האמיתית }\צמקור{[צ״צ א קמא].}

\paragraphs

\משנה{כבוד }\צהגדרה{- }\צמשנה{ענינו }\צהגדרה{- סדור היחס החיצוני\mycircle{°} החברתי אל המהות העצמית של נושא הכבוד }\צמקור{[ב״א ד ח (ג״ר 128)].}

\paragraphs

\ערך{כבוד }\הגדרה{- שלמות הנפש }\מקור{[ע״ר ב קכב (ע״א א 79)]}\צהגדרה{. }

\הגדרה{הדר\mycircle{°} מוסרי\mycircle{°} ושכלי }\מקור{[אג׳ א מז]}\צהגדרה{. }

\paragraphs

\משנה{כבוד }\צהגדרה{- }\צמשנה{״אין כבוד אלא תורה״}\myfootnote{ אבות ו ג.\label{1}}\צהגדרה{ - ההופעה והגילוי האמיתי של צלם\hebrewmakaf אלקים\mycircle{°} שבאדם, כבודו של מלך\hebrewmakaf הכבוד\mycircle{°}. <הכבוד התורני הריהו במהותו גילוי בריאות נפשו ונשמתו הישראלית\hebrewmakaf תורנית של האדם> }\צמקור{[ק״ת כה].}

\paragraphs

\ערך{כבוד }\הגדרה{- }\משנה{הכבוד הרוחני }\הגדרה{- האידיאליות\mycircle{°} בטהרתה\mycircle{°} }\מקור{[ע״ר א קמו]}\צהגדרה{. }

\משנה{הכבוד האמיתי}\myfootnote{ \textbf{הכבוד האמיתי }\textbf{-}\textbf{ גלוי השכל }\textbf{האלהי}\textbf{ הטהור} - בפיה״מ לרמב״ם, חגיגה ב א: ״באמרם ״כל שלא חס על כבוד קונו״ רוצה בו מי שלא יחוס ויחמול על שכלו, כי השכל הוא כבוד השם״.\label{2}}\הגדרה{ - גלוי השכל\mycircle{°} האלהי\mycircle{°} הטהור\mycircle{°}, ״אין כבוד אלא תורה\mycircle{°}״ }\מקור{[שם עניני תפילה כג]}\צהגדרה{. }

\הגדרה{שלמות חכמה\mycircle{°} ועבודה\mycircle{°} שלמה להשי״ת\mycircle{°} }\מקור{[ה׳ רנא]}\צהגדרה{. }

\ערך{כבוד }\הגדרה{- }\משנה{הכבוד הלמודי }\הגדרה{- תמציתם הכללית של מעמקי החכמה\mycircle{°} והדעה\mycircle{°}, המוכללים בהדר\mycircle{°} המצוה\mycircle{°} האלהית\mycircle{°} }\מקור{[ע״ר א טז\hebrewmakaf יז]}\צהגדרה{. }

\משנה{יסוד הכבוד בכל מלא החיים }\הגדרה{- (היסוד) הנותן ברק לתרבות הרוחנית\mycircle{°} בכלל, בתוכן של מחשבה\mycircle{°}, רגש\mycircle{°} והגיון\mycircle{°}. תכונת השיגוב\mycircle{°} האצילי\mycircle{°} }\מקור{[עפ״י שם רח]}\צהגדרה{. }

\משנה{הכבוד הרוחני\mycircle{°} הנאצל }\הגדרה{- יסוד המחשבה, הרגש וההגיון }\מקור{[שם]}\צהגדרה{. }

\paragraphs

\משנה{כבוד }\הגדרה{-}\צמשנה{ במובנו האלוהי }\הגדרה{- }\צהגדרה{התגלות. משל לאדם מכובד המופיע בחברה. התגלות אלוהים בנבראים התחתונים, בריאת האדם בצלם\hebrewmakaf אלהים\mycircle{°} }\צמקור{[עפ״י שי׳ ה 405]. }

\ערך{כבוד }\הגדרה{- }\משנה{כבוד עליון }\הגדרה{- אמיתת הופעת\mycircle{°} אלהים\mycircle{°} אמת\mycircle{°} }\מקור{[ע״ר א רז]}\צהגדרה{. }

\הגדרה{היצירה המעשית בכל צדדיה שתשתלם <(שתנאי לכך הוא ש)רוח האדם החפשי\mycircle{°} ישתלם במעז ההשלמה היותר טובה ומושלמת שבו, שהוא כח הבחירה\mycircle{°} בתכלית הטוב, על ידי כח עצמי בטוח ונצחי> }\מקור{[קבצ׳ א קכט]}\צהגדרה{. }

\ערך{כבוד }\הגדרה{- }\משנה{״שהכל ברא לכבודו״}\myfootnote{ ״שבע ברכות״, כתובות ח.\label{3}}\הגדרה{ - לתכלית נשגבה כ״כ שהוא כבוד לו ית׳ על מה שבראם }\מקור{[מ״ש שנ (ה׳ קעה)]}\צהגדרה{. }

\הגדרה{למטרה היותר טובה\mycircle{°} קדושה\mycircle{°} ומפוארה. המגמה\mycircle{°} הכללית המקודשת שהיא נזר הבריאה כתרה ועטרתה }\מקור{[עפ״י ע״א ד יב כח]}\צהגדרה{. }

\הגדרה{השלימות הגמורה שתמצא בברואים\mycircle{°} לכ״א לפי מה שיוכל לקבל ולהתעלות }\מקור{[שם א ה קב]}\צהגדרה{. }

\משנה{״כולו אומר כבוד״}\myfootnote{ תהילים כט ט.\label{4}}\הגדרה{ - כולו תכליתי, כולו אורה\mycircle{°} עונג\mycircle{°} ושמחה\mycircle{°} }\מקור{[ע״א ד ד ה]}\צהגדרה{. }

\משנה{״כבודו״, הכבוד האלהי }\הגדרה{- האלהות הכללית, המובעת מכל היקום כולו בסידורו ותדיריותו }\מקור{[ע״ר א רו]}\צהגדרה{. }

\משנה{כבודו }\הגדרה{- אור אלהותו והשגחתו\mycircle{°}, ודעת שמו יתברך }\מקור{[ל״ה 100]}\צהגדרה{. }

\משנה{כבוד אלהים }\צהגדרה{- גילוי הערך\mycircle{°} הכללי\mycircle{°} של תוכן הקדושה\mycircle{°} האלהית }\צמקור{[ל״י ב מב].}

\צהגדרה{הכבוד העליון האמיתי, גילוי ערך ההויה, העצמיי האמתי, בבחינת שכל הנמצאים אינם אלא מאמיתת המצאו }\צמקור{[א״ל קיג].}

\paragraphs

\ערך{״כבוד״ }\הגדרה{- חיים עד העולם, כי שם צוה ד׳\mycircle{°} את הברכה\mycircle{°} }\מקור{[א״א 77]}\צהגדרה{.}

\הגדרה{הכח החיוני, החי והפועל של המציאות כולה, הנורא בהודו\mycircle{°} והדרו\mycircle{°}. גדלות\mycircle{°} ההויה ויפי\mycircle{°} סדורה, בעוצם חסד\mycircle{°} עליון\mycircle{°} אשר בה, ורום האידיאליות\mycircle{°} שבמגמתה\mycircle{°} הכללית }\מקור{[עפ״י ע״ר א קצא]}\צהגדרה{. }

\paragraphs

\ערך{כבוד }\הגדרה{- }\משנה{״הכבוד אשר ראיתי על נהר כבר״ }\הגדרה{- ע׳ במדור פסוקים ובטויי חז״ל. }

\paragraphs

\ערך{״כבוד אל״}\myfootnote{ תהילים יט א.\label{5}}\הגדרה{ - כבוד נשגב\mycircle{°}, גנוז בחביון\hebrewmakaf עז\mycircle{°}, שהשכל האנושי אינו שולט להכירו, אלא ששפעים\mycircle{°} כוללים מהתוכן הנעלם הזה הולכים ומפכים, ומכים גלים על נשמת\mycircle{°} האדם, להיות מתרוממת על ידם למרום התשוקה הקדושה\mycircle{°} של נועם\hebrewmakaf ד׳\mycircle{°} }\מקור{[ע״ר ב נג]}\צהגדרה{. }

\משנה{הדר כבוד אל הכבוד }\הגדרה{- הארת זיו\hebrewmakaf שכינת\hebrewmakaf אל\mycircle{°}, עלית אור עולמי עולמים }\מקור{[א״ק ב תקל]}\צהגדרה{.}

\paragraphs

\ערך{״כבוד גויים״ }\הגדרה{- ע׳ במדור פסוקים ובטויי חז״ל.}

\paragraphs

\ערך{״כבוד ד׳״ }\הגדרה{- ״}\משנה{מראה דמות כבוד ד׳״ }\הגדרה{- ע׳ במדור פסוקים ובטויי חז״ל, מראה דמות כבוד ד׳.}

\paragraphs

\ערך{כבוד ד׳ בעולם }\הגדרה{- האור\mycircle{°} העתיד }\מקור{[א״ק ג קפב]}\צהגדרה{. }

\paragraphs

\ערך{כבוד המקדש }\הגדרה{- ע׳ במדור משכן ומקדש. }

\paragraphs

\ערך{״כבוד מלכותו\mycircle{°}״}\myfootnote{ תהילים קמה יא.\label{6}}\הגדרה{ - התכלית הרוחנית\mycircle{°} הנרצית מכל החוקים כולם, ומכל המצואים והנהגתם החוקית }\מקור{[ע״ר א קס]}\צהגדרה{. }

\משנה{כבוד המלוכה }\הגדרה{- אמצעי לתכלית המלוכה שהוא ההמשכה אחרי רצון\mycircle{°} המלך\mycircle{°} }\מקור{[מ״ש רמז]}\צהגדרה{. }

\הגדרה{ע׳ במדור שמות כינויים ותארים אלהיים, בהגדרות המבוא, שם השי״ת. ע׳ במדור פסוקים ובטויי חז״ל, ברוך שם כבוד מלכותו לעולם ועד. }

\paragraphs

\ערך{כבוד שם ד׳ }\הגדרה{- הטוב\hebrewmakaf הכללי\mycircle{°}, להועיל לכלל המציאות }\מקור{[ע״א ב ט שכא]}\צהגדרה{. }

\paragraphs

\ערך{״כבוד שמו״}\myfootnote{ תהילים כט ב. דבה״י א טז כט.\label{7}}\הגדרה{ - ההשפעה\mycircle{°} האצילית\mycircle{°} המרוממת את הנפש בצורה מקיפה בראשית ההתודעות של הכבוד\hebrewmakaf האלהי\mycircle{°}, הנשא ומרומם מכל רם ונשא. יפעת\mycircle{°} האצילות והשערת\mycircle{°} הגודל\hebrewmakaf העליון\mycircle{°}, המעורר בנשמה כבוד\mycircle{°} אין קץ למלך הכבוד ב״ה וגדולת שמו\hebrewmakaf הגדול\mycircle{°} }\מקור{[ע״ר א רח]}\צהגדרה{.}

\paragraphs

\ערך{כבוד שמים}\myfootnote{ סנהדרין פה. תוספתא יומא פרק ב ח. ירושלמי, שקלים כב:.\label{8}}\הגדרה{ - השתלמות השמות\mycircle{°}. השלמות האלהית העליונה, שכל הטוב\mycircle{°} והיושר\mycircle{°}, כל מדות החסד\mycircle{°} והגודל, כל התפארת\mycircle{°} הנצח\mycircle{°} וההוד\mycircle{°}, ברום מעלתם, יזהירו\mycircle{°} ויופיעו בכלל שלל צבעי\mycircle{°} רקמת חמדתם }\מקור{[עפ״י ט״ר צז, ע״א ד ט קמג]}\צהגדרה{. }

\הגדרה{יסוד ההשלמה התכליתית לכל נוצר. המכון והבסיס של כל ארחות הצדק\mycircle{°} והמשרים\mycircle{°}, <שע״י כבוד\hebrewmakaf שמים נעשים כל דרכי היושר\mycircle{°} מכוונים בעולם> }\מקור{[ע״א א ג לא]}\צהגדרה{.}

\הגדרה{שיגוב שם\hebrewmakaf ד׳\mycircle{°} לבדו }\מקור{[א׳ קנח]}\צהגדרה{. }

\ערך{כבוד שמים }\הגדרה{- }\משנה{יסודו }\הגדרה{- תיקון ההנהגה והיות בני אדם מתקרבים אל הטוב\hebrewmakaf האלהי\mycircle{°} ללכת בדרכיו\mycircle{°} }\מקור{[ע״א א ג לא]}\צהגדרה{. }

\הגדרה{תיקון מעשי בריותיו (ית׳) }\מקור{[שם]}\צהגדרה{. }

\paragraphs

\ערך{״כבוד תורה״ }\הגדרה{- ר׳ במדור תורה.}

\paragraphs

\ערך{כבודו }\הגדרה{- }\משנה{(של האדם) }\הגדרה{- רצונו, מהותו היותר פנימית של עצמיותו }\מקור{[א״ק ג לט]}\צהגדרה{.}

\paragraphs

\ערך{״כבודי״ }\הגדרה{- ע׳ במדור תיאורים אלהיים.}

\ערך{״כבודי״, ״תהילתי״ }\הגדרה{- }\משנה{״כבודי לאחר לא אתן ותהילתי״ }\הגדרה{- ע׳ שם.}

\paragraphs

\ערך{״כבודך״ }\הגדרה{- ע׳ במדור תיאורים אלהיים.}

\ערך{״כבודך״ }\הגדרה{- }\משנה{״פי כבודך״ }\הגדרה{- ע׳ שם. }

\paragraphs

\ערך{כהונה }\הגדרה{- }\משנה{הכהונה }\הגדרה{- האמצעות בין האדם לאלהים\mycircle{°} ע״י הבחירים היותר עליונים שבאדם <(ש)איננה אמצעות כ״א תכיפה\mycircle{°} הגונה> }\מקור{[א׳ נד]}\צהגדרה{. }

\הגדרה{התבלטות הצורה\mycircle{°} הרוחנית העליונה של ישראל\mycircle{°}, בקביעות אישיה נושאי דגלה, בתכונה מכוננת מראשית היצירות הנפשיות }\מקור{[עפ״י ע״א ד ט נז]}\צהגדרה{. }

\ערך{כהונה }\הגדרה{- }\משנה{קדושתה }\הגדרה{- האינסטינקט העמוק של האמונה\mycircle{°} והאהבה\hebrewmakaf האלהית\mycircle{°}, שהיראה\hebrewmakaf העליונה\mycircle{°} מקושרת עמה בהדר\mycircle{°} גאונה, הספוג\mycircle{°} במשפחת\hebrewmakaf הכהונה\mycircle{°} מכל האומה\mycircle{°} כולה. חוש פנימי\mycircle{°} הגנוז בטבע הנפש משורש היצירה, שהוא התכונה האינסטינקטיבית ליחסי הקדושה ושורשיה העליונים }\מקור{[עפ״י ע״ט יט]}\צהגדרה{. }

\הגדרה{ע׳ במדור אישים, אהרן. ושם, ״אהרן ובניו״. ושם, ״בני אהרן״. ע״ע בחירה, בחירת ד׳ בכהנים.}

\paragraphs

\ערך{כהונה }\הגדרה{- }\משנה{שבט הכהונה }\הגדרה{- מפלגה מיוחדת להעמדת הרוחניות\mycircle{°} באומה\mycircle{°} }\מקור{[עפ״י שם ד ה יא]}\צהגדרה{.}

\משנה{משפחת הכהונה }\הגדרה{- המשפחה, המשוקה מטל החיים הקדושים העליונים, של אור\hebrewmakaf ד׳\mycircle{°} ועבודתו\mycircle{°} הקבועה ברום טהרתה\mycircle{°}}\צהגדרה{ }\מקור{[ע״ר א קעז]}\צהגדרה{.}

\paragraphs

\ערך{כהונה }\הגדרה{- }\משנה{זכרי כהונה}\הגדרה{ - בני\hebrewmakaf אהרן\mycircle{°} המקודשים, המוכנים לעבודת הקדש בפעל }\מקור{[ע״ר א קעא\hebrewmakaf ב]}\צהגדרה{.}

\הגדרה{כח הפועל בקדושה }\מקור{[עפ״י ע״ר א קעה]}\צהגדרה{.}

\הגדרה{ע׳ במדור משכן ומקדש, לויים. }

\paragraphs

\ערך{כהן }\הגדרה{- מורה המלמד דעה לאחרים }\מקור{[מ״ש שטז]}\צהגדרה{. }

\הגדרה{מורה הנגש אל ד׳ ודבר קדשו }\מקור{[ע״א ג ב קעט]}\צהגדרה{.}

\הגדרה{עומד לשרת בקודש, לברך את עם ד׳ בברכת שלום כוללת ומקפת}\צהגדרה{ }\מקור{[ע״א ד ה נד]}\צהגדרה{.}

\הגדרה{החסיד\mycircle{°} העליוני, המלא חסד\mycircle{°} ודעה\mycircle{°} עליונה\mycircle{°}, היודע את האלהים\mycircle{°} באמת, ועל סמך דעתו והרגשתו העליונה נסמך הוא העם כולו, בעתים שזרם החיים השלם מאחד את העם עם העליונים שבבניו }\מקור{[א׳ נד]}\צהגדרה{. }

\משנה{כהנים }\הגדרה{- ממשיכי פאר\mycircle{°} החסד בעולם }\מקור{[ע״ר א קלד]}\צהגדרה{. }

\הגדרה{החלק המקודש אשר בעצם תכונתו - ד׳ הוא נחלתו, ומשאלותיו עולות ממעל לחוג החיים המצומצמים וגבוליהם }\מקור{[ע״א ד ט סג]}\צהגדרה{. }

\הגדרה{החיל הטוב, הלבבות הנאמנים\mycircle{°} והברים\mycircle{°} הנושאים עין למה שהוא למעלה מכל חפצי האדם המגושמים, ומתאמצים תמיד לחזות בנעם\hebrewmakaf ד׳\mycircle{°} ולבקר בהיכלו, ברגשי\hebrewmakaf קודש\mycircle{°} של קדושה\mycircle{°} וטהרה\mycircle{°}, של חכמה ושל ענוה\mycircle{°} כ״א לפי מעלתו. אנשי הקודש, שעליהם עבודת הקודש מוטלת, הפועלים על הכלל כולו, בתיקון המדות והמעשים, בהתישרות הדיעות והמושגים, שמטיבים את החיים ההוים וממשיכים ברכתם לחיי הנצח }\מקור{[עפ״י שם ג ב רכד]}\צהגדרה{.}

\הגדרה{העומדים להאיר לעם, לחבר את טהרת הרגש עם השכל <ע״כ גם עבודת ד׳ שבמקדש\mycircle{°}, גם העבודה השכלית של ההוראה עליהם מוטלת> }\מקור{[ע״א ג ב מ]}\צהגדרה{. }

\הגדרה{ע״ע בחירה, בחירת ד׳ בכהנים. ע׳ במדור אישים, אהרן. ושם, ״אהרן ובניו״. ושם, ״בני אהרן״.}

\paragraphs

\ערך{כהן }\הגדרה{- }\משנה{(תורת הכהן לעומת תורת השופט) }\הגדרה{- ע׳ במדור תורה, דרישת התורה בדרך הכהן. }

\paragraphs

\ערך{״כהן גדול״ }\הגדרה{- האיש היחידי המרכז את קדושת\mycircle{°} האומה\mycircle{°} בכללותה. האישיות הקדושה המיוחדה, העומד[ת] לנס של שלמות העבודה\mycircle{°} של כל קהל ישראל\mycircle{°} הגוי כולו, [האיש] העומד ברום קדושתו, שכל תכונתו היא קדש\hebrewmakaf לד׳\mycircle{°} כאשר הוא חרות על נזר הקדש, הרי הוא מופקד להיות המשלים והממלא, מלוי אור\mycircle{°} קדש, ושאיפה אלהית\mycircle{°} של דבקות\mycircle{°} תדירית באור\hebrewmakaf ד׳\mycircle{°}, מקור חיי החיים ב״ה, את כל העבודה כולה, של כל האומה }\מקור{[עפ״י ע״ר א קנה]}\צהגדרה{. }

\הגדרה{האדם היחידי הראשי, החובק בקרבו את הצד העליון היותר טהור\mycircle{°} שבצורה האנושית }\מקור{[עפ״י ע״א ד ו עב]}\צהגדרה{. }

\paragraphs

\ערך{כוכבים }\הגדרה{- }\משנה{קיבוץ החוקיות הכוכביית }\הגדרה{- הזרמים והכוחות הפועלים בפגישתם הסדורית על פי משטרם }\מקור{[א״ק ג לג]}\צהגדרה{. }

\הגדרה{ע״ע מזל, יסוד המזל. ע״ע גד. }

\paragraphs

\ערך{כונניות}\myfootnote{ חולין נו: ״מלמד שברא הקב״ה כונניות באדם שאם תהפך אחת מבני מעיים ימות״. וברש״י שם: \textbf{כונניות} - לשון ״ואת כנו״ (שמות לח), שברא להם בסיס לישב עליו, ואם ירדו מבסיסן שוב אין מתיישב״. ובדעת תבונות לרמח״ל, קכד: ״והרי אלה כונניות גדולות כאורלוגין הזה שאופניו פוגשות זו בזו, ואופן קטן מנועע אופנים גדולים ורבים - כך קשר האדון ב״ה כל בריאותיו קשרים גדולים, והכל קשר באדם, להיות הוא מנועע במעשיו, וכל השאר מתנועעים ממנו״.\textbf{שהם מכוונים גם כן }\textbf{להעיכול}\textbf{ הנפשי, במזון הרוחני}\textbf{ }- ע״ע א״ק ג פח-פט.\label{9}}\הגדרה{ - (בסיסי מערכת אחוזים וסבוכים זה בזה, שבהם) כל המעשים והמסיבות פועלים זע״ז ונפעלים זה מזה, כגוף אורגני אחד ועצם מיוחד שלם }\מקור{[עפ״י ע״א ג ב צד]}\צהגדרה{.}

\ערך{שעשה הקב״ה כונניות באדם}\הגדרה{ - נדרש על הסיבוך האורגני, של כלי העיכול הגופניים. <שהם מכוונים גם כן להעיכול הנפשי, במזון הרוחני, המחיה איש ועם> }\מקור{[א״ק א עו]}\צהגדרה{.}

\paragraphs

\ערך{כזב }\הגדרה{- דבר שאין לו קיום, היפוך האמת\mycircle{°} }\מקור{[מא״ה ג (מהדורת תשס״ד) שט]}\צהגדרה{.}

\מעוין{◊ }\הגדרה{יתיחס על דבר שאין לו בעצמו קיום ומעמד תמידי }\צהגדרה{[מא״ה ענייני תפילה שג}\הגדרה{].}

\הגדרה{ע״ע שקר. ע״ע מרמה. ע״ע שוא. ע״ע בד.}

\paragraphs

\ערך{כח }\הגדרה{-}\משנה{ (לעומת חומר\mycircle{°}) }\הגדרה{- (הענין הפועל) על החומר לשנותו ולשכללו ולהניעו }\מקור{[מ״ש קיז (מא״ה ב יא)]}\צהגדרה{. }

\paragraphs

\ערך{כח המפעל }\הגדרה{- יסוד פעולת החיים}\צהגדרה{ }\מקור{[ע״ר א לט]}\צהגדרה{.}

\paragraphs

\ערך{כח הרעיון }\הגדרה{- יסוד המחשבה\mycircle{°} וההרגשה }\מקור{[ע״ר א לט]}\צהגדרה{. }

\paragraphs

\ערך{כימה }\הגדרה{- (קבוצת כוכבים שמבטאת בעולם הרוחני\mycircle{°} את) מושג ההעמדה וההתקיימות. הכח המעמיד ונותן המקום. (הכימה היא) בעלת הצינה הפועלת להתקבצות החלקים ולהתגלות בנין המקום, היפך כח החמה המרקיע ומקליש }\צהגדרה{[עפ״י ע״א ב ט קלג, קלד }\צמקור{310\hebrewmakaf 309}\צהגדרה{]. }

\הגדרה{ע״ע עש. ע׳ במדור פסוקים ובטויי חז״ל, כסלא.}

\paragraphs

\ערך{כיעור }\הגדרה{- היפך הנוי\mycircle{°} }\מקור{[קבצ׳ א מג]}\צהגדרה{. }

\paragraphs

\ערך{כלות הנפש לאלהים\mycircle{°}}\הגדרה{ - הרגשת המתק והנועם\hebrewmakaf העליון\mycircle{°} בכל עומק הנשמה\mycircle{°}, בכל יפעת\mycircle{°} תענוגיו\mycircle{°} }\מקור{[עפ״י א׳ קלח]}\צהגדרה{. }

\הגדרה{ע״ע בטול. ע״ע כניעה.}

\paragraphs

\ערך{כלי }\הגדרה{- }\משנה{הכלי }\הגדרה{- בית הקיבול, החומר, הגשם המגביל, שהוא מוכשר לקבל בתוכו }\מקור{[ר״מ קכט]}\צהגדרה{. }

\paragraphs

\ערך{כלל }\הגדרה{- }\משנה{הכלל }\הגדרה{- הקיבוץ}\myfootnote{ \textbf{הכלל, הקיבוץ} - ביחס שבין הכלל, במובן זה של קיבוץ, לפרט, כותב הרב בע״א ג ב נה: ״כאשר נחדור לעומק ענין החיים נדע שהכלל אם שהוא נכבד מהפרט מפני ריבויו, אבל עצם מעמד הכלל הלא הוא הפרט, יסוד כל התנועות הכלליות הוא להיטיב ולשכלל את הפרטים הרבים הגנוזים בתוך הכלל הגדול. כי חיי הפרט הרם והנשא הלא הוא הוא כל המגמה והתכלית של הכלל״. ״כי למה ניתן לנו כל העמל הכללי, כדי להביא את הפרטים לאושר גדול, לשלמות אמתית בדעת ובהנהגה״. ״כי הלא זה הפרט הוא הוא היסוד הכללי״. ע״ע ע״ט נא ד״ה יש, וד״ה כשמתעוררת. א״ב פרק ג, ושם פרק יג (א״ה מהדורת תשס״ב עמ׳ 106). ע״א א א קכ, שם ב ח ו, שם ג ב מט. ע״ר א תטו. אג׳ ג יב\hebrewmakaf יג. א׳ קסה\hebrewmakaf ו, קיז. ש״ק, קובץ א תרנ. ע״ע כזרי ח״ב מד, נו. \label{10}}\הגדרה{ }\מקור{[קובץ א קלג]}\צהגדרה{. }

\הגדרה{הסביבה והקיבוץ והמציאות בכללה }\מקור{[ע״א ב ט קה]}\צהגדרה{.}

\paragraphs

\ערך{כלל }\הגדרה{- הכח הלאומי }\מקור{[מ״ר 217]}\צהגדרה{. }

\הגדרה{האומה }\מקור{[א״ק ג קנט]}\צהגדרה{. }

\הגדרה{אומה שלמה מצד כללותה }\מקור{[ל״ה 211]}\צהגדרה{.}

\הגדרה{חיי הכלל והאומה }\מקור{[ל״ה 133]}\צהגדרה{. }

\הגדרה{ע׳ בנספחות, מדור מחקרים, צבור ושותפין. }

\paragraphs

\ערך{כלל }\הגדרה{- כללות נשמת\hebrewmakaf האומה\mycircle{°} כולה }\מקור{[א״ת יג ג]}\צהגדרה{. }

\משנה{כח הכלל של הנשמות }\הגדרה{- כללות קדושת\mycircle{°} האומה, כנסת\hebrewmakaf ישראל\mycircle{°} }\מקור{[עפ״י אג׳ א שסט]}\צהגדרה{. }

\הגדרה{ע״ע נשמת האומה, נשמת הכלל.}

\paragraphs

\ערך{כלל }\הגדרה{- }\משנה{הכלל }\הגדרה{- המציאות בכל היקיפה }\מקור{[א״ק ב שפג]}\צהגדרה{. }

\paragraphs

\ערך{כלל }\הגדרה{- המקור של הפרטים }\מקור{[ע״ט קכב]}\צהגדרה{. }

\הגדרה{מקורו ומוצאו של הפרט }\מקור{[פנק׳ ב רט]}\צהגדרה{.}

\משנה{כללות }\הגדרה{- הצינורות\mycircle{°} שהפרטים יונקים מהם את ליחם }\מקור{[פנק׳ ג שכז]}\צהגדרה{. }

\paragraphs

\משנה{כלל }\הגדרה{- }\צהגדרה{יצירה אלקית יסודית מקורית ושמיימית, שמתגלה בריבוי פרטים }\צמקור{[שי׳ ת״ת 19].}

\paragraphs

\ערך{כלל }\הגדרה{- האור\mycircle{°} האצילי\mycircle{°}, הנובע ממקור האורה\hebrewmakaf העליונה\mycircle{°}, שהוא יסוד הכל, ומכון ההיות של הכל }\מקור{[ע״ר א קפג]}\צהגדרה{. }

\הגדרה{חפץ\hebrewmakaf ד׳\hebrewmakaf העליון\mycircle{°} בכל יצוריו, בכל העולמות\mycircle{°} שברא}\myfootnote{ \textbf{כלל }\textbf{-}\textbf{ חפץ ד׳ העליון בכל יצוריו וכו׳} - י׳ מאמרות לרמ״ע, מאמר המדות, מדה ד ״הנה הכתר כלל... וידוע בשפת אמת שהכתר הוא הרצון״.\label{11}}\הגדרה{ }\מקור{[שם קפא]}\צהגדרה{. }

\paragraphs

\ערך{כלל ישראל }\הגדרה{- האומה, כנסת\hebrewmakaf ישראל\mycircle{°} }\מקור{[ע״ט יח]}\צהגדרה{. }

\צמקור{הקדושה האלהית הכללית של עם ישראל [שי׳ ה 34].}

\משנה{קדושת הכלל }\הגדרה{- כח הלאומי המיוחד הישראלי התלוי בארץ\hebrewmakaf ישראל\mycircle{°} }\מקור{[ע״א ב ט שכו]}\צהגדרה{. }

\צהגדרה{נשמת\hebrewmakaf האומה\mycircle{°}. עם ישראל שהוא עם מיוחד, בעל סגולה\mycircle{°} ישראלית, סגולה מיוחדת }\צמקור{[שי׳ ב 305]. }

\הגדרה{ע׳ בנספחות, מדור מחקרים, כלל ומדינה, ערכם בגויים ובישראל.}

\paragraphs

\ערך{כלל ישראל האמיתי}\myfootnote{ באג׳ ג קעח: ״העולם הכשר ורוחו, הוא יסוד האומה כולה״. ושם רסב: ״תוכן הקודש המתגלה ע״י אור התורה וחיי הכשרות של שלמי אמונים בישראל, זהו יסוד האומה כולה, וכל התנועות והמפלגות הנן ענפים לה, בין כשהם מודים ביחוסם אליה, או שהם כופרים בזה וכו׳״.\label{12}}\הגדרה{ - החלק הכשר, היותר עמוק ולבבי של אומתנו, אותו החלק שהוא חי, חושב, מרגיש, ומאמין, ביהדות מחוורה, אשר לפי האמת הנה הוא ורק הוא הנו הערובה הנאמנה על קיומה האמיתי של אומתנו בצביונה הברור ובאופי העצמי הטהור שלה }\צמקור{[אג׳ ג קצא].}

\paragraphs

\ערך{כללות }\הגדרה{- הצבור האנושי בעמקו והיקף גדלו }\מקור{[א״ק ג ב]}\צהגדרה{. }

\paragraphs

\ערך{כללות }\הגדרה{- }\משנה{הכללות }\הגדרה{- המגמה\mycircle{°} ההויתית }\מקור{[קובץ ו סו]}\צהגדרה{. }

\paragraphs

\משנה{כלליות }\צהגדרה{- המציאות, הטבעיות, החיים }\צמקור{[צ״צ קפב]. }

\הגדרה{ע״ע כלליות.}

\paragraphs

\ערך{כללי }\הגדרה{- }\משנה{(אדם כללי, לעומת פרטי\mycircle{°}) }\הגדרה{- מדיני, משותף בחיי החברה ופועל עליה }\מקור{[ע״ר א רכא]}\צהגדרה{. }

\הגדרה{ע׳ במדור פסוקים ובטויי חז״ל, אדם מדיני בטבע. }

\paragraphs

\ערך{כללי }\הגדרה{- מקיף את האנושיות וכל ההויה }\מקור{[עפ״י א״ש יג ב]}\צהגדרה{.}

\הגדרה{כלל אנושי, (אוניברסאלי) }\צהגדרה{[עפ״י א׳ י, מ״ר }\צמקור{97\hebrewmakaf 96}\צהגדרה{].}

\צמשנה{כללית - }\הגדרה{אנושית מקפת }\מקור{[ע״א ד ו צו]}\צהגדרה{.}

\הגדרה{ע״ע פרטי. }

\paragraphs

\ערך{כללי }\הגדרה{- }\משנה{היסוד הכללי }\הגדרה{- ׳צרורא דלעילא דביה חיי כולא׳\mycircle{°}. הכללות, כללות הכל, כללות העולם, האדם, כללות\hebrewmakaf ישראל\mycircle{°}, כל היקום }\מקור{[עפ״י א״ק ג קמז]}\צהגדרה{. }

\paragraphs

\ערך{כללי }\הגדרה{- אלהי\mycircle{°} }\מקור{[עפ״י ע״ר א טו]}\צהגדרה{. }

\הגדרה{אמיתי\mycircle{°} }\מקור{[ע״א ג ב קסו]}\צהגדרה{. }

\צהגדרה{פנימי, אלהי}\צמקור{ [א״ל רכט].}

\הגדרה{ר׳ פרטי.}

\paragraphs

\ערך{כלליות }\הגדרה{- }\משנה{(באדם)}\הגדרה{ - הכנסת האדם את עצמו בחיי הכלל. כשהאדם שוכח מעט את עצמו, את פרטיותו, והטוב הכללי לוקח את לבבו. כשהגורם העיקרי בדחיפת החיים המוסריים\mycircle{°} אצלו היא התשוקה\hebrewmakaf האידיאלית\mycircle{°}, לחיים שיש בהם תוכן מדעי ומוסרי במלא מובנו }\מקור{[עפ״י א״ק ב תקט, ושם ג שכא\hebrewmakaf שכב]}\צהגדרה{.}

\paragraphs

\ערך{כלליות }\הגדרה{- ע״ע כללות. }

\paragraphs

\משנה{כלליות }\צהגדרה{- משיחיות }\צמקור{[שמעתי מהרצי״ה].}

\paragraphs

\ערך{כלליות }\הגדרה{- העליונות\mycircle{°} האלהית\mycircle{°} }\מקור{[פנק׳ ד נב (א״ה 913)]}\צהגדרה{. }

\הגדרה{האלהות\mycircle{°} }\מקור{[א׳ קכה]}\צהגדרה{. }

\paragraphs

\ערך{כללים}\הגדרה{ - תעופות של הארות רעיונות }\מקור{[אג׳ א קסד]}\צהגדרה{.}

\paragraphs

\ערך{כללית }\הגדרה{- }\משנה{אומה כללית }\הגדרה{- ע״ע אומה כללית. }

\paragraphs

\ערך{כמיהה}\הגדרה{ - (}\צהגדרה{כמהון אלהי}\הגדרה{) - פנית הנשמה אל הרוממות האלהית העליונה. התשוקה אל הגודל והאור\mycircle{°}}\צהגדרה{ }\מקור{[ע״ר א קנט\hebrewmakaf קס]}\צהגדרה{.}

\הגדרה{ע״ע עריגה אלהית. ע״ע צמאון אלהי.}

\paragraphs

\ערך{״כנור״ }\הגדרה{- נימי נשמת\mycircle{°} אדם <ששתי ההשפעות - זו שמתחילה בזרמה מהציור הדמיוני והולכת עד השכל היותר טהור, ועוד הגבה למעלה יותר מהשכל ושרשיו, וממלאת באושר האלהי את כל החיים של כל שרשי הנשמה, באדם ובעולם; וזו שמתחילה מהופעה שכלית, או עוד עליונה ממנה, והיא מתפשטת והולכת עד הציור הדמיוני, וענפיו ההרגשים הגופניים כולם - פוגשות בו זו את זו. שהולכת הזרמים והובאתם מנעימות עליהם נועם קולם, קול עז> }\מקור{[עפ״י א״ק א רמא]}\צהגדרה{. }

\הגדרה{ע׳ במדור מלאכים ושדים, שיר המלאכים. ע׳ במדור פסוקים ובטויי חז״ל, סולם שמלאכי אלהים עולים ויורדים בו. }

\paragraphs

\ערך{כניעה}\myfootnote{ ע׳ חובות הלבבות שער הכניעה.\label{13}}\הגדרה{ - }\משנה{ההכנעה מפני האלהות }\הגדרה{- ההתבטלות\mycircle{°} בפני הכללות\mycircle{°} שהיא הטבע שבכל נברא\mycircle{°}, - ק״ו כאשר ההתבטלות היא בפני מקור ההויה של הכללות, שהוא חש בו עליונות של אין קץ לגבי הכלליות, - שבה עונג\mycircle{°} וקוממיות, שלטון וגבורה\mycircle{°} פנימית מעוטרת בכל יפי\mycircle{°}. שעבודם הטבעי של החיים, הקטנתה הטבעית של הנשמה\mycircle{°} לפני יוצרה. ההתבטלות הנתבעת מכל זויות הנשמה, מכל כללותה ופרטיותיה, כשגדולת\hebrewmakaf האלהות\mycircle{°} מצטירת יפה בתוך הנשמה }\מקור{[עפ״י א׳ קכה\hebrewmakaf ו]}\צהגדרה{.}

\משנה{ענין ההכנעה לפני אדון\hebrewmakaf כל\mycircle{°} ב״ה}\צהגדרה{ - }\הגדרה{הכנעת הגוף וכחותיו לפני אור\hebrewmakaf העליון\mycircle{°} של הקדושה\hebrewmakaf העליונה\mycircle{°} שבכלל, <יסוד ההדרכה הכללית הראויה לכל אדם החפץ להתדבק בקדושה העליונה, להיות מתהלך\hebrewmakaf את\hebrewmakaf ד׳\mycircle{°}> }\מקור{[ע״א ד ו ג]}\צהגדרה{.}

\הגדרה{ע״ע כלות הנפש לאלהים. ע״ע בטול.}

\paragraphs

\ערך{כנסת ישראל}\myfootnote{ ע״ע במדור מונחי קבלה ונסתר, שכינה, ובהערה על שכינה - כנסת ישראל.\label{14}}\הגדרה{ - אור השכינה\mycircle{°}, האידיאל\mycircle{°} הישראלי השורה באומה\mycircle{°} כולה, העושה אותה לחטיבה\mycircle{°} אחת בכל דורותיה. האידיאל הלאומי הנצחי\mycircle{°}, מהוד\mycircle{°} מקורו האלהי\mycircle{°}, השוכן בישראל ומקושר עם נשמתם המצוחצחה }\מקור{[א׳ קמ]}\צהגדרה{. }

\הגדרה{חטיבת השכינה העליונה\mycircle{°}, לאלהי כל הארץ מעולם ועד עולם }\מקור{[אג׳ ב שלד]}\צהגדרה{. }

\הגדרה{החטיבה האלהית בחיי האדם והעולם }\מקור{[א״ק ג קפט]}\צהגדרה{. }

\הגדרה{החטיבה העולמית שסגולת התשובה\mycircle{°} תתגלה בה תחלה. <היא נדחפת להיות מתואמת עם האורה\hebrewmakaf האלהית\mycircle{°} בעולם, שאין בה חטא\mycircle{°} ועון\mycircle{°}, משום הרגשתה הרוחנית העודפת; ואור התשובה יופיע בה בראשונה ואח״כ תהיה היא הצנור\mycircle{°} המיוחד המשפיע את לשד החיים של תאות התשובה העדינה על כל העולם כולו, להאיר אותו ולקומם מצבו> }\מקור{[עפ״י א״ש ה ח]}\צהגדרה{.}

\הגדרה{שורש נשמתם של ישראל, ספירת המלכות\mycircle{°} }\מקור{[פנק׳ ד תמ]}\צהגדרה{.}

\הגדרה{נוקבא\mycircle{°} }\מקור{[א״ק ב תו]}\צהגדרה{.}

\הגדרה{מרכז האנושיות במובן הרוחני, ״הכלה״\mycircle{°}, ״דכלילא מכל גוונין״}\myfootnote{ \textbf{״}\textbf{דכלילא}\textbf{ מכל גוונין״} - ת״ז י. בקהילות יעקב ערך כנסת ישראל: ״כנסת ישראל נודע דבחינה זו נקראת מלכות שכונסת ישראל דלעילא אליה והיא בית לו (זוהר בלק קצ״ז.), גם ישראל דלתתא נשמותיהן מכונסין בה מתחילת הבריאה, וגם בכל לילה חוזרין לתוכה בפקדון״.\label{15}}\צהגדרה{ }\מקור{[קובץ א כו]}\צהגדרה{. }

\הגדרה{המפתח הקטנה של פלטרין האחוזה בשלשלת הגדולה, של שם\hebrewmakaf ד׳\hebrewmakaf האלהים הגדול}\myfootnote{ ירושלמי תענית י:\hebrewmakaf יא. ״שיתף הקדוש ברוך הוא שמו הגדול בישראל. למלך שהיה לו מפתח של פלמנטריא קטנה אמר המלך אם אני מניחה כמות שהיא אבידה היא. אלא הריני עושה לה שלשלת שאם אבדה, השלשלת תהא מוכחת עליה. כך אמר הקדוש ברוך הוא, אם מניח אני את ישראל כמות שהם נבלעין הן בין העכו״ם, אלא הרי אני משתף שמי הגדול בהם והן חיים. מה טעמא ״וישמעו הכנעני וכל יושבי הארץ ונסבו עלינו והכריתו את שמנו מן הארץ ומה תעשה לשמך הגדול״ שהוא משותף בנו״.  \label{16}}\הגדרה{ }\מקור{[ע״ה קנד]}\צהגדרה{. }

\משנה{כללות כנסת ישראל }\הגדרה{- דיוקן\mycircle{°} הקדושה האלהית של כל ההויה כולה, אור נשמת ד׳\mycircle{°} בעולם }\מקור{[קבצ׳ א קעה]}\צהגדרה{.}

\מעוין{◊ }\משנה{כנסת ישראל}\הגדרה{ היא מוכנת בטבע נשמתה לתפוש ברוחה את האיווי של הגדולה\mycircle{°} האלהית, המתיחסת רק לאל\mycircle{°} אמת\mycircle{°} אשר לגדולתו אין חקר }\מקור{[ע״ר א רו]}\צהגדרה{. }

\הגדרה{נקודה\mycircle{°} אחת מיוחדה בחיובים נעלים, הראויים להמשיך עז הקודש\hebrewmakaf העליון\mycircle{°} בעולמים כולם }\מקור{[ע״א ד ט קלז]}\צהגדרה{. }

\הגדרה{בית לד׳, היכל מלך מלכי המלכים, מילוי השם הקודש באורותיו, אשתו כגופו, אחותי רעיתי יונתי תמתי, מלכת עולמים, השלמת ההופעה האורית החיה והמפוארה, הטהורה והעזיזה, בכל הנשמות, בכל הבריות, בכל העולמים. האידיאה\hebrewmakaf הלאומית\mycircle{°} הישראלית החודרת בעצמת חייה בכל פרט ופרט מישראל ובכל מעשיו ותנועותיו, שיחיו ושיגיו, שאיפותיו וקניניו הפרטיים }\מקור{[עפ״י קובץ ו קמג]}\צהגדרה{. }

\הגדרה{תמצית הנשמה המאוחדת של ישראל הנובעת מאור המאוחד אשר לתורה הכלולה ומאוחדת בקדושתה העליונה בקול\hebrewmakaf אלהים\hebrewmakaf חיים\mycircle{°}, שאור ד׳\hebrewmakaf אחד\mycircle{°} מופיע בה בבליטה מחורת, באש שחורה על גבי אש לבנה, והיא שמה כל מעינה במקור חייה זה }\מקור{[עפ״י א״ת ד א]}\צהגדרה{.}

\הגדרה{שם\hebrewmakaf ד׳\mycircle{°}, מלך ישראל, מלכות\hebrewmakaf שמים\mycircle{°}. עצם הופעת הפעולה הכללית, הפעולה הממשית וכל דרכיה, של אותה הארה\mycircle{°} של הנהגת העולם המטהרת\mycircle{°} את הנשמות מיסודן ועוקרת את כל הטומאה\mycircle{°} והעבודה\hebrewmakaf זרה\mycircle{°} מן העולם }\מקור{[עפ״י קובץ ו רעג]}\צהגדרה{. }

\הגדרה{נשמת\hebrewmakaf העולם\mycircle{°}, נשמת העמים כולם, הודם\mycircle{°} תפארתם\mycircle{°} וברכתם\mycircle{°} }\מקור{[קובץ ז קסט]}\צהגדרה{.}

\משנה{כנס״י הכוללת כל ברוחה }\הגדרה{- ההיכל המקודש להשם\hebrewmakaf הגדול\mycircle{°} המאיר את העולם כולו בכבודו }\מקור{[מ״ר 21]}\צהגדרה{.}

\הגדרה{יסוד ההויה, (תמונת\hebrewmakaf כל\mycircle{°}) [בחינת כל\mycircle{°}] של כל הברואים כולם }\מקור{[עפ״י (חד׳ (מהדורת תשס״ח) פד) פנק׳ ג שסו]}\צהגדרה{.  }

\הגדרה{יסוד יסודה של כללות ההויה כולה בעומק תמציותה }\מקור{[עפ״י א״ק ב תק]}\צהגדרה{. }

\הגדרה{תמצית ההויה כולה, הנשפעת בעולם הזה באומה הישראלית ממש, בחומריותה\mycircle{°} ורוחניותה, בתולדתה ואמונתה\mycircle{°} }\מקור{[עפ״י א׳ קלח]}\צהגדרה{. }

\הגדרה{גולת הכותרת של ההויה בכללותה }\מקור{[א״א 160]}\צהגדרה{. }

\הגדרה{התמצית של הטוב\mycircle{°} והמעולה שבכל העולם כולו }\מקור{[א״ק ג שמט]}\צהגדרה{. }

\הגדרה{הגלוי הרוחני העליון שבההויה האנושית. גלוי זרוע\hebrewmakaf ד׳\mycircle{°} בעולם, יד\hebrewmakaf ד׳\mycircle{°} בהויה, בבנין הלאומים }\מקור{[א׳ קלח]}\צהגדרה{. }

\הגדרה{מרכז האנושיות <במובן הרוחני\mycircle{°}, המיוחד למשאות נפש המתרוממת למרומי האשר הטהור\mycircle{°}> }\מקור{[קובץ א כו]}\צהגדרה{. }

\הגדרה{אוצר הנשמה של הכלל\mycircle{°} כולו }\מקור{[א״ש תוספות תשובה ז]}\צהגדרה{. }

\הגדרה{יסוד נשמת\hebrewmakaf האומה\mycircle{°} בכללה }\מקור{[עפ״י א׳ קמא, אג׳ א עא, ושם ב שסה]}\צהגדרה{. }

\הגדרה{נשמת האומה }\מקור{[ע״ר ב קנח, מ״ר 283 (ח״ה קא)]}\צהגדרה{. }

\הגדרה{נשמת האומה, אור\hebrewmakaf החיים\mycircle{°}, האצילות\mycircle{°}, הזוהר\mycircle{°}, היופי\mycircle{°}, הטוהר\mycircle{°}, הפאר\mycircle{°}, הקודש\mycircle{°}, הגבורה\mycircle{°}, הדעה, החופש\mycircle{°}, העז\mycircle{°}, והענוה\mycircle{°} הישראלית, המדע והחכמה הישראלית, אור האמונה, דעת העולם והחיים הישראלים, מהות החיים הישראליים ממקורם מיסוד הוייתם }\מקור{[קובץ ו קמב]}\צהגדרה{. }

\הגדרה{כללות קדושת האומה }\מקור{[אג׳ א שסט]}\צהגדרה{. }

\הגדרה{הרוחניות\mycircle{°} הישראלית של כל הדורות ושל כל העולמים\mycircle{°} }\מקור{[ע״ר ב שה]}\צהגדרה{. }

\הגדרה{הטבעיות העצמית של האומה הישראלית }\מקור{[א״ק ג קיז]}\צהגדרה{. }

\הגדרה{טבע האומה במהותה הפנימית }\מקור{[א׳ קמא]}\צהגדרה{. }

\מעוין{◊}\הגדרה{ החפץ לרומם את הכלל, את האומה כולה, את כל העולם, את כל ההויה, את הנשמות כולן, את כל החושים, הנטיות, לאחד את כל העולמים, לנצח את המות, להעשיר את חיי החיים ממרומי מקוריותם, כל החפצים ההם וכל מה שלמעלה מהם, בלא פרוד וקיצוץ, קבועים בנשמת הכלל, בצורת כנסת ישראל, בדמות\hebrewmakaf דיוקנו\hebrewmakaf של\hebrewmakaf יעקב\mycircle{°} }\מקור{[א״ק ב רפט (ע״ט ג)]}\צהגדרה{. }

\משנה{הכנסיה הישראלית }\הגדרה{- חלקה פסיכית שהזקוק המוסרי והשכלי, זקוק הבא עם יחס אבות וכור הברזל של עני מצרים, פעל הרבה על ההסטוריה שלה, עד אשר הוכנה לקלט בקרבה את ההשפעה האלהית המחלטת בתוקף הויתה בחיי הכנסיה ובנימוסיה, בשאיפותיה ותקוותיה, ותשרש שורשיה בנשמות יחידיה וכללותה לדורות עולמים }\מקור{[מ״ר 3]}\צהגדרה{. }

\משנה{כנסת ישראל - העצמיות הפנימית של כנסת ישראל }\הגדרה{- התשוקה הבוערה להשתעשע במקור חייה }\מקור{[קובץ ו רכז]}\צהגדרה{.}

\משנה{כנסת ישראל בעצמה, בתוכיות נשמתה }\הגדרה{- אור האלהים בעולם }\מקור{[קבצ׳ ג צד]}\צהגדרה{.}

\משנה{הפועל הגדול של כנסת ישראל בעולם }\הגדרה{- הקדושה\mycircle{°} בתוך החיים, להחיות את כל החיים, שיהיה ״הודו על ארץ ושמים״ }\מקור{[פנק׳ א רסה]}\צהגדרה{.}

\מעוין{◊}\הגדרה{ אין להגדיר את מהותה של }\צהגדרה{כנסת ישראל }\הגדרה{בגבולים מיוחדים ובתוארים מוגבלים. כוללת היא את הכל, והכל מיוסד על כלות\hebrewmakaf נפשה\hebrewmakaf לאלהים\mycircle{°} }\מקור{[א׳ קלח]}\צהגדרה{. }

\הגדרה{ע׳ במדור מונחי קבלה ונסתר, קיר. ושם, שושנה עליונה. ושם, מלכות (ספירה). ע״ע ישראל. ע׳ במדור שמות כינויים ותארים אלהיים, קב״ה. }

\paragraphs

\משנה{כנסת ישראל}\myfootnote{ ע׳ בנספחות, מדור מחקרים, אידיאה אלהית ואידיאה לאומית.\label{17}}\הגדרה{ }\צהגדרה{- האידיאה הלאומית}\צמקור{ [שי׳ ב 235].}

\paragraphs

\ערך{כנסת ישראל נחתת לאשראה בארעא}\myfootnote{ עפ״י זוהר ח״ב קמג:\label{18}}\הגדרה{ - האידיאל הלאומי הנצחי, מהוד מקורו האלהי, השוכן בישראל, המקושר עם נשמתם המצוחצחה, דופק ופועם, ומכה גלים בים החיים}\צהגדרה{ }\מקור{[א׳ קמ]}\צהגדרה{.}

\paragraphs

\ערך{כנסת ישראל }\הגדרה{- האומה\mycircle{°} ישראל}\צהגדרה{ }\מקור{[עפ״י א״ה ו 167]}\צהגדרה{.}

\הגדרה{האומה האלהית }\מקור{[קובץ ד מט]}\צהגדרה{. }

\הגדרה{האומה, הלוחמת מלחמת ד׳ והופעת הטהרה\mycircle{°} של האורה\hebrewmakaf האלהית\mycircle{°} בעולם }\מקור{[קובץ ח קפח]}\צהגדרה{. }

\הגדרה{אומת המחשבה\mycircle{°} האצילית\mycircle{°} בעולם }\מקור{[ע״ט קלג (א״א 61)]}\צהגדרה{. }

\הגדרה{עדה המיוחדת לקשר את כל ההויה אל החפץ הקדוש\mycircle{°} העליון - החפץ\hebrewmakaf האלהי\mycircle{°} הכללי בעולמיו כולם }\מקור{[עפ״י ע״ר א קנח]}\צהגדרה{.}

\משנה{כנסת ישראל בתור עם ד׳}\הגדרה{ - הנושא של הטוב\hebrewmakaf הגדול\mycircle{°} בתכונת נצחיותו}\צהגדרה{ }\מקור{[ע״ר א צט]}\צהגדרה{.}

\משנה{כנסת ישראל בכלל}\הגדרה{ - ״האם״, הנוסדת על בניה לימודי\hebrewmakaf ד׳\mycircle{°} אלו תלמידי\hebrewmakaf חכמים\mycircle{°} [}\צהגדרה{ע״א ג א טז].}

\הגדרה{האומה הסופגת\mycircle{°} את הכל }\צהגדרה{<הנוחה לקבל ולהסתגל לרוחם של כל עם ולשון> }\הגדרה{שיסוד החיים שלה בנוי מיסוד הדעה\hebrewmakaf האלהית\mycircle{°} היותר עליונה, בבחינה זו שהיא מאירה על הדעת הלאומית שלה ברום מעלתה. }\צהגדרה{<היא איננה אומה בפני עצמה, כפי אותו המובן של הלאומיות היבשה העומדת בלא ההארה של הדעה\hebrewmakaf האלהית בקרבה, שבכל עם ולשון, אבל היא>}\הגדרה{ התמצית הכנוסה של כל מה שהוא ראוי להקבץ לטובה מכל העמים שתחת כל השמים}\צהגדרה{ }\מקור{[מ״ר (מהדורת תשע״ו) 210 (קבצ׳ ג עג-עד)]}\צהגדרה{.}

\מעוין{◊ }\משנה{כנסת ישראל}\הגדרה{ איננה אומה כפי המובן הרגיל אלא התמצית האידיאלית של האדם, המופעה בתור קיבוץ חברתי ממולא בכל תכסיסיו, שנקרא ״לאום״ בדרך\hebrewmakaf השאלה, מפני שכל הקיבוצים המיוחדים שבבני\hebrewmakaf אדם נקראים כן. את הופעותיה המרובות מגלה היא בתקופות שונות בגוונים\mycircle{°} שונים. יש מהן שהיא מגלה אותן ע״י עצמה, ויש מהן שגילויה יוצא אל הפועל ע״י חלקים אחרים של האנושיות - ע״י גרמתה. והיא שואפת תמיד להתעלות\mycircle{°} למדרגה\mycircle{°} עליונה זו של רוחב\hebrewmakaf נשמה, שלא תצטרך עוד לפזר את כוחותיה בהופעות מפורדות פעם אחר פעם ומקום זולת מקום, כי\hebrewmakaf אם שכולם יתגלו בתוכה בהופעה ברורה ובולטת בבת אחת, בצורה של יצירת היסתוריה חדשה. אז ״ילכו גוים לאורה ומלכים לנגה זרחה״ ״ולה יקרא שם חדש אשר פי ד׳ יקבנו״ }\מקור{[אג׳ ב סה\hebrewmakaf ו]}\צהגדרה{. }

\הגדרה{ע׳ בנספחות, מדור מחקרים, צבור ושותפין.}

\paragraphs

\ערך{כנסת ישראל }\הגדרה{- }\משנה{(לעומת תפארת\hebrewmakaf ישראל\mycircle{°})}\הגדרה{ - מורה התכלית והשאיפה האלהית בכל המון גלגולי המקרים, לבנות את בית ישראל, לכוננו ולסעדו בשלמות גדולה ומופלאה, שיהיה סגולה מכל העמים, ויודע דעת\hebrewmakaf האלהים\mycircle{°} ידיעה אמתית, בא לתכונת האדם בצד השלמות היותר עליונה. וכל זה בטפוס המיוחד לאומה הישראלית. <וכל מעשי התורה\mycircle{°} והמצוה\mycircle{°}, הכל הכנה גדולה ורבת עליליה היא אל התכלית המעולה הזו לרומם קרן כנסת ישראל}\צהגדרה{> }\מקור{[פנק׳ א נח]}\צהגדרה{.}

\ערך{כנסת ישראל עולה למעלה מתפארת\hebrewmakaf ישראל\mycircle{°}}\myfootnote{ בפנק׳ ד תמ: ״שלמות הז״א, שהוא בחי׳ דכר היא שעל ידו נשפע כל טוב ושלימות לכל העולמות. ושלימות המל׳, שהיא מקבלת, וממילא מתמלאים כל העולמות אורה, כי בברכתה יבורכו, א״כ שלמותה בעלייתה היא עטרת לשלמות ז״א, שתכלית השפע בכל העולמות היא שעי״ז יהיה נשפע ענין קבלתה״. ע״ש תלט-תמא כל העניין. \label{19}}\הגדרה{ - תכלית ישראל לעצמם מצד איכותם נשגב יותר מתכלית הכללי של שלימות כל מין האנושי, ״אשת חיל עטרת בעלה״ }\מקור{[פנק׳ א טז (ב״ר שכו)]}\צהגדרה{.}

\paragraphs

\ערך{כנסת ישראל הטבעית }\הגדרה{- החמרית\mycircle{°}, המעשית, והחיצונית\mycircle{°} }\מקור{[עפ״י א״א 34]}\צהגדרה{. }

\הגדרה{ע״ע יהדות טבעית. ע׳ בנספחות, מדור מחקרים, כנסת ישראל ניסית וטבעית וכו׳. ע״ע חול.}

\paragraphs

\ערך{כנסת ישראל הניסית }\הגדרה{- הרוחנית\mycircle{°}, המוסרית\mycircle{°}, והפנימית\mycircle{°} }\מקור{[עפ״י א״א 34]}\צהגדרה{. }

\הגדרה{ע״ע יהדות ניסית. ע׳ בנספחות, מדור מחקרים, כנסת ישראל ניסית וטבעית וכו׳. ע״ע קודש.}

\paragraphs

\ערך{כנף }\הגדרה{- }\משנה{כנפיים }\הגדרה{- הכנפיים הן החיצונות המתפרטות להוציא אל הפועל את הפעולה של העפיפה\mycircle{°} }\מקור{[ע״א ב 242]}\צהגדרה{.}

\הגדרה{ע״ע אבר.}

\paragraphs

\ערך{כסות }\הגדרה{- }\משנה{(לעומת לבוש\mycircle{°}) }\הגדרה{- בגד עליון }\מקור{[ע״א א 78]}\צהגדרה{.}

\paragraphs

\ערך{כפיפה }\הגדרה{- }\מעוין{◊}\הגדרה{ מכווצת בתוכה את המהותיות, מונעת את התגלותה, אע״פ שהיא נמצאה בעינה בצביונה וערכה. תכונה זו היא מעוטת האורה\mycircle{°}, אבל עלולה היא לשימור מכל פגע, שיוכל להזדמן בתכונת הזקיפה\mycircle{°} }\מקור{[ע״ר א עג]}\צהגדרה{.}

\paragraphs

\ערך{כפירה }\הגדרה{- }\משנה{(בענין או בנושא) }\הגדרה{- אמונה באפסיות, בריקנות, בביטול ושלילות המוחלטה בעומק ההעדר }\מקור{[עפ״י א״א 23]}\צהגדרה{.}

\הגדרה{כל ענין מדעי, שמאיזו סבה אין לו דרישה אצל בני האדם, ואי הדרישה הגיעה לידי מדרגה גדולה, הרי הוא נחשב כאלו אינו, ובלשון הרגילה היכולה להתלבש בספרות יקרא לזה שלילה\mycircle{°} והעדר, או כפירה }\מקור{[מ״ר 10]}\צהגדרה{.}

\paragraphs

\ערך{כפירה }\הגדרה{- }\משנה{(שלילת האמונה\mycircle{°}) }\הגדרה{- מעמד נפש לפיו אין אידיאל\mycircle{°} במציאות הכללית }\מקור{[עפ״י א״א 39]}\צהגדרה{.}

\הגדרה{הקשחת הלב\mycircle{°} מהענין\hebrewmakaf האלהי\mycircle{°}, ונטיית החפץ מלהיות קשור במחשבה\mycircle{°} בהתוכן האלהי\mycircle{°}, המוביל לשקר\mycircle{°} המצוייר\mycircle{°} שהעולם הוא טומטום, תהו\mycircle{°}, באין נשמה, וכל הבנין, מלא חכמה חסד\mycircle{°} ותפארת\mycircle{°}, הוא פעולה מקרית }\מקור{[עפ״י קובץ א תג]}\צהגדרה{.}

\מעוין{◊ }\הגדרה{הסריסות הגמורה מכל שירה\mycircle{°}, ואי האפשרות להסתגל להאמת הגדולה מכל חקר פרוזאי }\מקור{[קובץ ג שח]}\צהגדרה{.}

\הגדרה{סילוק הדעה, הסחת הדעת מכל קודש\mycircle{°}. <אותה המגרעת המנוולת, שהיא מש(וּ)מרת לכלות ולבער באש חלאתה את כל דופי של של איזה הצטיירות (סמלונית), ע״י שלילותה העקשנית; שורפת את צחצוחי הטומאה\mycircle{°} של עבודה\hebrewmakaf זרה\mycircle{°}, של תמונה וסמל, שהתדבק בהתוכן המחשבי מתקופת התוהו\mycircle{°} של עבודה זרה> }\מקור{[עפ״י קובץ ח קפח]}\צהגדרה{.}

\משנה{כפירה, תעודתה}\הגדרה{ - הכנה שלילית לצרך העלוי העליון, שלא יהיה שום צרך לחשב על דבר אלהות, כ״א עצם החיים יהיה אור אלהים  }\מקור{[מ״ר 41]}\צהגדרה{.}

\הגדרה{להפיג את הפחדנות}\myfootnote{ ע״ע קבצ׳ ב ס 21.\label{20}}\הגדרה{ שהאמונה שאינה מזוקקת מביאה לעולם}\צהגדרה{ }\מקור{[קבצ׳ ב קלח 91]}\צהגדרה{.}

\הגדרה{להסיר את הצורות המיוחדות מהמחשבה המהותית של כל החיים ושורש כל המחשבות כולן (המחשבה על דבר המושג האלהי) }\מקור{[א׳ קכו]}\צהגדרה{.}

\משנה{ניצוץ הקודש היותר נשגב הטבוע במעמקי הכפירה }\הגדרה{- לעשות דברים לשם פעלם, את הטוב לשם הטוב\mycircle{°}}\צהגדרה{. }\משנה{הכפירה בשכר ועונש}\הגדרה{ מחנכת את הבריות לעשות טוב מצד עצם הטוב. }\צהגדרה{רעיון הכפירה}\הגדרה{ איננו כי אם אמצעי תרבותי לזה המצב}\צהגדרה{ }\מקור{[עפ״י קובץ א תרכז, קבצ׳ ב קסז 110]}\צהגדרה{.}

\משנה{ניצוץ הקדושה המחיה את הכפירה}\צהגדרה{ -}\הגדרה{ ניצוץ אהבת האמת המוטבע בלב האדם. מפני חפץ האמת איננו יכול לאמר קדוש לכל אשר הנחילוהו אבותיו, ע״כ הוא מפקפק וכופר }\מקור{[א״ה ו 154 (קבצ׳ ג צב)]}\צהגדרה{.}

\הגדרה{ע״ע אפיקורסות. ע״ע שלילה. ע״ע כופר. ע׳ במדור פסוקים ובטויי חז״ל, כפירים. ע׳ במדור מונחי קבלה ונסתר, כפור שמים. }

\paragraphs

\ערך{כפירה }\הגדרה{- }\משנה{הכפירה האפיקורית }\הגדרה{- דלדול מצד נטילת החיים, מיעוט החיים ושלילתם. מות, שיסודו הן הכפירות כולן. השקר. רוח מות שלא רק בצורה מופשטת שולטת, כי-אם בצורה חיונית. וכפי אותה התכונה שהחיים נערכים על פיה, והתשוקות הנפשיות מתמלאות ממאויה הפנימיים, משתלשלים דורות, ששאיבות נשמותיהם באות ממקור מושחת כזה, והמאויים הפנימיים ותשוקות החיים ההם של תוכני זוהמא של המות, עושים את היסוד לכל התוכן היצירתי גם בהערכים הפועלים של החיים והתולדה, ההתרבות וצדדיה, וכל גורמיה הפנימיים והחיצוניים נדחפים מתוך אותו עומק הרע. והמונים המונים של קיבוצים מתיצרים בערכי אמונה, בערכי ישוב, בערכי קיבוץ וחברה, בערכי תרבות ושלטנות, בערכי יופי ושירה, שתוכיותם היא אותו הלשד הרשעתי, הבא ממהומת המות }\מקור{[עפ״י קובץ ו רז-רט]}\צהגדרה{.}

\הגדרה{ע׳ בנספחות, מדור מחקרים, רשעה, מינות, נוצריות, נצרות.}

\paragraphs

\ערך{כפירה בתורה מן השמים}\myfootnote{ סנהדרין צט.\label{21}}\הגדרה{ - לאמר שמשה\mycircle{°} מפי עצמו אמרה (את התורה), בלא השפעה\mycircle{°} אלהית\mycircle{°}}\צהגדרה{ }\מקור{[עפ״י ל״ה 255]}\צהגדרה{.}

\צהגדרה{הכחשת אלהיותה של התורה\mycircle{°}. אי חדירה להכרת דברי\hebrewmakaf אלהים\hebrewmakaf חיים\mycircle{°} החובקים זרועות עולמות ודורות, המעריכה את התורה כספר חוקים או מוסר\mycircle{°} אנושי, שהיה מתאים למצבי הרוח של תקופות ידועות ולא אחר\hebrewmakaf כך בשנויי הזמנים והתגברות ההשכלה והתפשטותה }\צמקור{[עפ״י א״ל לט].}

\הגדרה{ע׳ במדור תורה, תורה מן השמים.}

\paragraphs

\ערך{כפירה בתורה שבעל פה}\myfootnote{ רמב״ם ה׳ תשובה פרק ג ח.\label{22}}\הגדרה{ - (כך נקרא) בכללות כל צד של בעיטה בדברי חכמים}\צהגדרה{ }\מקור{[עפ״י א״ה (מהדורת תשס״ב) ג 85]}\צהגדרה{.}

\הגדרה{ע׳ במדור מדרגות והערכות אישיותיות, כופר בתורה שבעל פה, כופר בפירושה. ושם, מכחיש מגידיה, בכפירה בתורה שבעל פה.}

\paragraphs

\ערך{כפרה }\הגדרה{- המעצור של כל פרץ בשטף החיים, והפיכת כל הכוחות הגנוזים, אשר יוצאים לפעמים בצורה של קלקול והריסה, הממיטים חרפה על נשמת\mycircle{°} האדם ואור\mycircle{°} קדושתו\mycircle{°} - לכוחות נשאים, שביכולתם הוא העז\mycircle{°} לעצור כל ענין רע\mycircle{°} ולפעול את המפעל של החיים האלהיים\mycircle{°} הטהורים\mycircle{°} במרוצתם, זהו יסוד הכפרה בצורה שנוטה לבטוי של כופר וכפורת }\מקור{[עפ״י ע״ר א קכח]}\צהגדרה{. }

\משנה{התכפרות עוונות }\הגדרה{- התעלות ניצוצי החיים שירדו\mycircle{°}, התרוממות הכוחות הירודים וההרגשות והרצונות, המחשבות והמעשים, הנגררים עמהם, עד כדי הימחקות רישומם והתהפכותם לזכות\mycircle{°} }\מקור{[עפ״י שם תז\hebrewmakaf ח]}\צהגדרה{.}

\מעוין{◊ }\משנה{הכפרה}\הגדרה{ באה ביחש לקלקול הרצון, המבטל את קשר האהבה, ומביא זעם ומשטמה תחת נעימות האהבה. מתוך הוספת החדוש של ההופעה\mycircle{°} התדירית של קדושת הנשמה, עובר הרושם הזה, ובמקום רוח של זעם בא רוח של נעם\mycircle{°}, מלא רגשי אהבה וחבת קדש\mycircle{°}, וממלא את כל הפגם אשר החסיר הרוח הזועף במרירותו, ורגשי הזעם, המתפרצים להרס מעשי, נשארים טמונים במעמקים, ששם אינם פועלים כ״א את המפעל הזועם הראוי, הפועל לסור מן הרע. ונטיה זו היא כמו כופר על הרצון אשר נתפרץ לקלקל, להוציא את המפעל הרע אל הפעל, וגם הוא מכסה וחוצץ כמו כפורת, בעד הנטיות הטמונות, שלא לפעול את הפעולות המהרסות, כי אם להחליף את צביונן ע״פ הוספת כח האור\hebrewmakaf האלהי\mycircle{°}, הבא ממקור הרחמים\mycircle{°} על שורש הנשמה האצילית, רק להרבות עז הקדושה ואור\hebrewmakaf החיים\mycircle{°} הטהורים, בעולם ובנפש, והעוון מתכפר }\מקור{[עפ״י שם קכו\hebrewmakaf ז]}\צהגדרה{.}

\משנה{קדושת הכפרה}\הגדרה{ - השבת הנפשות לקדושת\mycircle{°} מקורן }\מקור{[עפ״י שם קעא]}\צהגדרה{.}

\משנה{חק הכפרה התדירית }\הגדרה{- העלאת\mycircle{°} כל ערכי\mycircle{°} החיים למקור התמידי של תכונת חייהם }\מקור{[עפ״י שם קנד]}\צהגדרה{.}

\הגדרה{ע״ע מחילה. ע״ע סליחה. ר׳ מחילה סליחה וכפרה. ע׳ בנספחות, מדור מחקרים, מחילה סליחה וכפרה.}

\paragraphs

\ערך{כפרה }\הגדרה{- }\משנה{(כפרת\hebrewmakaf הדם בזריקתו על יסוד\hebrewmakaf המזבח) }\הגדרה{- ע׳ במדור משכן ומקדש.}

\paragraphs

\ערך{כפרה }\הגדרה{- }\משנה{ענין כפרת ראשי חדשים}\הגדרה{ - ע׳ במדור מועדים וחגים.}

\paragraphs

\ערך{כרת }\הגדרה{- כריתות הנפש, שגרמה להחריב את פעולת הדמיון\mycircle{°}, מהדברים היסודיים\mycircle{°}. כאשר לא נשמרה שלא יזוחו הדברים היסודיים הטובים\mycircle{°} מלעשות את התפשטות ענפיהם גם על הדמיון, כמו שהם מושרשים בשרשיהם בעומק השכל\mycircle{°} }\מקור{[עפ״י א״ק א רלה]}\צהגדרה{.}

\paragraphs

\ערך{כשוף }\הגדרה{- הוצאה מאוצר הרצון שבנפש, ע״י חדירה פנימית בהתלמדות סגולית\mycircle{°} פנימית לכך, את עזוז הכח הספון בקרב הנפש פנימה, המתגבר לסגולת אותם דבורים, שאוצרים בהם את תכני המחשבה והכח הללו, עד כדי פעולה אדירה על החוג הסובב משפעת כח הנפשי הכולל אשר בקרב המכשף }\מקור{[עפ״י ע״א ד ח מח]}\צהגדרה{.}

\ערך{כשפים }\הגדרה{- }\משנה{מעשי כשפים }\הגדרה{- המון מעשים רבים ותכסיסים שונים העשויים להפגיע את כל כחות המציאות הראויים לפעל מצד הדמיון\mycircle{°} הכללי כפי מדת ריחוקו של אותו המושפע מן האמת היותר עליונה. ההשלטה של כחות הדמיוניים, שהם שולטים בכללות בני אדם כל זמן שלא התעלו על רוממות המדרגה השכלית הטהורה הנאותה להם, ופוגעת בפרטים ע״י }\משנה{מעשי הכשפים }\מקור{[שם ג ב רלז]}\צהגדרה{.}

\משנה{מעשי כשפים וכל פלגותיהם וענפיהם }\הגדרה{- התמכרות אדם לגודל הרשעה\mycircle{°} והזוהמא\mycircle{°} והוצאה אל הפועל על ידם, בגודל המרץ הנפשי הפנימי באפלתו, את שלטונו האצור בקרבו על היקום הסובבו }\מקור{[עפ״י שם ד ח מח]}\צהגדרה{.}

\צהגדרה{באדם בכללותו, הגופני והנפשי, מקובצות הן התמציות של כל ההויה וכחותיהן. אמנם מקומצים הם בקרבו הכחות הגדולים, שמלא עולם עזם המתפשט. ע״י חדירה פנימית, בהתלמדות סגולית פנימית, להוציא אל הפועל את גדולת הכח, הספון בקרב הנפש פנימה, יש אופן להשים משטר ושלטון על החוג הסובב. המרץ הנפשי הפנימי גדול הוא בשתי צורותיו, באורו ובאפלתו. אין קץ לגודל הרשעה והזוהמא שיוכל האדם להתמכר לה ולהוציא אל הפועל בהם ועל ידם, את שלטונו האצור בקרבו על היקום הסובבו, והן הן }\צהגדרהמודגשת{מעשי כשפים וכל פלגותיהם וענפיהם }\צמקור{[שם].}

\צהגדרה{הכח המתפשט במציאות, המתגלם, ועלול אל התוהו והרשעה, נטיית הלב שבאדם מוצאת גם בו יחש גדול. רע\mycircle{°} מנעוריו הוא יצר\mycircle{°} לב האדם, לנפלאות יחשוק, להעמיק ברשע\mycircle{°}, לעולל עלילות נוראות, שהרבה יחריבו, ידאיבו, יבלבלו ויחשיכו. ומין הנאה של פועל, של שורר ומושל, הולכת ונצמחת מזה. והרשעה עושה לה כנפים ומתגדלת בלב\mycircle{°}, במעמקי הנשמה, ובמעמקי המציאות. ה}\צהגדרהמודגשת{כישופים וכל זוהמת מפעלם}\צהגדרה{ הנם פרי העמקה זו שמקורה הוא הרע שבמציאות המכוון לעומת ייצר לבו של האדם הרע, המתיחש(ים) זה אל זה ע״י יחושים שונים, שההעמקה של הרע המציאותי מוצאת בהם את המון חוקיה, כחותיה ומסילותיה וארחות לימודיה }\צמקור{[קובץ א קעב].}

\ערך{מכשף }\הגדרה{- אדם שהוא בפנימיות נטיות גופו ומזגו, נוטה לרעה, (ש)חית האדם שבקרבו, הרעה, נוטה היא אל ההסתכלות, וההתקשרות העצמית, אל צד החושך\mycircle{°} והבלבול והתוהו\mycircle{°} שבההויה. החפץ להחריב, לשבר, לכלות, אמיץ הוא מאד בתוכן רוחו. ועל פי משקל זה הולך הוא זרם המדע, וההשערה השכלית, ההצטיירות של הדמיון, וכל מערכי ההויה המתיחשת אליו, בקרבו וחוצה לו. כשהוא מסתכל בעמקי המחשכים שבמציאות, בראשי צוקין\mycircle{°}, שכל קלקולי החיים, כל הבלי השוא, וכל מרורות הרשעה, מהן יוצאין }\מקור{[עפ״י קובץ א קעג, קעב]}\צהגדרה{. }

\משנה{מקור הכשפים }\הגדרה{- כח השליטה האישית, הנוטה לצד הרשעה, <המיוסד על היסוד הדמיוני בעולם, שלא פחות מבאדם ונפשו, רבו עלילותיו ביקום> }\מקור{[ע״א ד ח מט]}\צהגדרה{. }

\הגדרה{ע׳ בנספחות, מדור מחקרים, כשפים לעומת קסמים. ע׳ במדור פסוקים ובטויי חז״ל, אמגושי.}

\paragraphs

\ערך{כתר }\הגדרה{- התכלית המקיפה הפרטים }\מקור{[מ״ש קכא (מא״ה ב טו)]}\צהגדרה{.}\mylettertitle{ל}

\paragraphs

\ערך{לאומיות החדשה }\הגדרה{- }\משנה{הלאומיות החדשה}\myfootnote{ ע״ע ש״ק, קובץ א תרנ.\label{1}}\הגדרה{ - הלאומיות, שהניצוצות המחיים את האומה מעולם ועד עולם, היורדים תמיד כטל חרמון על הררי ציון, משמי קדם של אמונה טהורה ואהבת דת קודש ותורת אמת, נדעכו ממנה }\מקור{[אג׳ א קפב]}\צהגדרה{.}

\paragraphs

\ערך{לאחור }\הגדרה{- }\משנה{ההנהגה האלהית שהיא לאחור }\הגדרה{- לצד ההתחסרות}\צהגדרה{ }\מקור{[עפ״י ע״ר ב סז]}\צהגדרה{.}

\הגדרה{ע״ע לפנים.}

\paragraphs

\ערך{לב }\הגדרה{- הרגש העליון, הרוח הנשגב, רוח\hebrewmakaf הקודש\mycircle{°} הכוללת השורה על ישראל\mycircle{°} ועל מלכותו\mycircle{°} }\מקור{[ע״א ד ה עא]}\צהגדרה{.}

\הגדרה{ע׳ במדור אישים, דוד.}

\paragraphs

\משנה{לב טוב }\צהגדרה{- שורש הטוב שבאדם, המתגלה בכל מהלך החיים שלו }\צמקור{[עפ״י שי׳ ת״ת 192].}

\הגדרה{ע׳ במדור מדרגות והערכות אישיותיות, טוב לב.}

\paragraphs

\ערך{לב טוב }\הגדרה{- הלב המלא אורה\mycircle{°} של צדק\mycircle{°} ומישרים\mycircle{°}, הלב אשר תורת\hebrewmakaf ד׳\mycircle{°} עשתה אותו ללב חי חיי\hebrewmakaf אמת\mycircle{°}, החש ומרגיש את הטובה האמיתית של חיים המפכים מקדושת\mycircle{°} דעת\mycircle{°} אל\mycircle{°} אמת וקדושתם של ישראל\mycircle{°} ונצחם\mycircle{°} }\מקור{[א״י נב]}\צהגדרה{. }

\paragraphs

\ערך{לב רע }\הגדרה{- הלב הזונה, הנשפל עד שפל כל רגש עכור, אשר לא יוכל לראות בגיאות\hebrewmakaf ד׳\mycircle{°}, להתרומם לחוג רעיונות נישאים בכלל, ולא יוכל להתנשא עד מרום קדושתם\mycircle{°} של ישראל\mycircle{°}, ולהרגיש עם זה חיים מלאים ובהירים }\מקור{[א״י נב]}\צהגדרה{. }

\paragraphs

\ערך{לבבו }\הגדרה{- }\משנה{לקח את לבבו }\הגדרה{- רוחניותו ואומץ חיל נשמתו }\מקור{[קובץ ח עח]}\צהגדרה{. }

\paragraphs

\ערך{לבוש }\הגדרה{- }\משנה{(לעומת כסות\mycircle{°}) }\הגדרה{- בגד תחתון }\מקור{[ע״א א 78]}\צהגדרה{. }

\paragraphs

\ערך{לבטא את השם\mycircle{°}}\הגדרה{ - לפרש את האור העליון }\מקור{[א״ק א עט]}\צהגדרה{.}

\paragraphs

\ערך{לבנה }\הגדרה{- }\משנה{(בחינתה שבאדם) }\הגדרה{- כל מה שמקבל (אדם) בתורה ומצות - לטוב, ובעבירות - להפכו, ע״י הצטרפותו לזולתו ואין לו ענין זה מצד עצמו ומעשיו וחכמתו }\מקור{[עפ״י מ״ש קעט (מא״ה א קכ)]}\צהגדרה{. }

\paragraphs

\משנה{לבצר}\הגדרה{ - }\צהגדרה{לצמצם למנוע }\צמקור{[הרצי״ה א״י כו].}

\paragraphs

\ערך{לדמות לדרכיו\mycircle{°}}\הגדרה{ - לקנות השלמות האמיתית }\מקור{[ע״א א א נג]}\צהגדרה{. }

\paragraphs

\ערך{להתגלם}\הגדרה{ - להתעטף בהגדרה מיוחדת }\מקור{[א׳ קלג]}\צהגדרה{.}

\paragraphs

\ערך{״להתענג על ד׳״}\הגדרה{ - ע״ע עונג, להתענג על ד׳.}

\paragraphs

\ערך{להתפאר }\הגדרה{- ע״ע פאר.}

\paragraphs

\משנה{לוגוס }\צהגדרה{- }\צמשנה{בפילוסופיה הדתית העברית של פילון האלכסנדרוני}\צהגדרה{ - יצור בעל אישיות אמצעיית מיוחדת להנהגת העולם ומחוברת ולא נפרדת מעצמיות האלהות, ונקראת וקבועה ג״כ בשמה ובכחותיה של זו האחרונה. כח בעל אישיות מיוחדה אלהית }\צמקור{[צ״צ קלא].}

\צהגדרה{כח הפועל וגובר בעולם האידיאות\mycircle{°} הנחות <שבו מסודר העולם המחוקה, קוסמוס,> כחוק, הנשמע  כמו }\צהגדרהמודגשת{לוגוס}\צהגדרה{, מאמר, אידיאה\hebrewmakaf דינמיס, כח, סבה פועלת, כסבה הכי פועלת, המתנועעת לעולם. השכל הטהור של הכל, המעולה על כל המעלות הטובות, על הטוב והיפה. <עולם האידיאות בתור כחות פועלים בסדר\hebrewmakaf עולם, הקוסמוס, אינו אלא במקום הלוגוס האלהי, האידאות הצפויות הן מעבר, סימבולון, ללוגוס, הנשמע מתוכן> }\צמקור{[עפ״י ק״ה נא].}

\הגדרה{ע׳ במדור פסוקים ובטויי חז״ל, שכל הפועל. ר׳ בנספחות, מדור מחקרים, לוגוס.}

\paragraphs

\משנה{לויה}\myfootnote{ ע׳ ישעיה יד א.\label{2}}\הגדרה{ }\צהגדרה{- התחברות חיובית של אהבה וקשור ממשי }\צמקור{[ק״ת סג].	}

\הגדרה{ר׳ הסתפחות.}

\paragraphs

\ערך{לחם }\הגדרה{- היסוד המזין, הממשיך את חיי האדם בצורתם המפותחת }\מקור{[מ״ר 159]}\צהגדרה{.}

\מעוין{◊}\הגדרה{ מזין ומקשר ע״י כחו הפלאי את נשמת\hebrewmakaf החיים\mycircle{°} עם הגויה וכחותיה }\מקור{[ע״ר א קכט]}\צהגדרה{.}

\הגדרה{ע״ע פת.}

\paragraphs

\ערך{לילה }\הגדרה{- }\מעוין{◊}\הגדרה{ עת\mycircle{°} שהטבעים החמריים\mycircle{°} מתגברים בעולם, עת שאור\mycircle{°} הנשמה\mycircle{°} האצילי\mycircle{°} נדעך מעט, מפני הגסות\mycircle{°} הפרושה על האדם ועל החי }\מקור{[ע״ר א קכב]}\צהגדרה{.}

\מעוין{◊ }\הגדרה{עוה״ז, או עת חשכת אור החכמה}\צהגדרה{ }\מקור{[פנק׳ ג כו]}\צהגדרה{.}

\מעוין{◊ }\משנה{מתיחד הלילה}\הגדרה{ - בבדידותו האישית המיוחדת של כל אחד ואחד }\מקור{[ע״ר א קעב]}\צהגדרה{.}

\הגדרה{ע״ע יום.}

\paragraphs

\ערך{לילה }\הגדרה{- }\משנה{חצי הלילה הראשון }\הגדרה{- הזמן של רשם היחידיות והבדידות הנפשית, אחרי עמל היום, ותנועת החברה ורשמיה }\מקור{[ע״ר א קעז]}\צהגדרה{.}

\מעוין{◊ }\הגדרה{עומק התכונה הבדידותית, היא קשורה ביסוד המחצית הראשונה של הלילה, בטבעה }\מקור{[שם קעב]}\צהגדרה{.}

\paragraphs

\ערך{לילה }\הגדרה{- }\משנה{מחצית האחרונה של הלילה }\הגדרה{- נקודת החצות היא המכריעה להתחיל את הקשר האטי, שהולך היחיד ומתנער לקראת חיי הצבור והחברה. ההכשרה שהיא באה בחצות הלילה נמשכת במחצית האחרונה של הלילה }\מקור{[עפ״י ע״ר א קעב]}\צהגדרה{.}

\הגדרה{המחצה המוכן להיות פונה ומתעתד לקראת ההתקנות החברתיות, העתידות לבא ביום }\מקור{[עפ״י שם קעט]}\צהגדרה{.}

\paragraphs

\ערך{לילה }\הגדרה{- }\משנה{״מדת לילה״ }\הגדרה{- ע׳ במדור פסוקים ובטויי חז״ל.}

\paragraphs

\ערך{לילה }\הגדרה{- }\משנה{״מדת לילה}\הגדרה{״, }\צהגדרה{עילוי מדת\hebrewmakaf לילה, שבאה אחר היום\mycircle{°}}\הגדרה{ - התגלות חושך\mycircle{°} גדל\hebrewmakaf ערך, <והנשמות מתענגות ע״י מה שמאיר להם אח״כ מתוכו> }\מקור{[ע״ר א תיא]}\צהגדרה{.}

\משנה{מדת הלילה ברוח }\הגדרה{- למעלה מההופעה השכלית, במקום שכח השכל כבר אפס, והציור\mycircle{°} המוגבל לא יוכל להגיע שמה, רק הנשמה בהגיון הפנימי שלה, הרי היא הולכת ומתאדרת, בתפארת הקודש החבוי של הדעת\hebrewmakaf הפנימית\mycircle{°}, המתעלה ממדת\hebrewmakaf היום\mycircle{°}, שאע״פ שהיא לכאורה כהה בבהירותה, היא בפנימיותה מלאה רחבות ונהורות מני אפל, בתכונה רוממה לאין קץ }\מקור{[ע״ר ב נג\hebrewmakaf ד]}\צהגדרה{.}

\הגדרה{ע׳ במדור מונחי קבלה ונסתר, ״סכלות מעט״.}

\paragraphs

\ערך{לימוד }\הגדרה{- }\משנה{חמשת חלקי עסק הלימוד }\הגדרה{- א. שיטה; ב. ריהטא; ג. גירסא; ד. לימוד; ה. עיון }\מקור{[א״ק א עג]}\צהגדרה{.}

\הגדרה{ע׳ במדור תורה.}

\paragraphs

\ערך{לימוד }\הגדרה{- }\משנה{האידיאל הלימודי }\הגדרה{- כבוד הדעת, אהבת החכמה, היסוד הפילוסופי, התשוקה האלהית\mycircle{°}, האהבה\mycircle{°} בתענוגיה\mycircle{°} }\מקור{[ר״מ קעח]}\צהגדרה{. }

\paragraphs

\משנה{״ללמוד אמונה באמונה״ }\צהגדרה{- יש ללמוד אמונה באמונה, כמו שלומדים עברית בעברית, מתוך כִּווּנָה העצמי של תפיסת\hebrewmakaf עולם\hebrewmakaf ואדם שלמה וכוללת, בטבעיותו הרוחנית והחיונית. יש לקבוע, ללמד ולהדריך, לבאר ולברר את האמונה מתוכה בעצמה, לא מתוך שטחי עניינים אחרים של רגשיות או מוסריות או שכליות, של תועלתיות או חברתיות או הגיוניות, אלא מתוך רוממות תכנה הפנימי של מציאותה הנפשית }\צמקור{[ל״י ב (מהדורת בית אל תשס״ג) תז].}

\paragraphs

\ערך{לֶמֶד }\הגדרה{- ההתרחבות הציורית מתוך הספיגה שכל נושא סופג אל תוכו מחוצה לו }\מקור{[ר״מ יד]}\צהגדרה{. }

\paragraphs

\ערך{״למהלך האידיאות״ }\צהגדרה{- ר׳ אידיאה.}

\paragraphs

\ערך{לפני ולפנים }\הגדרה{- הרגשות הטמירים ומהות החיים אשר לנשמה }\מקור{[פנק׳ א שצז]}\צהגדרה{.}

\paragraphs

\ערך{לפנים}\myfootnote{3 \textbf{כמו ב״הם לא עשו אלא לפנים״ מגילה }\textbf{יב}\textbf{.}\label{3}}\הגדרה{ - המעשים הגלויים }\מקור{[פנק׳ א שצז]}\צהגדרה{.}

\paragraphs

\ערך{לפנים }\הגדרה{- }\משנה{ההנהגה האלהית שהיא לפנים }\הגדרה{- לצד ההשתלמות }\מקור{[עפ״י ע״ר ב סז]}\צהגדרה{.}

\הגדרה{לתכלית ההתעלות }\מקור{[ע״א ג ב נה]}\צהגדרה{.}

\הגדרה{ע״ע לאחור.}

\paragraphs

\ערך{לצחצח }\הגדרה{- להחיות להבהיר ולזקק יותר }\מקור{[א׳ לד]}\צהגדרה{. }

\paragraphs

\ערך{לרומם }\הגדרה{- לחבר במקור אור\hebrewmakaf החיים\mycircle{°}, יסוד חֵי\hebrewmakaf העולמים\mycircle{°}}\צהגדרה{ }\מקור{[א״ק ב תקעג]}\צהגדרה{.}

\משנה{לרומם שם אלהי\hebrewmakaf ישראל\mycircle{°}}\הגדרה{ - להעלות שטפי חיי העולמים אל באר מקור מי חייהם }\מקור{[א״ק ב תקעד]}\צהגדרה{. }

\הגדרה{ע׳ במדור פסוקים ובטויי חז״ל, רוממות שם\hebrewmakaf ד׳.}

\paragraphs

\ערך{לשוב בתשובה}\צהגדרה{ - }\הגדרה{ע״ע תשובה (באדם). }

\paragraphs

\ערך{לשון }\הגדרה{- }\משנה{(לעומת דבור\mycircle{°}) }\הגדרה{- }\מעוין{◊}\הגדרה{ הוראת לשון הוא דוקא בהבנה, ודבור ג״כ כשאינו מבין הלשון }\מקור{[עפ״י ט״ר עח]}\צהגדרה{. }

\paragraphs

\ערך{לשון הקודש }\הגדרה{- השפה הנועדת להליט בה רגשי קודש\mycircle{°} נעלים ורוממים, השפה שמלאכי\hebrewmakaf השרת\mycircle{°} משתמשים בה }\מקור{[ע״א ב ט כ 255]}\צהגדרה{. }

\הגדרה{השפה שעל ידה, על ידי ניביה ואותיותיה נגלה לנו כל מחמד העולם המחשבי היותר עליון ונשגב. הרוח הפילולוגי שהביא אל הנפש את המבטאים והצלצולים הללו הוא לפי״ז היותר עדין ויותר פנימי. הרוח שעל ידו משתקפים החיים בצדם הנשגב והעליון }\צהגדרה{[ג״ר }\הגדרה{140}\צהגדרה{]. }

\הגדרה{השפה הברורה המכוונת להשקפת העולם של האספקלריא\hebrewmakaf המאירה\mycircle{°} הכוללת בתוכה את סגולת הדעת בתור כח פועל, לא רק כח סוקר ומצייר את היש, כ״א כח מחולל ומוליד, מוציא את ההוייות לאור עולם, מן האין אל המציאות ״העולם בלשון הקודש נברא״}\myfootnote{ בר״ר פר׳ יח ד. פר׳ לא ח.\label{4}}\הגדרה{ }\מקור{[קובץ ו ז]}\צהגדרה{. }

\משנה{שפת הקודש }\הגדרה{- (השפה) היותר טבעית לאדם להוציא בה לעצמו מושגיו היותר נשגבים שיסודם הוא כח יראת\hebrewmakaf שמים\mycircle{°} }\מקור{[ע״א ג ב קעה]}\צהגדרה{. }

\paragraphs

\צמשנה{לשון הרע}\myfootnote{ \textbf{שכוונתה להרע} - ע׳ חפץ חיים, הלכות לשון הרע, עמ׳ מח, כלל א, באר מים חיים ס״ס יג: ״שכוונתו למיסבר קראי ולא לגנותה בזה, וזה דומה למי שמכוון לתועלת בדבר שהוא אמת״. ושם עמ׳ פ כלל ג, באר מים חיים סי׳ יא: ״דאילו לא היה משה זכאי בזה, בוודאי היה מותר למרים לדבר בענין זה, דהא כיוונה רק לקנא לאמת ולא ח״ו לגנותו של מרע״ה, ורק לבנינו של עולם וכו׳, ודבר זה איננו לשון הרע מצד הדין״ וכו׳ ע״ש.\label{5}}\צהגדרה{ }\הגדרה{- (התבטאות מתוך טינת לב שכונתה להרע) }\מקור{[עולה עפ״י א״ק ג רמה, ע״ר א עניני תפילה יח]}\צהגדרה{.}

\paragraphs

\ערך{״לשון יונית״}\myfootnote{ בבא קמא פג.\label{6}}\הגדרה{ - (הלשון) המיוחדת להוציא את המושגים הגנוזים אל הפועל החיצוני, להציגם בעולם המורגש. זהו יסוד היופי\mycircle{°} המקנן בה }\מקור{[ע״א ג ב קעה]}\צהגדרה{. }\mylettertitle{מ}

\paragraphs

\ערך{מאד }\הגדרה{- }\משנה{(הוראתו) }\הגדרה{- הרבוי וההפלגה }\מקור{[ע״ר א רנ]}\צהגדרה{. }

\paragraphs

\ערך{מאור }\הגדרה{- }\מעוין{◊}\הגדרה{ הפעולה היוצאת אל המשיגים החשים את האור }\מקור{[ע״א ב 238]}\צהגדרה{. }

\הגדרה{ע׳ במדור מונחי קבלה ונסתר, מאור.}

\paragraphs

\ערך{מאיר }\הגדרה{- }\משנה{מאיר את העולמים }\הגדרה{- מחיה את העולמים, פועל אותם, מסדר אותם, מקשר אותם זה בזה, ומעריך את כל ערכיהם }\מקור{[עפ״י קובץ ה פז]}\צהגדרה{. }

\paragraphs

\ערך{מאכל }\הגדרה{- כל חומר הצריך לבנין הגוף הנהרס ע״י כח החיים והמעשה }\מקור{[ע״א ב ט ריד]}\צהגדרה{.}

\הגדרה{ע״ע אכילה.}

\paragraphs

\ערך{מאמר }\הגדרה{- המשפט הגוזר, מושגי התבונה וערכי השכל וההשכלה בהקצבתם }\מקור{[עפ״י ר״מ קעד]}\צהגדרה{. }

\paragraphs

\ערך{מבאר }\הגדרה{- ע״ע מפרש. }

\paragraphs

\ערך{מבוע }\הגדרה{- מעין הנובע }\מקור{[ר״מ קלד]}\צהגדרה{. }

\paragraphs

\ערך{מבט }\הגדרה{- }\משנה{(פנית מבט) }\הגדרה{- נטיית דעת }\מקור{[קובץ א קעב]}\צהגדרה{. }

\paragraphs

\ערך{מברך }\הגדרה{- משפיע ומשלים }\מקור{[ע״א ג ב מט]}\צהגדרה{.}

\הגדרה{ע״ע ברכה.}

\paragraphs

\ערך{מגביל }\הגדרה{- מצמצם\mycircle{°} ומשים הפרשים והבדלים בין נושא לנושא }\מקור{[עפ״י ע״ר א לג]}\צהגדרה{. }

\הגדרה{ע״ע הגבלה. ע״ע גבולים. ע״ע מוגבל.}

\paragraphs

\ערך{מגדף }\הגדרה{- ע״ע גדפנות. }

\paragraphs

\ערך{מגמה }\הגדרה{- תעודה ותכלית }\מקור{[עפ״י ע״א ד ט כו]}\צהגדרה{. }

\צהגדרה{שפע\mycircle{°} פנימי\mycircle{°}, חיוני, ממקור\hebrewmakaf החיים\mycircle{°}. שאיפה, כונה, או תשוקה, גם תכלית, מקום חפצו. שפע החיות העליונה }\צמקור{[עפ״י א״ק א, מבוא הרד״ך 6\hebrewmakaf 35]. }

\צמשנה{מגמה ותכלית ההבדל ביניהם }\הגדרה{- }\צהגדרה{האחרונה היא חיצונית\mycircle{°}, והמגמה היא פנימית, חיונית, נפשית }\צמקור{[שם 35]. }

\paragraphs

\ערך{מגמה אלהית\mycircle{°}}\הגדרה{ - }\משנה{המגמה האלהית }\הגדרה{- הטוב\mycircle{°} הכללי המקיף את כל ומתגלה בהשלמתו של כל פרט ושל כל קבוץ }\מקור{[א׳ נו]}\צהגדרה{. }

\הגדרה{ההשתלמות וההשתבחות המוסרית והחומרית }\מקור{[ע״א ג ב נה]}\צהגדרה{.}

\paragraphs

\משנה{מגמה עליונה}\הגדרה{\mycircle{°} - }\צהגדרהמודגשת{המגמה העליונה, באיכותה}\צהגדרה{ - רצון\hebrewmakaf העולם\mycircle{°}, המתגלה בתור נפש\hebrewmakaf החיים\hebrewmakaf שבהויה\mycircle{°}. המגמה המשיחית, ההתעלות\mycircle{°} הכוללת }\צמקור{[א״ק א, מבוא הרד״ך 35, 38, ועפ״י שם ב שסט]. }

\הגדרה{ע׳ במדור משיח וגאולה, ״רוחא דמלכא משיחא״. ושם, משיחיות. ור׳ שם, משיח, המשיח.}

\paragraphs

\ערך{מגמת החיים }\הגדרה{- שיהיה אור תפארת\hebrewmakaf ישראל\mycircle{°} מאיר בכל זיו\mycircle{°} טהרתו\mycircle{°} }\מקור{[קובץ ז קפו]}\צהגדרה{.}

\paragraphs

\ערך{מגמת כל המגמות}\הגדרה{ - להאיר את האור של המחשבה\hebrewmakaf הקדומה\mycircle{°} שנברא העולם בה מצד פנימיותו\mycircle{°} }\מקור{[עפ״י קובץ ח קפג]}\צהגדרה{.}

\paragraphs

\ערך{מדבר }\הגדרה{- }\משנה{(בדצח״מ) }\הגדרה{- המדרגה שבה יד הבחירה\hebrewmakaf החפשית\mycircle{°} שולטת, שבה צריכים כל הצדדים שבו מצד שהוא חי, להעשות על פי התכלית של כח הדיבור השכלי, דהיינו היושר\mycircle{°} והצדק\mycircle{°} }\מקור{[עפ״י ה׳ רו]}\צהגדרה{. }

\paragraphs

\ערך{מדות }\הגדרה{- אופי הרצון}\צהגדרה{ }\מקור{[קובץ ג קצז]}\צהגדרה{.}

\paragraphs

\ערך{מדות חסידות }\הגדרה{- ע״ע מדות תרומיות. ע״ע מעשים של קדושה. }

\paragraphs

\ערך{מדות תרומיות }\הגדרה{- המעשים האציליים\mycircle{°} העדינים. תכונות נשאות, רמות וקדושות }\מקור{[ע״ר א קנג]}\צהגדרה{. }

\paragraphs

\ערך{מדיני }\הגדרה{- כללי, משותף בחיי החברה ופועל עליה }\מקור{[ע״ר א רכא]}\צהגדרה{. }

\paragraphs

\ערך{מדע}\הגדרה{ - }\משנה{המדע}\הגדרה{ - ידיעת העולם והחיים }\מקור{[פנק׳ ב השמטות מפנקס מ״א]}\צהגדרה{.}

\משנה{המדע בכללו}\הגדרה{ - ההסתגלות של רוח האדם אל המציאות כולה }\מקור{[קובץ א תתיא]}\צהגדרה{.}

\paragraphs

\ערך{מה }\הגדרה{- מלת השאלה המבררת }\מקור{[ר״מ קפח]}\צהגדרה{. }

\paragraphs

\ערך{מואב }\הגדרה{- }\משנה{(תכונתם הלאומית) }\הגדרה{- אהבת הגזע }\מקור{[קבצ׳ א לט]}\צהגדרה{. }

\הגדרה{ע׳ במדור מונחי קבלה ונסתר, קליפת מואב. ע״ע עמון. }

\paragraphs

\ערך{מוגבל }\הגדרה{- }\ערך{מוגבל }\הגדרה{- מרוצף מגמה\mycircle{°} וכונה, מצרים והגבלות }\מקור{[ע״ר א קפה]}\צהגדרה{.}

\הגדרה{אסור בחוג מדתו }\מקור{[ע״ר א לד]}\צהגדרה{.}

\הגדרה{ע״ע הגבלה. ע״ע גבולים. ע״ע מגביל. }

\paragraphs

\משנה{מומרות }\צהגדרה{- }\צמשנה{המומרות הגמורה בכל תוצאותיה }\צהגדרה{- ההתיחסות הרוחנית ההערצה הדתית וההודאה באלהותו, של אותו ״פושע\hebrewmakaf ישראל\mycircle{°} שעשאוהו נכרים עבודה זרה״, שהיא ההפנאה מן תוקף נצחם האלהי של ישראל אל אותו אל זר ומתנכר, במקומו }\צמקור{[ל״י ב (מהדורת בית אל תשס״ג) תקלה].}

\הגדרה{ע״ע המרה. ר׳ במדור אליליות ודתות, יש״ו.}

\paragraphs

\ערך{מוסר }\הגדרה{- הטבת\mycircle{°} ארחות החיים, אורח הצדקה\mycircle{°} והישרות\mycircle{°} בחיי היחיד והכלליות }\מקור{[א״ק א ב\hebrewmakaf ג]}\צהגדרה{. }

\paragraphs

\ערך{מוסר }\הגדרה{- נטיית החפץ הטוב\mycircle{°} }\מקור{[א״ק ג פט]}\צהגדרה{.}

\משנה{המבוקש המוסרי }\הגדרה{- בנין עולם שכלולו ופארו }\מקור{[קבצ׳ א קעו]}\צהגדרה{. }

\משנה{רגש המוסר}\הגדרה{ - }\מעוין{◊}\הגדרה{ בא משאיפת הטוב המעשי המוכן ושישנו לשעתו }\מקור{[עפ״י פנק׳ א קכג (מא״ה ב מד)]}\צהגדרה{. }

\הגדרה{ע״ע דת, בישראל, רגש הדת בישראל.}

\paragraphs

\ערך{מוסר }\הגדרה{- }\משנה{איש מוסרי }\צהגדרה{- }\הגדרה{ע׳ במדור מדרגות והערכות אישיותיות.}

\paragraphs

\ערך{מוסר }\הגדרה{- }\משנה{חסרון המוסר }\הגדרה{- שאין הרצון של האדם עומד על מעמד כזה שיוכל לחפוץ באמת בהטוב\mycircle{°} }\מקור{[עפ״י קובץ א תקנח]}\צהגדרה{. }

\paragraphs

\ערך{מוסר }\הגדרה{- }\משנה{מוסר טהור\mycircle{°} של הדמות\hebrewmakaf הצורה\hebrewmakaf ליוצרה\mycircle{°}}\הגדרה{ - המשכת האדם להטוב\hebrewmakaf המוחלט\mycircle{°} והעדין }\מקור{[א״ק ג מא]}\צהגדרה{. }

\הגדרה{תכונה פנימית עדינה, שוכנת בנשמה\mycircle{°}, לבקש את הטוב\mycircle{°}, את הטוב\hebrewmakaf המוחלט, להיות בעצמו טוב, להיות דבק אל הטוב }\מקור{[א׳ קלו]}\צהגדרה{. }

\משנה{הכח המוסרי }\הגדרה{- הדעת\mycircle{°} את ד׳\mycircle{°} וההליכה\mycircle{°} בדרכיו\mycircle{°} }\מקור{[עפ״י ע״א א 153]}\צהגדרה{. }

\הגדרה{ע״ע רוח המוסר. ע״ע דת בישראל, רגש הדת בישראל. }

\paragraphs

\ערך{מוסר - }\משנה{העולם המוסרי}\הגדרה{ - הכח היותר עקרי ויותר מגמתי במציאות. <כי תיקון המוסר הוא באמת תיקון הכל, ואין חפץ בכל תיקוני הליכות הבריאה וכל הרחבות החיים שבה, אם לא יהיה תוכה רצוף מוסר השכל, ענות\hebrewmakaf צדק\mycircle{°}, חסד\mycircle{°} ומישרים\mycircle{°}. ע״כ הכרח הוא שהמגמות המוסריות פזורות הן בכל פינה שבצדדי החיים ולב חכם ישכילן> }\מקור{[ע״א ג ב רנ]}\צהגדרה{.}

\משנה{האור המוסרי }\הגדרה{- נשמת ההויה, היותר זורחת\mycircle{°} וחביבה }\מקור{[קובץ א קכח]}\צהגדרה{.}

\מעוין{◊ }\משנה{המוסר והשכל }\הגדרה{- העולמים\mycircle{°} היסודיים של החיים והמציאות }\מקור{[א״ק ג יט]}\צהגדרה{.}

\paragraphs

\ערך{מוסר }\הגדרה{- }\משנה{העליה\mycircle{°} המעולה בחיי המוסר }\הגדרה{- ההתיצבות הישרה\mycircle{°} בכל הכוחות, ״רגלי עמדה במשור״, ההתאמה המפוארה של כל נטיות החיים, עד שהשכל\hebrewmakaf העליון\mycircle{°} נמצא בתור גלוי עליון\mycircle{°} של סכום החיים כולם, וכל אשר מתחת לו הרי הם ענפיו המתפשטים ממנו, שבים אליו ומתרפקים עליו, מוכנים לרצונו, ולעבודתו כסופה ירדופו}\myfootnote{ \textbf{כסופה }\textbf{ירדופו} - לשון ס״י פרק א ו.\label{1}}\הגדרה{, וכל המהלכים הטבעיים של הנפשיות והגופניות\mycircle{°} מוארים הם באור\mycircle{°} עליון ובמהות הקדש\mycircle{°} המנצח, המלא הוד\mycircle{°} ויפעת\mycircle{°} קדשים\mycircle{°} של זיו ההשכלה הטהורה\mycircle{°} המאירה באור חכמה\mycircle{°} ודעת\mycircle{°} יסודית\mycircle{°} }\מקור{[א׳ כח]}\צהגדרה{. }

\paragraphs

\ערך{מוסר }\הגדרה{- }\משנה{עז המוסר }\הגדרה{- קדושת\mycircle{°} אמת, בהכרה עליונה, ובתכונה עמוקה החודרת עד עומק לב ונפש }\מקור{[ע״א ד ה עב]}\צהגדרה{. }

\paragraphs

\ערך{״מוסר אביך״ }\הגדרה{- ע׳ במדור פסוקים ובטויי חז״ל. }

\paragraphs

\ערך{מוסר האלהי }\הגדרה{- }\משנה{המוסר האלהי}\הגדרה{ - המוסר הנצחי, <לא כמוסר האנושי המוגבל בתחומים צרים, הבנוי על יסודות שהנפש חופשת להשתחרר מהם, לא כרצועות על קרני שור מועד> אור\hebrewmakaf החיים\mycircle{°} הוא מצד עצמו. עטרת\mycircle{°} תפארת\mycircle{°} בראשו של כל צדיק\mycircle{°}}\צהגדרה{ }\מקור{[א״ק ב תקסג]}\צהגדרה{.}

\paragraphs

\ערך{מוסר הקודש }\הגדרה{- המוסר\mycircle{°} שממקורה של תורה, המרומם ממעל למוסר הטבעי הפשוט }\מקור{[עפ״י ע״א ד ד א]}\צהגדרה{. }

\הגדרה{המוסר של החיים, ההולך ממקור\hebrewmakaf החיים\mycircle{°}, שמפעם ביחידים ממעין האומה\mycircle{°}, ובאומה מחי\hebrewmakaf העולמים\mycircle{°} }\מקור{[עפ״י א״ק ג י]}\צהגדרה{. }

\הגדרה{ע׳ במדור פסוקים ובטויי חז״ל, הר ד׳. ע״ע אורות הקֹדש. ר׳ חכמת הקודש.}

\paragraphs

\ערך{מוסר טבעי }\הגדרה{- המוסר הנטוע בטבע הישר\mycircle{°} של האדם }\מקור{[א״ק ג, ראש דבר כז]}\צהגדרה{. }

\הגדרה{קול\hebrewmakaf ד׳\mycircle{°} הפשוט הקורא בכח על ידי שער\hebrewmakaf האורה\mycircle{°} הטבעי אשר לאדם, שהוא נחלת כל האדם. אור\hebrewmakaf החיים\mycircle{°} הפרושים בנשמת האדם בהדרה}\צהגדרה{ }\מקור{[עפ״י א״ת יב ה]}\צהגדרה{.}

\הגדרה{ע׳ במדור פסוקים ובטויי חז״ל, דרך ארץ הטבעית. }

\paragraphs

\ערך{מוסר כובש\mycircle{°}}\הגדרה{ - האופי המוסרי\mycircle{°} המוכרח לאסור מלחמות פנימיות, או להסיח דעה ורעיון ממהותו הטבעית ומכל תפקידיו, <אשר אז המוסר מוסר צולע הוא, מוסר מסוכן לנפילה למהמורות> }\מקור{[עפ״י א׳ כט]}\צהגדרה{.}

\הגדרה{ע״ע מוסר, העליה המעולה בחיי המוסר.}

\paragraphs

\משנה{מוסר כליות}\הגדרה{ - }\צמשנה{הרגשת מוסר כליות}\צהגדרה{ - ההרגשה הסובייקטיבית הפנימית של אותה ההערכה הטבעית של הפגימה והתיקון ביחס אליה }\צמקור{[ב״א 11].}

\paragraphs

\ערך{מוסר נמוך }\הגדרה{- (מוסר\mycircle{°}) שאינו כי אם משום יראת\hebrewmakaf העונש\mycircle{°} ואהבת השכר כשאינו מתעלה למה שהוא רם ונשגב ממנו, שהוא כח העבודה\hebrewmakaf מאהבה\mycircle{°} }\מקור{[ל״ה 188]}\צהגדרה{.}

\paragraphs

\ערך{מופת }\הגדרה{- הבהיקה המפעלית היוצאת מהסדר הרגיל, המראה את המגמה\mycircle{°} של ההויה בעומק מוסריותה\mycircle{°} }\מקור{[ע״ר א רב]}\צהגדרה{. }

\משנה{המסות האותות והמופתים אשר נגלו על ישראל }\הגדרה{- התנועה הפתאומית, האפשרות המתגלה להעתיק איתני הטבע ממשמרתם למען תכלית מוסרית נשגבה ואדירה אשר נגלתה בעולם ע״י ישראל ופעולתו בעולם }\מקור{[ע״א ג ב קט]}\צהגדרה{. }

\מעוין{◊ }\משנה{המופתים}\הגדרה{ מביאים לידי הכרת המציאות הרוחנית, והרכוש הרוחני בכללות }\מקור{[קובץ ז קמט]}\צהגדרה{.}

\הגדרה{ע״ע נס. }

\משנה{מופת }\הגדרה{- }\משנה{(לעומת אות\mycircle{°}) }\הגדרה{- }\מעוין{◊}\הגדרה{ מראה על עוצם הכח של המשלח ב״ה }\מקור{[ע״ר ב פב]}\צהגדרה{. }

\paragraphs

\ערך{מוצק }\הגדרה{- }\משנה{לפי הצורך של הכלכלה הגופנית, נחשב למוצק}\הגדרה{ - כל דבר שהוא צריך לגופו של אדם בתור בנין לאחד מחלקי גופו, בין יהיה נוזל מוצק או אויר}\צהגדרה{ }\מקור{[ע״א ב ט ריד]}\צהגדרה{.}

\הגדרה{ע״ע נוזל, לפי הצורך של הכלכלה הגופנית.}

\paragraphs

\ערך{מושכל }\הגדרה{- מצויר\mycircle{°}, מופשט }\מקור{[ע״ר א פג]}\צהגדרה{. }

\paragraphs

\ערך{מושכל}\myfootnote{ \textbf{מושכל, המוסר העצמי} - הכוונה למוסר\hebrewmakaf ישר (אוטונומי), בניגוד לחוקי התורה המהווים מוסר\hebrewmakaf כובש (הטרונומי), כשימוש חובות הלבבות, שער העבודה פ״ג, במושג ״הערת השכל״ לעומת ״הערת התורה״.  \label{2}}\הגדרה{ - המוסר\mycircle{°} העצמי הטהור\mycircle{°}}\צהגדרה{ }\מקור{[ל״ה 125]}\צהגדרה{.}

\paragraphs

\ערך{מוּתָרִים }\הגדרה{- ע׳ במדור מצוות, הלכות, מנהגים וטעמיהן, בהגדרות המבוא. }

\paragraphs

\ערך{מזבח רוחני }\הגדרה{- הלב\mycircle{°} הישראלי, המלא חיי קודש }\מקור{[א״ק ג רי]}\צהגדרה{.}

\paragraphs

\ערך{מזון }\הגדרה{- }\משנה{לקיחת מזון }\הגדרה{- ספוח אל כחו (העצמי של הניזון) אוצר כח מהמציאות החיצונית }\מקור{[עפ״י קובץ א קעג]}\צהגדרה{. }

\הגדרה{ע״ע אכילה.}

\paragraphs

\ערך{מזון אסתתי }\הגדרה{- }\משנה{(שבחידוד הפלפולי\mycircle{°}) }\הגדרה{- הסיפוק הנפשי הגדול בקדושה\mycircle{°}, הממלא את האדם בדבקותו בתורה באהבת שקידתה והרחבת פלפולה }\מקור{[רצי״ה א״י כח]}\צהגדרה{. }

\הגדרה{ע׳ במדור פסוקים ובטויי חז״ל, מילתא דבדיחותא. }

\paragraphs

\ערך{מזוקק }\הגדרה{- ע״ע זקוק.}

\paragraphs

\ערך{מזל }\הגדרה{- המערכות, הזרמים הטבעיים הדוחפים (בעולם) }\מקור{[עפ״י א״ק ג לד (קובץ ז לט)]}\צהגדרה{.}

\ערך{מזל }\הגדרה{- }\משנה{יסוד המזל }\הגדרה{- הערך שיש להתגלות השטחית של הצורה הקוסמית, עם המהות האישית, וממילא הקישור עם הפרצופיות, והיחס עם המוסר והתוכן המאורעתי. <סקירת המעמד הפנימי של האדם על פי ההתגלות השטחית של הצורה הקוסמית אפשרית וכמו כן הבחנת היחס שבין הסקירה האסטרולוגית, עם ההכרה הפרצופית האישית}\myfootnote{ \textbf{ההכרה הפרצופית האישית }- מאמר קדמות הזוהר לרד״ל, עמ׳ עו. נר באישון לילה  לרב אבינר עמ׳ 146. ע׳ א״ש ח יג. פנק׳ ד קעט. א״ל קמט.\label{3}}\הגדרה{, והם ביחס עם ההכרה של התנועות וצביונם בכלל, עד לידי אותה הכרה המונחת בהתגלות צביון כתב היד, כיון שהכללות פועלת פעולה סדרנית על כל תוכן תמציתי לפי ערך ריכוזו. והצביון האנושי, הוא המרוכז ביותר, וסידור כחותיו מתבלטים על ידי אותו הסידור הכללי, שתוכנית המערכה הקוסמית היא אחת מחלקיו> }\מקור{[עפ״י קובץ ז קעה]}\צהגדרה{. }

\הגדרה{ע״ע גד. ע״ע כוכבים. ע׳ במדור פסוקים ובטויי חז״ל, מערכת, (מערכת הכוכבים).}

\paragraphs

\ערך{מזל }\הגדרה{- }\משנה{(מזלו של אדם) }\הגדרה{- כח הכבוד(}\צהגדרה{\hebrewmakaf העצמי}\הגדרה{) הנפשי, <המוכן למען טובת האדם וגבורתו\mycircle{°} הנפשית הנחוצה לו לטובתו החמרית\mycircle{°} והרוחנית\mycircle{°}> }\מקור{[עפ״י ע״א ב ט סד]}\צהגדרה{. }

\הגדרה{ר׳ במדור פסוקים ובטויי חז״ל, אדם אית ליה מזלא.  }

\paragraphs

\ערך{מחוקה }\הגדרה{- מחוקק, מצויר }\מקור{[רצ״יה א״ש יא הערה 27]}\צהגדרה{.}

\paragraphs

\ערך{מחילה }\הגדרה{- }\משנה{(מחילת ד׳) }\הגדרה{- כאשר גוברים הרחמים\hebrewmakaf העליונים\mycircle{°} על סבלותיו של האור\hebrewmakaf האלהי\mycircle{°} המוטל בתוך מעצורים, שעי״ז הנשמה\mycircle{°} מחדשת את אורה\mycircle{°}, וכחה הרוחני\mycircle{°} מתגלה בשדרת החיים עם כל הטבעתם החמרית\mycircle{°}, אז משתלם הקלקול הבא מתוך חוסר האורה, והחוב (של החטא\mycircle{°}) נמחק. <מושג המחילה מורה על איזה דבר שבחוב חצוני, המוטל לפרעון> }\מקור{[ע״ר א קכו קכז]}\צהגדרה{.}

\מעוין{◊ }\משנה{עיקר המחילה}\הגדרה{ היא שתעשה נפשו מוכשרת לקבל האור הטוב\mycircle{°} של החכמה והצדק\mycircle{°} כמו שהיתה מוכשרת לזה קודם החטא }\מקור{[ע״א ג ב לט]}\צהגדרה{.}

\משנה{מחילה סליחה וכפרה (אלהית) }\צהגדרה{- }\צמשנה{סליחה }\צהגדרה{היא מצד ערכו המרומם (של הסולח) לבדו, ומצד ענינו של זה שנסלח לו זה נקרא בשם }\צהגדרהמודגשת{מחילה}\צהגדרה{, והפעולה הפנימית שהיא ביחש לקלקול הרצון, וביחוד זו שהיא באה מיסודה של הקדושה\mycircle{°} התמידית (נקראת }\צהגדרהמודגשת{כפרה}\צהגדרה{) }\צמקור{[עפ״י ע״ר ב תנב].}

\הגדרה{ע״ע כפרה. ע״ע סליחה. ר׳ בנספחות, מדור מחקרים, מחילה סליחה וכפרה.}

\paragraphs

\ערך{מחשבה }\הגדרה{- }\משנה{המחשבה הקודמת להאותיות }\הגדרה{- הסבה\mycircle{°} הרוחנית\mycircle{°} היותר אחרונה, שהיא סמוכה להיצירה הברואית של הוית האותיות\mycircle{°} בקצבתן }\מקור{[ר״מ קסו\hebrewmakaf ז]}\צהגדרה{. }

\ערך{המחשבה האידיאלית }\הגדרה{- השאיפה הקדמוניה מראש\hebrewmakaf מקדם\mycircle{°} }\מקור{[ר״מ קסח]}\צהגדרה{. }

\הגדרה{ע״ע מקור, המקור העליון ששם שרויה היא המחשבה האידיאלית. }

\paragraphs

\ערך{מחשבה }\הגדרה{- }\משנה{להפרות את המחשבה}\הגדרה{ - להוציא רעיונות חדשים מן הכח אל הפועל }\מקור{[ע״א ד ד ג]}\צהגדרה{.}

\paragraphs

\ערך{מחשבה אלהית על דבר העולם }\הגדרה{- (}\צהגדרה{כשחושבים על דבר העולם מתוך מחשבה אלהית}\הגדרה{) - המחשבה על דבר התהוותו והמשכת הוייתו ושאיפת עתידו (של העולם), מחשבה מציירת\mycircle{°} זיו\mycircle{°} שכל\mycircle{°} והדרת\mycircle{°} חפץ של טוב אדיר ונעים}\צהגדרה{ }\מקור{[עפ״י קבצ׳ ב פו 55]}\צהגדרה{.}

\הגדרה{ע״ע בריאה.}

\paragraphs

\ערך{מחשבת שם ד׳ }\הגדרה{- המחשבה\hebrewmakaf העליונה\mycircle{°} של התגשמות הטוב\mycircle{°} בעולם. מקור הטוב\mycircle{°} וצורתו\mycircle{°} היותר נשגבה\mycircle{°} }\מקור{[ע״א ד ו ס]}\צהגדרה{. }

\paragraphs

\ערך{מחשכים }\הגדרה{- תוכנים שליליים, כחות הזוהמא\mycircle{°} והמאפלים }\מקור{[ע״א ד ט צט]}\צהגדרה{. }

\הגדרה{דינים\hebrewmakaf קשים\mycircle{°}, מארות וקטיגורים, המכחישים פמליא\hebrewmakaf של\hebrewmakaf מעלה\mycircle{°}}\צהגדרה{ }\מקור{[א״ק ד תקה]}\צהגדרה{.}

\paragraphs

\משנה{מטה }\צהגדרה{- ההכללה היסודית והשרשית }\צמקור{[א״ל קכב].}

\הגדרה{ע״ע מעלה.}

\paragraphs

\ערך{מטה }\הגדרה{- }\משנה{מלמטה למעלה }\הגדרה{- מתחתית מדרגת ההויה עד רומה }\מקור{[אג׳ א קסג]}\צהגדרה{. }

\הגדרה{מ(התוכנים) הפרטי(י)ם על הכללים, מהתחתונים על העליונים }\מקור{[א״ק ב תטז]}\צהגדרה{. }

\משנה{ממטה למעלה }\הגדרה{- ממילוי של חיים במדה מועטת למילוי חיים במדה מרובה, מזהר\mycircle{°} קלוש לזהר עז ומקרין }\מקור{[א׳ נב]}\צהגדרה{. }

\הגדרה{מיצור שפל לעליון\mycircle{°} }\מקור{[א״ק ב תקמז]}\צהגדרה{. }

\הגדרה{ממהות ירודה למהות עליונה\mycircle{°} }\מקור{[שם תקכב]}\צהגדרה{. }

\הגדרה{מתכונת החשבון עד ״תאות גבעות עולם״ ״בטרם הרים יולדו״ }\מקור{[א״ש טז א]}\צהגדרה{. }

\הגדרה{ע״ע מעלה. }

\paragraphs

\ערך{מטה }\הגדרה{- }\משנה{דרך ההכרה ממטה למעלה }\הגדרה{- דרך ההשגה באור\hebrewmakaf חוזר\mycircle{°}. ההכרה המתחלת מדרך הרצאה רגילה, מאורח עולם ודרכם של בני אדם, בתכונת גופם, חושיותם ורגש נפשותם, או בדרך הגיון\mycircle{°} שכלם, והיא בוקעת ועולה, מקשרת סבה בסבה, מעין במעין, עד שהיא באה לידי התגלות עליונה\mycircle{°}, אשר ממנה ינהרו פלגי חכמה ודעת\hebrewmakaf קדושים\mycircle{°} }\מקור{[א״ק א ע]}\צהגדרה{. }

\משנה{מלמטה למעלה בדרך ההשכלה }\הגדרה{- קישור המחשבות המתפשטות לשרשם המקורי }\מקור{[עפ״י ע״א ד יב נה]}\צהגדרה{. }

\משנה{מלמטה למעלה }\הגדרה{- מהתוכן המעשי אל התוכן הרוחני\mycircle{°} }\מקור{[ע״ט קיא]}\צהגדרה{. }

\הגדרה{ע״ע מעלה, מלמעלה למטה בדרך ההכרה. }

\paragraphs

\ערך{מטיב - }\הגדרה{מאיר\mycircle{°} ומרחיב }\מקור{[עפ״י ע״ר א קמו]}\צהגדרה{.}

\paragraphs

\ערך{מטיב }\הגדרה{- }\משנה{״הקב״ה מטיב לך״ }\הגדרה{- ע׳ במדור פסוקים ובטויי חז״ל.}

\paragraphs

\ערך{מיל}\הגדרה{ - התחום המתיחש למקומו של אדם למצבו הקבוע }\מקור{[ע״א ג ב רצד]}\צהגדרה{.}

\paragraphs

\ערך{מיחוש }\הגדרה{- התחלשות הפעולות מהתעוררותן הצריכה להיות מלאת חיים בגוף בריא ונפש בריאה }\מקור{[ע״א ג א לא]}\צהגדרה{.}

\הגדרה{ע״ע חוֹלי.}

\paragraphs

\ערך{מים }\הגדרה{- היסוד הראשון, המוחש הרפה במגעו ואין לו מכון להסמך עליו שום דבר בעל כובד }\מקור{[ע״א ד יב ו (מא״ה ב קלב)]}\צהגדרה{. }

\paragraphs

\ערך{״מים״}\myfootnote{ עקידת יצחק, לרי״ע, שער לד: ״כבר נתפרסם בדברי החכמים ומנהג התורה האלהית בהשתמשה במלת מים לשום אותה כנוי אל כלל נמצאות, אם עליונות ואם תחתונות״.\label{4}}\הגדרה{ - כינוי לנמצאות בכללן, כל זמן  שהן בכח הצורה הכללית }\מקור{[מ״ש קטז (מא״ה ב יא)]}\צהגדרה{. }

\הגדרה{היצירה הבאה ממזוגו של השטף הגדול, ״מקולות מים רבים אדירים״\mycircle{°} }\מקור{[עפ״י ר״מ פח]}\צהגדרה{. }

\הגדרה{השטף המלא של התמזגות כל הפרטיות אל הכלליות האיתנה, ״מקולות מים רבים אדירים משברי ים״ }\מקור{[שם כא]}\צהגדרה{. }

\paragraphs

\ערך{״מים״}\myfootnote{ חגיגה יד:.\label{5}}\הגדרה{ - צורה\mycircle{°} עצמית <הנוהגת במציאות רוחנית כבגשמית>. }\צהגדרה{ומה שתחתיה חומר\mycircle{°} הוא }\מקור{[פנק׳ א שעג]}\צהגדרה{.}

\הגדרה{ע׳ במדור פסוקים ובטויי חז״ל, אבני שיש טהור.}

\paragraphs

\ערך{מים }\הגדרה{- הרגש הנפשי שמרומם את האדם בטבע הנפש הזכה לכסוף לאור\hebrewmakaf ד׳\mycircle{°} }\מקור{[ע״א ב ו כז]}\צהגדרה{. }

\משנה{מים היותר טהורים}\הגדרה{ - דעות היותר טהורות\mycircle{°} ונשגבות }\מקור{[קובץ א שמח]}\צהגדרה{.}

\הגדרה{ע׳ במדור מונחי קבלה ונסתר, מי הבריכה\hebrewmakaf העליונה. ע׳ בנספחות, מדור מחקרים, אמונה.}

\paragraphs

\ערך{מים}\הגדרה{ - ע״ע גשמים, הבאים מיסוד המים.}

\paragraphs

\משנה{מיעוט}\הגדרה{ - הצערת הנושא ממובנו הקדום, קציצת איזה סעיפים מתכונתו }\מקור{[ר״מ קכא]}\צהגדרה{. }

\הגדרה{ע״ע ממעט.}

\paragraphs

\ערך{מישרים }\הגדרה{- }\משנה{״הולך למישרים״ }\הגדרה{- לאושר\mycircle{°} ולטוב\mycircle{°} }\מקור{[א״ק ג שלט]}\צהגדרה{. }

\paragraphs

\ערך{מכבד את ד׳ }\הגדרה{- מוציא אל הפועל את התכלית הטובה הצפונה ב(ענינים), כפי הכונה האלהית העליונה }\מקור{[ע״א א ג ג]}\צהגדרה{. }

\paragraphs

\ערך{מכשף }\הגדרה{- ע״ע כשוף. }

\paragraphs

\ערך{מלא כל }\הגדרה{- אוצר החיים השלמים העליונים }\מקור{[ע״ר ב עח]}\צהגדרה{. }

\paragraphs

\ערך{מלאכה }\הגדרה{- כל תנועה שמצלת איזה חלק מן ההויה מן שליטת התוהו\mycircle{°} }\מקור{[א׳ סז]}\צהגדרה{. }

\הגדרה{חדוש כלום בההויה והיצירה }\מקור{[עפ״י ע״א ד ט ק]}\צהגדרה{. }

\הגדרה{תיקון והבאת איזה דבר לידי שכלול }\מקור{[עפ״י שם ז י]}\צהגדרה{. }

\הגדרה{עיבוד ושינוי באיזה חומר. דבר הפועל שינוי בעצם החומר בטבעו }\מקור{[שם יא כ]}\צהגדרה{. }

\משנה{״מלאכת מחשבת״}\myfootnote{ ביצה יג:.\label{6}}\הגדרה{ }\צהגדרה{- יצירה של האדם מתוך החופש של האדם. האדם חושב מחשבות ועושה מלאכה, מלאכה בעיר או מלאכה בשדה. מלאכה שייכת למלאכיות של המלאכים השמימיים }\צמקור{[עפ״י שי׳ ב 378].}

\הגדרה{ר׳ בנספחות, מדור מחקרים, מלאכה לעומת עבודה.}

\paragraphs

\ערך{מלאכת ד׳ }\הגדרה{- השתכללות כל היקום והוספת יפעתו\mycircle{°} ועלייתו התדירית העליונה העולה למעלה למעלה }\מקור{[ע״א ד ט ק]}\צהגדרה{. }

\paragraphs

\ערך{מלוכה }\הגדרה{- }\משנה{הנהגת המלוכה }\הגדרה{- ההנהגה המתפשטת בענייני המקרים }\צהגדרה{<לעומת הנהגת התורה\mycircle{°} המיוחדת לעצמותם של ישראל, או ההנהגה הכללית של שבע\hebrewmakaf מצות\hebrewmakaf ב״נ\mycircle{°} שהן ראויות לעצמות כלל האנושיות ומוסרם> }\מקור{[עפ״י ע״א ג ב לח]}\צהגדרה{.}

\paragraphs

\ערך{מלח }\הגדרה{- }\משנה{(מציין ברוחניות) }\הגדרה{- הידיעה הפשוטה שמחייבת את כל המון הדרישה המעשית, השומרת את היראה\mycircle{°} והאמונה\mycircle{°} }\מקור{[עפ״י ע״א ב ו כז]}\צהגדרה{. }

\הגדרה{היסוד הגס של יראת\hebrewmakaf העונש\mycircle{°} היושב בתחתיתה היותר שפלה של שדרת הרוח <שצריך הוא רק שם לקחת את מקומו, (כדוגמת) שיכלול היצירה, שם הצרות והרעות שבעולם, הפורעניות והעלבונות הנם מלח\hebrewmakaf העולם\mycircle{°}>. מחשבות וציורים שמיסוד יראת העונש. זהירות פנימית, שוה עם טבע היראה מכל אסון, והזהירות הטבעית ממנו, <זהו היסוד הנלוה אל כל אמוצי החיים, והוא הגורם בקורטוב המעורב שלו, דמשתכח כי קורטיתא בכורא, לחבב את החיים כולם, לעשותם רעננים ומאושרים> }\מקור{[עפ״י א״ק ד תיז]}\צהגדרה{. }

\ערך{מלח החיים }\הגדרה{- הפחד הדמיוני מהמות\mycircle{°} שכל ענין אהבת החיים בא רק על ידו }\מקור{[עפ״י א״ק ב שפא]}\צהגדרה{. }

\paragraphs

\ערך{מלח העולם}\הגדרה{ - }\משנה{בשיכלול היצירה}\הגדרה{ - הצרות והרעות שבעולם, הפורעניות והעלבונות }\מקור{[עפ״י א״ק ד תיז]}\צהגדרה{.}

\paragraphs

\ערך{מלך }\הגדרה{- מנהיג לעם בעצה\mycircle{°} }\מקור{[קבצ׳ ב לה (פנק׳ א לא)]}\צהגדרה{. }

\הגדרה{ע״ע נשיא. }

\paragraphs

\ערך{מלך}\הגדרה{ - }\משנה{״משיח ד׳״ }\הגדרה{- ע׳ במדור משיח וגאולה, ״משיח ד׳״.}

\paragraphs

\ערך{מלך ישראל }\הגדרה{- הנפש המשוכללת המשוקה מהלח התמציתי של חיי האומה כולה. הנעשית ראויה בכוח התרכזות התמציתיות הלאומית הגדולה שבה להתנחלות השאיפה האלהית בקרבה ביסוד של נצח\mycircle{°} ושל הוד\mycircle{°}}\צהגדרה{ }\מקור{[עפ״י מ״ר 38]}\צהגדרה{.}

\הגדרה{ע׳ במדור מדתם ועניינם הרוחני של אישי התנ״ך, דוד, נפש דוד. ע״ע שליטה.}

\paragraphs

\ערך{מלכות }\הגדרה{- }\משנה{ההשגה מצד המלכות }\הגדרה{- ההכרה מצד הבריאה }\מקור{[ע״ר א תיב]}\צהגדרה{. }

\הגדרה{כבודו\mycircle{°} ית׳ מצד ההתגלות שבעולמות\mycircle{°} }\מקור{[עפ״י שם רמט]}\צהגדרה{. }

\הגדרה{ע׳ במדור שמות כינויים ותארים אלהיים, ״מלכנו״. ע׳ במדור פסוקים ובטויי חז״ל, ברוך שם כבוד מלכותו לעולם ועד. }

\paragraphs

\ערך{מלכות}\הגדרה{ - }\משנה{המילוי המלכותי}\הגדרה{ - התכוננות כסא\hebrewmakaf ד׳\mycircle{°} על מכונו, על כסא דוד ועל ממלכתו, להכין אותה ולסעדה}\צהגדרה{ }\מקור{[א״ק ב תקסב]}\צהגדרה{. }

\paragraphs

\ערך{מלכות בית דוד }\הגדרה{- }\משנה{(לעומת מלכות\hebrewmakaf ישראל\mycircle{°}) }\הגדרה{- יסודה של צד הקדושה שבסדר האומה להנהגתה הפנימית והחיצונית בתור ממלכת ד׳ ממלכת\hebrewmakaf כהנים\mycircle{°}. כשרון צד הקדושה של הממלכה }\מקור{[עפ״י ע״א ד ה סט]}\צהגדרה{. }

\מעוין{◊ }\ערך{מלכות בית דוד }\הגדרה{היא דוגמת מלכות\hebrewmakaf שמים\mycircle{°} העליונה, והסגולות\mycircle{°} השמימיות גנוזות בה, והן מוכרחות לצאת לאור בכל מילואם, במילוי\hebrewmakaf המלכותי\mycircle{°}, בהתכוננות כסא\hebrewmakaf ד׳\mycircle{°} על מכונו, על כסא דוד ועל ממלכתו, להכין אותה ולסעדה }\מקור{[א״ק ב תקסב]}\צהגדרה{.}

\paragraphs

\ערך{מלכות הארץ }\הגדרה{- האידיאל\mycircle{°} החברתי }\מקור{[א״ק ב תקסב]}\צהגדרה{. }

\הגדרה{ע׳ במדור תיאורים אלהיים, ״מלכות שמים״. }

\paragraphs

\ערך{מלכות ישראל }\הגדרה{- }\משנה{(לעומת מלכות\hebrewmakaf בית\hebrewmakaf דוד\mycircle{°}) }\הגדרה{- יסודו של הצד העולמי שבסדר האומה להנהגתה הפנימית והחיצונית ברוממות פוליטית ככל העמים. כשרון צד החול\mycircle{°} של הממלכה }\מקור{[עפ״י ע״א ד ה סט]}\צהגדרה{. }

\paragraphs

\ערך{ממארת }\הגדרה{- עצובה ומחשכת }\מקור{[א״ק ג קעה]}\צהגדרה{.}

\paragraphs

\ערך{״ממולח״ }\הגדרה{- המיזוג והעירוב העמוק. העירוב והמיזוג הגמור עד לכדי ההתבוללות הגמורה והמוחלטת }\מקור{[עפ״י ע״ר א קלז]}\צהגדרה{.}

\paragraphs

\ערך{ממוצעים }\הגדרה{- המסילות, האספקלריאות הממצעים ומתווכים בין הראשית\mycircle{°} והאחרית\mycircle{°} }\מקור{[עפ״י א״ק א רז, ע״ר א פד]}\צהגדרה{.}

\paragraphs

\ערך{ממלא }\הגדרה{- }\משנה{(לעומת מקיף\mycircle{°}) }\הגדרה{- מתפשט ופועל בתוכיותן של כל ההויות כולן }\מקור{[עפ״י ע״ר ב פב]}\צהגדרה{.}

\paragraphs

\ערך{ממעט }\הגדרה{- צמצום כח, בכמות או באיכות }\מקור{[אג׳ ד (מהדורת תשע״ח) רמח]}\צהגדרה{.}

\הגדרה{ע״ע מיעוט.}

\paragraphs

\ערך{מנביע }\הגדרה{- ממשיך את תכנו הנובע }\מקור{[רצי״ה א״ש יא הערה 15]}\צהגדרה{.}

\paragraphs

\ערך{מנוחה }\צהגדרה{- }\הגדרה{שלוה פנימית }\מקור{[א״ק א קפט]}\צהגדרה{.}

\צמשנה{מנוחה אנושית}\צהגדרה{ - ענין נפשי\hebrewmakaf גופני הכרחי של התאפשרות והתכוננות להמשך העבודה, אחרי הפסקת העיפות, ״הניחו לעיף״, ומתוך השעבוד והקשור של האדם, אשר ״לעמל יולד״, אל הכרחיות העבודה, הלא הפסקת מנוחה זו היא החוזרת ומכשירה ומכינה אותו להתחדשותה ושכלולה  מתוך התעכבות זו של ״עמידה לפוש״ או גם ״עמידה לכתף״}\צמקור{ [ל״י ב (מהדורת בית אל) שמה].}

\צהגדרה{שלילת עמל כאמצעי להתחזקות }\צמקור{[פע׳ לו].}

\הגדרה{ר׳ מנוחה של יום השבת.}

\paragraphs

\ערך{מנוחה}\myfootnote{ ירמיה מה ג. ע״ע ע״ר ב תצ, ס״ק סא. \textbf{מנוחה של יום השבת }- בס׳ בעל שם טוב המהודר, על התורה ומועדים, ח״א עמ׳ צב מביא מכתר שם טוב ח״ב כ.: ״מנוחה היינו בהירות הוייתם של הברואים, הנעלמת, שהוא מעצמותו יתברך שמו״. ״כשהשם יתברך מבהיק זיו הדרו אל הברואים״. ס׳ בת עין, בהעלותך ד״ה ויסעו מהר ד׳. ״לתור להם מנוחה וגו׳ (במדבר י לג) [...] להמשיך עליהם שפע״.\label{7}}\הגדרה{ - עצם קדושת\mycircle{°} הנפש, השפעה מלמעלה הבאה מכח עצם גילוי הקדושה שבנפש }\מקור{[ה׳ רי]}\צהגדרה{.}

\הגדרה{שפע\mycircle{°} הנבואה\mycircle{°}, שעיקר מעלתה היא ע״י כח עצם קדושת הנפש }\מקור{[עפ״י שם]}\צהגדרה{.}

\הגדרה{השפע הבא ע״י רצון שמים מטבע הנפש }\מקור{[עפ״י ה׳ ריב]}\צהגדרה{.}

\משנה{מנוחה רוחנית }\הגדרה{- התקלטות פנימית בחזיונות רצויים ומקובלים, שההרגשה העדונית [הרוחנית] היא מתאחזת בהם בצורה בסיסית }\מקור{[ר״מ נד]}\צהגדרה{.}

\משנה{מנוחה שלמה }\הגדרה{- הקודש\mycircle{°}, הנשמה\mycircle{°} האלהית (ה)מופיעה\mycircle{°} באדם בכל הדרה\mycircle{°} בעת תגבורת הקודש, ואז הוא מכיר כי הוא חי חיי\hebrewmakaf אמת\mycircle{°}, חיים של נועם\hebrewmakaf ד׳\mycircle{°} וההתענג\mycircle{°} מזיו\mycircle{°} כבודו\mycircle{°} }\מקור{[ע״א ג ב סט]}\צהגדרה{.}

\משנה{מנוחה של יום השבת\mycircle{°} }\צהגדרה{- גילויה של קדושת ישראל\mycircle{°} ואצילות\mycircle{°} השראת שכינתם\mycircle{°} }\צמקור{[ל״י ב (מהדורת בית אל) שמז].}

\צהגדרה{המנוחה הישראלית המיוחדת, המופיעה בקביעת והתגלות ערך זמנו וסדור חייו של האדם, לא בשעבודו אל מסיבותיו ההכרחיות החיצוניות, אלא בקשורו וריכוזו אל חרות עצמותו הפנימית, אשר ביסוד הווייתו ויצירתו המקורית }\צמקור{[ל״י ב (מהדורת בית אל) שמה]. }

\צהגדרה{פסיביות של התבטלות\mycircle{°} כלפי מעלה, לשם הופעת הקודש}\צמקור{ [פע׳ לו].}

\paragraphs

\ערך{מנוחה}\הגדרה{ - }\משנה{(מתכונתה הראויה של המלוכה) }\הגדרה{- ההנהגה הישרה בשבט מישור לכלל העם כולו }\מקור{[ע״א ב ט רצד]}\צהגדרה{. }

\paragraphs

\ערך{מנוחה שלמה }\הגדרה{- }\משנה{(מתכונתה הראויה של מלכות ד׳) }\הגדרה{- השקט הגמור האהוב והרצוי הבא בדעת\hebrewmakaf את\hebrewmakaf ד׳\mycircle{°} לאמתתו, בדברים הנאמרים בלחישה לפי חקר כבודם, קול דממה דקה. }\צהגדרה{כשכבר יתוקן טבעו של אדם ותכונותיו, ולא יהי׳ עוד צריך לא למלחמה פנימית ולא למלחמה חיצונית, והמצות\mycircle{°} לא יהיו משמשות לאיבוד ועקירת מדות בהמיות, כ״א לרומם את רוח האדם הרבה מאד במנוחה שלמה של דעת ד׳ }\מקור{[עפ״י ע״א ב ט רצד]}\צהגדרה{. }

\paragraphs

\ערך{״מנוחתי״ }\הגדרה{- }\משנה{(מנוחת הקב״ה) }\הגדרה{- ע׳ במדור פסוקים ובטויי חז״ל.}

\paragraphs

\ערך{מסורת }\הגדרה{- }\משנה{(לעומת דת) }\הגדרה{- תוצאתה הממשית של הדת\mycircle{°} }\מקור{[עפ״י ב״א 11]}\צהגדרה{.}

\paragraphs

\ערך{מסתעף }\הגדרה{- מתפשט למרחק בסטיה מן השורש }\מקור{[רצי״ה א״ש יב הערה 18]}\צהגדרה{. }

\paragraphs

\ערך{מעולף }\הגדרה{- מסותר, מוסווה }\מקור{[רצי״ה א״ש יב הערה 58]}\צהגדרה{. }

\paragraphs

\ערך{מעילה }\הגדרה{- }\משנה{בקדשי שמים}\הגדרה{ - שינוי, החלפת תעודת המציאות, והשפלת הערך }\מקור{[עפ״י ע״א ב ו ג]}\צהגדרה{.}

\paragraphs

\ערך{מעילה }\הגדרה{- }\משנה{(לעומת חטא\mycircle{°}) }\הגדרה{- החטא מצד גדלו ועצם מריו, שמעל בחוקי ד׳ ומרד באדון כל המעשים יתברך }\מקור{[ע״ס 34]}\צהגדרה{. }

\paragraphs

\ערך{מעין }\הגדרה{- }\משנה{המעין העולמי העליון\mycircle{°}}\הגדרה{ - מקור צדיקו\hebrewmakaf של\hebrewmakaf עולם\mycircle{°}, מקור ראש צדיק, משם הברכות\mycircle{°} שופעות }\מקור{[א״ק א יב]}\צהגדרה{. }

\paragraphs

\ערך{מעין }\הגדרה{- }\משנה{המעין הפנימי }\הגדרה{- שורש החיים שבמקור\hebrewmakaf ישראל\mycircle{°} }\מקור{[קובץ ו רג]}\צהגדרה{.}

\paragraphs

\משנה{מעלה }\צהגדרה{- ההעלאה וההקפה המעשית המפורטת }\צמקור{[א״ל קכב].}

\הגדרה{ע״ע מטה.}

\paragraphs

\ערך{מעלה }\הגדרה{- אור\mycircle{°} העולמים\mycircle{°} }\מקור{[א״ק ג מה]}\צהגדרה{.}

\paragraphs

\ערך{מעלה }\הגדרה{- }\משנה{מלמעלה }\הגדרה{- ממקור עליון\mycircle{°} }\מקור{[א״ק ב של]}\צהגדרה{. }

\ערך{מעלה }\הגדרה{- }\משנה{מלמעלה למטה }\הגדרה{- ממהות עליונה\mycircle{°} למהות ירודה }\מקור{[שם תקכב]}\צהגדרה{. }

\משנה{מלמעלה למטה בדרך ישרה }\הגדרה{- מהתוכנים הכלליים העליונים על הפרטיים השפלים }\מקור{[שם תטז]}\צהגדרה{. }

\הגדרה{מיצור עליון לשפל }\מקור{[שם תקמז]}\צהגדרה{. }

\הגדרה{מההתעלות מכל חשבון אל יסוד ההתגלות החשבונית }\מקור{[א״ש טז א]}\צהגדרה{. }

\הגדרה{מעולם נהדר ונאדר בקודש\mycircle{°}, על עולם חול\mycircle{°} וחלוש }\מקור{[ע״א ד ח ח]}\צהגדרה{. }

\הגדרה{ע״ע מטה. }

\paragraphs

\ערך{מעלה }\הגדרה{- }\משנה{מלמעלה למטה בדרך ההכרה }\הגדרה{- דרך ההשגה באור\hebrewmakaf ישר\mycircle{°}. השגה הבאה בדרך התגלות עליונה, ממעמקי האצילות\mycircle{°}, ותוכני טהרה\mycircle{°} וקדושה\mycircle{°} עליונה\mycircle{°} השוטפים ועוברים, גלי נשמה המתגברים ומעינותיה ההולכים וזורמים ממקור הטוהר\mycircle{°} העליון, אל השכל\mycircle{°} ההגיוני\mycircle{°} ורגשי הנפש, כח\hebrewmakaf המדמה\mycircle{°}, והחושים, וכל ארחות ההכרה, הבאים וממשיכים להם פלגים מנחל\hebrewmakaf עדנים\mycircle{°} זה, עד שבאים הדברים לגבולם ותחומם הם }\מקור{[עפ״י א״ק א ע]}\צהגדרה{.}

\הגדרה{ע״ע מטה, דרך ההכרה ממטה למעלה.}

\paragraphs

\ערך{מעלה }\הגדרה{- }\משנה{פונה למעלה }\הגדרה{- לשלמות כללית ונצחית }\מקור{[ע״א ג א יג]}\צהגדרה{.}

\משנה{ללכת מעלה מעלה }\הגדרה{- מאידיאל\mycircle{°} אחד למשנהו וממגמה\mycircle{°} חשובה מרהיבה וכוללת לרעותה }\מקור{[שם ד ה נב]}\צהגדרה{.}

\paragraphs

\ערך{״מעמד הר עיבל״ }\הגדרה{- הבעת מחאתנו הכללית על כל פורע מוסר\mycircle{°} ועושה נבלה, בהתחלת כניסתנו לארץ\hebrewmakaf ישראל\mycircle{°} כדי להוציא מן הכח הגנוז אל הפועל את הטבע הפנימי\mycircle{°} הישראלי למוסדות הצדק\mycircle{°}, מצד ההכרה הכללית הנעוצה בעומק הטבע של הנשמה\hebrewmakaf של\hebrewmakaf האומה\mycircle{°} כולה}\צהגדרה{ }\מקור{[ע״א ד ה מה]}\צהגדרה{. }

\paragraphs

\ערך{מענג }\הגדרה{- מעדן ומחיה }\מקור{[ע״ה קלא]}\צהגדרה{. }

\paragraphs

\ערך{מערכה }\הגדרה{- דבר שהוא מונח בערך ויחש זה לזה }\מקור{[מ״ש עז (מא״ה א קו)]}\צהגדרה{. }

\paragraphs

\ערך{מעשה בראשית }\הגדרה{- }\משנה{המעמד היותר תכליתי להשלמת מעשה בראשית }\הגדרה{- הטבת המצב החברותי במעמדו המוסרי\mycircle{°} }\מקור{[עפ״י ע״א ב א ט]}\צהגדרה{.}

\paragraphs

\ערך{מעשה הטוב והעבודה האלהית }\הגדרה{- }\משנה{יסודם}\הגדרה{ - ע״ע עבודה אלהית והמעשה הטוב.}

\paragraphs

\ערך{מעשים }\הגדרה{- }\משנה{המעשים }\הגדרה{- הצד המגולה של ארחות החיים }\מקור{[עפ״י קובץ ח קכא]}\צהגדרה{. }

\paragraphs

\ערך{״מעשים״ }\הגדרה{- ע׳ במדור מצוות, הלכות, מנהגים וטעמיהן, הגדרות מבוא, מצוות, מעשה המצוות.}

\paragraphs

\ערך{מעשים טהורים\mycircle{°}}\הגדרה{ - המעשים שההרמוניה\mycircle{°} הנצחית מתגלה על ידם }\מקור{[א״ק ג ז (א״ש יב ט)]}\צהגדרה{. }

\paragraphs

\ערך{מעשים רצויים}\הגדרה{ - (מעשים) משוקלים בשקל של מוסר\mycircle{°} מאיר ומתוקן}\צהגדרה{ }\מקור{[א׳ קנד]}\צהגדרה{.}

\paragraphs

\ערך{מעשים של קדושה\mycircle{°}}\הגדרה{ - }\משנה{שהחסידות\hebrewmakaf המעשית\mycircle{°} מתלבשת בהם כשהיא באה מצד הרגש האלהי העולה על המדה ההגיונית המבקש(ת) מרג}\הגדרה{ֹ}\צהגדרה{ע בקרבת\hebrewmakaf אלהים\mycircle{°} התיאורית בצורתה הרוחנית }\הגדרה{- דרכי חיים מלאים אור\hebrewmakaf ד׳\mycircle{°} וטובו. צינורות\mycircle{°} יקרים להזיל על ידם טל חיים של שמחת\hebrewmakaf עולמים\mycircle{°}, של תום אמון בתוכיות הנפש, <לרוממה מעל כל ההנחות המוגבלות על דבר הטוב והיושר, שבני אדם נלכדים בהם, ולעתים קרובות מאד רגליהם מתמוטטות על ידם, מפני הלחץ של חקי ההגיון הצר של השכל האנושי, המוגבל ביותר> }\מקור{[עפ״י א״ק ג שיא]}\צהגדרה{.}

\הגדרה{ע״ע מדות תרומיות.}

\paragraphs

\ערך{מְפַלֶּשֶׂת }\הגדרה{- חודרת ומתפשטת }\מקור{[רצי״ה א״ש ח הערה 15, ושם יב הערה 17]}\צהגדרה{. }

\paragraphs

\ערך{מפרש, מבאר }\הגדרה{- הדורש בעמקי תורה, להאיר אור על דברי כתובים או מאמרי חכמים ז״ל }\מקור{[ע״א א, הקדמה, יג]}\צהגדרה{. }

\הגדרה{ע״ע פרוש. ע״ע באור. }

\paragraphs

\ערך{מצומצם }\הגדרה{- מוגבל }\מקור{[ע״ר א סז, א׳ כח, ועוד]}\צהגדרה{.}

\הגדרה{מוגבל במצרי גבולות חוקיים }\מקור{[עפ״י א״ק א קד]}\צהגדרה{. }

\paragraphs

\ערך{מצומצם }\הגדרה{- }\משנה{רצון מצומצם ומאופל }\הגדרה{- סגור במצרים של ההשגות והמאוויים שעל פי התנאים המצומצמים של ההויה }\מקור{[א״ק ב רפח (ע״ט ג)]}\צהגדרה{. }

\paragraphs

\ערך{מציאות }\הגדרה{- }\משנה{המציאות}\הגדרה{ - בית ד׳\mycircle{°}, הקשר הכללי הזה (של) כל המעשים והיצורים בקשורם ותכניתם, במהותם ומטרת הויתם }\מקור{[עפ״י ע״ר א נג]}\צהגדרה{.}

\משנה{היסוד הפנימי של המציאות}\הגדרה{ - ההתעלות הבלתי פוסקת. אור\hebrewmakaf ד׳\mycircle{°} המהוה הישות, זרוע\hebrewmakaf ד׳\mycircle{°} אשר נגלתה }\מקור{[עפ״י א״ק ב תקכט-תקל]}\צהגדרה{.}

\הגדרה{ע״ע תכלית המציאות.}

\paragraphs

\ערך{מציאות מלאה }\הגדרה{- מציאות שהתגלות\mycircle{°} המהות היא מלאה בקרבו של המצוי פנימה }\מקור{[עפ״י אג׳ ב מא]}\צהגדרה{. }

\הגדרה{חיים\mycircle{°}, רצון\mycircle{°} }\מקור{[עפ״י א׳ יא]}\צהגדרה{. }

\paragraphs

\ערך{מציאות עליונה }\הגדרה{- }\משנה{המציאות העליונה }\הגדרה{- החיים האציליים\mycircle{°} וההופעות\mycircle{°} האלהיות\mycircle{°} אשר ממפלאות תמים דעים. המציאות הטמירה העליונה המתבודדת בדומיה הקדושה העליונה. הסתום\mycircle{°} של רז העליון }\מקור{[עפ״י מא״ה ב קל]}\צהגדרה{. }

\paragraphs

\ערך{מצייר }\הגדרה{- ע״ע ציור. }

\paragraphs

\ערך{מצרי גיהנם }\הגדרה{- ע״ע גיהנם. }

\paragraphs

\ערך{מְקָדֵשׁ}\הגדרה{ - מעלה את (ה)רוח ומקרבהו אל הקדושה\mycircle{°} בהרגשת לב ושכל פנימי }\מקור{[א״ת י ג]}\צהגדרה{.}

\paragraphs

\ערך{מקודש }\הגדרה{- המגיע אל התכלית העליונה שכיון השי״ת\mycircle{°} }\מקור{[עפ״י מ״ש שנ (ה׳ קעו)]}\צהגדרה{. }

\הגדרה{ע״ע מתקדש. ע״ע קדוש. }

\paragraphs

\ערך{מקום }\הגדרה{- בית הקיבול אל הנושא המוכן להופיע }\מקור{[ר״מ קכח]}\צהגדרה{. }

\הגדרה{השטח החמרי של דבר, המכיל בתוכו איזה ענין מיוחד, וירמוז ברוחניות\mycircle{°} על המדרגה המיוחדה של איזה נושא }\מקור{[ע״ר א קסו]}\צהגדרה{. }

\הגדרה{ערך\mycircle{°} }\מקור{[עפ״י שם מ]}\צהגדרה{. }

\הגדרה{מדרגה ומעלה ידועה }\מקור{[מ״א ב ג]}\צהגדרה{. }

\הגדרה{מה שתופסת ההויה (האישית) }\מקור{[עפ״י ע״ר ב עו]}\צהגדרה{. }

\הגדרה{צורה\mycircle{°} תכליתית }\מקור{[ח״פ מה.]}\צהגדרה{. }

\משנה{מקומו ושרשו }\הגדרה{- תעודתו }\מקור{[עפ״י א״י לא]}\צהגדרה{. }

\paragraphs

\ערך{מקום }\הגדרה{- }\משנה{מקומו האמיתי של האדם }\הגדרה{- מדרגתו, (סדרי) הרגשתו הפנימית\mycircle{°} }\מקור{[ע״א ב 232]}\צהגדרה{. }

\paragraphs

\ערך{״מקום״ }\הגדרה{- }\משנה{״מקומו של עולם״ }\הגדרה{- ע׳ במדור שמות כינויים ותארים אלהיים.}

\paragraphs

\ערך{מקור }\הגדרה{- }\משנה{המקור האלהי }\הגדרה{- ע׳ במדור שמות כינויים ותארים אלהיים, אלהי, המקור האלהי. ע״ע אור אלהי, מקור האור האלהי.}

\paragraphs

\ערך{מקור הברכה\mycircle{°}}\הגדרה{ - יסוד האור העליון אשר ממנו נתייסדה תולדת ישראל, יסוד עולם שהוא קודם ונעלה מכל הגבלה\mycircle{°} וחוקיות מוטבעה\mycircle{°} }\מקור{[עפ״י אג׳ ג נח]}\צהגדרה{. }

\paragraphs

\ערך{מקור העליון }\הגדרה{- }\משנה{המקור העליון ששם שרויה היא המחשבה\hebrewmakaf האידיאלית\mycircle{°}}\הגדרה{ - אוצר החיים של השעשועים\mycircle{°} העליונים, אותה האידיאליות\mycircle{°} הנשגבה\mycircle{°} המכלילה את כל היקום בכל חטיביותו, שמשיגובה העליון של תועפות עזה\mycircle{°}, הוּפעה היצירה כולה לכל המון פלגיה, מראשית ההויה האצילית בתוקף מעלתה, עד הצורך האידיאלי העליון המביא בתוקף גבורת\mycircle{°} הודו\mycircle{°} את התכונה היצירתית לידי הגמרה עד תחתית שלביה }\מקור{[ר״מ קסח]}\צהגדרה{. }

\paragraphs

\ערך{מקור הקודש\hebrewmakaf העליון\mycircle{°}}\הגדרה{ - מקום\mycircle{°} החישוף של כל עדן\mycircle{°} ועדן, שם תערב\mycircle{°} מנחה\mycircle{°}, שם תתענג\mycircle{°} כל נשמה\mycircle{°} בחדות\hebrewmakaf ד׳\mycircle{°} מעוזה }\מקור{[ע״ר א קנא]}\צהגדרה{. }

\paragraphs

\ערך{מקוריות }\הגדרה{- }\משנה{המקוריות }\הגדרה{- העוצם\mycircle{°} שאין בו ערך\mycircle{°} של התעלות\mycircle{°} מרוב תעצומו }\מקור{[א״ק ב תקכח]}\צהגדרה{. }

\הגדרה{ע׳ במדור שמות כינויים ותארים אלהיים, מקור המקורות.}

\paragraphs

\ערך{מקוריות }\הגדרה{- }\משנה{(המקוריות המציאותית) }\הגדרה{- מקור\hebrewmakaf החיים\mycircle{°} וההויה העליונה\mycircle{°} בשלמותם, באין גבולים\mycircle{°} ומיצרים }\מקור{[א״ש יב ח1]}\צהגדרה{. }

\הגדרה{הראשית\mycircle{°}, השורש אשר משם יוצאים כל ענפי החיים ולשם ישובו ויחוברו }\מקור{[עפ״י מ״ר 144 (א״ש תוספות תשובה ו)]}\צהגדרה{. }

\הגדרה{ע״ע עצם, עצמיות. }

\paragraphs

\ערך{מקיף }\הגדרה{-}\משנה{ (לעומת ממלא\mycircle{°}) }\הגדרה{- שולט בהויית היצורים בכללותם }\מקור{[עפ״י ע״ר ב פב]}\צהגדרה{. }

\paragraphs

\ערך{מקיף }\הגדרה{- }\משנה{(לעומת פנימי) }\הגדרה{- ע׳ במדור מונחי קבלה ונסתר.  }

\paragraphs

\ערך{מקיר }\הגדרה{- מנביע ממקוריותו\mycircle{°} }\מקור{[עפ״י רצי״ה א״ש יא הערה 6]}\צהגדרה{. }

\paragraphs

\ערך{מרום }\הגדרה{- השאיפה האלהית הטהורה\mycircle{°} }\מקור{[ע״א ד ט קמו]}\צהגדרה{. }

\paragraphs

\ערך{״מרום״ }\הגדרה{- מקום\mycircle{°} פעולת המעשה שמגיע למרום\mycircle{°}, רמז לדבר ״רום ידיהו נשא״ }\מקור{[ע״א א א ו]}\צהגדרה{. }

\paragraphs

\ערך{מרומים }\הגדרה{- מקור הדעת\mycircle{°}, אוצר החיים אשר בנשמת\mycircle{°} חי\hebrewmakaf העולמים\mycircle{°} }\מקור{[א״ק א קעד]}\צהגדרה{. }

\תהגדרה{שפע\mycircle{°} זיו\mycircle{°} וזוהר\mycircle{°} הידיעה\hebrewmakaf האלקית\mycircle{°} העליונה }\תמקור{[נ״א ד 38].}

\paragraphs

\ערך{מרומים עליונים }\הגדרה{- מדרגה עליונה המתעלה מעל פרטיותנו המוגבלה\mycircle{°} }\מקור{[ע״ר א יא]}\צהגדרה{. }

\משנה{מרומים }\הגדרה{- עלוי הדרגות }\מקור{[עפ״י ר״מ ו]}\צהגדרה{. }

\paragraphs

\ערך{״מרוצה״ }\הגדרה{- }\משנה{(לפני ד׳) }\הגדרה{- ע״ע רצויה. }

\paragraphs

\ערך{מרוק }\הגדרה{- השתנות הטבע הנפשי\mycircle{°} לטוב\mycircle{°} בהעמדת התכונות לטוב, כאילו לא נעשה החטא\mycircle{°} שפוגם את התכונות בכללן }\מקור{[עפ״י ע״ר ב שנז]}\צהגדרה{. }

\הגדרה{זקוק מכל כעור החיצוניות, והעמדה על היסוד הפנימי\mycircle{°} החי חיי\hebrewmakaf אמת\mycircle{°} וקדש }\מקור{[עפ״י א״ש יא ו]}\צהגדרה{. }

\הגדרה{ע׳ במדור פסוקים ובטויי חז״ל, יסורים ממרקים.}

\paragraphs

\ערך{מרות }\הגדרה{- }\משנה{קבלת מרות}\הגדרה{ - הכנעה ושיבור הטבע העצמי, לעומת התנאים החיצונים }\מקור{[עפ״י ע״א ד יג יא]}\צהגדרה{.}

\paragraphs

\ערך{מרמה }\הגדרה{- דבר שאינו לפי הכונה הפנימית, ושעם זה מביא היזק לתועים, שאין הדברים כלבו }\מקור{[ע״א א 81]}\צהגדרה{. }

\הגדרה{ע״ע שקר. ע״ע כזב. ע״ע שוא.}

\paragraphs

\ערך{משבר }\הגדרה{- החשכה והסתרה של האור\mycircle{°} מפני שלא יוכלו הבריות בהשפלתן לקלוט אותו בבהירותו, למען יתן מקום לאור ממועט, שרק הוא יכשיר את החיים באחרונה לאור הגנוז היותר עליון ומגמתי\mycircle{°} }\מקור{[ע״א ד ט מג]}\צהגדרה{. }

\הגדרה{ע׳ בנספחות, מדור מחקרים,  ״שבירת הכלים״. }

\paragraphs

\ערך{משגב }\הגדרה{- }\משנה{(המשגב האלהי) }\הגדרה{- ישע\mycircle{°} השגוב, העילוי\mycircle{°} למקור\hebrewmakaf החיים\mycircle{°}, הנקלט בעצמיותנו בעריגת הקדש\mycircle{°} של נשמתנו\mycircle{°}. התחושה ברוחניות הפנימית את קרבתו ואחודו של ד׳\hebrewmakaf צבאות\mycircle{°} עמנו. <אחרי העלאתה של הנשמה, במרום חייה העליונים, אחרי שאיפת הקדש של אחדות\mycircle{°} ההויה במרום האצילות\mycircle{°} שלה> }\מקור{[עפ״י ע״ר א קמט]}\צהגדרה{. }

\paragraphs

\ערך{משובב דעתו של אדם }\הגדרה{- מאבד את הרגש של האחריות המוסרית }\מקור{[קובץ ד ב]}\צהגדרה{. }

\paragraphs

\ערך{משכיל }\הגדרה{- חושב, מצייר\mycircle{°} ומרגיש }\מקור{[עפ״י ע״ה קיז]}\צהגדרה{. }

\paragraphs

\ערך{משכן }\הגדרה{- שם בית הדירה, העלול להיות מוכן למסעות, שמרשם בתוכן הרוחני העליון (של האדם) את העליות הנכספות. המשכן מסמן, אם גם בצורה המחוברת באיזה אופן להכנה של נסיעה, אבל דוקא את היסוד המנוחתי שיש בין מסע למסע. ובנפשיות היא ההרגשה המרגעת, העוצרת את ההתנועעות הבלתי פוסקת, לשם היסוד של המטרה העליונה, קביעות האורה ובסיסיותה בערכה}\צהגדרה{ }\מקור{[ע״ר א מג]}\צהגדרה{.}

\הגדרה{ע״ע אוהל.}

\paragraphs

\משנה{משפט עליון }\צהגדרה{- מדת\hebrewmakaf הדין\mycircle{°}, שעלתה במחשבה\mycircle{°}, לפני בריאת העולם. מקור המשפט\mycircle{°} והצדקה\mycircle{°} }\צמקור{[א״ל קכח].}

\paragraphs

\ערך{משפט עליון}\צהגדרה{ - מדת המשפט העליונה}\הגדרה{ - כח אותה ההכרת המוסרית, החקוקה בנפש האדם הנברא בצלם\hebrewmakaf אלהים\mycircle{°} שהמשפט המושרש בטבעו של אדם ובארחות חיי החברה <לענוש את הרשע, הגונב, הגוזל והמרצח, באפן פרטי> מיסודה הוא בא}\צהגדרה{. }\הגדרה{״שופט הארץ״\mycircle{°} }\מקור{[עפ״י ע״ר א רטו]}\צהגדרה{.}

\משנה{תכונת המשפט\hebrewmakaf העליונה}\הגדרה{ - החשק היותר נאצל של המעמד היותר רוחני ורם של החברה האנושית }\מקור{[עפ״י ע״א ג א י]}\צהגדרה{.}

\paragraphs

\ערך{״משפטים״ }\הגדרה{- חוקי חיי החברה }\מקור{[פנ׳ צג]}\צהגדרה{.}

\צהגדרה{סידורים, סידורי חיים, סדרי הענינים}\myfootnote{ כמו בביטוי ׳משפט הנער ומעשהו׳ (שופטים יג יב), וכן ׳כמשפטם׳ ( במדבר כט ו, לג).\label{8}}\צהגדרה{ }\צמקור{[שי׳ ב 248].}

\משנה{משפטי התורה }\הגדרה{- ע׳ במדור מצוות, הלכות, מנהגים וטעמיהן, בהגדרות המבוא, מצוות, כל מצוותיה של התורה. ע׳ בנספחות, מדור מחקרים, משפט לעומת חק.}

\paragraphs

\ערך{משקל }\הגדרה{- }\משנה{תכונת המשקל}\הגדרה{ - צדוק הערכים, זה לעומת זה, כח ההכרעה }\מקור{[עפ״י ר״מ פז]}\צהגדרה{. }

\הגדרה{השוואת הערכים הזרים בהיותם נוכחים }\מקור{[עפ״י שם קלב]}\צהגדרה{. }

\paragraphs

\ערך{משקל עליון }\הגדרה{- }\משנה{המשקל העליון }\הגדרה{- בירורה של הכרעת ההנהגה העליונה }\מקור{[רצי״ה א״ש יא הערה 18]}\צהגדרה{.}

\הגדרה{ע׳ במדור מונחי קבלה ונסתר, מתקלא.}

\paragraphs

\ערך{מתבסם }\הגדרה{- מתקן, מתנעם, משתכלל }\מקור{[רצי״ה א״ש ט הערה 5]}\צהגדרה{. }

\הגדרה{ע״ע בסום.}

\paragraphs

\ערך{מתברך }\הגדרה{- מתמלא שפעת\mycircle{°} חיים טובים, נעימים ורעננים\mycircle{°} }\מקור{[עפ״י א״ק ג קפח]}\צהגדרה{. }

\הגדרה{ע״ע ברכה. ע״ע מברך.}

\paragraphs

\ערך{מתודה }\הגדרה{- אופן הטפול, ההתנהגות }\מקור{[רצי״ה א״ש יב הערה 36]}\צהגדרה{. }

\paragraphs

\ערך{מתוק}\הגדרה{ - מתוקן ומקובל}\צהגדרה{ }\מקור{[א׳ מא]}\צהגדרה{.}

\ערך{מתיקות }\הגדרה{- ההתאמה המכוונת של פרטי (הענין) לכל מה שהחיים וההויה בכללה צריכים לו שהוא }\משנה{יסוד המתיקות}\הגדרה{, הבא מתוך הרגש המתאים להתכונה שבנין הגוף דורש }\מקור{[עפ״י ע״ר ב נח]}\צהגדרה{. }

\paragraphs

\ערך{מתנודד }\הגדרה{- }\משנה{(לעומת קבוע\mycircle{°}) }\הגדרה{- משתנה ומתנועע בתנועה תדירה }\מקור{[עפ״י א״ק א רעד]}\צהגדרה{. }

\paragraphs

\ערך{מתנשא }\הגדרה{- }\משנה{תכונה מתנשאת }\הגדרה{- מתעלה מההתפשטות הגבולית }\מקור{[ר״מ קעד]}\צהגדרה{. }

\paragraphs

\ערך{מתפלש }\הגדרה{- ע״ע מפלשת. }

\paragraphs

\ערך{מתקדש }\הגדרה{- נבדל ונפרש }\מקור{[א״ק ב רצז]}\צהגדרה{. }

\הגדרה{נקשר בקשר האידיאליות\mycircle{°} של הקדושה\hebrewmakaf העליונה\mycircle{°} }\מקור{[עפ״י ע״א ד ט כו]}\צהגדרה{. }

\הגדרה{ע״ע קדוש. ע״ע מקודש.}\mylettertitle{נ}

\ערך{נאזר בגבורה }\הגדרה{- מלא עז\mycircle{°} חיים אמיצים ומכובדים }\מקור{[פנק׳ ד תמח]}\צהגדרה{.}

\paragraphs

\ערך{נאמן לעמו }\הגדרה{- שישים ישעו וחפצו לטובת כלל עמו בחמריותו\mycircle{°} ורוחניותו\mycircle{°} }\מקור{[עפ״י ע״ר א רפז (ע״א ב 322, מעשר שני, כד)]}\צהגדרה{. }

\paragraphs

\ערך{נבוב }\הגדרה{- שבתוכו אין כלום }\מקור{[ע״א ד יב ח (מא״ה ב קלג)]}\צהגדרה{. }

\הגדרה{מקום ריק בלא בניה, בלא מילוי }\מקור{[שם שם]}\צהגדרה{. }

\paragraphs

\ערך{נביבות }\הגדרה{- }\משנה{(במובן רוחני) }\הגדרה{- הריקניות מציורי\mycircle{°} הבנה }\מקור{[ע״א ד יב ח (מא״ה ב קלג)]}\צהגדרה{. }

\paragraphs

\ערך{נברא }\הגדרה{- הויה שפרטיות מתגלה בקרבה }\מקור{[א׳ קכה]}\צהגדרה{. }

\הגדרה{ניצוץ חיים אחד שפוע מחסד מקור\hebrewmakaf החיים\mycircle{°} כולם }\מקור{[קובץ ג נו]}\צהגדרה{.}

\הגדרה{ע״ע בריאה. }

\paragraphs

\ערך{״נגדלך״ }\הגדרה{- }\משנה{(באמירה כלפי ד׳) }\הגדרה{- כשאנחנו מסתכלים במפעלותיו של יוצר כל ב״ה, כל מה שהתכן של הדברים הנתפסים אצלנו בציור ורעיון הולך ומתרבה, ועשרה של המחשבה מתגדל, לפי מדה זו התפיסה הכללית מתגדלת, והציור המאחד את כל המון הפרטים הנפלאים לגודל אחד מתאחד בקרבנו, בצורה של אספקלריא גדולה, המביעה ביקר תפארתה את הגודל של מעשה ד׳ בהיקפה [}\צהגדרה{אצ״ל:}\הגדרה{ בהיקפו] הכללי, (ה)[ו]אומרת: הבו גודל לאלהינו. והננו אומרים לעומת ההגיון הקדוש הזה: }\משנה{נגדלך}\הגדרה{. ההיקף הכמותי של כל פלאי הפלאות של מפעלותיו של יוצר כל הנכנסים בדרך איחוד וגודל בציורינו }\מקור{[עפ״י ע״ר א קצח]}\צהגדרה{. }

\הגדרה{ע״ע גדולה. ע׳ בנספחות, מדור מחקרים, ״מה גדלו מעשיך ד׳״, ״מה רבו מעשיך ד׳״. ע״ע ״נפארך״.}

\paragraphs

\ערך{נֹגה }\הגדרה{- התחלת מציאות קצת קרן אור\mycircle{°} במעוט האפשרי לפי הערך, שהוא הנהו מקביל אל האור היותר גדול, אלא שלפי הרגשת חסרון האור הוא נקרא חושך }\מקור{[ע״ר א רלח]}\צהגדרה{. }

\הגדרה{[המקום\mycircle{°} שבו מתחיל קצת קרן אור להכנס. אמנם לפי המושג בתחילת הציור\mycircle{°} ידומה שנמצא החושך, מפני שע״י מעט האור אנו מרגישים מה שחסר באותו מקום מהאור }\צהגדרה{[עפ״י ע״א א 63]}\הגדרה{]. }

\משנה{נוגה }\צהגדרה{- מצב ביניים שבין מציאות אור\mycircle{°} לחושך\mycircle{°} }\צמקור{[מה״ה ד ר]. }

\הגדרה{ע״ע לילה, ״מדת לילה״.  }

\paragraphs

\ערך{נגע }\הגדרה{- רעה חלקית, במקרים ובתוכנים הסובבים את האדם, הפוגמת את הכל, ע״י החלק הנשחת המעורה ביתר החלקים }\מקור{[עפ״י ע״ר ב עו]}\צהגדרה{. }

\הגדרה{מכה חיצונית שאינה לפי סדר תהלוכות הגויה כולה }\מקור{[ע״א ב ט מה]}\צהגדרה{.}

\paragraphs

\ערך{נהלאה }\הגדרה{- עיפה }\מקור{[רצי״ה א״ש ג הערה 1]}\צהגדרה{. }

\paragraphs

\ערך{נהר }\הגדרה{- מורה על אחדות\mycircle{°} הכללית }\מקור{[ע״א א א מב]}\צהגדרה{. }

\paragraphs

\ערך{נהר }\הגדרה{- }\משנה{(רמיזתו) }\הגדרה{- החכמה\mycircle{°} המקורית, והטהרה\mycircle{°} הנמשכת ממנה }\מקור{[ע״א ב ז נב]}\צהגדרה{. }

\paragraphs

\ערך{נוזל}\הגדרה{ -}\משנה{ לפי הצורך של הכלכלה הגופנית, יחשב נוזל }\הגדרה{- מה שהוא דרוש לגופו של אדם כדי לתן מקום להתנועה התמידית שבקרבו למען המשכת החיים, <כי זאת היא תעודת המצב הדמי\mycircle{°}}\צהגדרה{> }\מקור{[ע״א ב ט ריד]}\צהגדרה{.}

\הגדרה{ע״ע מוצק, לפי הצורך של הכלכלה הגופנית.}

\paragraphs

\ערך{נוי }\הגדרה{- }\משנה{הנוי }\הגדרה{- הסדר, ההתאמה של החלקים זה בצד זה, זה אחר זה, וזה לעומת זה }\מקור{[ע״ר א קנב]}\צהגדרה{. }

\משנה{יסוד הנוי }\הגדרה{- הכוון והיחש שבין החלקים }\מקור{[שם רלג]}\צהגדרה{. }

\הגדרה{אמיתת ההוד\mycircle{°} }\מקור{[עפ״י ח״פ לז.]}\צהגדרה{. }

\הגדרה{שלמות הציור\mycircle{°} }\מקור{[ע״א א 157]}\צהגדרה{. }

\משנה{הנוי האמיתי }\הגדרה{- השלמת ציור המדמה\mycircle{°} כפי השורה השכלית\mycircle{°} }\מקור{[ע״ר א עניני תפילה כג, (ע״א א 149)]}\צהגדרה{. }

\מעוין{◊}\הגדרה{ הנוי ימצא תמיד מחיבור כל ההפכים }\מקור{[קבצ׳ א מג]}\צהגדרה{. }

\הגדרה{ע״ע יופי. ע׳ בנספחות, מדור מחקרים, נוי לעומת יופי. }

\paragraphs

\ערך{נועם }\הגדרה{- }\משנה{שטף נועם אלהי }\הגדרה{- מוסר\mycircle{°} עליון\mycircle{°}, קדושה\mycircle{°} וטהרה\mycircle{°} }\מקור{[א״ה 913]}\צהגדרה{. }

\משנה{״נועם ד׳״ }\הגדרה{- אהבת החכמה\mycircle{°} והטוב\mycircle{°} }\מקור{[ע״א א א צו]}\צהגדרה{. }

\משנה{הנעם העליון }\הגדרה{- השכל\mycircle{°} }\מקור{[ע״א ד ו כג]}\צהגדרה{.}

\הגדרה{עדן\hebrewmakaf גן\mycircle{°} סתרי חכמת הנשמה\hebrewmakaf העליונה\mycircle{°} של חמדת העולמים }\מקור{[א״ת ב 215]}\צהגדרה{. }

\משנה{נועם ד׳ }\הגדרה{- שפריר הרוחניות, ההכרה הפנימית העלומה}\צהגדרה{ }\מקור{[ע״א ד ט נ]}\צהגדרה{.  }

\הגדרה{ע׳ במדור פסוקים ובטויי חז״ל, לחזות בנועם ד׳ ולבקר בהיכלו. }

\paragraphs

\ערך{נועם }\הגדרה{- אידיאל המגמה\mycircle{°}, המקציבה הגבלה\mycircle{°} בקצה תכליתה. מופע תכונת החרות\mycircle{°} של בינת\hebrewmakaf כל\hebrewmakaf העולמים\mycircle{°}, עדן\mycircle{°} הציורים\mycircle{°} ויצירת האידיאלים\mycircle{°} באופן הראוי לרדת לתפיסת הוית אידיאלים מוגבלים, יוצרי עולמים, מהוי חיים, ממציאי נשמות, ומחדשי צבאי צבאות, המגלה את הוד\hebrewmakaf הבינה\mycircle{°} בכל מערכי ההויה }\מקור{[עפ״י א״ק ב רפג\hebrewmakaf ה]}\צהגדרה{. }

\משנה{הנועם העליון של עולם\hebrewmakaf הבא\mycircle{°}}\הגדרה{ - אור הבינה\mycircle{°}, המתגלה במחשבות הגדולות\mycircle{°}, הכלליות }\מקור{[שם תקט]}\צהגדרה{. }

\הגדרה{הופעת נשמות\hebrewmakaf חדשות\mycircle{°} אשר יאירו באור המופלא, עד שהתוכן של כל הנשמות\mycircle{°} היותר עליונות של עכשיו רק כנפש הבהמה יחשבו לעומת אותו אור החכמה\mycircle{°}, אותו עז\mycircle{°} החיים בבהירות קדשם, לגבי אותן הנשמות החדשות אשר תופענה בתיקון\hebrewmakaf העולם\mycircle{°} }\מקור{[קבצ׳ א ריג (פנק׳ א תקי)]}\צהגדרה{. }

\הגדרה{ע״ע שכינה עליונה. }

\paragraphs

\ערך{נועם }\הגדרה{- }\משנה{נועם אור אלוה נורא הוד }\הגדרה{- ע״ע אור אלהי. }

\paragraphs

\ערך{נועם }\הגדרה{-}\משנה{ העונג והנועם האלהי\mycircle{°}}\הגדרה{ - התכן היותר עליון\mycircle{°}, המופיע מתוך האורה\hebrewmakaf האלהית\mycircle{°}, המתגלה על מערכת החיים וההויה כולה ומשכלל את המפעלים כולם, שיהיו ראויים כל אחד להביא את פועלו לתכלית האושר\mycircle{°} }\מקור{[ע״ר ב עד]}\צהגדרה{. }

\הגדרה{ע״ע עונג, (עונג אלהי בהויה). }

\paragraphs

\ערך{נועם ד׳ }\הגדרה{- הנועם הנשאב אצל כל אחד ואחד משפעת התורה והמצות שלו, מפעולתם וכונתם, שכשם שהוא נוהג בכל יחיד, ככה הוא נוהג בכללות האומה <היונקת תמיד ממתק טוב הקודש הפנימי של נועם ד׳>. הנועם הממתיק את החיים ומשפיע צורה כבודה ומתוקנה, וחיים מלאים תוכן של קורת רוח פנימית בעומק הנשמה הכללית של כנסת\hebrewmakaf ישראל\mycircle{°}, שקויה וניצוצי אורה משתלחים הם אל כל יחיד ויחיד, לפי ערך הכנתו, לפי זוהר נשמתו, ולפי רוב קישורו העצמי אל הכלל. הנועם המרגיע ונותן עונג פנימי לנפש גם בתוך מחשכיה }\מקור{[עפ״י א״ק ג קעד]}\צהגדרה{.}

\paragraphs

\ערך{נועם ד׳ }\הגדרה{- }\משנה{העונג של נועם ד׳ }\הגדרה{- ע״ע עונג של נועם ד׳.}

\paragraphs

\ערך{נְחִיָּה }\הגדרה{- הנהגה של נחת, מנוחה והשקטת התרגשות שאפשר להמצא בצבא גדול }\מקור{[ע״א ב 309]}\צהגדרה{. }

\paragraphs

\ערך{נחל ד׳ }\הגדרה{- זיו\mycircle{°} החסד\mycircle{°} והטהר\mycircle{°} של המדות היותר נעלות, וחפץ הטוב\mycircle{°}, החסד והצדקה\mycircle{°} }\מקור{[נ״א יב 10]}\צהגדרה{. }

\paragraphs

\צמשנה{ניצב }\הגדרה{- התמדת הקיום במצב אחד }\מקור{[ע״א א א מט]}\צהגדרה{.}

\paragraphs

\ערך{נמוך }\הגדרה{- }\משנה{(לעומת גבוה\mycircle{°}}\הגדרה{) - פרטי\mycircle{°} ומעשי }\מקור{[עפ״י קובץ ג קז]}\צהגדרה{. }

\paragraphs

\ערך{נמנע}\myfootnote{ ע׳ ש״ק, קובץ ו נ. קובץ ז מא, נג, נח, קנא. קובץ ח קנה. פנק׳ א שיא סי׳ ו. פנק׳ ג קלד סי׳ ב. ע״א א ג לה. ע״ע מו״נ ח״ג פט״ו ח״ב פי״ג [עמ׳ 299 במהדורת שורץ]. שו״ת הרשב״א סי׳ תיז, ס׳ העקרים ח״ג פרק כה. אלימה לרמ״ק פרק יט, דף ז:, י׳ מאמרות לרמ״ע, חקו״ד ח״ד פ״ט. דף קטז. קלח פ״ח לרמח״ל, פתח ל ״והטבע הזה שאנחנו רואים אותו בעינינו בכל המוחשות שבו הוא עשאו כך במוחשות האלה. והוא אינו מוגבל תחת שום סדר או חק כלל ועיקר. אלא כל מה שאנו אומרים אי אפשר בלא הכי, רצה לומר לפי דרך ההדרגה אי אפשר כן. וזה אמת״. והעיר על דברי רמח״ל אלה ר״א מרקוס בקסת הסופר בראשית א, ב עמ׳ כא: ״וזה מה שכתב הרמב״ם במורה: ״הנמנע יש לו טבע קיים״״. ע״ע לשב״ו ספר הדע״ה חלק א דרוש ה סימן ז, ד״ה אמנם אין כוונתי בנמנעות כמחשבת המחקרים בזה, דף נח.. ובס׳ הדע״ה עמ׳ 115. והרחרל״פ במי מרום ח״א עמ׳ קס: ״מה שנמצאו אומרים שאין הנמנעות מכלל היכולת, אין שום צדק בזה להעלות על הדעת שאפשר להגדיר את היכולת העליונה בשום הגדרה איזו שהיא, אלא שיש צדק לומר שמבראשית בהתחוללות קביעות גדרי ההוי׳ שקבע הוא יתברך, קבע ככה ברצונו בדברים האלה שהשינוי בהם הוא מן הנמנעות, ואין שום הגדרה שאיננה רצונית״. וע׳ רופא הנפשות, לקט דברי אמונה ומוסר מהכתבים הרפואיים לרמב״ם, למו״ר הר״ש אבינר, עמ׳ 34, 157-158, 169-170. ובמעבר לפילוסופיה עמ׳ 32-38.\label{1}}\הגדרה{ - }\משנה{חק הנמנע }\הגדרה{- הכלי\mycircle{°} הראשון שהוצרך להיות מוקדם להתפשטות הטוב הראשון של הארת היכולת\mycircle{°} שעל פי עומק החכמה\mycircle{°} והחסד\hebrewmakaf העליון\mycircle{°} }\מקור{[עפ״י קובץ ז נג]}\צהגדרה{. }

\הגדרה{ע׳ במדור מונחי קבלה ונסתר, רשימו. ושם, צמצום, הצמצום הקדמון.}

\paragraphs

\ערך{נס }\הגדרה{- ע״ע דגל.}

\paragraphs

\ערך{נס }\הגדרה{- דליגה חפשית רצונית בתוך המעגל שבטבע\mycircle{°} עצמו }\מקור{[ע״א ד יא יח]}\צהגדרה{. }

\הגדרה{קפיצות כמה מדריגות בבת אחת במקום הצורך }\מקור{[ה׳ רכ]}\צהגדרה{. }

\ערך{נס }\הגדרה{- התגלות האור\mycircle{°} של חיי האחדות\mycircle{°} באדם ובעולם, עם תהלוכות מוסרו\mycircle{°} וכל שאיפותיו היותר עמוקות, המקיפות\mycircle{°} וממלאות\mycircle{°} את הכל }\מקור{[עפ״י א״ק ב תנח]}\צהגדרה{. }

\הגדרה{גילוי חפץ אידיאלי\mycircle{°} ורצון חי כללי, מדויק ומפואר, בכל פאר\mycircle{°} העדינות היותר עליונה }\מקור{[ע״ט ה (קובץ ב ז)]}\צהגדרה{. }

\הגדרה{מקום שאור\mycircle{°} החיים הפנימיים\mycircle{°} הבהיק במלא אורו }\מקור{[א״ק ב שנט]}\צהגדרה{. }

\הגדרה{התנועה הפתאומית, האפשרות המתגלה להעתיק איתני הטבע ממשמרתם למען תכלית מוסרית נשגבה ואדירה אשר נגלתה בעולם ע״י העמדת ישראל ופעולתו בעולם }\מקור{[עפ״י ע״א ג 128]}\צהגדרה{. }

\הגדרה{פלאי אדון\hebrewmakaf כל\mycircle{°}, בהריסת הסדרים הקבועים הטבעיים למטרה רוממה נשגבה במוסר וחכמה}\צהגדרה{ }\מקור{[ע״א ד ה ג]}\צהגדרה{.}

\משנה{נסים }\הגדרה{- התגלות\mycircle{°} כח סדר המצב המוסרי\mycircle{°} של המציאות }\מקור{[עפ״י ע״ר א רסח\hebrewmakaf ט]}\צהגדרה{. }

\מעוין{◊}\הגדרה{ ההופעה של ה}\משנה{ניסים }\הגדרה{באה מההתגלות\mycircle{°} העליונה של הארת עולם נשגב\mycircle{°} בחוקיו העומד בסדוריו ע״פ דעת\hebrewmakaf עליון\mycircle{°} וסוד רצונו, למעלה למעלה מכל חק וסדר הנהוג בעולם המוגבל שחושינו מגיעים אליו }\מקור{[ע״א ד יא יח (ה׳ רל)]}\צהגדרה{. }

\משנה{נסים }\צהגדרה{- מה שלמעלה ממהלך קביעות תנאי ההויה וסדריה}\צמקור{ [צ״צ קמג].}

\הגדרה{ע״ע מופת. }

\משנה{תכלית הניסים }\הגדרה{- להכין את האדם, להעלותו, להרמת השלמות האנושית ממעלה שפלה, למעלה שהיא רמה ממנה }\מקור{[מ״ש קה]}\צהגדרה{. }

\הגדרה{לצורך ההעלאה אל השלמות, למדרגה עליונה שאינה לפי ההדרגה, ולהכין למעלה יותר גבוהה }\מקור{[שם]}\צהגדרה{. }

\הגדרה{להיות פועלים הגברת הכח המוסרי על שאר כחות הנפש }\מקור{[ע״א ב ט ק]}\צהגדרה{. }

\הגדרה{ע׳ בנספחות, מדור מחקרים, הנהגת הטבע והנהגה הנסיית. }

\paragraphs

\ערך{נס }\הגדרה{- }\משנה{הנהגת הנס }\הגדרה{- בהשגחה\mycircle{°} פרטית }\מקור{[עפ״י מ״ש רפו (מא״ה ג קפח)]}\צהגדרה{. }

\הגדרה{ע׳ בנספחות, מדור מחקרים, הנהגת הטבע והנהגה הנסיית. }

\paragraphs

\ערך{״נס הניתן לכתוב״, ״נס שלא ניתן לכתוב״ }\הגדרה{- ע׳ במדור פסוקים ובטויי חז״ל. }

\paragraphs

\ערך{נס כללי }\הגדרה{- }\משנה{״נס בתוך נס״}\myfootnote{ שבת צז., פסחים קיח:, סוטה מז:, חולין קכז..\label{2}}\הגדרה{ - נס\mycircle{°} כפול, נס להתנוסס, העוקר את כל ההגבלות של כל חק שבטבע הנסתר ושבטבע הגלוי }\מקור{[עפ״י ע״א ד יא יח]}\צהגדרה{. }

\ערך{נס }\הגדרה{- }\משנה{נס מוחלט }\הגדרה{- אור\hebrewmakaf ד׳\mycircle{°} המתעלה מעל כל מסגרת של כח טבע\mycircle{°} מצומצם\mycircle{°}, ומרכז ע״י גדולתו\mycircle{°} את כל מהות החיים וכל יסוד השאיפות להנצח\mycircle{°} וההוד\mycircle{°} העליון\mycircle{°} }\מקור{[עפ״י ע״ר א מט]}\צהגדרה{. }

\הגדרה{הלמעלה מהטבעיות, אור ד׳ הרחב והחפשי }\מקור{[אג׳ ג נט]}\צהגדרה{. }

\משנה{יסוד הנס באצילותו העליונה }\הגדרה{- הכח המעמיד את כל ההויה הרוחנית\mycircle{°} והחומרית\mycircle{°} לא בתכונה של קישורי סיבות\mycircle{°} ומסובבים כי אם ישר\mycircle{°} בדבר\hebrewmakaf ד׳\mycircle{°} יוצר כל }\מקור{[ע״א ד יב ו (מא״ה ב קלב)]}\צהגדרה{. }

\משנה{נס }\הגדרה{- הכח המנצח את כל מסיבות הטבע\mycircle{°}. עטרת הטבע, חייו, נשמתו הפנימית\mycircle{°}, האוצר האידיאלי\mycircle{°} שלו }\מקור{[עפ״י ע״ר א מט, א״ק ד תקד (א״א 73)]}\צהגדרה{. }

\הגדרה{פנימיות\mycircle{°} הטבע. הנהגה שעל פי תכלית המשפט והיושר\mycircle{°} האמיתי, שגלוי רק לפניו יתברך שמו\mycircle{°} }\מקור{[ח״פ כו.]}\צהגדרה{. }

\משנה{עולם הנס }\הגדרה{- העולם\hebrewmakaf העליון\mycircle{°}, הופעת השליטה הנשמתית }\מקור{[עפ״י א׳ כז\hebrewmakaf ח]}\צהגדרה{. }

\הגדרה{ע״ע טבע, עולם הטבע. ע׳ בנספחות, מדור מחקרים, הנהגת הטבע והנהגה הנסיית. ע״ע טבע הניסי, הטבע הניסי. ע׳ במדור פסוקים ובטויי חז״ל, תנאי התנה הקב״ה עם כל מה שנברא בששת ימי  בראשית וכו׳. }

\ערך{נס }\הגדרה{- }\משנה{יסוד הנסים המתגלים ברצון\hebrewmakaf ד׳\mycircle{°}}\הגדרה{ - השכל\hebrewmakaf העליון\mycircle{°}, האלהי\mycircle{°}, המאיר להדמיון\mycircle{°}. כל ההויה המתוארת בחוקיה אינה כי אם לפי הדמיון שלנו, שרק אור השכל של חכמה תתאה מאיר בו, להיות חקי הטבע\mycircle{°} מסודרים על ידו, ואור השכל העליון שולט הוא על הדמיון ומבטלו, ובתוך ארג זה נעשו הנסים }\מקור{[עפ״י א״ק א רלב]}\צהגדרה{. }

\ערך{נס }\הגדרה{- }\משנה{הנסים והנפלאות שביציאת מצרים }\הגדרה{- ע׳ במדור מועדים וחגים, פסח, יציאת מצרים. }

\paragraphs

\ערך{נס קריעת ים סוף }\הגדרה{- דבר מופלא, בהתפשטות הנס בכל המציאות של המים\mycircle{°}, והי׳ ביטול לסדר טבע כולל, המורה שסילק ית׳ יד הטבע בכללו מחוק רשום כולו של המציאות. <והנה סר בזה כח הסיבות והמסובבים, ונמשכה ההנהגה עפ״ז לשעה זו בכל המציאות שלא בדרך עילה ועלול וקישור סיבות, כ״א בתוקף יד השי״ת והשגחתו הגלויה הבלתי מתלבשת כלל בלבושי הסיבות וקישורי העלולים. }\צהגדרה{כי כיון שכל חלקי המציאות ע״פ דרכי סיבותיהם ועילותיהם בהנהגה הטבעית אחוזים וקשורים זה בזה, וכל דבר מנהיג את אחיו ומונהג ממנו, ופועל עליו כמו שנפעל ממנו, וכענין ביאור הרמב״ם במו״נ שהמציאות כולה כאדם גדול בעל איברים, א״כ בביטול כח עיקרי ממנו לגמרי הרי עושה רושם בכל סדר המציאות כולו, וא״א בשו״א שינהג מנהגו בשעה זו ע״פ מנהג קשר הסיבות ואחיזתן ע״פ עילה ועלול בהדרגת ההנהגה הנהוגה. ונמצא}\הגדרה{> שיצא הטבע בשעה זו מלבושי קישור הטבעיות, ונמשך כולו רק ע״פ גילוי החפץ העליון בלא שום הלבשה טבעית מסודרת }\מקור{[מ״ש שנד]}\צהגדרה{.}

\הגדרה{הנס ממדרגה עליונה, שתנאי התנה הקב״ה בעת יצירת הטבע שיהיה מקום לנסים ולהשתנות הטבע, <ומשום זה }\צהגדרה{הנס של קריעת ים סוף }\הגדרה{נקרא מקור הנסים שהשתתפו בו נסים עליונים ותחתונים, ע״י הרצון והמחשבה העליונה שהיתה בזה לפני בריאת העולם>. מקור המחשבה\mycircle{°} ושבירת הטבע }\מקור{[עפ״י טוב רואי, שבת פרק ב מה. דרשות לביאור ניסי חג החנוכה]}\צהגדרה{.}

\paragraphs

\ערך{נסים נסתרים}\הגדרה{ - נסתרים במערכות סדרי הטבע}\צהגדרה{ }\מקור{[עפ״י ל״ה 177]}\צהגדרה{.}

\paragraphs

\ערך{נסיון }\הגדרה{- כשהאדם פתאום הוא מוצא את עצמו במצב חדש, בסביבה חדשה, בעולם חדש, נשטף מזרמים חדשים המתיצבים לעומתו, מה שלא עלו כלל על לבו שהם יבואו כנגדו. שצריכים אז לחדש את כל מעמד החיים הרוחניים\mycircle{°}, ולהתאזר בגבורה חדשה מקורית ממבוע הנשמה\mycircle{°}, להיות נשאר עומד בקדושתו\mycircle{°} ומלא צביון תומתו }\מקור{[עפ״י ע״ר א עז]}\צהגדרה{. }

\הגדרה{ההכשרה לגלות את האופי הרצוני החפשי\mycircle{°} של האדם }\מקור{[שם פה]}\צהגדרה{. }

\משנה{יסוד הנסיון }\הגדרה{- רצון חפשי\mycircle{°} מוחלט, בלתי מקושר בשום הכרח פנימי\mycircle{°} או חצוני\mycircle{°}, כדי שעצמיותו הפנימית הבלתי תלויה (של המנוסה) תופיע בהדר\mycircle{°} גאון עזה\mycircle{°} }\מקור{[עפ״י שם]}\צהגדרה{. }

\משנה{תכלית הנסיון }\הגדרה{- הוצאת השלמות הצפונה בעומק הנפש מן הכח אל הפועל הגמור }\מקור{[ע״א א 135]}\צהגדרה{. }

\paragraphs

\ערך{״נעבדך ביראה״ }\הגדרה{- נרשים את היראה\mycircle{°} הקדושה במפעל }\מקור{[עפ״י ע״ר א קפה]}\צהגדרה{.}

\paragraphs

\ערך{נעלם }\הגדרה{- ענין גדול ומושכל זך וצח }\מקור{[עפ״י ע״א א ב ז]}\צהגדרה{. }

\הגדרה{מתעלה מכל רעיון ומכל מחשבה, ובא מיסוד התגלות אלהית עליונה }\מקור{[עפ״י ע״ר א קיא]}\צהגדרה{.}

\paragraphs

\ערך{״נפארך״ }\הגדרה{- }\משנה{(באמירה כלפי ד׳) }\הגדרה{- התוכן האיכותי, שהוא מתאים אל המפעלים עצמם, שבהם הרעיון בא לגבול השלמת קדושתו, שעל ידם אנו מציצים מן החרכים הרוחניים, על תפארת המגמות האידיאליות, הנאדרות בקודש, של כל ההויות הנפלאות, ואנו מתנשאים ע״י זה לתפארת\hebrewmakaf העליונה\mycircle{°} ברום\mycircle{°} עזה\mycircle{°} }\מקור{[עפ״י ע״ר א קצח]}\צהגדרה{.}

\הגדרה{ע׳ במדור שמות כינויים ותארים אלהיים, ״מפואר״. ע״ע פאר (לד׳). ע״ע ״נגדלך״.}

\paragraphs

\ערך{נפרז}\הגדרה{ - העובר את הגבול הראוי }\מקור{[עפ״י ע״א ב ט ריז]}\צהגדרה{.}

\paragraphs

\ערך{נפש החיים שבהויה }\הגדרה{- רצון\hebrewmakaf העולם\mycircle{°} הפועל ושואף }\מקור{[א״ק ב שסט]}\צהגדרה{.}

\הגדרה{ע׳ במדור משיח וגאולה, ״רוחא דמלכא משיחא״. ע׳ במדור נפשיות, נפש החיים.}

\paragraphs

\ערך{נפשי }\הגדרה{- }\משנה{(לעומת גופני\mycircle{°}) }\הגדרה{- איכותי פנימי\mycircle{°} (לעומת כמותי חיצוני\mycircle{°}) }\מקור{[רצי״ה א״ש יד הערה 31]}\צהגדרה{.}

\paragraphs

\צמשנה{נצב }\הגדרה{- ע״ע ניצב.}

\paragraphs

\ערך{נצח }\הגדרה{- הוראת נצוח המורה התגברות של פגישה על ההתנגדות }\מקור{[ע״ר א רלד]}\צהגדרה{. }

\paragraphs

\ערך{נצח}\הגדרה{ - כח הוספת הכח שהושפע מצד המנגד עצמו. תכונת התנועה הנצחית שמכל מפריז ומתנגד לה מוסיפה אומץ להגביר חיל }\מקור{[ע״ר א רלא (ע״א ב ט שז)]}\צהגדרה{.}

\הגדרה{ע׳ במדור פסוקים ובטויי חז״ל, ויז נצחם על בגדי.}

\paragraphs

\ערך{נצח }\הגדרה{- מצייר\mycircle{°} את ההתמדה העשירה של שטף בלתי פוסק, המנצח את כל אפיסה וכל כליון }\מקור{[ע״ר א קצו]}\צהגדרה{. }

\הגדרה{ע״ע עד. }

\paragraphs

\ערך{נצחיות עליונה }\הגדרה{- }\משנה{מידתה}\הגדרה{ - התממת החיות עם הקיום, התמידיות שאיננה בעלת שינוי ותמורה <מה שאין לו ציור בשום ציור של חיים בעולם> עם הקיום והעמדה המתמידה שבה שטף של ההתמדה הזמנית, מבלי להקשר עם החסרון וההעדר שלה }\מקור{[עפ״י ע״ר א קצו]}\צהגדרה{.}

\הגדרה{ע״ע עד.}

\paragraphs

\ערך{נקבה }\הגדרה{- }\משנה{נקבי}\הגדרה{ - חומר המקבל פעולה, נושא מתפעל, ההפך ממושג נושא פועל }\מקור{[ר״מ קל]}\צהגדרה{.}

\הגדרה{ע״ע זכר, זכרי.}

\paragraphs

\ערך{נקודה }\הגדרה{- הערך היסודי }\מקור{[עפ״י רצי״ה א״ש ד הערה 14]}\צהגדרה{. }

\paragraphs

\ערך{נקודה עצמית }\הגדרה{- (}\משנה{הנקודה העצמית באדם}\הגדרה{) - ע׳ במדור נפשיות.}

\paragraphs

\ערך{נקודת האמונה }\הגדרה{- ע״ע אמונה, נקודת האמונה. }

\paragraphs

\ערך{נקיות }\הגדרה{- }\משנה{הנקיות (במסילת ישרים) }\הגדרה{- כשידקדק בנקיון הלב וישעבד עיניו ללבו ושכלו במה שאין החטא גלוי, כי כאשר התאוה מושלת בלב גם מעט, תכסה על הפשע ולא ירגיש, אבל כשידקדק יראה ויכיר וינקה }\מקור{[עפ״י מ״ר 274]}\צהגדרה{.}

\paragraphs

\ערך{נשגב }\הגדרה{- מה שהוא למעלה בחיים ובהויה מהחושים המטומטמים שלנו, שנאטמו מרוב טומאה\mycircle{°} וצרה }\מקור{[א׳ קכח]}\צהגדרה{. }

\משנה{הנשגב }\הגדרה{- הטוב\mycircle{°} המוחלט בשכל ובחיים }\מקור{[שם מח]}\צהגדרה{. }

\paragraphs

\ערך{נשגב }\הגדרה{- }\משנה{הנשגב הכללי }\הגדרה{- אור\hebrewmakaf החיים\mycircle{°} העליונים\mycircle{°}, המסדרים את הכל לאושר\mycircle{°} עליון ונעלה }\מקור{[א״ק ג רפ]}\צהגדרה{. }

\paragraphs

\ערך{נשיא }\הגדרה{- נושא אל הכבוד הלאומי, נושא נזר, <שכשמולך בגאון\hebrewmakaf ד׳ הוא שוה מאד לכבוד העם והדרו> }\מקור{[קבצ׳ ב לה\hebrewmakaf ו (פנק׳ א לא)]}\צהגדרה{.}

\הגדרה{נושא כבוד כלל האומה}\צהגדרה{ }\מקור{[שם]}\צהגדרה{.}

\הגדרה{ע״ע מלך.}

\paragraphs

\ערך{נשיה }\הגדרה{- העדר התביעה הטבעית, קהיון החושים }\מקור{[ע״א ג ב סה]}\צהגדרה{. }

\הגדרה{העקירה שנעקרה תביעה טבעית מן הלב }\מקור{[עפ״י שם סז]}\צהגדרה{. }

\ערך{״נשיית טובה״}\myfootnote{ איכה ג יז, שבת כה:.\label{3}}\הגדרה{ - הדילדול היסודי שמביא להטביע את כל הנפש האנושית במרירות של מגור ופחד, של דאגת צרה ויגון עוני וחוסר כל, עד שהתביעות הטבעיות, הראויות לאדם החי חיים שלמים הראויים לאיש בעל דעה וטעם העומד להיות לנס בחכמה ויושר לב, הם משתכחים ואינם נתבעים כלל }\מקור{[עפ״י שם סו]}\צהגדרה{. }

\paragraphs

\ערך{נשים }\הגדרה{- }\משנה{(לעומת אנשים\mycircle{°}) }\הגדרה{- היסוד הנפעל בחברה }\מקור{[ע״א ד ו כז]}\צהגדרה{. }

\paragraphs

\משנה{נשמעים לו לעצמו }\הגדרה{- }\צמשנה{״כל מי שיש בו יראת שמים דבריו נשמעים״}\myfootnote{ ברכות ו:. \textbf{דהיינו נשמעים לו לעצמו} - כדמרגלא בפומיה דהרצי״ה, ע׳ א״ל כה. ע״ע ע״א א ד לז. ש״ק, קובץ ו נד. ע״ע אור החיים עה״ת, בראשית א א, אופן יז. נפה״ח, הקדמת בן המחבר, ד״ה היה אוהב תוכחת מוסר, ובהערה שם.\label{4}}\צהגדרה{ - }\צהגדרהמודגשת{דהיינו נשמעים לו לעצמו}\צהגדרה{ - שמחזיק ומדריך עצמו כראוי בהנהגה ישרה ואמיצה ומרוכזת ומעוררת }\צמקור{[א״ל שט].}

\צהגדרה{להוציאם לפועל ממש ולקביעות מעשה }\צמקור{[א״ל כה].}

\paragraphs

\משנה{נשמת אומה }\צהגדרה{- הפסיכולוגיה\hebrewmakaf הצבורית\mycircle{°} האמיתית}\צמקור{ [ל״י א (מהדורת בית אל תשס״ב) ח]. }

\הגדרה{ע׳ במדור מלאכים ושדים, שר האומה. }

\paragraphs

\משנה{נשמת האומה }\הגדרה{- }\צהגדרה{נצח\hebrewmakaf ישראל\mycircle{°}. הפסיכולוגיה\hebrewmakaf הצבורית\mycircle{°}. משהו שמחיה אותנו, קיים ונצחי, עילאי אלוהי, טהור עליון }\צמקור{[עפ״י שי׳ ה 485].}

\ערך{נשמת האומה}\הגדרה{ - כנסת\hebrewmakaf ישראל\mycircle{°} }\מקור{[ע״ר ב קנח, אג׳ א עא, שם ב שסה]}\צהגדרה{. }

\הגדרה{הצורה\mycircle{°} המזוקקת של האומה\mycircle{°}, חפץ ואמץ עליונות האלהית\mycircle{°} בעולם }\מקור{[א״ה ו 167]}\צהגדרה{.}

\הגדרה{רוח התולדה המפורט, המפעם בהסתוריה המפליאה שלנו, במזג דמנו וטבע גופניותנו, וביחוד שתופנו הקוסמי, כלומר תעודת חיינו בכח ההויה בכלליותה, המשתרעת ממעל לכל הגבולים, עד תאות גבעות עולם, שאין כל הזיה שבעולם מוכשרה לסמל צבעיה, ושם חביון הכח ועצם החיים וגבורת האמת המפליאה לעשות, זרוע\hebrewmakaf ד׳\mycircle{°} המנהלת חדת העולמים והלאומים }\מקור{[עפ״י אג׳ ג יג\hebrewmakaf יד]}\צהגדרה{. }

\הגדרה{אור\mycircle{°} אלהים\hebrewmakaf חיים\mycircle{°} בעולמים כולם }\מקור{[שם קכו]}\צהגדרה{. }

\הגדרה{התורה\mycircle{°} }\מקור{[א׳ צד]}\צהגדרה{. }

\משנה{נשמת האומה כולה }\הגדרה{- מקור החיים הנצחיים }\מקור{[קבצ׳ א קעה]}\צהגדרה{. }

\משנה{ראש הפסגה של נשמת האומה }\הגדרה{- המגמה\mycircle{°} הכללית שהאומה שואפת אליה בעצם הויתה }\מקור{[א״ש ד ו]}\צהגדרה{. }

\משנה{נשמת אומתנו הפנימית}\myfootnote{ \textbf{נשמת אומתנו הפנימית }\textbf{-}\textbf{ שם ד׳ }\textbf{אלהי}\textbf{ ישראל} - ויק״ר פר׳ יא ג ״שלחה נערותיה תקרא - אלו ישראל, על גפי מרומי קרת - שהטיסן הקב״ה וקרא אותן אלהות שנאמר אני אמרתי אלהים אתם״. במד״ר פר׳ ה ו ״למען שמי אאריך אפי אלו ישראל שייחד הקב״ה שמו עליהם אנכי ה׳ אלהיך ושיתף שמו בשמם ישראל״. ע״ע שפע טל הקדמה בן מאה שנה. אור החיים עה״ת, דברים ד ד״ה גם רמז ״שדבקות ישראל הוא בשמות עצמם מתאחדים אור נפשם באור שם הנכבד וכו׳ והוא מ״ש כי חלק ה׳ עמו הרי כי חלק א׳ הם ישראל משם הוי״ה ב״ה״. וכן שם שמות כ כ ד״ה אכן. ובנפה״ח, שער א יט ״ומה נעמו אמרי חז״ל בירושלמי תענית פ״ב. ר״ל בשם ר״י אמר שיתף הקב״ה שמו בישראל. משל למלך שהיה לו מפתח של פלטרין קטנה וכו׳ אלא הריני משתף את שמי הגדול בהם. והם ז״ל דברו לענין כלל האומה יחידה. וכו׳. שהג׳ דרגין נר״ן ושרש הנשמה מקור שרשם הוא מהד׳ אותיות השם הגדול ית״ש״.\label{5}}\הגדרה{ - שם\hebrewmakaf ד׳\hebrewmakaf אלהי\hebrewmakaf ישראל\mycircle{°}, מקור כל הופעת הרוח של חיינו הלאומיים בטהרתם\mycircle{°} }\מקור{[עפ״י אג׳ ג קצג]}\צהגדרה{. }

\משנה{המאור הפנימי של כללות האומה }\הגדרה{- הכח הכמוס האלהי שיש במגמת הוייתה של האומה בעולם, שהוא הסוד של כל ההויה כולה במקור האור\hebrewmakaf האלהי\mycircle{°}, שהוא נחל עדנים שאין לו סוף, ומקור עדן נצחי לכל נשמת חיים, המהפך את הכל לאור\hebrewmakaf חיים\mycircle{°} }\מקור{[עפ״י קבצ׳ א קעה]}\צהגדרה{. }

\משנה{הנשמה הלאומית }\הגדרה{- ההקשבה\mycircle{°} הפנימית\mycircle{°} הישראלית של כל הזמנים והתקופות התולדתיים, כח הקודש\mycircle{°}, הכח האלהי המנהיג את העולם כולו }\מקור{[אג׳ ג קסז]}\צהגדרה{. }

\הגדרה{החפץ הגדול והאדיר, המחוכם והנאור, הכולל חיי עולמים כולם, עז\mycircle{°} וגבורה\mycircle{°}, שלום\mycircle{°} וברכה\mycircle{°} לאדם רב, חפץ החופש\hebrewmakaf העליון\mycircle{°} והנשגב, המאשר את האנושיות כולה, ומוציאה מן השפלות הבזויה, של שיעבוד האליליות\mycircle{°} ושל העבדות\mycircle{°} של הכפירה\mycircle{°} והחמריות\mycircle{°} השפלה גם יחד, החופש האלהי הנשא והנעלה המרומם את האדם כולו, בחיי היחיד ובחיי המשפחה, בחיי המדינה והממלכה ובחיי הצבור האנושי בכללו, למעלת החרות\mycircle{°} והדרור\mycircle{°} ״לעלמא דיובלא\mycircle{°} עלמא\hebrewmakaf דחירו\mycircle{°}״, המכשירו לחיי שלום, נדבה, גבורה ואהבה גם יחד, עדי\hebrewmakaf עד, המרכיבו על במותי ארץ, המשכלל את חיי החומר וחיי הרוח גם יחד, המאדיר את חיי הזמן\mycircle{°} עם חיי הנצח\mycircle{°} בחוברת הוד\mycircle{°} גדולה\mycircle{°} ועז\mycircle{°} תפארת\mycircle{°}, ע״י האורה האלהית הטהורה\mycircle{°} והברורה המאירה בהרחבתה ורוממות תעופתה כל מחשכים. זאת הנשמה הקדושה הגנוזה בטבע וסגולת\mycircle{°} העם כולו, המוטבעת בסגולת חייו ועצמות הוייתו, המוכרחת לצאת, להתעופף\mycircle{°} ולהתעלות\mycircle{°}, להתרחב, להרבות פארות וענפים עד אין סוף, בשכל, בחיים, בסדרי הבית, המשפחה, הממלכה והלאום, בתכונות הנפש לכל גלויי החיים שבהם לכל רחב תנאיהם והיקפיהם }\מקור{[עפ״י ע״ר ב רסג]}\צהגדרה{. }

\ערך{נשמת ישראל }\הגדרה{- שורש\mycircle{°} התגלות\mycircle{°} האלהות\mycircle{°} והרצון של אחדות\mycircle{°} ההויה בעולם, בהיקף כללותו היותר עליונה }\מקור{[ע״ט יא]}\צהגדרה{.}

\משנה{נשמתן של ישראל מצד הפנימיות\mycircle{°}}\הגדרה{ - שורש\hebrewmakaf התורה\mycircle{°} ופנימיותה\mycircle{°} }\מקור{[קובץ ח קנז]}\צהגדרה{. }

\הגדרה{מקורה של תורה\mycircle{°} }\מקור{[א״ק ג קלח]}\צהגדרה{. }

\ערך{נשמת ישראל\mycircle{°}}\הגדרה{ - }\משנה{מגמתה }\הגדרה{- אחדות כל היקום, בהכרה וברגש, בהלך חיים ובנטיות הנפש האנושית. נטית אחדות העולמים\mycircle{°} זה עם זה, ע״י משך שטף אור\hebrewmakaf האלהות, ע״י התאחדות באור\hebrewmakaf האלהי\mycircle{°} במקורו, עם תשוקת שאיפת כל הטוב\mycircle{°} הזמני והנצחי\mycircle{°} האידיאלי\mycircle{°} }\מקור{[עפ״י ע״ט יא]}\צהגדרה{.}

\משנה{הנשמה הישראלית }\הגדרה{- }\משנה{הודה}\הגדרה{ - תפארת\mycircle{°} העולם ופאר\mycircle{°} הדר\mycircle{°} כל עמי התבל}\צהגדרה{ }\מקור{[אג׳ ג מה]}\צהגדרה{. }

\משנה{המגמה היסודית של נשמת ישראל }\הגדרה{- לחיות חיים אלהיים במובן החברותי, כמו שהחיים הללו הנם מצויים למעלה מן החברה, וכמו שהם נסקרים במשאת נפשם של יחידי\hebrewmakaf סגולה\mycircle{°} המקודשים }\מקור{[א״ק ב תקסב]}\צהגדרה{.}

\ערך{נשמת הכלל\mycircle{°}}\הגדרה{ - צורת\mycircle{°} כנס״י\mycircle{°}. דמות\hebrewmakaf דיוקנו\hebrewmakaf של\hebrewmakaf יעקב\mycircle{°} }\מקור{[א״ק ב רפט (ע״ט ג)]}\צהגדרה{.}

\הגדרה{ע׳ במדור מונחי קבלה ונסתר, מלכות. ע״ע שכינה. ע״ע נשמת כנסת ישראל. ר׳ במדור פסוקים ובטויי חז״ל, נצח ישראל.}

\paragraphs

\ערך{נשמת האמונה}\הגדרה{ - ע״ע אמונה, נשמת האמונה.}

\paragraphs

\ערך{נשמת ההויה }\הגדרה{- }\משנה{הנשמה האלהית העצמית שבהויה המחיה אותה }\הגדרה{- העילוי\mycircle{°} התמידי של ההויה, שהוא יסודה האלהי\mycircle{°}, הקורא אותה להמצא ולהשתכלל }\מקור{[א״ק ב תקלב]}\צהגדרה{.}

\הגדרה{ע״ע נשמת העולם.}

\paragraphs

\ערך{נשמת הכלל }\הגדרה{- ע״ע נשמת האומה.}

\paragraphs

\ערך{נשמת העולם }\הגדרה{- אור\hebrewmakaf החיים\mycircle{°} של השכינה\hebrewmakaf האלהית\mycircle{°} }\מקור{[א׳ צ]}\צהגדרה{. }

\הגדרה{הנשמה האלהית, אצילות השכינה, מלכות שמים, העשירה המתפשטת }\מקור{[עפ״י א׳ צח]}\צהגדרה{.}

\הגדרה{כנסת\hebrewmakaf ישראל\mycircle{°}, נשמת העמים כולם, הודם\mycircle{°} תפארתם\mycircle{°} וברכתם\mycircle{°} }\מקור{[קובץ ז קסט]}\צהגדרה{.}

\צהגדרה{פנימיות\mycircle{°} העולם ואצילותו\mycircle{°} }\צמקור{[שי׳ ד סדרה ב - תשל״ג}\צהגדרה{ 6}\צמקור{].}

\משנה{נשמת העולמים\mycircle{°}}\הגדרה{ - חֵי\hebrewmakaf העולמים\mycircle{°} }\מקור{[א״ק ב שמז]}\צהגדרה{.}

\הגדרה{אור\hebrewmakaf ד׳\mycircle{°} }\מקור{[קבצ׳ ב קנה]}\צהגדרה{.}

\הגדרה{אור ד׳ בעולמו, אור\hebrewmakaf השכינה, הוד\mycircle{°} האידיאליות\mycircle{°} האלהית\mycircle{°} החיה בכל }\מקור{[א״ק ב שסח]}\צהגדרה{.}

\משנה{נשמת חֵי העולמים}\הגדרה{ - נשמת כל היקום, כל היש }\מקור{[קובץ ו סד]}\צהגדרה{.}

\משנה{נשמת כל האצילית }\הגדרה{- אור חֵי\hebrewmakaf העולמים, הטוב\hebrewmakaf הכללי\mycircle{°}, הטוב האלהי השורה בעולמות\mycircle{°} כולם }\מקור{[עפ״י א״ש פרק ב]}\צהגדרה{.}

\משנה{הנשמה הכללית של כל נשמות החיים }\הגדרה{- האור\mycircle{°} המחיה את כל העולמים, והתכונה המקימת את כל בניניהם החמריים\mycircle{°} והרוחניים\mycircle{°}, וכל כליהם, והתלבשויותיהם המפורטות }\מקור{[ע״ר א ח]}\צהגדרה{.}

\הגדרה{ע״ע נשמת ההויה.}

\paragraphs

\ערך{נשמת ישראל }\הגדרה{- ע״ע נשמת האומה.}

\paragraphs

\ערך{נשמת כנסת\hebrewmakaf ישראל\mycircle{°}}\הגדרה{ - הצדק\hebrewmakaf המוחלט\mycircle{°} }\מקור{[א״ש ד ז]}\צהגדרה{.}

\paragraphs

\ערך{נשק }\הגדרה{- ע״ע זין.}\mylettertitle{ס}

\ערך{סבלנות }\הגדרה{- מדת החסד\mycircle{°} והענווה\mycircle{°}, הממתקת\mycircle{°} את כל הדינים\mycircle{°} ובונה את העולם בשכלולו }\מקור{[א׳ נג]}\צהגדרה{. }

\paragraphs

\ערך{סגולה }\הגדרה{- האופי הטבעי }\צהגדרה{[עפ״י קובץ ה קמח}\הגדרה{]. }

\הגדרה{כח נפשי אופייני }\מקור{[רצי״ה א״ש ב הערה 2]}\צהגדרה{. }

\צהגדרה{אופי, תכונה, כח פנימי }\צמקור{[שי׳ סגולת ישראל 115].}

\paragraphs

\ערך{סגולה }\הגדרה{- כח קדוש\mycircle{°} פנימי\mycircle{°} מונח בטבע הנפש ברצון ד׳, כמו טבע כל דבר מהמציאות, שאי\hebrewmakaf אפשר לו להשתנות כלל }\מקור{[אג׳ ב קפו]}\צהגדרה{. }

\paragraphs

\משנה{״סגולה״}\myfootnote{ של״ה, בית אחרון, בהגהה ״נקרא סגולה כי מסוגל כן אבל נעלם הסבה״.\label{1}}\הגדרה{ }\צהגדרה{- ביטוי מיוחד המציין תופעה שהיא מעל כל פירוש }\צמקור{[שי׳, חוב׳ 48, שיר הכבוד, 8]. }

\paragraphs

\ערך{סגולה הצפונה העליונה }\הגדרה{- }\משנה{(לעומת בחירה\mycircle{°} בישראל) }\הגדרה{- משגב העצם של הקודש\mycircle{°} בנקודתו העליונה, <שבא בתאר ישראל\mycircle{°}, סגולה זו עומדת למעלה מכל בחירה> }\מקור{[ע״ר ב פ]}\צהגדרה{. }

\הגדרה{ע״ע סגולת ישראל. ע׳ במדור פסוקים ובטויי חז״ל, עם סגולה. ושם בית יעקב לעומת בני ישראל. ושם ממלכת כהנים וגוי קדוש. ושם בני בכורי לעומת בנים. ע׳ במדור מדתם ועניינם הרוחני של אישי התנ״ך, ישראל, מדת התאר ישראל (לעומת יעקב), ושם, יעקב, מדת התאר יעקב (לעומת ישראל). ע״ע ישראל לעומת ישורון. ע׳ במדור מונחי קבלה ונסתר, קוב״ה דרגא על דרגא סתים וגליא וכו׳. ע׳ בנספחות, מדור מחקרים, ״בחרתי בכם ויחדתי שמי עליכם״ לעומת ״אני בכבודי מתהלך ביניכם״.}

\paragraphs

\ערך{סגולת התורה }\הגדרה{- הסגולה האלהית העליונה, הארת\mycircle{°} המגמה\mycircle{°} האלהית\mycircle{°} בכל יצוריו, בפרטיהם ובכלליהם, המובלטת באותיותיה של תורה\mycircle{°}, במצותיה\mycircle{°}, חוקיה\mycircle{°} ומשפטיה\mycircle{°}, וכל התגלותה\mycircle{°} האלהית האין סופית }\מקור{[ע״ר א ס\hebrewmakaf סא]}\צהגדרה{. }

\paragraphs

\ערך{סגולת ישראל}\myfootnote{ \textbf{סגולה }\textbf{-}\textbf{ התפארת }\textbf{האלהית}\textbf{ אשר נתגלתה }\textbf{בכנס״י} - ע׳ פרדס, ערכי הכנויים, ערך סגולה. וכן בקהילות יעקב. וע׳ פתחי שערים, ד, בית נתיבות סוף סי׳ א.\label{2}}\הגדרה{ - האופי המקודש המיוחד, של האומה\mycircle{°} הקדושה\mycircle{°} והנפלאה, אשר אין לו ערך ודוגמא בכל מהות חיים ונפשיות של כל האדם אשר על פני האדמה. התוכן הפנימי\mycircle{°} של הנשמה, במהותו העצמית של מזג החיים, משונה ונפלא הוא בבני עם קדוש זה בפליאת עולם של קדושה אלהית עליונה, שעיקר הודה\mycircle{°} ופארה\mycircle{°} טמון וגנוז הוא במעמקי הפאר החבוי בסתר מהותיותו הרוחנית\mycircle{°}, וזהרורי אורים ממנו בוקעים וזוהרים בכל יפי הצדק\mycircle{°} של המדות והתכונות הקדושות, של גוי קדוש, עם נורא, עם אלהים זה, שאין דוגמא לגדולת תפארתו בכל העמים תחת כל השמים. הסגולה הגנוזה הזאת היא תוצאת שלשלת הקודש של קדושת עולם, אשר התגלתה בזוהר כבודה ע״י תולדתה האלהית בראשית יצירתה להיות לעם, בפליאת עולם של יציאת\hebrewmakaf מצרים\mycircle{°} וגילוי\hebrewmakaf שכינה\mycircle{°} שנתלותה עמה, וכל סדרי ההוד הנפלאים שנמשכו ממנה ועל ידה, שהם הולכים ונמשכים, כנחלים מלאי מי מרומים ונחלי עדני קודש, מאז ועד סוף\hebrewmakaf כל\hebrewmakaf הדורות\mycircle{°}. והטבעיות המיוחדה של ישראל, ההגיון האלהי הטהור, היושב בחדרי הלב של הכנסיה, והטוהר המדותי, היושר\mycircle{°} הנפשי, החקוק בעומק עצמותה, אורם התוכי לא יועם לעולם בפנימיות ערכו, אף אם רבו מאד המכשולים על דרך החיים, אשר יחשיכו את הדרם\mycircle{°} מהופיע בכל הבהקתם בכל ארחות המעשים, הדעות והמדות, בהתגלותם. גאות קודש זאת מורשה היא לקהלת יעקב\mycircle{°}, ואיתנה היא סגולה\mycircle{°} זו, אדירה ואמיצה היא יותר מכל חקות של כל היקום. (כל זאת הוא מיסוד) התפארת\hebrewmakaf האלהית\mycircle{°} אשר נתגלתה בכנסת\hebrewmakaf ישראל\mycircle{°}, במהות נשמתה, ביסוד חייה הכלליים, (ה)חודרת בכל נשמת כל יחיד ויחיד מיחידיה }\מקור{[עפ״י ע״ר א כב\hebrewmakaf ג]}\צהגדרה{. }

\הגדרה{כח אלהי שיש בקרב ישראל }\מקור{[עפ״י ע״א ד ו מו]}\צהגדרה{. }

\הגדרה{סגולתם הפנימית העליונה של ישראל, סגולה אלהית גנוזה שגרמה להופעת\mycircle{°} תורה\hebrewmakaf מן\hebrewmakaf השמים\mycircle{°} עליהם }\מקור{[עפ״י א״ת א ב]}\צהגדרה{.}

\משנה{סגולתה הפנימית של אומת ישראל }\הגדרה{- יסוד מחשבתה ומעמק הגיונה, הטבעי התולדתי מנפש עד בשר, (ש)הוא רק התוכן של המחשבה\hebrewmakaf האלהית\mycircle{°} }\מקור{[עפ״י קובץ ז קע]}\צהגדרה{. }

\משנה{סגולת האומה }\הגדרה{- שכלה הבהיר המלא ברוחה העצמי, המאיר את דרכה, לדעת באפיה הפנימי ידיעה עצמית מקורית את ד׳\mycircle{°} אל אמת, עד כדי נשיאת דגל קודש זה, שהוא תפארת\mycircle{°} כל ההויה כולה, ברמה לעיני כל אפסי ארץ }\מקור{[ע״ר א עה]}\צהגדרה{.}

\הגדרה{הטוב\hebrewmakaf האלהי\mycircle{°} הטבוע בקרבה, סדר\hebrewmakaf העולם, החיים הישרים\mycircle{°} והטובים\mycircle{°} המתאימים אל הצדק והיושר, השקט והשלוה, החן\mycircle{°} והאומץ הממולאים בהסתכלות אלהית מקפת, כפי מה שהיא נמצאת בנשמת האומה }\מקור{[ש״ה, הקדמה, ז]}\צהגדרה{. }

\הגדרה{השאיפה להאידיאלים\hebrewmakaf האלהיים\mycircle{°}, שנתגלתה באומה שלמה המוכשרת לה, ישראל\mycircle{°}, בטבע הנשמה הלאומית הכללית }\מקור{[ע״ה קלו]}\צהגדרה{.}

\הגדרה{סגולה\mycircle{°} נטועה וקבועה בנפש הלאומית בעצם טבעה. התכונה האלהית\mycircle{°} הפנימית של אהבת היושר\mycircle{°} והצדק\mycircle{°} והשאיפה האמיצה להאידיאלים\mycircle{°} האלהיים הללו ברום עזם, המלאה אהבת האמת\mycircle{°} השלמה והמאירה המתיחדת עם אור השלום\mycircle{°} וחפץ התעלותו לעד, בחיי האומה כולה, וביחוד בחייה הלאומיים }\מקור{[עפ״י ע״ה קמח]}\צהגדרה{. }

\משנה{סגולתה הפנימית של האומה }\הגדרה{- קדושת ישראל העליונה\mycircle{°} הפנימית, שהיא באה מכח שורש הנשמה בקדושתה, שבשביל כך היסוד של האצילות\mycircle{°} הנפשית אינו זז גם מהפרטים הכושלים שבהם, מפני שהם קשורים בקרב נפשם עם קדושת\hebrewmakaf השם\mycircle{°} יתברך העליונה }\מקור{[עפ״י ע״ר ב פד\hebrewmakaf ה]}\צהגדרה{. }

\משנה{הסגולה הפנימית של ישראל }\הגדרה{- אהבת\mycircle{°} השם יתברך, ואהבת תורתו ומצותיו באהבה של נטיה טבעית }\מקור{[ע״ר ב רפט]}\צהגדרה{. }

\הגדרה{יראת\hebrewmakaf ד׳\mycircle{°} }\צהגדרה{[מ״ר }\צמקור{402}\צהגדרה{ (קובץ ח קצח)]. }

\הגדרה{החוש הקדוש\mycircle{°} של האמונה\mycircle{°} בטהרתה\mycircle{°}, שהוא בישראל ענין סגולי\mycircle{°}, לא מצד בחירת נפשם בפרט אלא מצד מחצב הקדושה וסגולת ירושת אבות\mycircle{°} שלהם, שבה אין שום הפרעה יכולה לשלוט וגם במורד היותר גדול של התוארים החצונים של החיים המעשיים ההרגשיים וההכריים, חיה היא סגולת הקודש של נחלת ד׳ זאת, אמונת\hebrewmakaf אומן\mycircle{°}, בכל זיקוק סגולתה. האור הגנוז של שלמות חיי האמונה }\מקור{[עפ״י ע״א ד יא, יג יד]}\צהגדרה{. }

\הגדרה{טבע הקדושה שבנשמת\hebrewmakaf ישראל\mycircle{°} מירושת אבות }\מקור{[אג׳ ב קפו]}\צהגדרה{. }

\הגדרה{העצמיות של הטבע הנפשי שהיא בישראל מורשת עולמים מקדושת\hebrewmakaf האבות\mycircle{°}, הנשמרת כל כך באופן אופיי בעם ד׳ בכללו }\מקור{[ע״א ד ט קלג]}\צהגדרה{. }

\הגדרה{המדה הישראלית שהוטבעו בה האבות במעלת רוחם, ושבניהם באו אליה בתעלומה פנימית כשנזדככו בכור הברזל, של הידיעה הפנימית המרגשת את אי האפשרות של היפוך הסידור העליון המקיף והחודר של האורה\hebrewmakaf האלהית\mycircle{°}, ההולכת וזורמת בעולמים ומגיעה עד חידור התוכי של החיים, שהתורה\mycircle{°} כולה היא מסקנתה האחרונה של תוצאתה }\מקור{[עפ״י א׳ קמב\hebrewmakaf ג וההגדרות הקודמות]}\צהגדרה{.}

\משנה{יסוד סגולת האומה הפנימי }\הגדרה{- החפץ האדיר של הצדק האלהי והתפתחותו, שבעומק נשמת\hebrewmakaf האומה\mycircle{°}, הבא לה בירושת מורשת אבות - הסתוריה גזעית }\מקור{[עפ״י ע״ה קמח\hebrewmakaf ט]}\צהגדרה{. }

\הגדרה{הידיעה ״שאתה הוא ראשון ואתה הוא אחרון, ומבלעדיך אין אלהים״ ושרש האמונה\mycircle{°} בטהרתה, המחבר את הקדושה\mycircle{°} הרוממה, של אור\hebrewmakaf אין\hebrewmakaf סוף\mycircle{°}, עם הקדושה החודרת בכל העולמים ובכל היצורים כולם }\מקור{[עפ״י ע״ר א קיד]}\צהגדרה{.}

\מעוין{◊}\הגדרה{ ההערכה האלהית העליונה, המלאה בסגולת החיים של צביון האומה וסגולתה הפנימית, נשאבת לא ממקור איזו תכונה מוגבלה, ואיזו מורשת אבות מוקצבת, כ״א ממקור המית הנשמה במקורה, לחמדת פאר מקור התפארת של יסוד חיי כל, חיי עולמי עד, מקור כל טוב, קץ כל נועם, אחרית וראשית כל מחמד }\מקור{[עפ״י ע״א ד ט קמג]}\צהגדרה{. }

\מעוין{◊}\הגדרה{ הסגולה  הפנימית שבכלל ישראל אינה נמדדת לפי ערך של כל דור ודור בפרטיות, כ״א היא סוקרת בסקירה אחת כל הדורות מראש ועד סוף }\מקור{[שם ג ב ז (ע״ר א תלב)]}\צהגדרה{. }

\משנה{סגולת ישראל }\צהגדרה{- אופי, תכונה, כח פנימי. אור הלבבות והנשמות שמתגלה בהשראת\hebrewmakaf השכינה\mycircle{°} בישראל, שאינו מצדנו המוסרי. ״שההתחלה ממנו }\צמקור{[- ד׳] }\צהגדרה{ולא ממנו }\צמקור{[- אנחנו]}\צהגדרה{״. ערך היצירה, ערך הבריאה, שכך נוצרנו }\צמקור{[עפ״י שי׳ סגולת ישראל 115-116].}

\צהגדרה{סגולת\mycircle{°} אורגניות מהותו (של ישראל), קדושתו האלהית הנבחרת, ישראליותו, המתגלית עתה בצורת האומה הקמה לתחיה. מקום\mycircle{°} אחדותן הטבעית של ה״קוסמופוליטיות״ והלאומיות והאישיות היחידית בעמק\hebrewmakaf פנימיותה, ישראל ותורות ד׳ אשר לו תורת\hebrewmakaf חייו }\צמקור{[צ״צ א ק].}

\הגדרה{ע״ע צדק. ע״ע ישראל. ע״ע נשמת האומה. ע׳ במדור פסוקים ובטויי חז״ל, עם סגולה. ע׳ במדור מדתם ועניינם הרוחני של אישי התנ״ך, יצחק, מדתו של יצחק. ע׳ בנספחות, מדור מחקרים, סגולת ישראל. }

\paragraphs

\משנה{סגירו}\הגדרה{ - }\משנה{(צרעת בלשון הזוהר)}\הגדרה{ - ע״ע צרעת.}

\paragraphs

\ערך{סדר }\הגדרה{- הנוי\mycircle{°}, ההתאמה של החלקים זה בצד זה, זה אחר זה, זה לעומת זה }\מקור{[ע״ר א קנב]}\צהגדרה{. }

\paragraphs

\ערך{סופנת }\הגדרה{- גונזת, כורכת, כוללת }\צהגדרה{[רצי״ה א״ש ט הערה }\הגדרה{17}\צהגדרה{]. }

\paragraphs

\ערך{סיבה }\הגדרה{- }\משנה{הסיבות }\הגדרה{- הסדר עילה ועלול בקישור הנמצאים וסידורם כולם }\מקור{[עפ״י מ״ש שנד (ה׳ שטז\hebrewmakaf שיז)]}\צהגדרה{. }

\משנה{הסיבות המקוריות }\הגדרה{- המקוריות של מהלכי החיים והרשמים העמוקים, הרחבים והרמים\mycircle{°}, הגורמים למחשבות ולנטיות (של החיים הרוחניים\mycircle{°} שאנו רואים בעולם) שתצאנה אל הפועל, ושתחזקנה את מעמדן }\מקור{[עפ״י א״ק ב שנז]}\צהגדרה{. }

\הגדרה{ע׳ בהגדרות מבוא למדור מלאכים ושדים, נשמות, מלאכים, אורות, נצוצות, או כחות שכליות, סבות, עלולים, וכיו״ב.}

\paragraphs

\ערך{סיבת הכל }\הגדרה{- מחולל כל ומחיה את כל }\מקור{[א״ק ב תמב]}\צהגדרה{. }

\paragraphs

\ערך{סיגים רוחניים }\הגדרה{- הצללים, מחשבות השקר והחולשה, החנופה והתאוה }\מקור{[א״ק ג רמח]}\צהגדרה{. }

\paragraphs

\תערך{סימנים מקריים - }\תהגדרה{מסמנים את החצוניות של הדבר המסומן על ידם, שבאה במקרה ושעלולה להתחלף }\תמקור{[מ״ר 461]. }

\paragraphs

\תערך{סימנים עצמיים }\תהגדרה{- מסמנים את מהות הדבר המסומן על ידם ותכונתו הפנימית\mycircle{°} }\תמקור{[מ״ר 461]. }

\paragraphs

\ערך{סלה }\הגדרה{- המשכת השפעה\mycircle{°} שתהיה מתמשכת ומנצחת בקרבנו עדי\hebrewmakaf עד\mycircle{°} }\מקור{[עפ״י ע״ר א יא]}\צהגדרה{. }

\הגדרה{הנצח }\מקור{[עפ״י ר״מ קעח]}\צהגדרה{.}

\צהגדרה{לאין הפסק}\myfootnote{ ערובין נד.\label{3}}\צמקור{ [ל״י ב (מהדורת בית אל תשס״ג) של-שלא].}

\הגדרה{ע׳ במדור פסוקים ובטויי חז״ל, אֹמֶר סלה. }

\paragraphs

\ערך{סליחה }\הגדרה{- }\משנה{(סליחת ד׳) }\הגדרה{- תוכן תקון הבא מתוך מקור הרחמים\mycircle{°}, המאיר על מחשכי הנשמה\mycircle{°}, להעביר את יסוד הרשע העמוק (של הפשע\mycircle{°}) ממקורו, ע״י התגלות עוצם הטוהר\mycircle{°}, הרוממות\mycircle{°} והגודל\mycircle{°}, של אור\hebrewmakaf ד׳\mycircle{°} עליון\mycircle{°} }\מקור{[עפ״י ע״ר א קכז]}\צהגדרה{. }

\הגדרה{ע״ע מחילה. ע״ע כפרה. ר׳ מחילה סליחה וכפרה. ובנספחות, מדור מחקרים, מחילה סליחה וכפרה. }

\paragraphs

\ערך{סמבולי }\הגדרה{- מזכיר ומחקה איזה ענין על הדמיון לבד }\מקור{[עפ״י קובץ א שעו]}\צהגדרה{.}

\הגדרה{מזכיר על ידו את המחשבות }\מקור{[עפ״י א״ק ג צו]}\צהגדרה{.}

\הגדרה{מעורר מחשבה ורעיון }\צהגדרה{[עפ״י קובץ ה רסט].}

\משנה{דבר סימבולי - }\הגדרה{(דבר) שעל ידו האדם מתעורר אל הדברים המחשביים. תוכן מעשי המשמש ערך סימני, בתור מעורר לכחות המחשביים. ענין להכניס על ידו רעיונות כבודים בנפש }\מקור{[עפ״י ע״א ד ט עט]}\צהגדרה{.}

\paragraphs

\ערך{סנה }\הגדרה{- מקור הוראת ההכנעה וענות\mycircle{°} הלב }\מקור{[ע״א ד ו פד]}\צהגדרה{. }

\paragraphs

\ערך{״ספור״ }\הגדרה{- }\מעוין{◊}\הגדרה{ כולל ענין שכלי שמתלבש בדברים גשמיים }\מקור{[ע״א א ג כא]}\צהגדרה{. }

\paragraphs

\ערך{ספור תהילת\hebrewmakaf ד׳\mycircle{°}}\myfootnote{ עפ״י ישעיה מג כא.\label{4}}\הגדרה{ - }\משנה{כשרון ספור תהילת ד׳ אשר לישראל\mycircle{°} }\הגדרה{- החוש הקדוש והבריא הנאדר בקודש של הסתכלות מלאה במציאות ושרשיה }\מקור{[קובץ ה מא]}\צהגדרה{. }

\paragraphs

\ערך{ספיגה }\הגדרה{- הישפעות וקבלה, התפעלות }\מקור{[עפ״י ע״א ד ט צב]}\צהגדרה{. }

\paragraphs

\ערך{ספק }\הגדרה{- }\משנה{(סיבת התפשטותו) }\הגדרה{- }\מעוין{◊}\הגדרה{ צללי הספקות מתפשטים לפי אותה המדה שהאורה\hebrewmakaf האלהית\mycircle{°} אינה תפוסה בפנימיות\mycircle{°} מהות החיים }\צהגדרה{[א״ק א רט (א״א }\צמקור{39}\צהגדרה{)]. }

\paragraphs

\ערך{ספרות }\הגדרה{- האמנות המחשבתית ובטויה }\מקור{[עפ״י פנק׳ ב רכ (חד׳ עז)]}\צהגדרה{.}

\מעוין{◊ }\הגדרה{ראי החיים }\צהגדרה{[מ״ר }\צמקור{505}\צהגדרה{].}

\ערך{ספרות }\הגדרה{- }\משנה{כחה הגדול והעדין של הספרות }\הגדרה{- הרמת היסוד הרוחני\mycircle{°} בעולם בכל עילויו\mycircle{°} }\מקור{[א׳ פב]}\צהגדרה{. }

\paragraphs

\ערך{ספרים }\הגדרה{- מקורי המחשבה של החרות\mycircle{°} העליונה}\צהגדרה{ }\מקור{[פנק׳ ב רו]}\צהגדרה{.}

\מעוין{◊ }\משנה{הספרים}\הגדרה{ אוצרים את כל המילוי של הרכוש המדעי, המוסרי\mycircle{°}, וכל האורה\mycircle{°} שכל התעוררות חיים פועלים מתגלית בהם ועל ידם }\מקור{[ר״מ קעד]}\צהגדרה{. }

\צהגדרה{אנו משוטטים בעולם  הרוחני\mycircle{°}, מטיילים\mycircle{°} בפרדס האצילי, ונשמתנו מופעה בהופעת\mycircle{°} אורי קודש, ומתאוים אנו לקלוט את שברירי האורים, לצירם\mycircle{°} ולסמנם. והננו לואים מרוב טובה, ומבקשים מקום מנוח, אשר יגדיר את מאוינו, ויציג לפנינו את עדרי הרעיונות במספר\mycircle{°} ותוכן, בהשתרגות אורגנית\mycircle{°} והסתמלות ממודדה. ומתוך איווי חשוב זה, הננו עטים אל }\צהגדרהמודגשת{הספרים}\צהגדרה{, ונפשנו מתחממת לאורם, ורוחנו מתקרר בנחת הרוח של ההצטיירות הגונית\mycircle{°}, שהם מציירים לנו במלא אוצרות המפיקים זיו\mycircle{°} ונוגה\mycircle{°}, עושר גדול של תפארת\mycircle{°} זיו\hebrewmakaf החיים\mycircle{°} העליונים, מדושני העונג\mycircle{°} ורווי השמחות\mycircle{°} }\צמקור{[קובץ ח קלח]. }

\משנה{עיקר מטרתם של כל הספרים }\הגדרה{- לעורר את האדם לשום את הגיונו שאיפתו ומחשבתו אל המקום\mycircle{°} הרוחני\mycircle{°} העליון שממנו הם נשאבים }\מקור{[פנק׳ ב רטו]}\צהגדרה{.}

\paragraphs

\ערך{סקירה חצונית }\הגדרה{- }\משנה{(לעומת סקירה\hebrewmakaf פנימית\mycircle{°}) }\הגדרה{- לא בדרך האור\mycircle{°} של החיים העצמיים, כ״א בדרך הכלי\mycircle{°} המקבל אל תוכו את האור }\מקור{[ע״ר א כו]}\צהגדרה{. }

\paragraphs

\ערך{סקירה פנימית }\הגדרה{- }\משנה{(לעומת סקירה\hebrewmakaf חצונית\mycircle{°}) }\הגדרה{- דרך האור\mycircle{°} של החיים העצמיים }\מקור{[ע״ר א כו]}\צהגדרה{. }

\paragraphs

\ערך{סרחון }\הגדרה{- }\משנה{(ענינו) }\הגדרה{- דבר שמתפשט מקלקול הרוח\mycircle{°} לקלקול הגוף }\מקור{[ע״ר ב רו]}\צהגדרה{. }

\paragraphs

\ערך{״סתום״ }\הגדרה{- רז העליון, אשר מצד עוצם מעלתו לא ניתן להתגלות לכל יצור. המציאות\hebrewmakaf העליונה\mycircle{°}, החיים האציליים\mycircle{°} וההופעות\mycircle{°} האלהיות\mycircle{°} אשר ממפלאות תמים דעים. המציאות הטמירה העליונה המתבודדת בדומיה הקדושה העליונה. מקור החזון הנעלם, שהוא נועד להיות עיקר יסוד ההויה ועילויה האחריתי, המהות של אור התורה\mycircle{°} השרשי, האור המחולל את הנסים\mycircle{°} בכל הופעותיהם העומד בתור כח שרשי, למעלה מכל ההופעות שכל ארחות המציאות בכל מדרגותיה מסתעפים מהן. האורה הסתומה שהיא יסודה של תורה }\מקור{[עפ״י ע״א ד יב, ב ג (מא״ה ב קל)]}\צהגדרה{. }

\הגדרה{נעלם }\מקור{[שם שם שם יח]}\צהגדרה{. }

\משנה{הסתימה }\הגדרה{- הגניזה. החותמת של כל היסוד המחשבתי }\מקור{[עפ״י ר״מ צד]}\צהגדרה{. }

\הגדרה{ע׳ במדור פסוקים ובטויי חז״ל, מאמר סתום. ע׳ במדור מונחי קבלה ונסתר, ״סתים וגליא״. ע״ע גלוי. ע״ע ״פתוח״. ע׳ במדור אותיות, מ״ם סתומה. }

\paragraphs

\ערך{סִתּוּמִים }\הגדרה{- עיכובים לחדירת שפע\mycircle{°} חיים\mycircle{°} ואור\mycircle{°} אמת\mycircle{°} }\צהגדרה{[רצי״ה א״ש יב הערה }\צמקור{6}\צהגדרה{].}\mylettertitle{ע}

\paragraphs

\ערך{עבדות }\הגדרה{- }\משנה{ההכשר לה }\הגדרה{- ההכנה למדות שפלות ותאות גסות, שאינן מתעדנות אפילו ע״י ההרגש של כבוד המדומה }\מקור{[אג׳ א קב]}\צהגדרה{. }

\הגדרה{ע׳ במדור מדרגות והערכות אישיותיות, עבד. ושם, בן חורין.}

\משנה{עבדות}\צהגדרה{ - הגבלת התביעה הפנימית, עצירתה מהופיע בחיים החיצוניים}\צמקור{ [שי׳ מועדים ב 129].}

\ערך{עבדות }\הגדרה{- ההכרח והעדר החופש\mycircle{°} הבחירי\mycircle{°} }\מקור{[א״ק ג לה]}\צהגדרה{. }

\משנה{תכונת העבדות }\הגדרה{- שאין לה חיים עצמיים ורצון מקורי }\מקור{[עפ״י ע״ר א עא]}\צהגדרה{. }

\הגדרה{הנטיה אל רצונות זרים }\מקור{[עפ״י ע״א ג א ד]}\צהגדרה{. }

\הגדרה{ר׳ חרות.}

\paragraphs

\ערך{עבדות }\הגדרה{- }\משנה{(ביחס לד׳) }\הגדרה{- גדלות הרוחב של המפעל המעשי והליכות החיים }\מקור{[ע״ר א קצז]}\צהגדרה{. }

\מעוין{◊ }\ערך{עבדות עליונה\mycircle{°}}\הגדרה{ - <}\צהגדרה{מתעלה מכל תאר של יחס חביב, עולה על תאר של בנים\mycircle{°}, כמדתו של עבד\hebrewmakaf ד׳\mycircle{°} שנקרא בו משה\mycircle{°} רבנו ע״ה}\הגדרה{> באה מתוך עומק ההכרה, שהננו צריכים למלא תמיד את דבר\hebrewmakaf ד׳\mycircle{°} בשכלול עולמו }\מקור{[עפ״י ע״ר א רב]}\צהגדרה{. }

\מעוין{◊ }\משנה{העבדות האלהית של ד׳ אלהי ישראל }\הגדרה{- התכונה העליונה מכל שאיפה שבחירות\mycircle{°} }\מקור{[א״ק ג לה]}\צהגדרה{.}

\paragraphs

\ערך{עבודה }\הגדרה{- ר׳ בנספחות, מדור מחקרים, מלאכה לעומת עבודה.}

\paragraphs

\ערך{עבודה }\הגדרה{- הרצון של האדם, בהיותו מתעלה אל החשק\mycircle{°} האלהי, ובהיותו דורך באלה הדרכים שמביאים אותו לרום מטרתו, <שהתפילה\mycircle{°}, המחוברת עם חיים טהורים\mycircle{°} תוריים ושכליים, היא אחת מהם> }\מקור{[ע״ט ט\hebrewmakaf י]}\צהגדרה{. }

\משנה{העבודה העליונה של האישיות הפרטית }\הגדרה{- עלוי כוחות החיים הטבעיים, הרצוניים, שיהיו ממוזגים בטבעם עם המחשבתיים השכליים. ומזוג זה אינו נעשה בהדר\mycircle{°} השלמתו כי אם לפי אותה המדה שהעריגה האלהית\mycircle{°}, על פי תנאיה ותפקידיה, ממלאה את הנשמה\mycircle{°} }\מקור{[א״ק ג פו]}\צהגדרה{. }

\משנה{העבודה השלמה }\הגדרה{- תיקון כל המעשים והמדות כולם עם הסדר הנאות להנהגה שלמה ונהדרת הרצויה מאת הבורא ב״ה, הכולל שני חלקים: האחד, השגת החכמה וידיעת המדות עניניהן ותכונותיהן וגבולותיהן וכל התפשטות כחותיהן ואיך להשתמש בכל כח ובכל מדה. והשני, להרגיל היטיב את רצונו והכחות הממוסכים בגופו, שיסכימו כולם בלב שלם אל הטוב ההוא }\מקור{[עפ״י מ״א ב א]}\צהגדרה{.}

\משנה{העבודה }\צהגדרה{- }\צמשנה{בהא\hebrewmakaf הידיעה }\צהגדרה{- התמסרות\hebrewmakaf הנפש\mycircle{°} והקרבה המכונת למעלה לכל מדרגותיה }\צמקור{[עפ״י ל״י א (מהדורת בית אל תשס״ב) רנז].}

\משנה{העבודה}\הגדרה{ - השלמת הרגש\mycircle{°} בתכלית רוממותו }\מקור{[ע״א ג ב מ]}\צהגדרה{.}

\ערך{עבודה דמיונית }\הגדרה{- }\משנה{מהותה }\הגדרה{- להמשיך רגשי האדם לענינים גדולים, והגדול מכל, הוא הענין\hebrewmakaf האלהי\mycircle{°} }\מקור{[א״ק ג רג]}\צהגדרה{. }

\ערך{עבודה ממשית }\הגדרה{- }\משנה{מהותה }\הגדרה{- תחית הרצון, הגברת הנשמה האלהית, באדם ובעולם, על ידי מעשים טובים ושכלים אמיתיים וכל דבר קדוש ואמת, על פי הטהרה\mycircle{°} השכלית והופעת\mycircle{°} אור\hebrewmakaf התורה\mycircle{°} }\מקור{[א״ק ג רג]}\צהגדרה{. }

\ערך{עבודה פנימית }\הגדרה{- סידור המחשבות, חיי ההתבוננות, וסידור הרגשות, חיי הפיוט והשירה, ויחוסם של אלה התכונות השונות זה לזה לכוין איך הם מתלכדים ביחד, ופועלים אלה על אלה, ובאיזה אופן הרכבתם עולה יפה, ובאיזה אופן הם צריכים להתחלק כל אחד במערכה לבדו }\מקור{[א״ק ג רד]}\צהגדרה{. }

\paragraphs

\ערך{עבודה האלהית }\הגדרה{- }\משנה{מגמתה }\הגדרה{- העליה העולמית}\צהגדרה{ }\מקור{[עפ״י קובץ א תרטז]}\צהגדרה{.}

\ערך{עבודה האלהית והמעשה הטוב }\הגדרה{- }\משנה{יסודם }\הגדרה{- לגשם בפעל ובחיים את ציורי\mycircle{°} הצדק\mycircle{°} והיושר\mycircle{°} האלהיים היותר רמים וקדושים }\מקור{[ע״ר א עניני תפילה כב]}\צהגדרה{.}

\paragraphs

\ערך{עבודה }\הגדרה{- }\משנה{המטרה העליונה של העבודה (בבית\hebrewmakaf המקדש\mycircle{°})}\הגדרה{ - להעלות את כל מה שירד, לעדן את כל מה שנעשה גס\mycircle{°} ומוגשם להאיר את אור חסד\hebrewmakaf עליון\mycircle{°}, ההולך למשרים ומפלש נתיבו להאיר ולהחיות, מראש עולמי אורה, עד תחתית כל תהומות מאפליה }\מקור{[עפ״י ע״ר א קנו]}\צהגדרה{.}

\paragraphs

\ערך{עבודה אמונית }\הגדרה{- ע״ע אמונה, עבודת האמונה. }

\paragraphs

\ערך{עבודה העליונה}\הגדרה{ - ע׳ בנספחות, מדור מחקרים.}

\paragraphs

\ערך{עבודה מאהבה }\הגדרה{- }\משנה{עבודת ד׳ מאהבה ותלמוד תורה\hebrewmakaf לשמה\mycircle{°}}\הגדרה{ - הקדמת הדרכה תמידית של כשרון מעשים טבעיים, נימוסיים, שכליים, מוסריים, אלקיים, כהכשרה עצמית לתעודת השלמות של מצב הנפש בשמחת דעת\hebrewmakaf ד׳\mycircle{°} הטהורה\mycircle{°}, כדי שילכו ההשקפות בדרך ישרה, בדעת\hebrewmakaf אלהים\mycircle{°}, ושלא יכבד זוהר האמת על הנפש }\מקור{[עפ״י פנק׳ א נב]}\צהגדרה{.}

\משנה{עבודה מאהבה }\הגדרה{- לתועלת כל המציאות ולא תועלתו העצמית. בלתי פנותו אפילו לטובה הנפשית של עצמו כ״א מילוי רצונו של מקום ע״י טובת המציאות }\מקור{[פנק׳ ה קנא, קנב-קנג]}\צהגדרה{. }

\משנה{כח העבודה מאהבה}\הגדרה{ - ע״ע אהבה, עבודת אהבה. }

\paragraphs

\ערך{עבודת אלהים\mycircle{°}}\הגדרה{ - העבוד}\myfootnote{ \textbf{עבודת }\textbf{אלהים}\textbf{, }\textbf{העבוד}\textbf{ של האידיאלים} - תורה אור לרש״ז, שמות דף נא: ד״ה וזהו קול דודי ״ונק׳ עבודה מלשון עורות עבודים שצריך לעבד את המדות שיהיו מתוקנים לה׳ לבדו״. ליקוטי באורים לר׳ הלל מפריטש לקונטרס ההתפעלות דף כ. פרק ב סקכ״ה ״ועוד זאת נקראת עבודה מלשון עורות עבודים מצד שעובד ומתקן דבר מה״.\label{1}}\הגדרה{ של האידיאלים\hebrewmakaf האלהיים\mycircle{°}, לעבדם, לשכללם, להשתדל לשגבם, להאדירם בעם, באדם, ובעולם }\מקור{[ע״ה קמה]}\צהגדרה{. }

\הגדרה{שכלול האדם את עצמו, את הוייתו, ואת כל ההויה המתיחשת אליו יחש קרוב או רחוק, במה שהוא מעבד את רצונו, בהטביעו בקרבו הטבעה עליונה עדינה של הטוב\hebrewmakaf האלהי\mycircle{°} }\מקור{[עפ״י קובץ א קעג]}\צהגדרה{. }

\הגדרה{הכשרת האדם להיות טוב\mycircle{°} לכל }\מקור{[קבצ׳ ב קע (פ״א קנג)]}\צהגדרה{.}

\הגדרה{אהבת עולמים להאידיאלים האלהיים, טפוחם, רבוים והתעלות\mycircle{°} בהם ועל ידם }\מקור{[עפ״י ע״ה קמז]}\צהגדרה{. }

\הגדרה{העלאת כל המעשים וכל ערכי\mycircle{°} החיים כולם למעלה רוממה\mycircle{°}, התעלות אל החפץ\hebrewmakaf האלהי\mycircle{°} העליון\mycircle{°} }\מקור{[עפ״י ע״ר א כז]}\צהגדרה{.}

\paragraphs

\ערך{עבודת עבד }\הגדרה{- }\משנה{(בעבדות\hebrewmakaf עליונה\mycircle{°} את ד׳) }\הגדרה{- הפעולה להכין את עצמו ואת חלקו במציאות הכללית לאופן נעלה המתעלה יותר מציוריו היותר נעלים, בשאיפה למה שהוא נעלה מציורו\mycircle{°}, הצפוי לאדון\hebrewmakaf כל\mycircle{°}. שחפצו נעוץ רק בחפץ צור\hebrewmakaf העולמים\mycircle{°} שלפניו צפוי ערך ההתעלות באין תכלית }\מקור{[עפ״י ע״א ג ב לו]}\צהגדרה{.}

\משנה{עבודת ד׳ הטהורה }\הגדרה{- לעבוד ולפעול כפי חק היושר הנמשך מצד הצפיה\hebrewmakaf האלהית\mycircle{°}, שהוא עומק אמיתת הצדק\mycircle{°} והמשרים\mycircle{°} }\מקור{[ל״ה 110]}\צהגדרה{.}

\משנה{עבודת ה׳ }\הגדרה{- להמציא בעולם אותה התכונה השלמה, שהיא פעולה יקרה כזאת, שיקרותה גלויה היא רק לד׳\mycircle{°} לבדו. ובשביל המצאתה של תכונה זו וכל אגפיה, להמסר לכל הטוב\mycircle{°} והיושר\mycircle{°}, ולכל מעשה התורה\mycircle{°} והעבודה\mycircle{°}, וכל פרטי פרטיהם }\מקור{[עפ״י א״ק ג שמא]}\צהגדרה{. }

\הגדרה{עבודת קודש עליונה, למעלה למעלה מכל חק של טבע. להכניס בחשק הטבע כולו, את האור החפצי של תשוקת קודש\hebrewmakaf הקדשים\mycircle{°}, של האידיאליות\hebrewmakaf האלהית\mycircle{°} }\מקור{[עפ״י א״ק ג רכח (קובץ ח קכז)]}\צהגדרה{. }

\משנה{עבודת ד׳ }\הגדרה{- שכלול עולמים\mycircle{°}, עשית\hebrewmakaf רצונו\mycircle{°} של מקום\mycircle{°} }\מקור{[עפ״י א״ק א מב (קובץ ה יד)]}\צהגדרה{.}

\משנה{עבודת ד׳, תכונתה}\הגדרה{ - שקיקה לאידיאל של ידיעה ולידי שקיקה שאין אנו יכולים גם להגות בה מפני גדלה }\מקור{[א״ק ב תקנח]}\צהגדרה{.}

\משנה{עבודת השי״ת, ענינה}\הגדרה{ - ליחד כל הפעולות לשמו\mycircle{°}}\צהגדרה{ }\מקור{[מא״ה ג קסג]}\צהגדרה{.}

\הגדרה{ ע׳ במדור פסוקים ובטויי חז״ל, לעשות רצון\hebrewmakaf ד׳ בעולמו. ע״ע רצון ד׳, מה שנוכל להבין בו. ע׳ במדור מונחי קבלה ונסתר, תקון, יסוד תקון העבודה.}

\paragraphs

\ערך{עבודת ד׳ }\הגדרה{-}\משנה{ (המצויה גם בעמים) }\הגדרה{- מערכת מעשים המגלה את ההכרה, המוטבעת וקבועה בנפש האלהית שבאדם, שמתוכה הוא מרגיש שהוא צריך לתת כבוד\mycircle{°} ליוצרו, להודות\mycircle{°} לו על טובו\mycircle{°} ולפאר שם\mycircle{°} תפארתו\mycircle{°}. מעשים דתיים\mycircle{°}, מוגדרים ומפורטים, ומחוזקים בקביעות צבורית שוה, המוציאים אל הפועל את ההתגלות של כבוד\hebrewmakaf ד׳\mycircle{°} יתעלה }\מקור{[עפ״י א׳ קסז]}\צהגדרה{.}

\הגדרה{עשיית מצותיו\mycircle{°} על ידי החרדות אל דברו\mycircle{°} של ד׳\mycircle{°}, המגלה את ההרגשה הכללית של הכרת כבוד ד׳ }\מקור{[עפ״י א׳ קסז-קסח]}\צהגדרה{.}

\הגדרה{צורת דת, שיש בה הבעה של קדושה. סדרי עבודה על\hebrewmakaf פי ההרגשה הטהורה של הכרת כבוד ד׳, היוצאת אל הפועל על\hebrewmakaf ידי מעשים כאלה, שלפי מצב הנפש יש בהם יחש וסמך לאותה ההכרה }\מקור{[עפ״י א׳ קסח]}\צהגדרה{.}

\הגדרה{ע״ע דת.}

\paragraphs

\ערך{עבודת ד׳ }\הגדרה{- }\משנה{המיוחדת לישראל }\הגדרה{- העבודה המיוחדת בתעודתה הטובה, (לא רק למטרת עבודת\hebrewmakaf ד׳ כפי שהיא בעמים, אלא גם) לכונן בפועל את חפץ\hebrewmakaf ד׳\mycircle{°} בעולמו, לתקון העולם בפועל להמשיך את גלוי הטוב האלהי הכללי הקים והנצחי, במעטה המעשי של סדרי החיים המוגבלים, להטבת החיים של הפרט ושל הכלל }\מקור{[עפ״י א׳ קסח]}\צהגדרה{.}

\הגדרה{ע׳ במדור פסוקים ובטויי חז״ל, משפטים בל ידעום.}

\paragraphs

\ערך{עבודת ד׳\mycircle{°}, עבודת בנים\mycircle{°}}\הגדרה{ - הוצאת האור הצפון, של מקצת תכנם של האידיאלים\hebrewmakaf האלהיים\mycircle{°} החקוק וקבוע בנפשו של האדם, מן הכח אל הפועל, לקרב יותר ויותר לאותה השלמות הבלתי מוגבלת של האידיאלים האלהיים בצורות החיים עצמם, של האיש, ושל הקבוץ הכללי, של המעשה, ושל החפץ והרעיון\mycircle{°}; זאת היא }\משנה{עבודת ד׳}\הגדרה{ הנאורה, }\משנה{עבודת בנים }\הגדרה{החשים בקרבם יחס פנימי\mycircle{°} אל אביהם\mycircle{°} מחוללם מקור הטוב החיים והאורה\mycircle{°} }\מקור{[עפ״י ע״ה קמה]}\צהגדרה{. }

\משנה{עבודת ד׳}\הגדרה{ - כל פנייה אל ההשלמה של ההכרה, של הטוב, של האומץ, של כל הנשגב. קישור מעשי ואמוני לעומת המגמות\mycircle{°} הטובות הניכרות לנו, <שהכל הוא גילוי אלהות\mycircle{°}}\צהגדרה{> }\מקור{[קבצ׳ ב קכג]}\צהגדרה{.}

\הגדרה{התאמת הוייתנו אל התכונה המגמתית היותר עליונה, שהיא ודאי מצויה, קיימת ועומדת לעד }\מקור{[עפ״י קבצ׳ ב קמ (פנק׳ ד שעז-ח)]}\צהגדרה{.}

\הגדרה{לקיחת כוחות הנפש ושעבודם בחבלי אדם ובעבותות אהבה אל הטוב\mycircle{°} ואל היושר\mycircle{°}, אל מרומי המגמות\mycircle{°} היותר נשאות, שהם הינם החפצים\hebrewmakaf האלהיים }\מקור{[עפ״י ע״ה קיב\hebrewmakaf ג]}\צהגדרה{. }

\הגדרה{העבודה שעל האדם לעשות בתור חלק אחד מכלל המציאות, בהוה וגם בעתיד לתכלית השלמתה. ויתירה מזאת בתור איש יהודי, בתור בן לעמו, עליו החובה לקחת חלק בעבודת עמו כדי להביאו להשלמתו, כדי ללכת בדרך אשר התוה ד׳ לכלל עמו <אשר בהיות כל הפרטים מתאמצים לצאת ידי חובתם, רק אז יוציא הכלל כחו הנשגב אל הפועל> }\מקור{[עפ״י מ״ר 140]}\צהגדרה{. }

\משנה{עבודת אמת }\הגדרה{- עבודת\hebrewmakaf ד׳ הבהירה בעד עמנו וארצנו, בעד אור עולם }\מקור{[אג׳ א קנד]}\צהגדרה{.}

\משנה{עבודת ד׳ התמימה במובנה האמיתי }\הגדרה{- עבודת הכלל\mycircle{°}, להיטיב כפי כחו אל הכלל כולו כפי אשר תשיג ידו}\צהגדרה{ }\מקור{[ע״א ג א ט]}\צהגדרה{.}

\משנה{(עשית) רצון השם יתברך ועבודתו }\הגדרה{- מילוי החובה המוסרית לחיי האומה ועתידותיה, בנצירת תעודה ועשיה וקיום את כל המצוות שבתורה, על פי אשר הורונו חכמנו ז״ל מיסדי התלמוד <הוא בדיוק העמוק של אחרית המלה של האמת - }\משנה{רצון השם יתברך ועבודתו}\הגדרה{, שהוא חפץ בצדק ובמשפט וכל מעגל טוב> }\מקור{[עפ״י ל״ה 90]}\צהגדרה{.}

\ערך{עבודת ה׳ הטהורה }\הגדרה{- }\משנה{יסודה }\הגדרה{- ההטבעה באדם את החפץ הפנימי להיות תמיד הולך ומשתלם ולהכיר באמת שתכלית ההצלחה היא רק בהיות האדם תמיד קשור בחפץ של הוספת שלימות. <כי הלא תכלית החיים היא קרבת\hebrewmakaf אלוהות\mycircle{°} והוא ית׳ אין סוף לשלמותו, ע״כ כל מעלת האדם היא שבכל עת יוסיף מעלה בקרבתו אל השם ית׳ - וזאת השאיפה אין לה תכלית ולעולם לא יוכל האדם לומר בזה די, כי כל מעלה שיעלה בשלמות האמיתית תעוררהו לדעת איך לקנות עוד כמה מעלות הסמוכות לה - ומתוך השאיפה של עבודת\hebrewmakaf ד׳ יכיר שאין תכלית ההצלחה להאדם לומר שהשיג כל השלמות הראויה לו להשיגה כ״א שהוא תמיד שוקק להוסיף שלמות. ע״כ יוסיף באמת העובד האמיתי מעלות יקרות דבר יום ביומו, ובזה האדם מתרומם לתכליתו האמיתית>}\myfootnote{ ע״ע ש״ק, קובץ ח לו. ושם, קובץ ד ס. א״ק ג פט. פנ׳ כא.\label{2}}\צהגדרה{ }\מקור{[ע״א א ה מג]}\צהגדרה{. }

\ערך{עבודת ד׳ וכל מעגל טוב מאהבה }\הגדרה{- ע״ע אהבה, עבודת ד׳ וכל מעגל טוב מאהבה. }

\paragraphs

\ערך{עבודת אלהים אצל כל עם ולשון }\הגדרה{- ע׳ במדור אליליות ודתות, דת, אצל כל עם ולשון (חוץ מישראל). }

\paragraphs

\ערך{עבודת ד׳ בלב ההמון }\הגדרה{- ע׳ במדור אליליות ודתות, דת, אצל כל עם ולשון (חוץ מישראל).}

\paragraphs

\ערך{עבודת ד׳ }\הגדרה{- ע׳ במדור משכן ומקדש, קרבנות. }

\paragraphs

\ערך{עבודת הקודש }\הגדרה{- עבודה שכלית }\מקור{[ע״א ב ט רסח]}\צהגדרה{. }

\paragraphs

\ערך{עבודת חכמי\hebrewmakaf אמת\mycircle{°} }\הגדרה{- לעבד את היסוד האמתי של החלום הנאמן האחד, אשר רק בשגגה יראה כשתי מערכות סותרות ומנגדות זו את זו - הקולטורה העצמית שלנו, שאין לנו בה שותפות של כל גוי בעולם, ותחיתנו הלאומית על אדמת\hebrewmakaf הקודש\mycircle{°}, היעודה להיות גי חזיון עולמים, להראותו ביד רמה וקומה זקופה, בגדולה והוד }\מקור{[עפ״י אג׳ א ריד]}\צהגדרה{.}

\ערך{עבודת עבד את ד׳ }\הגדרה{- הפועל להכין את עצמו ואת חלקו במציאות הכללית לאופן נעלה המתעלה יותר מציוריו היותר נעלים, בשאיפה למה שהוא נעלה מציורו\mycircle{°} הצפוי לאדון\hebrewmakaf כל\mycircle{°}. שחפצו נעוץ רק בחפץ צור\hebrewmakaf העולמים\mycircle{°} שלפניו צפוי ערך ההתעלות באין תכלית }\מקור{[עפ״י ע״א ג ב לו]}\צהגדרה{.}

\paragraphs

\ערך{עבותות }\הגדרה{- חוזק החבל הדרוש במקום שיש לחוש לניתוק }\מקור{[מ״ש רמא]}\צהגדרה{. }

\paragraphs

\ערך{עֲבֵרָה }\הגדרה{- (עברה) על דין ומשפט שבתורה\mycircle{°} }\מקור{[ע״ר א עז]}\צהגדרה{. }

\ערך{עֲבֵרָה }\הגדרה{- }\משנה{פגם העברה}\הגדרה{ - פגם חלישות הקשר עם החוק. תכונת העבריינות, שלעבור על דין ומשפט שבתורה\mycircle{°} הוא ענין אפשרי לה }\מקור{[ע״ר א עז]}\צהגדרה{. }

\הגדרה{ע״ע עוון. ע״ע חטא. ע״ע פשע. }

\paragraphs

\ערך{עַד }\הגדרה{- }\משנה{העַד }\הגדרה{- המעמד הקבוע שהוא מתפשט ומגיע בכל השדרות, המצוירות\mycircle{°} והמתעלות מכל ציור של אורך זמן או הערכה זמנית }\מקור{[ע״ר א קצו]}\צהגדרה{. }

\הגדרה{ע״ע נצח. ע׳ במדור פסוקים ובטויי חז״ל, עדי עד.}

\paragraphs

\ערך{עדה }\הגדרה{- }\משנה{(לעומת קהל\mycircle{°})}\myfootnote{ אדיר במרום ח״ב (מהדורת ספינר) עמ׳ יב: ״נגד נר״ן יש בישראל קהל, עדה וישראל (פסחים סד.)״.\label{3}}\הגדרה{ - רושם הקיבוץ שע״י שיווי ההרגשות הנפשיות בעבודת\hebrewmakaf ד׳\mycircle{°} }\מקור{[ע״א א 97]}\צהגדרה{. }

\הגדרה{ע׳ בנספחות, מדור מחקרים, צבור ושותפין.}

\paragraphs

\ערך{עדן העתיד}\הגדרה{ - עתיד ההתעלות העליונה, שהכל מתעלה, הכל מתפתח, והכל דולג ודוהר למעלה, הכל שמח, מתרעע, מרנן, ומלא גילה פנימית, ושבע שמחות, ונחת עדנים, והוד\mycircle{°} הכל, והדר\mycircle{°} מקור הכל מאיר בחלקיותו של כל אחד. אורו וישעו של הקץ היותר מרומם, סוכת\hebrewmakaf עורו\hebrewmakaf של\hebrewmakaf לויתן\mycircle{°} }\מקור{[עפ״י א״ק ב תקסד]}\צהגדרה{.}

\הגדרה{ע״ע ״גן עדן״. ע׳ במדור פסוקים ובטויי חז״ל, עדן.}

\paragraphs

\ערך{עוון }\הגדרה{- עיוות דרך הישרה באיזה פרט מיוחד }\מקור{[ע״ר א עז]}\צהגדרה{.}

\הגדרה{פעולה בלתי הוגנת המונחת ברוב בתאוה הבאה מצד תגבורת היצר }\מקור{[עפ״י ע״א ב ו מג]}\צהגדרה{. }

\הגדרה{הזדת התפרצות מהמסגרת שכבר הגיע יתרון החיים אליה, מתוך הכרת האדם רק את רצונו לקו להליכות חייו ולא לחק עליון המתוה לו את דרכו. ההתקוממות בעלת הפנים הקצרים, התעלמות מהחוג הקרוב שבמעט השקפה לא יוכל להתעלם. התקוממות על החוקים שאינם רחוקים במטרתם מחוש ההרגשה של האדם }\מקור{[עפ״י ע״א ד ה לו]}\צהגדרה{.  }

\ערך{עוון}\צהגדרה{ - פגם העוון}\הגדרה{ - פגם הנטיה המיוחדת לעיוות דרך הישרה באיזה פרט מיוחד. נטיה מיוחדת לאחת מהתאוות שאין פחיתות המדות כ״כ ניכרת בהן, אבל העיוות ביחש לקדושת\mycircle{°} התורה\mycircle{°} הוא גדול עי״ז. }\צהגדרה{העוון}\הגדרה{ בא מתוך הרצון הזועם הנוטה להרע\mycircle{°} מפני קלקלת נטיתו החמרית\mycircle{°}, ומתפרץ מפני נטית הגויה אל הזוהם\mycircle{°} הגופני }\מקור{[עפ״י ע״ר א עז, קכו, כח]}\צהגדרה{. }

\משנה{עיקר פגם העון }\הגדרה{- שפועל להחשיך את מאורו של כח הדמיון\mycircle{°}, שאין השכל יכול לו }\מקור{[פנ׳ יז]}\צהגדרה{. }

\הגדרה{ע״ע פשע. ע״ע חטא. ע״ע עברה. ע״ע כפרה. ע׳ בנספחות, מדור מחקרים, מחילה סליחה וכפרה. }

\paragraphs

\ערך{עוז }\הגדרה{- ע״ע עז. }

\paragraphs

\ערך{עולֶה }\הגדרה{- }\משנה{עולה אל תכונה עליונה }\הגדרה{- מתעלה ומתרומם על ידה }\מקור{[עפי ר״מ קסב]}\צהגדרה{. }

\paragraphs

\ערך{עולם }\הגדרה{- }\משנה{העולם }\הגדרה{- שמים וארץ וכל צבאם, המון בריות לאין תכלית, המון כוחות לאין חקר, המון הופעות\mycircle{°} והזרחות\mycircle{°}, המון תעופות\mycircle{°} מתגברות ומתחדשות, מתלטשות ומתגבשות\mycircle{°} }\מקור{[ע״ר ב קנח]}\צהגדרה{. }

\הגדרה{המציאות כולה, שבה נכללים גם העולמות\hebrewmakaf העליונים\mycircle{°} }\מקור{[מא״ה א קלז]}\צהגדרה{. }

\משנה{עולמים }\הגדרה{- חיים, נשמות\mycircle{°}, צבאי צבאות\mycircle{°}, כל היצורים וכל חוג החיים וההויה שלהם }\מקור{[עפ״י א״ק ב רפג, תמב]}\צהגדרה{. }

\paragraphs

\ערך{עולם }\הגדרה{- שביל רוחני במערכה\mycircle{°} מיוחדת }\מקור{[עפ״י קובץ ה קנב]}\צהגדרה{. }

\paragraphs

\ערך{עולם העתיד }\הגדרה{- העולם הגדול של הטוב\hebrewmakaf המוחלט\mycircle{°}, שערכי ההבדלה בין טוב\mycircle{°} לרע\mycircle{°} מתבטלים בו מרוב טובו ותגבורת שפעת חייו האדירים. העתיד המאושר הרחוק, חיי\hebrewmakaf העולם\hebrewmakaf הבא\mycircle{°} במילואם, שם הטוב והרע משתוים }\מקור{[עפ״י א״ק ב תצט]}\צהגדרה{. }

\הגדרה{ע׳ במדור פסוקים ובטויי חז״ל, עולם הבא. ושם, חיי עולם הבא. ע׳ במדור מונחי קבלה ונסתר, אור ההשואה. }

\paragraphs

\ערך{עולם הרעיון }\הגדרה{- המון התיאורים, הצעת המחשבות }\מקור{[עפ״י א״ק קפג]}\צהגדרה{. }

\paragraphs

\ערך{עולם פנימי }\הגדרה{- }\משנה{העולם הפנימי }\הגדרה{- העולם הרוחני\mycircle{°}, הציורי\mycircle{°}, האידיאלי\mycircle{°} והאמוני\mycircle{°} }\מקור{[עפ״י א״א 21]}\צהגדרה{. }

\משנה{עולמו הפנימי של האדם }\הגדרה{- ע״ע פנימי, פנימיות האדם. }

\paragraphs

\ערך{עולמות }\הגדרה{- }\משנה{״ש״י עולמות״\mycircle{°}}\הגדרה{ - מקומות\mycircle{°} מחזיקים }\מקור{[פנק׳ ג רפו]}\צהגדרה{.}

\paragraphs

\ערך{עונג }\הגדרה{- }\מעוין{◊ }\משנה{העונג }\הגדרה{מתיחש לכל דבר ע״פ תכונתו, כשיהי׳ נשלם כפי התעודה היותר נשגבה המוכנת לו }\מקור{[ע״א ב ט שס]}\צהגדרה{. }

\הגדרה{ע׳ במדור מונחי קבלה ונסתר, שעשוע, עונג, ושחוק. ושם, ״אין בטובה למעלה מעונג״.}

\paragraphs

\ערך{עונג }\הגדרה{- }\משנה{(עונג אלהי בהויה) }\הגדרה{- מלא רוח\mycircle{°} האצילות\mycircle{°} }\מקור{[ע״ר א קנא]}\צהגדרה{. }

\משנה{העונג העליון\mycircle{°}}\הגדרה{ - העדן\mycircle{°} האלהי\mycircle{°}, בכל מילואו\mycircle{°} וטובו\mycircle{°}, היורד לההויה כולה בשפע\mycircle{°} אורו\mycircle{°} }\מקור{[א״ק ב רפח (ע״ט ג)]}\צהגדרה{. }

\ערך{עונג העליון }\הגדרה{- }\משנה{החפשי של עלמא\hebrewmakaf דחירו\mycircle{°} }\הגדרה{- אור התענוג\mycircle{°} העליון של חדות\hebrewmakaf ד׳\mycircle{°} העליונה, חדות עולמים, של זיו\mycircle{°} אור חיי העולמים\mycircle{°} }\מקור{[קובץ ח רלח]}\צהגדרה{. }

\הגדרה{ע״ע נועם, העונג והנועם האלהי. }

\paragraphs

\ערך{עונג ה׳ }\הגדרה{- זיו\mycircle{°} החכמה\hebrewmakaf העליונה\mycircle{°} ושפע האהבה\hebrewmakaf בתענוגים\mycircle{°} הנמשכת ובאה עם יראת\hebrewmakaf ד׳\hebrewmakaf האמתית\mycircle{°} }\מקור{[אג׳ א עח]}\צהגדרה{.}

\paragraphs

\ערך{עונג }\הגדרה{- }\משנה{ההתענגות על ד׳ }\הגדרה{- שעשועי\mycircle{°} קודש\hebrewmakaf הקדשים\mycircle{°} העליונים }\מקור{[ע״ר א כד]}\צהגדרה{. }

\הגדרה{למעלה מכל מבטא ומכל מושג, לעילא מכל ברכתא\mycircle{°} ושירתא\mycircle{°}, תושבחתא\mycircle{°} ונחמתא, הגנוז והצפון, המתגלה באור\mycircle{°} האמונה\mycircle{°} והאהבה\mycircle{°}, המתראה בתכסיס התורה\mycircle{°} והמצוה\mycircle{°} כולה }\מקור{[א׳ ע]}\צהגדרה{. }

\הגדרה{התוכן הטהור\mycircle{°} שבאידיאליות\mycircle{°} היותר טהורה שאי\hebrewmakaf אפשר לבטאה בשום הגה אנושי. השאיפה\mycircle{°} העליונה, הנקיה מכל ערוב גס\mycircle{°}. מורגשת היא הרגשה מועטת בשאיפת הלב הטהור\mycircle{°} בהתעלותו מעל כל המצרים המצמצמים\mycircle{°} אותו, וזהו רק צל של צל צללה של השאיפה העליונה }\מקור{[עפ״י א״ק ג קסז]}\צהגדרה{. }

\משנה{ההתענגות על ד׳ התוכית }\הגדרה{- הצפיה בפליאי שמות\mycircle{°} ותארי הוד\mycircle{°}, המעלים אותנו להשגב בקשר חי מלא האחדות\hebrewmakaf המוחלטה\mycircle{°}, במקור חיי כל החיים אור\hebrewmakaf אין\hebrewmakaf סוף\mycircle{°} }\מקור{[ע״ר א סה]}\צהגדרה{. }

\הגדרה{מקור התענוג\mycircle{°} העליון, הגבוה מכל תכונות השמות, ולא תפיס בשום מחשבה כ״א ברעותא דלבא }\מקור{[עפ״י אג׳ ב לח]}\צהגדרה{. }

\משנה{עונג הטוב\mycircle{°}}\הגדרה{ - מקור כל קדושת השמות כולם }\מקור{[א״ק א קצב]}\צהגדרה{. }

\הגדרה{ע״ע בטול. ע׳ במדור מונחי קבלה ונסתר, ״אין בטובה למעלה מעונג״.}

\paragraphs

\ערך{עונג }\הגדרה{- }\משנה{להתענג על ד׳}\הגדרה{ - להרחיב את המאור האלהי בעולם, לזכך את הדעות וההרגשות בההשגה האלהית }\מקור{[קבצ׳ א קע]}\צהגדרה{.}

\הגדרה{ללכת בדרך אור\hebrewmakaf ד׳\mycircle{°} לטוב לו לעולם }\מקור{[ע״א ג ב ריז]}\צהגדרה{.}

\משנה{להתענג }\הגדרה{- להתעדן ולהתחיות }\מקור{[עפ״י ע״ה קלא]}\צהגדרה{. }

\paragraphs

\ערך{עונג }\הגדרה{- }\משנה{כח העונג הטהור }\הגדרה{- מנוחת הלב וישוב הדעת של הדבקות\hebrewmakaf האלהית\mycircle{°}, המתעלה מכל רגשנות רגילה }\מקור{[ע״ט לד]}\צהגדרה{. }

\הגדרה{ע״ע תענוג. ע״ע ערב, הטעם הערב (במובן רוחני).}

\paragraphs

\ערך{עונג }\הגדרה{- }\משנה{עונג הגן\mycircle{°}}\הגדרה{ - העונג\mycircle{°} הנמשך מהארת הידיעה בסדרי החכמה\hebrewmakaf העליונה\mycircle{°}, מצד סידור פעולות החכמה\hebrewmakaf העליונה }\מקור{[ע״א א 168]}\צהגדרה{.}

\הגדרה{ע׳ במדור מונחי קבלה ונסתר, עדן.}

\paragraphs

\ערך{עונג הכללי}\הגדרה{ עונג\hebrewmakaf של\hebrewmakaf נועם\hebrewmakaf ד׳\mycircle{°} }\מקור{[א״ק ג קפו]}\צהגדרה{.}

\paragraphs

\ערך{עונג של נועם\hebrewmakaf ד׳\mycircle{°}}\הגדרה{ - העונג\mycircle{°} הכללי, הכולל את כל העינוגים הנצחיים, שכל העינוגים הזמניים נשפעים גם כך ממנו }\מקור{[א״ק ג קפו]}\צהגדרה{.}

\משנה{עונג על נועם ד׳ }\הגדרה{- מדרגת הסתכלות אלהית על כל העולם כולו, וכל המחשבות וההרגשות כולן, כמלאים אור\hebrewmakaf ד׳\mycircle{°} וקודש של החיים העליונים }\מקור{[א״ת ב 81]}\צהגדרה{.}

\paragraphs

\ערך{עוצם }\הגדרה{- ע״ע עצם.}

\paragraphs

\ערך{עִוֵּר }\הגדרה{- }\משנה{כח עִוֵּר }\הגדרה{- כח בלתי חש ומרגיש, וק״ו בלתי משכיל\mycircle{°} ובלתי חדור מרוח\mycircle{°} הקדושה\mycircle{°} של החפץ\hebrewmakaf האלהי\mycircle{°} הטהור\mycircle{°} }\מקור{[עפ״י ע״ר א קנט]}\צהגדרה{.}

\הגדרה{כח פועל בלא שום השכלה ושאיפה}\צהגדרה{ }\מקור{[קבצ׳ ב פו 55 (פנק׳ ד רב)]}\צהגדרה{.}

\משנה{עורון }\הגדרה{- פראות, הכרחיות עבדותית\mycircle{°}, שלילת כונה, וחסרון אידיאליות\mycircle{°} }\מקור{[עפ״י א״ק ג קמג]}\צהגדרה{.}

\ערך{עִוֵּר }\הגדרה{- ע׳ במדור מונחי קבלה ונסתר, ״תנין עור״.}

\paragraphs

\ערך{עושר }\הגדרה{- }\משנה{העושר, לאיזה מטרה טובה הוא נמצא בעולם}\הגדרה{ - להרחבת הדעת ושלות הנפש, שעל ידן יוכל האדם לעסוק במושכלות, בתורה\mycircle{°} וחכמה וכל טוב }\מקור{[ע״א ג ב ע]}\צהגדרה{.}

\משנה{המגמה האמיתית של העושר }\הגדרה{- שלות הנפש הנמצאת על ידו. עצם העושר אינו המטרה כ״א התולדה של מעמד הנפש השקט והדעה המתרחבת היוצא ממנו }\מקור{[שם]}\צהגדרה{.}

\הגדרה{ע׳ במדור מספרים, שנים עשר אלף, (שימוש במספר לעומת החיים האקונומיים).}

\paragraphs

\ערך{עָז }\הגדרה{- מלא גבורה ואומץ }\מקור{[ע״ר א רא]}\צהגדרה{. }

\paragraphs

\ערך{עֹז }\הגדרה{- יסוד המפעל והתפשטות כל מצוי בממשיות ובגבורת\mycircle{°} מעשה. התכן המעשי, המתגבר במפעלים, ומוציא אל הפועל את כל מחשב וכל הערכה רוחנית\mycircle{°}, העומד בחזקה נגד כל מפריע ביצירה והתחדשות מתעלה }\מקור{[ע״ר א רח]}\צהגדרה{. }

\הגדרה{חוזק המציאות }\מקור{[מ״ש קיט (מא״ה ב יג)]}\צהגדרה{. }

\paragraphs

\ערך{עֹז }\הגדרה{- חיים מלאים ענין }\מקור{[ע״ר א שלא]}\צהגדרה{. }

\paragraphs

\ערך{עֹז }\הגדרה{- מה שלא יושג בסבת כוחות הגוף }\מקור{[ח״פ כו:]}\צהגדרה{. }

\משנה{עֹז }\צהגדרה{- הנהגת החכמה\mycircle{°} בתורת השגחה\mycircle{°}, מה שלא נוכל להשיג, פנימיות\mycircle{°} הפעולה, נקראת במשל }\צהגדרהמודגשת{- עֹז }\צמקור{[עפ״י שם]. }

\צהגדרה{המשל להנהגת הפעולות הנעשות מצד החכמה, המסדרת את הפרטים בשכל שחוץ מהם, למעלה ממה שמושכל דוקא ע״י הפעולות העוברות }\צמקור{[עפ״י שם, ושם כז.]. }

\paragraphs

\ערך{עֹז }\הגדרה{- }\משנה{עז נשמתי }\הגדרה{- גבורה\mycircle{°} אצילית\mycircle{°} }\מקור{[ע״ר א רז]}\צהגדרה{. }

\ערך{עֹז }\הגדרה{- }\משנה{״התפילין עז לישראל״}\myfootnote{ ברכות ו.\label{4}}\הגדרה{ - חוזק ותקיפות שלא ימוטו לעולם }\מקור{[פ״א רעו]}\צהגדרה{. }

\הגדרה{ע׳ במדור מצוות, הלכות, מנהגים וטעמיהן, תפילין. }

\paragraphs

\ערך{עֹז }\הגדרה{- גבורת האמת\mycircle{°} בצורתה היותר שלימה, מציאות חיה וקיימת, עליונה, בחביון\hebrewmakaf העז\mycircle{°} האלהי, באופן יותר עשיר, יותר קיים ויותר אמיץ בהוייתו, ויותר נעלה בהופעת רוממות אצילות שלמותו, מכל מה שכל רעיון יוכל לצייר ולתפוס }\מקור{[עפ״י ע״ט נ]}\צהגדרה{. }

\הגדרה{ע׳ בנספחות, מדור מחקרים, עז והדר. }

\paragraphs

\ערך{עֹז אלהים }\הגדרה{- עז עליון, עזוז האמת והאורה }\מקור{[קובץ ג כ]}\צהגדרה{.}

\הגדרה{ע׳ במדור שמות כינויים ותארים אלהיים, אלהים, עז אלהים. }

\paragraphs

\ערך{עז ד׳ }\הגדרה{- }\משנה{בעז ד׳ }\הגדרה{- בהתנשאותה של הנשמה\mycircle{°} למקור חייה מקור חיי כל }\מקור{[ע״ר ב עד]}\צהגדרה{. }

\paragraphs

\ערך{עז החיל }\הגדרה{- ע״ע חיל, עז החיל. }

\paragraphs

\ערך{עז החיים הגופניים }\הגדרה{- חדות\mycircle{°} הנפש המתאמת עם סערת הרוח הצוהל לקראת כל חמדת לב, בחיי הרוח וחיי החומר }\מקור{[ע״א ד ו ע]}\צהגדרה{.}

\paragraphs

\ערך{עזיז }\הגדרה{- משפיע חיל רב לפעול ולעבוד עבודה ברב חוסן\mycircle{°} }\מקור{[עפ״י ע״ר א קמה]}\צהגדרה{.}

\הגדרה{מתמיד בפעולתו לטובה\mycircle{°} קבועה ההולכת ומתעלה }\מקור{[א״ק ב תקס]}\צהגדרה{. }

\הגדרה{עז\mycircle{°} ביותר }\מקור{[רצי״ה א״ש יב הערה 4]}\צהגדרה{. }

\paragraphs

\ערך{עִטוּף}\myfootnote{ ע׳ נתיבות עולם א קלד. (הערת הרב שלום הימן).\label{5}}\הגדרה{ - הלבשה של כבוד }\מקור{[ע״א א ב כח 78]}\צהגדרה{. }

\ערך{עיטוף}\myfootnote{ ע״ע רש״י שבת יב: ד״ה מתעטף. מהר״ל באר הגולה, עמ׳ עד\hebrewmakaf ה. (הערת הרב שלום הימן).\label{6}}\מקור{ }\הגדרה{- מורה על צמצום התנועה. הגבלה הפנימית שיהיו המעשים מצומצמים לפי מדת ההשגה הטהורה בדעת\hebrewmakaf ד׳\mycircle{°} הבלתי מעורבבת בגסות\mycircle{°} החומריות\mycircle{°} }\מקור{[ע״א ג א נא]}\צהגדרה{. }

\paragraphs

\ערך{עטרה }\הגדרה{- }\מעוין{◊}\הגדרה{ משכללת את המלך להורות שמצד השלמות הכל נכנעים אליו לא מצד הכח לבד }\מקור{[מ״ש רצו (ה׳ רנב)]}\צהגדרה{. }

\משנה{עטרה (אלהית), תגא }\הגדרה{- מורה על הכנעת העושים נגד טבעם ברצון מפני ביטולם אל השלמות הנמצאת בפועל הנס\mycircle{°} }\מקור{[מ״ש רצו (ה׳ רנב)]}\צהגדרה{. }

\paragraphs

\ערך{עטרה }\הגדרה{- מעלה יתירה על החק הטבעי, מעלה נוספת למתעטר חוץ לאנושיותו }\מקור{[עפ״י ע״א א 86]}\צהגדרה{. }

\הגדרה{מה שהקנתה הנפש לעצמה השלמות ע״י הבחירה\mycircle{°} נוסף על שלמותה בטבע }\מקור{[עפ״י שם]}\צהגדרה{. }

\משנה{״עטרין״}\myfootnote{ באור הגר״א על משלי, ד ה, ח כא, יד יח, יט יד, כז כז. ממקור זוהר ח״ג רצא. (הערת מו״ר הרצב״י טאו). ובאדיר במרום (מהדורת ספינר) ח ״״צדיקים יושבים ועטרותיהם בראשיהם״ שהם התיקונים שתקנו בחייהם, כי כל זמן הבחירה הדבר צריך להיות מסור ביד האדם״.\label{7}}\הגדרה{ - קדושה\mycircle{°} יתירה באדם בענינים חדשים בנוסף על כח קדושת טבע הנפש שהיא מורשה לישראל }\מקור{[עפ״י ה׳ רי]}\צהגדרה{. }

\הגדרה{הקדושה שיש לאדם בענין עבודת\hebrewmakaf ד׳\mycircle{°} ממה שהאדם זוכה ע״י ההשתדלותו }\מקור{[עפ״י מא״ה ג קעה]}\צהגדרה{.}

\משנה{עטרות }\הגדרה{- תוספת קדושה, קדושה הנקנית ע״י הכנה עם עשיית המצוה}\צהגדרה{ }\מקור{[עפ״י ע״ר ב רסה]}\צהגדרה{.}

\הגדרה{ע״ע ״אחסנתין״. ע׳ במדור פסוקים ובטויי חז״ל, כתר לעולם הבא. ושם, נסירת הפרצופים.}

\paragraphs

\משנה{עטרה עליונה }\צהגדרה{- הציורים\mycircle{°} והכונות של ההשגות האציליות\mycircle{°}, של קהל\mycircle{°} ועדה\mycircle{°} המצטרפות ע״י כח אחד קדוש\mycircle{°} המאחדם, מצד הפנימי\mycircle{°} האיכותי, מתעלות לציור קדוש, מרומם ונעלה, מאוחד ביחודו, מתעלה ומתרומם מעל לכל חלקי הציורים המצטרפים, והוה }\צהגדרהמודגשת{עטרה עליונה}\צהגדרה{, לרומם בה פאר\mycircle{°} חי\hebrewmakaf העולמים\mycircle{°},  במושג עליון\mycircle{°} ונשגב מאד נעלה מכל הציורים הפרטיים, שנצטרפו ע״י הקבוץ בתחילת התאחדותו }\צמקור{[עפ״י ע״ר ב סב]. }

\משנה{עטרה }\צהגדרה{- קבוץ יחד כל הרצונות הטובים והכוונות וטהרות\mycircle{°} הלבבות של הכלל כולו ומכולם תעשה }\צהגדרהמודגשת{עטרה }\צמקור{[}\הגדרה{עפ״י מ״ש עז (מא״ה א קו)}\צמקור{]. }

\paragraphs

\ערך{עטרת המלוכה העליונה }\הגדרה{- העליה של שיבת ההויה לשיא גדלה. השיבה של כללות ההויה של הכל אל מחוז חפצה העליון\mycircle{°}, השרוי ממעל להצמצום\mycircle{°} וההגבלה\mycircle{°} ההויתית }\מקור{[ע״ר א מז]}\צהגדרה{. }

\הגדרה{ע׳ במדור מונחי קבלה ונסתר, ״לאשתאבא בגופא דמלכא״. }

\paragraphs

\משנה{עיבוט}\myfootnote{ יואל ב ז.\label{8}}\הגדרה{ - התעכבות. הצרת מהלך אחד על חברו }\מקור{[עפ״י קבצ׳ א ריד (פנק׳ א תקיג)]}\צהגדרה{. }

\paragraphs

\ערך{עילוי }\הגדרה{- ע״ע עלוי. }

\paragraphs

\ערך{עין בהירה}\הגדרה{ - }\משנה{העין הבהירה}\הגדרה{ - העין החודרת אל התוכן הרוחני, המסבב את הפעולות }\מקור{[א״ק ג שיג]}\צהגדרה{.}

\הגדרה{ע׳ בנספחות, מדור מחקרים, בהיר לעומת צלול.}

\paragraphs

\ערך{עלוי }\הגדרה{- זיכוך, עידון, התרוממות }\מקור{[עפ״י ע״ר א לט, קנו, שם ב שה, א״ק ב שעד]}\צהגדרה{. }

\הגדרה{ע״ע התעלות. ע׳ במדור פסוקים ובטויי חז״ל, עליה. ע״ע ירד. }

\ערך{עלוי גמור }\הגדרה{- }\משנה{העלוי הגמור שיחול בעולם }\הגדרה{- אור\mycircle{°} הצדק\mycircle{°} והמישרים\mycircle{°}, המתאחד עם ההוד\mycircle{°} והתפארת\mycircle{°}, עם הגבורה\mycircle{°} והנצח\mycircle{°}, והשלמת היצור כולו, והאדם וכל אגפיו בתחילה }\מקור{[מ״ה אהבה ה]}\צהגדרה{. }

\הגדרה{ע״ע התעלות, ההתעלות הגמורה. }

\paragraphs

\ערך{עלוי העולם }\הגדרה{- ההשתכללות הצורתית של הכלל ושל הפרט }\מקור{[קובץ ד מד]}\צהגדרה{. }

\paragraphs

\ערך{עלומה }\הגדרה{- }\משנה{העלומה }\הגדרה{- אשר בפנימיות הנעלמת }\מקור{[רצי״ה א״ש יא הערה 5]}\צהגדרה{. }

\paragraphs

\ערך{עליון }\הגדרה{- אידיאלי\mycircle{°}, אצילי\mycircle{°},  אלהי\mycircle{°} }\מקור{[עפ״י א״ק ב שיג, ושם א רלב]}\צהגדרה{. }

\הגדרה{רם\mycircle{°} במעלתו מכל העניינים הזעירים של בני אדם }\מקור{[עפ״י ע״ר א קיג]}\צהגדרה{. }

\paragraphs

\ערך{עליוניות המוחלטה }\הגדרה{- האמת\mycircle{°} כשהיא לעצמה }\מקור{[א״ק ג ל]}\צהגדרה{.}

\paragraphs

\ערך{עם }\הגדרה{- }\משנה{(אומה\mycircle{°}) }\הגדרה{- האופן הכללי של (הכלל\mycircle{°}, הקיבוץ) בתור גוש אחד, המחבר את כל האישים הפרטיים להיות ל}\צהגדרה{עם }\הגדרה{אחד }\מקור{[ע״ר ב פד]}\צהגדרה{.}

\paragraphs

\ערך{עם }\הגדרה{- הערך הצבורי מצדו הכללי של ריבוי האנשים, האופי המובלט בכלליותם, וההנהגה הכוללת אותם ומטביעה רישומה כלפי חוץ }\מקור{[רצי״ה ע״ר ב תב, עפ״י שם א רד, רה\hebrewmakaf ו]}\צהגדרה{.}

\ערך{עם }\הגדרה{- }\משנה{״עמים״ לעומת ״גויים״}\הגדרה{ - בעלי הקבוץ והחברה, הנתונים יותר לרשמים הבאים מבחוץ }\מקור{[עפ״י ע״ר א רו]}\צהגדרה{.}

\צמשנה{עם }\צהגדרה{- }\צמשנה{(לעומת גוי\mycircle{°})}\צהגדרה{ - מציאות אפיו העצמי הכללי}\צמקור{ [עפ״י א״ל עו].}

\צהגדרה{תרבות של עם. ביטוי של איכות, לאום}\צמקור{ [עפ״י שי׳ ב 404].}

\הגדרה{ע״ע גוי. ע״ע אומה. ע׳ במדור פסוקים ובטויי חז״ל, משפחות עמים.}

\paragraphs

\ערך{עמון }\הגדרה{- }\משנה{(תכונתם הלאומית) }\הגדרה{- אהבת האומה }\מקור{[קבצ׳ א לט]}\צהגדרה{. }

\הגדרה{ע׳ במדור מונחי קבלה ונסתר, קליפת עמון. ע״ע מואב. }

\paragraphs

\ערך{עמוק }\הגדרה{- בהיר ויסודי }\מקור{[ע״ר א רכ]}\צהגדרה{. }

\paragraphs

\ערך{עמידה - }\משנה{במהלך שלמות האדם }\הגדרה{- שהדברים שקנה יהיה קנינם חזק בנפשו, ולא יפסידם איזה שינוי וגרעון במצבו }\מקור{[ע״ר א רסב]}\צהגדרה{.}

\הגדרה{ע״ע הליכה.}

\paragraphs

\ערך{עמילן}\הגדרה{ - כח המקיים המחזיק והמעלה [}\צהגדרה{א״ה ב (מהדורת תשס״ב) 81 (א״ב ג)].}

\paragraphs

\ערך{עמים }\הגדרה{-}\משנה{ הצד המהותי בחיי העמים }\הגדרה{- אותו הצד שהננו מוצאים ערך של מציאות חשוב לכללות האומה\mycircle{°} }\מקור{[פנק׳ ד עד]}\צהגדרה{.}

\הגדרה{ע״ע אומה, טבע האומה, הרוחני והחומרי.}

\paragraphs

\ערך{עמלק }\הגדרה{- שטנם של ישראל ומנגדם }\מקור{[ע״א ב ט י]}\צהגדרה{. }

\הגדרה{ההיפוך ממש מכל מגמת הקודש של ישראל. מחזיק בטבע, בעיקר בעבור אהבתו להרע שנמצא גָבוּל בשמרי הטבע, והוא קשור בו. ואינו יכול לסבול ניסים\mycircle{°} בעולם ולא שינוי טבע במוסר האדם }\מקור{[עפ״י קובץ ו רנב, רנא]}\צהגדרה{.}

\משנה{המחשך של עמלק }\הגדרה{- התפשטות כל המידות הרעות בהנפשיות הלאומית של כל עם\mycircle{°} וגוי\mycircle{°}, המביאה את כל האסונות הפרטיים והכלליים }\מקור{[אג׳ ג פז]}\צהגדרה{.}

\הגדרה{מקור הגאוה\mycircle{°} הטמאה\mycircle{°} }\מקור{[ע״ט קי]}\צהגדרה{.}

\צהגדרה{הפכו המהותי הגמור של ישראל, כח הטומאה שבעולם }\צמקור{[פע׳ קא]. }

\הגדרה{ע׳ במדור מצוות, הלכות, מנהגים וטעמיהן, מחית עמלק. }

\paragraphs

\ערך{ענפים }\הגדרה{-}\משנה{ (לעומת שורשים\mycircle{°}) }\הגדרה{- פרטים (לעומת כללים) }\מקור{[פנ׳ ג שנ]}\צהגדרה{.}

\paragraphs

\ערך{ענפים }\הגדרה{- }\משנה{ענפי שורש העולם }\הגדרה{- האנשים כולם, וסדריהם, ותכונות רוחם\mycircle{°} הפנימיות\mycircle{°}, וכן היצורים כולם, מדומם צומח וכל החיים כולם }\מקור{[א״ק ב תקכא]}\צהגדרה{.}

\paragraphs

\ערך{עפיפה }\הגדרה{- אור\mycircle{°} מתעלה }\מקור{[ע״ר א כ]}\צהגדרה{.}

\הגדרה{ע״ע תעופה.}

\paragraphs

\ערך{עפר }\הגדרה{- }\משנה{ירידה אל העפר}\הגדרה{ - התקשרות בסדרים המעשיים ובהערכות הגופניות }\מקור{[עפ״י א״ק ב שכא\hebrewmakaf ב]}\צהגדרה{.}

\paragraphs

\ערך{עצמיות }\הגדרה{- }\משנה{הופעת העצמיות העושה את התוכן המציאותי}\הגדרה{ - לשד היש, מעמק ההויה, משרש פנימיותה, כמו שהוא המקור שמשם ההתהוות הנשמתית }\מקור{[א״ק ב תצה]}\צהגדרה{.  }

\paragraphs

\ערך{עצמיות }\הגדרה{- }\משנה{העצמיות הפנימית\mycircle{°} באדם ובעם }\הגדרה{- התכונה הנפשית של צלם\hebrewmakaf אלהים\mycircle{°} אשר בקרבם }\מקור{[עפ״י ע״ר ב רמה]}\צהגדרה{. }

\משנה{עצמיות האדם }\הגדרה{- פנימיות נפשו המיוחדה. שורש נשמתו\mycircle{°} }\מקור{[עפ״י שם נג, א״ש טו י]}\צהגדרה{. }

\הגדרה{ע״ע ״יראה עליונה״, יראה עילאה. }

\paragraphs

\ערך{עקדה }\הגדרה{- }\משנה{העקדה}\myfootnote{ בראשית פרק כב.\label{9}}\הגדרה{ - מסירות\hebrewmakaf הנפש\mycircle{°} המיוחדה לקדושתם של ישראל, הבוערת תמיד באש קדש בלב האומה, בכללותה וביחידיה }\מקור{[ע״ר א קיז]}\צהגדרה{. }

\הגדרה{אותה הפעולה, שכל חמדת אהבת הקודש, בכל שלהבתה האלהית, יצאה על ידה, לאור עולם בחיים ובפעל. האהבה הגמורה והמסירות האלהית העליונה, המופעת מקדושת הנשמה התוכית, כפי מה שהיא ברום אורה העליונה, אשר כל ישעה וכל חפצה הוא רק מלוי החפץ האלהי, בתשוקת חפצה הפנימי, בהשגתה הבהירה. המפעל הנצחי, אשר עדי עד יעמוד לנס עולם }\צהגדרה{<מתוך יסוד מפעל מלא אש אהבה העליונה הזאת, יונקות הן כל הקדושות של כל הקרבנות\mycircle{°} כולם, לכל עבודותיהם העליונות, העומדות ברום עולם ומעטירות את הנשמות, את החיים כולם, ואת כל ההויה, בתפארת כליל הודם> }\מקור{[ע״ר א פד]}\צהגדרה{.}

\הגדרה{המצוה העקרית שהשרישה בנו הכח למסירות\hebrewmakaf נפש }\מקור{[מ״ש סט]}\צהגדרה{.}

\ערך{נסיון\mycircle{°} העקדה}\הגדרה{ - המאורע הגדול, אשר נהיה ליסוד העליה הרוחנית\mycircle{°} של כל האנושיות כולה, שפעל להוציא אל אור החיים את תעופת הרוח היותר קדושה\mycircle{°}, הדבקה באלהים\hebrewmakaf חיים\mycircle{°} בכל לבבה, בכל נפשה, ובכל מאודה\mycircle{°} }\מקור{[ע״ר א פד]}\צהגדרה{.}

\משנה{יסוד העקדה }\הגדרה{- הגברת הרגשות היותר עדינים והתכללותם באור האהבה המרוממה והעדינה לאין\hebrewmakaf סוף\mycircle{°}, אהבת\hebrewmakaf ד׳\mycircle{°} הבהירה, כביטול טפה אחת בים }\מקור{[קובץ ה צה]}\צהגדרה{.}

\paragraphs

\ערך{עקום}\הגדרה{ - ע׳ בנספחות, מדור מחקרים, קוים עקומים.}

\paragraphs

\ערך{עִקָּר }\הגדרה{- }\משנה{(בדת)}\הגדרה{ - יסוד שרשי }\מקור{[ל״ה 160]}\צהגדרה{.}

\ערך{עִקָּרִים }\הגדרה{- }\משנה{(באמונה) }\הגדרה{- התמצית העולה מכל הדעות\mycircle{°} וההרגשות הנטיות ומהלכי החיים שבאור\mycircle{°} האמונה\mycircle{°} }\צהגדרה{<והחיים של כל פרטי הדעות האמוניות מכונסים בהם> }\מקור{[מ״ר 14 (פנק׳ ד מא)]}\צהגדרה{. }

\paragraphs

\ערך{ערב }\הגדרה{- }\מעוין{◊ }\משנה{ערבות}\הגדרה{ באה כשהדברים הנפגשים ברגש מתאימים אל המהות העצמית של המקבל את הרושם מהדברים ההם }\מקור{[ע״ר א נט]}\צהגדרה{. }

\מעוין{◊}\הגדרה{ מה שנאות מאד לטבע }\מקור{[עפ״י מ״ש קנח (ה׳ קצו)]}\צהגדרה{. }

\paragraphs

\ערך{ערב }\הגדרה{- }\משנה{הטעם הערב (במובן רוחני) }\הגדרה{- בליטת התענוג\mycircle{°} במילואו, הבא כשרוח הקדושה\mycircle{°} מתמלא בכל לשדו העליון\mycircle{°}, בכח ובפועל: הקודש הנעלם שעודנו בכח, והקודש המתגלה בכליל הדרו\mycircle{°} בפועל }\מקור{[עפ״י ע״ר א קנא]}\צהגדרה{. }

\paragraphs

\ערך{ערבה }\הגדרה{- יסוד הגיהנם\mycircle{°}, רע בה, הב רע }\מקור{[קובץ ו ערה]}\צהגדרה{. }

\paragraphs

\ערך{עריגה אלהית }\הגדרה{- נועם\mycircle{°} ד׳ המפעם בנשמה\mycircle{°} וקרבת\hebrewmakaf אלהים\mycircle{°} שלה טוב מכל מחמדים. ההרגשה העליונה הכוללת שבנשמה, המורגשת גם בלב החיים. העריגה שכל בעל רוח כביר משתוקק אליה וחפץ להרשימה בכל תכני חייו, שהיא מושרשת בצורה מאירה הרבה בחיי האומה\mycircle{°} המיוחדת מני אז לנשיאת דגל\hebrewmakaf קודש זה ברמה, בחירוף נפש ומלחמה כבדה, וכל השאיפות הרוחניות\mycircle{°} העדינות הנן קשורות לעריגה עליונה זו בקשר של קיימא ובעילויה הן מתעלות }\מקור{[עפ״י א״ת ג א, אג׳ ב קיט]}\צהגדרה{. }

\הגדרה{הצמאון\mycircle{°} הנורא לאור אלהי אמת, השקיקה לאור הנעים לפאר חי\hebrewmakaf העולמים\mycircle{°}. הכמיהה\mycircle{°}, פנית הנשמה אל הרוממות האלהית העליונה. התשוקה אל הגודל והאור\mycircle{°}}\צהגדרה{ }\מקור{[ע״ר א קנט\hebrewmakaf קס]}\צהגדרה{.}

\משנה{העריגה הנצחית }\הגדרה{- דרישת השלמות העליונה }\מקור{[פנק׳ ב קצד]}\צהגדרה{.}

\הגדרה{ע״ע צמאון אלהי. ע״ע כמיהה.}

\paragraphs

\ערך{עריות }\הגדרה{- ע׳ במדור מצוות, הלכות, מנהגים וטעמיהן, גילוי עריות.}

\paragraphs

\ערך{ערך }\הגדרה{- }\משנה{(ערכו של דבר, של שיטה) }\הגדרה{- מקומו\mycircle{°} וענינו }\מקור{[ע״ר א של]}\צהגדרה{. }

\paragraphs

\ערך{עש }\הגדרה{- קבוצת כוכבים שמבטאת בעולם הרוחני את כח ההתעלות והיצירה הכללית, כח כללי בההרחבה הגדולה של היש, כח המצייר הכללי, כח המעשה הכללי, הנתון מאדון כל המעשים החותך חיים לכל חי, כח הפועל הגומר השתלמותם של היצורים ע״י ההערכה המסודרת של הכוחות, הכח הפועל המשכלל בכללות. כח ההתגדלות והתעלות. הכח המשכלל הכללי שבו נעוץ הכלל בהפרט, הסוף בהתחלה. העש נגזר מכח המרכיב המפתח והמצייר, ממעשה הפועל בכחו בהרכבה, בהצטיירות, בהגבלה של כל כח לערכו, החם והקר, החיוב והשלילה. הכח המחולל והפועל השולט בפעלו על כל נקודה פרטית מנקודות המציאות, לחבר ולהפריד, לחמם ולקרר, להעמיד ולשנות. הכח המחבר ומאחד את הכח הפרטי עם הכח הכללי של היחש והחיבור, התכלית הסובבת מראש המעשים עד סופם, המתגלה בפעולתו על היסודות והחומרים או על הכוחות והציורים המופשטים, בתור מערב ומהפך את סדרם הבודד הטבעי הנעזב, לכוננם בסדר מלאכותי לתוצאות החיים. הכח המוציא אל הפועל את ההויות הטבעיות, המורגשות והשכליות, בכח החיבור והציבור, ההכללה והעירוב של הכוחות והפזורות הרבות שבמציאות. הכח המוציא אל הפועל כל מעשה ד׳ הגדול אשר עשה עושה ויעשה, שיש בו את תכונת ההשבחה וההשבתה, להגביל את הכוחות הסוערים העוברים גבול, שהפעולה ניכרת ע״י חסרון והעדר וחלישות של הכוחות, ובזה היא מסדרת ומערבת את העצמים הגדולים בכח, המפוזרים איש לעברו להיות במערכה }\מקור{[עפ״י ע״א ב ט קלג, קלד]}\צהגדרה{. }

\הגדרה{ע״ע כימה. }

\paragraphs

\משנה{עשיה }\הגדרה{- }\צמשנה{(לעומת בריאה\mycircle{°} ויצירה\mycircle{°}) }\צהגדרה{- גמר הדבר }\צמקור{[ק״ת עז].}

\paragraphs

\ערך{״עשן״}\הגדרה{ - }\משנה{סוד ע׳קרב ש׳רף נ׳חש}\הגדרה{ - }\משנה{עקרב}\הגדרה{ רומז על כפירה בתורה שבכתב שממית ודאי, ו}\צהגדרה{נחש }\הגדרה{סוד הז ילזול בתורה שבעל פה, אע״פ שלפעמים בי״ד מבטל דברי בי״ד חברו. אבל הבא בכפירה נזוק ומת בלחישת הנחש, ו}\צהגדרה{שרף }\הגדרה{הוא הכולל מקום יחוד תורה שבכתב ותורה שבעל פה, והמחלק ביניהם ומפרידם נשרף בסוד אש אוכלה }\מקור{[מ״ר 445]}\צהגדרה{. }

\paragraphs

\ערך{עת }\הגדרה{- הופעה יוצאת מכלל של מאורעות של דברים משתנים. ענין מופרד ועומד בפני עצמו }\מקור{[ע״ר א מו]}\צהגדרה{. }\mylettertitle{פ}

\ערך{פאר }\הגדרה{- }\מעוין{◊}\הגדרה{ יבוא כשההשפעה של הטובה היא מזוגה ומתאמת באופן נפלא, שקולה בפלס ישר, ומכוונת במדותיה, בצבעיה השונים באופן הרמוני, שהתפארת\mycircle{°}, והיופי\hebrewmakaf העליון הנשגב\mycircle{°} מאד מזהיר ממנה }\מקור{[ע״ר א קח]}\צהגדרה{. }

\ערך{פאר אמיתי }\הגדרה{- ע״ע תפארת. }

\ערך{פאר עליון }\הגדרה{- התפארת\hebrewmakaf המחלטה\mycircle{°} }\מקור{[א״ק א קנג]}\צהגדרה{. }

\הגדרה{ע׳ במדור פסוקים ובטויי חז״ל, ארוסין (בין השי״ת וישראל).}

\paragraphs

\ערך{פאר }\הגדרה{- ״}\משנה{להתפאר״}\myfootnote{ ישעיה ס כא. סא ג.\label{1}}\הגדרה{ - לפרח ולהצמיח ענפים וענפי ענפים לפאר\mycircle{°} ולתהילה\mycircle{°} }\מקור{[ע״ר ב קנט]}\צהגדרה{. }

\הגדרה{להוסיף אור\mycircle{°} וחיי עולמים על כל אשר יצר ד׳ מראש ועד סוף }\מקור{[שם]}\צהגדרה{. }

\הגדרה{להוסיף פאר ושלמות }\מקור{[עפ״י שם קנה]}\צהגדרה{. }

\הגדרה{להגדיל יתר פאר וענף }\מקור{[שם קנז]}\צהגדרה{. }

\הגדרה{מלשון פארות וענפים, להגדיל יותר }\מקור{[פנק׳ ג צח]}\צהגדרה{.}

\paragraphs

\ערך{פאר }\הגדרה{- }\משנה{(לד׳) }\הגדרה{- השבח\mycircle{°} העליון, של התהילה\hebrewmakaf האלהית\mycircle{°}, בבואו לתכלית ההשלמה שלו כשהוא מופיע בחיים, לשפר את המעשים ביתרון פאר\mycircle{°}. גדלות הרוחב של המפעל המעשי והליכות החיים המפאר את שבח התהילה }\מקור{[עפ״י ע״ר א קצז]}\צהגדרה{. }

\הגדרה{ע״ע ״נפארך״. ע׳ במדור שמות כינויים ותארים אלהיים, ״מפואר״. }

\paragraphs

\ערך{פגר}\הגדרה{ - }\משנה{ענינו}\הגדרה{ - בטול עבודה }\מקור{[ע״ר א ע]}\צהגדרה{.}

\paragraphs

\ערך{פגר }\הגדרה{- ע׳ במדור מוות ועניינו.}

\paragraphs

\ערך{פדיה }\הגדרה{- }\משנה{(פדית ד׳) }\הגדרה{- ההצלה מכל רע באופן התאורי, בריאת החוק המכיל את תוכן הישועה\mycircle{°}. ההצלה המסירה את הרעה החוקית }\מקור{[עפ״י ע״ר א קצו]}\צהגדרה{. }

\הגדרה{ע׳ במדור שמות כינויים ותארים אלהיים, מציל. }

\paragraphs

\משנה{פוליטיקה שלנו}\צהגדרה{ - בטוי צד הצורה של מהלך חיינו}\צמקור{ [ל״י א 10].}

\paragraphs

\משנה{פופולרי}\הגדרה{ - שוה לכל נפש }\מקור{[א״ק א ד]}\צהגדרה{.}

\paragraphs

\ערך{פטום }\הגדרה{- השמנת הגוף והגדלתו הכמותית }\מקור{[ע״ר א קלט]}\צהגדרה{. }

\paragraphs

\ערך{פיוט ושירה\mycircle{°}}\הגדרה{ - }\משנה{חיי הפיוט והשירה }\הגדרה{- סידור הרגשות }\מקור{[א״ק ג רד]}\צהגדרה{. }

\הגדרה{ע״ע התבוננות. }

\paragraphs

\ערך{פינוק}\הגדרה{ - עידון, הוצאה לפעל באופן שלם וחשוב}\צהגדרה{ }\מקור{[הקדמת ש״ה ז]}\צהגדרה{.}

\paragraphs

\ערך{פנים }\הגדרה{- הצד העקרי והחשוב בכל דבר הוא פני הדבר }\מקור{[ע״א א ב מג]}\צהגדרה{. }

\הגדרה{ע״ע אחור.}

\paragraphs

\ערך{פנים }\הגדרה{- הצד שהחיים הפנימיים של רוח\hebrewmakaf ד׳\mycircle{°} אשר במלא עולמו יונקים ממנו }\מקור{[מ״ר 249]}\צהגדרה{. }

\הגדרה{ ע״ע אחור.}

\paragraphs

\ערך{פנים }\הגדרה{- }\משנה{ההנהגה האלהית שהיא לפנים }\הגדרה{- לצד ההשתלמות }\מקור{[עפ״י ע״ר ב סז]}\צהגדרה{.}

\הגדרה{ע״ע אחור.}

\paragraphs

\ערך{פנים }\הגדרה{- }\משנה{פנים מחשביים }\הגדרה{- הכחות המפכים ועולים כמעין המתגבר מעצמת טבעם הרוחני, שמהות החיים חתומה בהם בחותם מלא ובולט. ההבהקה התדירית העומדת במלואה וחדושה המתמר ועולה בכל עת ובכל שעה, המפשטת ממנו אורים גדולים <הנעשים לנחלי חכמה מוקשבת מחוללת אורה בינה והשכל לימודיים> }\מקור{[עפ״י ר״מ קפד]}\צהגדרה{. }

\ערך{פנים}\הגדרה{ - }\משנה{נושאי החושים העליונים}\הגדרה{ - המזריחים\mycircle{°} את המאור הנשמתי מתוך הגויה המלאה את אוצר החיים, אל כל העברים הרחוקים }\מקור{[ר״מ קפ]}\צהגדרה{.}

\הגדרה{ע״ע אחוריים. }

\paragraphs

\ערך{פנים ואחור בנבראים }\הגדרה{- }\משנה{אחור}\הגדרה{ - טבע טוב וחפץ טוב. רצון שכלי טוב - }\צהגדרה{פנים }\מקור{[פנק׳ ג קפ]}\צהגדרה{.}

\paragraphs

\ערך{פנים ואחור במציאות הרוחניות }\הגדרה{- הטבע והבחירה. הטבע הוא אחוריה\mycircle{°} של הבחירה\hebrewmakaf החפשית\mycircle{°} }\מקור{[קובץ א שמב]}\צהגדרה{. }

\הגדרה{טבע, ושכל ובחירה }\מקור{[פנק׳ ג קעט]}\צהגדרה{.}

\הגדרה{ע׳ במדור פסוקים ובטויי חז״ל, נסירת הפרצופים. }

\paragraphs

\ערך{פנימי }\הגדרה{- שמימי\mycircle{°}. נשמתי\mycircle{°} }\מקור{[עפ״י ע״ר א קכד, אג׳ ג ד]}\צהגדרה{. }

\הגדרה{קשור בעצמות הנשמה}\צהגדרה{ }\מקור{[עפ״י ע״ר א קכה]}\צהגדרה{.}

\הגדרה{עצמי וטפוסי }\מקור{[א״א 133]}\צהגדרה{. }

\הגדרה{צורתי\mycircle{°} }\מקור{[ע״א ד ט עו]}\צהגדרה{. }

\הגדרה{רוחני\mycircle{°}, ציורי\mycircle{°}, אידיאלי\mycircle{°} ואמוני\mycircle{°} }\מקור{[עפ״י א״א 21]}\צהגדרה{.}

\הגדרה{מוסרי\mycircle{°} ושכלי }\מקור{[ע״ה קז]}\צהגדרה{. }

\צהגדרה{חיוני נפשי }\צמקור{[עפ״י א״ק א, מבוא הרד״ך }\צהגדרה{35}\צמקור{]. }

\paragraphs

\ערך{פנימי }\הגדרה{- תכליתי }\מקור{[עפ״י ע״א ג ב ר]}\צהגדרה{. }

\paragraphs

\ערך{פנימי }\הגדרה{- }\משנה{פנימיות האדם }\הגדרה{- עומק הויתו }\מקור{[ע״ר ב ז (א״ק א קפח)]}\צהגדרה{. }

\הגדרה{בא מתוך מקורה של הנשמה}\צהגדרה{\mycircle{°} }\מקור{[ע״ר א קנא]}\צהגדרה{.}

\משנה{הצד הפנימי של האדם }\הגדרה{- הנפש החכמה ונטיותיה }\מקור{[ע״א ג ב קא]}\צהגדרה{. }

\משנה{הכח הפנימי באדם }\הגדרה{- רוח\hebrewmakaf ד׳\mycircle{°} אשר על האדם ודעת\mycircle{°} ויראת\hebrewmakaf ד׳\mycircle{°} אשר בקרבו }\מקור{[ע״ר ב רנז]}\צהגדרה{.}

\משנה{מצבו הפנימי}\הגדרה{ - יחושו הנאמן אל המושגים\hebrewmakaf האלקיים\mycircle{°}, הנטייה אל האשר\mycircle{°} והטוב\mycircle{°} מצד עצמם}\צהגדרה{ }\מקור{[אג׳ א פה]}\צהגדרה{.}

\משנה{החיים הפנימיים }\הגדרה{- החיים השכליים\mycircle{°} והמוסריים\mycircle{°} של האדם מצד צורתו וצלמו, צלם\hebrewmakaf אלהים\mycircle{°} }\מקור{[שם כט (ע״א ג א מב)]}\צהגדרה{.}

\משנה{עולמו הפנימי של האדם }\הגדרה{- מחשבותיו, רגשותיו, מדותיו, רצונו }\מקור{[עפ״י קובץ א תרנא]}\צהגדרה{. }

\הגדרה{ע״ע חצוני, חצוניות באדם, הצד החצוני. ע״ע הגיון הפנימי.}

\מעוין{◊ }\משנה{עולמנו הפנימי}\צהגדרה{ הוא עולמנו במציאות, בקישור בחפץ סמוי, ועולם שאינו עולמנו, בהכרה, בידיעה, בחדירה }\צמקור{[א״ק ב שעד]. }

\paragraphs

\ערך{פנימי }\הגדרה{- }\משנה{בנין פנים ברוח\mycircle{°} האדם (לעומת בנין\hebrewmakaf חוץ\mycircle{°}) }\הגדרה{- כל מה שהוא מוצע על פי יסוד החפץ הפנימי\mycircle{°} לטובה. הטבעת הרצון היותר עמוק. זעזוע הנטיות של הרצון מהדממה שלהן והתעוררותן לטובה\mycircle{°} }\מקור{[עפ״י א״ק ג פח]}\צהגדרה{. }

\paragraphs

\ערך{פנימי }\הגדרה{- }\משנה{ההכרה הפנימית (לעומת חיצונית\mycircle{°}) }\הגדרה{- ההכרה המתגברת בתוכיותו של האדם כמעין הנובע }\מקור{[א״ק א נח]}\צהגדרה{. }

\משנה{מושג פנימי }\הגדרה{- מושג שבא לאדם מתוכיותו, ומתגלה בו בתור דבר ההולך ומתגלה מפנים לחוץ }\מקור{[עפ״י שם שם]}\צהגדרה{. }

\paragraphs

\ערך{פנימי }\הגדרה{- }\משנה{רוממות פנימית }\הגדרה{- רגש נעלה וידיעה מקפת }\מקור{[עפ״י ע״ה קמה]}\צהגדרה{. }

\paragraphs

\ערך{פנימיות }\צהגדרה{- }\הגדרה{הסבה העליונה הגורמת את (המציאות החיצונית) ותולדותיה}\צהגדרה{ }\מקור{[עפ״י אג׳ ד (מהדורת תשע״ח) קסט-קע]}\צהגדרה{.}

\paragraphs

\ערך{פנימיות היש}\הגדרה{ - העצמותיות הנפשית של עצמו ושל העולם כולו במהותו הפנימיותית}\צהגדרה{ }\מקור{[א״ק ג שמח]}\צהגדרה{.}

\paragraphs

\ערך{פנימיות העולם }\הגדרה{- }\משנה{(לעומת חיצוניותו\mycircle{°}) }\הגדרה{- כח הכלל\mycircle{°} של כללות קדושת\mycircle{°} האומה\mycircle{°} כנס״י\mycircle{°} }\מקור{[אג׳ א שסט]}\צהגדרה{. }

\הגדרה{ע״ע שכלול פנימי. }

\paragraphs

\ערך{פסוק }\הגדרה{- מאמר\mycircle{°} המורכב משלובי\mycircle{°} משפטים }\מקור{[עפ״י ר״מ קעד]}\צהגדרה{. }

\paragraphs

\ערך{פסוק }\הגדרה{- }\משנה{הפסוק, האותיות של שמו (של האדם) שבתורה }\הגדרה{- הארת הנשמה\mycircle{°} העצמית. היסוד\mycircle{°}, הצנור\mycircle{°} המשפיע לעצמיות ממקור החיים המיוחד לאדם, שהוא כל מרכז התורה (של האדם) }\מקור{[עפ״י א״ק ג קלז, קלט]}\צהגדרה{. }

\הגדרה{ע״ע שם. ע׳ במדור מונחי קבלה ונסתר, חבוט הקבר.}

\paragraphs

\ערך{פסיחה }\הגדרה{-}\משנה{ פסיחה על (דבר) }\הגדרה{- שריה וביאה על אותו הדבר דרך דילוגו, <אין הפסיחה מתיחסת על מה שימנע מלבא שם ויעבור עליו דרך קפיצת דילוגו כ״א על מה שינוח עליו> }\מקור{[עפ״י מ״ש קלט (ה׳ קפא)]}\צהגדרה{. }

\ערך{פסיחה ברעיון}\myfootnote{ \textbf{פסיחה, דילוג, קפיצה }- רש״י שמות יב יג ״כל פסיחה לשון דלוג וקפיצה [...] כל הפסחים הולכים כקופצים״. סוכה מט: ״א״ר אלעזר כל העושה צדקה ומשפט כאילו מילא כל העולם כולו חסד, שנאמר (תהלים לג ה) אוהב צדקה ומשפט חסד ה׳ מלאה הארץ. שמא תאמר כל הבא לקפוץ קופץ? – ת״ל (תהלים לו ח) ״מה יקר חסדך אלהים״. וע׳ בביאור המהרש״א שם ובע״ז כ:. ע׳ במדור פסוקים ובטויי חז״ל, קופץ ליטול\hebrewmakaf את\hebrewmakaf השם. הערת משה עשת.\textbf{עולם חיצוני} - ע׳ בנספחות, מדור מחקרים, חיצון, עולם חיצוני.\label{2}}\הגדרה{ - דילוג האדם מפעולתו הפרטית תיכף אל גבהי עולם\hebrewmakaf החיצוני. בדרך קפיצה יאבה לצייר את מעשיו בגבהי מרומים ובעליית העולמות, והסולם\mycircle{°} הנפשי שלו נשמט מתחת רגליו <אז את החסרון המציאותי ימלא על ידי דמיונות והזיות}\myfootnote{ מ״ה קנא ״מי שמצייר בדעתו שהוא מתקן את העולמות בעבודתו, ואינו יודע את ערך נפשו ואת סדר הרוחניות הנפשית בכללה, הוא מלא הזיה ודמיונות כוזבים״.\textbf{הזי}\textbf{ות}\textbf{ }- ע״ע קובץ א יט, ובפנק׳ א מב. \label{3}}\הגדרה{, אשר יובילוהו בארחות חושך וצלמות בלא סדרים> }\מקור{[פנק׳ ד קד (קבצ׳ ב סא)]}\צהגדרה{.}

\paragraphs

\ערך{פעור }\הגדרה{- }\משנה{התמצית הגנוזה בפעור, שבשביל כך קבורתו\hebrewmakaf של\hebrewmakaf משה\mycircle{°} היא ממולו}\הגדרה{ - ההנהגה האלהית שבכפירה\mycircle{°} ותוצאותיה החירוף והגידוף שהם המדרגה היותר עליונה שמיסודות הגילוי האידיאלי באלהות, גילוי יסוד החסד הפשוט האידיאלי שאין מקום לו גם להכרת טובה}\myfootnote{ \textbf{הגילוי האידיאלי }\textbf{באלהות}\textbf{ וכו׳ שאין מקום לו גם להכרת טובה} - ע״ע קבצ׳ א קצט יז.\label{4}}\הגדרה{ }\מקור{[עפ״י קובץ ב סד]}\צהגדרה{. }

\paragraphs

\ערך{פקידה }\הגדרה{- }\משנה{(אלהית) }\הגדרה{- הפקידה היא ביחס למצבה ההוי של נקודה נשמתית קדושה שבכל פרט מפרטינו. עלוי\mycircle{°} הנקודה ההוית, הנעלמת במציאותה, ובלועה בעומק הסתרתה, עד כדי כהות אורה\mycircle{°} }\מקור{[עפ״י ע״ר א פג]}\צהגדרה{. }

\הגדרה{ע״ע זכרון. }

\הגדרה{זכות\mycircle{°} שמתעוררת ע״י איזה ענין להביא ענין שהוא כבר מוכן מראש\hebrewmakaf מקדם\mycircle{°} וגמור, אלא שצריך זכות להגיעו ליד הזוכה בו, <ולא יאמר על דבר שלא נתחדש עדיין וצריך לצאת מן הכח אל הפועל בהדרגה> }\מקור{[מ״ש רפד (ה׳ רמב)]}\צהגדרה{. }

\הגדרה{ע׳ בנספחות, מדור מחקרים, פקידה וזכרון. }

\paragraphs

\ערך{פקידה }\הגדרה{- }\משנה{״יפקד ד׳ על צבא המרום במרום ועל מלכי האדמה על האדמה״}\myfootnote{ ישעיה כד כא.\label{5}}\הגדרה{ - חלוף מעמדים ותכונות, <לא הכבדה וכליון> }\מקור{[א׳ מא]}\צהגדרה{.}

\paragraphs

\ערך{פרוד }\הגדרה{- ערך\mycircle{°} הדברים הניתק מעל הכל, שהוא מכון של כל עמל ויגיעת בשר }\מקור{[ע״א ד יא יט (ע״ר ב מח)]}\צהגדרה{. }

\הגדרה{החיים שהשאון והנגוד הוא מקור שבעם }\מקור{[א״ק א קפט]}\צהגדרה{. }

\מעוין{◊ }\משנה{הפרוד}\הגדרה{ מביא את המות\mycircle{°} ואת הכליון, את הרשעה ואת השקר }\מקור{[שם]}\צהגדרה{. }

\משנה{פרודים }\הגדרה{- התשוקות והציורים\mycircle{°} האלהיים בחלקיותם, הפוגמים את רוח האדם, ומעכבים את השלמתו הזמנית והנצחית }\מקור{[עפ״י א״א  78 (קובץ ז עו)]}\צהגדרה{. }

\paragraphs

\ערך{פרוזה }\הגדרה{- ספור של מאורע שאינו מרותק עם רגשי הנפש והתפעלותיה השיריות }\מקור{[ר״מ קיט]}\צהגדרה{. }

\paragraphs

\ערך{פרוש }\הגדרה{- להבין אל נכון יסוד מאמר מצד עצמו, את המונח בו בכללו ובפרטיו, זאת היא תכונת הפרוש, ״מפורש״ מלשון פרשׂ, ״יפרשׂ השׂוחה לשׂחות״, הרחבה של עצם הדברים שכבר ישנם בתוכנו של המאמר אלא שהם בו מקופלים, ע״כ עלינו להרחיב הקמטים כדי לעמוד על כל הרחבתו של המאמר }\מקור{[ע״א א, הקדמה, יד (מ״ר 208)]}\צהגדרה{. }

\הגדרה{ע׳ במדור פסוקים ובטויי חז״ל, מ״ם סתומה, מאמר סתום. ע״ע באור. ע׳ במדור תורה, דרש.}

\paragraphs

\ערך{פרטי }\הגדרה{- }\משנה{(הצד הפרטי באדם, לעומת הכללי\mycircle{°}) }\הגדרה{- סגור בעניני עצמו, מרגיש לעצמו בעצמו את מכאוביו וענוייו, וכמו כן הוא בעצמו, יותר מאחרים, מרגיש את שמחותיו }\מקור{[ע״ר א רכא]}\צהגדרה{. }

\paragraphs

\ערך{פרטי }\הגדרה{- יחיד מוגבל }\מקור{[א״ק א רנז]}\צהגדרה{.}

\paragraphs

\ערך{פרטי }\הגדרה{- יחודי. יחוד הצורה העצמית הלאומית }\צהגדרה{[עפ״י א׳ י, מ״ר }\צמקור{97\hebrewmakaf 96}\צהגדרה{].}

\הגדרה{מיוחד לנו }\מקור{[ע״ה קכג]}\צהגדרה{.}

\הגדרה{ע״ע כללי. }

\paragraphs

\משנה{פרטי }\צהגדרה{- גלוי אנושי}\צמקור{ [א״ל רכט].}

\הגדרה{ר׳ כללי.}

\paragraphs

\ערך{פרטי }\הגדרה{- }\משנה{העצמיות הפרטית של כל פרט מפרטי ההויה}\הגדרה{ - התגלות אלהות\mycircle{°} הזורחת\mycircle{°} בגוונים\mycircle{°} שונים לפנינו }\צהגדרה{[א״ק ב שצח]}\הגדרה{.}

\paragraphs

\ערך{פרי השפתים }\הגדרה{- הפעולה שהדבור\mycircle{°} עושה, שהוא מחשיף את האור הפנימי של הנשמה\mycircle{°}, המלאה תשוקה עליונה וחזון קדש נורא. ובכח הקדושה הגנוזה בסוד פרי השפה, פעולת הצמיחה עוברת היא ושוטפת ג״כ במלא עולמים ופועלת בעולם להשפיע שפע ברכה וחיים מאירים }\מקור{[עפ״י ע״ר א קסב]}\צהגדרה{.}

\הגדרה{ע׳ במדור מצוות, הלכות, מנהגים וטעמיהן, תפילה, (התפילה המעשית).}

\paragraphs

\ערך{פרישות }\הגדרה{- הפרדת הנשמה הזכה מהסתבכות ברגשי הגויה וכוחותיה, התפעלויותיה ונטיותיה הגסות, שבתחתית המצע של טבע החי אשר באדם }\מקור{[עפ״י ע״א ד ט כה]}\צהגדרה{. }

\משנה{הנטיה לצד פרישות וקדושה}\הגדרה{ - (}\משנה{עניינה}\הגדרה{) - הגנה על הנפש מנטיה קצונית להפקרות ולטמיעה בתוך חיים חמריים }\מקור{[עפ״י ע״ר א קעה]}\צהגדרה{.}

\משנה{פרישה }\הגדרה{- הכשרה של החיים הטבעיים אל ההסתגלות של הקודש\hebrewmakaf העליון\mycircle{°} }\מקור{[ע״א ד ט כח]}\צהגדרה{. }

\paragraphs

\ערך{פרעות }\הגדרה{- חסרון הסדרים וזעזוע ההרגשות }\מקור{[ע״א ב ט קכה]}\צהגדרה{. }

\paragraphs

\ערך{פשטות עליונה }\הגדרה{- האחדות\hebrewmakaf המוחלטה\mycircle{°}, יסוד העדן\hebrewmakaf העליון\mycircle{°} }\מקור{[ר״מ קג]}\צהגדרה{. }

\paragraphs

\ערך{פשע}\myfootnote{ בסדור הגר״א, קלג: אמרי שפר ״והנה העון הוא מזיד והיינו כי יודע שדבר זה אסור ומ״מ עובר עליו. ולא שמתכוון למרוד דא״כ היינו פשע. אלא שמתכוון להנאתו, אם בשביל תאוותו וחמדת ממון או כבוד או שאר הנאת הערב או המועיל״. במגיד משרים, פרשת מקץ, מהדורא בתרא, יחס החטא לנפש, עון לרוח ו״פשע למאן דחב להכעיס, ההוא גברא טניף נשמתיה״. ובחסד לאברהם, מעין חמישי, עין משפט, נהר טז ״והענין שיש חטא ועון ופשע, על החטאים הנפש מתגלגלת, העונות הרוח מתגלגלת, ועל הפשעים נשמה מתגלגלת, והיינו ובפשעכם שולחה אמכם, פשע הגיע למעלה שפגם באם עליונה״. ע״ע י׳ מאמרות לרמ״ע, חקו״ד ח״ד יד, וביד יהודה ס״ק ג. ושם ח״ה ג.\label{6}}\הגדרה{ - ענין שאיננו מוסיף כלל על גלוי המפעל יותר מעוון\mycircle{°}, אלא בחוג הפנימי\mycircle{°} של הנשמה יש בזה ציור\mycircle{°} יותר מדאיב, המביא לידי התפרצות כזאת, שהיא מכשירה את המקולקל בזה לידי עברינות\mycircle{°} גם במקום שאין המית הנטיה החמרית\mycircle{°} גורמת. ״פשעים אלו מרדים״ }\מקור{[ע״ר א קכז]}\צהגדרה{. }

\הגדרה{ההתקוממות על החוקים היותר רחוקים במטרתם מחוש ההרגשה של האדם }\מקור{[עפ״י ע״א ד ה לו]}\צהגדרה{. }

\משנה{תכונת הפשע }\הגדרה{- חפץ המרידה ופריקת עול הקודש\mycircle{°}, מתוך כעסנות הדיוטית, הבאה מתוך הריחוק הנורא של האדם, מצד עובי חומרי\mycircle{°}, מהאצילות\mycircle{°} הקדושה\mycircle{°} של הוד\mycircle{°} החיים, המוארים מאור\hebrewmakaf ד׳\mycircle{°} אשר במכמני המשפטים הקדושים }\מקור{[ע״ר ב ס]}\צהגדרה{. }

\הגדרה{ע״ע חטא. ע״ע עברה. ע״ע סליחה. ע׳ במדור מדרגות והערכות אישיותיות, ״שבי פשע ביעקב״. ושם, פושע. ע׳ בנספחות, מדור מחקרים, מחילה סליחה וכפרה. }

\paragraphs

\ערך{פשתן}\הגדרה{ - }\משנה{סגולתו הפנימית}\myfootnote{ \textbf{סגולת הפשתן} - ע״ע חסד לאברהם, מעין רביעי, עין יעקב, נהר נב.\label{7}}\הגדרה{ - הארת כח החיים בהחטיביות הפרטית, ושמירת הפרט באור נשמתו מהתערובות של פרטי הכלל, שהם מחשיכים את אור הקודש מצד מאפל הפרטיות שלהם }\מקור{[ר״מ קכט]}\צהגדרה{.}

\הגדרה{ע״ע בוץ.}

\paragraphs

\ערך{פת }\הגדרה{- האמצעי העקרי שעמו קונים את החיים }\מקור{[ע״א ב ז מא]}\צהגדרה{. }

\הגדרה{ע״ע לחם.}

\paragraphs

\ערך{״פתוח״ }\הגדרה{- מפורש }\מקור{[ע״א ד יב יט]}\צהגדרה{. }

\הגדרה{ע״ע גלוי. ע׳ במדור פסוקים ובטויי חז״ל, מאמר פתוח. ע״ע ״סתום״.}\mylettertitle{צ}

\paragraphs

\ערך{צבאות ד׳\mycircle{°}}\הגדרה{ - ניצוצות\hebrewmakaf הקודש\mycircle{°} השייכים בכל יום לעומק החפץ הטוב של האדם }\מקור{[א״ק ג רכה]}\צהגדרה{. }

\paragraphs

\ערך{צביון }\הגדרה{- רוח ותכונה }\מקור{[עפ״י ע״א ב ט רלא]}\צהגדרה{.}

\הגדרה{התוכן היסודי, הכלי\mycircle{°} הקיים בצורתו\mycircle{°}, קביעות רצון פנימי ואופי עצמי }\מקור{[עפ״י קובץ ו קצג]}\צהגדרה{. }

\צהגדרה{התאר הפנימי הספוג (בבעל הצביון) }\מקור{[עפ״י ע״ר א לו]}\צהגדרה{.}

\paragraphs

\משנה{צבע }\צהגדרה{- גילוי חיצוני של הפנים }\צמקור{[שי׳ ה 265]. }

\הגדרה{ע״ע גוון. }

\paragraphs

\ערך{צדדיות }\הגדרה{- הרישום של היחש של כל נושא בכל ערכיו אל מה שחוץ לו }\מקור{[ר״מ כא]}\צהגדרה{. }

\הגדרה{הערך שבין תוכן לתוכן מצד חצוניותו והתפרטותו }\מקור{[שם צח]}\צהגדרה{. }

\paragraphs

\משנה{״צדיק}\הגדרה{״ }\צהגדרה{- }\צהגדרהמודגשת{יסוד הקשור הנפשי אל אישיות גדולה, של ״צדיק״, של ״רבי״ }\צהגדרה{- משען נפשי אל אישיות גדולה ונאדרה\hebrewmakaf בקדושה, למען החזקת עמדתו הרוחנית של האדם\hebrewmakaf מישראל, מתוך כל שלמות\hebrewmakaf קדושתה וחיוניות\hebrewmakaf תקפה של תורה, בהמשך התגלות דבר\hebrewmakaf ד׳ על ידי תופשיה ועוסקיה הנאמנים }\צמקור{[עפ״י ל״י ב (מהדורת בית אל תשס״ג) תו].}

\הגדרה{ע׳ במדור תורה, שמוש תלמידי חכמים. ושם, ״גדולה שמושה של תורה יותר מלימודה״.}

\paragraphs

\ערך{צדיקים }\הגדרה{- }\משנה{יסוד נשמת הצדיקים המתהלכים\hebrewmakaf לפני\hebrewmakaf האלהים\mycircle{°}, ומתענגים\hebrewmakaf על\hebrewmakaf ד׳\mycircle{°}}\הגדרה{ - התביעה להיות תמיד נתון ביסוד\hebrewmakaf הכללי\mycircle{°}, ב׳צרורא דלעילא דביה חיי כולא׳\mycircle{°} }\מקור{[עפ״י א״ק ג קמז]}\צהגדרה{.}

\paragraphs

\ערך{צדק}\הגדרה{ - }\מעוין{◊ }\הגדרה{אותה השאיפה הכללית היותר נעלה <המוצאת מקום בנפשות היותר נבחרות שבמין האנושי, שהיא העמילן\mycircle{°} האנושי, כח המקיים המחזיק והמעלה>, אותה השאיפה ההרגשית לעבודת הכלל, לטובו העתיד}\צהגדרה{ }\מקור{[א״ה (מהדורת תשס״ב) ב 81 (א״ב ג)]}\צהגדרה{.}

\מעוין{◊ }\הגדרה{השאיפה הכללית לעתיד, היסוד הנושא בכוחו את כל הערך המוסרי היותר רם, הכולל בתוכו את עליית הנשמה האנושית}\צהגדרה{ }\מקור{[א״ה (מהדורת תשס״ב) ב 80 (א״ב ג)]}\צהגדרה{.  }

\משנה{מדת הצדק }\הגדרה{- המוסר\mycircle{°} והקדושה\mycircle{°}}\צהגדרה{ }\מקור{[ע״א ד יב מא]}\צהגדרה{.}

\משנה{כח הצדק }\הגדרה{- כח המגשם את הטוב\mycircle{°} ומגדירו }\מקור{[עפ״י ע״ר א קג]}\צהגדרה{.}

\הגדרה{ע״ע חסד, כח החסד. }

\ערך{צדק }\הגדרה{- שלא יהיה לשום אדם היזק בדבר }\מקור{[פנק׳ ה מ]}\צהגדרה{.}

\ערך{צדק }\הגדרה{- }\מעוין{◊}\הגדרה{ ציורי\mycircle{°} הצדק מבוטאים מצד היושר\mycircle{°} שמתכונן על ידם ביחש הצדדי שבין כל מצוי ומצוי, היחש השלם והתמים שבין כל נושא ונושא אל זולתו, בכל מערכות החיים וההויה }\מקור{[ר״מ קלה]}\צהגדרה{. }

\הגדרה{קיבוץ הצבעים\mycircle{°} כולם, בעריכת כל הכחות כולם על מכונם ועל סדרם, במקומם ושיעורם }\מקור{[שם כא]}\צהגדרה{. }

\הגדרה{ע״ע בנספחות, מדור מחקרים, צדק ואמת. }

\ערך{צדק}\myfootnote{ זוהר ח״ג נט:.\label{1}}\הגדרה{ - }\מעוין{◊ }\הגדרה{הנטיה הפנימית\mycircle{°} האדירה שבתוכיות ההויה, שהיא הסגולה\mycircle{°} הישראלית\mycircle{°} היותר פנימית\mycircle{°} ועליונה\mycircle{°} }\מקור{[עפ״י א׳ קסב, ע״ה קטו]}\צהגדרה{. }

\ערך{צדק }\הגדרה{- }\משנה{הצדק המופשט }\הגדרה{- }\מעוין{◊}\הגדרה{ עומד הוא למעלה מכל ערך של איזה מפעל ומעשה, כי הוא עטור בעטרת שלמות תפארתו\mycircle{°} האצילית\mycircle{°}. מתוך זהרו הברור יקבלו כל המעשים כולם בהערכתם\mycircle{°} המוגבלת, את צורת משפטם\mycircle{°} }\מקור{[ע״ר א לה]}\צהגדרה{. }

\ערך{צדק }\הגדרה{- }\משנה{הצדק המוחלט }\הגדרה{- נשמתה\mycircle{°} של כנסת\hebrewmakaf ישראל\mycircle{°}. בהתגשמותו כולל הצדק המוחלט את כל הטוב\mycircle{°} המוסרי\mycircle{°} שבפעל }\מקור{[עפ״י א״ש ד ז]}\צהגדרה{. }

\הגדרה{ע״ע בנספחות, מדור מחקרים, ״צדיק וצדק״. }

\paragraphs

\ערך{צדקה }\הגדרה{- }\מעוין{◊}\הגדרה{ הצדקה באה רק למלאות את חסרון העוני, ׳די מחסורו אשר יחסר לו׳ }\מקור{[לא׳ קע]}\צהגדרה{. }

\הגדרה{ע׳ בנספחות, מדור מחקרים, חסד לעומת צדקה. }

\paragraphs

\ערך{צדקה }\הגדרה{- כוללת בקרבה את כל ערכי המוסר\mycircle{°} והצדק\mycircle{°} המעשי, שממנו באות המדות הישרות וכל מעגל ישר\mycircle{°} }\מקור{[עפ״י ע״ר א כא]}\צהגדרה{. }

\paragraphs

\ערך{צדקוּת }\הגדרה{- הכשרה להישרות\mycircle{°} }\מקור{[מ״ר 378]}\צהגדרה{. }

\paragraphs

\ערך{צדקוּת }\הגדרה{- }\משנה{הצדקות האמיתית}\הגדרה{ - השקיקה האלהית\mycircle{°} הבלתי פוסקת, הפועלת תמיד בלא הרף במעמקי הנשמה\mycircle{°} }\מקור{[פנק׳ א תכב]}\צהגדרה{. }

\משנה{הצדקות היסודית של צדיק\hebrewmakaf יסוד\hebrewmakaf עולם\mycircle{°} }\הגדרה{- התביעה האלהית\mycircle{°} התדירה שברוחו\mycircle{°}, להיות קשור בכל חפצו, שכלו, רצונו, והרגשתו, לקונו, לצור\hebrewmakaf כל\hebrewmakaf העולמים\mycircle{°} מקור חיי כל החיים והויית כל ההוייות }\מקור{[פנק׳ ב רכז]}\צהגדרה{.}

\ערך{צדקוּת }\הגדרה{- }\משנה{כשרון הצדקות }\הגדרה{- הלב\hebrewmakaf הטוב\mycircle{°} וההשגה\mycircle{°} הבהירה הפנימית\mycircle{°} של צדיקים\hebrewmakaf יסודי\hebrewmakaf עולם\mycircle{°} }\מקור{[א״ק ג שיב]}\צהגדרה{.}

\paragraphs

\משנה{צהלה }\צהגדרה{- הגילוי החיצוני הבא לידי היכר גם ברננת\mycircle{°} הקול וגם בהארת הפנים}\myfootnote{ ע׳ רש״י שבת קמ. שהליבון של הבגדים מצהיל.\label{2}}\צהגדרה{, המביע את השמחה\mycircle{°} הפנימית }\צמקור{[ע״ר ב תלח].}

\הגדרה{ע׳ במדור נפשיות, גילה. ושם, עליצות. ר׳ שם, שמחה לעומת ששון. ר׳ שם, חדוה. }

\paragraphs

\משנה{צו אלהי }\צהגדרה{-}\צמשנה{ (העושה את האדם כשליח אלהי)}\צהגדרה{ - }\מעוין{◊ }\צהגדרה{לפי הכללת ההכרה, ברירות אמתותה ויציבותה, של הצורך והחפץ ללמד ולהשפיע, להיטיב ולגמול חסד, כשבא לא מצד שהוא יהיה העושה את זה, וגם לא באפשרות הרגשת עצמו בזה, אלא מצד התמלאותם של הדברים האלה כראוי - <כתבערת דבר\hebrewmakaf ד׳\mycircle{°}, המכוונת למסירת מצות שליחותו, שלא יוכל הנביא כַּלְכֵלהו}\myfootnote{ ירמיה כ ט: ״ואמרתי לא אזכרנו ולא אדבר עוד בשמו והיה בלבי כאש בערת עצר בעצמתי ונלאיתי כַּלְכֵל ולא אוכל״.\label{3}}\צהגדרה{ ומוכרח הוא בה להביעו> - היא מאירה לאדם את }\צהגדרהמודגשת{ערכו כמצווה ושלוח}\צהגדרה{ למענם בכונניות\mycircle{°} ההנהגה האלהית\mycircle{°} }\צמקור{[עפ״י א״ל רלו].}

\צמשנה{צווי פנימי }\צהגדרה{- }\מעוין{◊ }\צהגדרה{כפי מה שמורגש במעמקי\hebrewmakaf הלב הצורך לבטא דברי ההרגשות והמחשבות לעצמו ולאחרים, בתור צרך עצמי עניני, לא לשם בקשת תועלת הנאה ממשית או כבוד מדומה, כן מקבל צרך זה ערך של }\צהגדרהמודגשת{חיוב, }\צהגדרה{הנמשך מתוך }\צהגדרהמודגשת{צווי פנימי }\צהגדרה{המופיע בתכונת הנפש ודבק בה, והתעכבות מילואו מתקרבת לדוגמא של כבישת נבואה\mycircle{°}, בבחינת ניב דבריה ותפקיד קבלת שליחותה }\צמקור{[א״ל קמד]. }

\paragraphs

\ערך{צורה }\הגדרה{- המחשבה\hebrewmakaf העליונה\mycircle{°} העושה את (הנושא לנושא) }\מקור{[עפ״י ע״א ד ט יז]}\צהגדרה{. }

\מעוין{◊ }\הגדרה{אמתת שם צורה יש בה גדרים מחולקים, שצורת הבית הנראית לעין לא תקרא שם צורה עליה כ״א ע״ד מקרה והשאלה, אבל אמתת שם צורה היא הכח של הדבר הנותן קיום אל העצם,}\myfootnote{ מורה נבוכים ח״א א ״הצורה הטבעית וכו׳ הענין אשר בו נתעצם הדבר והיה מה שהוא, והוא אמיתתו, מאשר הוא הנמצא ההוא״.\label{4}}\הגדרה{ כמו כח החיים בחי, וכח הצמיחה בצומח, וכח הדיבור והשכל במדבר המשכיל\mycircle{°}, וכח המחבר היסודות בדומם, זהו עצם שם צורה. וזאת הצורה היא עיקר קיום הדברים שתנשא בהם, כשהיא מתחזקת יתחזקו על ידה הדברים שהיא נשואה בהם ותתחזק מציאותם, וברפיונה יהיו גם הם רפויים. והצורה הזאת היא באמת הכח המוציא את הנושא מן הכח אל הפועל, כי צורת הצומח מוציאה את הצמח וכח החיים את החי, להביאו אל מדרגתו במציאות }\מקור{[מ״ש צא (מא״ה ב רמג\hebrewmakaf ד)]}\צהגדרה{. }

\הגדרה{המטרה }\מקור{[עפ״י ע״ר א קד]}\צהגדרה{.}

\הגדרה{הנפש }\מקור{[פנק׳ ג רפא]}\צהגדרה{.}

\הגדרה{נשמה\mycircle{°} המחיה את הגויה }\מקור{[פנק׳ א תלד]}\צהגדרה{.}

\משנה{צורה (לעומת חומר\mycircle{°}) }\הגדרה{- כח פועל }\מקור{[ע״א ב ט רמ]}\צהגדרה{. }

\הגדרה{ע׳ בנספחות, מדור מחקרים, אידיאה. ושם, צלם, דמות, תבנית, צורה.}

\paragraphs

\ערך{צינור }\הגדרה{- ע׳ במדור מונחי קבלה ונסתר, צנור.}

\paragraphs

\ערך{צללים }\הגדרה{- הסיגים הרוחניים, מחשבות השקר והחולשה, החנופה והתאוה }\מקור{[עפ״י א״ק ג רמח]}\צהגדרה{. }

\paragraphs

\ערך{צלם }\הגדרה{- }\משנה{״צולמא״ }\הגדרה{- הסגנון }\מקור{[עפ״י ע״ה קכב]}\צהגדרה{. }

\הגדרה{ע׳ במדור מונחי קבלה ונסתר, ״דלא ליערב צולמיה בצולמא דארמאה״. }

\paragraphs

\ערך{צלם האדם האמיתי}\myfootnote{ \textbf{צלם }\textbf{אלהים} - להבחנה שבין ״צלם אלהים״ ל״צלם האדם״ ע׳ אור החיים עה״ת, בראשית א כז.\label{5}}\צהגדרה{ }\הגדרה{- המסקנות האחרונות (של האותיות}\צהגדרה{\mycircle{°}, מניות החיים העצמיים שבפנימיות ועצמיות}\הגדרה{), יסוד הכל המסקנא והתמצית }\מקור{[פנק׳ א סג\hebrewmakaf ד]}\צהגדרה{.}

\הגדרה{ע׳ במדור פסוקים ובטויי חז״ל, צלם אלהים.}

\paragraphs

\ערך{צלף }\הגדרה{- הצמח העומד למוסר על מסיגי גבול, שבו תיחם יהושע לישראל את הארץ}\צהגדרה{ }\מקור{[ע״א ג ב קי]}\צהגדרה{.}

\paragraphs

\ערך{צמאון אלהי}\myfootnote{ \textbf{דרישת השלמות העליונה, שאינה מתמלאת, ואינה צריכה להתמלאות} - ע״ע ע״ר א ר-רא ד״ה התהללו בשם קדשו.\label{6}}\הגדרה{ - הגעגועים\mycircle{°} ההומים לזוהר\mycircle{°} הצחצחות\mycircle{°} הנשמתיות\mycircle{°}, החפיצה הפנימית\mycircle{°} להתעלות\mycircle{°}, להשאב\mycircle{°} בחיי חיי מקור חי כל חיי העולמים\mycircle{°} }\מקור{[א״ק ג ס]}\צהגדרה{.}

\הגדרה{יסוד העריגה הנצחית, שתכונתה דרישת השלמות העליונה, שאינה מתמלאת, ואינה צריכה להתמלאות }\מקור{[עפ״י פנק׳ ב קצד (חד׳ יט)]}\צהגדרה{.}

\הגדרה{היסוד של מציאות ההכרח הפנימי, להיות האדם זקוק לאלהיו, להקשר בו בקשרי עבודה\mycircle{°}, הכוללים את כל היחוסים הרמים שבין יציר ליוצרו, שבין העולם לקונו }\מקור{[ע״א ד ט א]}\צהגדרה{.}

\הגדרה{אור\hebrewmakaf ד׳\mycircle{°}}\צהגדרה{ }\מקור{[עפ״י קובץ ז רח]}\צהגדרה{.}

\ערך{צמאון לאלהים }\הגדרה{- }\משנה{שבכל עומק נשמתנו }\הגדרה{- רוח פנימי המתמלא בקרבנו, האופי הטבעי האמוני, סגולה\mycircle{°} טבעית, הטבעיות הצמאונית הנשגבה והקדושה, הסבה הקרובה הנפשית להתגלות כח האמונה, חוש האמונה, וההפעמות האמונית, <שכשהיא מתמלאת בכל סגולותיה, בכל עצמתה וחילה, יצא מזה כלי התפארה של יראת\hebrewmakaf ד׳\hebrewmakaf הטהורה\mycircle{°}, הנהדרה והמפוארה> }\מקור{[קובץ ה קמח]}\צהגדרה{.}

\משנה{הצמאון האלהי, כליו }\הגדרה{- הארות\mycircle{°} המוסר\mycircle{°} המעשי והתכוני במהות הנפש }\מקור{[א״ש יב ג]}\צהגדרה{.}

\ערך{צמאון האלהי }\הגדרה{- }\משנה{השאיפה לצמאון האלהי }\הגדרה{- התשוקה הטבעית לאושר ולשלמות}\צהגדרה{ }\מקור{[פנק׳ ד שסח]}\צהגדרה{.}

\משנה{צמאון לאל חי }\צהגדרה{- הפניה כלפי מעלה, ההשתוקקות להופעת אורו וההתקשרות לגלוי כבודו, של תביעת מדות האמת והצדק, הטוב והנאה, מתוך ההתכונות אל רבון העולמים, המתגלה באומתנו הישראלית וחיי דורותיה וצבוריותה }\צמקור{[ל״י ב (מהדורת בית אל תשס״ג) סד].}

\הגדרה{ע״ע עריגה אלהית. ע״ע כמיהה.}

\paragraphs

\ערך{צמחני }\הגדרה{- }\משנה{כח הצמחני }\הגדרה{- יסוד הטבע שכח החיים מתגלה עליו באורו }\מקור{[ע״ר א קלב]}\צהגדרה{.}

\paragraphs

\ערך{צמצום }\הגדרה{- ע״ע מצומצם.}

\paragraphs

\ערך{צעקה }\הגדרה{- }\משנה{(צעקת הנשמה בפנימיות תהומה אל ד׳)}\הגדרה{ - }\מעוין{◊ }\הגדרה{באה מתוך הרגשת המכאוב וצער המחשכים, שהיא טבועה ושקועה בהם. מתוך הצער של המאסרים והשריה באפלה. מתוך עומק המחשך, שהנשמה מוצאת את עצמה קשורה בהגבלתו }\מקור{[ע״ר א קס]}\צהגדרה{.}

\הגדרה{ע״ע שועה.}

\paragraphs

\ערך{צער הגיהנם בעולם הזה }\הגדרה{- ע״ע גיהנם.}

\paragraphs

\ערך{צער עליון }\הגדרה{- }\משנה{הצער העליון }\הגדרה{- מקרה שהוא יוצא מחוץ לטבע בהנהגה הכללית. יציאה מסדר ההנהגה הטבעית. דבר מקרי שלא כסדר ההנהגה. חסרון בהנהגה הכוללת <הבא מצד כח הבחירה\mycircle{°} וסדרי עניינים אחרים שחייבו את ירידתם של ישראל\mycircle{°} לשעה>, והוא כמו חולי שולט בגוף\mycircle{°} ההנהגה }\מקור{[עפ״י ע״א א א ח]}\צהגדרה{. }

\הגדרה{צער העולם כולו וצערה של כנס״י\mycircle{°}, מחמת סיתום\mycircle{°} המאורות\hebrewmakaf העליונים\mycircle{°} ומיעוט שפעם\mycircle{°}, על לב כל בשר ועל המון גויים וממלכות בארץ מתחת ועל שריהם\mycircle{°} ומנהליהם ממעל }\מקור{[אג׳ א קיד]}\צהגדרה{. }

\הגדרה{צער\hebrewmakaf השכינה\mycircle{°}, צער המחשך הגדול של העולמים כולם וכל בריותיהם, מפני ההסתרה של האור\hebrewmakaf האלהי\mycircle{°} מהם}\צהגדרה{ }\מקור{[קובץ ח קצה]}\צהגדרה{.}

\הגדרה{ע׳ במדור פסוקים ובטויי חז״ל, שכינה, ״צער השכינה״. }

\paragraphs

\ערך{צרעת }\הגדרה{- }\מעוין{◊ }\משנה{(מופיע) ענין הצרעת }\הגדרה{- בעת הבוסר, שלא נגמרו הדעות להבין ערכם הנשגב של ישראל\mycircle{°} }\מקור{[ע״א ב ט ו]}\צהגדרה{.}

\משנה{סגירו}\הגדרה{ - }\משנה{(בלשון הזוהר)}\הגדרה{ - }\מעוין{◊ }\הגדרה{שסוגר את האור האלקי מלכנוס ולהאיר לחוץ, ע״כ מתגברת הקליפה\mycircle{°} מצד הגוף }\מקור{[פנק׳ ג פט]}\צהגדרה{.}\mylettertitle{ק}

\paragraphs

\ערך{קבוע }\הגדרה{- }\משנה{(לעומת מתנודד\mycircle{°}) }\הגדרה{- עומד בצביונו בלא שינוי ותמורה }\מקור{[א״ק א רעד]}\צהגדרה{. }

\הגדרה{ע׳ במדור פסוקים ובטויי חז״ל, קבע, ״תורתך קבע״. }

\paragraphs

\ערך{קבוץ }\הגדרה{- }\משנה{הקיבוץ }\הגדרה{- הכלל\mycircle{°} }\מקור{[קובץ א קלג]}\צהגדרה{. }

\paragraphs

\ערך{קבלה }\הגדרה{- }\משנה{מצב של קבלה }\הגדרה{- מצב של הכשרה להיות כלי\mycircle{°} לספג אל תוכיותו }\מקור{[ר״מ עג]}\צהגדרה{. }

\paragraphs

\ערך{קבלת לשון הרע }\הגדרה{- }\משנה{היסוד הכללי לקבלת לה״ר\mycircle{°} }\הגדרה{- הכח של היות נפעל מכח הרע\mycircle{°} החיצוני }\מקור{[ע״א ד ה סט]}\צהגדרה{. }

\הגדרה{ההכשר להיות מושפע ע״י הערה רעה מרעי\hebrewmakaf אדם\mycircle{°}, והנטיה להביט על העולם בעין סוקרת את הרע שבו, זה הוא היסוד לכחות של אבירות וקשיות לב. מוסדם של הכחות הרעים ועיקרם }\מקור{[שם סו]}\צהגדרה{. }

\paragraphs

\ערך{קדום }\הגדרה{- שמימי\mycircle{°} }\מקור{[ע״ר א פג]}\צהגדרה{. }

\הגדרה{ע׳ במדור פסוקים ובטויי חז״ל, מראש מקדם. }

\paragraphs

\ערך{קדוש }\הגדרה{- מובדל מכל ציור\mycircle{°} ורעיון\mycircle{°}, ומכל מצר והגבלה\mycircle{°}, כמכל שינוי ותמורה }\מקור{[ע״ר א ריב]}\צהגדרה{. }

\הגדרה{מופרש ונבדל למעלה מהשגה }\מקור{[לא׳ קפב]}\צהגדרה{. }

\הגדרה{מופרש ומובדל }\מקור{[עפ״י פנק׳ ג קב]}\צהגדרה{.}

\הגדרה{מובדל משאר דברים}\צהגדרה{ [פנק׳ ה קמט]}

\הגדרה{מובדל }\מקור{[מ״ש קכג]}\צהגדרה{. }

\הגדרה{מובדל ומופרש, והכונה שלם בתכלית השלימות }\מקור{[ע״א א ג מ]}\צהגדרה{. }

\הגדרה{נצחי ונשגב }\מקור{[עפ״י שם ג ב צט]}\צהגדרה{. }

\הגדרה{ע״ע מתקדש. ע״ע מקודש. }

\paragraphs

\ערך{״קדוש״}\הגדרה{ - ע׳ במדור שמות כינויים ותארים אלהיים, ״חסין קדוש״.}

\paragraphs

\ערך{קדוש }\הגדרה{- }\משנה{קדוש. טהור. טמא }\הגדרה{ - ע׳ בנספחות, מדור מחקרים.}

\paragraphs

\ערך{קדוש }\הגדרה{- }\משנה{הקדוש }\הגדרה{- זך המחשבה והרעיון\mycircle{°}, המתאחד בחושו הפנימי\mycircle{°} עם הטוב\mycircle{°} הרוחני, המלא כל }\מקור{[עפ״י א״ק א פד]}\צהגדרה{. }

\הגדרה{דבק עם האור\mycircle{°} הגדול, המרוה והמענג, המקדש והמטהר, שכל העולמים כולם שואבים מזיוו }\מקור{[קובץ ד צו]}\צהגדרה{. }

\הגדרה{מי שהשלים הכונה התכליתית שכיונה החכמה\hebrewmakaf העליונה\mycircle{°} ביצירתו }\מקור{[עפ״י מ״ש שמט (ה׳ קעה)]}\צהגדרה{. }

\הגדרה{איש טהור\mycircle{°}, אידיאלי\mycircle{°}, שהיושר\mycircle{°} והטוב המעשי והמוסרי\mycircle{°} הוא כל משוש חייו }\מקור{[עפ״י א״ק א קכד]}\צהגדרה{. }

\הגדרה{מי שכבר הגיע למעלה רמה, ששכלו שליט עליו תמיד גם בעת התעוררות ההרגשות, וזאת היא הקדושה\mycircle{°} במובנה האחד שהיא התעלות טבעית. ויש עוד מעלה: שהאדם כבר הפך טבעו לטוב כולו, ואז אין צריך כ״כ לשמירה שכלית, ועצם חומרו נמשך אל הטוב ולא יגורהו רע }\מקור{[ע״א א א קמח]}\צהגדרה{.}

\הגדרה{ע״ע קדושה (ענינה באדם)}\צהגדרה{. }

\paragraphs

\ערך{קדושה }\הגדרה{- }\משנה{יסוד ענין קדושה }\הגדרה{- הבדלה ורוממות }\מקור{[ע״א א א קלב]}\צהגדרה{. }

\paragraphs

\ערך{קדושה }\הגדרה{- דעת\hebrewmakaf ד׳\mycircle{°} ודרכיו\mycircle{°} }\מקור{[ע״ר א רד]}\צהגדרה{. }

\הגדרה{אהבת\hebrewmakaf ד׳\mycircle{°} מצד עצמות דרכיו\mycircle{°} העליונים הטובים\mycircle{°}, ואהבת הטוב והחסד\mycircle{°} }\מקור{[א״ה ב (מהדורת תשס״ב) 194]}\צהגדרה{.  }

\משנה{הקדושה האלהית }\הגדרה{- ההליכה בדרכי\hebrewmakaf ד׳\mycircle{°} ועשות הטוב\hebrewmakaf האמיתי\mycircle{°} }\מקור{[ע״א ג א טו]}\צהגדרה{. }

\משנה{כללות הקדושה בכל גווניה }\הגדרה{- השמיעה\hebrewmakaf לקול\hebrewmakaf ד׳\mycircle{°}, ההתמסרות לרצון אדון\hebrewmakaf כל\mycircle{°} }\מקור{[ע״ר א צו]}\צהגדרה{. }

\משנה{נקודת הקדושה }\הגדרה{- הנקודה הראשית שבמציאות, מרכז החיים של כל המצוי, חיי הדעה\mycircle{°} והוד\mycircle{°} האמת\mycircle{°} }\מקור{[עפ״י ע״א ד ד ד]}\צהגדרה{. }

\משנה{קדושה }\הגדרה{- ד׳\mycircle{°} בכבודו\mycircle{°} }\מקור{[ע״ר א קפז]}\צהגדרה{. }

\הגדרה{השאיפה הבלתי גבולית לצד האור\mycircle{°} והטוב\mycircle{°}. מקור הישע\mycircle{°} והחיים, יסוד כל האושר\mycircle{°} לכל נשמה\mycircle{°} }\מקור{[ע״א ד יב כו, כט]}\צהגדרה{. }

\הגדרה{היסוד האידיאלי האלהי המטהר את גסות\mycircle{°} הבהמיות }\מקור{[ע״א ד ח לג]}\צהגדרה{. }

\הגדרה{עליית השמים\mycircle{°}, התגלות האלהות\mycircle{°} בעולם ובחיים, במדע וברגש, ביצירה ובתולדה }\מקור{[קובץ ב קפ]}\צהגדרה{. }

\הגדרה{תכונה מודדת את האצילות\mycircle{°} הנפשית ומרוממת אותה במעלות העליונות ששם הכל הוא בנין ושכלול, אין פרץ ואין יוצאת, אין מקדיח תבשילו ברבים ואין יוצא מן השורה הישרה, עד שהמדידה הקדושה מביאה את גאולת האמת\mycircle{°} שלמעלה מכל גבולים\mycircle{°} }\מקור{[א״א 77\hebrewmakaf 76]}\צהגדרה{. }

\צהגדרה{עילאיות }\צמקור{[}\מקור{שי}\צמקור{׳ א 114].}

\paragraphs

\ערך{קדושה }\הגדרה{- }\משנה{(בדברי קדושה) מטרתה }\הגדרה{- להשריש באדם את הרוממות והכבוד שראוי לו שישער בלבו בכל דבר הנוגע לכבודו של מקום\mycircle{°} ב״ה, למען תחקק בלבבו האמת הנעלה שהוד האדם ותפארת נשמתו היא נערכת באותה המדה שכבוד ד׳ הוא מושג ומשוער בנפשו פנימה }\מקור{[ע״א ג ב טו]}\צהגדרה{. }

\paragraphs

\ערך{קדושה }\הגדרה{- }\משנה{(ענינה באדם) }\הגדרה{- שהאדם מתעלה כ״כ עד שאינו צריך לשום עמל להתגבר על יצרו, כי\hebrewmakaf אם ממילא ע״י רבוי ההרגל במעשים טובים ובלמוד, הוא הולך בדרך טובה ורצויה, ודרך\hebrewmakaf ד׳\mycircle{°} הטובה קבועה היא בלבו יותר מכל חמדה חושית, ואז מאליו הוא הולך וחושב דרכי האמת, והעולה על לבבו יהיה הכל לדרך הטוב\mycircle{°} והישר\mycircle{°} }\מקור{[עפ״י ע״ר ב קמו (פנ׳ נא)]}\צהגדרה{. }

\הגדרה{שלא תהיה נפשו נפרדת כלל מקדושת\mycircle{°} הבורא ית׳ גם בעסקו בעניני החומר }\מקור{[מ״ר 276]}\צהגדרה{. }

\הגדרה{שכל הפעולות הן לתכלית טובה\mycircle{°} וקדושה, ואפילו הפעולות השפלות, כי אין בפעולות הקדוש\mycircle{°} פעולה שהיא לתכלית שפלה כלל }\מקור{[ע״ר א רעג (פנק׳ ג ערב)]}\צהגדרה{. }

\הגדרה{התרוממות הדברים השפלים למעלה רמה}\צהגדרה{ }\מקור{[מא״ה ג (מהדורת תשס״ד) רצה]}\צהגדרה{.}

\הגדרה{שגם הכוחות הטבעיים יצליחם השי״ת שיכוונו תמיד אל הטוב והקדוש }\מקור{[ע״א א ב ט]}\צהגדרה{. }

\הגדרה{שמהפך כל הדברים הגשמיים לקדושים }\מקור{[פנק׳ ה קה]}\צהגדרה{.}

\הגדרה{ההשלמה העליונה של היפוך כל טבעו לטוב במה שאין החושיות שולטת. להתרומם מעל לתאות המורגשות, באופן שלא יפעלו על האדם כלל, ואף בהיותו עסוק בהם לא ימשיכו לבבו כלל להיות נוטה אחריהם }\מקור{[ע״א א א קמח]}\צהגדרה{. }

\הגדרה{שאין עניני עוה״ז לו ענינים הכרחיים לבד, כי אם ענינים מעולים, והיינו שאין צריך להשתמש בהם לשם אמצעי בלתי נכבד, כי אם גבר כל כך בעוצם קדושתו עד שהפכם לאמצעי נכבד, מתאים עם ענין התכלית בעצמו, לא סותרו }\מקור{[מ״ש רמו]}\צהגדרה{. }

\הגדרה{ע״ע קדוש. }

\מעוין{◊ }\משנה{הקדושה, עקר תוכן מעלתה }\הגדרה{- שיתרומם האדם לשום כל מגמתו ומעיניו לא לצרכי עצמו, כי\hebrewmakaf אם לכבודו של אדון\mycircle{°} כל העולמים\mycircle{°} ית׳ וקדושתו\mycircle{°} והשלמת רצון קונו. ההתרוממות אל הטוב\mycircle{°} והישר\mycircle{°} מצד עצמו כפי האמת המבוררת }\מקור{[ע״ר א רעא\hebrewmakaf ב]}\צהגדרה{. }

\הגדרה{שיתרומם האדם אל רוממות השלימות, לשום כל מגמתו לא צרכי עצמו כ״א כבודו ית׳ של אדון כל העולמים וקדושתו. ההתרוממות אל הטוב\mycircle{°} והישר\mycircle{°} מצד עצמו כמו שהשכל גוזר }\מקור{[עפ״י ע״א א ג מ]}\צהגדרה{. }

\הגדרה{ההתמסרות אל הענין הרוחני\mycircle{°} הטהור\mycircle{°} במלא טהרתו, בתוכן מסולק מכל עכירות וזוהם רצוני משופל }\מקור{[ע״ר א ח]}\צהגדרה{. }

\הגדרה{רום המעלה של האצילות\mycircle{°} האידיאלית\mycircle{°}, שהיא מכוונת רק אל הטוב\hebrewmakaf הגמור\mycircle{°} והישרנות המוחלטת }\מקור{[קובץ ב א]}\צהגדרה{.}

\משנה{הקדושה שבחיים }\הגדרה{- ההתגשמות המעשית של החיים הרוחניים }\מקור{[א״ק א קצו]}\צהגדרה{. }

\הגדרה{ע״ע קדושת החיים. ע׳ במדור משכן ומקדש, ״כהנים אוכלים ובעלים מתכפרים״.}

\משנה{תכלית הקדושה }\הגדרה{- לבא אל מקום\mycircle{°} מדרגות\mycircle{°} הכונה הטובה\mycircle{°} שכיון השי״ת\mycircle{°} ביצירתו }\מקור{[מ״ש שמט (ה׳ קעה)]}\צהגדרה{. }

\משנה{הקדושה העליונה }\הגדרה{- הכרת ערך איכות התכלית העליונה, של כללות ההנהגה העליונה }\מקור{[ע״ר א שח]}\צהגדרה{. }

\הגדרה{רוממות נפש וגדולת קדש\mycircle{°} }\מקור{[שם רכ]}\צהגדרה{. }

\משנה{הקדושה באדם }\הגדרה{- באה מהאור\mycircle{°} הגדול הזורח\mycircle{°} אל בתי נפשו בהכרה גדולה, עמוקה\mycircle{°} ואדירה בתוכן החיים ומגמתו\mycircle{°}, אז כל רגש בריא גופני שמתעורר לחיים ומתפשט בכוחו, הוא הוספה ליסוד האורה\mycircle{°}, השתכללות לבנין האושר\mycircle{°}, הטוב\mycircle{°}, היושר\mycircle{°} והצדק\mycircle{°} }\מקור{[שם ב רנט]}\צהגדרה{. }

\משנה{הקדושה}\הגדרה{ - קשור בהירות השכל, עדינות הרצון, טוהר המוסר וזיקוק המדות בחטיבה\mycircle{°} אחת }\מקור{[עפ״י קובץ ה קכג]}\צהגדרה{.}

\משנה{הקדושה העליונה }\הגדרה{- השאיפה\hebrewmakaf האצילית\mycircle{°} הבאה אחרי שהשכל\mycircle{°} מתמזג עם הרצון\mycircle{°}, עד שאין להם כלל אותו הטבע ואותה התכונה שיש לכל יסוד מהם בפני עצמו, אלא אור\mycircle{°} של איזה מה חדש יוצא ממיזוגם בצביון של דעת\mycircle{°} המלא תפארת\mycircle{°} }\מקור{[א״ק ג פה]}\צהגדרה{. }

\משנה{עוצם הקדושה }\הגדרה{- הגברת התשוקה האלהית, ואור הרוחני, ברצון האדם וטבעו, עד שהחפץ של הקודש והמגמה האלהית, יהיה יותר עמוק בנפש, מכל הרצונות הטבעיים }\מקור{[קובץ א שכט]}\צהגדרה{. }

\משנה{כח הקדושה }\הגדרה{- הכח הכללי שהוא שורש הקדושה בנפש האדם, הוא הקישור הפנימי, כ״א לפי ערכו, בדביקות\hebrewmakaf האלהות\mycircle{°}. והמושג הכללי הזה הוא האב לכל החכמות, וכל המעשים הטובים\mycircle{°}, וכל התורה\mycircle{°} כולה, וכל לימודי השכל\mycircle{°} והיושר\mycircle{°} כולם }\מקור{[פנ׳ י]}\צהגדרה{. }

\משנה{הקדושה האמיתית}\הגדרה{ - השלמת רצון קונו }\מקור{[ע״א א ג מ]}\צהגדרה{. }

\משנה{קדושה }\צהגדרה{- }\צמשנה{מהותה }\צהגדרה{- החשיבות הרוחנית שלפני האדם, ובאה ומאירה לו מלמעלה ממנו }\צמקור{[צ״צ קסו].}

\משנה{קדושה }\צהגדרה{- גבורת מציאות ושלמות מציאות היותר גדולה }\צמקור{[שי׳ ג 182]. }

\צהגדרה{הקיום המלא והמשוכלל של המציאות }\צמקור{[א״ל רעו]. }

\צהגדרה{המציאות בשלמות תוקפה, בנצחיותה. מה שנמשך ממקור החיים }\צמקור{[מה״ה א ה].}

\צהגדרה{איחוד שמים וארץ, גשר אלוהי, ״סֻלם מֻצב ארצה וראשו מגיע השמימה״, נמשכת משמי מרומים עד הממשיות הקרקעית }\צמקור{[שי׳ ג 235].}

\צהגדרה{מדרגת מציאות, קיימוּת, שכלול מציאות, נצח ותוקף מציאות, גדלות ואמיתיות מציאות }\צמקור{[שי׳ ג 239].}

\הגדרה{ע׳ בנספחות, מדור מחקרים, טהרה וקדושה ההבדל ביניהן. }

\paragraphs

\ערך{קדושה }\הגדרה{- }\משנה{הקדושה העליונה (באדם) }\הגדרה{- כשרון גאוני מוטבע ביסוד החיים ושורשי הנשמה\mycircle{°}, שהוא יסוד הצדקות המופלגה הרוחנית\mycircle{°} }\מקור{[עפ״י א״ק ב ש]}\צהגדרה{. }

\משנה{יסוד התגלות הקדושה בנשמת האדם }\הגדרה{- ההתלהבות הפנימית מאור\mycircle{°} הקודש\mycircle{°}, הצמאון\mycircle{°} התוכי לאור\hebrewmakaf ד׳\mycircle{°} }\מקור{[א״ק ג רט]}\צהגדרה{. }

\הגדרה{ע״ע קדושת הדומיה. }

\paragraphs

\ערך{קדושה }\הגדרה{- }\משנה{יסוד הקדושה }\הגדרה{- התאחדות הפרטים לכלל\mycircle{°} אחד }\מקור{[ע״א א א מו]}\צהגדרה{. }

\paragraphs

\ערך{קדושה }\הגדרה{- תוכן הברכה\mycircle{°}, התעלות, והבדלה מכל פחיתות ורשעה, מכל חשך וצמצום\mycircle{°} }\מקור{[עפ״י א״ק ב תקלד]}\צהגדרה{. }

\הגדרה{המדרגה היותר נעלה, (שבה) ראוי להיות החיים תכלית עצמית }\מקור{[עפ״י ע״ר א תלד]}\צהגדרה{. }

\משנה{הקדושה העליונה }\הגדרה{- }\מעוין{◊}\הגדרה{ הקדושה העליונה היא חפשית\mycircle{°} מכל מעצור של מצר וכל מועקה של איזו הגבלה\mycircle{°} והכבדה שעבודית לכל כח וכל חומר\mycircle{°} }\מקור{[עפ״י שם ח]}\צהגדרה{. }

\הגדרה{הבעת החיים השלמים בכל מילואם }\מקור{[ע״א ד ט כב]}\צהגדרה{. }

\משנה{רוח הקדושה }\הגדרה{- החביון\mycircle{°} האיתן והבטוח בצל\hebrewmakaf שדי\mycircle{°}, אל אלהי אמת }\מקור{[ע״ר א קיז]}\צהגדרה{. }

\משנה{אור\mycircle{°} הקדושה }\הגדרה{- עדן\mycircle{°} רוח\hebrewmakaf אלהים\hebrewmakaf חיים\mycircle{°} וזיו\mycircle{°} כבודו\mycircle{°}, בעצמת נעימת ענוגי\mycircle{°} קדשו\mycircle{°} }\מקור{[שם ב מט]}\צהגדרה{. }

\הגדרה{העולם\hebrewmakaf העליון\mycircle{°}, המוסריות\mycircle{°} הטהורה\mycircle{°}, שלטון היושר\mycircle{°}, שפעת הנשגב\mycircle{°} }\מקור{[א״ק א ערה]}\צהגדרה{. }

\paragraphs

\ערך{קדושה }\הגדרה{- }\משנה{הקדושה הגנוזה }\הגדרה{- המבט הפנימי החודר בעצמותם של כל המגמות הכלליות, אור\hebrewmakaf האלהי\mycircle{°} המתעלה מכל הגיון ורעיון, הקדושה הפנימית המסוקרת בנשמתנו שלא מצד חלק אחד של דעה והשגה, מצד ערך אחד של מציאות, בין ביחש הערך, בין ביחש הזמן והמקום, כ״א הכל\hebrewmakaf יוכל\hebrewmakaf וכוללם\hebrewmakaf יחד\mycircle{°} }\מקור{[ע״א ד ו יח (ח״פ מד: מה.)]}\צהגדרה{. }

\הגדרה{כל הכחות הנפלאים האלהיים שהם נעלמים וכמוסים מאד, הבאים ומופיעים באורם\mycircle{°} העליון מעטרים ומעלים את פנימיות\mycircle{°} הערך הקדוש של הכלל ושל הפרט, של האומה\mycircle{°} ושל העולמות\mycircle{°} כולם }\מקור{[עפ״י ע״ר א קכט]}\צהגדרה{. }

\משנה{קדושה גלויה וקדושה גנוזה }\הגדרה{- ע׳ בנספחות, מדור מחקרים. }

\paragraphs

\ערך{קדושה }\הגדרה{- }\משנה{קדושה טבעית }\הגדרה{- יסוד עריגת האמונה\mycircle{°} על פי מקוריותה הנפשית }\מקור{[א״א 115 (ע״ר ב שפד)]}\צהגדרה{. }

\paragraphs

\ערך{קדושה }\הגדרה{- }\משנה{יחש הקדושה לשמו ית׳ }\צהגדרה{- }\הגדרה{שהוא מובדל ומופרש, והכונה שלם בתכלית השלימות. <ע״כ אין לו יחש לנבראים, כי בהיותם אפשרי\mycircle{°} המציאות לא ישוו כלל אל ערך מחויב\hebrewmakaf המציאות\mycircle{°} יתעלה> }\מקור{[ע״א א ג מ]}\צהגדרה{. }

\paragraphs

\ערך{קדושה }\הגדרה{- }\משנה{הקדושה התלויה בשם השם ית׳ }\הגדרה{- דבר נעלם (ו)אינו מושג, והוא נקרא על ישראל ומורה על דבקותם הגדולה בשם ית׳ בזו המעלה הנעלמת שאינה תלוי׳ בשום בחירה\mycircle{°} כלל }\מקור{[עפ״י מא״ה ג רד]}\צהגדרה{.}

\הגדרה{ע״ע שם ד׳ המיוחד לישראל.}

\paragraphs

\ערך{קדושת האכילה }\הגדרה{- }\משנה{(עניינה של האכילה המקודשת)}\הגדרה{ - ע׳ במדור מצוות, הלכות, מנהגים וטעמיהן, אכילה מקודשת, (עניינה).  }

\paragraphs

\משנה{קדושת הארץ }\צהגדרה{- קדושת הכלל\mycircle{°} }\צמקור{[שי׳ ה 212]. }

\paragraphs

\ערך{קדושת הדומיה}\myfootnote{ \textbf{קדושת }\textbf{הדומיה} - ע׳ היכל הברכה, דברים דף לה:.\textbf{אם יפיל עצמו לעבודה מצומצמת וכו׳ יסבול וידוכא וכו׳ כל הדרכים יחד הנם לפניו פתוחים} - ע״ע מ״ר 432, ע״ט טז, ד״ה ישנם, מא״ה ב רפא, ע״ע ע״א ב ז לז, א״ק ג שה, ובהיכל הברכה, ויקרא דף עט: ובאהבת\hebrewmakaf יהונתן לר״י אייבשיץ, פו..\label{1}}\הגדרה{ - הקדושה העליונה, קדושת ההויה שהאדם מכיר את עצמו בטל בפנימיותו הפרטית, וחי חיים כלליים, חיי כל. מרגיש הוא חיי הדומם, הצומח והחי, חיי הכלל כולו, של כל מדבר, של כל איש מאישי האדם, חיי כל שכל וכל מכיר, כל משיג וכל מרגיש, וההויה כולה מתעלה עמו למקורה, והמקור מתגלה תמיד עליה ועליו ברוב הדר, בהוד קדושה באמת ובנחת. כל האושר, כל הטוב והיושר, כל העז\mycircle{°} והתפארת\mycircle{°}, כל החיל והגבורה שופעים עליו, אורו\hebrewmakaf של\hebrewmakaf עולם\mycircle{°} הוא, יסודו ואומץ המשכת חייו, בזכותו נזון העולם כולו, וכאין וכאפס הוא בעיניו. הוא אינו מתקדש\mycircle{°}, נבדל ונפרש, חי הוא וכל חייו קודש קדשים, חיי חיים הם. דופקי לבבו, מרוצת דמו, שאיפות נפשו, הסתכלותו ומבט עיניו, הכל חיי\hebrewmakaf אמת\mycircle{°}, חיי גבורה אלהית שוטפים בהם ועל ידם. אם יפיל עצמו קדוש הדומיה לעבודה מצומצמת, בתפילה\mycircle{°}, בתורה\mycircle{°}, בצמצום מוסריות\mycircle{°} ודיקנות פרטיות, יסבול וידוכא, יחוש כי נשמה מלאה כל היקום לוחצים בצבתים, להסגירה במועקה מצומצמת של מדה, של התוית דרך מיוחד, בשעה שכל הדרכים יחד הנם לפניו פתוחים, כולם מלאים אור, כולם אוצרים חיים }\מקור{[א״ק ב רצז]}\צהגדרה{. }

\paragraphs

\ערך{קדושת החיים }\הגדרה{- }\משנה{החיים עם קדושתם }\הגדרה{- החיים שהאדם יתרומם עמהם לתענוגים שהם מכובדים גם נצחיים, למעשים שהם עומדים לעד וראויים להיות נחשקים מצד השכל\mycircle{°} והיושר\mycircle{°} הטהור\mycircle{°} }\מקור{[ע״א ג א יד]}\צהגדרה{. }

\paragraphs

\ערך{קדושת הטבע\mycircle{°}}\הגדרה{ - קשרי הסיבות בעלילותיהן כעצת אדון\hebrewmakaf עולם\mycircle{°} אשר לתבונתו אין חקר }\מקור{[קובץ ז לה]}\צהגדרה{. }

\paragraphs

\ערך{קדש}\הגדרה{ - ע״ע קודש.}

\paragraphs

\ערך{קדש חמור}\הגדרה{ - }\משנה{בהערכת הקדושה, השופעת על חיי החברה המוגבלים}\הגדרה{ - הקדש הזורם לרומם את החיים המוגבלים למעלה מטבעם, להעמיד אותם במחיצתו של הקדש\hebrewmakaf העליון\mycircle{°} ושאיפותיו }\מקור{[ע״ר א קעה]}\צהגדרה{. }

\paragraphs

\ערך{קדש קל }\הגדרה{- }\משנה{בהערכת הקדושה, השופעת על חיי החברה המוגבלים }\הגדרה{- הקדש השופע בשפעו להעמיד את סדרי החיים החברתיים בתכן של קדש, המשוער להם לפי ערכם }\מקור{[ע״ר א קעה]}\צהגדרה{. }

\הגדרה{ע׳ במדור משכן ומקדש, קדשים קלים.}

\paragraphs

\ערך{קהל }\הגדרה{- }\משנה{(לעומת עדה\mycircle{°}) }\הגדרה{- הקיבוץ }\מקור{[ע״א א 97]}\צהגדרה{. }

\הגדרה{ע׳ בנספחות, מדור מחקרים, צבור ושותפין.}

\paragraphs

\ערך{קודש }\הגדרה{- }\משנה{רגשי קודש }\הגדרה{- רגשי צדק\mycircle{°} מלאי חיים לעשות הטוב\mycircle{°} והישר\mycircle{°} }\צהגדרה{[ע״א א }\צמקור{2\hebrewmakaf 71}\צהגדרה{]. }

\משנה{התוכן של הקודש }\הגדרה{- המוסר\mycircle{°} וחפץ הטוב }\מקור{[א״ק א צ]}\צהגדרה{.}

\paragraphs

\ערך{קודש }\הגדרה{- הענין המבדיל דבר מדבר בהפלאה מיוחדת }\מקור{[ע״א ד ט קכד]}\צהגדרה{.}

\הגדרה{עצם השכל המובדל מחומר\mycircle{°} }\מקור{[פנק׳ ה קמט]}\צהגדרה{.}

\paragraphs

\ערך{קודש }\הגדרה{- }\משנה{צד הקודש }\הגדרה{- הדעת\mycircle{°} הברורה שמשתקפת בתוכן ההויה }\מקור{[קובץ א קעג]}\צהגדרה{. }

\הגדרה{הנשגב והטוב\mycircle{°} }\מקור{[קובץ ה קסז]}\צהגדרה{. }

\הגדרה{המוסר\mycircle{°} העולמי }\מקור{[א״ק ג יב]}\צהגדרה{. }

\הגדרה{שאיפת המוסר בעומק הגודל\mycircle{°} שלו, שאיננו מוצא חסיון כי אם באור\hebrewmakaf ד׳\mycircle{°} וישעו\mycircle{°} }\מקור{[קובץ ז קעה]}\צהגדרה{. }

\הגדרה{האידיאליות\mycircle{°} העליונה הראויה להיות נכספת מכל צדיקים, מכל ברי לב }\מקור{[מ״ר 406 (פנק׳ א תלח)]}\צהגדרה{. }

\paragraphs

\ערך{קודש }\הגדרה{- אור\hebrewmakaf העולם העומד למעלה מהטבע\mycircle{°}, מהגופניות\mycircle{°} והחברתיות }\מקור{[א״ק ב שיז (ע״ט קיד)]}\צהגדרה{. }

\paragraphs

\ערך{קודש }\הגדרה{- }\משנה{מדת הקודש }\הגדרה{- השפעתה\mycircle{°} של תורה\mycircle{°} מצד הקדושה\mycircle{°} שנתגלה לנו }\מקור{[ע״ר א קלח]}\צהגדרה{. }

\הגדרה{ע״ע קודש\hebrewmakaf קדשים. }

\תערך{קודש }\הגדרה{- }\תהגדרה{דבר\hebrewmakaf ד׳\mycircle{°} }\תמקור{[מ״ר 490]. }

\paragraphs

\ערך{קודש, אור קודש}\הגדרה{ - }\משנה{בו מלאה הנשמה }\הגדרה{- מאויי נצח והוד עליון ושאיפת חיים אצילית לאין חקר }\מקור{[ע״ר א סח]}\צהגדרה{.}

\ערך{קודש, אור קודש }\הגדרה{- }\משנה{אדם המלא אור קודש }\הגדרה{- אדם שרצון עצמו הוא טהור\mycircle{°} ואידיאלי\mycircle{°} }\מקור{[ע״ט קכד]}\צהגדרה{. }

\paragraphs

\ערך{קודש}\הגדרה{ - }\משנה{תנועת קודש}\הגדרה{ - תנועת אמת\mycircle{°}, מלאה עז\mycircle{°} וגבורה\mycircle{°} }\מקור{[קובץ ו רמה]}\צהגדרה{.}

\paragraphs

\ערך{קודש }\הגדרה{- }\משנה{(לעומת חול\mycircle{°}) }\הגדרה{- כל הדברים בהנהגת העולם, המביאים לשכלל את הכח המוסרי\mycircle{°} במציאות, לרומם את הכרת האמת ולקדש את הדרכים ואת המדות }\מקור{[עפ״י ע״א א 146]}\צהגדרה{. }

\הגדרה{הַמְכוּוָן }\מקור{[אג׳ ג מא]}\צהגדרה{. }

\הגדרה{מְכוּוָן אידיאלי\mycircle{°}, מראשית\mycircle{°} עד אחרית\mycircle{°} }\מקור{[אג׳ ג מב]}\צהגדרה{.}

\הגדרה{צורת\mycircle{°} החול }\מקור{[א״ק א קמה (מ״ר 400)]}\צהגדרה{. }

\הגדרה{המגמה\mycircle{°} של החול, תעודתו ותכליתו }\מקור{[ע״א ד ט כו]}\צהגדרה{. }

\הגדרה{תכלית החול, המנוחה\hebrewmakaf השלמה\mycircle{°}, שהנשמה האלהית מופיעה באדם בכל הדרה\mycircle{°} בעת תגבורת הקודש\mycircle{°}, ואז הוא מכיר כי הוא חי חיי\hebrewmakaf אמת\mycircle{°}, חיים של נועם\hebrewmakaf ד׳\mycircle{°} וההתענג\mycircle{°} מזיו\mycircle{°} כבודו\mycircle{°} }\מקור{[שם ג ב סט]}\צהגדרה{. }

\הגדרה{כל מה שבא אל תכלית הכונה העליונה במציאותו וכבר קשור הוא בתכליתו, (ש)קדוש\mycircle{°} יאמר לו }\מקור{[עפ״י מ״ש שמט (ה׳ קעה)]}\צהגדרה{. }

\משנה{אור הקודש }\הגדרה{- דגל שם\hebrewmakaf ד׳\mycircle{°} וקדושת לבת קודש של אור התורה\mycircle{°} ואמונה\mycircle{°} גדולה\mycircle{°} וקדושה\mycircle{°}, כמו שהיא חקוקה על לוח לבם ונשמתם של כללות שומרי אמונים }\מקור{[אג׳ ג קעו]}\צהגדרה{.}

\משנה{הקודש העליון }\הגדרה{- האידיאליות\hebrewmakaf האלהית\mycircle{°} החיה במילואה בכל פינה ונקודה, של חיים, זמן, ומקום, ומקוריהם ותולדותיהם לאין תכלית בגודל ובקוטן. זהרי האורה\mycircle{°} של החיים הנחמדים והחביבים, שמנועם\hebrewmakaf ד׳\mycircle{°} המה מפכים והולכים במלא עולמים, במלא נשמות ובמלא חיינו פנימה}\צהגדרה{ }\מקור{[עפ״י אג׳ ג לה (ושם רסו, ובע״ר ב רנה)]}\צהגדרה{.}

\צהגדרה{פנימיות\mycircle{°} החיים }\צמקור{[שי׳ ד 58].}

\הגדרה{ע״ע כנסת ישראל הניסית.}

\paragraphs

\ערך{קודש }\הגדרה{- }\משנה{יסוד הקודש במילואו}\הגדרה{ - שלמות הקדושה הישראלית העורגת אל חייה השלמים, כפי המבוטא בדברי הנביאים ורוח הקודש שבכתובים ושבדברי חז״ל באמת ותמים, וכן הדברים חרותים על הלב הכללי של האומה}\צהגדרה{ }\מקור{[אג׳ ג קעג]}\צהגדרה{.}

\paragraphs

\משנה{קודש }\צהגדרה{- }\צמשנה{שם תאר }\צהגדרה{- חטיבה\mycircle{°} חיה שלמה }\צמקור{[צ״צ צה\hebrewmakaf ו].  }

\paragraphs

\ערך{קודש }\הגדרה{- אידיאליות\mycircle{°} של אור\hebrewmakaf ד׳\mycircle{°}, של טוב\hebrewmakaf העליון\mycircle{°} }\מקור{[ע״ר א שעב]}\צהגדרה{. }

\משנה{קֹדש }\צהגדרה{- האצילות\mycircle{°}, מקור כל ההויה }\צמקור{[ח״ר 34]. }

\משנה{מרום הקודש }\הגדרה{- אור שם\hebrewmakaf ד׳\mycircle{°}, חי\hebrewmakaf העולמים\mycircle{°} }\מקור{[א״ק ג רט]}\צהגדרה{. }

\משנה{קודש עליון }\הגדרה{- שם\hebrewmakaf ד׳ במילואו\mycircle{°}. הופעת\mycircle{°} העצמיות\mycircle{°}, לשד היש\mycircle{°}, מעומק ההויה, משורש פנימיותה\mycircle{°}, המקור שמשם ההתהוות הנשמתית, אושר\mycircle{°} הנשמות. שורש המקוריות\mycircle{°} העליונה\mycircle{°} }\מקור{[עפ״י שם ב תצה]}\צהגדרה{. }

\הגדרה{אור\hebrewmakaf ד׳ וכבודו\mycircle{°}. קודש\hebrewmakaf הקדשים\mycircle{°} }\צהגדרה{[מ״ר }\צמקור{345, 399, 405}\צהגדרה{]. }

\הגדרה{אור\hebrewmakaf החיים\mycircle{°} }\מקור{[ע״א ד ח לג]}\צהגדרה{. }

\הגדרה{זוהר\mycircle{°} הנוצץ מאור\hebrewmakaf אין\hebrewmakaf סוף\mycircle{°}}\צהגדרה{ }\מקור{[קובץ א שט]}\צהגדרה{.}

\משנה{אור הקודש העליון }\הגדרה{- ההופעה האלהית הכוללת כל, שמתוכה יוצאים ענפים מלאי עז חיים, וחסינות\mycircle{°} הויה\mycircle{°} ומציאות\mycircle{°}, היורדים לתוך המורד של כל תהלוכות החיים, שמתגלגל בהם האדם, שחייו קשורים להקדושה\hebrewmakaf האלהית\mycircle{°} }\מקור{[עפ״י ע״ר ב עז]}\צהגדרה{. }

\משנה{הקודש }\הגדרה{- מקור החיים והיש האלהי <שהכל בא מצד הופעתו, }\צהגדרה{מצד הסתעפות של חיים וישות ממנו}\הגדרה{> }\מקור{[עפ״י א״ק א ב (מ״ר 401)]}\צהגדרה{.}

\הגדרה{הישות הכבירה והאיתנה, חיי החיים בעוצם מקוריותם, והזרמת שפעתם ההומיה }\מקור{[עפ״י שם ב שח]}\צהגדרה{. }

\משנה{רום הקודש }\הגדרה{- מקור ההויה. תוכן היש המוחלט. הישות המוחלטה, שאין עמה העדר וכליון. מקום\mycircle{°} השאיבה המקורית, ששם היש המוחלט שוכן בחביונו, האידיאליות שממעל למדת כל חרות\mycircle{°}, מפני שאין תוכן של עבדות מוכשר לצאת משם, אפילו אחרי רבוא רבבות של הורדות והשתלשלויות. מקור החיים, שהוא למעלה מן החיים (משם) שואב ישראל\mycircle{°} את רוחו }\מקור{[עפ״י שם רפד\hebrewmakaf רפה]}\צהגדרה{. }

\משנה{מכון עז הקדש }\הגדרה{- הרוממות המוחלטה שהיא מזהרת בשטפי אוריה, בתחיה חדשה גם את הגלמים\mycircle{°} כולם, למלא אותם בתוך תוכיותם הוד חיים, להשאיבם במעמק אור אלהים\hebrewmakaf חיים\mycircle{°}, להפכם מתכונת המות\mycircle{°} והשאיה הדוממת להדרת כבוד וחיי\hebrewmakaf אמת\mycircle{°} של תחית נצחים }\מקור{[ר״מ קעב\hebrewmakaf ג]}\צהגדרה{. }

\משנה{קודש }\הגדרה{- הרוממות הנשגבה }\מקור{[קובץ ח א]}\צהגדרה{. }

\משנה{הקודש העצמי }\הגדרה{- החכמה\mycircle{°} }\מקור{[א״ק ב רפג]}\צהגדרה{. }

\הגדרה{הטוב\mycircle{°}. רום\hebrewmakaf חביון\mycircle{°} }\מקור{[א״ש ו ו]}\צהגדרה{. }

\משנה{אור קודש }\הגדרה{- הנצח\mycircle{°} והשיגוב }\מקור{[ע״ר א פה]}\צהגדרה{. }

\משנה{הקודש }\הגדרה{- העולם הנצחי, הרוחני\mycircle{°}, השכלי, הקיים }\מקור{[א״ק ב שכג]}\צהגדרה{. }

\משנה{הקודש העליון }\הגדרה{- המחשבה האידיאלית הרוממה, המחשבה\mycircle{°} האלהית החפשית\mycircle{°} מכל צמצומים\mycircle{°} }\מקור{[עפ״י א׳ עב]}\צהגדרה{. }

\הגדרה{מחשבתו של יוצר בראשית ברוך הוא }\מקור{[עפ״י ע״א ד יב כח]}\צהגדרה{.}

\משנה{קודש ד׳ }\הגדרה{- תוכן המחשבה\hebrewmakaf העליונה, מגמת\mycircle{°} כל המגמות, שבכללותה של ההנהגה האלהית\mycircle{°} בכל ההויה כולה }\מקור{[ע״ר א ר]}\צהגדרה{. }

\משנה{הקודש העליון }\הגדרה{- החפץ הבהיר בעליוניות מגמתו, שכולו אומר כבוד\hebrewmakaf אל\mycircle{°} }\מקור{[א״ק ג קפא]}\צהגדרה{.}

\הגדרה{צחצחות\mycircle{°} אור הנשמה\mycircle{°} בכל חופשה\mycircle{°} ועצמה }\מקור{[קובץ א תשפא]}\צהגדרה{. }

\הגדרה{המגמה\mycircle{°} הרוחנית\mycircle{°} העליונה\mycircle{°} של אצילות החיים וההויה\mycircle{°}, שישראל\mycircle{°}, מעומק טבע נשמתם\mycircle{°} אחוזים באחיזתה }\מקור{[עפ״י א׳ צ]}\צהגדרה{. }

\הגדרה{שורש נשמתם והויתם של ישראל }\מקור{[עפ״י ע״ר א קנט]}\צהגדרה{. }

\הגדרה{מקור חיי האומה ומציאותה בעולם ויסוד כל תקוותיה }\מקור{[מ״ר 418 (נאדר בקודש, לתחית הקודש סי׳ יב)]}\צהגדרה{. }

\משנה{ראש הררי קודש }\הגדרה{- מעמקי נשמת\hebrewmakaf ישראל\mycircle{°} }\מקור{[מ״ר 298]}\צהגדרה{. }

\משנה{אור הקודש היותר מְעוּמָק שבעולם }\הגדרה{- התוכן הישראלי המיוחד, החיים הקדושים, השופעים בפנימי פנימיות\mycircle{°} ההבהקה של אור אלהי אמת\mycircle{°} והולכים בדרך ישרה על כנסת\hebrewmakaf ישראל\mycircle{°} ופיתוח נשמתה, אחוזים בלשד החיים של קדושת אמונתה\mycircle{°}, הטהורה\mycircle{°} בטוהר\hebrewmakaf עליון\mycircle{°}, שרק העולם העתיד להתחדש ברום טהרת קודשו יוכל לספגו ולהאיר את עלילות החיים על ידו. רוח עליון זה קובע בכחו, בחיים המעשיים שבישראל מעבר מזה ובחיי האמונה ותוכן שפעת\mycircle{°} הלב והסתעפות\mycircle{°} הרוח מעבר מזה, את תביעתה הפנימית של האומה, גבורת עמדתה וחשק נצחונה, מעוז בטחת תקותה ואור עתידה }\מקור{[א׳ כא\hebrewmakaf ב]}\צהגדרה{.}

\משנה{אור הקודש}\הגדרה{ - הזוך והזוהר הנפשי העומד מוכן להופיע בעולם ע״י אורם של ישראל, ההולך לקראת תחיתו ועלייתו }\מקור{[קובץ ח רה]}\צהגדרה{.}

\משנה{רוממות הקודש}\הגדרה{ - יסודות המוסר העליונים, מרחבי האצילות הנשמתיים }\מקור{[קובץ ח קכ]}\צהגדרה{.}

\משנה{הקודש המרומם }\הגדרה{- המטרה האלהית העליונה בכל ההויה, והמגמה הפנימית\mycircle{°} של אור צלם\hebrewmakaf האלהים\mycircle{°} אשר לאדם }\מקור{[ע״ר א כח]}\צהגדרה{. }

\משנה{מרומי הקודש }\הגדרה{- אור\hebrewmakaf התורה\mycircle{°} בשורש שרשה }\מקור{[א״ק א נו]}\צהגדרה{. }

\משנה{קודש עולמים }\הגדרה{- אור\hebrewmakaf ד׳\mycircle{°} ותורתו\mycircle{°}, חשק האורה\mycircle{°} הרוחנית, הגבורה\mycircle{°} הגמורה המנצחת את כל העולמים וכל כחותיהם }\מקור{[עפ״י א׳ פד]}\צהגדרה{.}

\הגדרה{ע״ע קודש קדשים. ע״ע מקור הקודש העליון. ע״ע מקוריות, (המקוריות המציאותית). ע״ע ראשית הכל. ע״ע מחשבה, המחשבה העליונה. ע׳ במדור מונחי קבלה ונסתר, אבא. ושם, מעין עליון, המעין הרוחני של פנימיות הקודש. ע׳ במדור פסוקים ובטויי חז״ל, שמי שמי קדם. }

\paragraphs

\ערך{קודש העליון}\הגדרה{ - הקודש החבוי בחביון\hebrewmakaf העז\mycircle{°} האלהי, המופיע לא ע״י שום הכרה שכלית בעולם, כ״א בהתגלות דבר\hebrewmakaf ד׳\mycircle{°} בנבואה\mycircle{°} ורוח\hebrewmakaf הקודש\mycircle{°} הבהיר העליון, העומד למעלה מציור הקודש המוכר, המתגלה בצורה הכרית גם ביותר שלם שבבני אדם מקדישי\hebrewmakaf שם\hebrewmakaf ד׳\mycircle{°} }\מקור{[עפ״י ע״ר א קיב]}\צהגדרה{.}

\משנה{יסוד הקודש העליון }\הגדרה{- היסוד הנצחי המתרומם מכל הגבלה\mycircle{°}, ומעלה את כל אופי שבחיים אל הרום\hebrewmakaf העליון\mycircle{°}, עד שהכל אובד את התוכן החילוני\mycircle{°} המוגבל, ותופס לו מכון בקודש, ונעשה גם הקודש\mycircle{°} היסודי עומד ברום עזו הנערץ וגם התוכן החילוני כולו הופך אור קודש }\מקור{[ע״ר א קעב]}\צהגדרה{. }

\paragraphs

\משנה{קודש הקדשים}\צהגדרה{ - קיים שבקיים, נשמת המציאות }\צמקור{[מה״ה א ה].}

\מעוין{◊ }\הגדרה{חי\hebrewmakaf העולם\mycircle{°}, האלהות\mycircle{°}, מקור ההויה בעומק תוכן רומו, שם הוא }\צהגדרה{קודש הקדשים }\מקור{[א״ק א קמג]}\צהגדרה{. }

\paragraphs

\ערך{קודש קדשים }\הגדרה{- יסוד הכל, יסוד היסודות, למעלה מכל שם\mycircle{°}, תואר וכנוי, המתרומם מעל כל הגיון\mycircle{°}, גם של קודש\mycircle{°}, באותה אין\hebrewmakaf הסופיות שהוא מתרומם מעל כל חול\mycircle{°} }\מקור{[עפ״י מ״ר 408]}\צהגדרה{. }

\הגדרה{קודש\hebrewmakaf העליון\mycircle{°}, ששם גם החול וגם הקודש באים מתוכן אחד ומתגלים בתור צביון אחד, ״כי לית שמאלא\mycircle{°} בהאי עתיקא\hebrewmakaf קדישא\mycircle{°}, דכוליה ימינא\mycircle{°}״ }\צהגדרה{[מ״ר 406}\הגדרה{]. }

\הגדרה{מדרגה המופיעה מיסוד הקדש\hebrewmakaf העליון, שהוא היסוד הנצחי המתרומם מכל הגבלה, ומעלה את כל אופי שבחיים אל הרום\hebrewmakaf העליון\mycircle{°}, עד שהכל אובד את התכן החלוני\mycircle{°} המוגבל, ותופס לו מכון בקדש, ונעשה גם הקדש היסודי עומד ברום עזו הנערץ וגם התכן החלוני כולו הפך אור קדש }\מקור{[ע״ר א קעב]}\צהגדרה{.}

\מעוין{◊ }\הגדרה{נושא בתוכו את הנושא של הקודש ואת הנושא של החול, את החומר\mycircle{°} ואת הצורה\mycircle{°}, בצורה עליונה אידיאלית\mycircle{°}, באחדות מעולה }\מקור{[קובץ ח צא]}\צהגדרה{.}

\הגדרה{ע׳ במדור פסוקים ובטויי חז״ל, רום עולם.  }

\paragraphs

\ערך{קודש קדשים }\הגדרה{- יסוד הקדושה\hebrewmakaf הגנוזה\mycircle{°}, מקור השפעה\mycircle{°} העליונה\mycircle{°} של תורה\mycircle{°} מצד הופעתה מפי הגבורה באספקלריא\hebrewmakaf המאירה\mycircle{°} לרועה נאמן\mycircle{°}. הבחינה הפנימית (של הקדושה), סוד ברכתה\hebrewmakaf של\hebrewmakaf תורה\mycircle{°}, קשורה של תורה למקורה האלהי מצד נותן התורה ב״ה }\מקור{[עפ״י ע״ר א קלח]}\צהגדרה{. }

\הגדרה{ע״ע קודש, מדת הקודש. }

\paragraphs

\ערך{קודש הקדשים }\הגדרה{- }\משנה{(בבית המקדש) }\הגדרה{- ע׳ במדור משכן ומקדש, בית קודש הקדשים.}

\paragraphs

\ערך{קוטן }\הגדרה{- }\משנה{הקוטן לעומת הגודל\mycircle{°}}\הגדרה{ - הפרטיות }\מקור{[א״ק א עט]}\צהגדרה{. }

\הגדרה{ע״ע קטנות. }

\paragraphs

\ערך{קול }\הגדרה{- היסוד היותר פנימי, שהוא מתגלה בהארה של שני יסודות צמיחת ההבטאה: האמר\mycircle{°} והדבור\mycircle{°} }\מקור{[עפ״י ע״ר ב נד]}\צהגדרה{. }

\ערך{״קול״ }\הגדרה{- }\משנה{רמיזת ענין הקול }\הגדרה{- אל גילוי הרצון והשכלת דבר <שהתחלתו הקול, שממנו יוצאים דברים> }\מקור{[נ״א ג 25]}\צהגדרה{. }

\paragraphs

\ערך{קול }\הגדרה{- }\משנה{קול ההויה כולה }\הגדרה{- האֹמר הכללי, שכל ההויה כולה מדברת ברוחה המיוחד, שכבוד\hebrewmakaf אל\mycircle{°} מתבטא על ידה, הנעלה מהענפים של הבטוי ההויתי - מהענף העליון של האמר\mycircle{°}, ומהענף השני של הדבור\mycircle{°}. והוא למעלה מההקשבה של כל היצורים בפרטיותיהם. כי הוא אמר כללי, היוצא מתוך המציאות בהקיפה הכולל, ולא יוכל להפרט בפרטיותם בשום תיאור מתגלה לכל נוצר מפורט }\מקור{[עפ״י ע״ר ב נד]}\צהגדרה{. }

\paragraphs

\ערך{קול אלקים הקורא אל האדם מטהרת נפשו פנימה }\הגדרה{- הרעיונות המתרוצצים פנימה באדם שמוצא נפשו מתרגשת ומתגעגעת להתפרץ להיטיב יותר מחוג יכולתה. הרעיונות הרמים, שהם למעלה מחוג יכולתו, שמתרגשים בקרבו }\מקור{[עפ״י ל״ה 117]}\צהגדרה{.}

\paragraphs

\ערך{קול פנימי}\הגדרה{ - }\משנה{הקשבת הקול הפנימי }\הגדרה{- תפיסת הרגשות הטמירים ואת מהות חיי הנשמה }\מקור{[עפ״י פנק׳ א שצז]}\צהגדרה{.}

\paragraphs

\ערך{קומה }\הגדרה{- גמר שלם בכל דרכי השכליות והמדות הנאותות לאדם מצד שהוא אדם, נברא בצלם\mycircle{°} ובדמות\mycircle{°} לעבוד לקונו }\צהגדרה{[מ״ר 82}\צמקור{ (}\צהגדרה{פנק׳ ג רלא)]. }

\הגדרה{ע׳ במדור מונחי קבלה ונסתר, שעור קומה. ע״ע קימה. }

\paragraphs

\ערך{קופיות }\הגדרה{- ההתחקות\mycircle{°} }\מקור{[ר״מ כד]}\צהגדרה{. }

\משנה{התוכן הקופי }\הגדרה{- ההתחקות העוזבת את ערכי עצמותה ופונה אל מה שהיא רואה מתוכנים חוציים, נפעלת על ידם להקלט בהופעה החיה והפועלת על פיהם }\מקור{[שם קלו\hebrewmakaf ז]}\צהגדרה{. }

\משנה{היסוד הקופי }\הגדרה{- הערכים המחוקים }\מקור{[שם קלז]}\צהגדרה{. }

\paragraphs

\ערך{קופיות }\הגדרה{- }\משנה{הקופיות האנושית }\הגדרה{- הצטלמות\mycircle{°} החיקוי\mycircle{°} החודרת בכל השטחים, הפועל בכל השדרות לפי ערכיהם }\מקור{[ר״מ כב]}\צהגדרה{. }

\paragraphs

\ערך{קופיות קדושה }\הגדרה{- הקופיות\mycircle{°} המעולה ומפליאה, מתחילה מהתנסחות שבלונית, ועולה ובאה, מתגדלת\mycircle{°} ומתרוממת עד לכדי חידור פנימי פנימיות, מקור קודש\hebrewmakaf קדשים\mycircle{°}. החיקוי הטוב המשכלל את הישות, מקור הפאר וההתהדרות, המעלה את כל שפל אל רום, מכון המנהגים הטובים והקבלות היקרות שהן הולכות משרים וקולעות אל מטרת רוממות צביונו }\מקור{[עפ״י ר״מ כב כג]}\צהגדרה{. }

\משנה{התוכן הקופי המטוהר ונאדר בקודש }\הגדרה{- יסוד ההעתקה המעולה }\מקור{[שם כד]}\צהגדרה{. }

\הגדרה{יסוד הקודש שבתכונת החיקוי\mycircle{°}, להיות האדם וכל מדרגה של מציאות שוקקים להתדמות לצד עילאה, ובזה הם מתרוממים ומשתגבים }\מקור{[עפ״י שם צ]}\צהגדרה{. }

\paragraphs

\ערך{קופיות רעה }\הגדרה{- ההריסה של החקוי\mycircle{°} שאינה לפי התוכן, הענין, והכונה היוצרת, החמושה בעדי הצדק\mycircle{°} המוחלט }\מקור{[ר״מ כב]}\צהגדרה{. }

\הגדרה{הקופיות\mycircle{°} המורידה, המסתפקת בחיקויה, שלא תרומם למעלה, לא תדאג על שטחיותה, ולא תרים את רוחה, אל ההתעמקות הנשמתית במעמק מעין החיים }\מקור{[שם כג]}\צהגדרה{. }

\הגדרה{יסוד החול\mycircle{°} שבתכונת החקוי\mycircle{°}, שממנו באה כל הירידה של הכלל והפרט, וכל השקיעה במצולות הסכלות והרשעה, בזה שהוא זוקף את כל התעלותו על חשבון החקוי, ואינו חודר לתוך ההתרוממות הפנימית, שיש בתוך התוכן של חקוי זה ונטיתו }\מקור{[שם צא]}\צהגדרה{. }

\paragraphs

\ערך{קטנות }\הגדרה{- }\משנה{(לעומת גדלות\mycircle{°} באדם) }\הגדרה{- פרטיות\mycircle{°}. כשאדם עומד במצב הפשטות של אמונה, וסדר מוסרי מחובר ביראת\hebrewmakaf העונש\mycircle{°} ואהבת שכר, בתור דבר העיקרי, (הוא) הגורם לכל סיבוב החיים הרוחניים\mycircle{°} שלו, וכל המעשים הטובים והמדות הטובות יונקים אז משורש קטן ושפל זה. כאשר השכר ועונש הינם הגורם העיקרי בדחיפת החיים המוסריים (של האדם) }\מקור{[עפ״י א״ק ג שכא-ב]}\צהגדרה{.}

\צמשנה{בדרך הקטנות}\צהגדרה{ - }\הגדרה{במובן פרטי }\מקור{[קבצ׳ ג יז]}\צהגדרה{.}

\paragraphs

\ערך{קטנות }\הגדרה{- הבינוניות\mycircle{°}. האנשים הנדים בחפצם כאשר ינוד הקנה, שהם אינם שלמי היצירה. עלילות האדם לנטות פעם לטוב ופעם לרע, שהיא אות על רפיון נפשו ו}\משנה{קטנותה }\מקור{[עפ״י ע״א ב ט רכד]}\צהגדרה{. }

\paragraphs

\ערך{קטנות, נערות}\myfootnote{ \textbf{קטנות נערות, שיפה הוא לחנוך} - ע׳ פרדס, ערכי הכנויים, ערך נער, מובא בשל״ה, ח״ב דפים יז. כה:.\label{2}}\הגדרה{ - }\משנה{תוכנם }\הגדרה{- שיקועה של המחשבה רק בעולם המעשי, אף על פי שנראים גם בו סימני גדולה רוחב והעמקה, כללותו איננו יכול לעלות בכל זה מערכו המזער, והוא תוכן נערות וקטנות, שיפה הוא לחנוך, ומסוכן הוא אם ישאר בתור מטרה אחריתית }\מקור{[א״ק ג קמא]}\צהגדרה{. }

\paragraphs

\ערך{קטנות }\הגדרה{- פרטיות }\מקור{[עפ״י מ״ש ס]}\צהגדרה{.}

\הגדרה{המהות המוזערה, ביחושה הפרטי למציאותנו המיוחדת }\מקור{[עפ״י ע״ר א יט]}\צהגדרה{. }

\הגדרה{ע״ע התקטנות. ע״ע גדלות, במעמד הבהירות שלה.}

\paragraphs

\ערך{קטרוג }\הגדרה{- מיעוט צביון\mycircle{°} ועושק כחות, אורות\mycircle{°} ונשמות\mycircle{°} }\מקור{[עפ״י אג׳ ב קע]}\צהגדרה{.}

\paragraphs

\ערך{קיום }\הגדרה{- העמדה הקבועה, המתעלה מכל שינוי ותמורה }\מקור{[ע״ר א ב]}\צהגדרה{. }

\משנה{קיימות }\הגדרה{- העמדה הבלתי משתנה כלל }\מקור{[שם קצה]}\צהגדרה{. }

\משנה{מעמד קיים }\הגדרה{- בלא שום שינוי, חלוף ותמורה }\מקור{[שם]}\צהגדרה{. }

\paragraphs

\ערך{קיחה }\הגדרה{- מורה על ענין שייכות הרשות שתהיה מסורה לו }\מקור{[מא״ה א קלג]}\צהגדרה{. }

\paragraphs

\ערך{קימה}\הגדרה{ - העמדת כל שעור\hebrewmakaf הקומה\mycircle{°} }\מקור{[ע״ר א ריט]}\צהגדרה{. }

\הגדרה{ע״ע קומה. }

\paragraphs

\ערך{קימה }\הגדרה{- }\משנה{(קימה באדם) }\הגדרה{- מלא הקומה ואזירת החיל\mycircle{°} }\מקור{[עפ״י ע״ר א פז]}\צהגדרה{. }

\הגדרה{שלמות הנפש\mycircle{°} והשכלת האמת\mycircle{°} שהוא עומד לעד }\מקור{[מא״ה א קעז]}\צהגדרה{. }

\paragraphs

\ערך{קלח }\הגדרה{- שורש הטמון בארץ }\מקור{[קבצ׳ ג קכג]}\צהגדרה{.}

\paragraphs

\ערך{קללה }\הגדרה{- ציור\mycircle{°} שלילת מציאות הנפש, בכללה או בחלקיה }\מקור{[ע״ר א רצא (ע״א א 81)]}\צהגדרה{. }

\paragraphs

\ערך{קם להרע }\הגדרה{- (הקם) לבלע ולהשחית, בפועל ובמעשה הגופניות והחומריות }\מקור{[ע״ר א לד]}\צהגדרה{.}

\הגדרה{ע״ע אויב.}

\paragraphs

\ערך{קנאה טהורה }\הגדרה{- גבורה\mycircle{°} חמה וצוהלת\mycircle{°}, המדרכת עז\mycircle{°} ומאזרת חיל\mycircle{°} }\מקור{[ע״ה קנג]}\צהגדרה{. }

\paragraphs

\ערך{קנס }\הגדרה{- דבר משפט שהצורך נותן לעבור על השורה הבינונית כדי לתקן לא את אותו המאורע כ״א את כלל החיים }\מקור{[עפ״י ע״א ג א עא]}\צהגדרה{. }

\paragraphs

\משנה{קרבה אל ד׳ }\צהגדרה{- ההכרה העליונה בהדר\mycircle{°} גאון\hebrewmakaf ד׳\hebrewmakaf ועוזו\mycircle{°}, ההכרה בכלליות\mycircle{°} המציאות והאהבה\mycircle{°} }\צמקור{[צ״צ א כד (א״ל ריב)]. }

\הגדרה{ע׳ במדור פסוקים ובטויי חז״ל, ״קרבת אלהים״. }

\paragraphs

\ערך{קרוב }\הגדרה{- מגושם\mycircle{°} ומצומצם. מורגש }\מקור{[עפ״י א׳ לא, נה]}\צהגדרה{.}

\הגדרה{מוחש }\מקור{[פנק׳ ד קנב]}\צהגדרה{.}

\הגדרה{קטן }\מקור{[א״ק ג רכ]}\צהגדרה{.}

\הגדרה{מעשי [}\צהגדרה{ע״א ד ו סא].}

\צמקור{פרוזאי, מעשי. לא כוונות נסתרות, לא דברים פנימיים (- כלייחד ייחודים ולכוון כוונות), אלא קרוב לביצוע וקל לביצוע ושייך לענייני היום יום, ולאפשרות שינוי היום יום [רצב״י טאו].}

\משנה{קרובה }\הגדרה{- מורגשת מיד לעיני הדור בהוה }\צמקור{[פנק׳ ד רכ]}\הגדרה{.}

\הגדרה{ע״ע רחוק. }

\paragraphs

\ערך{קריאה }\הגדרה{- תפילה\mycircle{°} ובקשה }\מקור{[ע״ר א רטז]}\צהגדרה{.}

\משנה{קריאה למלך\mycircle{°}}\הגדרה{ - תפילה והבעת ההכרה של הממלכה האלהית\mycircle{°}, באדם ובעולם, בחיים ובמציאות }\מקור{[ע״ר א קנ]}\צהגדרה{. }

\paragraphs

\ערך{קריאה אל ד׳}\myfootnote{ שופטים טו יח.\label{3}}\הגדרה{ - השתדלות שיהיה ד׳ קרוב\mycircle{°} לו בהשגחתו\mycircle{°}. הגברת החיל שיאיר\hebrewmakaf ד׳\hebrewmakaf פניו\mycircle{°} לו ותהיה השגחתו חופפת עליו לשמרו מכל רע, בין בעניני גופו בין בעניני נפשו }\מקור{[עפ״י מ״ש נו (מא״ה א קסו)]}\צהגדרה{. }

\הגדרה{ע׳ במדור פסוקים ובטויי חז״ל, דרישת ד׳. }\mylettertitle{ר}

\paragraphs

\ערך{ראיה }\הגדרה{- ע׳ במדור גוף האדם אבריו ותנועותיו, עין. ע׳ במדור אותיות עי״ן (הוראת האות עי״ן).}

\paragraphs

\משנה{ראיה לעומת חזון}\צהגדרה{ - }\צמשנה{ראיה}\צהגדרה{ - ראית עין, ראיה ממשית,  }\צהגדרהמודגשת{חזון}\צהגדרה{ - ראיה של מעלה, ראיה פנימית רוחנית נפשית }\צמקור{[שי׳ מועדים ב 123-4].}

\הגדרה{ע׳ במדור הכרה והשכלה והפכן, חזון.}

\paragraphs

\משנה{ראיה של מעלה}\myfootnote{ בתפארת ישראל למהר״ל, פרק מח עמ׳ קמט: ״כאשר הדבר קרוב בהשגת האדם והוא אליו מושכל ראשון נקרא זה שרואה הדבר״. בקל״ח פתחי חכמה, פתח ט, מסביר הרמח״ל את מראה הכבוד העליון, כנראה בראיית הנשמה – כלומר, שהיא יותר הבנה מראיה, ״כמו המחשבות בשכל״. וע׳ כוזרי עם באורי, כרך א עמ׳ קכב-קכה, הראיה בספר הכוזרי.\label{1}}\הגדרה{ }\צהגדרה{- ראיה פנימית רוחנית נפשית, חזון\mycircle{°} }\צמקור{[שי׳ מועדים ב 123-4].}

\paragraphs

\משנה{״רֵאשׁ מִלִּין״}\myfootnote{ דניאל ז א.\label{2}}\הגדרה{ }\צהגדרה{- תורת מוצא המלין לראשיתן בחכמה }\צמקור{[ד״ל, (מהדורת תשס״ה) אגרת מא].}

\paragraphs

\ערך{״רב טוב״ }\הגדרה{- ע״ע טוב אלהי, ״רב טובך״. }

\paragraphs

\ערך{רבוי }\הגדרה{- }\משנה{הריבוי }\הגדרה{- ההתרבות וההתגדלות הכמותית }\מקור{[ר״מ כה]}\צהגדרה{. }

\paragraphs

\ערך{רבוי אחדותי }\הגדרה{- הריבוי\mycircle{°} העשיר שבעשירים, הריבוי של אחד של לפני\hebrewmakaf אחד\mycircle{°}, כלומר השלמות של הריבוי בהוד תפארת האחדות האלהית }\מקור{[א״ק  ב תג]}\צהגדרה{.}

\paragraphs

\ערך{רבנות ראשית }\הגדרה{- }\משנה{הרבנות הראשית}\הגדרה{ - רבנות עליונה שתעודתה השלום\mycircle{°} והאחדות והטבעת חותם היהדות\mycircle{°} על חיינו הנרקמים בארצנו }\מקור{[אג׳ ה רב]}\צהגדרה{.}

\paragraphs

\ערך{רבנים }\הגדרה{- }\משנה{עבודת הרבנים }\הגדרה{- השפעה לחבב את הדעות המביאות לאימוץ וסידור אורגני לאומי, לחזק את ההכרה הלאומית ולאמץ ידים רפות של פועלים מוכשרים לעבודה זו. לעדן את המחשבות ולחדדן ולהרבות בהן אמונה אמיצה ועמוקה משפעת הגיוני\hebrewmakaf קודש האצורים בנשמתם, ולהעלות בזה את התחיה הלאומית כולה למרומי הקודש שמשם לוקחה }\מקור{[אג׳ ב שכד]}\צהגדרה{.}

\paragraphs

\ערך{רגש }\הגדרה{- }\משנה{מקור הרגש }\הגדרה{- הנטיה שהיא מצומצמת לפי התכונה של הכלי\mycircle{°}, המקבלת את החיים בקרבה, כעין אור\hebrewmakaf הפנימי\mycircle{°}, ובכל העולמות\mycircle{°} כולם היא תכונת המלכות\mycircle{°} }\מקור{[עפ״י א׳ קסא]}\צהגדרה{. }

\הגדרה{ע׳ במדור שמות כינויים ותארים אלהיים, ה״א אחרונה של שם שהיא נעשית שם אדנות. ע״ע שכל, מקור השכל. }

\paragraphs

\ערך{רגש}\הגדרה{ - }\משנה{יסוד הרגש הטוב }\הגדרה{- לקבל את כל התמציות השכליות ולקבען יפה, עד שמה שע״פ המסקנא השכלית והתורית ראוי הוא שיהיה יפה ונחמד, יהיה הלב\mycircle{°} והרצון וכל הנטיות היותר ערות וחיות מסורות לו ברב כח. וכל מה שהתורה והשכל נותנים שהוא דבר מזיק ומפסיד, יהיה מיד נחקק בעצם הרגש הטבעי גיעולו וריחוקו, ממילא יהיה עומק הרגש מתעלה בין לטוב\mycircle{°} בין להפכו כפי אותו הגודל הניתן לדבר מצד האמת }\מקור{[ע״א ג ב רנג]}\צהגדרה{.}

\paragraphs

\ערך{רגש דתי}\הגדרה{ - ע״ע דת. ע״ע רוח האמונה, הנטיה הדתית. }

\paragraphs

\ערך{רגש לאומי}\הגדרה{ - לב ער ונפש מרגשת בחוש אמיץ ובריא את הטוב לעמו, והשתוקקות באהבה חיה וטבעית להגדיל ערכו והצלחתו }\מקור{[עפ״י ל״ה 132]}\צהגדרה{.}

\הגדרה{ע״ע רוח הלאומי. }

\paragraphs

\ערך{רגש עליון}\הגדרה{ - }\משנה{הרגש העליון }\הגדרה{- הלב\mycircle{°}, הרוח הנשגב, רוח\hebrewmakaf הקודש\mycircle{°} הכוללת השורה על ישראל\mycircle{°} ועל מלכותו\mycircle{°} }\מקור{[ע״א ד ה עא]}\צהגדרה{.}

\הגדרה{ע׳ במדור אישים, דוד.}

\paragraphs

\ערך{רגשות}\הגדרה{ - }\משנה{הרגשות}\הגדרה{ - הלח הפנימי והעצמי של החיים }\מקור{[ע״ה קכט]}\צהגדרה{.}

\paragraphs

\ערך{רגשי קודש }\הגדרה{- רגשי צדק\mycircle{°} מלאי חיים לעשות הטוב\mycircle{°} והישר\mycircle{°} }\מקור{[ע״א א ב ו]}\צהגדרה{. }

\paragraphs

\ערך{רדיה  }\הגדרה{- }\משנה{של מושל עריץ }\הגדרה{- השתמשות המושל במשועבדיו להנאתו ולהות בצעו }\מקור{[עפ״י ל״ה 54]}\צהגדרה{.}

\הגדרה{ממשלת מושל עריץ המדכא כל תחת רגליו להנאתו ואהבת בצעו }\מקור{[א״ה ב 241]}\צהגדרה{.}

\הגדרה{התעמרות בעמו ועבדיו רק להפיק חפצו הפרטי ושרירות לבו }\צהגדרה{[עפ״י א״ה ב }\צמקור{88, }\צהגדרה{א״ב ו].}

\paragraphs

\ערך{רואה}\הגדרה{ - שמרגיש מנפשו פנימה }\מקור{[ע״א ב ט שנט]}\צהגדרה{.}

\paragraphs

\ערך{רוח}\הגדרה{ - (}\צהגדרה{הרוח שלא נצחה את עשן\mycircle{°} המערכה}\הגדרה{)}\myfootnote{ אבות ה משנה ה.\label{3}}\הגדרה{ - מורה על כח הרצון המתגלה בנפש }\מקור{[ע״ר ב קעז]}\צהגדרה{. }

\הגדרה{ע׳ במדור משכן ומקדש, אש המערכה. ע״ע גשמים.}

\paragraphs

\ערך{רוח אלהי }\הגדרה{- }\משנה{הרוח האלהי }\הגדרה{- הרוח הכללי, החפשי והמוחלט, הרם מכל מדע ורגש; ודוקא מתוך רוממותו על כלם הוא מכשיר את המקום לכל התרחבות החיים של כל המדעים וההרגשות היותר אדירות ויותר כוללות ומקיפות את כל המצב הרוחני\mycircle{°} של ההויה כולה. מתגלה בקודש\mycircle{°} - בנבואה\mycircle{°}; ובחול\mycircle{°} - בפילוסופיא ובשירה\mycircle{°} העליונה }\צהגדרה{[מ״ר }\צמקור{21, 102}\צהגדרה{]. }

\משנה{הנטיה האלהית }\הגדרה{- דרישת\hebrewmakaf ד׳\mycircle{°} הבורא כל ומחיה כל, רבון כל העולמים ואלוה כל הנשמות, הצד האלהי המטה את ההשפעה של החיים לרוממות שאיפת דעת\hebrewmakaf אלהים\mycircle{°} בשכל או בהרגשה }\מקור{[עפ״י ב״ר שכו\hebrewmakaf ז]}\צהגדרה{. }

\הגדרה{ע׳ במדור פסוקים ובטויי חז״ל, ארבע רוחות. ע״ע רוח המוסר. ע״ע רוח האמונה. ע״ע רוח הלאומי. }

\paragraphs

\ערך{רוח אלהים }\הגדרה{- החיים הקוסמיים בכל עשרם והדר\mycircle{°} גאונם }\מקור{[מ״ר 23]}\צהגדרה{.}

\paragraphs

\ערך{רוח גן\hebrewmakaf עדן\mycircle{°} אלהים, המנשב בנשמה }\הגדרה{- הרצון להיות טוב\mycircle{°} }\מקור{[א״ש טז ג]}\צהגדרה{. }

\הגדרה{ע׳ במדור נפשיות, נשמה, (זמן) עליית הנשמה. ע׳ במדור פסוקים ובטויי חז״ל, מתרפקת על דודה.}

\paragraphs

\ערך{רוח ד׳\mycircle{°}}\הגדרה{ - רוח\hebrewmakaf ישראל\mycircle{°} המלא הכללי הממלא את כל חללה של הנשמה\mycircle{°} }\מקור{[א׳ יב]}\צהגדרה{. }

\משנה{רוח ד׳ אשר על האומה בכללה }\הגדרה{- (ה)הטבעה הטבעית הרוחנית\mycircle{°} אשר בנשמת\hebrewmakaf ישראל\mycircle{°} }\מקור{[שם ט]}\צהגדרה{. }

\משנה{רוח ד׳ }\הגדרה{- נשמת חיי העולם, יד האורה\hebrewmakaf העליונה\mycircle{°}, אור חכמת\mycircle{°} כל עולמים }\מקור{[שם כט]}\צהגדרה{. }

\הגדרה{החפץ העליון שהוא יסוד הכל, למעלה מכל תכונה של שלשלאות סבתיות\mycircle{°} }\מקור{[ע״א ד יב ו (מא״ה ב קלב)]}\צהגדרה{. }

\משנה{רוח ד׳ העליון}\הגדרה{ - דבר פיו ומחשבתו\mycircle{°} הגנוזה, המסדרת את כל היש}\צהגדרה{ }\מקור{[ע״ר א קמה]}\צהגדרה{.}

\הגדרה{ע׳ במדור פסוקים ובטויי חז״ל, רוח אלהים. }

\paragraphs

\ערך{רוח ד׳\mycircle{°}}\הגדרה{ - רוח האמת\mycircle{°} והחסד\mycircle{°} }\מקור{[א״י סח]}\צהגדרה{.}

\הגדרה{רוח קדושה\mycircle{°} וטהרה\mycircle{°} }\מקור{[ע״א ג ב קעט]}\צהגדרה{.}

\משנה{ברוח ד׳ }\הגדרה{ - בדעת והשכל }\מקור{[ע״א ג ב קסח]}\צהגדרה{. }

\paragraphs

\ערך{רוח האמונה\mycircle{°}}\הגדרה{ - האספקלריא שמתוכה תראינה באופן הקרוב למעשה והליכות החיים, כל התיאוריות הנכללות באורח אצילות ברוח\hebrewmakaf האלהי\mycircle{°} העליון, וברוח\hebrewmakaf מוסר\hebrewmakaf המוחלט\mycircle{°}, המתגלה במכוני הדתות, לפי חלוקי האופיים המיוחדים שלהן }\צהגדרה{[עפ״י מ״ר }\צמקור{21, 102}\צהגדרה{]. }

\משנה{הנטיה הדתית }\הגדרה{- הנטיה החפצה להתקדש בכל תכסיסי מעשים ומנהגים קדושים, הממולאה ביחש אלהי ומוסר אלהי ע״י השפעה קיבוצית המתהוה ע״י כח נשגב של אדם עליון המתנשא ממעל לכל העם כולו והמרומם אותם אל האלהים בכחו הרוחני העליון השופע מבהירות העליוניות של הנטיה\hebrewmakaf האלהית, והנטיה האנוכית העליונה העומדת על גבה בצירוף הנטיה\hebrewmakaf הלאומית. הצד האמוני, של הרבות דעת האמונה והרגשתה, התגלות תמציתה העליון המכוון מהוייתה, וזיכוך הלב הבא ע״י חיזוקה ואומץ הופעתה }\מקור{[עפ״י ב״ר שכו\hebrewmakaf ז]}\צהגדרה{. }

\הגדרה{ע׳ במדור פסוקים ובטויי חז״ל, ארבע רוחות. ע״ע רוח הלאומי. ע״ע דת. ע״ע דת, רגש דת. }

\paragraphs

\ערך{רוח הזמן }\הגדרה{- }\משנה{רוח הזמן האמיתי של היהדות\mycircle{°}}\הגדרה{ - שמירת וברכת\mycircle{°} הטבע של היהדות בתהלוכה הגינית הראויה לה - שהם כל מצוותיה\mycircle{°} של תורה\mycircle{°} ״דת משה ויהודית״\mycircle{°} בפועל, ואהבתם וכבודם ברעיון וברגש, בכל האמצעים החנוכיים הדרושים לזה. אלא שתחת כח המושך של החרדות התשה והרפויה, צריך להכניס כח מושך חי ובריא, יליד אומץ רוח ורעיון בהיר, שיעשה את כל הפעולות, שעשתה הראשונה, ביתר שאת ״ילכו מחיל אל חיל יראה אל אלהים בציון״ }\מקור{[עפ״י ע״ה קמח]}\צהגדרה{. }

\paragraphs

\ערך{רוח הלאומי }\הגדרה{- }\משנה{הרוח החברתי }\הגדרה{- קובע את התחום החברותי, בהשלמתו המוגבלת, מתגלה בכונניות\mycircle{°} מדינות, עמים ממלכות, וממנו תוצאות לרוח הלאומי המיוחד ולסדרי המשפחה. מקבל לתוכו את כל האורות\mycircle{°} במטבע המיוחדת לו ונוטל ג״כ תמציות רבות מהצדדים של ההכרות והנטיות הכלולות בשלשת צדדי הרוח, העליונים ממנו והקודמים לו (הרוח\hebrewmakaf האלהי\mycircle{°}, רוח\hebrewmakaf המוסר\hebrewmakaf המוחלט\mycircle{°} ורוח\hebrewmakaf האמונה\mycircle{°}) }\צהגדרה{[עפ״י מ״ר }\צמקור{21, 102}\צהגדרה{]. }

\משנה{הנטיה הלאומית }\הגדרה{- הנטיה המוצאת קורת רוח וחיים שלמים בהשתתפות הגמורה עם כללות האומה, עם קיומה והמשכת חייה לדורותיה בצביונה וברוחה הפנימי, שקוטב הענין הוא להבליט את כל הנטיות כולן, של האנוכיות ושל האלהות, של המוסריות ושל הדתיות, הכל בצורה המיוחדת הנאותה לטבע האומה, הרוחני והחומרי, דהיינו לפי הטבע הפסיכולוגי של האומה, ולפי טבע התולדה והמורשה של האבות והגזע, ולפי הטבע הגיאוגרפי של ארץ נחלתה, ולפי הערך של מצבה בכל אלה בין העמים כולם שעל פני כל הארץ. הצד הלאומי המקשר את האומה בצורתה הרוחנית ובמעמדים הרוחניים הנדרשים לה }\מקור{[עפ״י ב״ר שכו\hebrewmakaf ז]}\צהגדרה{. }

\הגדרה{ע׳ במדור פסוקים ובטויי חז״ל, ארבע רוחות. ע׳ בנספחות, מדור מחקרים, לאומיות. ע״ע רגש לאומי. }

\paragraphs

\ערך{רוח המוסר\mycircle{°}}\הגדרה{ - }\משנה{הנטיה המוסרית המוחלטת }\הגדרה{- הנטיה השואפת לצדק, לחיים, ליושר ולמשפט בכל. הצד המוסרי של הרבות היושר, החסד, הצדק וכל הטוב }\מקור{[עפ״י ב״ר שכו\hebrewmakaf ז]}\צהגדרה{. }

\משנה{רוח המוסר המוחלט }\הגדרה{- }\מעוין{◊}\הגדרה{ מתגלה בסדרי החיים, במשפט, בחסד ובצדק וכיוצא בהם. הוא בא באנושיות בתור תוצאה מסובבת מתוך התביעה האלהית, המתנוצצת בנשמתה הכללית של האנושיות ומתגלה לפעמים גם בתור חזיון אדיר לבדו בהעלמת מקורו }\צהגדרה{[עפ״י מ״ר }\צמקור{102, 21}\צהגדרה{]. }

\הגדרה{ע׳ במדור פסוקים ובטויי חז״ל, ארבע רוחות. ע״ע רוח אלהי. ע״ע רוח האמונה. ע״ע רוח הלאומי. }

\paragraphs

\ערך{רוח העולם }\הגדרה{- השאיפה המציאותית וכחותיה }\מקור{[א״ק ב תעו]}\צהגדרה{.}

\paragraphs

\ערך{רוח הקודש }\הגדרה{- }\משנה{אור רוח הקודש המעודד את החיים }\הגדרה{- רוח השמחה\mycircle{°} בנשמה\mycircle{°} }\מקור{[קבצ׳ א רכב]}\צהגדרה{.}

\paragraphs

\ערך{רוח הקודש }\הגדרה{- }\משנה{שפעת רוח הקודש המיוחדת לישראל }\הגדרה{- }\מעוין{◊}\הגדרה{ שפעה הנמשכת מפנימיותה של תורה\mycircle{°}, והיא רוממה הרבה במעלתה ומיוחדת במינה מכל מיני שפע של רוח הקודש, שכל אחד מבאי עולם זוכה לו על\hebrewmakaf פי מעשיו}\myfootnote{ \textbf{רוח הקודש, שכל אחד מבאי עולם זוכה לו על\hebrewmakaf פי מעשיו} - תדא״ר פרק ט: ״בין ישראל בין עכו״ם בין איש בין אשה בין עבד ובין שפחה הכל לפי המעשה שהוא עושה כך רוח\hebrewmakaf הקודש שורה עליו״.\label{4}}\הגדרה{ }\מקור{[עפ״י א״ק א כט]}\צהגדרה{. }

\paragraphs

\ערך{רוח הקודש }\הגדרה{- ״}\משנה{משמש ברוח הקודש}\הגדרה{״ - ע׳ במדור פסוקים ובטויי חז״ל, משמש ברוח הקודש.}

\paragraphs

\ערך{רוח הקודש של כנס״י\mycircle{°}}\הגדרה{ - החפץ של האידיאליה\mycircle{°} האלהית\mycircle{°} היותר גבוהה, לגשמה בחיים ובמעשה, בחיי החברה והאומה\mycircle{°}, ולהרגילה בעדנת הציורים\mycircle{°} התמידיים, ההולכים ושוטפים תמיד בכח המדמה\mycircle{°} אשר בנפש האדם, ולהרחיב צעדיה בחכמה וכשרון, עד שמצד רוב פיתוחה ושכלולה היא מוצאת את עצמה מבונה בכל טוב\mycircle{°}, אור חסד\mycircle{°}, גבורה\mycircle{°}, חכמה\mycircle{°}, שמחה\mycircle{°}, עדן\mycircle{°} ואהבה\mycircle{°}, ומתוך שלמות בנינה ומילואה היא מרגשת ג״כ בהארה\mycircle{°} פנימית\mycircle{°} את האור\mycircle{°} והטוב העולה למעלה למעלה מכל הכלים של יכלתה וחפציה, והיא מתנשאת תמיד לעומתם }\מקור{[א׳ קמה]}\צהגדרה{. }

\משנה{כח רוח הקדש }\הגדרה{- הרגש הקדוש הפנימי והדבקות\hebrewmakaf האלהית\mycircle{°} הזכה, ההולך ומפעם בישראל בתוכיות קדושת נשמתם העליונה, המלאה טל טהר של אהבת צור\hebrewmakaf כל\hebrewmakaf העולמים\mycircle{°}, אור ישראל וקדושו עדי עד }\מקור{[ע״ר א קסב]}\צהגדרה{.}

\משנה{רוחנו הקדוש }\הגדרה{- רוה״ק\mycircle{°} הבאה ע״י הופעת כל חיי הקודש\mycircle{°} הפנימיים\mycircle{°} והחיצוניים\mycircle{°} האחוזים במקור נשמת\hebrewmakaf ישראל\mycircle{°} }\מקור{[אג׳ ג קיא, קיט]}\צהגדרה{. }

\הגדרה{רוח קודש}\myfootnote{ \textbf{רוח קודש} - \textbf{לבקש את הטוב, וכו׳ להיות דבק אל הטוב} - ע׳ ש״ק, קובץ א תקסב. (הערת אסף אורנשטיין). \label{5}}\הגדרה{ <(ש)אינו נמצא לנו כ״א בתוכן הדבקות\hebrewmakaf האלהית\mycircle{°} הבאה לנו מתוך אמונת ישראל המעשית והעיונית. כאשר אנו מובלעים בתוך האומה, אומתנו, האומה שמכללה, מכלל כל דורותיה, יש לנו כל אוצר החיים שהם חיים באמת, שאנו חשים בקרב רוחנו פנימה>. לבקש את הטוב\mycircle{°}, את הטוב המוחלט, להיות בעצמו טוב, להיות דבק אל הטוב }\מקור{[עפ״י א׳ קלו (קובץ א קסב)]}\צהגדרה{.}

\הגדרה{ע׳ במדור מונחי קבלה ונסתר, שכינה. ושם, תפארת. }

\paragraphs

\ערך{רוח טהור }\הגדרה{- }\משנה{הרוח הטהור }\הגדרה{- הקודש\mycircle{°} הנשגב\mycircle{°} בחיים, בהרגשה ובידיעה }\מקור{[קובץ ח ע]}\צהגדרה{. }

\paragraphs

\ערך{רוח ישראל\mycircle{°}}\הגדרה{ - }\משנה{רוח ישראל המופשט }\הגדרה{- מלוא כל, אור\hebrewmakaf אלהים\mycircle{°}, תעודת ההויה, מקור הנשמות\mycircle{°} }\מקור{[א״ת יב א]}\צהגדרה{. }

\הגדרה{ע״ע רוח ד׳. }

\paragraphs

\ערך{רוח עכור }\הגדרה{- בעוד מצב הנפש מצומצם במאסרי הגוף כמו שהוא, בעוד לא זרחה\mycircle{°} הנפשיות הפנימית את אורותיה\mycircle{°}, בעוד הדם\mycircle{°} עושה את שלו ע״פ תכונותו הפראית }\מקור{[עפ״י קובץ א קעג]}\צהגדרה{. }

\paragraphs

\ערך{רוחני }\הגדרה{- ע׳ במדור נפשיות, רוח. }

\paragraphs

\ערך{רוחני }\הגדרה{- }\משנה{החזיון הרוחני שבהויה }\הגדרה{- האידיאל\mycircle{°} שלה, שממשותה הריאלית היא המסקנא שלו }\מקור{[עפ״י א״ת ד ב]}\צהגדרה{.}

\הגדרה{ע״ע חמר, חמרי, החזיון החמרי של ההויה. }

\paragraphs

\ערך{רוחני}\הגדרה{ - }\משנה{האור הרוחני}\הגדרה{ - מקור חיותם והוויתם של המוחשות והמורגשות }\מקור{[עפ״י קובץ ו רמז]}\צהגדרה{. }

\paragraphs

\ערך{רוחניות }\הגדרה{- }\משנה{הרוחניות }\הגדרה{- רום המעלה, תוכן התורה\mycircle{°} ורוממות קדושתה\mycircle{°} }\מקור{[ע״ר א קנו]}\צהגדרה{. }

\הגדרה{הכוחות הפועלים לרומם ולשכלל את המציאות }\מקור{[ע״א ב ט ל]}\צהגדרה{. }

\משנה{הרוחניות שבעולם }\הגדרה{- כוחות הרצון וההכרה שבהויה }\מקור{[ע״ט סו]}\צהגדרה{. }

\הגדרה{ע״ע חמר, החמריות שבעולם.}

\paragraphs

\ערך{רוחניות }\הגדרה{-}\משנה{ העולם הרוחני}\הגדרה{ - עולם הדעות והרעיונות }\מקור{[א׳ נ]}\צהגדרה{.}

\הגדרה{הרעיונות ומסבביהם}\צהגדרה{ }\מקור{[ע״א ב ט כא]}\צהגדרה{.}

\משנה{עניני הרוחניות }\הגדרה{- מעלתו ויחשו (של האדם) אל קרבת\hebrewmakaf אלהים\mycircle{°} }\מקור{[ע״א ב ט כב]}\צהגדרה{. }

\משנה{הענינים הרוחניים }\הגדרה{- תוכן המוסר\mycircle{°}, הצדק\mycircle{°} והיושר\mycircle{°} בדעת\hebrewmakaf האלהים\mycircle{°} וערך הקדושה\mycircle{°} ופעולתה בעולם }\מקור{[שם ג א לג]}\צהגדרה{. }

\הגדרה{הציורים\mycircle{°} המופשטים, ודרכי ההנהגה העליונה וחיי השכל והמוסר }\מקור{[ל״ה 203]}\צהגדרה{. }

\משנה{המציאות הרוחנית }\הגדרה{- המציאות היותר נבחרת, הציורים\mycircle{°} הנעלים והנשגבים שהם קיימים לעד לעולם והם הם יסוד החיים }\מקור{[שם ב קעא]}\צהגדרה{. }

\משנה{תכונות רוחניות}\הגדרה{ - ההויות השמימיות\mycircle{°} בכל מלא הדרן }\מקור{[ע״ר א רט]}\צהגדרה{.}

\הגדרה{ע׳ במדור נפשיות, רוחני. ע״ע חמרי. }

\paragraphs

\ערך{רוחניות }\הגדרה{- }\משנה{היצורים הרוחניים }\הגדרה{- המחשבות והציורים\mycircle{°} הרבים, שהם בריות חיות וקימות המבקשות פדיונן בכל עז ומרץ }\מקור{[א״ק א קצט]}\צהגדרה{. }

\paragraphs

\ערך{רוחניות }\הגדרה{- }\משנה{הרוחניות המציאותית }\הגדרה{- הפסיכיקה העולמית אשר רוח האדם ורוח העולם מתלכדים ובאים בה באחרית הימים לידי הרמוניה שלמה }\מקור{[מ״ר 3]}\צהגדרה{. }

\paragraphs

\ערך{רוחניות עליונה\mycircle{°}}\הגדרה{ - }\משנה{הרוחניות העליונה }\הגדרה{- שעור\hebrewmakaf הקומה\mycircle{°} השלמה }\מקור{[ע״ר א קלא]}\צהגדרה{. }

\הגדרה{האושר\mycircle{°} הנצחי, היסוד, המגמה\mycircle{°} התכליתית של ההויה }\מקור{[עפ״י א״ק ג שמד]}\צהגדרה{.}

\paragraphs

\ערך{רום}\הגדרה{ - החלק המעולה שבכל מדריגת ההשתלמות, המתגלה ביצורים בכלל, ובאדם לפרטיו בפרטיות }\מקור{[קובץ ח צח]}\צהגדרה{.}

\הגדרה{גודל רוח האדם, גדלו ברוממות קודש, בחוסן\mycircle{°} יה }\מקור{[ע״א ד ט עא]}\צהגדרה{. }

\הגדרה{האצלת\mycircle{°} הרוח והתגלותה של הנשמה }\מקור{[ע״ר א קעב]}\צהגדרה{. }

\משנה{רום העליון }\הגדרה{- הכרת האמת\hebrewmakaf העליונה\mycircle{°} של החסות באל\mycircle{°} אמת }\מקור{[ע״ר א קנא]}\צהגדרה{. }

\הגדרה{אור\hebrewmakaf פני\hebrewmakaf מלך\hebrewmakaf חיים\mycircle{°} }\מקור{[שם קכט]}\צהגדרה{.}

\paragraphs

\ערך{רוממות האידיאלים\mycircle{°} }\הגדרה{- הפיתוח של הכח הנסתר של האצילות\mycircle{°} שבנשמת\hebrewmakaf האומה\mycircle{°} }\מקור{[ע״ה קנב]}\צהגדרה{.}

\paragraphs

\ערך{רושם עליון\mycircle{°}}\הגדרה{ - }\משנה{הרושם העליון }\הגדרה{- }\מעוין{◊}\הגדרה{ הדברים הפרטיים, המתגלים בצורה מצומצמת\mycircle{°} בעולמנו, כשהם נובעים ממקור עליון, הרושם העליון הזה חל עליהם ונקרא עליהם תמיד }\מקור{[ע״ר א קטו]}\צהגדרה{. }

\paragraphs

\ערך{רחוק }\הגדרה{- עליון, עיוני ושכלי }\מקור{[עפ״י ע״א ד ו סא]}\צהגדרה{.}

\הגדרה{ע״ע קרוב. }

\paragraphs

\ערך{״רחימו״ }\הגדרה{- ע״ע אהבה אלהית. }

\paragraphs

\ערך{רחמים }\הגדרה{- תוכן של הגברת הרצון שיש בהחפץ להגשים ולהוציא לאור עולם את כל הטוב האצור בפנימיות נפשו, לטובת הנושא הידוע או הנמצא הידוע }\מקור{[ע״ר א עט]}\צהגדרה{. }

\paragraphs

\ערך{״רחמים״ }\הגדרה{- צדקה של ויתור }\מקור{[א״ה (מהדורת תשס״ב)  ב 93]}\צהגדרה{.}

\הגדרה{ע׳ במדור פסוקים ובטויי חז״ל, מדותיו של הקב״ה אינן רחמים אלא גזרות.}

\paragraphs

\ערך{ריח }\הגדרה{- }\משנה{הריח הטוב במובנו הרוחני\mycircle{°} }\הגדרה{- המושג הכללי של ההנאות היפות }\מקור{[מ״ר 213]}\צהגדרה{. }

\ערך{ריח }\הגדרה{- }\משנה{(במובנו הרוחני, לעומת טעם\mycircle{°}) }\הגדרה{- השאיפה האידיאלית\mycircle{°} לפי ערכה לפני התגשמותה במעשה }\מקור{[שם 237]}\צהגדרה{. }

\ערך{ריח }\הגדרה{- }\משנה{תכונת הריח }\הגדרה{- הרוחניות העליונה, הנאתה של הנשמה הפנימית וענוגה\mycircle{°} המקודש\mycircle{°} }\מקור{[ע״ר א קסד]}\צהגדרה{. }

\משנה{האור הפנימי\mycircle{°} של קדושת\mycircle{°} הריח }\הגדרה{- תוכן הופעת חיי הנשמה\mycircle{°} בצורתה העליונה, במדה של חרות\mycircle{°} מכל הזיקוקים והשעבודים של הגופניות }\מקור{[שם קלח]}\צהגדרה{. }

\מעוין{◊}\הגדרה{ מיוחס אל המחשבה הפנימית }\מקור{[מא״ה א קסב]}\צהגדרה{.}

\הגדרה{ע׳ במדור משכן ומקדש, קטרת. }

\paragraphs

\ערך{ריח בושם של גן\hebrewmakaf עדן\mycircle{°}}\הגדרה{ - הכרה פנימית מהערך התוכני של הרוחני\mycircle{°}, <שעיקר התשוקה הנשמתית להחיפוש הרוחני, בו היא תלויה>}\myfootnote{ \textbf{ריח בושם גן עדן} - ע״ע מלבי״ם, שיר השירים א ב\hebrewmakaf ג.\label{6}}\הגדרה{ }\מקור{[קובץ ז קמט]}\צהגדרה{. }

\הגדרה{ע׳ במדור פסוקים ובטויי חז״ל, לריח שמניך. }

\paragraphs

\ערך{ריח רע }\הגדרה{- ע״ע סרחון.}

\paragraphs

\ערך{רליגיוזיות }\הגדרה{- }\משנה{הטבעיות הרליגיוזית}\הגדרה{ - הצמאון לאלהים שבכל עומק נשמתנו. הצמאון שאיננו פוסק מאתנו, ולא מכל המין האנושי, גם לא מכל החי ומכל היצור. <לפעמים אנו פוגשים אותו בגלוי, הוא מלטף אותנו בהמון חבתו, לפעמים הוא מסעיר את נפשנו ברעש וסער, לפעמים מרוממנו מעל לשחקים, ולפעמים יושיבנו עד דכא, ובכל עת הוא חי בקרבנו, ביודעים ובלא יודעים>. הטבעיות הצמאונית הנשגבה והקדושה, הסבה הקרובה הנפשית להתגלות כח האמונה\mycircle{°}, חוש האמונה, וההפעמות האמונית, שבתוכה יש תערובת גדולה של טוב ורע, של אורות וצללים. <כשהיא מתמלאת בכל סגולותיה\mycircle{°}, בכל עצמתה וחילה, יצא מזה כלי התפארה של יראת\hebrewmakaf ד׳\hebrewmakaf הטהורה\mycircle{°}, הנהדרה והמפוארה. לפעמים יש שנפגמה הסגולה הטבעית מקלקול חיים, מרעתה של הנהגה, מגסות רוח, מזדון ורשע, ממעשים רעים ומתכונות רעות, רפיון וקלקול היסוד\mycircle{°}> }\מקור{[עפ״י קובץ ה קמח]}\צהגדרה{.}

\הגדרה{ע״ע דת, רגש דתי. ע״ע דת, בישראל, רגש הדת בישראל. ע״ע אמונה, היסוד הטבעי של האמונה, שני צדדים בטבע האמונה: צד הגבורה שבה.}

\paragraphs

\ערך{רם }\הגדרה{- }\משנה{(לעומת גבוה\mycircle{°}) }\הגדרה{- מונח במקום עליון ואיננו מתמשך להשתפל למטה }\מקור{[ע״ר א קיב]}\צהגדרה{. }

\paragraphs

\ערך{רנה }\הגדרה{- הארת\mycircle{°} הרוח\mycircle{°}, לעלות ברעיונה\mycircle{°} ודעותיה\mycircle{°} אל הרום\hebrewmakaf העליון\mycircle{°}, אל הכרת האמת\hebrewmakaf העליונה\mycircle{°} של החסות\mycircle{°} באל\mycircle{°} אמת }\מקור{[ע״ר א קנא]}\צהגדרה{.}

\הגדרה{הרוממות הנפשית הנכונה להיות מתלוה עם המוסיקה האלהית\mycircle{°}, בהיות האדם יודע שעם מלוי משאלו, ימלא ג״כ משאל הצדק\mycircle{°} שהוא עצם רנת ישרים\mycircle{°} }\מקור{[עפ״י ג״ר ז 6 (פנק׳ ד שפא)]}\צהגדרה{.}

\משנה{רנה קדושה\mycircle{°}}\הגדרה{ - הבעת הקדושה, בשפה ברורה, בנעימה ונגינה מוסיקלית, החודרת ומחשפת את היופי\mycircle{°} הטהור\mycircle{°} הפנימי\mycircle{°}, ומגלה (את) תכונתה הקדושה של נשמת\hebrewmakaf האומה\mycircle{°}, השרויה בתוכיותו של כל יחיד ויחיד מיחידי בניה, בכליל גוני\mycircle{°} אורותיה\mycircle{°} המזהירים }\מקור{[עפ״י ע״ר א קנז]}\צהגדרה{.}

\הגדרה{ע׳ במדור מצוות, הלכות, מנהגים וטעמיהן, תפילה, התפילה המעשית. ע״ע בנספחות, מדור מחקרים, רנה ותפילה. ע״ע זמר. ע״ע הלול.}

\paragraphs

\ערך{רננה }\הגדרה{- הרמת הקול בתנועה המקבילה לשמחה\mycircle{°} }\מקור{[ע״ר א רכא]}\צהגדרה{.}

\ערך{רננה}\myfootnote{ \textbf{התעוררות חיצונית וכו׳ ברינה של תורה} - ע׳ זוהר ח״א קסג. ח״ג ח.: \label{7}}\הגדרה{ - התעוררות חיצונית המביאה לסקור את הפנים. הרמת קול, ברינה של תורה המעוררת את הכונה וההרגשה הפנימית }\מקור{[עפ״י קבצ׳ ב קח\hebrewmakaf ט]}\צהגדרה{.}

\הגדרה{ר׳ צהלה.}

\paragraphs

\ערך{רע }\הגדרה{- האף\mycircle{°} והזעם, שמתגלה בעולם }\מקור{[ע״ר א קפח]}\צהגדרה{.}

\הגדרה{חרבן והריסה כיעור וניוול}\צהגדרה{ }\מקור{[פנק׳ א שכט]}\צהגדרה{.}

\הגדרה{שאיפת הריסה, חורבן, החשכה והשפלה }\מקור{[א״ק ב תעו]}\צהגדרה{.}

\משנה{נטיות הרע }\הגדרה{- (נטיות) החורבן, הרשעה\mycircle{°} והשיקוע בשפל }\מקור{[ע״ר ב שנח]}\צהגדרה{. }

\משנה{שאיפת רע }\הגדרה{- התנשאות עצמית, התפשטות חיל\mycircle{°} ושטף, מבלי הבט אל היושר\mycircle{°}, הצדק\mycircle{°}, המטרה העדינה }\מקור{[א״ק ב תצה]}\צהגדרה{.}

\הגדרה{ע׳ במדור נפשיות, יצר הרע. ע״ע טוב. ע׳ במדור מדרגות והערכות אישיותיות, ״אדם רע״. }

\paragraphs

\ערך{״רע״ }\הגדרה{- היפוך הטוב\mycircle{°}. כשמה שראוי להשפיע ולהתלבש באיברים ופעולות אינו משפיע }\מקור{[עפ״י מא״ה ד קסג]}\צהגדרה{.}

\הגדרה{ע״ע טוב.}

\paragraphs

\ערך{רע - }\משנה{(ענין הרע בעולם)}\myfootnote{ ע׳ קל״ח פ״ח לרמח״ל, פתח ב ופתח ג.\label{8}}\הגדרה{ - מציאותו היא מידי ד׳\mycircle{°} הטוב\mycircle{°} כדי לעקרו ולמגרו ולהעמיד את יסוד המציאות על הטוב והטוהר\mycircle{°} היותר נשגב\mycircle{°} }\מקור{[ע״א ג ב פד]}\צהגדרה{.}

\הגדרה{ע׳ במדור מונחי קבלה ונסתר, ״הרע מבחין את הטוב״.}

\paragraphs

\ערך{רעיון אלהי }\הגדרה{- }\משנה{הרעיון האלהי }\הגדרה{- כל היש כולו, ערכו, מהותו, תבניתו, גילוייו, עזוזו\mycircle{°} ועצמתו, המעופף ומשוטט, הולך וזורם, פועל ומתגלה בכל המערכה\mycircle{°} ההוייתית }\מקור{[עפ״י ע״א ד ט עא]}\צהגדרה{.}

\paragraphs

\ערך{רענן }\הגדרה{- יונק\mycircle{°} ממקור חיים }\מקור{[עפ״י א״ק ג רנח]}\צהגדרה{.}

\paragraphs

\ערך{רפואה }\הגדרה{- }\משנה{(לעומת ארוכה\mycircle{°}) }\הגדרה{- רפואה חריפה ומהירה למחלה חדה מתפרצת }\מקור{[מ״ר 473]}\צהגדרה{.}

\הגדרה{רפואה פתאומית הבאה על ידי נתוח או על ידי הקזה }\מקור{[שם 371]}\צהגדרה{.}

\הגדרה{רפואה חיצונית פעלית. דרך רפואת המחלות החיצוניות ברוב הענינים, שמחוץ לטבע הגוף, כמו חתוך וחבוש וכדומה }\מקור{[עפ״י משפט כהן, פתיחה טו]}\צהגדרה{.}

\paragraphs

\ערך{רצון ההויה }\הגדרה{- }\משנה{רצון ההויה כולה }\הגדרה{- אור\mycircle{°} הרצון של הופעת\mycircle{°} גלוי האלהות\hebrewmakaf המוחלטה\mycircle{°} }\מקור{[א״ק ב תקסה]}\צהגדרה{.}

\משנה{רצון כללי }\הגדרה{- אור\hebrewmakaf ד׳\mycircle{°} וכבודו\mycircle{°}, אשר בנשמת היקום כולו }\מקור{[שם ג לט]}\צהגדרה{.}

\הגדרה{הרצון העולמי, אור חי\hebrewmakaf העולמים\mycircle{°}, שבו כלולים כל המאויים }\מקור{[שם נ]}\צהגדרה{.}

\משנה{הרצון הכללי}\הגדרה{ - שפע הרצון המתעלה מעל לכל לראש, שפעת הרצון המחולל כל, הבורא כל עולמים, המחדש בכל יום תמיד מעשה בראשית}\צהגדרה{ }\מקור{[שם נד]}\צהגדרה{.}

\משנה{רצון העליון הכללי }\הגדרה{- רצון היוצר, אשר נעשה בחפצו כל }\מקור{[שם נב]}\צהגדרה{.}

\paragraphs

\ערך{רצון }\הגדרה{- }\משנה{אור הרצון }\הגדרה{- אור הפנימיות, הזורח\mycircle{°} בתורה, המייחש את האדם והעולם אל רצון\hebrewmakaf ד׳\mycircle{°}, יוצר כל ברצונו}\צהגדרה{ }\מקור{[א״ק ג לא]}\צהגדרה{.}

\משנה{הרצון הנעלה }\הגדרה{- עז\mycircle{°} המוסר\mycircle{°} המאיר מהאספקלריא של מעלה, המשים ביסודות, בכחות ובתולדות הנדחפים אל כל רוח, אור חיים שיטה וסדר, להפריח טוב, לסור ממוקש רע }\מקור{[א״ש ט ג]}\צהגדרה{.}

\משנה{הרצון}\הגדרה{ - הכח המרכזי של החיים, היסוד המניע את כל גלגלי החיים}\צהגדרה{ }\מקור{[א״ק ג סא]}\צהגדרה{.}

\paragraphs

\ערך{רצון העולם }\הגדרה{- המדרגה המתגלה בתור נפש\hebrewmakaf החיים\mycircle{°} שבהויה, רצון פועל ושואף, שסעיפיו מתגלים בכל, בדצח״מ, בכל פרטים ופרטי פרטים, ובכללי כללים, שמגמתו להתעלות\mycircle{°} }\מקור{[עפ״י א״ק ב שסט]}\צהגדרה{.}

\הגדרה{ע׳ במדור משיח וגאולה, ״רוחא דמלכא משיחא״.}

\paragraphs

\ערך{רציונלי }\הגדרה{- ע׳ בנספחות, מדור מחקרים.}

\paragraphs

\ערך{רקבון }\הגדרה{- פירוד יסודות של מדוה ומחלה, חולשה והרס }\מקור{[ע״א ד ג ד]}\צהגדרה{. }

\paragraphs

\ערך{רקיע }\הגדרה{- כללות כל החוג והמקום\mycircle{°} שכל הצבא הגדול (שהמה הברואים כולם) מתקומם בו }\מקור{[מ״ש שנב (ה׳ קעז)]}\צהגדרה{. }

\הגדרה{כלל העולם החיצוני <בערך אל עצם הכדור הארצי> }\מקור{[ע״א ב ט קמא]}\צהגדרה{.}

\צמקור{(השמים\mycircle{°}) מצד חומרם וחיצוניותם [פנק׳ ד תמד]. }

\paragraphs

\ערך{רשמיות }\הגדרה{- הסכמה מקפת כללית מכחם של האנשים העומדים תחת איזו חטיבה\mycircle{°} מיוחדת, המובעת באופן גלוי }\מקור{[עפ״י אג׳ ב ט]}\צהגדרה{.}

\paragraphs

\ערך{רֶשע - }\הגדרה{חפץ התגברות ללא שום אידיאל של צדק\mycircle{°} }\מקור{[א׳ קמג]}\צהגדרה{.}

\paragraphs

\ערך{רֶשע המוחלט }\הגדרה{- הרשעה\mycircle{°} בתור המגמה האחרונה }\מקור{[ע״א ד יב כז]}\צהגדרה{.}

\paragraphs

\ערך{רשעה }\הגדרה{- תהום החמריות והתאוות הגסות\mycircle{°} }\מקור{[ע״א ד יב כו]}\צהגדרה{.}

\הגדרה{ע״ע טומאה.}

\paragraphs

\ערך{רשעה }\הגדרה{- }\משנה{הרשעה }\הגדרה{- התוכן המשטין\mycircle{°} והמחבל }\מקור{[ע״א ד ט קח]}\צהגדרה{.}

\ערך{הרשעה }\הגדרה{- תשוקת הרע\mycircle{°}, הזעם, הרצח, הגסות\mycircle{°}, וכל סעיפיהם }\מקור{[א״ק ב תעח]}\צהגדרה{.}

\משנה{רשעה עולמית }\הגדרה{- }\משנה{הרשעה העולמית }\הגדרה{- כל יסוד הרע, האחוז כל כך בשרשים ענפים ומתרחבים, ביצירה הגדולה}\צהגדרה{. }\הגדרה{שאיפת הרע הגמורה, חפץ השלטתו בכל ערכי החיים והעולם }\מקור{[עפ״י א״ק ב תפח, תפט]}\צהגדרה{.}

\מעוין{◊ }\משנה{יסוד הרשעה}\הגדרה{ המתפצלת לעבודה\hebrewmakaf זרה\mycircle{°} ולמינות\mycircle{°}, הוא בא לבצר מקום לסיגי החיים, למותרות המציאותיות שבהויה ובאדם, במוסר ובחפץ, במפעל ובהנהגה, לתן להם גודל ושלטון בתוך הטוב והקודש; לא לטהר את הקודש, כ״א לטמאו ולסאבו }\מקור{[עפ״י א׳ לב]}\צהגדרה{.}

\הגדרה{ע׳ במדור מצוות, הלכות, מנהגים וטעמיהן, מחית עמלק. ע״ע טומאה. ע׳ בנספחות, מדור מחקרים, רשעה, מינות, נוצריות, נצרות. }\mylettertitle{ש}

\paragraphs

\ערך{שְׁאוֹל }\הגדרה{- הירידה\mycircle{°} של החיים, במיעוט הדמות\mycircle{°} החיונית, במחיקת כשרונותיה, המשאירה רק את הנקודה של החפץ, הרצון לחיות, השאיפה להתקיים, בלא כל ספוק אמצעי איך למלא את הדרישה. נשאר רק המשאל לבד בלא אפשריות ההפקה של המשאל, רק הרשימה החיונית במקורה - }\צהגדרה{השאול }\מקור{[ע״ר א קפז]}\צהגדרה{. }

\משנה{מצרי שאול הרוחניים\mycircle{°}}\הגדרה{ - כשהצמאון הנורא לשלמות העליונה\mycircle{°} בוער בקרב האדם, פגמיו המוסריים\mycircle{°} כולם מתבלטים אצלו, ואינו מוצא את נפשו בעלת יכולת להושיע\mycircle{°} את עצמה }\מקור{[עפ״י א״ק ג צד]}\צהגדרה{. }

\משנה{מוקדי שאול }\הגדרה{- יסודי הבליה וההעדרים בכל גווניהם\mycircle{°} }\מקור{[א״ק ב רפה]}\צהגדרה{. }

\הגדרה{ע״ע גיהנם. ע׳ במדור פסוקים ובטויי חז״ל, ירידת בור. }

\paragraphs

\ערך{שאיפה }\הגדרה{- }\משנה{במובן האצילות\mycircle{°} שלה }\הגדרה{- מדת הקדושה\hebrewmakaf העליונה\mycircle{°}, מזיגה של השכל\mycircle{°} והרצון\mycircle{°},  כשהם מתאחדים ביחד לצד הטוב\mycircle{°} האידיאלי\mycircle{°} }\מקור{[א״ק ג פה]}\צהגדרה{. }

\paragraphs

\משנה{שבח }\צהגדרה{- מובנו כפול: פאר ההילול ופעולת השכלול }\צמקור{[ע״ר ב תט].}

\ערך{שבח }\הגדרה{- }\משנה{(לד׳, כחובתנו מבחינת היותנו החלק הגורל והירושה, עם קרובו ומחשבת תחילתו) }\הגדרה{- מורה הכרה בהשלמות, והסגולות הרוממות\mycircle{°} וגודל העלוי של המיטיב }\מקור{[ע״ר א קח]}\צהגדרה{. }

\הגדרה{הוראת הכרת כבוד עליון }\מקור{[ע״א ב ט ד]}\צהגדרה{.}

\צהגדרה{הכרת השלמות והגדולה\mycircle{°} העצמית והנפלאה של המקור העליון ב״ה }\צמקור{[שם ב תסא]. }

\צהגדרה{הכרת עצם המעלה והגדולה של הבורא ב״ה }\צמקור{[שם תסב]. }

\ערך{שבח }\הגדרה{- }\משנה{(לד׳, כהתיחשות היחידית והמפורטת מצד הבריאה) }\צהגדרה{- העלאת והשבחת ערך\mycircle{°} }\צמקור{[ע״ר ב תסא]. }

\הגדרה{העליה ממעלה למעלה יותר עליונה ממנה }\מקור{[ע״א ג ב לו]}\צהגדרה{. }

\הגדרה{הוספת אומץ להכיר גדולתו\mycircle{°} ושלמות הנהגתו של יוצר הנשמה\mycircle{°} שבזה תדמה לקונה, שתישא פרי, להיות חושק לדבקה\mycircle{°} בדרכי\hebrewmakaf ד׳\mycircle{°} הישרים\mycircle{°} }\מקור{[עפ״י ע״א א א קלה]}\צהגדרה{. }

\הגדרה{ההלול\mycircle{°} בפעולתו על האדם ועל כל היקום כולו, להעלות אותו, להשביחו בהכרה זו שהוא מרומם את כל שפעת\mycircle{°} הטוב\mycircle{°} היורד לההויה, אל מקורה העליון\mycircle{°}, להדרת\mycircle{°} קודש\mycircle{°} }\מקור{[ע״ר א קצג]}\צהגדרה{. }

\הגדרה{באור בהרחבה של עלוי\mycircle{°}, המביא לידי התגלות\mycircle{°}, את היתרון המשובח של התהילה\hebrewmakaf האלהית\mycircle{°} בההכרה של הבינה והמדע הרוחני\mycircle{°} המזהיר }\מקור{[עפ״י שם קצז]}\צהגדרה{. }

\הגדרה{העדון האצילי\mycircle{°} האיכותי, השורה בכל המהלך של חיי ההויה, באור\mycircle{°} החיים\mycircle{°} השופעים מחסד\mycircle{°} אל וטובו }\מקור{[שם קצח]}\צהגדרה{. }

\הגדרה{גילוי ההתעלות\mycircle{°} וההשתלמות, שבמהלך הבריאה\mycircle{°} וההויה, שיש בו גם כן מענין השקטה, והורדת הדרגה (מלשון ״משביח שאון ימים״), המשקיט את המית הגדולה כשאנו באים לפאר בכל ביטוי של גודל את מלך הכבוד ב״ה. שהרי מצד האמת הלא לו דומיה\hebrewmakaf תהילה\mycircle{°}, ואין שום רעיון\mycircle{°}, וקל\hebrewmakaf וחומר שום דבור מתאים לפי רוממות התהילה בתפארתה }\מקור{[עפ״י שם, וע׳ שם ב תסב]}\צהגדרה{. }

\הגדרה{ע׳ במדור שמות כינויים ותארים אלהיים, ״משובח״. ע״ע רנה. }

\paragraphs

\ערך{שבטים }\הגדרה{- }\משנה{השבטים}\הגדרה{ - הבונים הפרטיים של כלל ישראל להיות על ידם לגוי קדוש הראויים לבחירה אלהית תמידית }\מקור{[מ״ש ריב]}\צהגדרה{.}

\paragraphs

\ערך{שביל }\הגדרה{-}\משנה{ שביל רוחני }\הגדרה{- עולם\mycircle{°} }\מקור{[קובץ ה קנב]}\צהגדרה{.}

\paragraphs

\ערך{שגגה }\הגדרה{- }\מעוין{◊}\הגדרה{ שגגה יוצאת לעולם, מפני שהיסוד הטוב שעליו בנויה אותה הזהירות המתיחשת להענין ההוא, איננו קנוי יפה בנפש קנין טבעי, אע״פ שהוא חפץ להשמר ממנו מצד הסכמתו וחפצו לקבל עליו את עול המצוה או המדה ההיא מ״מ השגגה רובצת לפתחו }\מקור{[ע״א ג ב רסב]}\צהגדרה{. }

\paragraphs

\ערך{שגוב }\הגדרה{- ע״ע משגב. ע״ע נשגב. }

\paragraphs

\ערך{שוא }\הגדרה{- הפכם של האמת\mycircle{°} והיושר\mycircle{°} }\מקור{[עפ״י ע״א ב ט שכה]}\צהגדרה{. }

\הגדרה{ע״ע מרמה. ע״ע שקר. ע״ע כזב.}

\paragraphs

\משנה{שואה }\צהגדרה{- }\צמשנה{השואה }\צהגדרה{- ניתוחנו הנורא, בנטות\hebrewmakaf יד\hebrewmakaf ד׳\mycircle{°} עלינו בתשפוכת חמתו להוציאנו מטומאת ארצות העמים ומתפוצות גלויותינו בתוך מחשכיהן, בהתגלות ערוות הגויות על ידי אומת התרבות האירופית הארורה, בהשמדת שליש גופנו ומבחרו }\צמקור{[ל״י א צד].}

\צהגדרה{אבדנות רבבות אלפינו וכל אשר אתם, בסערת נתוק גופו של כלל ישראל ממדבריות העמים }\צמקור{[ל״י א פ].}

\צהגדרה{שלילת ההריסה המחלטת, בכל נוראות ממשיותה, של מציאות ישראל בגלות\mycircle{°}, של אפשרות היותו בפרוד\hebrewmakaf פיזורה <השלילה המביאה את החיוב המחודש (של זווג ישראל וארצו) בגלוי העלייה מן טומאת אֲבַדּוֹנָה אל טהרת קיומו במקום קדשו> }\צמקור{[ל״י א ק]. }

\צהגדרה{פקודת היציאה מטומאת הגלות והעלייה לארץ\hebrewmakaf החיים במסירות נפשותיהם וגופותיהם של נידחי\hebrewmakaf ישראל }\צמקור{[עפ״י ל״י א ק].}

\צהגדרה{התמוטה הגמורה, העקירה המוחלטת, הנתוק הסופי, בכל פרפור זעזועו, של כנסת\hebrewmakaf ישראל\mycircle{°} ממציאותה בגויים ובארצותיהם, של גופותיה ונשמותיה, של גדוליה וקטניה, של רכושה ותרבותה, של חמריותה ורוחניותה, של קודשיה וחוליה. יד\hebrewmakaf ד׳\mycircle{°} החזקה והאיומה, במלא תשפוכת החמה ו״קשיות המלכות״ ההמנית, בהעברתה אל שפת המציאות האכזרית את תכן תודעתם בכל מערומי אמתותה של גדולי ישראל, אדיריו והמוניו בתודעת גרותם וזרותם בארצות לא\hebrewmakaf להם – שבה היתה מוכנה וגנוזה צפיית שיכותם אל הארץ הזאת אשר להם; והוצאתה אל הפועל הממשי (של יד\hebrewmakaf ד׳ זאת) את אותה התמוטה והעקירה, את אותו הנתוק, של האומה הישראלית מהיות בגויים ובארצותיהם, בדרך התקומה, ההנחה, ההתחדשות השרשית בארצה, נחלת ירושתה לעולמים}\צמקור{ [עפ״י ל״י א קב-ג (מהדורת בית אל קלה)].}

\paragraphs

\ערך{שוהם }\הגדרה{- }\משנה{(מציין את) }\הגדרה{- המחשבה\hebrewmakaf היסודית\mycircle{°}, נקודת גובהה של התגלותה\mycircle{°} של הנשמה\mycircle{°} }\מקור{[א״ק ג קנט, קס]}\צהגדרה{. }

\paragraphs

\ערך{שועה }\הגדרה{- ההבעה היותר חודרת ויוצאת מעומק הלב }\מקור{[ע״ר ב סז]}\צהגדרה{. }

\paragraphs

\ערך{שועה}\הגדרה{ - }\משנה{(שועת הנשמה בפנימיות תהומה אל ד׳)}\הגדרה{ - }\מעוין{◊}\הגדרה{ באה מתוך כמיהה, מתוך פניית הנשמה אל הרוממות האלוהית העליונה. מפני העריגה האלהית, מתוך הצמאון הנורא לאור אלהי אמת, מתוך השקיקה לאור הנעים לפאר חי העולמים. מתוך התשוקה אל הגודל והאור }\מקור{[עפ״י ע״ר א קס]}\צהגדרה{. }

\הגדרה{ע״ע צעקה.}

\paragraphs

\ערך{שורש }\הגדרה{- תעודה }\מקור{[עפ״י א״י לא]}\צהגדרה{. }

\paragraphs

\ערך{שורש האמונה }\הגדרה{- ע״ע אמונה, עיקר האמונה. }

\paragraphs

\ערך{שורשי הויותינו}\הגדרה{ - ענפי נשמותינו, קוי החיים השופעים ממהותנו, ומקורות החיים המפכים עלינו, עד הביאם אותנו אל המהותיות הפרטית שלנו }\מקור{[עפ״י קובץ ז קמד]}\צהגדרה{.}

\paragraphs

\ערך{שורשים}\הגדרה{ - }\משנה{(לעומת ענפים\mycircle{°}) }\הגדרה{- כללים (לעומת פרטים) }\מקור{[פנ׳ ג שנ]}\צהגדרה{.}

\paragraphs

\ערך{שחקים }\הגדרה{- כלי ההשפעה שהארץ מקבלת מהם, מצד הסיבוב של ההשפעה הרוחנית. }\צהגדרה{<מלשון שחיקה וטחינה, שענין השחיק(ה) הוא להחזיר את הדבר (ה)גס\mycircle{°} שיהיה דק. כמו כן צפן השי״ת> }\הגדרה{כח נפלא, שענייני מעשינו שהם גסים וחומריים, מתהפכים לדקים רוחניים, פועלים ברכה בעולם השכלים וסבות ההשפעה הרוחנית ג״כ }\מקור{[פנק׳ ד תל-תלא]}\צהגדרה{.}

\הגדרה{ע״ע שמים. ע׳ במדור פסוקים ובטויי חז״ל, שחקים, שבו רחיים שוחקות מן לצדיקים.}

\paragraphs

\ערך{שחר }\הגדרה{- החילוף בין מדרגת הלילה\mycircle{°} למדרגת היום\mycircle{°} }\מקור{[מ״ש קט]}\צהגדרה{. }

\paragraphs

\ערך{שיח }\הגדרה{- }\משנה{יסוד השיח}\myfootnote{ \textbf{שיח }- ע׳ מעלות התורה לר״א אחי הגר״א, מהד׳ ירושלים תשמ״ט, עמ׳ יט\hebrewmakaf כא. \textbf{עכשיו תורה נתתי לכם, לעתיד לבא חיים אני נותן לכם} - בשמו״ר פר׳ מח ד: ״בעוה״ז היתה רוחי נותנת בכם חכמה אבל לעתיד לבא רוחי מחיה אתכם שנאמר (יחזקאל לז) ונתתי רוחי בכם וחייתם״. ע״ע ילקו״ש שה״ש א, רמז תתקפא. ר׳ במדור פסוקים ובטויי חז״ל, ״בעוה״ז, תורה נתתי לכם ולעתיד, בעוה״ב, חיים אני נותן לכם״.\label{1}}\הגדרה{ - }\מעוין{◊}\הגדרה{ מדתו של יצחק\mycircle{°}, <העומדת למעלה גם ממדרגת שיר\mycircle{°}>. התהוות החיים בשלמותם, בהתעלות\mycircle{°} כל החול\mycircle{°} אל הקודש\mycircle{°}, כל הבטל אל הנשגב\mycircle{°} והמרומם מאד. ״עכשיו תורה נתתי לכם, לעתיד לבא חיים אני נותן לכם״\mycircle{°}. <}\משנה{ש}\הגדרה{כל, }\משנה{י}\הגדרה{כולת, }\משנה{ח}\הגדרה{יים>. התעלות ההשקפה עד כדי הכרה וציור פנימי של מקומן\mycircle{°} של כל הדברים, וכל הדברים הבטלים יתעלו, יצאו מכלל ארור ויבאו לכלל ברוך }\מקור{[עפ״י א״ק ג קו]}\צהגדרה{.}

\הגדרה{ע׳ בנספחות, מדור מחקרים, שיר וזמר ההפרש ביניהם. }

\ערך{שיחה }\הגדרה{- }\משנה{״שיחו בכל נפלאותיו״}\myfootnote{ תהילים קה ב\label{2}}\הגדרה{ - }\מעוין{◊}\הגדרה{ מתוך המפעל האצילי\mycircle{°} הטהור\mycircle{°}, היוצר בתוכה של הנשמה\mycircle{°} את השירה\mycircle{°} המחשבתית, ואת הזמרה\mycircle{°} הרגשית, בא אחר כך המהלך של ה}\משנה{שיחה}\הגדרה{ הרחבה, הפרוזית\mycircle{°}, המתרחבת והולכת במהלך של חיים רחבים ומסתעפים\mycircle{°} על כל המון הפליאות, ההולכות ומתגלות לעיני כל מעין ובוחן במפלאות תמים דעים }\מקור{[ע״ר א ר]}\צהגדרה{. }

\ערך{שיחה}\myfootnote{ \textbf{מה שצומח מאיליו} - רש״ר תהילים יד ה: ״לשוחח מלשון שיח, צמח״.\label{3}}\הגדרה{ - מה שצומח מאיליו בלא השתדלות מפרי הרוח, <וכן שיח השדה הבא מכח הצמיחה הטבעית הכוללת> }\מקור{[פנ׳ נו]}\צהגדרה{. }

\paragraphs

\ערך{שיחה }\הגדרה{- הרבות בדברים בעד גרעין קטן של רעיון }\מקור{[ע״א ג ב צט]}\צהגדרה{. }

\משנה{יסוד השיחה }\הגדרה{- מצב המורה על קניית השלימות דוקא מחוץ לנפש, למצא האושר אצל אחרים ולא אצל עצמו, <כי אושר הנפש הפנימי יבא בהגיונה עם עצמה> }\מקור{[ע״א א ה י]}\צהגדרה{. }

\משנה{המצב המביא את השיחה }\הגדרה{- בעת תשפל הנפש בכחה הציורי\mycircle{°} ולא תמצא בקרבה איך למלאות נפשה בציורים מרהיבים שהם ממלאים אותה שמחה\mycircle{°}, בשפלות ידים תחזור על ציוריה הנמוכים ההולכים ברפיון ובלא שום כח אדיר }\מקור{[עפ״י ע״א ג ב צט]}\צהגדרה{. }

\paragraphs

\ערך{שיחה }\הגדרה{- במדור מצוות, הלכות, מנהגים וטעמיהן, תפילה, שיחה. }

\paragraphs

\ערך{שילוב }\הגדרה{- תליית החלקים זה בזה }\מקור{[עפ״י א״א 63]}\צהגדרה{. }

\paragraphs

\ערך{שינה }\הגדרה{- הפסק ההרגשה, המביאה להיות ההרגשה עצמה מושלמת ובריאה }\מקור{[ע״א ב ט קיב]}\צהגדרה{.}

\מעוין{◊ }\הגדרה{משחררת את האדם מהשעבוד של העולם החיצוני, ועל ידי זה יכול הוא לפשט את נפשו בהפשטה פנימית, בכל אורך שיעור קומתו הרוחנית. ביחש החיים שכלפי העולם החיצוני נפשו מצטמצמת, אבל ביחש חייה הפנימיים הרי היא מתפשטת, מאריכה את מהותה. <היא יורדת עד כל עומק התחתית שבחיים מצד זה, }\צהגדרה{שכל זמן שהאדם הוא קשור עם העולם החיצוני אין העולם הסביבי מניח אותו לרדת כל כך לעומק התחתית}\הגדרה{, ולמעלה הרי הוא עולה לעומק הרום של החיים שלו, והרי הוא תופס את כללות החיים כולם>}\צהגדרה{ }\מקור{[א״ק ג שד]}\צהגדרה{. }

\משנה{שינה }\הגדרה{- (}\צהגדרה{לעומת תנומה}\הגדרה{) - מנוחה נפשית פנימית, המעתיקה מן העולם הסובב ומכל רשמיו המרעישים בחקיקתם. השינה העמוקה, הבאה מתוך תביעתה של המרגעה הנפשית. מנוחת השינה מאזרת חיל את הרוח }\מקור{[עפ״י ע״ר א עו]}\צהגדרה{.}

\הגדרה{ע״ע תנומה.}

\paragraphs

\ערך{שיר }\הגדרה{- }\מעוין{◊ }\הגדרה{מורה על השלמות, שהשמחה\mycircle{°} וההשגה\mycircle{°} כשהן מצורפות, הוא השלמות האמיתית, וכשהן במילואן ימצא ה}\צהגדרה{שיר }\צהגדרה{[ע״א א א ה]}\הגדרה{.}

\paragraphs

\ערך{שיר }\הגדרה{- השגת\mycircle{°} הנפש דרכי חייה האמיתיים בעת הבהקת החכמה בשלמות אורה\mycircle{°} }\מקור{[עפ״י פנק׳ ג כא (מא״ה ב רסט\hebrewmakaf רע)]}\צהגדרה{. }

\הגדרה{הדעת\mycircle{°} הכולל ציורי\mycircle{°} ה}\צהגדרה{ש}\הגדרה{כל\mycircle{°} ה}\צהגדרה{י}\הגדרה{כולת\mycircle{°} וה}\צהגדרה{ר}\הגדרה{צון\mycircle{°} ביחד }\מקור{[עפ״י פנק׳ ג כב (מא״ה ב רע)]}\צהגדרה{.}

\הגדרה{הארת השכל בחדרי הלב, והתמלאות אור דעת\mycircle{°} ויראת\hebrewmakaf ד׳\mycircle{°} אוצרו }\מקור{[עפ״י פנק׳ ג כו]}\צהגדרה{.}

\הגדרה{ע״ע שירה. ע״ע זמר. ע״ע שיח. ע׳ בנספחות, מדור מחקרים, שיר וזמר ההפרש ביניהם. }

\paragraphs

\ערך{שיר המלאכים }\הגדרה{- ע׳ במדור מלאכים ושדים.}

\paragraphs

\ערך{שירה }\הגדרה{- הרוח\mycircle{°} שמצטייר בנפש\mycircle{°}, בלבבו בכוחו המדמה\mycircle{°} השלם, שמקבל משיר\mycircle{°}, משלשת הציורים ה}\משנה{ש}\הגדרה{כל\mycircle{°}, ה}\משנה{י}\הגדרה{כולת\mycircle{°} וה}\משנה{ר}\הגדרה{צון\mycircle{°} ביחד, לפעול טוב וחסד }\מקור{[עפ״י מא״ה ב רע]}\צהגדרה{. }

\הגדרה{מקור החכמה\mycircle{°}, הבאה דרך הופעה\mycircle{°}, דרך המשכה\mycircle{°} של חסד\mycircle{°} מתכונת הנשמה\mycircle{°} <והחכמה באה אחריה לפרט את פרטיה> }\מקור{[קובץ ח קכד]}\צהגדרה{. }

\מעוין{◊ }\משנה{תכונת השירה}\הגדרה{ מתאימה להרגשה של התגלמות אורים\mycircle{°}, של הגבלת\mycircle{°} נעים זמרה כדי להכנס במשטר וקצב }\מקור{[עפ״י ר״מ סד]}\צהגדרה{. }

\מעוין{◊ }\משנה{השירה באה}\הגדרה{ כשהזמר\mycircle{°} הולך ומתפשט, הולך ומתגלה\mycircle{°}, הולך ומתחבר אל המחשבה\hebrewmakaf המיושבת\mycircle{°}, אוצר הדבור והבטוי בגובה גודלו }\מקור{[עפ״י ע״ר א קפו]}\צהגדרה{. }

\מעוין{◊ }\משנה{השירה}\הגדרה{ מורה על התרוממות הנפש למשאות נעלות, שמשקפת בהיותה מוכתרת בשכלה בבהירותו }\מקור{[ע״א ב 171]}\צהגדרה{. }

\מעוין{◊}\הגדרה{ כל האמת שאינה יכולה להתבטא בשום פרזיות, מתבטאת היא ב}\צהגדרה{שירה}\הגדרה{. ה}\צהגדרה{שירה}\הגדרה{ היא מכון להאמת\hebrewmakaf העליונה\mycircle{°} שממעל להחיים, שהיא יכולה לשיר את שירת\hebrewmakaf החיים\mycircle{°} }\מקור{[קבצ׳ ב קיב]}\צהגדרה{. }

\ערך{שירה }\הגדרה{- ההבעה השכלית\mycircle{°} העליונה\mycircle{°}, היוצאת מתוך ההסתכלות הרחבה והעמוקה באור\mycircle{°} אל עליון ופליאות מפעליו }\מקור{[ע״ר א ר]}\צהגדרה{. }

\צהגדרה{גילוי ההבנה האמיתית. השכל העליון. מרחב רוחני עילאי, עליוני, אלוהי. עילאיות נפשית, פסיכולוגית, פנימית. הגדלות במידה היותר גדולה }\צמקור{[עפ״י מה״ה ג, מאמר שני, הקדמה].}

\הגדרה{ההשגה היותר פנימית\mycircle{°}, היותר חודרת במעמקי מהות המושג, בתוכנו הפנימי }\מקור{[עפ״י א״א 66]}\צהגדרה{.}

\משנה{הדר השירה}\הגדרה{ - התוך של החיים והאמת שבהם }\מקור{[עפ״י א״א 66]}\צהגדרה{.}

\ערך{שירה חדשה}\הגדרה{ - הסתכלות בדבר פרטי מצד תוכן יישותו הפנימית, המתמלא מעולם הזוהר והאור העליון הבלתי מוגבל }\מקור{[עפ״י א״ת ד ד]}\צהגדרה{.}

\משנה{השירה האלהית }\הגדרה{- הרעיון הגדול של התוך המלא של אמונת\hebrewmakaf אלהים\mycircle{°} בצורתה הרוממה, המאשרת את החיים }\מקור{[קובץ א קע]}\צהגדרה{.}

\הגדרה{ע״ע שיחה, ״שיחו בכל נפלאותיו״.  ע׳ במדור פסוקים ובטויי חז״ל, ובלילה שירה עמי.}

\paragraphs

\ערך{שירה }\הגדרה{- }\משנה{חיי הפיוט והשירה }\הגדרה{- ע״ע פיוט. }

\paragraphs

\ערך{שירת החיים}\הגדרה{ - התוכן הנשגב האצור באמונה\mycircle{°}. מכון ששם האמת\hebrewmakaf העליונה\mycircle{°} שוכנת בכל יפעתה\mycircle{°} והדרה\mycircle{°} }\מקור{[קבצ׳ ב קיב]}\צהגדרה{.}

\paragraphs

\ערך{שירת ישראל}\הגדרה{ - התורה\mycircle{°} ושורשה העליון }\מקור{[קובץ ח קכד]}\צהגדרה{. }

\paragraphs

\ערך{שכינת האומה }\הגדרה{- רוח החיים של השאיפה האלהית המקושרת בתוכן הסגנון הצבורי של הצורה הלאומית }\מקור{[א׳ קו]}\צהגדרה{. }

\הגדרה{ע׳ במדור מונחי קבלה ונסתר, ״שושנה עליונה״. ע״ע אידיאה לאומית.}

\paragraphs

\ערך{שכל }\הגדרה{- }\משנה{(השכל האלהי) }\הגדרה{- הגלגלתא\mycircle{°}, המחשבה המגמתית העליונה }\מקור{[עפ״י א״ק ג עג\hebrewmakaf ד]}\צהגדרה{. }

\ערך{השכל העליון\mycircle{°}}\הגדרה{ - מקור החכמה והמדע ב״ה }\מקור{[ע״א ג ב רצה]}\צהגדרה{. }

\הגדרה{הזוהר הנשמתי\mycircle{°} ביסודו }\מקור{[א״ק א יא]}\צהגדרה{. }

\הגדרה{מקור החיים\mycircle{°} }\מקור{[עפ״י א׳ כט, א״ק ב תצ]}\צהגדרה{. }

\צהגדרה{העליה המעולה בחיי המוסר\mycircle{°}, של חיי האדם היחידי ובהליכות עולמים, מביאה לכך }\צהגדרהמודגשת{שהשכל העליון}\צהגדרה{ נמצא בתור גילוי עליון של סכום החיים כולם, וכל אשר מתחת לו הרי הם ענפיו המתפשטים ממנו, שבים אליו ומתרפקים עליו, מוכנים לרצונו, ולעבודתו כסופה ירדופו, וכל המהלכים הטבעיים\mycircle{°} של הנפשיות\mycircle{°} והגופניות\mycircle{°} מוארים הם באור עליון ובמהות הקדש\mycircle{°} המנצח, המלא הוד\mycircle{°} ויפעת\mycircle{°} קדשים של זיו\mycircle{°} ההשכלה הטהורה\mycircle{°} המאירה באור חכמה\mycircle{°} ודעת\mycircle{°} יסודית\mycircle{°} }\צמקור{[עפ״י א׳ כח\hebrewmakaf ט]. }

\paragraphs

\ערך{שכל }\הגדרה{- }\משנה{השכלים הגדולים }\הגדרה{- המציאות הרוחנית\mycircle{°}, כל ההויה הנשמתית במלא עולם }\מקור{[א״ק ב שנו]}\צהגדרה{. }

\הגדרה{ע׳ במדור מלאכים ושדים, בהגדרות המבוא, נשמות, מלאכים, אורות, נצוצות, או כחות שכליות, סבות, עלולים, וכיו״ב.}

\paragraphs

\ערך{שכל }\הגדרה{- }\משנה{מקור השכל }\הגדרה{- מגמת\mycircle{°} ההויה למה שהיא נוטה לאן תתעלה. תכונת התפארת\mycircle{°} שבכל העולמות כולם }\מקור{[עפ״י א׳ קסא\hebrewmakaf ב, ע״ר א עה]}\צהגדרה{. }

\הגדרה{ע״ע רגש, מקור הרגש. }

\paragraphs

\ערך{שכלול חצוני\mycircle{°}}\הגדרה{ - }\משנה{השכלול החצוני של העולמות\mycircle{°}}\הגדרה{ - תיקוני החברה, שיפור החיים, כונניותן\mycircle{°} של כל חמדה ועונג לבב, הבאים על ידי ההכרה המעשית, וכל התרבות, החברותית, האסתתית\mycircle{°} והאומנתית }\מקור{[עפ״י א״ק ג קפא]}\צהגדרה{. }

\paragraphs

\ערך{שכלול פנימי\mycircle{°}}\הגדרה{ - }\משנה{השכלול הפנימי של העולמות\mycircle{°}}\הגדרה{ - גדולת\mycircle{°} הדעת\mycircle{°}, קדושת\mycircle{°} הרצון\mycircle{°}, עילוי\mycircle{°} החיים, הנצח\mycircle{°} וההוד\mycircle{°}, הקדושה\mycircle{°} והתפארת\mycircle{°} המלכותית\mycircle{°} שבכל נשמה\mycircle{°} }\מקור{[א״ק ג קפא]}\צהגדרה{. }

\paragraphs

\ערך{״שכלים נבדלים״ }\הגדרה{- ע׳ בנספחות, מדור מחקרים. ע׳ במדור מלאכים ושדים, בהגדרות המבוא, נשמות, מלאכים, אורות, נצוצות, או כחות שכליות, סבות, עלולים, וכיו״ב. ע״ע שכל, השכלים הגדולים. }

\paragraphs

\ערך{שָׂכָר }\הגדרה{- }\משנה{(על עבודת\hebrewmakaf ד׳\mycircle{°})}\הגדרה{ - ההשגה והשלמות }\מקור{[מ״א ב ה]}\צהגדרה{. }

\משנה{קיבול השכר }\הגדרה{- השגת טעמי\hebrewmakaf מצות\mycircle{°} מצד החכמה\hebrewmakaf העליונה\mycircle{°} }\מקור{[מ״ש קסח (מא״ה ב קכב)]}\צהגדרה{. }

\paragraphs

\ערך{שֵׁכַר }\הגדרה{- }\משנה{יסוד תאות השכרון והשיטוף ביין}\הגדרה{ - השיעבוד של הכוחות הרוחניים לתשוקות החומריות, שיטה הנאותיו הרוחניות לצד הדברים החומריים. שהיין ישמח הלב, וחשק השמחה שהיא הנאה רוחנית וטובה במדתה, כשתעבור גבול תביא לידי כל השחתה וממנה תוצאות לכל הכוחות הרוחניים שישתעבדו להשתמש להרע, שמכאן באה ג״כ הגאוה שחברוה חז״ל עם שכרות ואמרו ע״ז}\myfootnote{ ערובין סה. וברש״י שם: \textbf{המפיק מגן} - המעביר תפלת מגן אברהם, שאינו אומרה. \textbf{בשעת גאוה} - בשעת שיכרות אינו מתפלל.\label{4}}\הגדרה{ ״כל המפיק מגן בשעת גאוה״ }\מקור{[ע״א ב ו לד]}\צהגדרה{.  }

\הגדרה{ע״ע תאנה.}

\paragraphs

\ערך{שכרות }\הגדרה{- }\משנה{(״שֵׁכָר לד׳״) }\הגדרה{- השמחה\mycircle{°} העליונה, כשהיא באה מרוב טובה למעלת השכרות, זהו אור\mycircle{°} החשק\mycircle{°} הנשגב\mycircle{°}, שכל החושים החצוניים מסתלקים מפני גדולת האור\hebrewmakaf האלהי\mycircle{°} }\מקור{[ע״ר א קלב]}\צהגדרה{. }

\הגדרה{ע׳ במדור משכן ומקדש, נסכים, (עניינם).}

\paragraphs

\ערך{שלום }\הגדרה{- האחדות הכללית }\מקור{[מ״ש קיט (מא״ה ב יג)]}\צהגדרה{. }

\הגדרה{הגברת הקשר בין אישי האומה והרבות היחש והטוב ביניהם, במעשה ובמחשבה, ברגש וברצון, כדי שיהיה הקשר אמיץ וכדי שיתעוררו הלבבות כולם להתאגד למטרה אחת, שהיא הטובה הכללית, בין אותה שהיא קרובה ומורגשת מיד לעיני הדור בהוה, בין אותה הטובה הכללית העתידה לבא על האומה לדורות הבאים}\צהגדרה{ }\מקור{[קבצ׳ ב צד (פנק׳ ד רכ)]}\צהגדרה{.}

\הגדרה{הסוג העליון הכולל את היושר ותיקון ההנהגה <השלום בין אדם לחבירו, בין משפחה למשפחה, בין עם לעם. שלכונן אותו באמת ובבטחה צריכין אנו לכל ההכנות הגדולות של הלימודים החכמים של דרכי ד׳ ועבודת\hebrewmakaf ד׳\mycircle{°}> }\מקור{[ע״א ב ט רעא]}\צהגדרה{. }

\משנה{שלום  (לעומת אמת\mycircle{°}) }\הגדרה{- קיום השלמת הקשר הכללי }\מקור{[מ״ש קכ (מא״ה ב יד)]}\צהגדרה{. }

\משנה{השלום האמיתי }\הגדרה{- היחש וההצטרפות של בני אדם לפעול ולהתפעל כל אחד מחברו בדרך הטובה\mycircle{°} }\מקור{[ע״ר א עניני תפילה כז]}\צהגדרה{.}

\הגדרה{(המצב שבו) ידרוש כל אחד טובת חברו, בין הרוחנית\mycircle{°} ובין הגשמית }\מקור{[ע״א א א מא]}\צהגדרה{.}

\משנה{שלום }\הגדרה{- היחש האמיתי שראוי להיות בין חלקי האומה\mycircle{°} הפרטיים, בין החלקים המשפיעים לבין החלקים המקבלים ההשפעה, שיהי׳ נמצא בפועל, בין ״העגל הצריך לינק״ לבין ״הפרה הרוצה להניק״, בין ה״עליא״ ל״אתכליא״, בין העוסקים בגמילות חסדים להעוסקים בעבודה, ובין שניהם לבין העוסקים בתורה; היחש הראוי להמצא בין כל אחד מישראל לרעהו, עד שיבוא ה}\צהגדרה{שלום}\הגדרה{ להיות נותן יחש גם בין העצה\mycircle{°} והתחבולה והמחשבה הטובה, הכמוסה בלב רבבות אלפי ישראל שלמי אמונים, לבין פעל ידים והמציאות הגמורה }\מקור{[או״ה כ (א״ה 926)]}\צהגדרה{.}

\paragraphs

\ערך{שלום }\הגדרה{- }\משנה{(השלום העולמי) }\הגדרה{- (כש)ישוב הניגוד האנושי להיות לכח אחד שכולו מקושר זה בזה בחטיבה\mycircle{°} של אחדות גמורה, ויבא הניגוד אל תעודתו שהיא תכלית שכלול העולם בהמון צבעיו\mycircle{°}. המדרגה הגבוהה אשר הניגוד חלף כולו ואין כאן כ״א חטיבה אחת מלאה גוונים\mycircle{°} מרובים היודעים להוקיר זה את זה ולכבד את כל נטיותיהם המשונות, כי (תוכרנה כולן) להיות עולות לתכלית עליונה אחת ומיוחדת, הגדולה בכבודה והדרה, שהיא השלמת העולם בכל תכונותיו כולן, שממנה יתראה הדר הממלכה האלהית במלא עולמים }\מקור{[עפ״י ע״א ד ו מה]}\צהגדרה{.}

\צמשנה{השלום העליון }\צהגדרה{- החבור השלם של כל באי עולם, ההרמוניה\mycircle{°} הפנימית והממשית, בכלליותם ובפרטיהם, השלמות האמיתית והבריאות הנכונה }\צמקור{[א״ל רמג]. }

\paragraphs

\משנה{שלום }\צהגדרה{- }\צמשנה{השלום המחלט}\צהגדרה{ - השלום הנובע מן העוז\mycircle{°} הנשגב, הנתן מאת ה׳ לעמו, (השלום) הבא מתוך עצם קביעות הערך של כל החלקים ואמתת קשורם ההויתי במקור שרשם העליון, הכוללם ומקיפם, מחים ומקימם, ומתוך כך מדריכם ארחות יושר ודרכי נעם }\צמקור{[נ״ה כא]. }

\paragraphs

\ערך{שלום}\הגדרה{ - }\משנה{האידיאל של השלום}\הגדרה{ - השכלול העליון החובק כל עולמי עד, המאחד את כל היש לכלל הויה אחת אדירה אצילית ונשגבה, כעצת תמים דעים אלוה נורא הוד }\מקור{[ע״ר א סה]}\צהגדרה{.}

\ערך{שלום }\הגדרה{- ההתאמה הגמורה בתוכן ההכרות והתאגדותן ההרמונית\mycircle{°}, והתאחדותן של כל התביעות המוסריות\mycircle{°} המתכנסות בלבו של כל יצור, והכוללות את כל הקבוצות החברתיות, והמאחדות את כל העולמים\mycircle{°} }\מקור{[עפ״י א״ק א יב]}\צהגדרה{.}

\הגדרה{הטוב\mycircle{°} הנערך מחבור כל גוני\mycircle{°} החיים }\מקור{[ע״ר א שמא]}\צהגדרה{.}

\משנה{השלום הפנימי\mycircle{°}}\הגדרה{ - התאמתן של הידיעות, ההרגשות והציורים\mycircle{°} השונים זה עם זה, עד שכולם עומדים בצורה אורגנית\mycircle{°} וחטיבה\mycircle{°} משוכללה, שכל חלק משלים את החלק השני, ואין פרץ ואין צוחה ברחובותיהם }\מקור{[עפ״י א״ק א יג, יד]}\צהגדרה{.}

\הגדרה{המצב של ההתאחדות הבא מקיבוץ כחות ודעות נפזרות }\מקור{[ע״ר א שלא]}\צהגדרה{.}

\הגדרה{צביונו של עולם, המגלה שכל רבויי הדעות והרצונות כולם פונים לתכלית אחת, וכל המעשים מתקבצים בו להביא לגילוי אור\mycircle{°} כבודו\mycircle{°} ית׳ ע״פ הדרכים שחקק הוא ית׳ }\מקור{[שם רנח]}\צהגדרה{.}

\משנה{הוד השלום האלהי\mycircle{°}}\הגדרה{ - המצב האיתן ששום כח טוב, גם הקטן שבקטנים, לא יצר את משנהו, ומכל שכן את הקודם לו בערך ובמעלה, כ״א יעמוד על ימינו לסעדו ולחזקו }\מקור{[שם ב רנח]}\צהגדרה{.}

\משנה{קול השלום}\הגדרה{ - הקול ההרמוני, המאחד, הקול התובע מהכל התאמה, כיוון והשתוות, הקול המעלה\mycircle{°} והמישר\mycircle{°} }\מקור{[עפ״י א״ק ד תצג]}\צהגדרה{. }

\משנה{שלום }\הגדרה{- ההתאמה של העולם החיצוני הכללי כולו עם העולם הפרטי הפנימי }\מקור{[ע״א ג א יג]}\צהגדרה{.}

\משנה{יסוד השלום }\הגדרה{- שיהיה מגמת הפנים של כל רצון פרטי הנמצא במציאות משתוה עם החפץ הכללי, שנובע מהתכלית\hebrewmakaf הכללית\mycircle{°} של כל המציאות כולה }\מקור{[שם שם]}\צהגדרה{.}

\משנה{שימת שלום }\הגדרה{- התאחדות כל החלקים לטוב\mycircle{°} ולקדושה\mycircle{°} }\מקור{[מ״ש נב]}\צהגדרה{.}

\משנה{ברכת השלום }\הגדרה{- שיווי המשקל של הכחות כולם, החומריים והרוחניים }\מקור{[ע״ר ב רצד]}\צהגדרה{. }

\מעוין{◊ }\משנה{השלום }\הגדרה{- מכליל את כל המחשבות\mycircle{°} הפזורות לרעיון\mycircle{°} נשגב\mycircle{°} ועשיר אחד, נורא בהודו\mycircle{°} ומלא חיים עזיזים בתכנו }\מקור{[א״ק א קיח]}\צהגדרה{.}

\הגדרה{ע׳ במדור מלאכים ושדים, מלאכי השלום בעולם.}

\paragraphs

\ערך{שלום }\צהגדרה{- }\הגדרה{השלמות האמיתית והוד\mycircle{°} הקדושה\mycircle{°}}\צהגדרה{ }\מקור{[מא״ה, ענייני תפילה, רלו]}\צהגדרה{.}

\paragraphs

\ערך{שלום }\הגדרה{- }\משנה{אור השלום הכללי העליון }\הגדרה{- }\מעוין{◊}\הגדרה{ יוצא לא מתוך דחיה של איזה כח, של איזה רעיון\mycircle{°}, של איזה זרם, של איזה נטיה, אלא מתוך הכנסתו של כל אחד מאלה לתוך הים הגדול של אור\hebrewmakaf אין\hebrewmakaf סוף\mycircle{°}, ששם הכל מתאחד\mycircle{°}, הכל מתעלה, הכל מתרומם\mycircle{°}, והכל מתקדש\mycircle{°} }\מקור{[א״ק ב תקע]}\צהגדרה{.}

\paragraphs

\ערך{שלום }\הגדרה{- }\משנה{הרבוי של השלום (״תלמידי\hebrewmakaf חכמים\mycircle{°} מרבים שלום״) }\הגדרה{- שיתראו כל הצדדים וכל השיטות, ויתבררו איך כולם יש להם מקום\mycircle{°}, כל אחד לפי ערכו\mycircle{°}, מקומו וענינו <}\צהגדרה{ואדרבא גם הענינים הנראים כמיותרים או כסותרים, יראו כשמתגלה אמתת החכמה לכל צדדיה, שרק ע״י קיבוץ כל החלקים וכל הפרטים, וכל הדעות הנראות שונות, וכל המקצעות החלוקים, דוקא על ידם יראה אור האמת והצדק, ודעת ד׳ יראתו ואהבתו, ואור תורת אמת> }\מקור{[ע״ר א של]}\צהגדרה{.}

\משנה{להרבות שלום בעולם}\הגדרה{ - עבודת הקודש של תלמידי\hebrewmakaf חכמים, להשיב את הזכרון הכללי על כנו ממקורו הרוחני, לקבץ את כל הפזורים ולאסוף את כל הנדחים, לעשותם חטיבה\mycircle{°} אחת בעולם, כמו שהם באמת בעיקר יסודם}\צהגדרה{ }\מקור{[קובץ א תתטו]}\צהגדרה{.}

\הגדרה{ע׳ במדור פסוקים ובטויי חז״ל, תלמידי חכמים מרבים שלום בעולם. }

\paragraphs

\ערך{שלום }\הגדרה{- }\משנה{זיו השלום }\הגדרה{- השפעת הטוב\mycircle{°}, המיסד עולם, ומכינו להיות עומד וחי בחיי היחיד ובחיי החברה, בחיי\hebrewmakaf שעה\mycircle{°} ובחיי\hebrewmakaf עולם\mycircle{°}, המתגלה ברעיונות ועצות המוסר וההדרכה המדותית וכל פרטיה של תורה }\מקור{[עפ״י א״ק ב תקט]}\צהגדרה{.}

\הגדרה{ע׳ במדור תורה, תורת השלום. }

\paragraphs

\ערך{שלום }\הגדרה{- }\משנה{שלום במעשה }\הגדרה{- שיפעל כל אחד לטובת רעהו כראוי ומכש״כ לטובת הכלל\mycircle{°} }\מקור{[מא״ה ג (מהד׳ תשס״ד) קנא]}\צהגדרה{. }

\paragraphs

\ערך{שלום }\הגדרה{- }\משנה{שלום במחשבה }\הגדרה{- שיהיה נתון בלבו ונפשו לאהבת אחיו ועמו }\מקור{[מא״ה ג (מהד׳ תשס״ד) קנא]}\צהגדרה{. }

\paragraphs

\משנה{שליחות אלהית }\צהגדרה{- }\צמשנה{ערכו של אדם כמצווה ושלוח בכונניות\mycircle{°} ההנהגה האלהית}\צהגדרה{ - }\מעוין{◊}\צהגדרה{ לפי הכללתה ברירות אמתותה ויציבותה של הכרתו בצורך והחפץ ללמד ולהשפיע, להיטיב ולגמול חסד, הבא לא מצד שהוא יהיה העושה את זה, וגם לא באפשרות הרגשת עצמו בזה, אלא מצד התמלאותם של הדברים האלה כראוי}\צמקור{ [עפ״י א״ל רלו]. }

\paragraphs

\ערך{שליטה }\הגדרה{- }\משנה{מהותה }\הגדרה{- כח חיים עצומים המפעמים בכלליות הקבוץ, המתגלה\mycircle{°} בעצמת חיים באיזה מרכזיות אישית מאוצר הכלל\mycircle{°} הנקשר בה והנסמך עליה }\מקור{[עפ״י ע״א ד ט פו]}\צהגדרה{. }

\הגדרה{ע׳ בנספחות, מדור מחקרים, מנהיגים (סוגי מנהיגים). ע״ע הנהגה צבורית. ע״ע מלך ישראל. ע״ע הכתרה.}

\paragraphs

\ערך{שלילה }\הגדרה{- }\משנה{השלילה }\הגדרה{- הכפירה\mycircle{°} באלהות }\מקור{[ל״ה 230]}\צהגדרה{.}

\הגדרה{הנכריות של העולם מהצורך של המדע והרגש האלהי }\מקור{[מ״ר 11]}\צהגדרה{.}

\הגדרה{הסרת הרגש של האהבה\hebrewmakaf האלהית\mycircle{°} מתוך הנפש\mycircle{°}}\צהגדרה{ }\מקור{[א״י כט]}\צהגדרה{.}

\הגדרה{הסרת הלב מן האמונה <(ואפילו) מצדה החלש> }\מקור{[עפ״י קבצ׳ ב קיט (פנק׳ ד ער-רעא)]}\צהגדרה{.}

\הגדרה{ע״ע אפיקורסות. ע״ע כפירה (שלילת האמונה). ר׳ במדור מדרגות והערכות אישיותיות, אפיקורס. ושם, כפירים. }

\paragraphs

\ערך{שלל שמים }\הגדרה{- שפעת ברכות\mycircle{°} קודש של הרגשות נעלות וחפצים כבירים }\מקור{[קובץ ו קכא]}\צהגדרה{. }

\paragraphs

\ערך{שלמות }\הגדרה{-}\משנה{ השלמות האמיתית}\הגדרה{ - השמחה\mycircle{°} וההשגה כשהן מצורפות }\צהגדרה{[ע״א א א ה]}\הגדרה{.}

\paragraphs

\ערך{שלמות }\הגדרה{- }\משנה{תכלית השלמות }\הגדרה{- תכלית החפץ של ההשתלמות העליונה\mycircle{°} לעד\mycircle{°}, המביאה להכרת הודעת שם\hebrewmakaf ד׳\mycircle{°} }\מקור{[מ״ר 281 (ח״ה צח)]}\צהגדרה{. }

\משנה{השלימות האמיתית }\הגדרה{- להתדמות לשם ית׳ כפי האפשרות בחק נברא\mycircle{°} }\מקור{[ע״א א ה קז]}\צהגדרה{. }

\paragraphs

\ערך{שלמות הכללית }\הגדרה{- }\משנה{ השלמות הכללית}\הגדרה{ - טובה העצום של שפעת המציאות במקורה האלהי }\מקור{[עפ״י קובץ א קסח]}\צהגדרה{.}

\paragraphs

\ערך{שם}\myfootnote{ \textbf{שם }- \textbf{המהותיות, העצמיות} \textbf{הנשמתית}\textbf{ }- רלב״ג, בראשית ב יט (דפ׳ ויניציה) טו: ״לראות מה יקרא לו ר״ל להשיג מהותם אשר ייוחס בו א׳ מהם והוא הסדור המושכל אשר ממנו השתלשל מציאותו וכל אשר השיגו האדם מזה בא׳ אח׳ מהנמצאות הטבעיות היה שמו נפש חיה ר״ל שהמושכל ההוא כשהגיע לאדם היה שכל נצחי״. ע״ע בחומש גור אריה השלם, מהד׳ מכון ירושלים, תשמ״ט, בראשית ו ד, סק״ט, ושם הערה 35. ובמקראות גדולות מהד׳ המאו״ר, באור החיים, דברים כט יט, ובליקוטי הערות על אור החיים, שם. בס׳ בעל שם טוב, המהודר, על התורה ומועדים, ח״א עמ׳ קיט. פתחי שערים, נתיב הצמצום, ב. דובר צדק פד:. א״ק ג קלז. ובשי׳ ב 16\hebrewmakaf 15.\label{5}}\הגדרה{ - צורת הדבר העצמית, במה שיוגדר הויתו, שיהיה בו הוא מה שהוא <ועיקר צורת כל הדבר נקרא על שם תכליתו, כי תכלית מציאות כל דבר היא צורתו ומהותו העצמית אשר רק מצד זה ראוי הוא להקרא בשם> }\מקור{[מ״ש קי (מא״ה ב ה)]}\צהגדרה{. }

\הגדרה{התעודה }\מקור{[פנ׳ כג]}\צהגדרה{. }

\הגדרה{ענין תכלית הדבר }\מקור{[פנק׳ ה פג]}\צהגדרה{.}

\הגדרה{התכלית והתכונה הפנימית }\צמקור{[פנק׳ ד תמה].}

\הגדרה{עיקר התכלית והמציאות, עצמות פעולתו (של בעל השם) }\מקור{[עפ״י ה׳ קעט, מא״ה א קו]}\צהגדרה{. }

\הגדרה{עיקר המציאות והתוכן }\צהגדרה{[פנק׳ ג רעא-ערב]}\הגדרה{.}

\הגדרה{התוכן המאגד את השפעה\mycircle{°} העליונה הרזית\mycircle{°} העומדת ממעל לכל הוראה אל התוכן של מדת ההוראה הקצובה של הלכתא גברתא }\מקור{[ר״מ קפו]}\צהגדרה{. }

\הגדרה{המהותיות, העצמיות הנשמתית }\מקור{[עפ״י ע״ר ב קנט]}\צהגדרה{.}

\הגדרה{פעולה}\צהגדרה{ }\מקור{[קבצ׳ ב כב]}\צהגדרה{.}

\הגדרה{פעולה על מה שחוץ לו, ביחש זולתו }\מקור{[עפ״י ע״א ג א סו]}\צהגדרה{.}

\משנה{יסוד השם }\הגדרה{- היסוד\mycircle{°} של ההויה ותכונת החיים המייחד את העצמיות של הנקרא הכולל בקרבו את הצד היותר יסודי שבהויותו }\מקור{[עפ״י ע״א ד ו ק]}\צהגדרה{. }

\מעוין{◊}\הגדרה{ ענין ה}\משנה{שם }\הגדרה{הוא תמיד להורות בהוראתו התכלית הנולד ממנו }\מקור{[מא״ה ג רצח]}\צהגדרה{. }

\משנה{שם }\צהגדרה{- }\צמשנה{שם של כל דבר }\צהגדרה{- אמצעי הבירור, ההוכחה והסימון, לאיזה דבר אנחנו מכונים בדברנו, שעל ידו, על ידי אמצעי זה, יצויר למדברים אותו המושג שהדבור מכוון אליו }\צמקור{[א״ל קיא]. }

\צהגדרה{טובו, טיבו, מהותו של הדבר, גופו של הדבר }\צמקור{[ק״ה קמז].}

\הגדרה{ע״ע פסוק, הפסוק, האותיות של שמו (של האדם) שבתורה. }

\paragraphs

\ערך{שם }\הגדרה{- }\משנה{(״לשם ולתהלה\mycircle{°}״)}\myfootnote{ צפניה ב כ.\label{6}}\הגדרה{ - התגלות בביטוי ובפרסום }\מקור{[א״ק ב שד]}\צהגדרה{. }

\הגדרה{הצד הפונה, בהשפעה חצונית\mycircle{°}, כלפי יניקתו מהתוכן העצמי הפנימי\mycircle{°}. מושג של עצם ושל אופי }\מקור{[עפ״י ע״ר א קטז]}\צהגדרה{. }

\paragraphs

\ערך{שם }\הגדרה{- }\ערך{לעשות שם }\הגדרה{- }\משנה{שם עולם }\הגדרה{- להכין החפץ האלהי היותר נשגב והיותר עתיד, <מתוך מצב ההוה עצמו> }\מקור{[ע״ר א שיח (ע״א ג ב מה)]}\צהגדרה{. }

\paragraphs

\ערך{שם }\הגדרה{- }\ערך{קריאת שם }\הגדרה{- ברור מגמה\mycircle{°} }\מקור{[עפ״י א׳ עט]}\צהגדרה{. }

\הגדרה{הבנת התכלית }\מקור{[עפ״י מא״ה ג רצח]}\צהגדרה{. }

\הגדרה{התודעות והודעת התכלית העליונה הכללית }\מקור{[עפ״י ע״א ב 407, מעשר שני יח]}\צהגדרה{.}

\משנה{קריאת (חידוש, שינוי) שם (״וקורא לך שם חדש״) }\הגדרה{- הבלטת הכוונה הפנימית בתארים חיים }\מקור{[א״ת ב 196]}\צהגדרה{. }

\ערך{קריאת שמות }\הגדרה{- }\משנה{חכמת אדם הראשון לקרוא שמות }\הגדרה{- שהבין לכל אחד מה תכליתו הפרטי(ת) ולאיזה צורך נברא }\מקור{[פנק׳ ה פד]}\צהגדרה{.}

\paragraphs

\ערך{שמאל }\הגדרה{- מתיחש יחוש של ערך ורושם אל הענינים הגשמיים, }\צהגדרה{<שראוי להכירם בתור חלושים מהענינים הרוחניים\mycircle{°}> }\מקור{[עפ״י ע״א ב ט רלח (ח״פ לח:)]}\צהגדרה{.}

\הגדרה{העוזר להמטרות העקריות, }\צהגדרה{<שהן ראויות להקרא בשם ימין\mycircle{°}> }\מקור{[ע״ר א נו]}\צהגדרה{.}

\הגדרה{(המסייע) להוצאת המגמות העליונות אל הפועל }\מקור{[עפ״י ע״א ד ו נח]}\צהגדרה{. }

\הגדרה{המעורר }\מקור{[עפ״י קובץ ז רג]}\צהגדרה{. }

\משנה{כח השמאל }\הגדרה{- מורה על חסרונות }\מקור{[פנ׳ יא]}\צהגדרה{.}

\הגדרה{הטפל }\מקור{[ע״א ד ו יז]}\צהגדרה{.}

\הגדרה{הכח מקבל פעולה, כערך החומר\mycircle{°} }\מקור{[עפ״י ע״א ב ט רמ]}\צהגדרה{.}

\הגדרה{הצד החומרי של החיים }\מקור{[קבצ׳ ג ע]}\צהגדרה{.}

\הגדרה{ע״ע ימין. ע׳ במדור גוף האדם אבריו ותנועותיו, יד שמאל. ושם, יד ימין. ושם, הצד השמאלי שבאדם. ושם, הצד הימני שבאדם. ע׳ במדור תיאורים אלהיים, שמאל. }

\paragraphs

\ערך{שמאל}\הגדרה{ - }\משנה{להשמאיל}\הגדרה{ - לצד ההכערה והמורד }\מקור{[קבצ׳ ב עו (פנק׳ ד קעט)]}\צהגדרה{.}

\הגדרה{ע״ע ימין, להימין.}

\paragraphs

\ערך{שמאל וימין }\הגדרה{- החול\mycircle{°} והקודש\mycircle{°} }\מקור{[עפ״י מ״ר 45, א״ק א סט]}\צהגדרה{.}

\paragraphs

\ערך{שמאל }\הגדרה{- }\משנה{הליכה משמאל לימין }\הגדרה{- היסוד העקרי הפונה מאוצר החול\mycircle{°} אל רום מבוע הקודש\mycircle{°}. הנטיה מאוצר רוח חכמה שבאדם אל רוח אלהים. כשמרימים את כל ערכי התבונה האנושית אל הרום\hebrewmakaf העליון\mycircle{°}, ומבארים את הכל על פי שפעת הקודש }\צהגדרה{<מדתו של בצלאל> }\מקור{[א״ק א סט]}\צהגדרה{.}

\הגדרה{ע״ע ימין, הליכה מימין לשמאל.}

\paragraphs

\ערך{שמים }\הגדרה{- נשמת\hebrewmakaf העולם\mycircle{°} }\מקור{[עפ״י א״ק ג קפט]}\צהגדרה{. }

\הגדרה{כל סדרי הקדושה\mycircle{°} בקביעותם }\מקור{[ע״ר א קסב]}\צהגדרה{. }

\הגדרה{המרחבים הגדולים בכל ערכיהם, בתכונה החמרית והרוחנית }\מקור{[ע״ר א יג]}\צהגדרה{.}

\צמקור{כל כח שכלי מופשט מכל חומר [פנק׳ ד תמד].}

\הגדרה{הרוחניות של המציאות, הכחות הפועלים לרומם ולשכלל אותה }\מקור{[ע״א ב ט ל]}\צהגדרה{. }

\הגדרה{המציאות הרוחנית }\מקור{[ע״ר א שכה]}\צהגדרה{. }

\הגדרה{כלל העולמים\mycircle{°} הרוחניים\mycircle{°} כולם }\מקור{[עפ״י ע״ר א קטו]}\צהגדרה{. }

\הגדרה{העולם הרוחני }\מקור{[ע״ר ב פ]}\צהגדרה{.}

\הגדרה{הענינים הרוחניים, תוכן המוסר\mycircle{°}, הצדק\mycircle{°} והיושר\mycircle{°}, דעת\hebrewmakaf האלהים\mycircle{°} וערך הקדושה\mycircle{°} ופעולתה בעולם }\מקור{[עפ״י ע״א ג א לג]}\צהגדרה{. }

\הגדרה{הטוב\mycircle{°}, החסד\mycircle{°}, הרחמים\mycircle{°} ונועם\hebrewmakaf ד׳\mycircle{°} }\מקור{[קובץ א תשכד]}\צהגדרה{.}

\הגדרה{רום הטוהר\mycircle{°} של האמת\mycircle{°}, של הצדק, של המשפט\mycircle{°} ושל החסד\hebrewmakaf העליון\mycircle{°} }\מקור{[ע״ר א צב]}\צהגדרה{. }

\הגדרה{העולם המחשבתי\mycircle{°} היותר מכולל\mycircle{°} ויותר שלם }\מקור{[ע״א ד ט עט]}\צהגדרה{.}

\הגדרה{השכלים הנבדלים }\מקור{[פנק׳ ג קסד]}\צהגדרה{.}

\הגדרה{מקור החרות\mycircle{°} }\מקור{[פנק׳ ב רו]}\צהגדרה{.}

\צהגדרה{אש ומים, סמל לכלליות העליונה ודוגמה לה }\צמקור{[א״ל קל]. }

\צהגדרה{שלמות מציאות אין\hebrewmakaf סופית. מלשון שם. ׳אי שם׳ - במרחקים הבלתי מוגבלים. והמלה }\צהגדרהמודגשת{״שמים״ }\צהגדרה{היא זוגיות  <כמו חיים או מים> ופירושה: ריבוי עצום של שם. ומתוך\hebrewmakaf כך המובן האלוהי של הביטוי ״}\צהגדרהמודגשת{שמים}\צהגדרה{״ }\צמקור{[שי׳ א 217].}

\משנה{עצמיות השמים }\הגדרה{- המושגים הקדושים\mycircle{°}, שהם מגמות\mycircle{°} כל ההויה כולה. הרום\hebrewmakaf העליון\mycircle{°} הרוחני }\מקור{[עפ״י ע״ר ב נג]}\צהגדרה{. }

\משנה{שמים }\הגדרה{- התחלת הסבות\mycircle{°} הראשיות, המסבבות את כל המון המעשים }\מקור{[שם א רמה (ע״א א 71)]}\צהגדרה{. }

\משנה{גבהי שמים }\הגדרה{- עולמים עליונים }\מקור{[ע״ר א יג]}\צהגדרה{. }

\משנה{המצפונים השמימיים }\הגדרה{- אורות\mycircle{°} החמדה העליונים שהם ממעל לעולמנו }\מקור{[קובץ ז קעז]}\צהגדרה{.}

\משנה{ההויות השמימיות }\הגדרה{- תכונות רוחניות\mycircle{°} }\מקור{[ע״ר א רט]}\צהגדרה{. }

\הגדרה{ע״ע ארץ. ע״ע רקיע. ע׳ במדור פסוקים ובטויי חז״ל, שמי שמי קדם. }

\paragraphs

\הגדרה{שמים }\צמקור{- הגרמים העליונים מצד הכוחות הפנימיים שלהם, ומצד כח השכל המושם בתוכם או הופקד עליהם שינהיגם לתכליתם שיהיו פועלים בזולתם תכליתם הנרצה [פנק׳ ד תמד]. }

\הגדרה{כלי ההשפעה שהארץ מקבלת מהם, מצד המשיכם ההשפעה החמרית לנו }\מקור{[פנק׳ ד תל]}\צהגדרה{.}

\הגדרה{ע״ע שחקים. }

\paragraphs

\הגדרה{שמים }\צמקור{- כינוי בדרך השאלה להשי״ת [פנק׳ ד תמד].}

\paragraphs

\ערך{שמימי }\הגדרה{- פנימי\mycircle{°} }\מקור{[עפ״י ע״ר א קכד]}\צהגדרה{. }

\הגדרה{אלהי עליון }\מקור{[עפ״י קובץ א צח]}\צהגדרה{. }

\צהגדרה{אלהי, אינסופי, נצחי, המאיר לארץ ולדרים עליה ומחיה את כלם }\צמקור{[ל״י א קנו]. }

\paragraphs

\ערך{שמיעה בקול\hebrewmakaf ד׳\mycircle{°}}\הגדרה{ - מה שמקשיבים את כל התהלוכה של דרכי החיים לכל פרטיהם, אל הקיבוצים הכלליים לפי הבדליהם, ולכל יחיד לפי ערכו, מתוך החכמה\hebrewmakaf הכללית\hebrewmakaf העליונה\mycircle{°}, החיה ומחיה את כל הויה. וכל מה שהפרטים נובעים מתוך החיים הרוחניים העליונים הכוללים, שהיא חכמת נשמת אל בעולם, בצורה יותר ברורה, האדם שומע ומאזין יותר בבירור את קול ד׳ הדובר אליו, מורהו ומצוהו ממש }\מקור{[ע״ט סב]}\צהגדרה{.}

\paragraphs

\ערך{שמיעה לקול\hebrewmakaf ד׳\mycircle{°}}\הגדרה{ - כללות הקדושה\mycircle{°} בכל גווניה, ההתמסרות לרצון\mycircle{°} אדון\hebrewmakaf כל\mycircle{°} }\מקור{[ע״ר א צו]}\צהגדרה{.}

\paragraphs

\ערך{שמן - }\צהגדרה{נקרא שמן וגם יצהר}\הגדרה{ - הוא על ב׳ הפעולות: }\צהגדרה{שמן}\הגדרה{ מלשון שמן ודשן על ענין הטעם והתועלת שממשיך על ידי עירובו במאכל, ו}\צהגדרה{יצהר}\הגדרה{ על שום הזוהר של האור. <ולפי זה גם חכמה, גם עושר, ימשל לשמן, אלא שהעושר ימשל לשמן מצד קריאתו בשם }\צהגדרה{שמן}\הגדרה{ שמשמין הגוף, וחכמה תמשל בשם שמן מצד קריאתו בשם }\צהגדרה{יצהר}\הגדרה{> }\מקור{[מציאות קטן קעח]}\צהגדרה{.  }

\paragraphs

\ערך{שמן }\הגדרה{- ענין מתאים להארה\mycircle{°} רוחנית\mycircle{°}, החמר המאיר ומגרש את החושך, המצהיל את הפנים, והמעורר את הכבוד\mycircle{°} במשיחתו, מזכיר הוא רוח\hebrewmakaf הקודש\mycircle{°} וסוד המנוחה\mycircle{°} וההופעה\hebrewmakaf האלהית\mycircle{°} באורה הצלול }\מקור{[ע״ר א קלא]}\צהגדרה{.}

\הגדרה{השפע\mycircle{°} }\מקור{[פנק׳ ה פג]}\צהגדרה{.}

\הגדרה{מורה על החכמה\mycircle{°} האלהית. ע״כ נבחר שמן למשחה לאות קודש ויתרון כח קדושה }\מקור{[ע״א ג א סז]}\צהגדרה{.}

\משנה{השמן האלהי }\הגדרה{– אור\hebrewmakaf השכל\mycircle{°} וקדושת המדות }\מקור{[מ״ר 219]}\צהגדרה{.}

\paragraphs

\ערך{שמן }\הגדרה{- }\משנה{כתית }\הגדרה{- קרוב לתכונת הארה, אע״פ שאיננו לגמרי במצב ההארה. }\צהגדרה{זך כתית}\הגדרה{ - לגמרי במצב ההארה}\צהגדרה{ }\מקור{[ע״ר א קלא]}\צהגדרה{.}

\paragraphs

\ערך{שמן זית }\הגדרה{- אור תורה ודעת}\צהגדרה{ }\מקור{[ע״א ב בכורים כט]}\צהגדרה{.}

\הגדרה{החכמה ומאור\hebrewmakaf השכל\mycircle{°}, החכמה האלהית, אורה של תורה}\צהגדרה{ }\מקור{[ע״א ג ב יח]}\צהגדרה{.}

\הגדרה{משל לחכמת התורה הפנימית, השכלת הבנות כלליות שופעות בבת אחת אור חכמה מרובה, ההבנה הרחבה של דעת גנזי תורה}\צהגדרה{ }\מקור{[עפ״י ע״א ג ב עז]}\צהגדרה{.}

\paragraphs

\ערך{״שמע״ }\הגדרה{- }\משנה{״קריאת שמע״ }\הגדרה{- ההקשבה\mycircle{°} הרוחנית של כלל\hebrewmakaf ישראל\mycircle{°} }\מקור{[ע״א ד יד א]}\צהגדרה{. }

\משנה{קבלת מלכות שמים שבפרשת שמע ישראל }\הגדרה{- ההודאה על אחדות\hebrewmakaf ד׳\mycircle{°}, (ש)היא הודאה מוחלטה, לא מצד הצירוף של התגלות\mycircle{°} האחדות\mycircle{°} המקפת את העולמים\mycircle{°} כולם, כ״א מצד עדות ד׳ הנאמנה\mycircle{°}, שהוא באמת המובן של היחידות\hebrewmakaf האלהית\hebrewmakaf בעולם\mycircle{°} }\מקור{[ע״ר א כה]}\צהגדרה{. }

\מעוין{◊}\הגדרה{ ב}\צהגדרה{קריאת שמע ד׳ אחד }\הגדרה{כלולה ההשגה המיוחדת למעלת ישראל\mycircle{°}, שהיא מצד התורה\mycircle{°}, המופיעה מצד עצם כבודו\mycircle{°} ית׳ בשפע\mycircle{°} רצונו שממעל לעולמות\mycircle{°}, שאין עוד מלבדו }\מקור{[שם רמט]}\צהגדרה{. }

\הגדרה{אור העליון ממקור התורה שקדמה\hebrewmakaf לעולם\mycircle{°} }\מקור{[שם קיא]}\צהגדרה{. }

\הגדרה{ע׳ במדור שמות כינויים ותארים אלהיים, ״אחד״. ע׳ במדור פסוקים ובטויי חז״ל, יחוד עליון. ושם, יחוד ד׳ בעולם. ושם, ברוך שם כבוד מלכותו לעולם ועד. ושם, קבלת מלכות שמים. ע׳ במדור מצוות, הלכות, מנהגים וטעמיהן, קריאת שמע. }

\paragraphs

\משנה{שפה }\צהגדרה{- המטבע הממשית של קבוציותה (של אומה), שפנימיות תכונתה האורגנית הכללית מתגלה ומתבלטת בו, בדרך\hebrewmakaf הדבור ובאופן\hebrewmakaf המחשבה המופיעים במהלך החיים וארחותיו ומאחדים וכוללים את בניה אחרי התקבצם בשטח\hebrewmakaf מקום הראוי להם. סגנון האומה ואוצר גילוי נשמתה <מכון ההסטוריה של האומה>}\צמקור{ [צ״צ א קג, קא]. }

\צהגדרה{ר׳ ארץ.}

\paragraphs

\ערך{שפה }\הגדרה{- }\משנה{אוצר השפה}\הגדרה{ - הקצבת דבור\mycircle{°} הפה\mycircle{°}, <הבא מיסוד ההתחמה וההגבלה> }\מקור{[ר״מ קכה]}\צהגדרה{. }

\paragraphs

\ערך{שפיכות דמים}\הגדרה{ - }\מעוין{◊ }\הגדרה{תמצית קלקול המוסריות }\מקור{[פנק׳ א סז]}\צהגדרה{.}

\paragraphs

\ערך{שפע }\הגדרה{- רצון חיים ומחשבה\mycircle{°} }\מקור{[א״ק ב תמו]}\צהגדרה{.}

\צמשנה{שפעה }\הגדרה{- חיים קדושים ורעננים\mycircle{°} }\מקור{[א״ק א קב]}\צהגדרה{.}

\הגדרה{ע״ע השפעה. }

\paragraphs

\ערך{שקוקה }\הגדרה{- מסורה בהשתוקקות }\מקור{[רצי״ה א״ש יב הערה 25]}\צהגדרה{. }

\paragraphs

\ערך{שקר }\הגדרה{- }\מעוין{◊ }\הגדרה{יתיחס על דבר שאין לו בעצמו קיום ומעמד תמידי }\מקור{[ע״ר א רכו\hebrewmakaf ז]}\צהגדרה{. }

\הגדרה{ע״ע אמת. ע״ע כזב. ע״ע מרמה. ע״ע שוא.}

\ערך{שקר }\הגדרה{- }\מעוין{◊}\הגדרה{ מכון השקר בציור\mycircle{°} הוא מיוסד על התחלקותו של הפרט מן הכלל, על אי התאמתו, והעדר התבלעותו בחיי הציור הכללי }\מקור{[ר״מ קכט]}\צהגדרה{. }

\הגדרה{ע״ע בד.}

\paragraphs

\ערך{שש שנים }\הגדרה{- חיי החברה המעשית }\מקור{[ע״א ג ב רסט]}\צהגדרה{.}

\paragraphs

\ערך{ששת הצדדים אשר לששת הקצוות }\הגדרה{- תיאור ההקפה הכוללת ביחושה להחושים }\מקור{[ע״א ג ב רפב]}\צהגדרה{. }

\הגדרה{ע׳ במדור מספרים, שתים עשרה.}\mylettertitle{ת}

\paragraphs

\ערך{תאוה רוחנית }\הגדרה{- השאיפה למרומים, לקודש\mycircle{°}, לנשגב\mycircle{°}, לטוהר\mycircle{°} וזוך }\מקור{[עפ״י א״ק ג קלב]}\צהגדרה{. }

\paragraphs

\ערך{תאוה רוחנית חשוכה }\הגדרה{- תולדתה של התאוה החומרית\mycircle{°}, המעכבת על כל מאור ממאורי הקודש העליונים, שלא יתפשט ולא יחדור בנפש ובעולם, מספקת את כל הודאו[יו]ת העליונות, בכח מחשכיה, המיוסדים על יסוד שקר שאין לו רגלים, אלא שכך היא בונה את עליותיה בעולה, בניני רשע וכסל, ומחזקת את ציורי האדם בהשקפת עולם מאופלת, עד שכל אור עולם נחשב לו כזר }\מקור{[א״ק ג קלב]}\צהגדרה{. }

\paragraphs

\ערך{תאוות }\הגדרה{- שמרי האהבות, <כלומר ניצוצות\hebrewmakaf קדושה\mycircle{°} שנפלו אל הקליפות\mycircle{°}> }\מקור{[קבצ׳ א קפח]}\צהגדרה{. }

\paragraphs

\ערך{תאנה }\הגדרה{- הפרי המתוק. שאין בה צורך כ״א שהיא נוטפת דבש. שכשמה כה תארה, תאנה יבקש הרודף אחרי המותרות במאכלים ערבים כדי ליהנות בהם הנאת הטעם לבדו בלא תועלת. מרמזת על בקשת ההנאות המותריות, שאין בהם צורך הכרחי לגוף רק כדי לבקש הנאה ופינוק, שזה יעיר את לב האדם לסור מדרכי יושר ולבקש רק הנאת השעה, והיא תתפשט בבקשות דברים ערבים וטעמים נעימים }\מקור{[ע״א ב ו לד]}\צהגדרה{.}

\הגדרה{ע״ע שכר, יסוד תאות השכרון והשיטוף ביין.}

\paragraphs

\ערך{תבונה }\הגדרה{- }\משנה{(ביצירה ובאדם) }\הגדרה{- הבנה של תולדה וחידוש דבר מדבר, כלומר השתלשלות של תולדות חדשות העולות ברום מעלות גבוהות }\מקור{[ע״א ב ט ל]}\צהגדרה{.}

\הגדרה{ע׳ במדור פסוקים ובטויי חז״ל, לצרף אותיות שנבראו בהן שמים וארץ.}

\paragraphs

\ערך{תגא }\הגדרה{- ע״ע עטרה, עטרה (אלהית).}

\paragraphs

\ערך{תדיריות }\הגדרה{- ההארה\mycircle{°} הבלתי פוסקת }\מקור{[ר״מ קנח]}\צהגדרה{.}

\הגדרה{הקביעות והנצחיות }\מקור{[שם קעח]}\צהגדרה{.}

\paragraphs

\ערך{תהילה }\הגדרה{- ההכרה הפנימית\mycircle{°} שהשפע\mycircle{°} הטוב\mycircle{°} ומקור\hebrewmakaf החיים\mycircle{°} נוזל תמיד באין הפסק, וכל הנשא, המרומם, היפה והנשגב\mycircle{°}, הטוב\mycircle{°} והחסד\mycircle{°}, הוא באמת בלא כל התעכבות, בתמידות גמורה באין הפסק של זמנים, ובאין הפרעה, בין חלקי העיתים\mycircle{°}. זאת היא הארה\mycircle{°} עליונה\mycircle{°} הנקראת בשם }\משנה{תהילה}\הגדרה{ מלשון ״יהל אור״}\myfootnote{ איוב לא כו ״אור כי יהל״.\label{1}}\הגדרה{, העומדת למעלה מכל בטוי, ואי\hebrewmakaf אפשר לצמצמה במטבע של ברכה\mycircle{°} }\מקור{[ע״ר ב סב]}\צהגדרה{. }

\הגדרה{ההכרה של ההארה העליונה ושל הופעת השלמות האדירה האלהית\mycircle{°}, המשלחת קוי נהרה ומהילה נרה על דרכי היצורים. הכניסה הפנימית יותר (מן התודה\mycircle{°}) בחצרות\hebrewmakaf ד׳\mycircle{°}, העומדת על היסוד של הגודל\mycircle{°}, אשר הנשמה\mycircle{°} חשה מהאור\mycircle{°} הכללי של שפעת ד׳ וטובו על כל יצוריו, בהנהגתו העליונה\mycircle{°} המלאה אור החסד\mycircle{°} }\מקור{[שם א רכב]}\צהגדרה{. }

\הגדרה{ההבעה הטבעית האצילה\mycircle{°}, הבאה מתוך נחלתה העליונה של סגולת\mycircle{°} הנשמה\mycircle{°} האלהית, מתת אל חי לעם קדשו, ההארה הצפונה המתגלה }\מקור{[עפ״י שם קצז]}\צהגדרה{. }

\משנה{תהילה אלהית }\הגדרה{- הבעת הכרת גודלו\hebrewmakaf העליון\mycircle{°} של נורא תהילות ברוך הוא, האצורה בתור סגולה טבעית קדושה בתכונת גוי קדוש בכללותו, מתוך אותה נקודת אורה\hebrewmakaf אלהית\mycircle{°} שמאירה בנשמתם\hebrewmakaf של\hebrewmakaf ישראל\mycircle{°} פנימה, <נקודה, המאירה את הציורים\mycircle{°} החשכים של הנבראים, הפודה, באור האמת\hebrewmakaf העליונה\mycircle{°} שבה, את התהילה מכל ציורי\mycircle{°} חשך> }\מקור{[עפ״י שם]}\צהגדרה{. }

\משנה{תהילת ד׳ }\הגדרה{- השׂבעת הנשמה\mycircle{°} בזיו\mycircle{°} הציור\mycircle{°} העליון מכל עליון, <נטייתם של ישראל\mycircle{°} מצד עצם החיים> }\מקור{[עפ״י קובץ א תשפא]}\צהגדרה{.}

\הגדרה{הסוד\mycircle{°} המושג בשיקופם הפנימי של בעלי היראה\hebrewmakaf הטהורה\mycircle{°} החיים בה }\מקור{[עפ״י ע״א ד ח לה]}\צהגדרה{. }

\משנה{תהילה עליונה }\צהגדרה{- }\הגדרה{ההסתגלות עם המחשבה\mycircle{°} הציור\mycircle{°} והדמיון\mycircle{°}, שיהיו מתאימים להגודל\mycircle{°}, ההוד\mycircle{°} והיושר\mycircle{°} האלקי }\מקור{[פנק׳ א שעג]}\צהגדרה{. }

\הגדרה{הסתכלות מלאה במציאות ושרשיה }\מקור{[קובץ ה מא]}\צהגדרה{.}

\משנה{תהילת ד׳ האמיתית }\הגדרה{- אור הדעה וההבנה בהגודל\hebrewmakaf האלהי\mycircle{°} }\מקור{[פנק׳ א תד]}\צהגדרה{. }

\מעוין{◊ }\משנה{תהילת ד׳}\הגדרה{ מאירה ומופיעה בשתי מערכות: האחת היא מצד ההארה השכלית, המוסיפה תמיד הכרה, ומדע עליון בגדולת השם יתברך ונוראותיו, שבזה הנשמה מתעלה, והאור\hebrewmakaf האלהי\mycircle{°} הולך ושופע עליה, והשניה היא התהילה הבאה מתוך המעשים הרצויים, מתוך כל מהלך החיים, שקדושת\hebrewmakaf השם\mycircle{°} יתברך ברעיון\mycircle{°} הקדוש של העובד האמיתי היא מישרת את כל חייו המפעליים, ועל ידם באה ברכה ותהילה לכל היצור כולו }\מקור{[ע״ר ב עט]}\צהגדרה{. }

\הגדרה{ע״ע ספור תהילת ד׳. ע״ע הלול. ע׳ בנספחות, מדור מחקרים, תפילה לעומת תהילה. ע׳ במדור מונחי קבלה ונסתר, בהגדרות המבוא, סוד. }

\paragraphs

\ערך{תהילה }\הגדרה{- }\משנה{(״לשם\mycircle{°} ולתהלה״)}\myfootnote{ צפניה ב כ.\label{2}}\הגדרה{ - הצד הפונה, בהשפעה חצונית, כלפי הפומביות וההשגה הגלויה, נחלת ההמון בענינים שאפשר להם לבא לענין פרסומי }\מקור{[עפ״י ע״ר א קטז]}\צהגדרה{. }

\paragraphs

\ערך{״תהילתי״ }\הגדרה{- ע׳ במדור תיאורים אלהיים, ״כבודי״, ״תהילתי״.}

\paragraphs

\ערך{תודה }\הגדרה{- }\משנה{(לד׳\mycircle{°}) }\הגדרה{- רגש הדוחק את הלב להביע את הכרת\hebrewmakaf הטובה\mycircle{°}, בעת\mycircle{°} שהחסד\mycircle{°} האלהי\mycircle{°} מבליט את גמולו הטוב ואת מפעל הצלתו והגנתו. הבלטת הכרת\hebrewmakaf הטובה }\מקור{[ע״ר א קעה]}\צהגדרה{. }

\הגדרה{רגש שהאדם חש בקרבו, שאי אפשר לו שלא יביע את רגשותיו הפנימיים\mycircle{°}, להודות לאדון החסד, רבון כל המעשים, על כל תגמולוהי עליו }\מקור{[עפ״י שם רכב]}\צהגדרה{. }

\הגדרה{רגש הודאה, שהתכן של הכרת\hebrewmakaf טובה מכריח להכיר מצד מקבל הטובה אל הנותן, המשפיע אותה ברב\hebrewmakaf חסדו\mycircle{°} }\מקור{[שם קפח]}\צהגדרה{. }

\הגדרה{ע״ע הודאה, תוכן ההודאה (לד׳). }

\paragraphs

\ערך{תוהו}\myfootnote{ רמב״ן בראשית א א: ״החומר הזה, שקראו היולי, נקרא בלשון הקדש ״תֹּהוּ״, והמלה נגזרה מלשונם (קידושין מ:): ״בתוהא על הראשונות״. מפני שאם בא אדם לגזור בו שֵם, תוהא ונמלך לקוראו בשם אחר, כי לא לבש צורה שיתפש בה השם כלל. [...] ״תהו״, כלומר, חומר אין בו ממש״. ובספר הברית, פתיחה, סי׳ טו: ״התורה קורא אותו תוהו באמת מטעם זה שפירושו מה שאדם תוהה בו ומשתומם עליו ולא יוכל לצייר אותו בדעתו״.\label{3}}\הגדרה{ - העדר סידור }\מקור{[ע״א ב ט שכה]}\צהגדרה{. }

\הגדרה{חסרון כל סדר }\מקור{[עפ״י שם רצד]}\צהגדרה{. }

\הגדרה{בלא צורה\mycircle{°} גמורה }\מקור{[קובץ ו קצג]}\צהגדרה{. }

\הגדרה{דבר שאין לו צורה כלל, שהלב נשאר מתהה עליו, מבלי להגדירו מפני חוסר צביונו והעדר צורתו. חוסר צורה ומהות עצמית, הראויה לעמידה ולהערכה }\מקור{[עפ״י ע״ר א קד]}\צהגדרה{. }

\הגדרה{בלא סדרים וחוקים מקושרים }\מקור{[שם רפט\hebrewmakaf רצ]}\צהגדרה{.}

\משנה{יסוד התוהו }\הגדרה{- התבוללות הצורות המביא לטשטושן ולהעדרן }\מקור{[ע״א ד ו קו]}\צהגדרה{.}

\הגדרה{ע׳ במדור פסוקים ובטויי חז״ל, יעלו בתוהו.}

\paragraphs

\ערך{תוכן החיים }\הגדרה{- ע״ע חיים, תוכן החיים.}

\paragraphs

\ערך{תועבות }\הגדרה{- תשוקות רעות\mycircle{°} בלתי טבעיות }\מקור{[ע״א א ה מד]}\צהגדרה{. }

\paragraphs

\ערך{תועלת בחיים }\הגדרה{- השפעה טובה\mycircle{°} על המהלך המוסרי\mycircle{°} הפרטי ועל מצב ההשכלה הפרטית, וכל על מהלך החיים החברתיים והמדיניים }\מקור{[ל״ה 160]}\צהגדרה{. }

\paragraphs

\ערך{תיו }\הגדרה{- הרישום החיצוני }\מקור{[ר״מ כו]}\צהגדרה{. }

\paragraphs

\ערך{תיומת }\הגדרה{- }\משנה{התיומת }\הגדרה{- שייכות ההתאמה הפנימית\mycircle{°} }\מקור{[רצי״ה א״ש יב הערה 53]}\צהגדרה{. }

\paragraphs

\ערך{תיכף }\הגדרה{- }\משנה{תכיפה }\הגדרה{- היחש המיוחד שיש לדבר הנסמך עם הדבר הסמוך לאחריו }\מקור{[ע״ר א שנו]}\צהגדרה{. }

\paragraphs

\ערך{תכונה אלהית }\הגדרה{- הטוב\hebrewmakaf המוחלט\mycircle{°}, מוסר\mycircle{°} טהור\mycircle{°} וכללי\mycircle{°}, המחוזק בדעה ושכל\hebrewmakaf טוב\mycircle{°} }\מקור{[א׳ קמה]}\צהגדרה{. }

\paragraphs

\ערך{תכלית }\הגדרה{- }\משנה{תכונת בקשתה }\הגדרה{- נקודת הרצון\mycircle{°} שהיא מחוברת לשכל\mycircle{°} }\מקור{[א״ק ב תקנט]}\צהגדרה{. }

\paragraphs

\ערך{תכלית החיים }\הגדרה{- שמחה\hebrewmakaf של\hebrewmakaf מצוה\mycircle{°}, העונג\mycircle{°} של החיים הפנימיים השלמים }\מקור{[ע״א א ה יא]}\צהגדרה{.}

\הגדרה{קרבת\hebrewmakaf אלוהות\mycircle{°} }\מקור{[ע״א א ה מג]}\צהגדרה{. }

\paragraphs

\ערך{תכלית }\הגדרה{- }\משנה{התכלית הכללית }\הגדרה{- החפץ להעמיד את כל הברואים כולם על מכון השלמתם היותר עליונה ומוצלחת }\מקור{[ע״א ג א יג]}\צהגדרה{. }

\משנה{היסוד התכליתי שבמציאות }\הגדרה{- החיים העליונים, חיי הקדושה\mycircle{°} וקרבת\hebrewmakaf אלהים\mycircle{°} }\מקור{[שם ד ד ד]}\צהגדרה{. }

\ערך{תכלית }\הגדרה{- ע״ע מגמה. }

\paragraphs

\ערך{תכלית הכללית של האומה הישראלית }\הגדרה{- ע״ע אומה הישראלית. }

\paragraphs

\ערך{תכלית המציאות }\הגדרה{- שתצא אל הפועל הבחירה\mycircle{°} הטובה\mycircle{°} של עשיית הטוב\mycircle{°} והחכמה\mycircle{°} בחופש\mycircle{°} גמור מרצון וחפץ פנימי }\מקור{[ע״א ב ו ז]}\צהגדרה{.}

\משנה{תכלית הבריאה }\הגדרה{- ההשתלמות המוסרית }\מקור{[ע״א א א נב]}\צהגדרה{.}

\הגדרה{הטבת המעשים בהנהגה הכללית והפרטית, }\צהגדרה{״}\הגדרה{והאלהים עשה שיראו מלפניו״ }\מקור{[מ״ר 395]}\צהגדרה{. }

\paragraphs

\ערך{תֹּם }\הגדרה{- }\משנה{תוכן הַתֹּם }\הגדרה{- המערכה\mycircle{°} העליונה, מערכת הקודש\mycircle{°}, העולה ממעל לכל ערכים מוסריים\mycircle{°}, התמימות השלמה, שאין בה דופי פיסוק וקיצוץ מכל ההופעות\mycircle{°}, יושר\mycircle{°} השכל\mycircle{°}, יושר הלב, יושר הרגש, יושר הרוח\mycircle{°}, יושר הטבע, יושר הבשר, יושר ההופעה, יושר ההקשבה }\מקור{[א׳ כט]}\צהגדרה{. }

\משנה{התם }\הגדרה{- המבוע העליון\mycircle{°}, המתעלה מכל הגה חברותי, מכל רעיון של צורך חוצי, ושל צורך עצמי, הוא הוא האושר\hebrewmakaf העליון\mycircle{°}, שכל חפצים לא ישוו בו, שכל תענוגים\mycircle{°} רוחניים\mycircle{°} לא יערכוהו. התם הפנימי הוא הרבה למעלה מהמוסר, כל המוסר כולו נובע הוא מנחליו המשתפכים מכל עבריו, וממלאים את כל אופקי הרעיונות והמעשים ישרנות אופית }\מקור{[עפ״י א״ק ג קמד]}\צהגדרה{. }

\paragraphs

\ערך{תמורה }\הגדרה{- }\משנה{תוכנה }\הגדרה{- חיוב ערך אחר שיבא במקום ערך קיים }\מקור{[עפ״י ע״ר א נד]}\צהגדרה{.}

\הגדרה{ע״ע חילוף.}

\paragraphs

\ערך{תמותם }\הגדרה{- מילואם }\מקור{[ע״ר א קסג]}\צהגדרה{.}

\paragraphs

\ערך{תמימות }\הגדרה{- השלמת טבעו של האדם שלא יהיה בטבעו שום דבר של התנגדות לקדושה\hebrewmakaf העליונה\mycircle{°}. כשהאדם ממליך\mycircle{°} את השי״ת\mycircle{°} על אבריו ומקדש\mycircle{°} את עצמו }\מקור{[מ״ש רסב]}\צהגדרה{. }

\הגדרה{ע׳ במדור מדרגות והערכות אישיותיות, תמים.}

\paragraphs

\ערך{תמצית }\הגדרה{- }\משנה{ישראל תמצית האנושיות כולה ותמצית היש בכללותו}\הגדרה{ - מאוסף בצורה מרכזית לרומם ולשגב הכל }\מקור{[ע״א ד ט נז]}\צהגדרה{.}

\משנה{עם תמציתי}\הגדרה{ - (עם) המוכשר לספוג בקרבו את כל כבוד\hebrewmakaf גוים\mycircle{°}, את מבחר האידיאליות\mycircle{°} האנושית המחולקת, ולהפיצו אח״כ כדרך כניסתו בתוספות שכלול ועבוד ומטבע הגון וקבוע של חותמו הבולט והמזהיר של העם }\מקור{[עפ״י מ״ר 25]}\צהגדרה{. }

\הגדרה{ע״ע אומה כללית. ע׳ במדור פסוקים ובטויי חז״ל,  עם לבדד.}

\paragraphs

\ערך{תמר }\הגדרה{- העץ הישר הזקוף בלא עיוותים }\מקור{[מ״ר 379]}\צהגדרה{. }

\paragraphs

\ערך{תנומה}\הגדרה{ - (}\צהגדרה{לעומת שינה\mycircle{°}}\הגדרה{) - מנוחה קלה וחיצונית הבאה בעקב הלאות של הכח החומרי }\מקור{[עפ״י ע״ר א עו]}\צהגדרה{.}

\מעוין{◊ }\הגדרה{תביעת התנומה ניכרת על האברים החיצונים, על ידה מתכוננים כלי הבשר להיות כלים מכשירי המפעלים הנכבדים לתעודת החיים}\צהגדרה{ }\מקור{[עפ״י ע״ר א עו]}\צהגדרה{. }

\הגדרה{ע״ע שינה.}

\paragraphs

\ערך{תסיסה }\הגדרה{- מצב שבו יסודות העומדים בהרכבה מזגית, לוחמים זה נגד זה }\מקור{[עפ״י א״ק א רלג]}\צהגדרה{.}

\הגדרה{ציורים\mycircle{°} הסותרים זה את זה }\מקור{[שם ד תקכב]}\צהגדרה{.}

\paragraphs

\ערך{תעוב }\הגדרה{- התרחקות פנימית }\מקור{[עפ״י ע״א ד יב לה]}\צהגדרה{. }

\paragraphs

\ערך{תעודת האדם }\הגדרה{- ע״ע אדם, תעודת האדם.}

\paragraphs

\ערך{תעופה }\הגדרה{- מעוף, היקף מרחבו הכללי }\מקור{[עפ״י רצי״ה א״ש יא הערה 22]}\צהגדרה{.}

\הגדרה{ע״ע עפיפה.}

\ערך{תעופה }\הגדרה{- }\משנה{התעופה העליונה }\הגדרה{- מעמד הרוח הגדול אשר לנשמת האדם, בו מתאחד הוא בשיא רוחו עם המהות הרוחנית העומדת למעלה מכל ערכיו, מתעלה הוא מגבולי הזמן והמקום, והנצחיות הבהירה היא נחלתו, בעזוז חסנה וקדושת ערכה }\מקור{[עפ״י א״ק ג קעט]}\צהגדרה{.}

\paragraphs

\ערך{תענוג }\הגדרה{- }\משנה{אור התענוג העליון }\הגדרה{- חיים האמיתיים\mycircle{°} של הדבקות\hebrewmakaf האלהית\hebrewmakaf האמיתית\mycircle{°} }\מקור{[א׳ צט]}\צהגדרה{.}

\משנה{התענוג הנשגב\mycircle{°} באין חקר }\הגדרה{- הצורה הרוחנית\mycircle{°} שהנשמה\mycircle{°}, לובשת על ידי ההשתקעות העמוקה בדעת\hebrewmakaf אלהים\mycircle{°}, מתוך דעה\mycircle{°} שלמה, המתלבשת בהרגש נאה ולבוש מזהיר של דמיון\mycircle{°} מבוסם\mycircle{°}, כשהיא, הנשמה, עומדת כולה בעולמה האלהי\mycircle{°} }\מקור{[עפ״י א״ק ג קעא]}\צהגדרה{.}

\הגדרה{ע״ע עונג, כח העונג הטהור. ע׳ במדור מונחי קבלה ונסתר, שעשועים. ע״ע ערב, הטעם הערב (במובן רוחני).}

\paragraphs

\ערך{תפארת }\הגדרה{- הערך הסדורי שמושך עליו חן\mycircle{°} ויופי\mycircle{°} ופאר\mycircle{°} }\מקור{[ע״ר א רל]}\צהגדרה{. }

\paragraphs

\ערך{תפארת }\הגדרה{- }\משנה{ענין תפארת }\הגדרה{- דבר של שבח\mycircle{°}, שיש בענינו תולדות שבחים אחרים נמשכים, כמו ״תפארת בנים אבותם״\mycircle{°} }\מקור{[ע״ר ב קסג]}\צהגדרה{.}

\paragraphs

\ערך{״תפארת בנים אבותם״ }\הגדרה{- ע׳ במדור פסוקים ובטויי חז״ל. }

\paragraphs

\ערך{תפארת }\הגדרה{- }\משנה{(התפארת העולמית) }\הגדרה{- קיום לדורות ברוחניות\mycircle{°} של פאר ושל אצילות\mycircle{°} מופשטה }\מקור{[ע״ר א קלב]}\צהגדרה{. }

\משנה{התפארת הכללית }\הגדרה{- חוקי חיים שיהיו מועילים לסדר הנהגה שלמה ומשוערת בין כל הכחות הנפשיים האנושיים, בין השכל\mycircle{°} והרגש\mycircle{°}, בין הגשם והרוח\mycircle{°}, בין העבר, ההוה והעתיד, בין הפרט אל הכלל, בין ההכרות הבאות מחובות האדם לרעהו אל ההכרות הבאות מיחש האלהים\mycircle{°} לברואיו, הכל בערך משוער, שהוא הפאר האמיתי }\מקור{[שם רלג]}\צהגדרה{. }

\paragraphs

\ערך{תפארת }\הגדרה{- }\משנה{הופעת אור תפארת\hebrewmakaf ישראל\mycircle{°} על האומה\mycircle{°}}\הגדרה{ - מרום מצבה של האומה, בו יוחל אור אלהי נקרא בשם המפורש להגלות, בו יגלה ויראה, שכל מה שהאיר\mycircle{°} וכל מה שיאיר, כל שחי וכל שיחיה בה, הכל אור אלהי עולם אלהי ישראל הוא. בו תדע האומה כי שם\hebrewmakaf ד׳\mycircle{°} נקרא עליה }\מקור{[עפ״י א עט\hebrewmakaf פ]}\צהגדרה{.}

\ערך{תפארת ישראל }\הגדרה{- המעלה היותר עליונה (של כנסת\hebrewmakaf ישראל\mycircle{°}, בה היא תפעל) באורה על כל העולם כולו, מפני שתכיר האנושות בכללה תפארתה והדרה וההוד הנאצל לכל הדבק בדעת\hebrewmakaf אלהים\mycircle{°} ודרכיו\mycircle{°}}\צהגדרה{ }\מקור{[עפ״י פנק׳ א נח]}\צהגדרה{.}

\הגדרה{ע״ע כנסת ישראל, לעומת תפארת ישראל. }

\paragraphs

\ערך{תקוני הקודש }\הגדרה{- }\משנה{מההופעות\mycircle{°} העליונות\mycircle{°}}\הגדרה{ - יניקת פרטי ההמשכות\mycircle{°} הרצוניות והשכליות, כשהם מתבארים, מהתהום העמוק מהם, מההקפה של כללות הרצון\mycircle{°} והשכל\mycircle{°}. <שבין המשכה להמשכה, של כל פרט, אנו מכירים איזה הפסק ממולא בשכל וברצון הכללי> }\מקור{[עפ״י א״ק ג עג]}\צהגדרה{. }

\הגדרה{ע׳ במדור מונחי קבלה ונסתר, זקן, ״תקוני דיקנא״.}

\paragraphs

\משנה{תרבות }\צהגדרה{- המובן הרחב והכולל של החינוך\mycircle{°}, שהוא כולל באמת את כל עבודות רוח האדם השונות, הוא השלמת היסודות הטבעיים (שבנפש החברה האומה והאדם), גילוים והוצאתם מכח לפועל, מהעלם לגילוי שכלולם ופיתוחם }\צמקור{[ל״י א ו\hebrewmakaf ז]. }

\paragraphs

\משנה{תרבות ישראלית }\צהגדרה{- }\צמשנה{התרבות הישראלית }\צהגדרה{- התרבות\mycircle{°}, הצומחת ועולה מתוך טהרת\mycircle{°} אוירא\hebrewmakaf דארץ\hebrewmakaf ישראל\mycircle{°} גופא, ואינה משתעבדת לעול הגויים שעל צוארנו בארצות הגולה, ואשר מתוך כך היא באה לעבוד בהשתכללות חיינו התרבותית בכל הדרכים המוכשרים לה ובריכוזם הספרותי של אנשי המדע והספרות, בעלי המחשבה והשירה הישראלית הבריאה והמקורית לשם זה }\צמקור{[ל״י א (מהדורת בית אל תשס״ב) יג\hebrewmakaf יד].}

\paragraphs

\ערך{תשובה }\הגדרה{- }\משנה{התשובה הכללית }\הגדרה{- עילוי\mycircle{°} העולם ותקונו\mycircle{°} }\מקור{[א״ש ד ג]}\צהגדרה{. }

\הגדרה{ע״ע תשובה הפרטית.}

\משנה{התשובה}\myfootnote{ שערי אורה, לר״י ג׳יקטיליה, השער השמיני, הספירה השלישית ״ומנין לנו כי סוד ספירת הבינה הוא סוד תשובה  בנין אב לכל התורה כולה שנאמר: ״ובלבבו יבי״ן וש״ב ורפא לו״. ואם כן התבונן בהיות התשובה סוד העולם הבא״. ובשל״ה, בית חכמה דף יג. ״תשובה חזרת והשבת הדברים לשרשן ומקורן הרומזים על ספירת הבינה שנקראת תשובה״.\label{4}}\הגדרה{ - שיבה אל המקוריות\mycircle{°}, אל הראשית\mycircle{°}, לחבר את כל ענפי החיים אל השורש אשר משם הם יוצאים }\מקור{[א״ש תוספות תשובה ו (מ״ר 144)]}\צהגדרה{. }

\הגדרה{תנועה לשוב אל המקוריות, אל מקור\hebrewmakaf החיים\mycircle{°} וההויה העליונה\mycircle{°} בשלמותם, באין גבולים ומיצרים, במגמה\mycircle{°} היותר אצילית\mycircle{°} וברוכה\mycircle{°} מזוהר\mycircle{°} העליון הפשוט\mycircle{°} והמבהיק }\מקור{[א״ש יב ח1]}\צהגדרה{. }

\הגדרה{ההערכה\mycircle{°} שעל ידה נעשה האדם וכל היש מוצב כולו בעתיד המאושר\mycircle{°} הרחוק, חיי\hebrewmakaf עולם\hebrewmakaf הבא\mycircle{°} במילואם\mycircle{°} }\מקור{[א״ק ב תצט]}\צהגדרה{. }

\ערך{תשובה }\הגדרה{- }\משנה{האידיאל של התשובה }\הגדרה{- תקון\mycircle{°} העולם, שיצא ממקורם\hebrewmakaf של\hebrewmakaf ישראל\mycircle{°} }\מקור{[א׳ קכז]}\צהגדרה{. }

\משנה{קדושת התשובה }\הגדרה{- אור\hebrewmakaf ד׳\mycircle{°} הפועל בתקון העולם, יסוד הבריאה\mycircle{°} }\מקור{[א״ש ז ז]}\צהגדרה{. }

\ערך{תשובה }\הגדרה{- }\משנה{עולם התשובה }\הגדרה{- עולם אידיאלי\mycircle{°} מסודר כפי טהרתה\mycircle{°} של המחשבה\hebrewmakaf הראשונה\mycircle{°} }\מקור{[עפ״י קובץ ז נא]}\צהגדרה{. }

\paragraphs

\ערך{תשובה הפרטית}\הגדרה{ - (התשובה) הנוגעת לאישיות הפרטית של כל אחד ואחד, עד כדי דקות הפרטים של תקוני התשובה המיוחדים, שרוח\hebrewmakaf הקודש יודע לפרטם לפרטים היותר בודדים}\צהגדרה{ }\מקור{[א״ש ד ג]}\צהגדרה{.}

\paragraphs

\ערך{תשובה }\הגדרה{- }\משנה{תשובה עילאה }\הגדרה{- חופש\hebrewmakaf עליון\mycircle{°}, עליצות דרור }\מקור{[א״ק ג ז (א״ש יב ט)]}\צהגדרה{. }

\משנה{תשובה עליונה }\הגדרה{- שאיפת החירות\mycircle{°} המוחלטה }\מקור{[ע״ט נז]}\צהגדרה{. }

\משנה{תשובה }\הגדרה{- עולם החירות }\מקור{[א״ת ה ו1 (קבצ׳ א סח)]}\צהגדרה{. }

\משנה{תשובה עליונה }\הגדרה{- }\משנה{תוכנה }\הגדרה{- המהלך העליון של שקיקת הנשמה }\מקור{[פנק׳ א תא]}\צהגדרה{.}

\משנה{תשובה עליונה הנובעת ממקור הראשית\mycircle{°}, תמצית תוכנה }\הגדרה{- הדרישה להטהר\mycircle{°} ולהזדכך בקדושה\mycircle{°} ובפרישות\mycircle{°} }\מקור{[עפ״י מ״ר 463]}\צהגדרה{. }

\paragraphs

\ערך{תשובה }\הגדרה{- }\משנה{התשובה הרוחנית היסודית }\הגדרה{- כל הוד\mycircle{°} החיים הדרת\mycircle{°} הקודש ושמחת ההויה }\מקור{[אג׳ ג קכה]}\צהגדרה{. }

\paragraphs

\ערך{תשובה }\הגדרה{- }\משנה{(תשובה באדם) }\הגדרה{- השבת אור\hebrewmakaf ד׳\mycircle{°} לטבע הנפש }\מקור{[ע״א ג ב רט]}\צהגדרה{. }

\הגדרה{התעלות האדם אחרי הירידות, שאפשר להן להמצא בחיים}\צהגדרה{ }\מקור{[עפ״י ע״ר ב ע]}\צהגדרה{.}

\הגדרה{שכלול הרצון <שאימוצו הוא אחד מתנאיו, ועדנו וקדושתו הוא עיקרו ומהותו> }\מקור{[א״ק ב תכח]}\צהגדרה{. }

\משנה{לעשות תשובה }\הגדרה{- להתקרב אל הרצון המקורי, להגשים בחיים את טבע הנשמה }\מקור{[עפ״י א״ש ו ב]}\צהגדרה{. }

\משנה{לשוב בתשובה }\הגדרה{- לקשר את הענפים של החיים המתגלים בכל העלילות, בכל המפעלים, בכל התכונות ובכל המדות אשר לנו, אל מקור שרשם, אל טהרתה\mycircle{°}  של הנשמה, אל ראשית\mycircle{°} הויתה, אל אצילות\mycircle{°} כבודה ומקור אור\hebrewmakaf חייה\mycircle{°} }\מקור{[מ״ר 145  (א״ש תוספות תשובה ו)]}\צהגדרה{.}

\הגדרה{לשוב להתקדש\mycircle{°} להטהר\mycircle{°} מכל חטא\mycircle{°} }\מקור{[א״ק ג סד]}\צהגדרה{. }

\הגדרה{ההטבה המוסרית\mycircle{°} במעשה, ברגש ובשכל }\מקור{[א׳ קמג]}\צהגדרה{.}

\משנה{מצות התשובה}\myfootnote{ רמב״ם, תשובה ב, ב.\label{5}}\הגדרה{ - שישוב האדם ויתחרט על עוונותיו באמת, ויקבל באמת לבל ישוב עוד לכסלה, <ואם יודע תעלומות יעיד עליו שחרטתו וקבלתו נאמנות, בודאי סר עוונו וחטאתו תכופר}\צהגדרה{> }\מקור{[מא״ה ב רנה]}\צהגדרה{.}

\משנה{התשובה והתיקון }\הגדרה{- להוסיף ביתר שאת על מה שחסר מקודם }\מקור{[ע״א א א לג]}\צהגדרה{. }

\תהגדרה{להשיב את הדברים והמעשים שנתקלקלו ונתרחקו, ויצאו מהכונה הראשונה מרצון הבורא, להשיבם לשורשם ע״י התשובה והחרטה, שעל ידם האדם מתחדש ונעשה כבריה חדשה, מזוכך, מצורף ומטוהר, ונכון לקבל עוד הפעם את האור\hebrewmakaf העליון\mycircle{°} אשר סר ממנו בסיבת מעשיו המקולקלים, וישוב לאור באור העליון }\תמקור{[נ״א ה 25]. }

\צמשנה{תשובה בישראל }\צהגדרה{- התחדשות הכחות כלפי טבעיותו העצמית, החזרה המתמדת אל קדושת הויתה וגלויי פניה האמתיים (של התורה\mycircle{°})}\צמקור{ [ל״י א מט].}

\צמשנה{להתעורר בתשובה}\myfootnote{ ע׳ רמב״ם, תשובה ג, ד.\label{6}}\צהגדרה{ - לחזור לאמיתיות האדם של תורה ושל ישראל, להיפגש עם נשמתנו ואמיתיותנו. לחזור לטבעיותנו – לעומת שטף החיצוניות שמשכיח את עצמנו}\צמקור{ [שי׳ מועדים א 16].}

\צמשנה{לשוב בתשובה }\צהגדרה{- לעשות אותה (את התורה) בכל מלואה, בידיעה ובכונה ובפועל ממש }\צמקור{[ל״י א מט]. }

\ערך{תשובה }\הגדרה{- }\משנה{התכונה היסודית של התשובה }\הגדרה{- ההשערה\mycircle{°} בגודל השלמות העליונה של האלהות\mycircle{°}, <ומתוך כך העוונות בולטים הרבה מאד> }\מקור{[א״ש טו ט]}\צהגדרה{. }

\משנה{יסוד התשובה }\הגדרה{- העריגה והחפץ הקבוע אל השלמות }\מקור{[שם ה ו]}\צהגדרה{. }

\הגדרה{התעלות\mycircle{°} הרצון והשתנותו לטובה }\מקור{[שם טו ב]}\צהגדרה{. }

\משנה{יסודות התשובה }\הגדרה{- התביעות שיש לאדם על עצמו }\מקור{[עפ״י שם ה]}\צהגדרה{. }

\משנה{התשובה הראשית}\הגדרה{ - שישוב האדם אל עצמו, אל שורש נשמתו, }\צהגדרה{<ומיד ישוב אל האלהים\mycircle{°}, אל נשמת כל הנשמות, וילך ויצעד הלאה מעלה מעלה בקדושה\mycircle{°} ובטהרה\mycircle{°}> }\מקור{[א״ש טו י]}\צהגדרה{.}

\ערך{תשובה }\הגדרה{- }\משנה{כאב התשובה }\הגדרה{- הצער הרוחני, ממעמד החיים הרוחנים\mycircle{°} של עצמו, ושל העולם כולו }\מקור{[א״ש ח ח]}\צהגדרה{. }

\הגדרה{ע״ע וידוי.}

\paragraphs

\ערך{תשובה }\משנה{- (״}\הגדרה{תכלית חכמה היא}\משנה{ תשובה }\הגדרה{ומעשים טובים״}\משנה{) - }\הגדרה{ללכת בדרך יותר נשגב(ה) ממה שהלך בדרך האמונה בלא החכמה }\מקור{[ע״א א ב סג]}\צהגדרהמודגשת{. }

\הגדרה{תיקון עיוות שקדם לו, קודם שעמד על החכמה }\מקור{[ע״א א ב סב]}\צהגדרה{.}

\paragraphs

\ערך{תשובה אמונית }\הגדרה{- התשובה אשר השפעת המסורת והקבלה, אם מפחד עונש או רשם כל דבר חק ומשפט הבא מהן אל פנים הנפש גורמים הם את התשובה }\מקור{[עפ״י א״ש א]}\צהגדרה{. }

\ערך{תשובה טבעית }\הגדרה{- התשובה אשר צער גופני או נפשי ורוחני גורמים הם את התשובה }\מקור{[עפ״י שם]}\צהגדרה{. }

\ערך{תשובה שכלית }\הגדרה{- התשובה שכבר באה למדרגה העליונה, שכבר רכשה לה את (התשובה) הטבעית והאמונית, (אך) לא רק (הן גורמות אותה), אלא שגורמת אותה הכרה ברורה, הבאה מהשקפת העולם והחיים השלמה, <אשר עלתה למעלתה אחרי אשר התפקיד הטבעי והאמוני כבר רשמו בה יפה את רישומיהם> }\מקור{[עפ״י שם]}\צהגדרה{.}

\הגדרה{תשובה\hebrewmakaf מאהבה\mycircle{°}, התשובה שתבא מתוך הכרת השכל ותשוקת היושר }\מקור{[עפ״י אג׳ א קעא]}\צהגדרה{.}

\הגדרה{ע״ע אמונה, אמונה טבעית הסתכלותית. אמונה ניסית מסורתית. אמונה תוכית. }

\paragraphs

\ערך{תשובה פרטית}\הגדרה{ - (תשובה) היוצאת מתוך חשבונות הנפש הקטנים הפרטיים <שאין בהם אלא מקומם, שעתם ופרטם, ולא ״חזיין לחפאה על כל עובדין״}\myfootnote{ זוהר ח״ב ריד:\label{7}}\הגדרה{> }\מקור{[א׳ צו]}\צהגדרה{.}

\paragraphs

\ערך{תשובה מאהבה }\הגדרה{- התשובה\hebrewmakaf השכלית\mycircle{°}, (הבאה מתוך) הכרה ברורה הבאה מהשקפת העולם והחיים השלמה, אשר עלתה למעלתה אחרי אשר תפקיד התשובה\hebrewmakaf הטבעית\mycircle{°}, והתשובה\hebrewmakaf האמונית\mycircle{°} כבר רשמו בה יפה את רשומיהם }\מקור{[עפ״י א״ש א]}\צהגדרה{. }

\הגדרה{תשובה שמסגלת אותנו בכל הליכות חיינו אל המחשבה\hebrewmakaf האלהית\mycircle{°} בכל מעוף ברקיה}\צהגדרה{ }\מקור{[קבצ׳ ב קנב (קבצ׳ ג קמח)]}\צהגדרה{.}

\הגדרה{תשובה הבאה בחסד\mycircle{°} ונדבה ודעת נפש}\צהגדרה{ }\מקור{[ל״ה 215]}\צהגדרה{.}

\הגדרה{תשובה גדולה, תשובה פנימית נובעת מעומק האמת שבאור\hebrewmakaf החיים\mycircle{°} של הנשמה\mycircle{°}}\צהגדרה{ }\מקור{[א׳ עט]}\צהגדרה{. }

\משנה{תשובה באהבה}\הגדרה{ - תשובה הבאה מתוך הכרה וידיעה }\מקור{[עפ״י אג׳ א צג]}\צהגדרה{.}

\הגדרה{תשובה שיודעת לברר דבר מתוך דבר, והיא נוטלת את כל הטוב ואת כל האֹמץ והעז הראוי להצטרף לטובה גם מעצם חשכתה של הרשעה, להוסיף על ידם תוספת חיים פאר והוד גדולת אמת }\מקור{[פנק׳ ג שיב]}\צהגדרה{.}

\paragraphs

\ערך{תשובה עילאה }\הגדרה{- (תשובה) שבאה מאהבה גדולה ומתוך הכרה ברורה, (מתוך) הבינה האלהית היסודית }\מקור{[עפ״י א״ת ו ג (א״ש יד כח), ע״ט קכג]}\צהגדרה{. }

\הגדרה{תשובה מלאה, אדירה וגדולה הבאה מאהבה\mycircle{°} גדולה ומתוך הכרה ברורה, שהכל מסיע להרמת הרוח אליה, כל התורה, כל המדע, כל הכח, כל ידיעת העולם והחיים, כל ההתערבות בין הבריות, כל תכונת היושר והצדק }\מקור{[עפ״י א״ש יד, כח כט לט]}\צהגדרה{. }

\משנה{תשובה עליונה }\הגדרה{- שאיפת תיקון לכל פגם, (תשובה) הכוללת כל התשובות כולן }\מקור{[עפ״י א״ק ג קכד]}\צהגדרה{. }

\הגדרה{(תשובה) מלאה דעה ושכל\hebrewmakaf טוב\mycircle{°}, המוארה באור\hebrewmakaf תורה\mycircle{°} שבחכמת\hebrewmakaf ישראל\mycircle{°} מורשת אבות, המלאה אצילות\mycircle{°} עולמים ״וחיי עולם נטע בתוכנו, זו תורה\hebrewmakaf שבעל\hebrewmakaf פה״\mycircle{°} בכל דרגותיה ובכליל תפארתה\mycircle{°} }\מקור{[א״ת ג ב (ע״ט צג)]}\צהגדרה{. }

\משנה{מדרגת תשובה מאהבה }\הגדרה{- שלבו טהור\mycircle{°} מכל מחשבת חטא\mycircle{°} <(ש)אות הוא כי נשתנה טבעו> }\מקור{[מא״ה א קנז]}\צהגדרה{. }

\paragraphs

\ערך{תשובה עליונה של בעל הנשמה העליונה }\הגדרה{- (התגברות) האמונה הכבירה בעוצם האני העצמי (ו)כחותיו העליונים }\מקור{[עפ״י א״ק א קעג]}\צהגדרה{.}

\paragraphs

\ערך{תשובה עליונה }\הגדרה{- }\משנה{(לעומת תשובה\hebrewmakaf תתאה\mycircle{°}) }\הגדרה{- התשובה בעד הנקודה\hebrewmakaf העצמית\mycircle{°} המיוחדת של האדם בתוכיותו, שיסודה ההשכלה הקדושה\mycircle{°} והתאדרות ההשגה בנעם\mycircle{°} ד׳. זכוך התוכיות להעמדה צרופה וחפשית\mycircle{°} באמת ע״י מחשבה נאצלת ובהירות הגיון, שעיקר כונתה אימוץ רצונו של האדם והגברת ערך אישיותו }\מקור{[עפ״י א״ש טז יא, ט ז, טו ו]}\צהגדרה{. }

\ערך{תשובה תחתונה }\הגדרה{- הישרת המעשה והתעלות\mycircle{°} עדינות המזג. קדושת המעשים הפרטיים וקדושת הטבע הגופני, טהרת\mycircle{°} המזג והתעלות התכונות הטבעיות }\מקור{[שם טו ו]}\צהגדרה{. }

\משנה{תשובה תתאה }\הגדרה{- התשובה בעד העולם, ביחס למה ששופע מן האדם ולחוץ }\מקור{[עפ״י שם טז יא]}\צהגדרה{. }

\paragraphs

\ערך{תשובה נפשית פנימית }\הגדרה{- הטבת החפץ הפנימי }\מקור{[א״ש ו א1 (קבצ׳ ב צג)]}\צהגדרה{.}

\הגדרה{טוהר\mycircle{°} רצון ורעיון }\מקור{[א״ק א קצח]}\צהגדרה{.}

\ערך{תשובה חיצונית }\הגדרה{- התשובה התלויה במעשים. עולה\hebrewmakaf של\hebrewmakaf תשובה\mycircle{°} }\מקור{[שם שם]}\צהגדרה{. }

\paragraphs

\ערך{תשובה שלמה }\הגדרה{- לא התשובה מאותו הפרט של (המעשה הרע או) המדה הרעה המורגשת כעת, אלא תשובה מכל חלקי העונות והמחשבות הרעות שהם גרמו (למעשה הרע) או לתכונה הרעה המורגשת עכשיו בנפש, שתתחזק }\מקור{[עפ״י קבצ׳ ב קז (פנק׳ ד רנ)]}\צהגדרה{.}

\paragraphs

\ערך{תשובה }\הגדרה{- }\משנה{״גדולה תשובה שמביאה רפואה לעולם״}\הגדרה{ - ע׳ במדור פסוקים ובטויי חז״ל, יחיד שעשה תשובה מוחלין לו ולעולם כולו.}

\paragraphs

\ערך{תשובה }\הגדרה{- }\משנה{״זדונות מתהפכות לזכיות״}\הגדרה{ - ע׳ במדור פסוקים ובטויי חז״ל.}

\ערך{תשובה }\הגדרה{-  }\משנה{״זדונות נעשים כזכויות}\הגדרה{״ - ע׳ במדור פסוקים ובטויי חז״ל.}

\paragraphs

\ערך{תשובה }\הגדרה{- }\משנה{״חתירה״ }\הגדרה{- ע׳ במדור פסוקים ובטויי חז״ל, חתירה.}

\paragraphs

\ערך{תשובה }\צהגדרה{- }\משנה{״יחיד שעשה תשובה מוחלין לו ולעולם כולו״ }\הגדרה{- ע׳ במדור פסוקים ובטויי חז״ל, יחיד שעשה תשובה וכו׳. }

\paragraphs

\ערך{תשובה }\הגדרה{- }\משנה{יד ד׳ הפשוטה לקבל שבים }\הגדרה{- ע׳ במדור תיאורים אלהיים, יד ד׳ הפשוטה לקבל שבים.}

\משנה{״פשט ידיה וקבליה״ }\הגדרה{- ע׳ שם. }

\paragraphs

\ערך{תשובה }\הגדרה{- }\משנה{״עולה של תשובה״ }\הגדרה{- ע׳ במדור פסוקים ובטויי חז״ל, עולה של תשובה. }

\paragraphs

\ערך{״תשובה ומעשים טובים״ }\הגדרה{- ר׳ במדור פסוקים ובטויי חז״ל.}

\paragraphs

\ערך{״תשובה קדמה לעולם״ }\הגדרה{- ע׳ במדור פסוקים ובטויי חז״ל. ושם, קדמה לעולם.}

\paragraphs

\ערך{״תשובה שמגעת עד כסא הכבוד״ }\הגדרה{- ע׳ במדור פסוקים ובטויי חז״ל, חתירה. }

\paragraphs

\ערך{תשוקה }\הגדרה{- }\משנה{התשוקה האידיאלית\mycircle{°}}\הגדרה{ - יסוד הכל, חשק הטוב\hebrewmakaf העליון\mycircle{°} של הדבקות\hebrewmakaf האלהית\mycircle{°}, העולה בטובו מכל שנקלט אצלנו במושג של תענוג\mycircle{°} ועידון\mycircle{°} }\מקור{[א״ק ג קסט]}\צהגדרה{.}