

\mychapter{מילון הראיה}{הראיה מילון}


\mylettertitle{א}


\ערך{אב }\הגדרה{- הנושא המוליד, המחולל את התולדות }\מקור{[ר״מ קיז]}\צהגדרה{. }

\משנה{אז }\צהגדרה{- במאמרי הראיה }\צמקור{165: }\צהגדרה{״כי }\צהגדרהמודגשת{אז, דוקא אז}\צהגדרה{, כאשר עוד הפעם כיום צאתינו ממצרים נעמוד על רגלי עצמנו בעצמה הנשמתית, להיות דורכים על במתי ארץ בגאון ד' צור ישראל, אז יראו כל העמים צדקנו, ומשפט חירותינו יגלה ויראה על פני כל מלא עולם״. ובש״ק, קובץ א תפח: בעלי הפנימיות משתוממים הם בעת שהחיצוניות נושאת את דגלה ברמה בחיים. אבל עליהם לדעת, כי הפנימיות היא }\צהגדרהמודגשת{אז, רק אז,}\צהגדרה{ מנצחת, כשהיא מוצאה לפניה עולם ערוך ומסודר, אברים בריאים, ולב אמיץ, חושים מפותחים, וסדרי יופי, נקיות וטהרה, מוסר נימוסי, ודרך ארץ, ומרץ, ואהבה לחיים ולעולם. }\צהגדרהמודגשת{אז}\צהגדרה{ תוכל הפנימיות לשלוט על ממלכה מלאה אונים״. וביחס לחידוש הסנהדרין, עיכובים שבזמננו, בא״ה ב לח: ״ישיבה המאוחדת שתהי' מאוגדת ביחד מגדולי רבני אה״ק וגדולי רבני הגולה תהי' האספה הגדולה הזאת נקראת ״הרבנות הכוללת״, והמובן יהי' הרבנות של כל ישראל, הגוי כולו. ואם חפץ השי״ת בידינו יצליח והכינוס הזה ואיגודו יבואו על נכון, ופעולות לטובה בין לחיזוק מצב התורה בכל עניני הדת, בין לפתרונן של השאלות היותר גדולות וכלליות הקשורות בחיי האומה בארץ ובגולה, ובין לתיקון מצב הכלל ביחש החיצוני כלפי האומות, בקשר עם חיזוק ידים של מליצינו ואוהבינו שבהן, בין להפרת עצת רשעי עולם שונאינו ומקטריגינו, בין בנוגע לחיי ישראל בארץ ובין בנוגע לחייו בכל תפוצות הגולה - כשכל אלה הדרכים יעוטרו באיזה מדרגה של הצלחה וכבוד, }\צהגדרהמודגשת{אז, רק אז,}\צהגדרה{ תוכל לעלות על הפרק גם כן שאלה זו של השבת שבותינו בדבר ערכה של הסמיכה ואפשרותה. ורק דוקא לאחר כל המעשים הגדולים אשר ייראו מקיבוצינו וסידורינו, כי }\צהגדרהמודגשת{אז}\צהגדרה{ יוכלו הדברים להיות במדרגה ״מידי דקיימא לשאלה״״. אמנם, במקומות רבים בהם משתמש הרב ב}\צהגדרהמודגשת{״אז״}\צהגדרה{, כוונתו }\צהגדרהמודגשת{״דוקא אז״}\צהגדרה{. בשם מו״ר הרב צב״י טאו.}\\

\ערך{אב }\הגדרה{- המקים את הבית, המדריך את התולדות, המאיר את ארחות חייהם בהשפעתו הרוחנית }\מקור{[שם קיח]}\צהגדרה{. }

\ערך{אב }\הגדרה{- הרועה הנאמן. מדריך, העומד במעלות נפשו הרבה יותר גבוה מהמעלה של הצעירות של }\הגדרה{הבנים°}\הגדרה{, הצריכה לקבל את }\הגדרה{השפעתו°}\הגדרה{ }\מקור{[עפ״י ע״ר ב סה]}\צהגדרה{.}\\

\ערך{אבנט}\הגדרה{ - מכוון בתור אמצעי, בין החלק העליון מקום הכחות הנפשיים, לבין החלק התחתון שבגוף, מקום הכחות הגופניים השפלים, שמורה אמנם על היחש החזק שיש לכחות השפלים אל הכחות הנפשיים, עד }\הגדרה{שהקדושה°}\הגדרה{ המעלה את הנטיות הנפשיות, פועלת להגביל יפה את סדרי הפעולות הטבעיות לצד המעלה והקדושה}\מקור{ [ע״א ג ב ד]}\צהגדרה{.}\\\הגדרה{ע״ע חגורה, יסוד הויתה.}\\

\ערך{אבר }\הגדרה{- }\משנה{אבר הכנף°}\הגדרה{ - הכח הפנימי המניע את העפיפה }\מקור{[עפ״י ע״א ב ח יב]}\צהגדרה{. }

\משנה{אגרת רב שרירא גאון}\צהגדרה{ - מסמך־היסוד לסדר ההשתלשלות של כל }\צהגדרה{התורה־שבעל־פה°}\צהגדרה{, שהוא כמגדל בנוי לתלפיות של היהדות, עליו תלוי כל שלטי הגבורים במלחמתה של תורה ואמתת דורותיה. בסיס האמונים לחתימת תורת אמת של ״חיי עולם הנטועים }\צהגדרה{בתוכנו״°}\צהגדרה{ מאז היותנו לעם ד' אלהינו, בקבלת מתנתה, ונמשכים לו באחרית חתימת התלמוד עם המשך דבריהם של }\צהגדרה{הגאונים°}\צהגדרה{ מוסרי עניניו }\צמקור{[ל״י ב (מהדורת בית אל תשס״ג) נ, נב].}\\

\ערך{אד }\הגדרה{- ענן }\מקור{[ר״מ קיח]}\צהגדרה{. }

\ערך{אד }\הגדרה{- לישנא דתברא }\מקור{[ר״מ קיח]}\צהגדרה{. }

\ערך{״אדם״ }\הגדרה{- כנוי לגויה. ציור האדם השפל והנבזה על שם האדמה אשר לוקח משם, להוראת היות חומרו שפל מאד, כי הוא המדרגה הפחותה מן הדצח״מ שהוא הדומם }\מקור{[עפ״י ע״א יבמות סג.]}\צהגדרה{.}\\\ערך{״אדם״ }\הגדרה{- כנוי לנפש. ציור מדרגה גבוהה, כמו שכתוב }\הגדרה{״בצלם°}\הגדרה{ אלקים עשה את האדם״, היינו מצד }\הגדרה{נשמתו°}\הגדרה{ הרוממה, אשר היא נאצלת מתחת }\הגדרה{כסא־הכבוד°}\הגדרה{, ועל שם ״אדמה לעליון״}\footnote{ ישעיה יד יד.\label{1}}\הגדרה{ }\מקור{[עפ״י ע״א יבמות סג.]}\צהגדרה{.}\\\הגדרה{ע״ע ״אנוש״. ע״ע גבר. ע״ע איש.}\\

\ערך{אדם }\הגדרה{- נפש שכלית קשורה בחומר }\מקור{[ע״א ג ב קצט]}\צהגדרה{. }\\\ערך{אדם }\הגדרה{- }\משנה{צורת° האדם }\הגדרה{- }\הגדרה{המחשבה־העליונה°}\הגדרה{ העושה את האדם לאדם, }\הגדרה{התורה°}\הגדרה{ }\מקור{[ע״א ד ט יז]}\צהגדרה{. }\\\משנה{צורת האדם הפנימית }\הגדרה{- שכלו ומוסרו }\מקור{[פנ' א]}\צהגדרה{. }\\\ערך{אדם }\הגדרה{- }\משנה{כחו הרוחני }\הגדרה{- ע' במדור נפשיות, רוח, הכח הרוחני (של האדם).}\\\ערך{אדם}\הגדרה{ - }\משנה{סגולת° האדם}\footnote{ \textbf{כונס בקרבו }\textbf{את}\textbf{ כל }\textbf{סגולת}\textbf{ ההויה }\textbf{וכו'} - ש״ק קובץ א קעב: ״האדם הוא תמצית מלאה שההויה כולה משתקפת בו״.\label{2}}\הגדרה{ - מציינת את רוממותו הבאה בעקב שפלותו }\משנה{- }\הגדרה{יצור מושפל עד עמקי החומר, ועם זה כונס בקרבו את כל סגולת ההויה הרוחנית המלאה. שדוקא בהשתפלותו אל המורד הארצי הרי הוא רוכס את כל ההויה מראש היש עד סופו}\מקור{ [ע״א ד ט קה]}\צהגדרה{.}\\\ערך{אדם }\הגדרה{- }\משנה{נשמת האדם בכל חגויה השונים}\הגדרה{ - פרח רז עולם החיבור הנעלה שממנו מתגלה הכבדות }\הגדרה{הארצית°}\הגדרה{ עם השאיפה השמימית המנצחתה, של שפעת החיים היציריים המשתפלת דרגה אחר דרגה, עד שיוצרת את }\הגדרה{החמריות°}\הגדרה{, עם }\הגדרה{המאור־העליון°}\הגדרה{, השפעה של הוית הישות, הרוחני, האצילי, השכלי, והמוסרי, הקדוש והמצוחצח }\מקור{[עפ״י א״ק ב תקכד]}\צהגדרה{.}\\\הגדרה{יצירה שבה מתגלה האור ההויתי בכל עזו ותקפו. הכח המרכזי, שההויה חודרת באורה כולה אל הויתו, ומשלמת את תכונתה על ידו }\מקור{[פנק' ג של]}\צהגדרה{.}\\\ערך{אדם }\הגדרה{- }\משנה{תעודת האדם שנוצר בגללה}\footnote{ \textbf{תעודת}\textbf{ האדם} - ע״ע ע״ר א קפא ד״ה מכלל ופרט וכלל. א״ק ב תקלד. ע' במדור מונחי קבלה ונסתר, ״תוספת״.\label{3}}\הגדרה{ - להוסיף }\הגדרה{אור°}\הגדרה{ }\הגדרה{רצוני°}\הגדרה{ }\הגדרה{עליון°}\הגדרה{ }\הגדרה{בעזוז°}\הגדרה{ החיים הפרטיים, }\הגדרה{להעלותם°}\הגדרה{ אל }\הגדרה{עלוי°}\הגדרה{ }\הגדרה{הכלל°}\הגדרה{, ולהוסיף בכלל }\הגדרה{זיו°}\הגדרה{ צביוני חדש ע״י עושר הבא ממשפלים. (לעסוק }\הגדרה{בתורה־לשמה°}\הגדרה{) }\מקור{[א״ק א מד]}\צהגדרה{. }\\\הגדרה{ע״ע חיי האדם. ע' במדור פסוקים ובטויי חז״ל, צלם אלהים, חותם צלם אלהים מוטבע באדם. ע״ע דמות האדם.}

\ערך{אדנות מוחלטה }\הגדרה{- }\הגדרה{היכולת°}\הגדרה{ }\הגדרה{החפשית°}\הגדרה{ האין־סופית, המצויה תמיד בפועל }\הגדרה{בגבורה־של־מעלה°}\הגדרה{, היא האדנות המוחלטה והמלוכה האמיתית שהיא עומדת למעלה מכל }\הגדרה{שם°}\הגדרה{, מכל בטוי ומכל }\הגדרה{קריאה°}\הגדרה{, שהרי האפשרות אין לה קץ ותכלית, והיכולת אין לה גבול והגדרה. }\הגדרה{מלכות־אין־סוף°}\הגדרה{ במובן העליון, }\הגדרה{המלוכה־העליונה°}\הגדרה{ }\מקור{[ע״ר א מו]}\צהגדרה{. }\\\הגדרה{ע' במדור שמות כינויים ותארים אלהיים, ״אדון עולם״}\footnote{ ע' עטרת ראש להרד״ב, שער ראש השנה סי' ה.\label{4}}\הגדרה{. }

\ערך{אדריכל }\הגדרה{- פועל (את) הבנין }\מקור{[א״ק ב שנ]}\צהגדרה{. }\\\צהגדרה{ }\\\משנה{אהבה }\צהגדרה{- ההתיחסות הנאמנה, הישרה, ההגונה, המתאימה אל האמת המציאותית, מתוך שייכות נכונה וזיקה רצויה, הכרה מלאה ושלמה של המציאות, של הענין שהיא מתייחסת אליו }\צמקור{[עפ״י ל״י ב רלד].}\\\צהגדרה{מצב גדלותי, רוחני, אינטלקטואלי הכרתי נשמתי, שייכות חיונית, קישור התדבקות והזדהות, מתוך חכמה אמיתית והכרה אמיתית }\צמקור{[עפ״י שי' 63, 4־5].}

\ערך{אהבה }\הגדרה{- עדן החיים, התשוקה האלהית של העלאת נר החיים }\צהגדרה{[מ״ר }\צמקור{24}\צהגדרה{]. }\\\משנה{שלימות האהבה}\הגדרה{ - השמחה הגמורה ואור הנפש, שעמה כל טוב ואושר ובה כלולים נועם החכמה וההשגה ואהבתה }\מקור{[ע״א א ד לו]}\צהגדרה{.}\\\מעוין{◊}\הגדרה{ }\הגדרה{האמונה°}\הגדרה{ והאהבה הן עצם החיים בעוה״ז }\הגדרה{ובעוה״ב°}\הגדרה{ }\מקור{[א' סט]}\צהגדרה{. }

\ערך{אהבה }\הגדרה{- }\משנה{שמרי האהבה }\הגדרה{- ע״ע תאוות. }

\משנה{ טוב מאהבה }\הגדרה{- מידיעת }\הגדרה{הטוב°}\הגדרה{ הגנוז בהם }\מקור{[עפ״י ע״ר א רפו]}\צהגדרה{. }\\\הגדרה{מהכרה אמיתית אל הטוב והשלימות }\מקור{[ע״ר א שסז־ח (ע״א א ג לב)]}\צהגדרה{.}\\\משנה{באהבה}\הגדרה{ - בדרך חפץ פנימי והכרה עצמית }\מקור{[ל״ה 55]}\צהגדרה{. }\\\משנה{כח העבודה מאהבה}\הגדרה{ - }\מעוין{◊ }\הגדרה{אינו בא כי אם לפי מדת הידיעה הבאה בלימוד של קביעות ועשירות רבה במקצעות השונים של תורת }\הגדרה{המוסר°}\הגדרה{ }\הגדרה{והיראה°}\הגדרה{, שאי אפשר כלל להמצא מבלעדי לימוד בסדר נכון, למגרס תחילה בבקיאות מלמטה למעלה, ואחר כך למסבר בעומק עיון ודעה שלמה }\מקור{[ל״ה 188]}\צהגדרה{.}\\\הגדרה{ע״ע עבודה מאהבה, עבודת ד' מאהבה ותלמוד תורה־לשמה.}\\

\ערך{אהבה אלהית }\הגדרה{- }\משנה{האהבה האלהית העליונה, המבוסמת בבשמי הדעה העליונה }\הגדרה{- ההרגשה הנשמתית היותר חודרת ופנימית, אשר }\הגדרה{בכנסת־ישראל°}\הגדרה{ בכללותה, בנשמות אישיה היחידים, }\הגדרה{בחביון־עז°}\הגדרה{ נשמת כללותה, ובכל אשד הרוח המשתפך בכל פלגות תולדותיה }\מקור{[ע״א ד ט פח]}\צהגדרה{. }\\\הגדרה{}\הגדרה{אהבת־ד'°}\הגדרה{ }\הגדרה{אלהי־ישראל°}\הגדרה{, עצם החיים (בישראל), }\הגדרה{נשמת־האומה°}\הגדרה{ ועצם חייה }\מקור{[אג' א מד]}\צהגדרה{.}\\\משנה{אהבה אלהית }\הגדרה{- הנטיה היותר }\הגדרה{חפשית°}\הגדרה{ }\הגדרה{ונצחית°}\הגדרה{ של רוח החיים, שהופעתה באה מסקירת הגודל הבלתי מוקצב, של }\הגדרה{אור°}\הגדרה{ }\הגדרה{הקודש°}\הגדרה{ המקיף עולמי נצח ממעל לכל חק וקצב, שאור }\הגדרה{החסד°}\הגדרה{ }\הגדרה{הנאמן°}\הגדרה{ מתעלה שם, השופע ויורד בכל מלא }\הגדרה{חנו°}\הגדרה{, ממעל לכל חק }\הגדרה{ומשפט°}\הגדרה{, וכל פנות שהוא פונה הכל הוא רק }\הגדרה{לטובה°}\הגדרה{ }\הגדרה{ולברכה°}\הגדרה{ לאור }\הגדרה{ולחיים°}\הגדרה{, וכל מעשה וכל תנועה מחוללת אך }\הגדרה{נועם°}\הגדרה{ }\הגדרה{והוד°}\הגדרה{ קודש }\מקור{[עפ״י א״י כט, ע״ר א יד]}\צהגדרה{.}\\\משנה{האהבה העליונה }\צהגדרה{- }\צהגדרה{אהבת־עולם°}\צהגדרה{ }\צהגדרה{ואהבה־רבה°}\צהגדרה{, אשר לישראל את ד' אלהיהם }\צהגדרה{ואביהם־שבשמים°}\צהגדרה{ מלך־עולמים, הבוחר בעמו ומלמדו ומדריכו }\צמקור{[ל״י א (מהדורת בית אל תשס״ב) צג]. }\\\ערך{האהבה}\footnote{ \textbf{ההכרה }\textbf{האמיתית}\textbf{ וכו' }\textbf{מכבוד}\textbf{־}\textbf{אל}\textbf{, הנשקף מהבריאה וכו' וכו' }\textbf{שומע קול ד' הקורא אליו }\textbf{וכו'}\textbf{ ומרגיש שהוא}\textbf{ וכו' }\textbf{שואף את חייו יחד עם }\textbf{מקור}\textbf{־}\textbf{החיים}\textbf{,}\textbf{ וכל היצור כולו ניצב לו כאורגן שלם אדיר נחמד ואהוב, שהוא אחד מאבריו, המקבל מכולו ונותן לכולו, ויונק יחד עמו זיו חייו ממקור החיים} - ע' במדור שמות כינויים ותארים אלהיים, ״מלכנו״. ושם, ״אבינו״. ע״ע ע״ר א רמט, ד״ה ברוך. ושם, רפט ד״ה ברכנו. ושם ב ג ד״ה אמר ר' עקיבא. קבצ' ב קלז [87]. פנק' ב רד מט. ע״ע ״שמע״. ע' במדור פסוקים ובטויי חז״ל, ״ברוך שם כבוד מלכותו״. (את ההבחנה בעניין האיר לי רוני שיין).\label{5}}\ערך{ }\הגדרה{- }\צהגדרה{תכלית התעודה האנושית}\הגדרה{. ההכרה האמיתית כשמתגברת באדם כראוי, }\הגדרה{מכבוד־אל°}\הגדרה{ הכללי, הנשקף מכל }\הגדרה{הדר°}\הגדרה{ הבריאה וסדריה }\הגדרה{הגשמיים°}\הגדרה{ }\הגדרה{והרוחניים°}\הגדרה{, בעבר, בהוה ובעתיד, }\צהגדרה{<שגם זה האחרון מוצץ הוא יפה למי שמבקש }\צהגדרה{ודורש־את־אלהים°}\צהגדרה{ באמת וחפץ שלם>}\הגדרה{ אותה ההכרה כשהיא מתעצמת יפה באדם, רק היא מטבעת עליו את חותמו האמיתי, את אופיו הטבעי להקרא בשם }\הגדרה{אדם°}\הגדרה{. }\צהגדרה{רק אז הוא מרגיש שהוא חי חיים נצחיים ומכובדים. <הוא מכיר כי הדרכים שהחיים מתגלים בהם, לפי ערכנו ביחש מצבנו החומרי, שונים המה, ובכל השינויים ההווים והעתידים לבבו }\צהגדרה{בוטח־בשם־ד'°}\צהגדרה{ אלהי עולם מחיה החיים }\צהגדרה{וחי־העולמים°}\צהגדרה{> מצב נפש כזה כשהוא מתאים גם כן לכל סדרי החיים הפנימיים, הנפשיים והגופניים, חיי המשפחה והחברה, וכשהוא צועד בעוזו להיות גם כן מתפלש להיות המוסר הציבורי עומד על תילו ומכונו, אז הארץ מוכרחת להתמלא דעה, ותורת ד' היא נובעת ממעמקי הלב }\צהגדרהמודגשת{- }\צהגדרה{כל }\הגדרה{אדם שומע קול ד' הקורא אליו ושש ושמח לעשות רצון קונו וחפץ צורו, שהוא צורו הפרטי וצור העולמים כולם; ומרגיש הוא אז, שהוא האדם, שואף את חייו יחד עם }\הגדרה{מקור־החיים°}\הגדרה{, וכל היצור כולו ניצב לו כאורגן שלם אדיר נחמד ואהוב, שהוא אחד מאבריו, המקבל מכולו ונותן לכולו, ויונק יחד עמו זיו חייו ממקור החיים }\מקור{[עפ״י ל״ה 149]}\צהגדרה{.}\\\משנה{מתק האהבה }\הגדרה{- רוחב }\הגדרה{הדעת°}\הגדרה{, והנועם אשר }\הגדרה{לעדן־העליון°}\הגדרה{ }\מקור{[א״ק ג ראש דבר כט]}\צהגדרה{. }\\\משנה{אהבת צור־העולמים°}\הגדרה{ - זיו }\הגדרה{השכינה°}\הגדרה{, הכרה שכלית והרגשית, ללכת }\הגדרה{בדרכי־ד'°}\הגדרה{ באהבת אמת והכרה עמוקה }\הגדרה{פנימית°}\הגדרה{ }\מקור{[עפ״י ע״א ג ב נ]}\צהגדרה{. }\\\משנה{זיקי אהבת אלהים }\הגדרה{- מציאת }\הגדרה{אור־ד'°}\הגדרה{ בעומק רגש, בתוכן דעה }\מקור{[א״ק ג ריא]}\צהגדרה{. }\\\הגדרה{ע״ע אהבת ד'. ע״ע אהבת ד' העליונה. ע״ע יראת הגודל. }

\ערך{אהבה לעומת טוב}\הגדרה{ - ע' בנספחות, מדור מחקרים.}\\

\ערך{אהבה מינית }\הגדרה{- }\הגדרה{האהבה°}\הגדרה{ בפנותה לקראת הויית החיים ושלשלת השלמתם מדור לדור ומתקופה לתקופה, התשוקה האלהית של העלאת נר החיים }\צהגדרה{[עפ״י מ״ר }\צמקור{24-25}\צהגדרה{]. }\\

\ערך{אהבה קדושה }\הגדרה{- }\משנה{האהבה הקדושה}\הגדרה{ - }\הגדרה{אהבת־ד'°}\הגדרה{ וכל העולמים, אהבת כל היקום וכל היצור }\מקור{[א״ק ג רעט־רפ]}\צהגדרה{.}\\

\ערך{״אהבה רבה״ }\הגדרה{- ע' במדור פסוקים ובטויי חז״ל.}\\

\ערך{אהבת אור ד'}\הגדרה{ - אהבת החיים של }\הגדרה{הצדיק־האמיתי°}\הגדרה{, <שאיננה כלל אותה הנטיה הגסה של אהבת החיים המרופדת בשכרון של נטיות החומר הגסים המצוי אצל רוב הבריות, כי אם> אהבת חיקוי }\הגדרה{לחסד־עליון°}\הגדרה{ בעולמו, המתפשטת על פני כל היצור }\מקור{[קבצ' א קעד]}\צהגדרה{.}\\

\ערך{אהבת ד' }\הגדרה{- הרגשת השתוקקות תמיד }\הגדרה{לטוב°}\הגדרה{ }\הגדרה{ולאמת°}\הגדרה{ שהאדם מרגיש באמת בנקודת }\הגדרה{נשמתו°}\הגדרה{ הפנימית, <שהכל הוא בכלל טוב או בכלל אמת> }\מקור{[עפ״י קבצ' ב קלג]}\צהגדרה{.}\\\הגדרה{(האהבה) מצד }\הגדרה{כבודו°}\הגדרה{ }\הגדרה{וחסדיו°}\הגדרה{ שעשה }\מקור{[מא״ה א פט]}\צהגדרה{. }\\\צהגדרה{גלויה היסודי של }\צהגדרה{האמונה°}\צהגדרה{ הגדולה }\צמקור{[נ״ה יא].}\\\מעוין{◊ }\משנה{אהבת ד'}\הגדרה{ באה כשישים האדם לבבו להדמות }\הגדרה{לדרכי°}\הגדרה{ }\הגדרה{השי״ת°}\הגדרה{, אז ע״י ההדמות תולד האהבה, וכפי רוב הדמיון יהי' רוב האהבה }\מקור{[ע״א א ב לו (ע״ר ב קכג)]}\צהגדרה{. }\\\מעוין{◊}\הגדרה{ האהבה באה מצד השלמות שבנמצאים, שמצדה הם כולם נמצאים באמיתת מציאותו יתברך }\מקור{[ע״א ג ב קעא]}\צהגדרה{.}\\\מעוין{◊ }\הגדרה{דעת כל המציאות לאמתתה לכל סעיפיה, כפי היכולת לאדם, <שמכלל }\הגדרה{דעת־ד'°}\הגדרה{>. דעת הטבע לכל סעיפיו גיאוגרפיה והתכונה, הרפואה וחכמת הנפש, תכונות העמים וכל הנלוה להם, המביאים גם כן }\הגדרה{לאהבה־העליונה°}\הגדרה{ הזכה בכללות האנושיות }\מקור{[עפ״י קבצ' ב קלא]}\צהגדרה{.}\\\הגדרה{ע״ע אהבה אלהית. ע״ע יראת ד'. ע״ע בנספחות, מדור מחקרים, אהבה ויראה. }

\ערך{אהבת ד' העליונה }\הגדרה{- אהבת השלמות המוחלטת והגמורה של }\הגדרה{סיבת°}\הגדרה{ הכל, מחולל כל ומחיה את כל }\מקור{[א״ק ב תמב]}\צהגדרה{. }\\\משנה{אהבת ד' הבהירה }\הגדרה{- האהבה המרוממה והעדינה }\הגדרה{לאין־סוף°}\הגדרה{ }\מקור{[קובץ ה צה]}\צהגדרה{. }\\\משנה{אהבת השי״ת }\הגדרה{- }\מעוין{◊}\הגדרה{ בהכרת האמת של המציאות האלהית מצד עצמה, המקור }\הגדרה{לשמו־הגדול°}\הגדרה{ ב״ה }\מקור{[קבצ' א קלז]}\צהגדרה{. }\\\משנה{אור אהבת ד' }\הגדרה{- }\הגדרה{עדן°}\הגדרה{ החיים, מגמת החיים, עצם החיים, בהירות החיים, ומעין חיי החיים, עליון מכל הגה, מכל רצון והסברה, מכל שאיפה }\הגדרה{פנימית°}\הגדרה{, ומכל }\הגדרה{הזרחה°}\הגדרה{ }\הגדרה{יפעתית°}\הגדרה{, הכל בה, והכל ממנה }\מקור{[קובץ ו רמא]}\צהגדרה{.  }\\\הגדרה{ע' במדור פסוקים ובטויי חז״ל, אהבה רבה. ע״ע אהבת שם ד'. }

\משנה{״אהבת חנם״}\footnote{ ע' א״ק ג שכד.\label{6}}\הגדרה{ }\צהגדרה{- אהבה שגם כשיש במציאות דברים שכאילו מעכבים לה אעפ״כ תתגבר על כולם ותקבע־חנם }\צמקור{[ל״י א קיג]. }\\\צהגדרה{אהבה שאינה תלויה בדבר <כאהבת ד' לישראל, ברית עולם> }\צמקור{[ק״ת נה].}

\ערך{״אהבת חסד״ }\הגדרה{- ע' במדור פסוקים ובטויי חז״ל.}\\

\ערך{״אהבת חסד״}\הגדרה{ - }\משנה{(לעומת ״תורת חיים״)}\הגדרה{ - ע' במדור פסוקים ובטויי חז״ל.}\\

\ערך{״אהבת עולם״ }\הגדרה{- ע' במדור פסוקים ובטויי חז״ל.}\\

\ערך{אהבת שם ד' }\הגדרה{- אהבת הלימוד והידיעה של מציאות השי״ת ודרכיו, וכל המכשירים המביאים לזה }\מקור{[קבצ' א קלו]}\צהגדרה{. }\\\הגדרה{ע״ע אהבת ד'. ע״ע אהבת ד' העליונה.}\\

\ערך{אהבת תורה }\הגדרה{- ע' במדור תורה.}\\

\ערך{אוהל }\הגדרה{- בית הדירה, העלול להיות מוכן למסעות, שמרשם בתוכן הרוחני העליון את העליות הנכספות. מסמן את היסוד המטלטל, את הצביון של ההכנה אשר לתנועה, שכונתה היא תמיד השתנות ועליה לצד }\הגדרה{האושר־העליון°}\הגדרה{, לקראת }\הגדרה{הזיו°}\הגדרה{ של מעלה}\מקור{ [ע״ר א מג]}\צהגדרה{.}\\\הגדרה{ע״ע משכן.}\\

\ערך{״אוהל״ לעומת ״בית״ }\הגדרה{- ע' במדור פסוקים ובטויי חז״ל,}\משנה{ ״}\הגדרה{יושב בבית״ לעומת ״יושב אוהל״.}\\

\ערך{אויב }\הגדרה{- מי שהשנאה (אצלו) בכח לא בפועל }\מקור{[מ״ש שכז]}\צהגדרה{. }

\ערך{אולפן }\הגדרה{- לימוד <בתרגום> }\מקור{[ר״מ ב]}\צהגדרה{. }

\ערך{אומה }\הגדרה{- }\משנה{טבע האומה, הרוחני והחומרי }\הגדרה{- הטבע הפסיכולוגי של האומה, וטבע התולדה והמורשה של האבות והגזע, וטבע הגיאוגרפי של ארץ נחלתה }\מקור{[עפ״י קבצ' ב מה (ב״ר שכו־ז)]}\צהגדרה{. }\\\ערך{אומה }\הגדרה{- }\משנה{צורת° האומה }\הגדרה{- נשמתה ואורח חייה }\מקור{[עפ״י ע״א ד ה סא]}\צהגדרה{. }\\

\ערך{אומה }\הגדרה{- }\משנה{האומה (הישראלית) כולה בצרופה הכללי }\הגדרה{- אֵם החיים שלנו. האופן }\הגדרה{הכללי°}\הגדרה{ של כל }\הגדרה{ישראל°}\הגדרה{ בתור גוש אחד, המחבר את כל האישים הפרטיים להיות }\הגדרה{לעם°}\הגדרה{ אחד, הכולל ג״כ את כל הדורות כולם בהערכה אחת }\מקור{[עפ״י א' עו, ע״ר ב פד]}\צהגדרה{. }\\\הגדרה{ע״ע עם. ע״ע גוי. }

\ערך{אומה }\הגדרה{- }\משנה{רוח האומה היחידי}\הגדרה{ - השאיפה אל הטוב האלהי המונח בטבע נשמתה }\מקור{[א' נב]}\צהגדרה{.}\\\הגדרה{ע״ע רוח ישראל. ע״ע רוח ד'. ע' במדור תורה, תורה שבכתב, תורה שבכתב ברום תפארתה ותורה שבעל פה שניהם יחד.}\\

\ערך{אומה }\הגדרה{- }\משנה{שכינת האומה }\הגדרה{- רוח החיים של השאיפה האלהית המקושרת בתוכן הסגנון הצבורי של הצורה הלאומית }\מקור{[א' קו]}\צהגדרה{. }\\\הגדרה{ע' במדור מונחי קבלה ונסתר, ״שושנה עליונה״. }

\ערך{אומה }\הגדרה{- }\משנה{ישראל}\הגדרה{ - }\הגדרה{כנסת־ישראל°}\הגדרה{ המוגבלה בגבול נחלת ישראל }\מקור{[א' מב]}\צהגדרה{.}\\\הגדרה{מקום מנוחתה של }\הגדרה{האידיאה־האלהית°}\הגדרה{ על המרחב ההיסתורי הכללי }\מקור{[א' קח]}\צהגדרה{.}\\\ערך{אומה הישראלית}\הגדרה{ - }\משנה{התכלית הכללית של האומה הישראלית}\הגדרה{ - להודיע את }\הגדרה{שם־ד'°}\הגדרה{ בעולם כולו ע״י מציאותה והנהגתה }\מקור{[ל״ה 119 (פנק' ב עו)]}\צהגדרה{.}\\\הגדרה{ע״ע ישראל, מהותם העצמית הנותנת להם את אופים המיוחד.}\\

\ערך{אומה כללית }\הגדרה{- }\הגדרה{תמצית°}\הגדרה{ של המין האנושי הפועלת עליו בלי הרף בעיבוד }\הגדרה{צורתו°}\הגדרה{ }\הגדרה{הרוחנית°}\הגדרה{ }\מקור{[קובץ ה קצו]}\צהגדרה{.}\\\הגדרה{ע' במדור פסוקים ובטויי חז״ל, עם לבדד.}\\

\ערך{אוצר החיים }\הגדרה{- }\הגדרה{אורה־של־תורה°}\הגדרה{ במקורה }\מקור{[ע״ר א קמז]}\צהגדרה{. }\\\הגדרה{הצד העליון של התורה, היקר בעצמו מכל החיים כולם, }\משנה{אוצר חיים}\הגדרה{ עליונים נעלים ונשאים מכל חיי זמן ועולם}\מקור{ [ע״א ד ט ז]}\צהגדרה{.}\\\הגדרה{ע' במדור מונחי קבלה ונסתר, ״אורייתא מבינה נפקת״. ע' במדור תורה, תורה, שורש התורה. }

\ערך{אוצר הטוב }\הגדרה{- מקור }\הגדרה{חי־העולמים°}\הגדרה{ }\מקור{[אג' א קי]}\צהגדרה{.}\\

\ערך{אוצר עליון}\הגדרה{ - }\משנה{האוצר העליון}\הגדרה{ - מקור }\הגדרה{הברכות°}\הגדרה{}\מקור{ [א״ק א קיט]}\צהגדרה{.}\\

\ערך{אור }\הגדרה{- יסוד ואומץ המשכת החיים }\מקור{[עפ״י א״ק ב רצז (ע״ט טז)]}\צהגדרה{. }\\\הגדרה{כל יסוד החיים, חיי החיים, }\הגדרה{זיום°}\הגדרה{ }\הגדרה{ותפארתם°}\הגדרה{ }\מקור{[עפ״י ע״א ד יא יג]}\צהגדרה{. }\\\הגדרה{כח }\הגדרה{הרוחניות°}\הגדרה{ של השכל הגדול, של החפץ הכביר, של המרץ }\הגדרה{הנשגב°}\הגדרה{ }\צהגדרה{[מ״ר }\צמקור{296}\צהגדרה{ (קבצ' ב עא)]. }\\\משנה{האור הגדול הכללי}\צהגדרה{ - שלמות החיים ובריאותם הנמשכת ממקור אמתתם, המתגלה על ידי כל פרטיותם של דברי התורה, טיפולם וקליטתם, במלא כל הנפש ובכל תפוצות חדריה }\צמקור{[א״ל מג].}\\\הגדרה{ע״ע ״אור החיים״. ע' במדור מונחי קבלה ונסתר, אורות. }\\\ערך{אור }\הגדרה{- }\משנה{האור בעצם }\הגדרה{- }\הגדרה{אור־חדש°}\הגדרה{ של }\הגדרה{תשובה־עליונה°}\הגדרה{, המ״ט }\הגדרה{שערי°}\הגדרה{ }\הגדרה{בינה°}\הגדרה{ [}\צהגדרה{ח״פ לב:].}\\\צהגדרה{גילוי אמיתת המציאות המשוכללת }\צהגדרה{בהופעת°}\צהגדרה{ הקרנת }\צהגדרה{הזרחתה°}\צהגדרה{ }\צמקור{[עפ״י פנק' א תרלו (ב״א ד יא)]. }\\\הגדרה{ע' בנספחות, מדור מחקרים, אור, זוהר, זיו. ושם, אור, זיו, ברק. }

\ערך{אור }\הגדרה{- }\משנה{(לעומת כלי°) }\הגדרה{- נשמתו הרוחנית של הכלי <שהוא לבושו המעשי החיצון> }\מקור{[עפ״י א' קנח]}\צהגדרה{.}\\\הגדרה{החיים העצמיים של מחשבת ההויה (לעומת ההויה) }\צהגדרה{[עפ״י ע״ר א כו, וא״ק ד ת (א״ה }\צמקור{1098}\צהגדרה{)]. }\\\הגדרה{אצילות האלהות בתור נפש ההויה, (מבחינתה הפנימית), בדיבורים מצד הסתכלות השירית שברוח הקודש }\מקור{[עפ״י א״ק ב שמח]}\צהגדרה{. }\\\צהגדרה{משמעות עמוקה, תוכן }\צהגדרה{רוחני°}\צהגדרה{ }\צמקור{[פנק' א תרלז (שי' 6, 25)]. }

\ערך{אור }\הגדרה{- }\משנה{(לעומת חיים°) }\הגדרה{- }\הגדרה{דעה°}\הגדרה{, רוח־הבטה }\מקור{[עפ״י א' יא]}\צהגדרה{. }

\ערך{'אור' לעומת 'מאור' }\הגדרה{- ע' בנספחות, מדור מחקרים. }\\

\ערך{אור}\הגדרה{ - כללות הרגשה וידיעת מציאות. ערך }\הגדרה{ההשגה°}\הגדרה{ והרצון הגמור }\צהגדרה{<כי מה שלמעלה מההשגה האנושית אין לקרות כ״א בשם }\צהגדרה{חושך°}\צהגדרה{ מצד ההעלם, וכשיש העלם לחושך של מעלה, נגבל בגדר השגה ונעשה }\צהגדרהמודגשת{אור}\צהגדרה{>}\צמקור{ [עפ״י מא״ה ד כא-כב]}\צהגדרה{.}\\

\ערך{אור }\הגדרה{- ההרגשה הנפשית וההבנה של הידיעה }\מקור{[ע״א ב ט קכד]}\צהגדרה{.}\\

\ערך{אור }\הגדרה{- }\משנה{אוצר האור }\הגדרה{- חיי החיים העליונים, מקור כל החיים ושרש כל ההויות }\מקור{[ע״ר א סז]}\צהגדרה{. }\\\ערך{אור }\הגדרה{- }\משנה{האור הפנימי (של המושג מאורו של אלקים־חיים, צור ישעינו, לגדולי המשיגים) }\הגדרה{- החיים האמיתיים שאין לנו מלה ליחסם, כמו שהם נמצאים }\הגדרה{במקור־החיים°}\הגדרה{ יתברך שמו }\מקור{[עפ״י ע״א ג ב נב]}\צהגדרה{. }

\ערך{אור אין סוף }\הגדרה{- ע' במדור מונחי קבלה ונסתר. או במדור שמות כינויים ותארים אלהיים.}\\

\ערך{אור אין סופי}\הגדרה{ - }\משנה{האור האין סופי}\הגדרה{ - ההארה }\הגדרה{האלהית°}\הגדרה{ המקיפה והממלאה את כל, את כל }\הגדרה{הנשמות°}\הגדרה{ ואת כל }\הגדרה{העולמים°}\הגדרה{ }\מקור{[פנק' א שצט]}\צהגדרה{.}\\

\ערך{אור ״אל עליון קונה שמים וארץ״}\footnote{ בראשית יד יט.\label{7}}\הגדרה{ - }\הגדרה{החפץ־האלהי°}\הגדרה{, המהוה את היש כולו, המעמידו ומחייהו, הדוחפו לעילוייו בכל קומתו המעשיית והרוחנית מריש דרגין עד סופם. }\הגדרה{הנבואה־העליונה°}\הגדרה{ של פה אל פה אדבר בו, הנשפעת }\הגדרה{לנאמן°}\הגדרה{ }\הגדרה{בית°}\הגדרה{, להקים עדות ביעקב לעולמי עולמים, לקומם תבל ומלאה, בנשמת ד' יוצר כל }\מקור{[עפ״י ע״א ד ט טז]}\צהגדרה{. }

\ערך{אור אלהי }\הגדרה{- }\הגדרה{אור°}\הגדרה{ }\הגדרה{האמת°}\הגדרה{ }\הגדרה{הצדק°}\הגדרה{ }\הגדרה{והדעת°}\הגדרה{ }\מקור{[ע״ה קכח]}\צהגדרה{. }\\\הגדרה{}\הגדרה{זוהר°}\הגדרה{ גדול של שכל בהיר וחשק אדיר של רצון כביר מאד }\מקור{[א״ק ג רטז]}\צהגדרה{. }

\ערך{אור אלהי }\הגדרה{- }\משנה{האור האלהי }\הגדרה{- המגמה }\הגדרה{השעשועית°}\הגדרה{ }\הגדרה{הפנימית°}\הגדרה{ של היצירה כולה, }\הגדרה{המזריחה°}\הגדרה{ }\הגדרה{באור°}\הגדרה{ }\הגדרה{יפעתה°}\הגדרה{ על פני כל היקום, מחייה הפנימיים }\מקור{[א״ק ג קפח]}\צהגדרה{. }\\\משנה{נועם אור אלוה נורא הוד }\הגדרה{- מקור הנעימות ומעין }\הגדרה{העדנים°}\הגדרה{, אוצר }\הגדרה{ההופעות°}\הגדרה{ ומקור מקורות החיים }\מקור{[ר״מ עו]}\צהגדרה{. }\\\משנה{האור האלהי }\הגדרה{-  }\הגדרה{מקור־החיים°}\הגדרה{ ומקור כל העדן ורוממות כל }\הגדרה{אושר־עליון°}\הגדרה{. אור }\הגדרה{חיי־החיים°}\הגדרה{ }\צהגדרה{[עפ״י קבצ' א רטז (ג״ר }\צמקור{124}\צהגדרה{)]. }\\\הגדרה{חיי החיים }\מקור{[ע״א ג ב רכו]}\צהגדרה{. }\\\משנה{מקור האור האלהי}\הגדרה{ - נחל עדנים שאין לו סוף, ומקור עדן נצחי לכל נשמת חיים, המהפך את הכל }\הגדרה{לאור־חיים°}\הגדרה{. המאור הפנימי, הכח הכמוס האלהי שיש במגמת הוייתה של האומה בעולם, שהוא הסוד של כל ההויה כולה }\מקור{[עפ״י קבצ' א קעה]}\צהגדרה{.}\\\משנה{אור אלהי עליון }\הגדרה{- המרחב של אין סוף לבהירות והשלמת חיי עולמים בעד הכל }\מקור{[ע״א ד ט סה]}\צהגדרה{. }\\\הגדרה{ע״ע אור עליון. ע״ע אור ד'.}\footnote{ \textbf{אור}\textbf{ אלהי, אור אלהים, אור ד', אור עליון }- בין מושגים אלה התקשתי למצוא הבדל, מכל מקום חולקו ההגדרות למחלקות שונות על פי המונחים השונים.\label{8}}\הגדרה{ ע' במדור שמות כינויים ותארים אלהיים, אלהי, המקור האלהי. }

\ערך{אור אלהי }\הגדרה{- }\משנה{האור האלהי }\הגדרה{- }\הגדרה{נשמת־האומה°}\הגדרה{ השרשית }\מקור{[א' קנח]}\צהגדרה{. }\\\הגדרה{}\הגדרה{הזיו°}\הגדרה{ }\הגדרה{הטהור°}\הגדרה{ הממלא נפשות טהורות }\מקור{[ע״א ג ב נ]}\צהגדרה{. }

\ערך{אור° אלהים° }\הגדרה{- תעודת ההויה, מקור }\הגדרה{הנשמות°}\הגדרה{, }\הגדרה{מלא־כל°}\הגדרה{, רוח }\הגדרה{ישראל°}\הגדרה{ המופשט }\מקור{[עפ״י א״ת יב א]}\צהגדרה{. }\\\הגדרה{אור החיים היותר יפים, היותר }\הגדרה{טהורים°}\הגדרה{ היותר }\הגדרה{מאירים°}\הגדרה{ }\מקור{[ע״א ד ו מ]}\צהגדרה{. }

\ערך{אור ד'°}\הגדרה{ - }\הגדרה{העילוי°}\הגדרה{ }\הגדרה{העליון°}\הגדרה{, שממעל }\הגדרה{למקור־החיים°}\הגדרה{, יסוד המרחב העליון של }\הגדרה{הזוהר°}\הגדרה{ הבלתי מוגבל שכל }\הגדרה{עולמי־עולמים°}\הגדרה{ אינם כדאיים לו, שהוא מובדל מכל אורות עולמים, שכל תכונה של אורה בהם הרי היא מכוונת לראות על ידה גופים חשכים, שבעצמם אינם מערך מהות האורה, אבל האור בעצמו איננו דבר נראה, כי לא נתגלה בעולם לפי מדתו הכח הרואה את מהות האור. אמנם }\משנה{אור ד'}\הגדרה{ במעלת הרחבת }\הגדרה{אצילות°}\הגדרה{ מקורו, הוא האור שאור נראה בו ועל ידו }\מקור{[עפ״י ע״ר א כא]}\צהגדרה{. }\\\הגדרה{אור האורים, שאי־אפשר לנו לבטאו ואיננו יכול להתלבש באותיות של שום מבטא גם לא של שום רעיון }\מקור{[א' קלא]}\צהגדרה{. }\\\הגדרה{מקור }\הגדרה{חיי־החיים°}\הגדרה{ ב״ה }\מקור{[ע״ר א קנה]}\צהגדרה{. }\\\הגדרה{}\הגדרה{חיי־החיים°}\הגדרה{, היסוד העליון מקור חיי אור }\הגדרה{העולמים°}\הגדרה{ }\מקור{[עפ״י א״ק ג צה]}\צהגדרה{. }\\\הגדרה{יסוד כל היש, ויותר מכל היש באין קץ }\מקור{[קובץ א תתיא]}\צהגדרה{.}\\\הגדרה{}\הגדרה{צרור־החיים°}\הגדרה{ }\מקור{[שם רמ]}\צהגדרה{. }\\\הגדרה{אור }\הגדרה{האמת°}\הגדרה{ }\מקור{[עפ״י ע״א ב ט קנא]}\צהגדרה{. }\\\משנה{אור ד' מחולל כל }\הגדרה{- זוהר האמת, }\הגדרה{הוד°}\הגדרה{ }\הגדרה{אור־החיים°}\הגדרה{, }\הגדרה{שבמקור־הקודש°}\הגדרה{ }\צהגדרה{[עפ״י א״ק א ג (מ״ר }\צמקור{402}\צהגדרה{)]. }\\\משנה{אור ד' וכבודו°}\הגדרה{ - הקודש־העליון }\צהגדרה{[מ״ר }\צמקור{345}\צהגדרה{]. }\\\משנה{אור ד' העליון }\הגדרה{- כולל הכל, ומקור הכל וחיי כל }\מקור{[ע״ר א רח]}\צהגדרה{. }\\\הגדרה{ע״ע אור עליון. ע״ע אור אלהי.}\footref{8}\\\ערך{אור ד' ממרומיו°}\הגדרה{ - }\הגדרה{היש־העליון°}\הגדרה{, }\הגדרה{הרוחניות°}\הגדרה{ }\הגדרה{והטוהר°}\הגדרה{ המעולה }\מקור{[עפ״י א״ק ג רפו]}\צהגדרה{. }\\\משנה{אור ד' }\הגדרה{- אור פני המלך המתנשא לכל לראש מעל כל ענין העולמות }\מקור{[ע״ר א רפט]}\צהגדרה{.}\\\הגדרה{}\הגדרה{הארת°}\הגדרה{ היש האמיתי }\הגדרה{וזיו°}\הגדרה{ החיים האלהיים }\מקור{[ע״א ד ט מז]}\צהגדרה{. }\\\הגדרה{הקדושה השרשית העצמית, המצואה בפועל, }\הגדרה{הוד°}\הגדרה{ חיי }\הגדרה{הקודש°}\הגדרה{, המרומם ונשא מכל שרעף ורעיון }\מקור{[עפ״י ע״ר א ט]}\צהגדרה{. }\\\הגדרה{}\הגדרה{הנס°}\הגדרה{ המוחלט }\מקור{[ע״ר א מט]}\צהגדרה{.}\\\משנה{אור ד' בעולמו }\הגדרה{- אור }\הגדרה{השכינה°}\הגדרה{, נשמת העולמים, הוד }\הגדרה{האידיאליות°}\הגדרה{ האלהית החיה בכל }\מקור{[א״ק ב שסח, א״ש יד ד]}\צהגדרה{. }\\\הגדרה{}\הגדרה{אורו־של־משיח°}\הגדרה{ }\מקור{[א״ק ב תקסא]}\צהגדרה{. }\\\משנה{אור ד' }\הגדרה{- }\הגדרה{גאולה°}\הגדרה{ רוחנית עליונה, נהירה אל }\הגדרה{ד'°}\הגדרה{ ואל טובו }\מקור{[א' צא]}\צהגדרה{.}\\\הגדרה{״אורן של ישראל״, }\הגדרה{רוח־ה'°}\הגדרה{ השורה על }\הגדרה{כלל°}\הגדרה{־ישראל }\הגדרה{ותפארתם°}\הגדרה{ הכללית }\מקור{[ע״א א ד לג]}\צהגדרה{. }\\\משנה{אור ד' אשר באומה°}\הגדרה{ - }\הגדרה{השראת־השכינה°}\הגדרה{ }\הגדרה{וקדושת°}\הגדרה{ }\הגדרה{התורה°}\הגדרה{ }\הגדרה{והמצוה°}\הגדרה{ }\מקור{[עפ״י ע״ר א קעג]}\צהגדרה{. }\\\משנה{אור ד' בעולם }\הגדרה{- }\הגדרה{האמת°}\הגדרה{ }\הגדרה{והצדק°}\הגדרה{ של דעת }\הגדרה{הקודש°}\הגדרה{ }\מקור{[ע״ר א רו]}\צהגדרה{. }\\\משנה{אור ד' וטובו }\הגדרה{- }\הגדרה{הטוב°}\הגדרה{ והצדק }\מקור{[ל״ה 118]}\צהגדרה{. }\\\ערך{אור ד' בארץ }\הגדרה{- התורה האמיתית והשכל הצלול והבהיר }\מקור{[אג' א קיז]}\צהגדרה{. }\\\משנה{אור ד' }\הגדרה{- קדושת התורה והיהדות הנאמנה, מורשה קהלת יעקב }\צהגדרה{[מ״ר }\צמקור{367}\צהגדרה{].}\\\הגדרה{אור צדק עולמים אשר }\הגדרה{בתורת־חיים°}\הגדרה{ }\צהגדרה{[מ״ר }\צמקור{366}\צהגדרה{].}\\\הגדרה{}\הגדרה{המוסר°}\הגדרה{ }\הגדרה{האלהי°}\הגדרה{ המתגלה }\הגדרה{בתורה°}\הגדרה{, במסורת, }\הגדרה{בשכל°}\הגדרה{ }\הגדרה{וביושר°}\הגדרה{ }\מקור{[א״ק ג א]}\צהגדרה{. }\\\הגדרה{}\הגדרה{נועם°}\הגדרה{ הקודש }\מקור{[א״ק ב שי]}\צהגדרה{. }\\\הגדרה{}\הגדרה{זיו°}\הגדרה{ אור החכמה והשגת האמת }\מקור{[פנ' ח]}\צהגדרה{. }

\ערך{אור ד' העליון }\הגדרה{- }\משנה{(לעומת אור־הדעת־התחתון°) }\הגדרה{- }\הגדרה{אור־המקיף°}\הגדרה{ הגדול ורחב מרחבי שחקים }\מקור{[ע״א ד ט כט]}\צהגדרה{. }

\ערך{אור האמת }\הגדרה{- ע״ע אמת. }

\ערך{אור הגדול}\הגדרה{ -}\משנה{ האור הגדול}\הגדרה{ - התוכן }\הגדרה{האלהי°}\הגדרה{ שאי אפשר להגותו }\הגדרה{ולשערו°}\הגדרה{ }\מקור{[פנק' א שסב]}\צהגדרה{.}\\

\ערך{אור הגלוי }\הגדרה{- }\משנה{האור הגלוי}\הגדרה{° - האור הנראה של }\הגדרה{אור־התורה°}\הגדרה{ וחכמת ישראל כולה }\הגדרה{בקדושתה°}\הגדרה{ }\הגדרה{וטהרתה°}\הגדרה{, בבינתה והכרתה, בכבודה וישרותה, בעושר סעיפיה בעומק הגיונותיה ובאומץ מגמותיה. תלמודה של תורה בכל הרחבתה והסתעפו(יו)תיה, בדעת וכשרון, ברגש חי וקדוש, וברצון אדיר }\הגדרה{וחסון°}\הגדרה{ לחיות את אותם החיים הטהורים והקדושים אשר האור המלא הזה מתאר אותם לפנינו. }\הגדרה{(אור־הקודש־החבוי°}\הגדרה{) בהיותו מתקרב מאד אל מושגינו, אל צרכינו הזמניים, ואל מאויינו הלאומיים }\מקור{[מא״ה ג (מהדורת תשס״ד) קכג, קכה]}\צהגדרה{. }\\\הגדרה{ע״ע אור קודש חבוי. }

\ערך{אור הדעת התחתון }\הגדרה{- }\משנה{(לעומת אור־ד'־העליון°) }\הגדרה{- }\הגדרה{אור־הפנימי°}\הגדרה{ המרוכז באוצר הדעת אשר לבן־אדם }\מקור{[ע״א ד ט כט]}\צהגדרה{. }

\ערך{אור ההשואה }\הגדרה{- ע' במדור מונחי קבלה ונסתר.  }\\

\ערך{אור החיים }\הגדרה{- מקור }\הגדרה{זיו°}\הגדרה{ החיים }\מקור{[עפ״י א״ק ב שכט]}\צהגדרה{. }\\\הגדרה{שאיפת הגדלת כחותיהם }\מקור{[ע״א ד ו מא]}\צהגדרה{.}\\\הגדרה{ע״ע אור. ע״ע אור חיים. }\\\ערך{אור החיים }\הגדרה{- }\הגדרה{אור־ד'°}\הגדרה{, }\הגדרה{זיו°}\הגדרה{ }\הגדרה{החכמה°}\הגדרה{ האלהית, ואור פני מלך חוטר מגזע ישי }\מקור{[ע״א ב ט קנב]}\צהגדרה{. }\\\צהגדרה{ }\\\ערך{אור החיים העליונים }\הגדרה{- }\הגדרה{הנשגב־הכללי°}\הגדרה{ }\מקור{[א״ק ג רפ]}\צהגדרה{.}\\

\ערך{אור העתיד}\הגדרה{ - }\הגדרה{הופעת°}\הגדרה{ }\הגדרה{כבוד־ד'°}\הגדרה{ בעולם }\מקור{[א״ק ב קפב]}\צהגדרה{.}\\

\ערך{אור הפנימי}\הגדרה{ - }\משנה{האור הפנימי (של המושג מאורו של אלקים־חיים, צור ישעינו, לגדולי המשיגים)}\הגדרה{ - ע״ע אור, האור הפנימי.}\\

\ערך{אור השכינה }\הגדרה{- ע' במדור מונחי קבלה ונסתר, שכינה. }

\ערך{אור התורה }\הגדרה{- ע' במדור תורה.}\\\משנה{אורה של תורה }\הגדרה{- ע' שם. }\\

\ערך{אור חדש }\הגדרה{- ע״ע אור קודש.}

\ערך{״אור חדש״ }\הגדרה{- אוצר חיים חדש ומלא }\הגדרה{רעננות°}\הגדרה{, }\הגדרה{נשמות־חדשות°}\הגדרה{, מלאות הופעת חיים גאיוניים, ממשלת }\הגדרה{עולמי־עולמים°}\הגדרה{, הפורחת ועולה, המשחקת בכל עת לפני }\הגדרה{הדר°}\הגדרה{ }\הגדרה{אל°}\הגדרה{ עליון, האצולות }\הגדרה{מזיו°}\הגדרה{ }\הגדרה{החכמה°}\הגדרה{ }\הגדרה{והגבורה°}\הגדרה{ של מעלה }\מקור{[א״ק ג שסח]}\צהגדרה{. }

\ערך{אור חיים }\הגדרה{- אור קיום של }\הגדרה{הדר°}\הגדרה{ נצח נצחים }\מקור{[א״ק ג נח]}\צהגדרה{. }

\ערך{אור חיים}\footnote{ \textbf{אור חיים} - לבירור ההבחנה בין ״\textbf{אור}״ ל״\textbf{חיים}״, ע' הוד הקרח הנורא פרק א סי' ג, ד, ובעיקר בעמ' לו. \label{9}}\ערך{ }\הגדרה{- }\הגדרה{דעה°}\הגדרה{ }\הגדרה{ורצון°}\הגדרה{, רוח־הבטה }\הגדרה{ומציאות־מלאה°}\הגדרה{ }\מקור{[עפ״י א' יא]}\צהגדרה{. }\\\הגדרה{ע״ע ״אור החיים״. ע' בנספחות, מדור מחקרים, אור וחיים.}

\ערך{אור חַי־העולמים° }\הגדרה{- }\הגדרה{אור־עליון°}\הגדרה{, מקור מקורות, חיי החיים }\מקור{[קובץ ה צט]}\צהגדרה{.}\\

\ערך{אור חֵי־העולמים° }\הגדרה{- }\הגדרה{הענין־האלהי°}\הגדרה{}\מקור{ [א' סו]}\צהגדרה{.}\\\הגדרה{}\הגדרה{הטוב־הכללי°}\הגדרה{, הטוב האלהי השורה }\הגדרה{בעולמות°}\הגדרה{ כולם. נשמת־כל, האצילית, }\הגדרה{בהודה°}\הגדרה{ }\הגדרה{וקדושתה°}\הגדרה{ }\מקור{[עפ״י א״ש פרק ב]}\צהגדרה{. }\\\הגדרה{החיים האלהיים ההולכים ושופעים, המחיים כל חי, השולחים אורם מרום גובהם עד שפל תחתיות ארץ, המתפשטים על אדם ועל בהמה יחד }\מקור{[עפ״י ע״ט י]}\צהגדרה{. }\\\הגדרה{הרצון הכללי, הרצון העולמי }\מקור{[א״ק ג נ]}\צהגדרה{. }

\ערך{אור עליון }\הגדרה{- }\משנה{האור העליון }\הגדרה{- חייו ומקור שפעו, מחוללו ומהוהו של העולם }\מקור{[עפ״י ע״א ד ט נב]}\צהגדרה{. }\\\הגדרה{יסוד הכל ומקורו }\מקור{[קובץ א תרלו]}\צהגדרה{.}\\\הגדרה{}\הגדרה{הזיו°}\הגדרה{ }\הגדרה{האלהי°}\הגדרה{, יוצר כל }\מקור{[א״ק א קצב]}\צהגדרה{. }\\\משנה{האור העליון שבהויה}\הגדרה{ - }\הגדרה{העילוי°}\הגדרה{ }\הגדרה{הרוחני°}\הגדרה{ }\מקור{[קובץ א קסח]}\צהגדרה{.}\\\משנה{האור העליון }\הגדרה{- }\הגדרה{זוהר°}\הגדרה{ }\הגדרה{הצחצחות°}\הגדרה{ של }\הגדרה{קדש־הקדשים°}\הגדרה{ }\מקור{[א״ק ג רח]}\צהגדרה{. }\\\הגדרה{מקור מקוריות כל חיים וכל יש }\מקור{[קובץ ה נ]}\צהגדרה{. }\\\הגדרה{מקור מקורות, חיי החיים, אור חי העולמים }\מקור{[שם צט]}\צהגדרה{. }\\\הגדרה{מקור החיים והעונג }\מקור{[שם כה]}\צהגדרה{. }\\\הגדרה{בהירות חיים והויה מלאה זיו }\הגדרה{קדש°}\הגדרה{. החיים }\הגדרה{העליונים°}\הגדרה{ ברום ערכם, בהופיעם ממכון }\הגדרה{הטוב°}\הגדרה{ }\הגדרה{והעלוי°}\הגדרה{ }\הגדרה{הנשגב°}\הגדרה{ }\מקור{[עפ״י ע״ר א קצג]}\צהגדרה{. }\\\הגדרה{לשד חיי העולמים הזולף בחסדי אבות ממקור }\הגדרה{הברכה°}\הגדרה{,  מיסוד עולם שהוא קודם ונעלה מכל }\הגדרה{הגבלה°}\הגדרה{ וחוקיות מוטבעה }\מקור{[אג' ג נח]}\צהגדרה{. }\\\משנה{האור העליון הבלתי מוגבל }\הגדרה{- }\הגדרה{המוסר°}\הגדרה{ האלהי המוחלט }\מקור{[א״ת ד ד]}\צהגדרה{. }\\\הגדרה{ע״ע אור אלהי. ע״ע אור ד'.}\footref{8}\הגדרה{ע' במדור מונחי קבלה ונסתר, אור אין סוף. }

\ערך{אור קודש}\footnote{ \textbf{אור }\textbf{קודש} - בא״ק ג רפו הנוסח הוא: אור חדש.\label{10}}\ערך{ }\הגדרה{- טללי }\הגדרה{שפעת°}\הגדרה{ }\הגדרה{חכמה°}\הגדרה{ }\הגדרה{וציורים°}\הגדרה{ }\הגדרה{עליונים°}\הגדרה{, }\הגדרה{נשגבים°}\הגדרה{ }\הגדרה{ונעימים°}\הגדרה{, שהם משתפכים לתוך }\הגדרה{הנשמה°}\הגדרה{, מודיעים לה }\הגדרה{זיוים°}\הגדרה{ עליונים, מנשאים אותה }\הגדרה{לרוממות°}\הגדרה{ }\הגדרה{מעלה°}\הגדרה{, מקרבים לה את }\הגדרה{היש־העליון°}\הגדרה{, את }\הגדרה{הרוחניות°}\הגדרה{ }\הגדרה{והטוהר°}\הגדרה{ המעולה, את }\הגדרה{אור־ד'־ממרומיו°}\הגדרה{ }\מקור{[קובץ ו קנ]}\צהגדרה{. }\\\הגדרה{ע״ע אור, האור בעצם.}\\

\ערך{אור קודש חבוי }\הגדרה{- }\הגדרה{האור°}\הגדרה{ }\הגדרה{הקדוש°}\הגדרה{ הגנוז, (ה)מקור }\הגדרה{האלהי°}\הגדרה{ של }\הגדרה{התורה°}\הגדרה{, }\הגדרה{החוסן°}\הגדרה{ של }\הגדרה{הנבואה°}\הגדרה{, }\הגדרה{סגולתה°}\הגדרה{ של }\הגדרה{רוח־הקודש°}\הגדרה{ והמחזה }\הגדרה{העליון°}\הגדרה{. האור הגנוז של אור הנבואה ורוח הקודש. מעין החיים של שורש התורה האלהית ומכון כל חזון ומראה עליון. אור הקודש של חמדת עולמים הגנוזה, שורש התורה האלהית ומקור הנבואה ורוח הקודש המיוחד לישראל }\מקור{[עפ״י מא״ה ג (מהדורת תשס״ד) קכב־ה]}\צהגדרה{.}\\\הגדרה{ע״ע אור הגלוי. ע' במדור אליליות ודתות, חושך חבוי.}

\ערך{אורגן }\הגדרה{- גוף חי, מסודר }\צהגדרה{[רצי״ה א״ש ה הערה }\צמקור{1}\צהגדרה{].}

\ערך{אורגניסמוס }\הגדרה{- הקישור העצמי שיש להגוף עם הנשמה }\מקור{[קובץ ה קנה]}\צהגדרה{.}\\\ערך{אורגניסמוס }\הגדרה{- }\משנה{(האורגניות הכללית שביצירה כולה) }\הגדרה{- קישור }\הגדרה{ושילוב°}\הגדרה{ החלקים זה בזה בכל צומח ובכל חי ובאדם. כל החלקים שביש (ה)צריכים זה לזה, ותהומות רבה והררי עד (ש)הם זה בזה משולבים ומצורפים }\מקור{[עפ״י א״ק ב תיז]}\צהגדרה{. }\\\משנה{חק האורגניות}\הגדרה{ - היחש החי של השפעה ושל קבלה, (ה)הולך וחורז ומקיף את כל המצוי, את החומריות ואת הרוחניות, את הפעולות, המנהגים, ההרגשות ואת המחשבות}\מקור{ [ע״א ד ו מא]}\צהגדרה{.}\\

\ערך{״אורה״ }\הגדרה{- ע' במדור פסוקים ובטויי חז״ל. }\\\ערך{אורה }\הגדרה{- }\משנה{האורה }\הגדרה{- }\הגדרה{העילוי°}\הגדרה{ }\הגדרה{הרוחני°}\הגדרה{ }\מקור{[עפ״י קובץ א קסח]}\צהגדרה{.}\\

\ערך{אורה }\הגדרה{- }\משנה{אורה אלהית }\הגדרה{- שלמות הכל, ושלמות }\הגדרה{העדן°}\הגדרה{ של מקור הכל, שאין לנו שום מושג ממנה כי־אם מה שאנו חשים את מציאותה ומתענגים מזיוה בכל עומק נפש רוח ונשמה }\מקור{[עפ״י א״ק ג רצ, א' קיא]}\צהגדרה{. }\\\הגדרה{הגודל והשיגוב האלהי }\מקור{[קובץ ו צ]}\צהגדרה{. }

\ערך{אורה }\הגדרה{- }\משנה{אורה אלהית }\הגדרה{- }\הגדרה{החוסן°}\הגדרה{ המלא, האור העליון, המון החיים ומקור יממיהם }\מקור{[עפ״י אוה״ק ב תמז]}\צהגדרה{.}\\\הגדרה{האורה האנושית בכללה המתגלה }\הגדרה{באורן־של־ישראל°}\הגדרה{}\מקור{ [אג' א מג]}\צהגדרה{.}\\\משנה{האורה הכללית }\הגדרה{- המשך החיים הנובע מהתשוקה העליונה והכללית של }\הגדרה{קרבת־אלהים°}\הגדרה{}\מקור{ [מ״ר 38]}\צהגדרה{.}\\\משנה{אורה עליונה }\הגדרה{- }\הגדרה{אור°}\הגדרה{ }\הגדרה{חכמת°}\הגדרה{ כל }\הגדרה{עולמים°}\הגדרה{ }\מקור{[א' כט]}\צהגדרה{. }

\ערך{אורה }\הגדרה{- }\משנה{האורה הכללית }\הגדרה{- דעת }\הגדרה{היהדות°}\הגדרה{ בכל הדרת נשמתה הפנימית, העולה מעומקה של }\הגדרה{תורה°}\הגדרה{, ומאוצר ההרגשה האלהית הבאה בהתמדת התלמוד והעיון בדברים שהם כבשונו של עולם, עם המכשירים המוסריים והעיוניים הדרושים לזה }\צהגדרה{[עפ״י א״ה }\צמקור{913}\צהגדרה{]. }

\ערך{אורה אלהית }\הגדרה{- החפץ }\הגדרה{הציורי°}\הגדרה{ והמעשי, לשלטון של }\הגדרה{עילוי°}\הגדרה{ כל }\הגדרה{עז°}\הגדרה{ של }\הגדרה{צדק°}\הגדרה{ }\הגדרה{ואור°}\הגדרה{ }\מקור{[ע״ה קלב]}\צהגדרה{. }\\\הגדרה{השלמות, המעשית והשכלית, הרגשית והתכונית, השלמות במילואה }\מקור{[א״ק ב שעה]}\צהגדרה{. }\\\משנה{אורה }\הגדרה{- שלמות בכל תיקונה. חיי }\הגדרה{קודש°}\הגדרה{ }\הגדרה{וטוהר°}\הגדרה{ }\מקור{[עפ״י שם רפז, שכט]}\צהגדרה{. }\\\משנה{האורה העליונה }\הגדרה{- גדולת החיים של הכרת האלהות האמיתית, הכוללת את כל תענוגי הרוח וכל העדנים עמם ברום עוזם}\מקור{ [קובץ ז עו]}\צהגדרה{.}\\

\ערך{אורה חיצונית }\הגדרה{- נימוסים אנושיים טובים ויפים, תיקוני מדינה וממלכה נוחים ונעימים. התקדמות, סדרים, }\הגדרה{פאר°}\הגדרה{ ונעימות חיצונית הדורשים עמם חכמה מעשית רבה ללכת קוממיות ולהיות גוי איתן מלא חכמה מעשית וכליל }\הגדרה{יופי°}\הגדרה{ }\מקור{[עפ״י ע״א ג ב טז]}\צהגדרה{.}\\\הגדרה{ע״ע אורה פנימית.}\\

\ערך{אורה פנימית }\הגדרה{- }\הגדרה{אורה־של־תורה°}\הגדרה{, }\הגדרה{רוח־הקודש°}\הגדרה{ }\הגדרה{והנבואה°}\הגדרה{, ששופע }\הגדרה{בישראל°}\הגדרה{ ביחוד ממקום }\הגדרה{בית־המקדש°}\הגדרה{ יצאה }\הגדרה{האורה°}\הגדרה{, היא }\הגדרה{האורה־האלהית°}\הגדרה{ שמאירה בישראל לבדם ואין לזרים חלק בו. כח }\הגדרה{הקדושה°}\הגדרה{ המיוחדת שהיא מעלה את ישראל למצב רם ברוח קדושה }\הגדרה{ודעת־אלהים°}\הגדרה{ }\הגדרה{ודרכיו°}\הגדרה{, שכולה אומרת }\הגדרה{כבוד־אלהים°}\הגדרה{ }\מקור{[עפ״י ע״א ג ב טז]}\צהגדרה{.}\\\הגדרה{ע״ע אורה חיצונית.}\\

\ערך{אורה רוחנית }\הגדרה{- }\משנה{האורה הרוחנית }\הגדרה{- }\הגדרה{הגבורה°}\הגדרה{ הגמורה המנצחת את כל העולמים וכל כחותיהם }\מקור{[א' פד]}\צהגדרה{. }

\ערך{אורה שכלית }\הגדרה{- }\משנה{האורה השכלית}\הגדרה{ - הדעות הקבועות וארחות }\הגדרה{הדעה°}\הגדרה{ }\מקור{[ע״ר א קסח]}\צהגדרה{.}\\\מעוין{◊ }\משנה{האורה השכלית}\הגדרה{ באה מרוב תורה ודעת, מהרבה }\הגדרה{שימוש־של־חכמים°}\הגדרה{,  ומהרבה דעת העולם והחיים }\מקור{[א״ק א רמ]}\צהגדרה{. }\\

\ערך{אורה של תורה }\הגדרה{- ע' במדור תורה.}

\ערך{אורה של תורה }\הגדרה{- ע' במדור תורה, אור התורה. }

\משנה{״אורות״ }\צהגדרה{- }\צמשנה{(עניינו של ספר אורות) }\צהגדרה{- שלמות גילוי אמתת קדושת עצמיותם של ישראל וערכם האלהי העליון הנצחי}\צמקור{ [א' קפז].}\\

\ערך{אורות הקדש }\הגדרה{- החיים בחיים (ה)עליונים ברום עולמים }\הגדרה{בצחצחות°}\הגדרה{ }\הגדרה{אידיאליהם°}\הגדרה{, ספוגי }\הגדרה{קדש־קדשים°}\הגדרה{ }\מקור{[ע״ר א קפ]}\צהגדרה{. }\\\הגדרה{ע״ע מוסר הקודש. ר' חכמת הקודש.}\\

\ערך{״אורך ימים״}\footnote{ תהילים כג ו, צג ה\label{11}}\הגדרה{ - כל }\הגדרה{הימים°}\הגדרה{ בעמדם בצביונם המלא, (בהיותם) בתוכן }\הגדרה{העליון°}\הגדרה{, בקשר החיים עם }\הגדרה{הנצחיות°}\הגדרה{ }\הגדרה{האלהית°}\הגדרה{, (ששם) אין }\הגדרה{הזמן°}\הגדרה{ עובר, הכל קיים, (כש)כל העשוי בהם עומד ומזהיר, ומשביע את }\הגדרה{הנשמה°}\הגדרה{ }\הגדרה{זיו°}\הגדרה{ }\הגדרה{וצחצחות°}\הגדרה{ ושובע נעימות. מלוא הימים, (כאשר) הצירוף של כל השיגוב, שנעשה מכל פרטי החיים }\הגדרה{בטוהר°}\הגדרה{ }\הגדרה{קדושתם°}\הגדרה{, מתעלה בזיו ונהורא בהירה }\מקור{[עפ״י ע״ר ב עח]}\צהגדרה{. }\\\משנה{באורך ימים}\צהגדרה{ כלולים הם כל הימים וכל }\צהגדרה{ההשפעות°}\צהגדרה{, כל }\צהגדרה{ההופעות°}\צהגדרה{ וכל }\צהגדרה{ההזרחות°}\צהגדרה{, כל המדעים וכל ההרגשות, כל צדדי ההסתכלות, וכל ארחות }\צהגדרה{הדעה°}\צהגדרה{ }\צמקור{[א״ק א סו]. }

\ערך{״אורך ימים״}\footnote{ ברכת קריאת שמע שבערבית.\label{12}}\ערך{ }\הגדרה{- }\משנה{(תאר למעלה שבמצוות כ״אורך ימינו״ לעומת ״חיינו״) }\הגדרה{- התועלת המגיעה בשלמות }\הגדרה{הנשמה°}\הגדרה{ במה שנעלם ואינו נרגש כלל אבל הוא מקנה לה קנין נשגב }\מקור{[ע״ר א תיב (פנק' ג רסח)]}\צהגדרה{. }\\\הגדרה{ע״ע ״חיים״, תאר למעלה שבמצוות (לעומת אורך ימים).}\\

\ערך{אורך ימים }\הגדרה{- השלמת החיים היוצאת חוץ לגבול התעודה הפרטית. שמאריכים הם על המדה המוגבלת לפרטיותו ויספיק האדם תעודת החיים בעד העתיד בעד דור יבוא }\מקור{[עפ״י ע״א ג א נח (ח״פ מב.)]}\צהגדרה{. }\\\הגדרה{הנביעה של }\הגדרה{האורה°}\הגדרה{ }\הגדרה{הרוחנית°}\הגדרה{ המתגברת ועולה על ידי }\הגדרה{הברכה°}\הגדרה{ הפנימית של }\הגדרה{הנשמה°}\הגדרה{, שהיא באה ביחוד מהמקור של ההוקרה הפנימית והתוכית של הצד החיצוני המוכר }\הגדרה{בחכמה°}\הגדרה{, שזהו התוכן הפרטי שבקניני הרוח, שהיא הכרה מפורדת לחלקים שונים, (העושה את) הימים מבורכים גם בפרטיותם }\מקור{[עפ״י שם ד יג ט]}\צהגדרה{.}\\\הגדרה{ע״ע אורך שנים. }

\ערך{אורך שנים }\הגדרה{- האורה הכללית של השנים. תפיסת חיים העולה בצורה כללית, מפני הוקרת התוכן }\הגדרה{האצילי°}\הגדרה{ של }\הגדרה{החכמה°}\הגדרה{ (ה)מביאה נהרה אחדותית בנפש האדם, ודחיפת החיים היוצאת ממנה היא משאת נפש לתוכן ההכללה של החכמה בצורתה הבהירה והמקפת }\מקור{[עפ״י ע״א ד יג ט]}\צהגדרה{. }\\\הגדרה{ע״ע אורך ימים.}

\ערך{אורן של צדיקים }\הגדרה{- }\הגדרה{האור°}\הגדרה{ }\הגדרה{הרענן°}\הגדרה{, }\הגדרה{שהקודש־העליון°}\הגדרה{ חי במלא }\הגדרה{חפשו°}\הגדרה{ הנאדר בתוכו }\מקור{[א״ק ג ק]}\צהגדרה{. }

\mylettertitle{ב}


\ערך{בא }\הגדרה{- הוראת התכנסות הנושא אל המקום הראוי ע״י תנועה מוקדמת }\מקור{[ר״מ קכח]}\צהגדרה{.}

\ערך{באור }\הגדרה{- יחשו של כל מאמר בודד, לא רק לפי ערכו והדבר המבוצר בתוכו בלבד, כ״א עפ״י ערך כל אותן ההשפעות שאפשר לו להשפיע, לכשיתבאר, כאשר ״מעין ישיתוהו״ על עולם הרעיונות ההולכים בדרך ישרה, הוא פתוח בדרך מפולש לעולם הגדול של ההשכלות מלאות }\הגדרה{זיו°}\הגדרה{, ומעורר בדרך פתחו להכניס אל תוכו ועל ידו תלי תלים של ידיעות והרחבות שכליות, שנותנות אומץ וגבורה לנפשות ההוגות בהם. ״מ״ם פתוחה }\משנה{- }\הגדרה{}\הגדרה{מאמר־פתוח°}\הגדרה{״ }\מקור{[ע״א א, הקדמה, יד־טו]}\צהגדרה{.}\\\הגדרה{ע״ע פרוש. ע״ע דרש.}\\

\ערך{באור }\הגדרה{- }\משנה{דרך הביאור }\הגדרה{- הדרישה הבנויה ע״פ ערכי הכללים, שהם דומים לדרישת כל רעיון לא רק מצד עצמו, כ״א מצד הרעיונות שמטבעו להוליד ע״פ דרך ישרה, כן דרישת התורה שע״פ הכללים, אין הפרטים נולדים ומסתעפים זה מזה, כ״א כולם יחד יוצאים הם מהכללים הראשיים יסודי ועקרי התורה וסתרי טעמיה הגדולים }\מקור{[ע״א א, הקדמה, יז]}\צהגדרה{.}\\\הגדרה{ע' במדור תורה, דרישת התורה בדרך הכהן. }

\ערך{בג }\הגדרה{- מזון }\מקור{[ר״מ קכח]}\צהגדרה{.}

\ערך{בד }\הגדרה{- מגזרת בדד, ההתבודדות הפרטית }\מקור{[ר״מ קכט]}\צהגדרה{.}

\mychapter{מדורים}{מדורים}



\mymotochapter{פסוקים כמושגים \protect\\ ובטויי ומושגי חז״ל וקדמונים}{פסוקים}



    \end{multicols}
    \thispagestyle{empty}
    \begin{minipage}{4.1in}{
\ערך{יתבררו ויקראו המושגים בשמותיהם, שיהיו מובנים בתכלית דיוקם דוקא ע״פ אותם השמות המדויקים שקראו להם בחירי חכמינו מני אז. ״וישב יצחק ויחפור את בארות המים אשר חפרו עבדי אביו בימי אברהם אביו, ויסתמום פלשתים אחרי מות אברהם ויקרא להן שמות כשמות אשר קרא להן אביו״.}\\\הגדרה{מאמרי הראיה,  נחמת ישראל, עמ' }\צהגדרה{287}\\


    }
    \end{minipage}
    \clearpage
    \begin{multicols}{2}
    \myletterslave{א}


\ערך{אבוקה }\הגדרה{- }\משנה{(בעולם השכלי, לעומת אור ירח° ואור השמש°)}\footnote{ ברכות מג:\label{13}}\משנה{ }\הגדרה{- החכמה ההקשית הבנויה מהרכבת הקדמות עיוניות }\מקור{[ע״א ב ז נג]}\צהגדרה{. }

\ערך{אבירי לב הרחוקים מצדקה }\הגדרה{- ע' במדור מדרגות והערכות אישיותיות.}\\

\ערך{אֶבֶן דִי לָא בִיְדַיִן}\footnote{ דניאל ב לד ״חָזֵה הֲוַיְתָ עַד דִּי הִתְגְּזֶרֶת אֶבֶן דִּי לָא בִידַיִן וּמְחָת לְצַלְמָא וגו'״. ע״ע פנק' ג קנ.\label{14}}\ערך{ }\הגדרה{- }\משנה{די לא בידין }\הגדרה{- }\מעוין{◊}\הגדרה{ אם היה רצון }\הגדרה{השי״ת°}\הגדרה{ לעכב את ישראל עד שישלימו הם בעצמם במעשיהם הטובים את הכשרם אל }\הגדרה{הגאולה°}\הגדרה{, היתה האבן נקראת }\משנה{אבן די בידין}\הגדרה{, שבידים שהיו ישראל עושים את תיקוני שלמותם, בזה }\משנה{ימחו לצלמא°}\הגדרה{, שהוא כח הרע והקלקול שבמציאות שאינו נותן מקום לגדולת ישראל. אבל כיון שרצון השי״ת הוא רק שיהיו ישראל מוכנים קצת ואז השי״ת ינטלם וינשאם אל קדושת המעלה הראויה להם אחרי }\הגדרה{שימורק°}\הגדרה{ }\הגדרה{עונם°}\הגדרה{ ויקבלו איזו שלמות להכנה של מעלתם, אע״פ שלא יהיו עדיין מוכנים לגמרי, א״כ תהיה השבירה של הצלם ע״י }\משנה{אבן די לא בידין }\הגדרה{שלא עשו ישראל בידיהם זאת המעלה }\מקור{[מ״ש רעז־ח]}\צהגדרה{.}\\\הגדרה{גמר ענין }\הגדרה{העלאת־השכינה°}\הגדרה{ שיהי' בידי }\הגדרה{שמים°}\הגדרה{ }\מקור{[ע״ר ב רעב]}\צהגדרה{. }\\\הגדרה{ע' במדור זה, בקש יעקב לגלות את הקץ לבניו. ושם, מדת הימים. ע' במדור משיח וגאולה, ימות המשיח שאין בהם לא זכות ולא חובה.}\\

\ערך{אבן מאסו הבונים היתה לראש פנה}\footnote{ תהילים קיח כב. זוהר ח״א כד.: ״לסלקא לה, לגבי דאתמר בה ״אבן מאסו הבונים היתה לראש פנה״, וכד איהי סליקת לעילא, ברישא דכל רישין סלקא ובגינה מלאכיא אמרין אי״ה מקום כבודו״.\label{15}}\הגדרה{ - }\הגדרה{ברתא°}\הגדרה{, התכלית היותר קרובה ומובנה, }\הגדרה{הנקודה°}\הגדרה{ התחתונה, עתידה }\הגדרה{לעלות°}\הגדרה{ }\הגדרה{לרום־חביון°}\הגדרה{ }\מקור{[עפ״י אג' א קמב]}\צהגדרה{.}\\\הגדרה{ע' במדור מונחי קבלה ונסתר, פרצופים, ״אבא יסד ברתא״.}\\

\ערך{אבני שיש טהור}\footnote{ חגיגה יד:.\label{16}}\ערך{ }\הגדרה{- החומר היותר נשגב }\מקור{[קבצ' א קצח]}\צהגדרה{.}\\\הגדרה{ע' במדור זה, מים.}\\

